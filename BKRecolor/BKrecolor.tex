\documentclass[12pt]{article}
\usepackage{amsmath, amsthm, amssymb}
\usepackage{hyperref}
\usepackage{verbatim}
\usepackage[top=1.0in, bottom=1.0in, left=1.0in, right=1.0in]{geometry}

\pagestyle{plain}

\usepackage{tkz-graph}
\usetikzlibrary{arrows}
\usetikzlibrary{shapes}
\usepackage[position=bottom]{subfig}

\usepackage{longtable}
\usepackage{array}

\usepackage{sectsty}
\allsectionsfont{\sffamily}

\setcounter{secnumdepth}{5}
\setcounter{tocdepth}{5}

\makeatletter
\newtheorem*{rep@theorem}{\rep@title}
\newcommand{\newreptheorem}[2]{
\newenvironment{rep#1}[1]{
 \def\rep@title{#2 \ref{##1}}
 \begin{rep@theorem}}
 {\end{rep@theorem}}}
\makeatother

\theoremstyle{plain}
\newtheorem{thm}{Theorem}[section]
\newreptheorem{thm}{Theorem}
\newtheorem{prop}[thm]{Proposition}
\newreptheorem{prop}{Proposition}
\newtheorem{lem}[thm]{Lemma}
\newreptheorem{lem}{Lemma}
\newtheorem{conjecture}[thm]{Conjecture}
\newreptheorem{conjecture}{Conjecture}
\newtheorem{cor}[thm]{Corollary}
\newreptheorem{cor}{Corollary}
\newtheorem{prob}[thm]{Problem}

\newtheorem*{SmallPotLemma}{Small Pot Lemma}
\newtheorem*{BK}{Borodin-Kostochka Conjecture}
\newtheorem*{BK2}{Borodin-Kostochka Conjecture (restated)}
\newtheorem*{Reed}{Reed's Conjecture}
\newtheorem*{ClassificationOfd0}{Classification of $d_0$-choosable graphs}


\theoremstyle{definition}
\newtheorem{defn}{Definition}
\theoremstyle{remark}
\newtheorem*{remark}{Remark}
\newtheorem{example}{Example}
\newtheorem*{question}{Question}
\newtheorem*{observation}{Observation}

\newcommand{\fancy}[1]{\mathcal{#1}}
\newcommand{\C}[1]{\fancy{C}_{#1}}
\newcommand{\IN}{\mathbb{N}}
\newcommand{\IR}{\mathbb{R}}
\newcommand{\G}{\fancy{G}}

\newcommand{\inj}{\hookrightarrow}
\newcommand{\surj}{\twoheadrightarrow}

\newcommand{\set}[1]{\left\{ #1 \right\}}
\newcommand{\setb}[3]{\left\{ #1 \in #2 \mid #3 \right\}}
\newcommand{\setbs}[2]{\left\{ #1 \mid #2 \right\}}
\newcommand{\card}[1]{\left|#1\right|}
\newcommand{\size}[1]{\left\Vert#1\right\Vert}
\newcommand{\ceil}[1]{\left\lceil#1\right\rceil}
\newcommand{\floor}[1]{\left\lfloor#1\right\rfloor}
\newcommand{\func}[3]{#1\colon #2 \rightarrow #3}
\newcommand{\funcinj}[3]{#1\colon #2 \inj #3}
\newcommand{\funcsurj}[3]{#1\colon #2 \surj #3}
\newcommand{\irange}[1]{\left[#1\right]}
\newcommand{\join}[2]{#1 \mbox{\hspace{2 pt}$\ast$\hspace{2 pt}} #2}
\newcommand{\djunion}[2]{#1 \mbox{\hspace{2 pt}$+$\hspace{2 pt}} #2}
\newcommand{\parens}[1]{\left( #1 \right)}
\newcommand{\brackets}[1]{\left[ #1 \right]}
\newcommand{\DefinedAs}{\mathrel{\mathop:}=}

\def\adj{\leftrightarrow}
\def\nonadj{\not\!\leftrightarrow}

\def\D{\fancy{D}}
\def\C{\fancy{C}}

\newcommand{\bigclique}[1]{\frac{2}{3}\Delta(#1) + 5}
\newcommand{\bigcliqueraw}{\frac{2}{3}\Delta + 5}
\newcommand{\cliqueparts}{\frac{2}{3}\Delta}


\title{Maybe?}
\author{Landon Rabern}

\begin{document}
\maketitle

\section{Introduction}
\section{The decomposition}
Let $\D_1$ be the collection of graphs without induced $d_1$-choosable
subgraphs.  Plainly, $\D_1$ is hereditary.  For a graph $G$ and $r \in \IN$, let
$\C_r$ be the maximal cliques in $G$ having at least $r$ vertices.

\begin{lem}\label{CliqueJoin}
If $B$ is a graph such that $\join{K_4}{B}$ is not $d_1$-choosable, 
then at least one of the following holds:
\begin{itemize}
  \item $\card{B} \leq 4$ and $B$ is $\join{E_3}{K_{\card{B} - 3}}$,
  \item $\omega(B) \geq \card{B} - 1$.
\end{itemize}
\end{lem}


\begin{lem}\label{parition}
If $G \in \D_1$ and $\frac{\Delta(G) + 5}{2} \leq r \leq \Delta(G) - 1$, then
$\bigcup \C_r$ can be partitioned into sets $F_1, \ldots, F_t$ such that for
each $i \in \irange{t}$ at least one of the following holds:
\begin{itemize}
  \item $F_i \in \C_r$,
  \item $F_i = C_i \cup \set{x_i}$ where $C_i \in \C_r$ and $\card{N(x_i) \cap
  C_i} \geq r-1$.
\end{itemize}
\end{lem}

When $F_i \in \C_r$, we put $K_i \DefinedAs C_i \DefinedAs F_i$ and when $F_i =
C_i \cup \set{x_i}$, we put $K_i \DefinedAs N(x_i) \cap C_i$.

\section{The recoloring technique}

\begin{lem}\label{SingletonSetTransversal}
Let $H$ be a graph and $V_1 \cup \cdots \cup V_r$ a partition of $V(H)$. 
Suppose that $\card{V_i} \geq 2\Delta(H)$ for each $i \in \irange{r}$.  If a
graph $G$ is formed by attaching a new vertex $x$ to fewer than $2\Delta(H)$
vertices of $H$,  then $G$ has an independent set $\set{x, v_1, \ldots, v_r}$
where $v_i \in V_i$ for each $i \in \irange{r}$.
\end{lem}
\begin{proof}
Suppose not. Remove $\set{x} \cup N(x)$ from $G$ to form $H'$ with induced
partition $V_1', V_2', \ldots, V_r'$. Then $V_1', V_2', \ldots, V_r'$ has no
independent transversal since we could combine one with $x$ to get our desired
independent set in $G$. Note that $\card{V_i'} \geq 1$. 
Create a graph $Q$ by removing edges from $H'$ until it is edge minimal without
an independent transversal. Pick $yz \in E(Q)$ and apply Lemma
\ref{BaseTransversalLemma} on $yz$ with the induced partition to get the guaranteed 
$J \subseteq \irange{r}$ and the totally dominating induced matching 
$M$ with $\card{M} = \card{J} - 1$. 
Now $\card{\bigcup_{i \in J} V_i'} > 2\Delta(H)\card{J} - 2\Delta(H) =
2(\card{J} - 1)\Delta(H)$ and hence $M$ cannot dominate, a contradiction.
\end{proof}


Call $v \in V(G)$ \emph{big} if it is contained in a $\bigclique{G}$
clique in $G$.  If a vertex is not big, it is \emph{small}.

\begin{lem}\label{StrengthenedHypotheses}
If $G \in \D_1$ with $\Delta(G) \geq 50?$ and $\omega(G) < \Delta(G)$, then
exactly one of the following holds:
\begin{enumerate}
  \item $\chi(G) \leq \Delta(G) - 2$,
  \item $\chi(G) = \Delta(G) - 1$ and $G$ has a $(\Delta(G) - 1)$-coloring that
  uses only $\Delta(G) - 2$ colors on the small vertices.
\end{enumerate}
\end{lem}
\begin{proof}
Suppose not and choose a counterexample $G$ minimizing $\card{G}$.  Put $\Delta
\DefinedAs \Delta(G)$, $\chi \DefinedAs \chi(G)$ $\omega \DefinedAs \omega(G)$.
Plainly, $\chi \in \set{\Delta - 1, \Delta}$. If $G$ has no big vertex, then
Lemma \ref{DeltaMinusOneBigClique} gives a contradiction since $\chi \geq
\Delta - 1$.

Hence $G$ contains a big vertices. First, suppose $\Delta(G - x) < \Delta$ for
every big vertex $x$.  Let $H$ be the vertices with degree $\Delta$ in $G$ and
let $B$ be the big vertices in $G$.  Then $B$ is joined to $H$ in $G$.  But then
we must have $H \subseteq B$ since any vertex joined to all big vertices must be
big.  Thus it must be that the partition given by Lemma \ref{parition} for $r
\DefinedAs \bigcliqueraw$ has only one part $F_1$ and $H \subseteq K_1$.  By
Lemma \ref{SmallColoringExtension}, we have $(\Delta - 1)$-coloring of $G$ that
uses only $(\Delta - 2)$ colors on the small vertices, contradicting (2).

Otherwise, we have a big vertex $x$ such that, with $Q \DefinedAs G - x$,
we have $\Delta(Q) = \Delta$. Applying minimality of $\card{G}$ shows that
either (1) or (2) holds for $Q$. It can't be (1), for then adding $\set{x}$ as a
color class to a $(\Delta - 2)$-coloring of $Q$ gives a $(\Delta - 1)$-coloring of $G$ satisfying (2).

Hence, by symmetry, we have a $(\Delta - 1)$-coloring $\pi$ of $Q$
and such that every vertex in $M \DefinedAs \pi^{-1}(1)$ is big.  We may as well
have taken such a coloring $\pi$ so as to minimize $\card{M}$.  By minimality of $M$, every vertex in $M$ has a
neighbor in $\pi(i)$ for each $i \in J \DefinedAs \irange{\Delta - 1} -
\set{1}$.  For each $z \in M$, put $O_z \DefinedAs \setb{v}{N_Q(z)}{\pi(z)
\not \in \pi(N_Q(z) - \set{v})}$.  Then $\card{O_z} \geq \Delta - 4$ for each
$z \in M$.  Since $z$ is big, it is in some $F_j$.  If $z = x_j$, then put $P_z
\DefinedAs K_j \cap O_z$, otherwise put $P_z \DefinedAs C_j \cap O_z$.  In
either case, $\card{P_z} \geq \bigcliqueraw - 1 - 2 \geq \cliqueparts$.

Suppose there are different $y, z \in M$ such that $P_y \cap P_z \neq
\emptyset$.  By the definition of our partition of the big vertices, it must be
that $P_y, P_z \subseteq F_j$ for some $j$.  But $y$ and $z$ are not adjacent so
one of them must be $x_j$ and the other in $C_j - K_j$.  In particular,
$\card{P_y \cap P_z} \geq \card{K_j} - 4 \geq \cliqueparts$. Plainly, no $P_z$
intersects two others $P_y$ and $P_w$.

For the $w \in M$ such that $P_w$ does not intersect some other $P_z$, put $S_w
\DefinedAs P_w$.  The other vertices in $M$ come in pairs $w, w'$, put $S_w \DefinedAs P_w \cap P_{w'}$.  
Consider the subgraph $T$ of $G$ induced on $\set{x} \cup \bigcup_{w \in M}
S_w$ with each $S_w$ made into an independent set. For $w \in M$ and $z \in S_w$, we
have $d_T(z) \leq \Delta - 1 - (\cliqueparts - 1) \leq \frac13 \Delta$.

\end{proof}

\begin{thm}
Every graph satisfying $\chi \geq \Delta \geq 50?$ contains a $K_\Delta$.
\end{thm}
\begin{proof}
Suppose not and choose a counterexample $G$ minimizing $\card{G}$.  Then $G$ is
vertex critical and hence $G \in \D_1$.  By Lemma \ref{StrengthenedHypotheses},
$G$ has a $(\Delta(G) - 1)$-coloring, a contradiction.
\end{proof}

\section{Some more details}
\begin{lem}\label{DeltaMinusOneBigClique}
Any graph satisfying $\chi \geq \Delta - 1 \geq 20?$ contains a $K_{\frac23
\Delta}$.
\end{lem}

\begin{lem}\label{SmallColoringExtension}
Let $G \in \D_1$ and put $r \DefinedAs \bigclique{G}$.  If the partition in
Lemma \ref{parition} of $\C_r$ has only one part $F_1$ and
$\setb{v}{V(G)}{d(v) = \Delta(G)} \subseteq K_i$, then $G$ has a $(\Delta(G) -
1)$-coloring using only $\Delta(G) - 2$ colors on the small vertices.
\end{lem}

\end{document}
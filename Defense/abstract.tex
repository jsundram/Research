\documentclass[12pt]{article}
\usepackage{amsmath, amsthm, amssymb}
\usepackage{hyperref}
\usepackage{verbatim}
\usepackage[top=1.0in, bottom=1.0in, left=1.0in, right=1.0in]{geometry}

\pagestyle{plain}

\usepackage{tkz-graph}
\usetikzlibrary{arrows}
\usetikzlibrary{shapes}
\usepackage[position=bottom]{subfig}

\usepackage{longtable}
\usepackage{array}

\usepackage{sectsty}
\allsectionsfont{\sffamily}

\setcounter{secnumdepth}{5}
\setcounter{tocdepth}{5}

\makeatletter
\newtheorem*{rep@theorem}{\rep@title}
\newcommand{\newreptheorem}[2]{
\newenvironment{rep#1}[1]{
 \def\rep@title{#2 \ref{##1}}
 \begin{rep@theorem}}
 {\end{rep@theorem}}}
\makeatother

\theoremstyle{plain}
\newtheorem{thm}{Theorem}[section]
\newreptheorem{thm}{Theorem}
\newtheorem{prop}[thm]{Proposition}
\newreptheorem{prop}{Proposition}
\newtheorem{lem}[thm]{Lemma}
\newreptheorem{lem}{Lemma}
\newtheorem{conjecture}[thm]{Conjecture}
\newreptheorem{conjecture}{Conjecture}
\newtheorem{cor}[thm]{Corollary}
\newreptheorem{cor}{Corollary}
\newtheorem{prob}[thm]{Problem}

\newtheorem*{SmallPotLemma}{Small Pot Lemma}
\newtheorem*{BK}{Borodin-Kostochka Conjecture}
\newtheorem*{BK2}{Borodin-Kostochka Conjecture (restated)}
\newtheorem*{Reed}{Reed's Conjecture}
\newtheorem*{ClassificationOfd0}{Classification of $d_0$-choosable graphs}


\theoremstyle{definition}
\newtheorem{defn}{Definition}
\theoremstyle{remark}
\newtheorem*{remark}{Remark}
\newtheorem*{problem}{Problem}
\newtheorem{example}{Example}
\newtheorem*{question}{Question}
\newtheorem*{observation}{Observation}

\newcommand{\fancy}[1]{\mathcal{#1}}
\newcommand{\C}[1]{\fancy{C}_{#1}}
\newcommand{\IN}{\mathbb{N}}
\newcommand{\IR}{\mathbb{R}}
\newcommand{\G}{\fancy{G}}
\newcommand{\CC}{\fancy{C}}
\newcommand{\D}{\fancy{D}}

\newcommand{\inj}{\hookrightarrow}
\newcommand{\surj}{\twoheadrightarrow}

\newcommand{\set}[1]{\left\{ #1 \right\}}
\newcommand{\setb}[3]{\left\{ #1 \in #2 \mid #3 \right\}}
\newcommand{\setbs}[2]{\left\{ #1 \mid #2 \right\}}
\newcommand{\card}[1]{\left|#1\right|}
\newcommand{\size}[1]{\left\Vert#1\right\Vert}
\newcommand{\ceil}[1]{\left\lceil#1\right\rceil}
\newcommand{\floor}[1]{\left\lfloor#1\right\rfloor}
\newcommand{\func}[3]{#1\colon #2 \rightarrow #3}
\newcommand{\funcinj}[3]{#1\colon #2 \inj #3}
\newcommand{\funcsurj}[3]{#1\colon #2 \surj #3}
\newcommand{\irange}[1]{\left[#1\right]}
\newcommand{\join}[2]{#1 \mbox{\hspace{2 pt}$\ast$\hspace{2 pt}} #2}
\newcommand{\djunion}[2]{#1 \mbox{\hspace{2 pt}$+$\hspace{2 pt}} #2}
\newcommand{\parens}[1]{\left( #1 \right)}
\newcommand{\brackets}[1]{\left[ #1 \right]}
\newcommand{\DefinedAs}{\mathrel{\mathop:}=}

\def\adj{\leftrightarrow}
\def\nonadj{\not\!\leftrightarrow}

\def\D{\fancy{D}}
\def\C{\fancy{C}}

\newcommand{\bigclique}[1]{\frac{2}{3}\Delta(#1) + 5}
\newcommand{\bigcliqueraw}{\frac{2}{3}\Delta + 5}
\newcommand{\cliqueparts}{\frac{2}{3}\Delta}


\title{Coloring graphs from almost maximum degree sized palettes}
\author{Landon Rabern}
\date{}

\begin{document}
\maketitle
\begin{abstract}
Every graph can be colored with one more color than its maximum degree. A
well-known theorem of Brooks gives the precise conditions under which a graph
can be colored with maximum degree colors.  It is natural to ask for the
required conditions on a graph to color with one less color than the maximum
degree; in 1977 Borodin and Kostochka conjectured a solution for graphs with maximum degree at least 9: as long as the graph doesn't contain a maximum-degree-sized clique, it can be colored with one
fewer than the maximum degree colors.

This study attacks the conjecture on multiple fronts.  The first technique is an
extension of a vertex shuffling procedure of Catlin and is used to prove the
conjecture for graphs with edgeless high vertex subgraphs.  This general
approach also bears more theoretical fruit.

The second technique is an extension of a method Kostochka used to reduce the
Borodin-Kostochka conjecture to the maximum degree 9 case.  Results on the
existence of independent transversals are used to find an independent set
intersecting every maximum clique in a graph.  

The third technique uses list coloring results to exclude induced subgraphs in a
counterexample to the conjecture. The classification of such excludable graphs
that decompose as the join of two graphs is the backbone of
many of the results presented here.

The fourth technique uses the structure theorem for quasi-line graphs of 
Chudnovsky and Seymour in concert with the third technique to prove the
Borodin-Kostochka conjecture for claw-free graphs.

The fifth technique adds edges to proper induced subgraphs of a minimum
counterexample to gain control over the colorings produced by minimality.

The sixth technique adapts a recoloring technique originally developed for
strong coloring by Haxell and by Aharoni, Berger and Ziv to general coloring. 
Using this recoloring technique, the Borodin-Kostochka conjectured is proved for
graphs where every vertex is in a large clique.

The final technique is naive probabilistic coloring as employed by
Reed in the proof of the Borodin-Kostochka conjecture for large maximum degree. 
The technique is adapted to prove the Borodin-Kostochka conjecture for list
coloring for large maximum degree.
\end{abstract}
\end{document}

\documentclass[12pt]{article}
\usepackage{amsmath, amsthm, amssymb}
\usepackage{hyperref}
\usepackage[margin=1cm]{caption}
\usepackage{verbatim}
\usepackage[top=1.0in, bottom=1.0in, left=1.0in, right=1.0in]{geometry}
\usepackage{graphicx}

\pagestyle{plain}

\usepackage{tkz-graph}
\usetikzlibrary{arrows}
\usetikzlibrary{shapes}
\usepackage[position=bottom]{subfig}

\usepackage{longtable}
\usepackage{array}

\usepackage{sectsty}
\allsectionsfont{\sffamily}

\setcounter{secnumdepth}{5}
\setcounter{tocdepth}{5}

\makeatletter
\newtheorem*{rep@theorem}{\rep@title}
\newcommand{\newreptheorem}[2]{
\newenvironment{rep#1}[1]{
 \def\rep@title{#2 \ref{##1}}
 \begin{rep@theorem}}
 {\end{rep@theorem}}}
\makeatother

\theoremstyle{plain}
\newtheorem{thm}{Theorem}[section]
\newreptheorem{thm}{Theorem}
\newtheorem{prop}[thm]{Proposition}
\newreptheorem{prop}{Proposition}
\newtheorem{lem}[thm]{Lemma}
\newreptheorem{lem}{Lemma}
\newtheorem{conjecture}[thm]{Conjecture}
\newreptheorem{conjecture}{Conjecture}
\newtheorem{cor}[thm]{Corollary}
\newreptheorem{cor}{Corollary}
\newtheorem{prob}[thm]{Problem}
\newtheorem{observation}{Observation}
\newtheorem{obs}[observation]{Observation}
\newtheorem*{mainconj}{Main Conjecture}
\newtheorem*{mainthm}{Main Theorem}
\newtheorem{problem}{Problem}
\newtheorem{clm}{Claim}
\newtheorem*{VAL}{Vizing's Adjacency Lemma (VAL)}

\theoremstyle{definition}
\newtheorem{defn}{Definition}
\theoremstyle{remark}
\newtheorem*{remark}{Remark}
\newtheorem{example}{Example}
\newtheorem*{question}{Question}


\newcommand{\fancy}[1]{\mathcal{#1}}
\newcommand{\C}[1]{\fancy{C}_{#1}}
\newcommand{\IN}{\mathbb{N}}
\newcommand{\IR}{\mathbb{R}}
\newcommand{\G}{\fancy{G}}
\newcommand{\CC}{\fancy{C}}
\newcommand{\D}{\fancy{D}}

\newcommand{\inj}{\hookrightarrow}
\newcommand{\surj}{\twoheadrightarrow}

\newcommand{\set}[1]{\left\{ #1 \right\}}
\newcommand{\setb}[3]{\left\{ #1 \in #2 \mid #3 \right\}}
\newcommand{\setbs}[2]{\left\{ #1 \mid #2 \right\}}
\newcommand{\card}[1]{\left|#1\right|}
\newcommand{\size}[1]{\left\Vert#1\right\Vert}
\newcommand{\ceil}[1]{\left\lceil#1\right\rceil}
\newcommand{\floor}[1]{\left\lfloor#1\right\rfloor}
\newcommand{\func}[3]{#1\colon #2 \rightarrow #3}
\newcommand{\funcinj}[3]{#1\colon #2 \inj #3}
\newcommand{\funcsurj}[3]{#1\colon #2 \surj #3}
\newcommand{\irange}[1]{\left[#1\right]}
\newcommand{\join}[2]{#1 \mbox{\hspace{2 pt}$\ast$\hspace{2 pt}} #2}
\newcommand{\djunion}[2]{#1 \mbox{\hspace{2 pt}$+$\hspace{2 pt}} #2}
\newcommand{\parens}[1]{\left( #1 \right)}
\newcommand{\brackets}[1]{\left[ #1 \right]}
\newcommand{\nint}[1]{\widetilde{N}\left(#1\right)}
\newcommand{\DefinedAs}{\mathrel{\mathop:}=}
\newcommand{\pot}{\operatorname{pot}}

\def\adj{\leftrightarrow}
\def\nonadj{\not\!\leftrightarrow}

\def\D{\fancy{D}}
\def\C{\fancy{C}}
\def\Q{\fancy{Q}}
\def\Z{\fancy{Z}}
\def\H{\fancy{H}}
\def\T{\fancy{T}}
\def\X{\fancy{X}}
\def\P{\fancy{P}}
\def\L{\fancy{L}}

% any changes to \claim should be mirrored in \claimnonum and \subclaim
\newcommand{\claim}[2]{{\bf Claim #1.}~{\it #2}~~}
\newcommand{\claimnonum}[1]{{\bf Claim.}~{\it #1}~~}
\newcommand{\subclaim}[2]{{\bf Subclaim #1.}~{\it #2}~~}

\newcommand\numberthis{\addtocounter{equation}{1}\tag{\theequation}}

%
%  If the proof ends with a displayed equation, use \aftermath just
%  before \end{proof} to put the halmos in the ``right'' place.  This
%  may not work near page boundaries. 
%
\def\aftermath{\par\vspace{-\belowdisplayskip}\vspace{-\parskip}\vspace{-\baselineskip}}

\def\fr{\frac}
\def\adj{\leftrightarrow}
\def\ch{\textrm{ch}}

\renewcommand{\restriction}{\mathord{\upharpoonright}}
\begin{document}
	
\section{Definitions}
A \emph{2-partition} of a set $S$ is a partition of $S$ into sets of size two and at most one set of size one.

\section{Fixable graphs}
For different colors $a,b \in P$, let $S_{L,a,b}$ be all the vertices of $G$ that have exactly one of $a$
or $b$ in their list; more precisely, $S_{L,a,b} = \setb{v}{V(G)}{\,\card{\set{a,b} \cap L(v)} = 1}$.   
 If $\X$ is a 2-partition of $S_{L,a,b}$ and $J \subseteq \X$, let $L_{J}$ be the list assignment formed
from $L$ by swapping $a$ and $b$ in $L(v)$ for every $v \in \bigcup J$.  If $J = \set{X}$, we also write $L_X$ for $L_{J}$.

\begin{defn}
$G$ is \emph{$(L, P)$-fixable} if either
\begin{enumerate}
\item[(1)] $G$ has an $L$-edge-coloring; or
\item[(2)] there are different colors $a,b \in P$ such that for every 2-partition
$\X$ of $S_{L,a,b}$ there exists $J\subseteq \X$ so that $G$ is $(L_J, P)$-fixable.
\end{enumerate}
\end{defn}

The meaning of (1) is clear.  Intuitively, (2) says the following.  There is
some pair of colors, $a$ and $b$, such that regardless of how the vertices of
$S_{L,a,b}$ are paired via Kempe chains for colors $a$ and $b$ (or not paired
with any vertex of $S_{L,a,b}$), we can swap the colors on some subset $J$ of
the Kempe chains so that the resulting partial edge-coloring is fixable.

We write $L$-fixable as shorthand for $(L, \pot(L))$-fixable. When $G$ is $(L,
P)$-fixable, the choices of $a,b$, and $J$ in each application of (2) determine
a tree where all leaves have lists satisfying (1).  The \emph{height} of $(L,
P)$ is the minimum possible height of such a tree.  We write $h_G(L, P)$ for
this height and let $h_G(L, P) = \infty$ when $G$ is not $(L,P)$-fixable. 

\begin{lem}\label{FixableCompletesColoring}
If a multigraph $M$ has a partial $k$-edge-coloring $\pi$ such that $M_\pi$ is $(L_\pi, \irange{k})$-fixable, then $M$ is $k$-edge-colorable.
\end{lem}

\subsection{A necessary condition}
Since the edges incident to a vertex $v$ must all get different colors, 
if $G$ is $(L, P)$-fixable, then $|L(v)| \ge d_G(v)$ for all $v \in V(G)$.

By considering the maximum size of matchings in each color, we get a more
interesting necessary condition.
For each $C \subseteq \pot(L)$ and $H \subseteq G$, let $H_{L, C}$ be the
subgraph of $H$ induced by the vertices $v$ with $L(v) \cap C \ne \emptyset$. 
When $L$ is clear from context, we write $H_C$ for $H_{L,C}$. If $C =
\set{\alpha}$, we write $H_\alpha$ for $H_C$.  For $H \subseteq G$, let

\[\psi_L(H) = \sum_{\alpha \in \pot(L)} \floor{\frac{\card{H_{L, \alpha}}}{2}}.\]
Each term in the sum gives an upper bound on the size of a matching in color
$\alpha$. So $\psi_L(H)$ is an upper bound on the number of edges in a
partial $L$-edge-coloring of $H$.  The pair $(H, L)$ is \emph{abundant} if
$\psi_L(H) \ge \size{H}$ and $(G,L)$ is \emph{superabundant} if for every
$H \subseteq G$, the pair $(H, L)$ is abundant.  

\begin{lem}
\label{SuperabundanceIsNecessary} 
If $G$ is $(L, P)$-fixable, then $(G, L)$ is superabundant.  
\end{lem}


\begin{defn}
$G$ is \emph{$(L, P)$-subfixable} if either
\begin{enumerate}
\item[(1)] $G$ is $(L, P)$-fixable; or
\item[(2)] there is $xy \in E(G)$ and $\tau \in L(x) \cap L(y)$ such that
$G-xy$ is $L'$-subfixable, where $L'$ is formed from $L$ by removing $\tau$ from
$L(x)$ and $L(y)$.
\end{enumerate}
\end{defn}

Superabundance is a necessary condition for subfixability because coloring an
edge cannot make a non-abundant subgraph abundant.  The conjectures in the rest
of this paper may be easier to prove with subfixable in place of fixable.  That would
really be just as good since it would give the exact same results for edge coloring.

This may be useful.  For a multigraph $H$, let $\nu(H)$ be the number of edges in a maximum matching
of $H$.  For a list assignment $L$ on $H$, let 
$$\eta_L(H) = \sum_{\alpha \in \pot(L)} \nu(H_\alpha).$$  
Note that always $\psi_L(H) \ge \eta_L(H)$.

\begin{lem}[Marcotte and Seymour]\label{MultiTreeHall}
	Let $T$ be a multitree and $L$ a list assignment on $V(T)$.  If $\eta_L(H) \ge
	\size{H}$ for all $H \subseteq T$, then $T$ has an $L$-edge-coloring.
\end{lem}

\section{Swappable pairs}
Suppose $(G,L)$ is superabundant.  We say that $a,b \in \pot(L)$ are \emph{swappable} if $(G,L_X)$ is superabundant for every $X \subseteq S_{L,a,b}$ with $\card{X} \le 2$.

\begin{lem}\label{SwappableCondition}
	Suppose $(G, L)$ is superabundant.  Then $a, b \in \pot(L)$ are swappable if for every $H \subseteq G$, at least one of the following holds:
	\begin{enumerate}
		\item $\psi_L(H) > \size{H}$; or,
		\item $\card{H_{L, a}}$ is odd; or,
		\item $\card{H_{L, b}}$ is odd.
	\end{enumerate}
	Moreover, if (2) or (3) holds for $G$, then $\psi_{L_X}(G) = \psi_L(G)$ for every $X \subseteq S_{L,a,b}$ with $\card{X} \le 2$.
\end{lem}

\begin{proof}
	Suppose not and choose $X \subseteq S_{L,a,b}$ with $\card{X} \le 2$ such that $(G,L_X)$ is not superabundant.  
	Then we have $H \subseteq G$ such that $(H,L_X)$ is not abundant.	Note that $\card{H_{L,a}}$ and
	$\card{H_{L_X,a}}$ differ by at most 2, so their contributions to
	$\psi_L(H)$ and $\psi_{L_X}(H)$ differ by at most 1; the same is true for
	$\card{H_{L,b}}$ and $\card{H_{L_X,b}}$.  
	If $\psi_L(H)>\size{H}$, then $\psi_{L_X}(H) \ge \psi_L(H)-1\ge \size{H}$, a contradiction.
	So (2) or (3) holds.	The only way that we can have
	$\psi_{L_X}(H)<\psi_L(H)$ is if $\floor{\frac{\card{H_{L_X, a}}}{2}} +
	\floor{\frac{\card{H_{L_X, b}}}{2}} < \floor{\frac{\card{H_{L, a}}}{2}}
	+ \floor{\frac{\card{H_{L, b}}}{2}}$.  
	Since $\card{H_{L, b}} + \card{H_{L, a}} = \card{H_{L_X, b}} +
	\card{H_{L_X, a}}$,   this requires that both $\card{H_{L, b}}$ and
	$\card{H_{L, a}}$ are even; since (2) or (3) holds, this is impossible.
\end{proof}

\section{Stars with one edge subdivided}
	We say a graph $G$ is a $\operatorname{LongStar}_{r,s,t}$ if $G$ is a star with one edge subdivided, where $r$ is the center of the
	star, $t$ the vertex at distance two from $r$, and $s$ the intervening vertex.  
	
	We want to prove the following conjecture.
	
\begin{conjecture}
	\label{StarWithOneEdgeSubdivided}
 	 Suppose $G$ is a $\operatorname{LongStar}_{r,s,t}$. If $(G,L)$ is superabundant and $|L(v)| \ge d_G(v)$ for all 
	 $v \in V(G)$, then $G$ is $L$-subfixable if at least one of the following holds:
	\begin{enumerate}
		\item[(a)] $|L(r)| > d_G(r)$; or
		\item[(b)] $|L(s)| > d_G(s)$; or
		\item[(c)] $\psi_L(G) > \size{G}$ (how do we do induction here?)
	\end{enumerate}
\end{conjecture}

For a list assignment $L$ on a $\operatorname{LongStar}_{r,s,t}$ graph $G$, create a bipartite graph $B_L(G)$ with parts $X_L(G) = \setb{uw}{E(G - t)}{L(u) \cap L(w) \ne \emptyset}$ and $Y_L(G) = \setb{\alpha}{\pot(L)}{\nu((G - t)_\alpha) = 1}$, where $uw \in X_L(G)$ is adjacent to $\alpha \in Y_L(G)$ if and only if $\alpha \in L(u) \cap L(w)$.  Put $F_L(G) = L(r) \setminus \bigcup_{v \in N(r)} L(v)$.

	
	We prove part (a). 
	
	
	For a $\operatorname{LongStar}_{r,s,t}$ graph $G$, let $\L(G)$ be all list assignments $L$ such that $(G,L)$ is superabundant and $|L(r)| > d_G(r)$ and $|L(v)| \ge d_G(v)$ for all $v \in V(G-r)$.  Suppose there is a $\operatorname{LongStar}_{r,s,t}$ graph $G$ and $L \in \L(G)$ such that $G$ is not $L$-subfixable.  Choose such a $G$ and $L$ to 
	
	\begin{enumerate}
		\item minimize $|G|$; and 
		\item subject to that to maximize $\eta_L(G - t)$; and
		\item subject to that to have $F_L(G) \cap L(t) \ne \emptyset$ if possible.
	\end{enumerate}
	
	Since $G$ is not $L$-subfixable, for each pair of colors $a,b \in \pot(L)$ there is a 2-partition $\X_{a,b}$ of $S_{L,a,b}$ such that $G$ is not $L_J$-subfixable for every $J \subseteq \X_{a,b}$.  Let $F = F_L(G)$, $B = B_L(G)$ and $Y = Y_L(G)$.
	
	\begin{lem}
		For every $v \in V(G)$ with $d_G(v) = 1$, we have $|L(v)| \ge 2$.
	\end{lem}
	\begin{proof}
		Suppose we have $v \in V(G)$ with $d_G(v) = 1$ and $L(v) = \set{\alpha}$.  Let $N(v) = \set{w}$.  Then $\alpha \in L(w)$ since $(G[v,w], L)$ is abundant.  Let $G' = G-v$ and let $L'$ be the list assignment on $G'$ where $L'(w) = L(w) - \alpha$ and $L'(x) = L(x)$ for all $x \in V(G' - w)$.  If $(G', L')$ is superabundant, then $G'$ is $L'$-subfixable by minimality of $|G|$ and we get that $G$ is $L$-subfixable by coloring $vw$ with $\alpha$, a contradiction.  So there is an induced subgraph $H'$ of $G'$ with $w \in V(H')$ such that $(H', L')$ is not abundant.  Consider $H = G[V(H') \cup \set{v}]$.  We have $\psi_{L}(H) \le \psi_{L'}(H') + 1 < ||H'|| + 1 \le ||H||$, so $(H,L)$ is not abundant, a contradiction.
	\end{proof}

	\begin{lem}\label{SwappersAreSparse}
		For any $\beta \in \pot(L) \setminus F$ that is swappable with some $\gamma \in F$, we have $\card{G_{L,\beta} - r - t} \le 2$.  Moreover, if $\beta \not \in Y$, then $\card{G_{L,\beta} - r - t} \le 1$.
	\end{lem}
	\begin{proof}
		First, suppose $\beta \in Y$ is swappable with $\gamma \in F$ and $\card{G_{L,\beta} - r - t} \ge 3$.  Pick $v \in V(G_\beta - r - t)$.  Let $X \in \X_{\beta,\gamma}$ with $v \in X$.  Note that $L_X(r) = L(r)$ since $\gamma, \beta \in L(r)$.  Since $\card{X \cap V(G_\beta - r - t)} \le 2$,  we have $\eta_{L_X}(G - t) > \eta_L(G - t)$, which contradicts the maximality of $\eta_L(G - t)$.
		
		Now, suppose $\beta \in \pot(L) \setminus \parens{Y \cup F}$ is swappable with $\gamma \in F$ and $\card{G_{L,\beta} - r - t} \ge 2$.  Let $X \in \X_{\beta,\gamma}$ with $r \in X$.  Since $\card{X \cap V(G_\beta - r - t)} \le 1$, we again contradict the maximality of $\eta_L(G - t)$.
	\end{proof}
	
	\begin{lem}\label{AllSwappableLowPsi}
		If every $\beta \in \pot(L) \setminus F$ is swappable with some $\gamma \in F$, then $\eta_L(G-t) \ge \psi_L(G - t) \ge \size{G - t}$.
	\end{lem}
	\begin{proof}
		Suppose every $\beta \in \pot(L) \setminus F$ is swappable with some $\gamma \in F$.  Then, by Lemma \ref{SwappersAreSparse}, the colors in $Y$ each contribute at most one to $\psi_L(G - t)$ and the colors not in $Y$ contribute nothing to $\psi_L(G - t)$.  Hence $\psi_L(G - t) \le \card{Y} = \eta_L(G-t)$.
	\end{proof}

	\begin{lem}\label{FIntersectingLt}
		If $\card{F \cap L(t)} \ge 2$, then every $\beta \in \pot(L) \setminus F$ is swappable with every $\gamma \in F \cap L(t)$.
	\end{lem}
	\begin{proof}
		Suppose $\card{F \cap L(t)} \ge 2$.  Then $\card{L(r) \cap L(t)} \ge 2$.  Fix $\gamma \in F \cap L(t)$. Let $H \subseteq G$.  If $\card{H_{L, \gamma}}$ is even, then $r, t \in V(H)$ and hence $\psi_L(H) \ge \size{H-t} + 2 \ge \size{H} + 1$.  Therefore $\gamma$ is swappable with $\beta$ by Lemma \ref{SwappableCondition}.
	\end{proof}
	
	\begin{lem}\label{SingleGammaMuchSwappage}
		If $\gamma \in F \setminus L(t)$ and $L(s) \cap L(t) \ne \set{\delta}$, then $\gamma$ and $\delta$ are swappable.
	\end{lem}
	\begin{proof}
			The only subgraph $H$ with edges where $\card{H_{L, \gamma}}$ is even is $G[s, t]$, so if $\gamma$ is not swappable with $\delta$, then it must be $H = G[s, t]$ that fails all conditions of Lemma \ref{SwappableCondition}.  Hence we have $L(s) \cap L(t) = \set{\delta}$.
	\end{proof}
	
	\begin{lem}\label{Odd_delta}
		Suppose there is $\gamma \in F \setminus L(t)$ and $\delta \in L(t) \setminus L(s)$ such that $\card{G_{L, \delta} - t}$ is odd.  Then there is a list assignment $L'$ such that
		\begin{itemize}
			\item $(G, L')$ is superabundant; and
			\item $G$ is not $L'$-subfixable; and
			\item $\eta_{L'}(G - t) = \eta_L(G-t)$; and
			\item $F_{L'}(G) \cap L'(t) \ne \emptyset$.
		\end{itemize}
	\end{lem}
	\begin{proof}
		If $\delta \in F$ then $L$ works for $L'$, so we may assume $\delta \not \in F$.  By Lemma \ref{SingleGammaMuchSwappage}, $\gamma$ and $\delta$ are swappable.
		
		First, suppose $\delta \in Y$. Then, by Lemma \ref{SwappersAreSparse} and since $\card{G_\delta - t}$ is odd, we have $\delta \in L(u) \cap L(w)$ for exactly two $u,w \in N(r) - s$.   Let $X \in \X_{\delta,\gamma}$ with $t \in X$.  We have $L_X(r) = L(r)$ and thus if $|X| = 2$, then $\eta_{L_X}(G - t) > \eta_L(G - t)$, a contradiction.  So, $|X| = 1$ and $L_X$ differs from $L$ only on $t$ where $L_X$ has $\gamma$ instead of $\delta$.  Hence we can use $L' = L_X$.
		
		Otherwise, by Lemma \ref{SwappersAreSparse}, we must have $\delta \in L(u)$ for exactly one $u$ in $N(r) - s$.  Let $X \in \X_{\delta,\gamma}$ with $t \in X$. As before, we conclude $|X| = 1$ and again we can use $L' = L_X$.
	\end{proof}
	
		
		\begin{lem}\label{SpannerSpecial}
			Let $G$ be a bipartite graph with nonempty parts $P$ and $Q$.  If $|P| \le |Q|$ and $Q$ has no isolated vertices, then $G$ contains a nonempty matching $M$ whose vertex set is $S \cup N(S)$ for some $S \subseteq Q$.
		\end{lem}
		
	\begin{lem}\label{TameEnoughColorsHits-t}
		For every $C \subseteq Y$ with $\card{C} \ge \card{N_B(C)}$, we have $C \cap L(t) \ne \emptyset$
	\end{lem}
	\begin{proof}
			Suppose not and let $C \subseteq Y$ with $\card{C} \ge \card{N_B(C)}$ and $C \cap L(t) = \emptyset$. Let $B'$ be the subgraph of $B$ induced on $C \cup N_B(C)$. Then we may apply Lemma \ref{SpannerSpecial} to get a nonempty matching $M$ of $B'$ whose vertex set is $S \cup N_B(S)$ for some $S \subseteq C$.  For each $\set{uw, \alpha} \in M$, color $uw$ with $\alpha$.   Let $G' = G - V(N_B(S) - r)$ and define $L'$ by $L'(v) = L(v) \setminus S$ for $v \in V(G')$.  Then $L'(v) = L(v)$ for $v \in V(G'-r)$. So, $(G', L')$ is superabundant and $|L'(r)| > d_{G'}(r)$.  Therefore, we can apply minimality of $|G|$ to $G'$ to conclude that $G'$ is $L'$-subfixable which implies that $G$ is $L$-subfixable, a contradiction.			
	\end{proof}
	
	\begin{lem}\label{NoPositiveSurplus}
		For every $C \subseteq Y$, we have $\card{C} \le \card{N_B(C)}$. In particular, $\eta_L(G-t) \le \size{G-t}$ and $F \ne \emptyset$.
	\end{lem}
	\begin{proof}
			Suppose not and choose $C \subseteq Y$ such that $\card{C} > \card{N_B(C)}$ so as to minimize $\card{C}$.  For all $\tau \in C$, by minimality of $\card{C}$, we have $N_B(C - \tau) = N_B(C)$.  Since $\card{N_B(C')} \ge \card{C}$ for every $C' \subseteq C - \tau$, Hall's theorem gives a nonempty matching $M_\tau$ whose vertex set is $(C - \tau) \cup N_B(C-\tau) = (C - \tau) \cup N_B(C)$.  So, for every $\tau \in C$, we can color $N_B(C - \tau)$ using $C - \tau$ as in Lemma \ref{TameEnoughColorsHits-t}; the key point is that each of these colorings colors the same edge set.
			
			Put $R = C \cap L(t)$.  By Lemma \ref{TameEnoughColorsHits-t}, $R \ne \emptyset$.  For $\tau \in R$, we have $\card{C - \tau} \ge \card{N_B(C - \tau)}$, so Lemma \ref{TameEnoughColorsHits-t} gives $\card{R} \ge 2$. 
			
			First, suppose $rs \in N_B(C)$. Pick $\tau \in R \cap L(s)$ if possible; otherwise pick $\tau \in R$ arbitrarily. For each $\set{uw, \alpha} \in M_\tau$, color $uw$ with $\alpha$.  Put $G' = G - V(N_B(C) - r - t)$ and $L'(v) = L(v) \setminus (C - \tau)$ for $v \in V(G')$.  Then $L'(v) = L(v)$ for $v \in V(G' - r)$. Then $(G', L')$ is superabundant and $|L'(r)| > d_{G'}(r)$. If we can now color $st$ with a color different than $rs$ received, then by minimality of $|G|$ we conclude that $G'$ is $L'$-subfixable which implies that $G$ is $L$-subfixable, a contradiction. If we cannot color $st$, then $R \cap L(s) \ne \emptyset$ and hence $\tau \in L(s) \cap L(t)$ and $\tau$ is not used on $rs$, so we can color $st$ with $\tau$, a contradiction.  
					
			Hence, we may assume that $rs \not \in N_B(C)$. So, $R \cap L(s) = \emptyset$. Pick $\tau \in R$. For each $\set{uw, \alpha} \in M_\tau$, color $uw$ with $\alpha$.  Put $G' = G - V(N_B(C) - r)$ and $L'(v) = L(v) \setminus (C - \tau)$ for $v \in V(G')$.  We claim that $(G', L')$ is superabundant.  Suppose otherwise that we have $H \subseteq G'$ such that $(H, L')$ is not abundant. Since $\tau \not \in L(s)$, we must have $r, t \in V(H)$.  Now $V(H_\tau - t) = \set{r}$ since $N_B(\tau) \subseteq N_B(C)$.  So, when we add $t$ back in, $\tau$ contributes one to $\psi_{L'}(H)$.  But $(H - t, L')$ is abundant, so $(H, L')$ is abundant, a contradiction.  Since $(G', L')$ is superabundant and $\card{G'} < \card{G}$, by minimality we conclude that $G'$ is $L'$-subfixable which implies that $G$ is $L$-subfixable, a contradiction.
	\end{proof}


	\begin{lem}\label{EtaIsBig}
		$\eta_L(G - t) = \size{G - t}$.
	\end{lem}
	\begin{proof}
		By Lemma \ref{NoPositiveSurplus}, $\eta_L(G - t) \le \size{G - t}$.	So, suppose $\eta_L(G - t) < \size{G - t}$. Then we have $\card{F} \ge 2$. By Lemma \ref{AllSwappableLowPsi}, there is $\beta \in \pot(L) \setminus F$ that is not swappable with any $\gamma \in F$.  Hence, by Lemma \ref{FIntersectingLt} there is $\gamma \in F \setminus L(t)$ and by Lemma \ref{SingleGammaMuchSwappage}, $\gamma$ is swappable with every color in $\pot(L) \setminus \parens{F \cup \set{\beta}}$ and $L(s) \cap L(t) = \set{\beta}$.
		
		\claim{1}{If $\beta \in Y$, then $\card{G_\beta - r - t} = 3$.}
		
			If $\beta \in Y$ and $\card{G_\beta - r - t} \le 2$, then the argument in Lemma \ref{AllSwappableLowPsi} gives $\psi_L(G - t) < \size{G - t}$, a contradiction.
			
			So, suppose $\beta \in Y$ and $\card{G_\beta - r - t} \ge 4$.  Pick $v_1, v_2, v_3 \in V(G_\beta - r - t - s)$.  Then there is $i \in \irange{3}$ and $X \in \X_{\beta,\gamma}$ with $v_i \in X$ such that $X \cap \set{s,t} = \emptyset$. Note that $L_X(r) = L(r)$, $L_X(s) = L(s)$ and $L_X(t) = L(t)$.  Since the only subgraph with edges where $\card{H_{L, \gamma}}$ is even is $G[s, t]$, the argument in Lemma \ref{SwappableCondition} shows that $(G,L_X)$ is superabundant.  Now $\set{\beta, \gamma} \subseteq L(v_1) \cup L(v_2) \cup L(v_3)$, so $\eta_{L_X}(G - t) > \eta_L(G - t)$, which contradicts the maximality of $\eta_L(G - t)$.
			
		\claim{2}{If $\beta \not \in Y$, then $\card{G_\beta - r - t} = 2$.}
		
		If $\beta \not \in Y$ and $\card{G_\beta - r - t} \le 1$, then the argument in Lemma \ref{AllSwappableLowPsi} gives $\psi_L(G - t) < \size{G - t}$, a contradiction.
		
		So, suppose $\beta \not \in Y$ and $\card{G_\beta - r - t} \ge 3$. Pick $v_1, v_2 \in V(G_\beta - r - t - s)$ and let $v_3 = r$.  Then there is $i \in \irange{3}$ and $X \in \X_{\beta,\gamma}$ with $v_i \in X$ such that $X \cap \set{s,t} = \emptyset$.  Since the only subgraph with edges where $\card{H_{L, \gamma}}$ is even is $G[s, t]$, the argument in Lemma \ref{SwappableCondition} shows that $(G,L_X)$ is superabundant.  Now $\set{\beta, \gamma} \cap L(r) \subseteq L(v_1) \cup L(v_2)$, so $\eta_{L_X}(G - t) > \eta_L(G - t)$, which contradicts the maximality of $\eta_L(G - t)$.
		
		\claim{3}{We have $F \cap L(t) \ne \emptyset$.  Pick $\delta \in F \cap L(t)$.}
		
		Remember that our initial choice of $L$ guarantees this if possible.  Suppose $F \cap L(t) = \emptyset$.  By Lemma \ref{SwappersAreSparse} and Lemma \ref{SwappersAreSparse}, the colors in $Y - \beta$ contribute at most $|Y - \beta|$ to $\psi_L(G - t)$.  By Claim 1 and Claim 2, the total contribution of $Y$ and $\beta$ to $\psi_L(G - t)$ is at most $\card{Y} + 1$.  Since nothing else contributes by  Lemma \ref{SwappersAreSparse}, we have $\psi_L(G - t) \le \eta_L(G - t) + 1 \le \size{G} - 1$.  Since $\psi_L(G) \ge \size{G}$, there must be $\delta \in L(t) \setminus L(s)$ such that $\card{G_\delta - t}$ is odd.
		
	    But now we can use Lemma \ref{Odd_delta} to get $L'$ and $F_{L'}(G) \cap L'(t) \ne \emptyset$.  This contradicts our initial choice of $L$.

		\claim{4}{The lemma is true.}
		
			Pick $\tau \in L(s) - \beta$.   We claim that $\delta$ and $\tau$ are swappable.  We know $\tau \not \in L(t)$ since $L(s) \cap L(t) = \set{\beta}$.  Suppose $\tau \not \in L(r)$.  Then, by Lemma \ref{SwappersAreSparse}, $\tau$ appears only in $L(s)$ and we conclude that $\delta$ and $\tau$ are swappable.  So, instead suppose $\tau \in L(r)$.  Then there is at most one $v \in N(r) - s$ with $\tau \in L(v)$ by Lemma \ref{SwappersAreSparse}.  If $\delta$ and $\tau$ are not swappable, then some subgraph of $G[r,s,t]$ must fail all conditions in Lemma \ref{SwappableCondition} since $\tau$ appears an odd number of times in $G[v,r,s,t]$.  But this is impossible since $\psi_L(G[r,s,t]) \ge 3$.  Hence $\delta$ and $\tau$ are swappable.
			
			Suppose $\tau$ appears only on $L(s)$. Then $\set{s,t} \in \X_{\delta,\tau}$ and we get $\eta_{L_{\set{s,t}}}(G - t) > \eta_L(G - t)$, a contradiction.  So, $\tau \in L(r)$.  There is at most one $v \in N(r) - s$ with $\tau \in L(v)$ by Lemma \ref{SwappersAreSparse}.  Suppose there is such a $v$.  Then $\X_{\delta,\tau}$ is one of $\set{\set{v,t}, \set{s}}$, $\set{\set{v}, \set{s, t}}$ or $\set{\set{v,s}, \set{t}}$.  For the first two, using the singleton set for $X$ gives increases $\eta_{L_X}(G - t) > \eta_L(G - t)$, a contradiction.  For the third using $X = \set{t}$ gives $L_X(s) \cap L_X(t) = \set{\beta, \tau}$.  But now $\gamma$ is swappable with every color in $\pot(L_X) \setminus F_{L_X}(G)$ by Lemma \ref{SingleGammaMuchSwappage} and hence $\eta_{L_X}(G-t) \ge \size{G-t} > \eta_L(G-t)$ by Lemma \ref{AllSwappableLowPsi}, a contradiction.
			
			Hence $\tau$ must appear only on $L(r)$ and $L(s)$.  That means that for any induced subgraph $H$ of $G$ containing $r,s,t$ we have $\psi_L(H) \ge 3 + \psi_L(H-s-t) \ge 3 + \size{H-s-t} = \size{H} + 1$.  But now $\delta$ is swappable with every color in $\pot(L) - F$ by Lemma \ref{SwappableCondition} since any subgraph $H$ with edges containing an even number of $\delta$'s must contain $r,s,t$, but then $\psi_L(H) > ||H||$.  Since $\delta \in F$, we have $\eta_{L}(G-t) \ge \size{G-t}$ by Lemma \ref{AllSwappableLowPsi}, a contradiction.
	\end{proof}
	
	
	\begin{lem}\label{CanColorG-t}
		$G-t$ has an $L$-edge-coloring $\pi$.  Also, $\pi(rs) \in L(t)$.
	\end{lem}
	\begin{proof}
		By Lemma \ref{NoPositiveSurplus}, Lemma \ref{EtaIsBig} and Hall's theorem, $B$ has a perfect matching which gives an $L$-edge-coloring of $G-t$.  If $\pi(rs) \not \in L(t)$, then there is another color $\tau \in L(s) \cap L(t)$, so we can complete $\pi$ to $G$, a contradiction.
	\end{proof}
	
	\begin{lem}\label{SIsSmall}
		We have $|L(s)| = 2$.
	\end{lem}
	\begin{proof}
		Suppose $|L(s)| \ge 3$.  Color $G-st$ using $\pi$ from Lemma \ref{CanColorG-t}.  We claim we can order the vertices in $G-t-r$ such that we have a Tashkinov tree with edge $st$ uncolored.  Since both $|L(r)| \ge 3$ and $|L(s)| \ge 3$, this implies that $(G,L)$ is not superabundant, a contradiction.  Now we get the ordering. Start $t,s,r$ noting that $\pi(rs) \in L(t)$ by Lemma \ref{CanColorG-t}.  Now build a sequence $x_1, \ldots, x_m$ inductively by picking $x_i$ such that $\pi(rx_i)$ is missing on one of $t,s,r,x_1, \ldots, x_{i-1}$.  Suppose at some point we get stuck and cannot make such a choice for $x_i$.  Let $E$ be the remaining edges. Then the colors used by $\pi$ on the $E$ appear only on the endpoints of edges in $E$, so we can color $E$ by $\pi$ and remove all those colors from $L(r)$ to get a list assignment $L'$ on $G' = G[t,s,r, x_1, \ldots, x_{i-1}]$ such that $(G', L')$ is superabundant.  But then minimality of $|G|$ shows that $G'$ is $L'$-fixable and hence $G$ is $L$-fixable, a contradiction.  So, we don't get stuck and hence we have our desired Tashkinov tree.
	\end{proof}
	
	\begin{thm}
		Conjecture \ref{StarWithOneEdgeSubdivided}(a) is true (if we assume $|L(v)| \le 2$ for all leaves, can likely remove this restriction with better bipartite graph handling, this is only needed in Claims 4 and 5)
	\end{thm}
	\begin{proof}
	
		\claim{1}{There is a color $\beta \in L(r)$ such that $L(s) \cap L(t) = \set{\beta}$.  For every $L$-edge-coloring $\pi$ of $G-t$, we have $\pi(rs) = \beta$.}
		
		Otherwise, we $L$-edge-color $G-t$ by using Lemma \ref{CanColorG-t} and then use one of the two colors in $L(s) \cap L(t)$ to color $st$, a contradiction. 
		
		\claim{2}{We have $F \cap L(t) = \emptyset$.  In particular, there is no $\delta \in L(t) \setminus L(s)$ such that $|G_\delta - t|$ is odd.}
		
		Suppose otherwise that there is $\gamma \in F \cap L(t)$.  Color the edges of $G - s - t$ via $\pi$ in Lemma \ref{CanColorG-t} and let $L'$ be the resulting list assignment on $rst$.  Then $\beta \in L'(r) \cap L'(s) \cap L'(t)$ and $\gamma \in 'L(r) \cap L'(t)$.  Hence $(G[r,s,t], L')$ is superabundant and thus $G[r,s,t]$ is $L'$-subfixable.  But then $G$ is $L$-subfixable, a contradiction.
		
		The final statement follows from out initial choice of $L$ using Lemma \ref{Odd_delta} with the fact that $F \ne \emptyset$ by Lemma \ref{NoPositiveSurplus}.
		
		\claim{3}{If $\delta \in L(t) - \beta$ with $|G_\delta| \ge 2$, then $\delta \in L(r)$ and $|G_\delta| = 3$.  Also, $|G_\beta| \ge 4$.}
		
		Let $\delta \in L(t)$ with $|G_\delta| \ge 2$.  Then $|G_\delta|$ is even by Claim 2.  So $|G_\delta| \ge 3$ and then Lemma \ref{SwappersAreSparse} shows that $\delta \in L(r)$ and $|G_\delta| = 3$.  
		
		Since $\eta_L(G-t) = \size{G-t}$ by Lemma \ref{EtaIsBig}, Lemma \ref{SwappersAreSparse} shows that $\psi_L(G-t) \le \eta_L(G-t) = \size{G-t}$.  Since $(G,L)$ is superabundant, $\psi_L(G) \ge \size{G}$ so we need one more, as we just saw the only way to get this is from $\beta$.  So we have $|G_\beta| \ge 4$.
		
		\claim{4}{We have $L(r) \cap L(s) = \set{\beta}$.}
		
			Then in the bipartite graph $B$, the vertex $rs$ has degree at least two.  By Claim 1, Lemma \ref{SingleGammaMuchSwappage} and Lemma \ref{SwappersAreSparse}, every $\alpha \in Y - \beta$ has degree at most two in $B$.  Let $\pi$ be a $L$-edge-coloring of $G-st$ from Lemma \ref{CanColorG-t}.  Then $\pi$ specifies a perfect matching $M$ in $B$.  Consider a maximum length path $P$ in $B$ starting at $rs$ alternating between edges in $M$ and edges not in $M$.  
			If $P$ ends in $Y$, then by swapping the $M$-edges for the non-$M$-edges and $rs$, we get a perfect matching $M'$ of $B$ that gives an $L$-edge-coloring $\pi'$ of $G-t$ with $\pi'(rs) \ne \beta$, contradicting Claim 1.  So, $P$ must end at some edge $rw$ of $G - t$.   So, $\beta \not \in L(v)$.  Let's say $v_1 = s$ and $v_m = w$ and $P$ is $rv_1,\tau_2, rv_2, \tau_3, \ldots, \tau_m, rv_m$.  Consider $G' = G[r,t,v_1, v_2, \ldots, v_m]$.  If $\beta \in L(v_i)$ for any $i > 1$ then we can again swap the $M$-edges and non-$M$-edges to win.   So, in $G'$ we have that $\beta$ only appears on $r$, $s$, and $t$.  Also, as noted above, every $\tau_i$ has degree at most two in $B$ and hence $\tau_i$ appears at most three times in $G'$.  Since every $\alpha \not \in Y$ appears at most once in $G'$ by Lemma \ref{SwappersAreSparse}, we conclude that $\psi_L(G') = m$.  But $||G'|| = m + 1$, so $(G', L)$ is not abundant, a contradiction.

		\claim{5}{The theorem is true.}
		
			 Since $L(r) \cap L(s) = \set{\beta}$ by Claim 1, there must be $\alpha \in L(r) \cap L(t)$ since $(G[r,s,t],L)$ is abundant and $L(s) \cap L(t) = \set{\beta}$ by Claim 4.  Also, by Lemma \ref{SwappersAreSparse}, any $\tau \in L(s) - \beta$ appears in exactly one list.  By Claim 3, $|G_\alpha| = 3$.  Let $v \in N(r) - s$ have $\alpha \in L(v)$.  Then to make $(G[v,r,s,t], L)$ abundant, there must be $\delta \in L(r) \cap L(v) - \alpha$ (another color in common between $L(r)$ and $L(t)$ cannot happen because we are assuming $|L(t)| = 2$).  Also, since $\delta \not \in L(t)$ (again using $|L(t)| = 2$), there must be $w \in N(r) \setminus \set{v,s}$ with $\delta \in L(w)$ for otherwise, we color $rv$ with $\delta$ and apply minimality of $|G|$ to win. If $L(r) \cap L(w) - \delta = \emptyset$, then $(G[w,v,r,s,t], L)$ is not abundant (again we are using $|L(v)| \le 2$ and $|L(t)| \le 2$).  So, there is $\rho_1 \in L(r) \cap L(w) - \delta$, it could be that $\rho_1 = \beta$.  If $\rho_1 \ne \beta$, we can repeat the argument with $\rho_1$ in place of $\delta$ to get $w_2 \in N(r) - \set{w, v,s}$ with $\rho_1 \in L(w_2)$.  Continuing this way, at some point we end with $w_m$ where $L(w_m) = \set{\rho_{m-1}, \beta}$.

			 Suppose $|G_\beta| = 4$. Then $\beta$ and $\tau$ are swappable since the only subgraph with edges and an even number of both $\beta$ and $\tau$ is $G[r,w_m]$, but $\psi_L(G[r,w_m]) \ge 2$ because $\rho_{m-1} \in L(r) \cap L(w_m)$.  Now $\card{S_{L, \beta, \tau}}$ is odd, so one of $\set{w_m}$, $\set{r}$ or $\set{t}$ is in $X_{\beta, \tau}$.  If it is either $\set{r}$ or $\set{t}$, then we can color $G$.  So, we must have $\set{w_m} \in X_{\beta, \tau}$.  But then $L_{\set{w_m}}$ contradicts Lemma \ref{SwappersAreSparse}.
			 
			 Therefore $|G_\beta| \ge 5$.  By the same argument as in Claim 1 of Lemma \ref{EtaIsBig} we also have $|G_\beta| \le 5$.  So, $|G_\beta| = 5$.  Let $z \in N(r) \setminus \set{w_m, s}$ with $\beta \in L(z)$.  Since $(G[z,r,s], L)$ is abundant, there must be $\zeta \in L(z) \cap L(r) - \beta$.  Now $\beta$ and $\tau$ are swappable since the only subgraphs with edges that have an even number of both $\beta$ and $\tau$ are $G[r,z]$ and $G[r,w_m]$, but both of these have $\psi_L$ at least two.  If $\set{z, t}$ or $\set{w_m,t}$ is in $X_{\beta, \tau}$ then after swapping we can color $G$ (just color $st$ with $\tau$).  So, we must have $\set{z,w_m} \in X_{\beta, \tau}$.   But then $L_{\set{z,w_m}}$ contradicts Lemma \ref{SwappersAreSparse}.
	\end{proof}
\end{document}

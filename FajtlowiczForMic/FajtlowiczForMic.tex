\documentclass[12pt]{article}
\usepackage{amsmath, amsthm, amssymb}
\usepackage{tkz-graph}
\usepackage{marginnote}
\usepackage{verbatim}
\usepackage[top=1.0in, bottom=1.0in, left=1.0in, right=1.0in]{geometry}
\usepackage{color}
\pagestyle{plain}

\usepackage[backref=page]{hyperref}

\usepackage{sectsty}
\allsectionsfont{\sffamily}

\setcounter{secnumdepth}{5}
\setcounter{tocdepth}{5}

\makeatletter
\newtheorem*{rep@theorem}{\rep@title}
\newcommand{\newreptheorem}[2]{
\newenvironment{rep#1}[1]{
 \def\rep@title{#2 \ref{##1}}
 \begin{rep@theorem}}
 {\end{rep@theorem}}}
\makeatother

\theoremstyle{plain}
\newtheorem{thm}{Theorem}[section]
\newreptheorem{thm}{Theorem}
\newtheorem{prop}[thm]{Proposition}
\newreptheorem{prop}{Proposition}
\newtheorem{lem}[thm]{Lemma}
\newreptheorem{lem}{Lemma}
\newtheorem{conjecture}[thm]{Conjecture}
\newreptheorem{conjecture}{Conjecture}
\newtheorem{cor}[thm]{Corollary}
\newreptheorem{cor}{Corollary}
\newtheorem{prob}[thm]{Problem}

\newtheorem*{KernelLemma}{Kernel Lemma}
\newtheorem*{BK}{Borodin-Kostochka Conjecture}
\newtheorem*{BK2}{Borodin-Kostochka Conjecture (restated)}
\newtheorem*{Reed}{Reed's Conjecture}
\newtheorem*{ClassificationOfd0}{Classification of $d_0$-choosable graphs}


\theoremstyle{definition}
\newtheorem{defn}{Definition}
\theoremstyle{remark}
\newtheorem*{remark}{Remark}
\newtheorem*{problem}{Problem}
\newtheorem{example}{Example}
\newtheorem*{question}{Question}
\newtheorem*{observation}{Observation}

\newcommand{\fancy}[1]{\mathcal{#1}}
\newcommand{\C}[1]{\fancy{C}_{#1}}


\newcommand{\IN}{\mathbb{N}}
\newcommand{\IR}{\mathbb{R}}
\newcommand{\G}{\fancy{G}}
\newcommand{\CC}{\fancy{C}}
\newcommand{\D}{\fancy{D}}
\newcommand{\T}{\fancy{T}}
\newcommand{\B}{\fancy{B}}
\renewcommand{\L}{\fancy{L}}
\newcommand{\HH}{\fancy{H}}

\newcommand{\inj}{\hookrightarrow}
\newcommand{\surj}{\twoheadrightarrow}

\newcommand{\set}[1]{\left\{ #1 \right\}}
\newcommand{\setb}[3]{\left\{ #1 \in #2 : #3 \right\}}
\newcommand{\setbs}[2]{\left\{ #1 : #2 \right\}}
\newcommand{\card}[1]{\left|#1\right|}
\newcommand{\size}[1]{\left\Vert#1\right\Vert}
\newcommand{\ceil}[1]{\left\lceil#1\right\rceil}
\newcommand{\floor}[1]{\left\lfloor#1\right\rfloor}
\newcommand{\func}[3]{#1\colon #2 \rightarrow #3}
\newcommand{\funcinj}[3]{#1\colon #2 \inj #3}
\newcommand{\funcsurj}[3]{#1\colon #2 \surj #3}
\newcommand{\irange}[1]{\left[#1\right]}
\newcommand{\join}[2]{#1 \mbox{\hspace{2 pt}$\ast$\hspace{2 pt}} #2}
\newcommand{\djunion}[2]{#1 \mbox{\hspace{2 pt}$+$\hspace{2 pt}} #2}
\newcommand{\parens}[1]{\left( #1 \right)}
\newcommand{\brackets}[1]{\left[ #1 \right]}
\newcommand{\DefinedAs}{\mathrel{\mathop:}=}

\newcommand{\mic}{\operatorname{mic}}
\newcommand{\AT}{\operatorname{AT}}
\newcommand{\col}{\operatorname{col}}
\newcommand{\ch}{\operatorname{ch}}
\newcommand{\type}{\operatorname{type}}
\newcommand{\nonsep}{\bar{S}}

\def\adj{\leftrightarrow}
\def\nonadj{\not\!\leftrightarrow}

\newcommand\restr[2]{{% we make the whole thing an ordinary symbol
  \left.\kern-\nulldelimiterspace % automatically resize the bar with \right
  #1 % the function
  \vphantom{\big|} % pretend it's a little taller at normal size
  \right|_{#2} % this is the delimiter
  }}

\def\D{\fancy{D}}
\def\C{\fancy{C}}
\def\A{\fancy{A}}
\def\chil{{\chi_\ell}}
\def\chiol{\chi_{\rm{OL}}}

\newcommand{\case}[2]{{\bf Case #1.}~{\it #2}~~}
\newcommand{\aside}[1]{\marginnote{\scriptsize{#1}}[0cm]}
\newcommand{\aaside}[2]{\marginnote{\scriptsize{#1}}[#2]}

\title{Generalizing Fajtlowicz to mic}

\begin{document}
\maketitle
\begin{abstract}
Generalizing Fajtlowicz's lower bound on the independence number, we show that every graph $G$ has an independent set incident to at least $2|G| - \alpha(G)(\omega(G) + 1)$ edges.  
Combined with the magic kernel lemma, this implies lower bounds on the average degree of online $k$-list-critical graphs.  In particular,  an online $k$-list-critical graph with $n$ vertices, independence number $\alpha$ and clique number $\omega$ has average degree at least $k - \frac{\alpha(\omega+1)}{n}$ and also has average degree at least $k-1 + \frac{k - \omega - 2}{k + \omega}$. 
\end{abstract}

\section{Introduction}
For a graph $G$ and disjoint $A, B \subseteq V(G)$, let $\size{A,B}$ be the number of edges between $A$ and $B$.

\begin{defn} The \emph{maximum independent cover number }of a graph $G$
	is the maximum $\mic(G)$ of $\size{I, V(G) \setminus I}$ over all independent sets $I$
	of $G$. 
\end{defn}

\section{The bound on mic}

\begin{thm}\label{FajtlowiczMic}
Every graph $G$ satisfies $\mic(G) \ge 2|G| - \alpha(G)(\omega(G) + 1)$.
\end{thm}
\begin{proof}
	Let $t$ be the maximum $\mic(G)$ of $\size{M, V(G) \setminus M}$ over all \emph{maximum} independent sets $M$
	of $G$.   Then $\mic(G) \ge t$, so it suffices to show that $t \ge 2|G| - \alpha(G)(\omega(G) + 1)$.  Let $A$ be a maximum independent set in $G$ with $\size{A,V(G)\setminus A} = t$.  For $i \in \irange{\Delta(G)}$,
	let $c_i$ be the number of vertices in $V(G) \setminus A$ with exactly $i$ neighbors in $A$.  Then
	\begin{equation}\label{eqn1}
		\sum_{i \in \irange{\Delta(G)}} c_i = \card{G} - \card{A},
	\end{equation}
	and
		\begin{equation}\label{eqn2}
		\sum_{i \in \irange{\Delta(G)}} i c_i = t.
		\end{equation}
	Subtracting twice \ref{eqn1} from \ref{eqn2} gives
	\begin{equation}\label{eqn3}
	  -c_1 \le -c_1 + \sum_{i \in \irange{\Delta(G) - 2}} ic_{i+2} = t - 2\card{G} + 2\card{A}.
	\end{equation}
	If $x,y \in V(G) \setminus A$ with $N(x) \cap A = N(y) \cap A = \set{z}$, then $x$ and $y$ are adjacent since otherwise $A \cup \set{x,y} \setminus \set{z}$ is an independent set in $G$ which is larger than $A$.  Therefore
	\begin{equation}\label{eqn4}
		c_1 \le (\omega(G) - 1)|A|.
	\end{equation}
	Plugging \ref{eqn4} into \ref{eqn3} gives
	\[-(\omega(G) - 1)|A| \le t - 2\card{G} + 2\card{A},\]
	and hence
	\[t \ge 2\card{G} - \alpha(G)(\omega(G) + 1).\]
\end{proof}

When $\delta(G) \ge \frac{\Delta(G)}{2}$, a similar argument works using $t = \mic(G)$, which gets a better bound when some non-maximum independent set $A$ is incident to the maximum number of edges.  The bound can also be improved when $c_i > 0$ for some $i \ge 3$ since we just disregarded those terms.

\begin{cor}\label{AlphaGone}
	Every graph $G$ satisfies $\mic(G) \ge \frac{2\delta(G)}{\delta(G) + \omega(G) + 1}\card{G}$.
\end{cor}
\begin{proof}
		For every graph $G$,
		\begin{equation}\label{triveqn}
			\mic(G) \ge \delta(G)\alpha(G).
		\end{equation}
Adding $\frac{\omega(G) + 1}{\delta(G)}$ times \ref{triveqn} to the bound in Theorem \ref{FajtlowiczMic} gives
\[\parens{1 + \frac{\omega(G) + 1}{\delta(G)}}\mic(G) \ge 2\card{G},\]
which is
\[\mic(G) \ge \frac{2\delta(G)}{\delta(G) + \omega(G) + 1}\card{G}.\]
\end{proof}

\section{Online list coloring}

\begin{thm}\label{ConsantListMicStrength} 
	Every OC-irreducible graph $G$ satisfies
	$$\mic(G)\leq2\size{G}-(\delta(G)-1)\card{G}-1.$$
	\end{thm}
Combining Theorem \ref{ConsantListMicStrength} with Theorem \ref{FajtlowiczMic} gives the following.

\begin{cor}\label{SizeBound1}
	Every OC-irreducible graph has average degree at least
	\[2\size{G} \ge (\delta(G) + 1)\card{G} + 1 - \alpha(G)(\omega(G) + 1).\]
\end{cor}

Combining Theorem \ref{ConsantListMicStrength} with Corollary \ref{AlphaGone} gives the following.

\begin{cor}\label{SizeBound2}
	Every OC-irreducible graph $G$ satisfies
	\[2\size{G} \ge \parens{\delta(G) + \frac{\delta(G) - \omega(G) - 1}{\delta(G) + \omega(G) + 1}}\card{G} + 1.\]
\end{cor}

\bibliographystyle{amsplain}
\bibliography{GraphColoring1}
\end{document} 
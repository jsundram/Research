\documentclass[12pt]{article}
\usepackage{amssymb, amsmath, amsthm}
\usepackage{fullpage}
\usepackage[hidelinks]{hyperref}
\usepackage{epigraph}
\usepackage{lmodern}
\usepackage[T1]{fontenc}
\usepackage{xspace}
\usepackage{termcal}
\usepackage{color}
\usepackage{fmtcount}

\makeatletter
\DeclareRobustCommand{\maybefakesc}[1]{%
  \ifnum\pdfstrcmp{\f@series}{\bfdefault}=\z@
    {\fontsize{\dimexpr0.8\dimexpr\f@size pt\relax}{0}\selectfont\uppercase{#1}}%
  \else
    \textsc{#1}%
  \fi
}
\newcommand\AM{\,\maybefakesc{am}\xspace}
\newcommand\PM{\,\maybefakesc{pm}\xspace}
\makeatother


\newcommand{\MWFClass}{%
\calday[Monday]{\classday}
\skipday % Tuesday (no class)
\calday[Wednesday]{\classday}
\skipday % Thursday (no class)
\calday[Friday]{\classday}
\skipday\skipday % weekend (no class)
}

\newcommand{\MTRClass}{%
	\calday[Monday]{\classday}
	\calday[Tuesday]{\classday}
	\skipday
	\calday[Thursday]{\classday}
	\skipday
	\skipday\skipday % weekend (no class)
}

\newcommand{\Holiday}[2]{%
\options{#1}{\noclassday}
\caltext{#1}{#2}
}


\pagestyle{plain}
\title{Math 109: Calculus 1\\ \bigskip\small{Fall 2015}}
\date{}
\begin{document}
\maketitle

\begin{tabular}{r l}
\textbf{instructor:}& landon rabern\\
\textbf{email:}& \href{mailto:lrabern@fandm.edu}{\nolinkurl{lrabern@fandm.edu}}\\
\textbf{class webpage:}& \href{http://bit.ly/FandM2015}{\nolinkurl{bit.ly/FandM2015}}\\
\textbf{class meetings:}& 12:30\PM--1:20\PM M in STA 316 \\
& 8:30\AM--9:50\AM TR in STA 112\\
\textbf{office hours:}& 1:20\PM--2:50\PM M in STA 316\\
\textbf{textbook:}&\textit{Calculus, Concepts and Contexts 4E}, by James Stewart\\
\end{tabular}

\bigskip

\section*{Why Calculus?}
For the purposes of this class, assume our universe runs on a computer with finite memory.  One way this could work is for both space and time to be \emph{discrete}; that is, there is a smallest possible distance $\Delta d$ between any two objects and there is a smallest unit of time $\Delta t$.  Let's work with a simple model where the universe is broken up into little cubes with side length $\Delta d$.  Objects (or parts of objects) must be located within some cube, there is no in between and in one time step an object can only move to an adjacent cube.  

Ok, so that's our universe.  Say you throw a rock straight up and want to know when it will come back down so you can look up from your phone and catch it.  If you threw the rock up with speed $v$ (in units of $\Delta d$ per $\Delta t$) and gravity increases its downward speed at a rate of $a$ (in units of $\Delta d$ per $\Delta t$ per $\Delta t$), then with a little work we can compute the equation for its height above the ground at time $t$ as:
\[h(t) = vt - \frac{a}{2}t^2 + \frac{a}{2}t\Delta t.\]
Setting $h(t)$ to whatever height you want to catch the rock at and solving for $t$ will give you your answer.  But your answer will contain $\Delta t$.  There are a few of problems with this:
\begin{enumerate}
	\itemsep 0em 
	\item we don't know the value of $\Delta t$,
	\item if we want to compute more complicated things like planetary orbits and rocket trajectories, these extra terms with $\Delta t$ will proliferate making a giant complicated mess,
	\item often the mess in (2) will lead to equations that we cannot solve exactly.
\end{enumerate}
But, we really want a usable way to compute rocket trajectories, so what can we do?  Well, fortunately we appear to live in a universe where $\Delta t$ is very small, so we can get very accurate approximations by replacing all $\Delta t$'s with some small value like a nanosecond.  That solves problem (1). To solve problems (2) and (3), we need to completely get rid of the $\Delta t$ terms.  In the 17th century, this problem was solved by the inventors of Calculus by using \emph{infinitesimals} which are new numbers that are bigger than zero but smaller than every positive real number.  While these seemed to work, they were not completely understood which led to mistakes in their application.  It wasn't until the 20th century that mathematicians gave a complete account of infinitesimals.  In the intervening years the concept of a \emph{limit} was introduced to replace the use of infinitesimals in getting rid of the $\Delta t$ terms.  Using limits we get an approximation to our function $h$ that solves (2) and (3):
\[h(t) = vt - \frac{a}{2}t^2 + \frac{a}{2}t\Delta t \approx vt - \frac{a}{2}t^2.\]
In effect, we have taken $\Delta t$ down to zero and we are modeling our discrete universe by a \emph{continuous} universe.  Calculus is the collection of tools we need to work with these continuous models.

\section*{Homework} 
\epigraph{I can only show you the door. You're the one that has to walk through it.}{}
To achieve fluency in this subject, you will need to immerse yourself in the material.  
Working tons of problems is a great way to do this.  How many problems?  
My recommendation is to work problems of a given type until they become easy for you.

I will put a list of practice problems for each class period on the class webpage. 
To encourage you to make working problems a regular activity, you will need to maintain a journal containing your practice work.
These journals will be turned in periodically for inspection.  There are many ways you could structure such a journal, we will go over some
basic guidelines in class.

Each Thursday, I will select a couple of the more interesting problems and assign them---due the following Thursday.
These will be graded both for correctness and clarity of exposition.
\newpage
\section*{Quizzes}
\epigraph{Pop quiz, hotshot.}{}
There will be tiny quizzes at random times throughout the course.  I will set the random number generator so that the expected number of quizzes is 10.
Quizzes are intended to reinforce basic concepts as well as encourage attendance.
Unlike exams, quizzes will be closed-book.  Your lowest quiz score will be dropped.

\section*{Computing devices}
We will be doing a lot of estimation, so you will need a graphing calculator.  This should be a separate device from your phone/tablet since you will not be allowed to use any device on exams that is capable of wireless communication.

\section*{Exams}
There will be two in-class exams and then a final exam during finals week. 
The purpose of the exams is to test your understanding of, and ability to reason about, the mathematical concepts. Since you can use your textbook as well as any other written material, no memorization is required; however, these exams occur in a finite time period, so rapid recall of facts will serve you well. 

\section*{Graded work breakdown}
\begin{tabular}{r | r | l}
what & \% & when \\
\hline
journal & 10 & TBA \\
graded homework & 10 & weekly\\
quizzes & 10 & random times\\
in-class exam \#1 & 15 & Thursday, October \ordinalnum{1} \\
in-class exam \#2 & 25 & Tuesday, November \ordinalnum{3} \\
final exam & 30 & TBA, in finals week \\
\end{tabular}

\section*{Help}
If you need help or just want to know more about something, please come to my scheduled office hours or set up another time to meet. In addition to my office hours, there are several undergraduate mathematics teaching assistants who hold regular hours.

\section*{Attendance}
Please be advised that Math Department and F\&M policy state that penalties (including
grade reduction and/or dismissal from the course) may be assessed for
excessive, unexcused absences.

\section*{Tentative Schedule}
\begin{center}
\begin{calendar}{8/31/2015}{16}
\setlength{\calboxdepth}{.3in}
\MTRClass

\options{8/31/2015}{\noclassday} 
\options{9/1/2015}{\noclassday} 

% schedule
\caltexton{1}{introduction}
\caltextnext{1.1, 1.2, 1.3 review}
\caltextnext{2.1 tangents and velocity}
\caltextnext{2.2 limits}
\caltextnext{2.3 limit laws}
\caltextnext{2.4 continuity}
\caltextnext{2.5 infinite limits}
\caltextnext{2.6 rates of change}
\caltextnext{2.7 derivatives}
\caltextnext{2.8 what does the derivative tell us?}
\caltextnext{\textcolor{blue}{exam \#1 review}}
\caltextnext{\textcolor{blue}{exam \#1 review}}
\caltextnext{\textcolor{blue}{\textbf{in-class exam \#1}}}
\caltextnext{3.1 derivatives of polynomials}
\caltextnext{1.5, 3.1 derivatives of exponentials}
\caltextnext{3.2 product and quotient rule}
\caltextnext{3.3, Appendix C derivatives of trig functions}
\caltextnext{3.4 chain rule}
\caltextnext{3.5 implicit differentiation}
\caltextnext{1.6, 3.6 derivatives of inverse trig functions}
\caltextnext{3.7 derivatives of logarithms}
\caltextnext{Taylor series, imaginary numbers, and magic}
\caltextnext{\textcolor{blue}{exam \#2 review}}
\caltextnext{\textcolor{blue}{exam \#2 review}}
\caltextnext{\textcolor{blue}{\textbf{in-class exam \#2}}}
\caltextnext{4.2, 4.3 max and min values}
\caltextnext{4.2, 4.3 max and min values}
\caltextnext{4.2, 4.3 max and min values}
\caltextnext{4.6 optimization}
\caltextnext{projectiles}
\caltextnext{projectiles}
\caltextnext{4.8 antiderivatives}
\caltextnext{5.1 areas and distances}
\caltextnext{5.2 definite integrals}
\caltextnext{5.2 definite integrals}
\caltextnext{5.3 evaluating definite integrals}
\caltextnext{5.3 evaluating definite integrals}
\caltextnext{\textcolor{blue}{final exam review}}
\caltextnext{\textcolor{blue}{final exam review}}
\caltextnext{\textcolor{blue}{final exam review}}

\Holiday{10/10/2015}{Fall Break}
\Holiday{10/11/2015}{Fall Break}
\Holiday{10/12/2015}{Fall Break}
\Holiday{10/13/2015}{Fall Break}

\Holiday{11/25/2015}{Thanksgiving}
\Holiday{11/26/2015}{Thanksgiving}
\Holiday{11/27/2015}{Thanksgiving}
\Holiday{11/28/2015}{Thanksgiving}
\Holiday{11/29/2015}{Thanksgiving}

\Holiday{12/12/2015}{Reading Day}
\Holiday{12/13/2015}{Reading Day}
\Holiday{12/14/2015}{Reading Day}
\Holiday{12/15/2015}{Reading Day}

\Holiday{12/16/2015}{Exam Week}
\Holiday{12/17/2015}{Exam Week}
\Holiday{12/18/2015}{Exam Week}
\Holiday{12/19/2015}{Exam Week}
\Holiday{12/20/2015}{Exam Week}

\end{calendar}
\end{center}
\end{document}
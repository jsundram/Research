% These lines can probably stay unchanged, although you can remove the last
% two packages if you're not making pictures with tikz.
\documentclass[11pt]{exam}
\RequirePackage{amssymb, amsfonts, amsmath, latexsym, verbatim, xspace, setspace}
\RequirePackage{tikz, pgflibraryplotmarks}

% By default LaTeX uses large margins.  This doesn't work well on exams; problems
% end up in the "middle" of the page, reducing the amount of space for students
% to work on them.
\usepackage[margin=1in]{geometry}


% Here's where you edit the Class, Exam, Date, etc.
\newcommand{\class}{Math 109}

% For an exam, single spacing is most appropriate
\singlespacing
% \onehalfspacing
% \doublespacing

% For an exam, we generally want to turn off paragraph indentation
\parindent 0ex

\newcommand*{\TrueFalse}[1]{%
\ifprintanswers
    \ifthenelse{\equal{#1}{T}}{%
        \textbf{TRUE}\hspace*{14pt}False
    }{
        True\hspace*{14pt}\textbf{FALSE}
    }
\else
    {True}\hspace*{20pt}False
\fi
} 
%% The following code is based on an answer by Gonzalo Medina
%% http://tex.stackexchange.com/a/13106/39194
\newlength\TFlengthA
\newlength\TFlengthB
\settowidth\TFlengthA{\hspace*{1.16in}}
\newcommand\TFQuestion[2]{%
    \setlength\TFlengthB{\linewidth}
    \addtolength\TFlengthB{-\TFlengthA}
    \parbox[t]{\TFlengthA}{\TrueFalse{#1}}\parbox[t]{\TFlengthB}{#2}}

\begin{document} 

% These commands set up the running header on the top of the exam pages
\pagestyle{head}

\begin{flushright}
\begin{tabular}{p{2.8in} r l}
\textbf{\class} & \textbf{Practice \#2} & \makebox[2in]{}\\
\end{tabular}\\
\end{flushright}
\rule[1ex]{\textwidth}{.1pt}

\begin{questions}
\question Differentiate each of the following functions with respect to $x$.
\begin{parts}
\part $f(x) = (x + \pi)^3$.
\part $g(x) = e^{\left(x^2 + \ln(x)\right)}$.
\part $r(x) = \ln(\arctan(x^2))$.
\part $f(x) = 2^x\ln(x)$.
\part $f(x) = x^{\sin(x)}$.
\end{parts}
\bigskip\bigskip\bigskip\bigskip\bigskip
	\question Find the equation of the line tangent to the ellipse $2x^2 + 3y^2 = 11$ at the point $(2,1)$.
	\bigskip\bigskip\bigskip\bigskip
	\question If $\sin(y) = \cos(x)$, what is $\frac{dy}{dx}$?
		\bigskip\bigskip\bigskip\bigskip

\question Circle True or False for each question.  
\begin{parts}
	\part\TFQuestion{T}{$V = 2^{\log_2(V)}$ for every positive integer $V$.}
\part\TFQuestion{T}{$\arctan(1) = \frac{\pi}{4}$.}
\part\TFQuestion{T}{$\frac{d}{dx}(\sqrt{x + y}) = \frac{d}{dx}(\sqrt{x}) + \frac{d}{dx}(\sqrt{y})$ for every positive integer $y$.}
\part\TFQuestion{T}{$f(x) = \pi x^{3} - 2^{\pi}x^{6}$ is not differentiable at $x = \pi - 2^{\pi}$.}
\part\TFQuestion{T}{If $n$ is divisible by $4$, then the $n$-th derivative of $\sin(x)$ with respect to $x$ is $\cos(x)$.}
\part\TFQuestion{T}{$\pi = \frac{7^7}{4^9}$.}
\part\TFQuestion{T}{$\sqrt{x + y} = \sqrt{x} + \sqrt{y}$ for all non-negative real numbers $x$ and $y$ with $x + y = 0$.}
\part\TFQuestion{T}{$\frac{d}{dy}(y^y) = yy^{y-1}$.}
\part\TFQuestion{T}{The correct answer to this question is neither `True' nor `False'.}
\end{parts}

\end{questions}
\end{document}
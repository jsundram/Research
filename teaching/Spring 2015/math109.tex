\documentclass[12pt]{article}
\usepackage{amssymb, amsmath, amsthm}
\usepackage{fullpage}
\usepackage[hidelinks]{hyperref}
\usepackage{epigraph}
\usepackage{lmodern}
\usepackage[T1]{fontenc}
\usepackage{xspace}
\usepackage{termcal}
\usepackage{color}
\usepackage{fmtcount}

\makeatletter
\DeclareRobustCommand{\maybefakesc}[1]{%
  \ifnum\pdfstrcmp{\f@series}{\bfdefault}=\z@
    {\fontsize{\dimexpr0.8\dimexpr\f@size pt\relax}{0}\selectfont\uppercase{#1}}%
  \else
    \textsc{#1}%
  \fi
}
\newcommand\AM{\,\maybefakesc{am}\xspace}
\newcommand\PM{\,\maybefakesc{pm}\xspace}
\makeatother


\newcommand{\MWFClass}{%
\calday[Monday]{\classday}
\skipday % Tuesday (no class)
\calday[Wednesday]{\classday}
\skipday % Thursday (no class)
\calday[Friday]{\classday}
\skipday\skipday % weekend (no class)
}

\newcommand{\Holiday}[2]{%
\options{#1}{\noclassday}
\caltext{#1}{#2}
}


\pagestyle{plain}
\title{Math 109: Calculus 1\\ \bigskip\small{Spring 2015}}
\date{}
\begin{document}
\maketitle

\begin{tabular}{r l}
\textbf{instructor:}& landon rabern\\
\textbf{email:}& \href{mailto:lrabern@fandm.edu}{\nolinkurl{lrabern@fandm.edu}}\\
\textbf{class webpage:}& \href{http://bit.ly/calculus2015}{\nolinkurl{bit.ly/calculus2015}}\\
\textbf{class meetings:}& 9:00\AM--9:50\AM MWF in STA 217 (lecture / group work)\\
& 1:15\PM--2:05\PM T in HAC 412 (working problems)\\
\textbf{office hours:}& 10:00\AM--10:50\AM MWF in STA 108\\
\textbf{textbook:}&\textit{Calculus, Concepts and Contexts 4E}, by James Stewart\\
\end{tabular}

\section*{What is this class about?}
You are driving your car down a lonely highway and out of pure boredom you start thinking about how far you have traveled and how long it has taken.  You remember
that semi an hour back that cut you off and made you let up on the gas, how much of your time did that waste?  What about all the other times you have braked or accelerated?
We can answer any such question by playing with functions.  Say $d(t)$ is the distance you have traveled after $t$ hours where $t = 0$ is when you got on the highway.   Say $s(t)$ is the value on your speedometer at time $t$ in miles per hour.  When $s(t)$ is small (when you are driving slow) $d(t)$ increases more slowly and when $s(t)$ is large (when you are driving fast) $d(t)$ increases more quickly.  So, to answer your original questions, what we need is a better understanding of the relationship between $d(t)$ and $s(t)$. This relationship is precisely what \emph{Calculus} is about. We study a procedure for determining $s(t)$ given $d(t)$ similarly to how your speedometer works (computing your average speed over a short period of time); in fancy math words, this procedure is called \emph{differentiation} and the function $s$ is the \emph{derivative} of the function $d$. We study this procedure for arbitrary functions. This might seem much more general, but if you move your foot just right on the gas pedal (and perhaps shift into reverse), you can make $d(t)$ any function you like.  So, in reality, the whole course is just about driving your car.

The procedure for getting $d(t)$ given $s(t)$ is the main topic of Calculus 2, we will see a little of this at the end of the course.  The idea is that if we add up our speed over time, we get how far we have traveled. Again switching to fancy math words, this adding up procedure is called \emph{integration} and the function $d$ is the \emph{integral} of the function $s$.  Differentiation and integration are inverses of each other. If we have $s(t)$ and apply our integration procedure, we get $d(t)$.  If we then apply our differentiation procedure we get $s(t)$ back.  That the procedures ``add speeds up'' and ``take average speed over short time period'' are inverses of each other is The Fundamental Theorem of Calculus.


\section*{Homework} 
\epigraph{I can only show you the door. You're the one that has to walk through it.}{}
To achieve fluency in this subject, you will need to immerse yourself in the material.  
Working tons of problems is a great way to do this.  How many problems?  
My recommendation would be to work problems of a given type until they become easy for you.

I will put a list of practice problems for each class period on the class webpage.  These will not be collected.  
Each Wednesday, I will select a couple of the more interesting problems and assign them---due the following Wednesday.
These will be graded both for correctness and clarity of exposition.

\section*{Quizzes}
\epigraph{Pop quiz, hotshot.}{}
There will 5 minute quizzes at random times throughout the course.  I will set the random number generator so that the expected number of quizzes is 10.
Quizzes are intended to reinforce basic concepts as well as encourage attendance.
Unlike exams, quizzes will be closed-book.  Your lowest three quiz scores will be dropped.

\section*{Exams}
There will be two in-class exams and then a final exam during finals week. 
The purpose of the exams is to test your understanding of, and ability to reason about, the mathematical concepts. Since you can use your textbook as well as any other written material, no memorization is required; however, these exams occur in a finite time period, so rapid recall of facts will serve you well.  All electronic devices should be stowed in your bag for the duration of the exam and any brain implants should be turned off.

\section*{Graded work breakdown}
\begin{tabular}{r | r | l}
what & \% & when \\
\hline
homework & 10 & weekly\\
quizzes & 10 & random times\\
in-class exam \#1 & 20 & Friday, February \ordinalnum{13} \\
in-class exam \#2 & 25 &  Friday, March \ordinalnum{13} \\
final exam & 35 & TBA, in finals week \\
\end{tabular}

\section*{Help}
If you need help or just want to know more about something, please come to my scheduled office hours or set up another time to meet. In addition to my office hours, there are several undergraduate mathematics teaching assistants who hold regular hours.

\section*{Attendance}
Please be advised that Math Department and F\&M policy state that penalties (including
grade reduction and/or dismissal from the course) may be assessed for
excessive, unexcused absences.

\begin{center}
\begin{calendar}{1/12/2015}{16}
\setlength{\calboxdepth}{.3in}
\MWFClass

\options{1/12/2015}{\noclassday} 

% schedule
\caltexton{1}{1.1, 1.2 Review}
\caltextnext{1.3 Review}
\caltextnext{2.1}
\caltextnext{2.2}
\caltextnext{2.3}
\caltextnext{2.4}
\caltextnext{2.5}
\caltextnext{2.6}
\caltextnext{2.7}
\caltextnext{2.8}
\caltextnext{\textcolor{blue}{exam \#1 review}}
\caltextnext{\textcolor{blue}{exam \#1 review}}
\caltextnext{\textcolor{blue}{\textbf{in-class exam \#1}}}
\caltextnext{3.1}
\caltextnext{1.5, 3.1}
\caltextnext{3.2}
\caltextnext{3.3, Appendix C}
\caltextnext{3.4}
\caltextnext{3.5}
\caltextnext{1.6, 3.6}
\caltextnext{3.7}
\caltextnext{Taylor series, imaginary numbers, and magic}
\caltextnext{\textcolor{blue}{exam \#2 review}}
\caltextnext{\textcolor{blue}{exam \#2 review}}
\caltextnext{\textcolor{blue}{\textbf{in-class exam \#2}}}
\caltextnext{4.2, 4.3}
\caltextnext{4.2, 4.3}
\caltextnext{4.2, 4.3}
\caltextnext{4.6}
\caltextnext{4.6}
\caltextnext{4.6}
\caltextnext{4.8}
\caltextnext{5.1}
\caltextnext{5.2}
\caltextnext{5.2}
\caltextnext{5.3}
\caltextnext{5.3}
\caltextnext{\textcolor{blue}{final exam review}}
\caltextnext{\textcolor{blue}{final exam review}}

% Holidays
\Holiday{1/19/2015}{Martin Luther King Day}
\Holiday{3/14/2015}{Spring Break}
\Holiday{3/15/2015}{Spring Break}
\Holiday{3/16/2015}{Spring Break}
\Holiday{3/17/2015}{Spring Break}
\Holiday{3/18/2015}{Spring Break}
\Holiday{3/19/2015}{Spring Break}
\Holiday{3/20/2015}{Spring Break}
\Holiday{3/21/2015}{Spring Break}
\Holiday{3/22/2015}{Spring Break}

\Holiday{4/24/2015}{Reading Day}
\Holiday{4/25/2015}{Reading Day}
\Holiday{4/26/2015}{Reading Day}
\Holiday{4/27/2015}{Reading Day}

\Holiday{4/28/2015}{Finals week}
\Holiday{4/29/2015}{Finals week}
\Holiday{4/30/2015}{Finals week}
\Holiday{5/1/2015}{Finals week}
\Holiday{5/2/2015}{Finals week}
\end{calendar}
\end{center}
\end{document}
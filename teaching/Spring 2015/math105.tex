\documentclass[12pt]{article}
\usepackage{amssymb, amsmath, amsthm}
\usepackage{fullpage}
\usepackage[hidelinks]{hyperref}
\usepackage{epigraph}

\pagestyle{plain}
\title{Math 105: Preparation for College Mathematics\\ \bigskip\small{Spring 2015}}
\date{}
\begin{document}
\maketitle

\begin{tabular}{r l}
\textbf{instructor:}& landon rabern\\
\textbf{email:}& \href{mailto:lrabern@fandm.edu}{\nolinkurl{lrabern@fandm.edu}}\\
\textbf{class webpage:}& \href{http://bit.ly/precalculus2015}{\nolinkurl{bit.ly/precalculus2015}}\\
\textbf{class meetings:}& TBA\\
\textbf{office hours:}& TBA\\
\textbf{textbook:}& \textit{Pre-Calculus, 6E}, by Stewart, Redlin, and Watson\\
\end{tabular}

\section*{What is this class about?}

\section*{Homework}
\epigraph{I can only show you the door. You're the one that has to walk through it.}{}
To achieve fluency in this subject, you will need to immerse yourself in the material.  
Working tons of problems is a great way to do this.  How many problems?  
My recommendation would be to work problems of a given type until they become easy for you.

I will put a list of practice problems for each class period on the class webpage.  These will not be collected.  
Each week, I will select a couple of the more interesting problems and assign them with a due date.  
These will be graded both for correctness and clarity of exposition.

\section*{Quizzes}
\epigraph{Pop quiz, hotshot.}{}
There will 5 minute quizzes at random times throughout the course.  
These are intended to reinforce basic concepts as well as encourage attendance.
Unlike exams, quizzes will be closed-book.  Your lowest three quiz scores will be dropped.

\section*{Exams}
There will be two in-class exams and a final exam. 
The purpose of the exams is to test your understanding of, and ability to reason about, the mathematical concepts. Since you can use your textbook as well as any other written material, no memorization is required; however, these exams occur in a finite time period, so rapid recall of facts will serve you well.  All electronic devices should be stowed in your bag for the duration of the exam and any brain implants should be turned off.

\section*{Graded work breakdown}
\begin{tabular}{r | r | l}
what & \% & when \\
\hline
homework & 10 & weekly\\
quizzes & 10 & weekly\\
in-class exam \#1 & 20 & TBD \\
in-class exam \#2 & 25 & TBD \\
final exam & 35 & TBA \\
\end{tabular}

\section*{Help}
If you need help or just want to know more about something, please come to my scheduled office hours or set up another time to meet. In addition to my office hours, there are several undergraduate mathematics teaching assistants who hold regular hours.

\section*{Attendance}
Please be advised that Math Department and F\&M policy state that penalties (including
grade reduction and/or dismissal from the course) may be assessed for
excessive, unexcused absences.
\end{document}
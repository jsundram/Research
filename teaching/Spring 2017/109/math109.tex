\documentclass[12pt]{article}
\usepackage{amssymb, amsmath, amsthm}
\usepackage{fullpage}
\usepackage[hidelinks]{hyperref}
\usepackage{epigraph}
\usepackage{lmodern}
\usepackage[T1]{fontenc}
\usepackage{xspace}
\usepackage{termcal}
\usepackage{color}
\usepackage{fmtcount}

\makeatletter
\DeclareRobustCommand{\maybefakesc}[1]{%
  \ifnum\pdfstrcmp{\f@series}{\bfdefault}=\z@
    {\fontsize{\dimexpr0.8\dimexpr\f@size pt\relax}{0}\selectfont\uppercase{#1}}%
  \else
    \textsc{#1}%
  \fi
}
\newcommand\AM{\,\maybefakesc{am}\xspace}
\newcommand\PM{\,\maybefakesc{pm}\xspace}
\makeatother


\newcommand{\MWFClass}{%
\calday[Monday]{\classday}
\skipday % Tuesday (no class)
\calday[Wednesday]{\classday}
\skipday % Thursday (no class)
\calday[Friday]{\classday}
\skipday\skipday % weekend (no class)
}

\newcommand{\MWRFClass}{%
\calday[Monday]{\classday}
\skipday % Tuesday (no class)
\calday[Wednesday]{\classday}
\calday[Thursday]{\classday}
\calday[Friday]{\classday}
\skipday\skipday % weekend (no class)
}

\newcommand{\MTRClass}{%
	\calday[Monday]{\classday}
	\calday[Tuesday]{\classday}
	\skipday
	\calday[Thursday]{\classday}
	\skipday
	\skipday\skipday % weekend (no class)
}

\newcommand{\Holiday}[2]{%
\options{#1}{\noclassday}
\caltext{#1}{#2}
}


\pagestyle{plain}
\title{Math 109: Calculus 1\\ \bigskip\small{Spring 2017}}
\date{}
\begin{document}
\maketitle

\begin{tabular}{r l}
\textbf{instructor:}& landon rabern\\
\textbf{email:}& \href{mailto:lrabern@fandm.edu}{\nolinkurl{lrabern@fandm.edu}}\\
\textbf{class webpage:}& \href{http://bit.ly/109FandM2017}{\nolinkurl{bit.ly/109FandM2017}}\\
\textbf{class meetings:}& 10:00\AM--10:50\PM MWF in Stager 217 (lecture)\\
& 1:15\PM--2:05\PM R in Stager 110 (practice)\\
\textbf{office hours:}& 9:00\AM--9:59\AM M in Stager 230 (and more)\\
\textbf{textbook:}&\textit{Calculus, Concepts and Contexts 4E}, by James Stewart\\
\end{tabular}

\bigskip

\section*{What is this class about?}
We will learn all about numbers as well as various machines that take numbers as input and output other numbers.
The machines that operate on our numbers we call \emph{functions}, there are a few basic types of such machines: \emph{polynomials}, \emph{exponentials} and \emph{trigonometric} functions.
We will learn a systematic way to draw diagrams of these machines' operation.  Some machines are easier to work with than others and our diagrams give us ways to talk about this.  For
example, if a machines' diagram can be drawn without lifting our pencil, then we call the machine \emph{continuous}.  More specific features of the diagrams such as peaks and valleys will turn out
to be useful for solving real-world optimization problems.  In this class, we develop a toolkit for identifying and working with such features.
\section*{Homework} 
\epigraph{I can only show you the door. You're the one that has to walk through it.}{}
To achieve fluency in this subject, you will need to immerse yourself in the material.  
Working tons of problems is a great way to do this.  How many problems?  
My recommendation is to work problems of a given type until they become easy for you.
\section*{Journal} 
To encourage you to make working problems a regular activity, you will need to maintain a journal containing your practice work.
These journals will be inspected periodically.  In addition to problems, your journal will contain your summaries of each week's video.
\section*{Practice}
On Thursdays we practice.  There are many ways we could structure practice time, I have some ideas on what works well, but the practice is
entirely for your benefit, so you will determine what works best.

\section*{Exams}
There will be two in-class exams and then a final exam during finals week. 
The purpose of the exams is to test your understanding of, and ability to reason about, the mathematical concepts. 
Since you can use your textbook as well as any other written material, no memorization is required; however, 
these exams occur in a finite time period, so rapid recall of facts will serve you well. 

\section*{Graded work breakdown}
\begin{tabular}{r | r | l}
what & \% & when \\
\hline
journal & 15 & random times \\
participation & 15 & always\\
exam \#1 & 15 & TBA, in class \\
exam \#2 & 25 & TBA, in class \\
final exam & 30 & TBA, in finals week \\
\end{tabular}

\section*{Help}
If you need help or just want to know more about something, please come to my scheduled office hours or set up another time to meet. 
In addition to my office hours, there are several undergraduate mathematics teaching assistants who hold regular hours.

\section*{Attendance}
If you do not attend class regularly, it is less likely that you will learn.
\end{document}
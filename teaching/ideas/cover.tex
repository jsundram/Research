\documentclass[10pt,stdletter,dateno]{newlfm}
\usepackage{url}

\widowpenalty=1000
\clubpenalty=1000

\newlfmP{headermarginskip=0pt}
\newlfmP{sigsize=12pt}
\newlfmP{dateskipafter=0pt}

\namefrom{Landon Rabern}
\addrfrom{%
    \today\\[10pt]
    497 Mine Road\\
    Lebanon, PA 17042}

\addrto{%
F\&M Search Committee\\
Department of Mathematics \\
Franklin \& Marshall College \\
Lancaster, PA 17604}

\greetto{Dear F\&M Search Committee,}
\closeline{Sincerely,}
\begin{document}
\begin{newlfm}

I am writing to apply for Franklin and Marshall's Visiting Assistant Professor position in Mathematics, a post Professor Michael McCooey brought to my attention. 
I was glad to learn of the listing; 
I have taught multiple sections of MATH 105 and MATH 109 at F\&M for the past two years, and would look forward to strengthening my connection with Math Department 
and the college. My general areas of research interest include structural and extremal graph theory, machine learning and philosophical logic. In particular, I am 
drawn to problems at intersections of computer science, math and philosophy. Given my research agenda, it is a pleasure to apply to a college that values interdisciplinarity, 
and to a department that houses both math and computer science.  

As an educator, I am able to offer a range of courses suitable for undergraduates of varying levels. In addition to MATH 105 and MATH 109, 
I would also welcome any opportunities to teach courses in differential equations, linear algebra, abstract algebra, analysis, or topology. In keeping with 
my research and industry background in computer science, I am also interested in teaching courses in discrete math or computer science, as department need allows. 
I do have some familiarity with F\&M’s Connections curriculum, and can also imagine contributing a course exploring the history of cryptography, for example, if that would be desirable. 

As a researcher, I can offer an active publishing record as well as a genuine interest in mentoring and potentially collaborating with undergraduate students. 
One of my favorite results from my graduate work in graph coloring and partitioning is a simultaneous generalization of Hall's marriage theorem and Vizing's edge 
coloring theorem in terms of winning strategies in a certain two-player game. In philosophy, I made progress on determining which relations of reference afford the 
structure necessary to support the infinite semantic paradoxes. This and other research appears in numerous journals including Journal of Combinatorial Theory Series B, 
Discrete Mathematics, Analysis, Journal of Philosophical Logic, and Journal of Pure and Applied Algebra. Much of my work emerges from collaborations, an ethos I would 
extend to interested students at Franklin and Marshall. In my primary research area of graph coloring, there are many open problems that would be amenable to collaborative 
work with undergraduate students, and likely the opportunity for co-authored articles. 

In sum, I would look forward to bringing my experience to an inspiring liberal arts community that values rigorous scholarship while centralizing pedagogy. Through 
teaching, collaborative research with students, and other mentorship, I would enjoy contributing to a department and college community that I have come to respect.
\end{newlfm}
\end{document}

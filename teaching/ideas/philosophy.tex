\documentclass[12pt]{article}
\usepackage{amsmath, amssymb}


\newtheorem{acknowledgement}{Acknowledgement}
\newtheorem{algorithm}{Algorithm}
\newtheorem{axiom}{Axiom}
\newtheorem{case}{Case}
\newtheorem{claim}{Claim}
\newtheorem{conclusion}{Conclusion}
\newtheorem{condition}{Condition}
\newtheorem{conjecture}{Conjecture}
\newtheorem{corollary}{Corollary}
\newtheorem{criterion}{Criterion}
\newtheorem{definition}{Definition}
\newtheorem{example}{Example}
\newtheorem{exercise}{Exercise}
\newtheorem{lemma}{Lemma}
\newtheorem{notation}{Notation}
\newtheorem{problem}{Problem}
\newtheorem{proposition}{Proposition}
\newtheorem{remark}{Remark}
\newtheorem{solution}{Solution}
\newtheorem{summary}{Summary}
\newtheorem{theorem}{Theorem}


\newcommand{\fancy}[1]{\mathcal{#1}}
\newcommand{\C}[1]{\fancy{C}_{#1}}


\newcommand{\IN}{\mathbb{N}}
\newcommand{\IZ}{\mathbb{Z}}
\newcommand{\IR}{\mathbb{R}}
\newcommand{\G}{\fancy{G}}
\newcommand{\CC}{\fancy{C}}
\newcommand{\D}{\fancy{D}}
\newcommand{\T}{\fancy{T}}
\newcommand{\B}{\fancy{B}}
\renewcommand{\L}{\fancy{L}}
\newcommand{\HH}{\fancy{H}}

\newcommand{\inj}{\hookrightarrow}
\newcommand{\surj}{\twoheadrightarrow}

\newcommand{\set}[1]{\left\{ #1 \right\}}
\newcommand{\setb}[3]{\left\{ #1 \in #2 : #3 \right\}}
\newcommand{\setbs}[2]{\left\{ #1 : #2 \right\}}
\newcommand{\card}[1]{\left|#1\right|}
\newcommand{\size}[1]{\left\Vert#1\right\Vert}
\newcommand{\ceil}[1]{\left\lceil#1\right\rceil}
\newcommand{\floor}[1]{\left\lfloor#1\right\rfloor}
\newcommand{\func}[3]{#1\colon #2 \rightarrow #3}
\newcommand{\funcinj}[3]{#1\colon #2 \inj #3}
\newcommand{\funcsurj}[3]{#1\colon #2 \surj #3}
\newcommand{\irange}[1]{\left[#1\right]}
\newcommand{\join}[2]{#1 \mbox{\hspace{2 pt}$\ast$\hspace{2 pt}} #2}
\newcommand{\djunion}[2]{#1 \mbox{\hspace{2 pt}$+$\hspace{2 pt}} #2}
\newcommand{\parens}[1]{\left( #1 \right)}
\newcommand{\brackets}[1]{\left[ #1 \right]}
\newcommand{\DefinedAs}{\mathrel{\mathop:}=}

\newcommand{\mic}{\operatorname{mic}}
\newcommand{\AT}{\operatorname{AT}}
\newcommand{\col}{\operatorname{col}}
\newcommand{\ch}{\operatorname{ch}}
\newcommand{\type}{\operatorname{type}}
\newcommand{\nonsep}{\bar{S}}
\newcommand{\type}{\operatorname{type}}
\def\adj{\leftrightarrow}
\def\nonadj{\not\!\leftrightarrow}
\newcommand{\gcd}{\operatorname{gcd}}

\newcommand\restr[2]{{% we make the whole thing an ordinary symbol
  \left.\kern-\nulldelimiterspace % automatically resize the bar with \right
  #1 % the function
  \vphantom{\big|} % pretend it's a little taller at normal size
  \right|_{#2} % this is the delimiter
  }}

\def\D{\fancy{D}}
\def\C{\fancy{C}}
\def\A{\fancy{A}}

\newcommand{\claim}[2]{{\bf Claim #1.}~{\it #2}~~}
\newcommand{\case}[2]{{\bf Case #1.}~{\it #2}~~}
\newcommand\numberthis{\addtocounter{equation}{1}\tag{\theequation}}

\def\gcd{\bigtriangledown}
\def\lcm{\bigtriangleup}
\def\no{\natural}


\usepackage{tikz}
\usetikzlibrary{calc}

\pgfdeclarelayer{background}
\pgfsetlayers{background,main}
\newcommand{\Bond}[6]%
% start, end, thickness, incolor, outcolor, iterations
{ \begin{pgfonlayer}{background}
        \colorlet{InColor}{#4}
        \colorlet{OutColor}{#5}
        \foreach \I in {#6,...,1}
        {   \pgfmathsetlengthmacro{\r}{#3/#6*\I}
            \pgfmathsetmacro{\C}{sqrt(1-\r*\r/#3/#3)*100}
            \draw[InColor!\C!OutColor, line width=\r] (#1.center) -- (#2.center);
        }
    \end{pgfonlayer}
}

\newcommand{\BlackBond}[2]%
% start, end
{   \Bond{#1}{#2}{0.7071mm}{black!25}{black!25!black}{10}
}

\title{teaching philosophy}
\author{landon rabern}
\begin{document}
\maketitle
I approach teaching as an ongoing engagement with a core problem: how can one enable students to balance the rigorous and sometimes repetitive work that is necessary to 
achieve thorough understanding, with motivation and enjoyment? Further, how can I maximize students' retention of material beyond the context of an upcoming exam? 
I seek to transmit to my students, who will come into the classroom with a range of interests and ability levels depending on the course, positive experiences that 
affirm why mathematics matters.

When a former student meets someone at a party who says they are a mathematician, I don't want their first response to be ``oh, I always hated that''.  Even if before my
class that was their state, afterwards I want them to have at least a little light to share about math, something positive. A good way to achieve positive feelings about math is to separate the math from the formalism in the students minds.  Often, when someone says they hated math, it isn't math they hate
but some specific formalism, maybe it was set builder notation, functions, those weird integral signs, the dreaded Leibniz derivative notation?  Or maybe it was even more basic at
the level of algebra, at some point something beautiful was replaced with something ugly and all they got to work with after that was the ugly stand-in.  An example: I tell someone
that I am a writer and they say they always hated writing, but not really, they always just had to do their writing in Microsoft Word and that tool is what they hate.  They have
not experienced writing in any other form, their only association is pain in Microsoft Word.  As a teacher, one of my jobs is to separate the tools from the topics so that the students can 
learn new friendlier tools to deal with topics they once thought they hated.

Throughout the process of teaching different courses, I actively search out applications that I hope will resonate with the group. In a multivariable 
calculus class, for example, 3D games give a perfect backdrop for a discussion of Euler angles and their susceptibility to gimbal lock.  This leads naturally into the topic of quaternions 
and their advantages in performing smooth 3D rotations.  Just one ``Whoa that's cool'' or ``Ah, I see how this could be useful'' moment for a student can bring about a drastic change 
in their attitude.

My background in industry also informs my approach to teaching and my emphasis on rigor. As a Senior Software Engineer and Scientific Programmer, 
I've mentored many junior software engineers. I've learned that, at least in this realm, the single most effective way to help a student improve is to 
get them to take great pride in their work. In my view, this pride derives from writing code that is the purest logical expression of the idea/algorithm in the
 given language. At Wall Street On Demand in Boulder, we instilled this pride through code reviews---a small subset of the engineers get together and look over
 each others code line by line.  This tedious-sounding practice is extremely effective in improving both code and engineer quality. Inspired by these results, 
 I tried a similar idea in a graduate graph theory course at Arizona State University. A student would present a homework solution on the board and the rest of the class 
 would give feedback.  To prevent progress being quashed by negative feedback, we took the ``Yes and...'' approach espoused by theatrical improvisation, wherein feedback 
 must build on the good parts of the solution instead of just pointing out the bad.
 
A class consists of the material to be taught and the classroom mechanics. Classroom mechanics are important.  
How much homework to assign? Many easy problems or just a few interesting ones? Should students work in groups or alone? How much help do I give them?
All important questions, I come in with a few prior assumptions about what works well, but not with any real certainty.  Faced with this, there is a clear solution: science.
Each time I teach a class, I like to experiment a little.  Last year, I had students keep a \emph{Practice Journal} in which they worked suggested exercises.  
I gave them a large list of exercises of each type and instructed them to keep working problems in their journals until they became easy.  I told the students that I would
be collecting their journals for inspection every few weeks.  It turned out that there was no way I was going to be able to collect and inspect that many notebooks that often.
I ended up only collecting them one time near the start of the semester and thought this journal idea was a total failure. But, it wasn't.  I didn't have to actually collect them,
I just had to have the students think I would. I was surprised that it actually worked for a large fraction of the students.  And it worked because it was vague, there
was no bare minimum number of exercises to do, so students ended up working many exercises to make their journals look respectible.
 
Beyond these mechanics, I bring a great love of and enthusiasm for mathematics to the classroom. 
My path into and within the discipline is unique to me; I seek to mentor the infinite possible paths that my individual students might take, 
from a standpoint of interest in and respect for their work. I aim to balance the rigor that is necessary for mastery of the discipline with the pleasure 
that will motivate students to think beyond the confines of a single class. Though practicing rigor is not always comfortable, I endeavor to communicate to
my students that the payoff is worth it, because the work opens up crucial windows into perceiving the structure of the world. 

\end{document}
\documentclass[12pt]{article}
\usepackage{amsmath, amssymb}


\newtheorem{acknowledgement}{Acknowledgement}
\newtheorem{algorithm}{Algorithm}
\newtheorem{axiom}{Axiom}
\newtheorem{case}{Case}
\newtheorem{claim}{Claim}
\newtheorem{conclusion}{Conclusion}
\newtheorem{condition}{Condition}
\newtheorem{conjecture}{Conjecture}
\newtheorem{corollary}{Corollary}
\newtheorem{criterion}{Criterion}
\newtheorem{definition}{Definition}
\newtheorem{example}{Example}
\newtheorem{exercise}{Exercise}
\newtheorem{lemma}{Lemma}
\newtheorem{notation}{Notation}
\newtheorem{problem}{Problem}
\newtheorem{proposition}{Proposition}
\newtheorem{remark}{Remark}
\newtheorem{solution}{Solution}
\newtheorem{summary}{Summary}
\newtheorem{theorem}{Theorem}


\newcommand{\fancy}[1]{\mathcal{#1}}
\newcommand{\C}[1]{\fancy{C}_{#1}}


\newcommand{\IN}{\mathbb{N}}
\newcommand{\IZ}{\mathbb{Z}}
\newcommand{\IR}{\mathbb{R}}
\newcommand{\G}{\fancy{G}}
\newcommand{\CC}{\fancy{C}}
\newcommand{\D}{\fancy{D}}
\newcommand{\T}{\fancy{T}}
\newcommand{\B}{\fancy{B}}
\renewcommand{\L}{\fancy{L}}
\newcommand{\HH}{\fancy{H}}

\newcommand{\inj}{\hookrightarrow}
\newcommand{\surj}{\twoheadrightarrow}

\newcommand{\set}[1]{\left\{ #1 \right\}}
\newcommand{\setb}[3]{\left\{ #1 \in #2 : #3 \right\}}
\newcommand{\setbs}[2]{\left\{ #1 : #2 \right\}}
\newcommand{\card}[1]{\left|#1\right|}
\newcommand{\size}[1]{\left\Vert#1\right\Vert}
\newcommand{\ceil}[1]{\left\lceil#1\right\rceil}
\newcommand{\floor}[1]{\left\lfloor#1\right\rfloor}
\newcommand{\func}[3]{#1\colon #2 \rightarrow #3}
\newcommand{\funcinj}[3]{#1\colon #2 \inj #3}
\newcommand{\funcsurj}[3]{#1\colon #2 \surj #3}
\newcommand{\irange}[1]{\left[#1\right]}
\newcommand{\join}[2]{#1 \mbox{\hspace{2 pt}$\ast$\hspace{2 pt}} #2}
\newcommand{\djunion}[2]{#1 \mbox{\hspace{2 pt}$+$\hspace{2 pt}} #2}
\newcommand{\parens}[1]{\left( #1 \right)}
\newcommand{\brackets}[1]{\left[ #1 \right]}
\newcommand{\DefinedAs}{\mathrel{\mathop:}=}

\newcommand{\mic}{\operatorname{mic}}
\newcommand{\AT}{\operatorname{AT}}
\newcommand{\col}{\operatorname{col}}
\newcommand{\ch}{\operatorname{ch}}
\newcommand{\type}{\operatorname{type}}
\newcommand{\nonsep}{\bar{S}}
\newcommand{\type}{\operatorname{type}}
\def\adj{\leftrightarrow}
\def\nonadj{\not\!\leftrightarrow}
\newcommand{\gcd}{\operatorname{gcd}}

\newcommand\restr[2]{{% we make the whole thing an ordinary symbol
  \left.\kern-\nulldelimiterspace % automatically resize the bar with \right
  #1 % the function
  \vphantom{\big|} % pretend it's a little taller at normal size
  \right|_{#2} % this is the delimiter
  }}

\def\D{\fancy{D}}
\def\C{\fancy{C}}
\def\A{\fancy{A}}

\newcommand{\claim}[2]{{\bf Claim #1.}~{\it #2}~~}
\newcommand{\case}[2]{{\bf Case #1.}~{\it #2}~~}
\newcommand\numberthis{\addtocounter{equation}{1}\tag{\theequation}}

\def\gcd{\bigtriangledown}
\def\lcm{\bigtriangleup}
\def\no{\natural}


\usepackage{tikz}
\usetikzlibrary{calc}

\pgfdeclarelayer{background}
\pgfsetlayers{background,main}
\newcommand{\Bond}[6]%
% start, end, thickness, incolor, outcolor, iterations
{ \begin{pgfonlayer}{background}
        \colorlet{InColor}{#4}
        \colorlet{OutColor}{#5}
        \foreach \I in {#6,...,1}
        {   \pgfmathsetlengthmacro{\r}{#3/#6*\I}
            \pgfmathsetmacro{\C}{sqrt(1-\r*\r/#3/#3)*100}
            \draw[InColor!\C!OutColor, line width=\r] (#1.center) -- (#2.center);
        }
    \end{pgfonlayer}
}

\newcommand{\BlackBond}[2]%
% start, end
{   \Bond{#1}{#2}{0.7071mm}{black!25}{black!25!black}{10}
}

\title{teaching philosophy}
\author{landon rabern}
\begin{document}
\maketitle
I approach teaching as an ongoing engagement with a core problem: how can one enable students to balance the rigorous and 
sometimes repetitive work that is necessary to achieve thorough understanding, with motivation and enjoyment? 
Further, how can I maximize students' retention of material beyond the context of an upcoming exam? 
I seek to transmit to my students, who will come into the classroom with a range of interests and ability levels
depending on the course, positive experiences that affirm why mathematics matters.

When I work with students in pre-calculus or calculus, I notice that it is more likely for students to enter the course with negative feelings about math. 
In addition to communicating content, I consider that part of my function is to help them accumulate positive associations with the subject. 
I find that one effective way of achieving this involves delineating math as such from formalistic tools we have developed to teach and communicate. 
For example, perhaps students shut down when they encounter set builder notation, integral signs, the Leibniz derivative notation, etc. 
We might liken this to students who perceive that they dread writing, when in fact they dread the tool of Microsoft Word. As an educator, 
I search out opportunities to separate the tools from the topics so that students can simultaneously develop respect for and interest in mathematical principles, 
while together we investigate friendlier tools. In order to excite some passion about concepts, I explore applications that I hope will resonate with the group. 
In a multivariable calculus class, for example, 3D games provide an ideal backdrop for a discussion of Euler angles and their susceptibility to gimbal lock. 
This can lead organically into the topic of quaternions and their advantages in performing smooth 3D rotations. Introducing applications like these can, 
I have found, provoke drastic changes in student attitudes that enable students to embrace the rigor of the discipline with more curiosity. 

My background in industry also informs my approach in classroom, in particular my emphasis on group work and peer review. As a Senior Software Engineer and Scientific Programmer,
I've mentored many junior software engineers. I've learned that, at least in this realm, the single most effective way to help an individual to improve is through communication to 
and with a group, with a clear shared goal in mind. For example, how can we as a team write code that is the purest logical expression of the idea/algorithm in the given language? 
How can each individual contribute maximally to this shared goal? At Wall Street On Demand in Boulder, we initiated code reviews, during which a small subset of the 
engineers – together - reviewed one another’s code line by line. This tedious-sounding practice was extremely effective in improving code and engineer quality, 
and produced a positive ethos of teamwork. Inspired by these results, I tried a similar idea in a graduate graph theory course at Arizona State University. 
A student would present a homework solution on the board and the rest of the class would give feedback. To prevent progress being quashed by negative feedback, 
we took the ``Yes and...'' approach espoused by theatrical improvisation, wherein feedback must build on the
good parts of the solution instead of just pointing out the bad. While students were initially shy, I found that by the end of the semester they were better 
able to articulate their own reasoning, communicate with others, and in general had a greater sense of pride and focus toward finding the best possible collective solution. 

Each time I teach a class of any level, I like to experiment with classroom mechanics such as the balance between working alone and in groups, moving toward 
best possible practices. Last year, I had students keep a \emph{Practice Journal} in which they worked suggested exercises. I gave them a large list of exercises of 
each type and instructed them to keep working problems in their journals until they became easy, without giving them further guidelines. I indicated that I would 
be collecting journals for inspection every few weeks, so as to gain greater insight into their individual processes and progress. I learned that it proved to be 
unfeasible to collect and review journals as often as I'd planned, and initially thought that the journal idea was a total failure. However, at the end of the course, 
a large fraction of the students remarked on the efficacy of this requirement.  Allowing students ownership over the number of problems seemed to inspire more rigor and 
repetition than I would have asked of them. As with the strategy of peer review, I continually look for ways in which I can be a facilitator of students' self-motivation, 
which I believe is essential to achieving solid conceptual understanding. 

Beyond these mechanics, I bring a great love of and enthusiasm for mathematics to the classroom. Knowing that my path within the discipline is unique to me, 
I seek to mentor the infinite possible paths that my individual students might take, from a standpoint of interest in and respect for their work. 
I aim to balance the rigor that is necessary for mastery of the discipline with the pleasure that will motivate students to think beyond the confines of a single class. 
Though practicing rigor is not always comfortable, I endeavor to communicate to my students that the payoff is worth it, because the work opens up crucial windows into perceiving 
the structure of the world. 

\end{document}
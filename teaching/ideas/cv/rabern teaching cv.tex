\documentclass[margin,line]{res}
\usepackage{enumerate}
\usepackage[hidelinks]{hyperref}
\urlstyle{same}


\oddsidemargin -.6in
\evensidemargin -.6in
\textwidth=6.0in
\itemsep=0in
\parsep=0in

\newenvironment{list1}{
  \begin{list}{\ding{113}}{%
      \setlength{\itemsep}{0in}
      \setlength{\parsep}{0in} \setlength{\parskip}{0in}
      \setlength{\topsep}{0in} \setlength{\partopsep}{0in} 
      \setlength{\leftmargin}{0.17in}}}{\end{list}}
\newenvironment{list2}{
  \begin{list}{$\bullet$}{%
      \setlength{\itemsep}{0in}
      \setlength{\parsep}{0in} \setlength{\parskip}{0in}
      \setlength{\topsep}{0in} \setlength{\partopsep}{0in} 
      \setlength{\leftmargin}{0.2in}}}{\end{list}}


\begin{document}

\name{{\sc\bf landon rabern} \vspace*{.1in}}

\begin{resume}
\section{\sc Contact Information}
\vspace{.05in}
\begin{tabular}{@{}p{2in}p{4in}}     
497 Mine Road & \qquad\qquad\textit{email}: landon.rabern@gmail.com \\   
Lebanon, PA 17042 & \qquad\qquad \textit{math}: \url{https://sites.google.com/site/landonrabern} \\
& \qquad\qquad \textit{code}: \href{https://github.com/landon?tab=repositories}{https://github.com/landon}
\end{tabular}

\section{\sc Research Interests}
Structural and extremal graph theory, particularly graph coloring. The structure of paradox.  Meditation.
Machine learning/discovery, automated proof writing.  Teaching mathematics via games and play.

\bigskip

\section{\sc Education}
{\bf Ph.D., Mathematics}. Arizona State University, 2011 - 2013.
\begin{list2}
\item Dissertation: \textit{\href{https://dl.dropbox.com/u/8609833/Papers/main_fancy.pdf}{Coloring graphs from almost maximum degree sized palettes}}
\item Advisor: Hal Kierstead
\end{list2}
{\bf M.A., Mathematics}. University of California, Santa Barbara, 2003 - 2005.\\
{\bf  B.A., Mathematics}. Washington University in St. Louis, 1999 - 2003.

\section{\sc Teaching}

{\bf Franklin \& Marshall College}

\vspace{-.4cm}
Department of Mathematics\\
{\em Adjunct Assistant Professor} \hfill {\bf 2015 - \phantom{2017}}

\begin{list2}
\item Precalculus
\item Differential Calculus (5 courses)
\end{list2}

{\bf Arizona State University}

\vspace{-.4cm}
School of Mathematical and Statistical Sciences \\
{\em Teaching Assistant} \hfill {\bf 2011 - 2012} 

\begin{list2}
\item Graph Theory
\end{list2}

{\bf University of California, Santa Barbara}

\vspace{-.4cm}
Department of Mathematics \\
{\em Teaching Assistant} \hfill {\bf 2003 - 2005} 

\begin{list2}
\item Abstract Algebra
\item Differential Equations \& Linear Algebra 
\item Logic and Proofs
\end{list2}

\section{\sc Industry Work}

{\bf LBD Data}

\vspace{-.4cm}
{\em Owner / Software Architect} \hfill {\bf 2008 - \phantom{2017}}

{\bf Wall Street On Demand}

\vspace{-.4cm}
{\em Senior Software Engineer} \hfill {\bf 2010 - 2011}

{\bf Synaptics}

\vspace{-.4cm}
{\em Software Engineer} \hfill {\bf 2009 - 2010}

{\bf Wall Street On Demand}

\vspace{-.4cm}
{\em Software Engineer} \hfill {\bf 2007 - 2009}

{\bf L-3 Communications - Applied Technologies Division}

\vspace{-.4cm}
{\em Scientific Programmer, Security Clearance---Secret} \hfill {\bf 2005 - 2007}

\section{\sc Math Publications}

\begin{enumerate}[{[}1{]}]

\item
L.~Rabern.
\newblock A better lower bound on average degree of 4-list-critical graphs.
\newblock {\em Electron. J. Combin.}, Accepted.
	
	\smallskip
	
	\item
H.~Kierstead and L.~Rabern.
\newblock Extracting list colorings from large independent sets.
\newblock {\em J. Graph Theory}, Accepted.

\smallskip

	\item
	D.W.~Cranston and L.~Rabern.
	\newblock Edge Lower Bounds for List Critical Graphs, via Discharging.
	\newblock {\em Combinatorica}, Accepted.
	\smallskip

\item
D.W.~Cranston and L.~Rabern.
\newblock Planar graphs have independence ratio at least 3/13.
\newblock {\em Electron. J. Combin.}, Accepted.
\smallskip

\item
D.W.~Cranston and L.~Rabern.
\newblock List-coloring claw-free graphs with $\Delta - 1$ colors.
\newblock {\em SIAM J. Discrete Math.}, Accepted.
\smallskip

\item
D.W.~Cranston and L.~Rabern.
\newblock Subcubic edge chromatic critical graphs have many edges.
\newblock {\em J. Graph Theory}, Accepted.
\smallskip

\item
D.W.~Cranston and L.~Rabern.
\newblock Painting squares in $\Delta^2 - 1$ shades.
\newblock {\em Electron. J. Combin.}, Accepted.
	
	\smallskip
	
	\item
H.~Kierstead and L.~Rabern.
\newblock Improved lower bounds on the number of edges in list critical and online list critical graphs.
\newblock {\em J. Combin. Theory Ser. B}, Accepted.

\smallskip
	
	\item
	D.W.~Cranston and L.~Rabern.
	\newblock The fractional chromatic number of the plane.
	\newblock {\em Combinatorica}, Accepted.
	\smallskip
	
	\item
	D.W.~Cranston and L.~Rabern.
	\newblock Graphs with $\chi = \Delta$ have big cliques.
	\newblock {\em SIAM J. Discrete Math.}, Accepted.
	\smallskip
	
	\item
	D.W.~Cranston and L.~Rabern.
	\newblock Brooks' Theorem and Beyond.
	\newblock {\em J. Graph Theory}, Accepted.
	
	\smallskip
	
\item D.W.~Cranston and L.~Rabern.
\newblock A note on coloring vertex-transitive graphs.
\newblock {\em Electron. J. Combin.}, \textbf{22} (2), 2015.

\smallskip

\item
D.W.~Cranston and L.~Rabern.
\newblock Conjectures equivalent to the Borodin-Kostochka conjecture that appear weaker.
\newblock {\em European J. Combinatorics}, Volume 44, Part A, February 2015, Pages 23–-42.

\smallskip

\item
L.~Rabern.
\newblock A game generalizing Hall's theorem.
\newblock {\em Discrete Math.}, \textbf{320}\penalty0 (6):\penalty0 87-91, 2014.

\smallskip

\item
L.~Rabern.
\newblock Coloring graphs with dense neighborhoods.
\newblock {\em J. Graph Theory}, \textbf{76}\penalty0 (4):\penalty0 323-340, 2014.

\smallskip

\item
L.~Rabern.
\newblock A different short proof of Brooks' theorem.
\newblock {\em Discuss. Math. Graph Theory},  \textbf{34}\penalty0 (3), 2014.

\smallskip

\item
L.~Rabern.
\newblock Partitioning and coloring graphs with degree constraints.
\newblock {\em Discrete Math.}, \textbf{313}\penalty0 (9):\penalty0 1028-1034, 2013.

\smallskip

\item
D.W.~Cranston and L.~Rabern.
\newblock Coloring claw-free graphs with $\Delta - 1$ colors.
\newblock {\em SIAM J. Discrete Math.}, \textbf{27}\penalty0 (1):\penalty0 534-549, 2013.

\smallskip

\item
L.~Rabern.
\newblock Destroying non-complete regular components in graph partitions.
\newblock {\em J. Graph Theory}, \textbf{72}\penalty0 (2):\penalty0 123-127, 2013.

\smallskip

\item A.V.~Kostochka, L.~Rabern and M.~Stiebitz.
\newblock Graphs with chromatic number close to maximum degree.
\newblock {\em Discrete Math.},  \textbf{312}\penalty0 (6):\penalty0 1273-1281, 2012.

\smallskip

\item L.~Rabern.
\newblock A strengthening of Brooks' Theorem for line graphs.
\newblock {\em Electron. J. Combin.}, N145, \textbf{18} (1), 2011.

\smallskip

\item L.~Rabern.
\newblock $\Delta$-Critical graphs with small high vertex cliques.
\newblock {\em J. Combin. Theory Ser. B}, \textbf{102}\penalty0 (1):\penalty0 126-130, 2012.

\smallskip

\item L.~Rabern. 
\newblock On hitting all maximum cliques with an independent set.
\newblock {\em J. Graph Theory}, \textbf{66}\penalty0 (1):\penalty0 32-37, 2011.

\smallskip

\item L.~Rabern.
\newblock A note on Reed's conjecture.
\newblock {\em SIAM J. Discrete Math.}, \textbf{22}\penalty0 (2):\penalty0 820-827, 
	2008.

\smallskip

\item L.~Rabern.
\newblock Applying Groebner basis techniques to group theory.
\newblock {\em J. Pure Appl. Algebra}, \textbf{210}\penalty0 (1):\penalty0 137-140, 2007.

\smallskip

\item
L.~Rabern.
\newblock The Borodin-Kostochka conjecture for graphs containing a doubly critical edge.
\newblock {\em Electron. J. Combin.}, N22, \textbf{14} (1), 2007.

\smallskip

\item
D.~Gernert and L.~Rabern.
\newblock A knowledge-based system for graph theory, demonstrated by partial proofs for graph-colouring problems.
\newblock {\em Comm. Math. Comput. Chem.}, \textbf{58}, N2 2007.

\smallskip

\item L.~Rabern.
\newblock On graph associations.
\newblock {\em SIAM J. Discrete Math.}, \textbf{20} \penalty0 (2):\penalty0 529--535,
  2006.

\smallskip

\item L.~Rabern.
\newblock Properties of magic squares of squares.
\newblock {\em Rose Hulman Undergraduate J. Math.}, \textbf{4}\penalty0
  (1), 2003. 

\end{enumerate}

\section{\sc Philosophy Publications}
\begin{enumerate}[{[}1{]}]
	\setcounter{enumi}{29}
\item
L.~Rabern, B.~Rabern, and M.~Macauley.
\newblock Dangerous reference graphs and semantic paradoxes.
\newblock {\em J. Philos. Logic}, \textbf{42}\penalty0 (5):\penalty0 727-765, 2013.

\smallskip

\item
B.~Rabern and L.~Rabern.
\newblock A simple solution to the hardest logic puzzle ever. 
\newblock {\em Analysis}, \textbf{68}\penalty0 (2), April 2008.

\end{enumerate}
\section{\sc Under Review}

\begin{enumerate}[{[}1{]}]
\setcounter{enumi}{31}


\item
L.~Rabern.
\newblock A better lower bound on average degree of k-list-critical graphs.

\smallskip

\item
D.W.~Cranston and L.~Rabern.
\newblock Planar graphs are 9/2-colorable.

\smallskip

\item
D.W.~Cranston and L.~Rabern.
\newblock Short fans and the 5/6 bound for line graphs..

\smallskip

\item
D.W.~Cranston and L.~Rabern.
\newblock Beyond Degree Choosability.

\smallskip

\item
D.W.~Cranston and L.~Rabern.
\newblock Edge-coloring via fixable subgraphs.


\end{enumerate}

\section{\sc Peer Reviews}
\begin{list2}
\item Journal of Combinatorial Theory, Series B
\item Electronic Journal of Combinatorics
\item Journal of Graph Theory
\item Discrete Math
\item Synthese
\item Minds and Machines
\item SIAM Journal on Discrete Mathematics
\end{list2}

\section{\sc Presentations}
\begin{list2}
\item \emph{A common generalization of Hall's theorem and Vizing's edge-coloring theorem.} Miami University Colloquium, 2015.
\item \emph{Extending Alon-Tarsi Orientations.} AMS Special Session on Structural and Extremal Problems, 2014.
\item \emph{Improving Brooks' theorem.} The 26th Clemson Conference on Discrete Mathematics and Algorithms, 2011.
\item \emph{An improvement on Brooks' theorem.} CU-Denver Discrete Math Seminar, 2011.
\end{list2}

\section{\sc Computer Skills} 
\begin{list2}
\item Languages: C\#, C/C++, JavaScript, Python, Java, Pascal, Scheme, x86 assembly.
\item Applications: GAP, Boost Graph Library, \LaTeX.
\item Operating Systems:  UNIX/Linux, Windows.
\end{list2}

\end{resume}
\end{document}





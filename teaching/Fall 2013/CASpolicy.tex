%THIS SECTION IS STANDARD POLICY....................................DO NOT ALTER
\Topic{Department, College and University Expectations and Policies}

\Subtopic{Adding/Dropping a Class}
Prior to the Add/Drop deadline, a student may withdraw from 
registration in the course through the Registrar's Office or system.
The students name is removed from the class roll, and the instructor is
not involved.
After the Add/Drop period,
a grade of W will be considered under appropriate
circumstances.
A grade of W will not be given once the final exam is attempted.
The Add/Drop deadline is \Addanddrop, as discussed at
\Urls{\AddanddropURL}, and the University Add/Drop policy is outlined at
\Urls{\AddanddroppolicyURL}.

\Subtopic{Attendance}
University attendance policy guidelines require that:
\smallskip

\Quote{
All students are expected to attend regularly and promptly all their classes,
appointments, and exercises. While the university recognizes that some absences
may occasionally be necessary, these should be held to a minimum. A maximum of
two weeks of absences will be permitted for illness and emergencies.
The instructor has the right to dismiss from class any student who has been
absent more than the maximum allowed. After the last date to drop as published
in the academic calendar, a student will receive a failure (F), if failing at
that point, or a W, if passing at the time of dismissal.
}

\vspace*{0.5em}
\noindent
Students are to adhere to the policy attendance policy guidelines
outlined in the University Catalog at
\Urls{\AttendanceURL}.

\Subtopic{Academic Integrity Policy}
This class fully adheres to the Academic Integrity Policy:

\Quote{
Academic integrity is a core university value that ensures respect for the academic reputation of the University, its students, faculty and staff, and the degrees it confers. The University expects that students will conduct themselves in an honest and ethical manner and respect the intellectual work of others. Please be familiar with the UNH policy on Academic Integrity. Please ask about my expectations regarding permissible or encouraged forms of student collaboration if they are unclear.
}

\vspace*{0.5em}
\noindent
Students are required to adhere to the Academic Integrity Policies specified in
the Student Handbook at
\hspace*{-0.08em} \Urls{\IntegrityURL}.


\Subtopic{Coursework Expectations}
This course will require significant in-class and out-of-class commitment from
each student.
The University estimates that a student should expect to spend two hours
outside of class for each hour they are in a class.
For example, a three credit course would average six [6] hours of additional
work outside of class.
Coursework expectations are detailed at
\Urls{\CourseworkURL}.


\Subtopic{University Support Services}
The University recognizes students often can use some help outside of class
and offers academic assistance through several offices.
In addition to talking with your instructor and advisor, we recommend you
contact the Office of Academic Services (OAS) for help with your academic
studies (call 203.932.7234 or visit Maxcy 208).
The Center for Learning Resources (CLR) in Peterson Library is equipped to help
you with writing, mathematics, biology and physics.  

\Subtopic{Special needs}
Students with disabilities are encouraged to share, in confidence, information
about needed specific course accommodations.
The Campus Access Services office (CAS) provides comprehensive services
and support that serve to promote educational equity and ensure that students
are able to participate in the opportunities available at the University
of New Haven.
Contact 203.932.7331, Sheffield Basement, or
\Urls{\AccessURL}.


\Subtopic{Religious Observance Policy for Students}
The University of New Haven respects the right of its students to observe
religious holidays that may necessitate their absence from class or from
other required university-sponsored activities.
This class fully adheres to these ideals and responsibilities:
\vspace*{0.2em}

\Quote{
Students who wish to observe such holidays should not be penalized for their
absence although, in academic courses, they are responsible for making up
missed work. Instructors should try to avoid scheduling exams or quizzes on
religious holidays, but where such conflicts occur, should provide reasonable
accommodations for missed assignment deadlines or exams. If a class,
an assignment due date, or exam interferes with the observance of such a
religious holiday, it is the student's responsibility to notify the class
instructor, preferably at the beginning of the term, but otherwise at least
two weeks before the holiday. In a similar vein, students who
will not participate in other required activities due to religious observance
should notify the staff or faculty member who oversees the program with the
same lead-time.
}

\vspace*{1.0em}
\noindent
More information about religious observance policies can be found in the
Student Handbook.
%THIS SECTION IS STANDARD POLICY....................................DO NOT ALTER

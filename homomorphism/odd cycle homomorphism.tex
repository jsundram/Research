% !TeX root = odd cycle homomorphism.tex
\documentclass[12pt]{amsart}
\usepackage{amsmath, amsthm, amssymb}
\usepackage[top=1.25in, bottom=1.25in, left=1.0in, right=1.0in]{geometry}
\usepackage{hyperref}
\usepackage{color}
\usepackage{verbatim}

\usepackage{tikz-cd}


\makeatletter
\newtheorem*{rep@theorem}{\rep@title}
\newcommand{\newreptheorem}[2]{
\newenvironment{rep#1}[1]{
 \def\rep@title{#2 \ref{##1}}
 \begin{rep@theorem}}
 {\end{rep@theorem}}}
\makeatother

\theoremstyle{plain}
\newtheorem{thm}{Theorem}
\newreptheorem{thm}{Theorem}
\newtheorem{prop}[thm]{Proposition}
\newreptheorem{prop}{Proposition}
\newtheorem{lem}[thm]{Lemma}
\newreptheorem{lem}{Lemma}
\newtheorem{conj}[thm]{Conjecture}
\newreptheorem{conj}{Conjecture}
\newtheorem{cor}[thm]{Corollary}
\newreptheorem{cor}{Corollary}
\newtheorem{prob}[thm]{Problem}
\theoremstyle{definition}
\newtheorem{defn}{Definition}
\theoremstyle{remark}
\newtheorem*{remark}{Remark}
\newtheorem{example}{Example}
\newtheorem*{question}{Question}
\newtheorem*{observation}{Observation}

\title{Sparse graphs admit homomorphisms into odd cycles}
\author{Landon Rabern}
\date{\today}

\newcommand{\fancy}[1]{\mathcal{#1}}
\newcommand{\C}{\fancy{C}}
\newcommand{\IN}{\mathbb{N}}
\newcommand{\IZ}{\mathbb{Z}}
\newcommand{\IR}{\mathbb{R}}
\newcommand{\G}{\fancy{G}}
\renewcommand{\H}{\fancy{H}}
\newcommand{\LB}{\mathcal{L}_B}
\newcommand{\col}{{\textrm{col}}}
\newcommand{\chil}{{\chi_{\ell}}}
\newcommand{\chiol}{{\chi_{OL}}}

\newcommand{\inj}{\hookrightarrow}
\newcommand{\surj}{\twoheadrightarrow}

\newcommand{\set}[1]{\left\{ #1 \right\}}
\newcommand{\setb}[3]{\left\{ #1 \in #2 \mid #3 \right\}}
\newcommand{\setbs}[2]{\left\{ #1 \mid #2 \right\}}
\newcommand{\card}[1]{\left|#1\right|}
\newcommand{\size}[1]{\left\Vert#1\right\Vert}
\newcommand{\ceil}[1]{\left\lceil#1\right\rceil}
\newcommand{\floor}[1]{\left\lfloor#1\right\rfloor}
\newcommand{\func}[3]{#1\colon #2 \rightarrow #3}
\newcommand{\funcinj}[3]{#1\colon #2 \inj #3}
\newcommand{\funcsurj}[3]{#1\colon #2 \surj #3}
\newcommand{\irange}[1]{\left[#1\right]}
\newcommand{\join}[2]{#1 \mbox{\hspace{2 pt}$\ast$\hspace{2 pt}} #2}
\newcommand{\djunion}[2]{#1 \mbox{\hspace{2 pt}$+$\hspace{2 pt}} #2}
\newcommand{\parens}[1]{\left( #1 \right)}
\newcommand{\brackets}[1]{\left[ #1 \right]}
\newcommand{\DefinedAs}{\mathrel{\mathop:}=}
\newcommand{\im}{\operatorname{im}}
\newcommand{\mic}{\operatorname{mic}}
\newcommand{\pot}{\operatorname{Pot}}
\newcommand{\mad}{\operatorname{mad}}

\begin{document}

\begin{abstract}
\end{abstract}
\maketitle

\section{Introduction}
All graphs under consideration are nonempty finite simple graphs. For graphs $G$ and $H$, we indicate the existence of a homomorphism from $G$ to $H$ or lack thereof by $G \rightarrow H$ and $G \not \rightarrow H$, respectively.  We write $H \unlhd G$ to indicate that $H$ is an induced subgraph of $G$, when we want the containment to be proper, we write $H \lhd G$. 

\section{Potential functions}
Kostochka and Yancey \cite{kostochkayancey2012ore} used ``potential functions'' to great effect in proving lower bounds on the number of edges in critical graphs. Here we generalize this idea and prove some basic facts.

\begin{defn}
For positive integers $a$ and $b$, the \emph{$(a,b)$-potential function} is the function from graphs to $\IZ$ given by $\rho_{a,b}(G) \DefinedAs a\card{G} - b\size{G}$.  Additionally, put
\[\hat{\rho}_{a,b}(G) \DefinedAs \min_{H \unlhd G} \rho_{a,b}(H).\]
\end{defn}

The invariant $\hat{\rho}_{a,b}(G)$ is a measure of the sparseness of $G$, the larger $\hat{\rho}_{a,b}(G)$ is, the sparser $G$ is.  For example, if $\hat{\rho}_{a,b}(G) \ge 0$, then $\mad(G) \le \frac{2a}{b}$ where $\mad(G)$ is the maximum average degree of $G$.

For any fixed graph $T$, we are interested in proving results of the form: any sufficiently sparse graph admits a homomorphism into $T$.  To do so, it will be useful to get the benefits of having a minimum counterexample without being bound to a fixed inductive context.  To achieve this, we use \emph{mules} as introduced in \cite{CranstonR-equiv, raberndiss}. 

\subsection{Mules}
\begin{defn}
\label{epi}
If $G$ and $H$ are graphs, an \emph{epimorphism} is a graph homomorphism $\funcsurj{f}{G}{H}$ such that $f(V(G)) = V(H)$.  We indicate this with the arrow $\surj$.
\end{defn}

\begin{defn}
\label{child}
Let $G$ be a graph.  A graph $A$ is called a \emph{child} of $G$ if $A \neq G$
and there exists $H \unlhd G$ and an epimorphism $\funcsurj{f}{H}{A}$.  
\end{defn}

Note that the child-of relation is a strict partial order on the set of (finite
simple) graphs $\fancy{G}$.  We call this the \emph{child order} on $\fancy{G}$
and denote it by `$\prec$'.  By definition, if $H \lhd G$ then $H \prec G$.

\begin{lem}\label{well-founded}
The ordering $\prec$ is well-founded on $\fancy{G}$; that is, every nonempty
subset of $\fancy{G}$ has a minimal element under $\prec$.
\end{lem}
\begin{proof}
Let $\fancy{T}$ be a nonempty subset of $\fancy{G}$.  Pick $G \in \fancy{T}$
minimizing $\card{V(G)}$ and then maximizing $\card{E(G)}$.  
Since any child of $G$ must have fewer vertices or more edges (or both), we see
that $G$ is minimal in $\fancy{T}$ with respect to $\prec$.
\end{proof}

\begin{defn}
Let $\fancy{T}$ be a collection of graphs.  A minimal graph in $\fancy{T}$
under the child order is called a \emph{$\fancy{T}$-mule}.
\end{defn}
\smallskip

\subsection{Basic facts}~\\
\begin{figure}
\begin{center}
\begin{Large}
\begin{tikzcd}
H \arrow[r, hookrightarrow, "\iota"] \arrow[d, twoheadrightarrow, "h"'] & G \arrow[d, twoheadrightarrow, "h'"] \\
Q \arrow[r, hookrightarrow, "\iota"'] & G_h
\end{tikzcd}
\end{Large}
\end{center}
\caption{The commutative diagram for $G_h$.}
\label{fig:G_h}
\end{figure}
For a graph $T$ together with positive integers $a$, $b$ and $c$, let $\C_{T,a,b,c}$ be the set of graphs $G$ such that $G \not \rightarrow T$ and $\hat{\rho}_{a,b}(G) \ge c$.  

\begin{lem}\label{Criticality}
Let $G$ be a $\C_{T,a,b,c}$-mule.  If $H \lhd G$, then $H \rightarrow T$.
\end{lem}
\begin{proof}
Since $\hat{\rho}_{a,b}(H) \ge \hat{\rho}_{a,b}(G) \ge c$ and $H \prec G$, we must have $H \rightarrow T$ since $G$ is a $\C_{T,a,b,c}$-mule.
\end{proof}

\begin{defn}\label{InducedHomomorphism}
Let $H$ be an induced subgraph of a graph $G$ and $\funcsurj{h}{H}{Q}$ an epimorphism onto some graph $Q$. Let $G_h$ be the image of the natural extension of $h$ to an epimorphism $h'$ defined on $G$; that is, $G_h$ and $h'$ are such that the diagram in Figure \ref{fig:G_h} commutes (where $\iota$ indicates the inclusion map).
\end{defn}

\begin{lem}\label{ArbitraryQ}
Let $G$ be a $\C_{T,a,b,c}$-mule and $Q$ an arbitrary graph.  If $H \unlhd G$ with $H \ne Q$ such that $H \surj Q$, then $\rho_{a,b}(H) > \hat{\rho}_{a,b}(Q)$.
\end{lem}
\begin{proof}
Suppose to the contrary that there is $H \unlhd G$ with $H \ne Q$ such that $H \surj Q$ and $\rho_{a,b}(H) \le \hat{\rho}_{a,b}(Q)$.  
Let $h$ be an epimorphism from $H$ onto $Q$. Since $G$ is a $\C_{T,a,b,c}$-mule, $G_h$ cannot be a child of $G$.  But we have an epimorphism $h'$ from $G$ onto $G_h$ and $G_h \ne G$ since $H \ne Q$, so it must be that $G_h \not \in \C_{T,a,b,c}$.  Since $G \rightarrow G_h$ and $G \not \rightarrow T$, we must have $G_h \not \rightarrow T$.  Therefore $\hat{\rho}_{a,b}(G_h) < c$.  Pick $W \unlhd G_h$ with $\rho_{a,b}(W) < c$.  Since $W \not \subseteq G$, we must have $V(W) \cap V(Q) \ne \emptyset$.  Hence $\rho_{a,b}\parens{G\brackets{(V(W) - V(Q)) \cup V(H)}} \le \rho_{a,b}(W) - \hat{\rho}_{a,b}(Q) + \rho_{a,b}(H) \le  \rho_{a,b}(W) - \hat{\rho}_{a,b}(Q) + \hat{\rho}_{a,b}(Q) = \rho_{a,b}(W) < c$, a contradiction since $\hat{\rho}_{a,b}(G) \ge c$.
\end{proof}

We can easily weaken the condition $H \surj Q$ to $H \rightarrow Q$.

\begin{cor}\label{ArbitraryQInto}
Let $G$ be a $\C_{T,a,b,c}$-mule and $Q$ an arbitrary graph.  If $H \unlhd G$ with $H \ne Q$ such that $H \rightarrow Q$, then $\rho_{a,b}(H) > \hat{\rho}_{a,b}(Q)$.
\end{cor}
\begin{proof}
Let $h$ be a homomorphism from $H$ into $Q$ and put $Q' \DefinedAs \im(h)$.  Then $H \ne Q'$ and $H \surj Q'$, so Lemma \ref{ArbitraryQ} gives $\rho_{a,b}(H) > \hat{\rho}_{a,b}(Q') \ge  \hat{\rho}_{a,b}(Q)$.
\end{proof}

We have the following basic bound on the potential of non-complete subgraphs of $G$.
\begin{cor}\label{CompleteQ}
Let $G$ be a $\C_{T,a,b,c}$-mule. If $H \unlhd G$ is not complete and $\chi(H) \le \frac{2a}{b}$, then $\rho_{a,b}(H) > a$.
\end{cor}
\begin{proof}
Suppose $\chi(H) = k \le \frac{2a}{b}$.  Then there is an epimorphism from $H$ onto $K_k$ given by contracting all color classes in a $k$-coloring of $H$.  Since $H \ne K_k$, Lemma \ref{ArbitraryQ} gives
$\rho_{a,b}(H) > \hat{\rho}_{a,b}(K_k)$.  But $\hat{\rho}_{a,b}(K_k) = \min_{t \in \irange{k}} at - b\binom{t}{2} = a$ since $k \le \frac{2a}{b}$, so we have the desired bound.
\end{proof}

% We may want to just do the following in the specific C_{2k+1} case.
%
%There is room for improvement in the proof of Lemma \ref{ArbitraryQ}; in particular, the bound 
%\[\rho_{a,b}\parens{G\brackets{(V(W) - V(Q)) \cup V(H)}} \le \rho_{a,b}(W) - \hat{\rho}_{a,b}(Q) + \rho_{a,b}(H)\] can be improved in many cases. Put $X \DefinedAs V(W) - V(Q)$ and $Y \DefinedAs h^{-1}(V(W) \cap V(Q))$.  We did not count any of the edges between $X$ and $H - Y$ in our estimate, so in fact we can improve the upper bound to \[\rho_{a,b}(W) - \hat{\rho}_{a,b}(Q) + \rho_{a,b}(H) - b\card{E_G(X,H - Y)}.\]  Furthermore, we gain when a vertex in $X$ has more than $\card{V(W) \cap V(Q)}$ neighbors in $Y$ because multi-edges get replaced by single edges.  Taking this into account, improves the upper bound to \[\rho_{a,b}(W) - \hat{\rho}_{a,b}(Q) + \rho_{a,b}(H) - b\card{E_G(X,H - Y)} - b\sum_{v \in X} \max\set{0, \card{N_G(v) \cap Y} - \card{V(W) \cap V(Q)}}.\]  Finally, we may be able to improve on the use of $\hat{\rho}_{a,b}(Q)$ when $\card{V(W) \cap V(Q)} > 1$.  There is a tension between this final improvement and the previous two.  For many $Q$ in the wild, $\rho_{a,b}(Q') > \rho_{a,b}(K_1)$ for all $Q' \unlhd Q$ with $\card{Q'} > 1$.  When this occurs, our upper bound is improved by $1$ unless $\card{V(W) \cap V(Q)} = 1$ and $\card{E(X,H - Y)} = 0$ and $\card{N(v) \cap Y} \le 1$ for all $v \in X$.  To apply this observation, we need some control over $W$. We can get this control when $Q = T$, we will be able to conclude that $(V(W) - V(Q)) \cup V(H) = V(G)$. 
%
%\begin{defn}
%Put $\tilde{\rho}_{a,b}(G) \DefinedAs \min\setbs{\rho_{a,b}(H)}{H \unlhd G \text{ with } |H| \ge 2}$.
%\end{defn}
%
%\begin{lem}\label{UsingT}
%Let $G$ be a $2$-connected $\C_{T,a,b,c}$-mule where $\tilde{\rho}_{a,b}(T) > \hat{\rho}_{a,b}(T) \ge b + c - 1$.  If $H \lhd G$ with $H \not \inj T$, then $\rho_{a,b}(H) > \hat{\rho}_{a,b}(T) + 1$.
%\end{lem}
%\begin{proof}
%Suppose to the contrary that we have such an $H$ with $\rho_{a,b}(H) \le \hat{\rho}_{a,b}(T) + 1$. Since $H \lhd G$, Lemma \ref{Criticality} shows that $H \rightarrow T$. Applying Corollary \ref{ArbitraryQInto} gives $\rho_{a,b}(H) > \hat{\rho}_{a,b}(T)$.  The same holds for any induced subgraph of $H$, so $\hat{\rho}_{a,b}(H) > \hat{\rho}_{a,b}(T)$ and hence $\hat{\rho}_{a,b}(H) = \hat{\rho}_{a,b}(T) + 1$.  
%
%Since $G$ is $2$-connected, $H$ has at least two vertices $x, y$ with neighbors outside $H$.  If $x$ is adjacent to $y$, put $H' \DefinedAs H$, otherwise put $H' \DefinedAs H + xy$.  We have $\hat{\rho}_{a,b}(H') \ge \hat{\rho}_{a,b}(H) - b \ge \hat{\rho}_{a,b}(T) + 1 - b \ge c$.   Since $H' \prec G$, we must have $H' \rightarrow T$ as $G$ is a $\C_{T,a,b,c}$-mule.
%
%Let $h$ be a homomorphism from $H'$ into $T$.  Then $h$ is a homomorphism from $H$ into $T$ with the property that $h(x) \ne h(y)$. Put $Q \DefinedAs \im(h)$. Now $H \ne Q$ since $H \not \inj T$ and $h$ is an epimorphism from $H$ onto $Q$. Since $G$ is a $\C_{T,a,b,c}$-mule, $G_h$ cannot be a child of $G$.  But we have an epimorphism $h'$ from $G$ onto $G_h$ and $G_h \ne G$ since $H \ne Q$, so it must be that $G_h \not \in \C_{T,a,b,c}$.  Since $G \rightarrow G_h$ and $G \not \rightarrow T$, we must have $G_h \not \rightarrow T$.  Therefore $\hat{\rho}_{a,b}(G_h) < c$.  Pick $W \unlhd G_h$ with $\rho_{a,b}(W) < c$.  Since $W \not \subseteq G$, we must have $V(W) \cap V(Q) \ne \emptyset$.  Hence $\rho_{a,b}\parens{G\brackets{(V(W) - V(Q)) \cup V(H)}} \le \rho_{a,b}(W) - \hat{\rho}_{a,b}(T) + \rho_{a,b}(H) \le c$.  Suppose $(V(W) - V(Q)) \cup V(H) \ne V(G)$.  Then Lemma \ref{Criticality} shows that $G\brackets{(V(W) - V(Q)) \cup V(H)} \rightarrow T$.  Applying Corollary \ref{ArbitraryQInto} gives $\rho_{a,b}\parens{G\brackets{(V(W) - V(Q)) \cup V(H)}} > \hat{\rho}_{a,b}(T)$ and hence $\hat{\rho}_{a,b}(T) < c$, a contradiction.  
%
%So, we have $(V(W) - V(Q)) \cup V(H) \ne V(G)$.  Since $\tilde{\rho}_{a,b}(T) > \hat{\rho}_{a,b}(T)$, if $\card{V(W) \cap V(Q)} > 1$, then our above estimate on $\rho_{a,b}(G)$ is decreased by one, giving $\hat{\rho}_{a,b}(G) < c$, a contradiction.  So we must have $V(W) \cap V(Q) = \set{z}$ for some $z$.  By the discussion before this lemma, we must also have $\card{E(G - H , H - h^{-1}(z))} = 0$.  But $h(x) \ne h(y)$, so by symmetry, we may assume that $h(y) \ne h(z)$.  Therefore $y \in V(H - h^{-1}(z))$ and $y$ has a neighbor in $G-H$, a contradiction.
%\end{proof}

%*** not sure we need/can use this more general statement for odd cycle homomorphisms
%*** need to be able to exclude small edge cuts to use it
%*** note for future reference if we want to use this stuff that the formulation below isn't quite right, the bound on $\tilde{\rho}_{a,b}(T)$ can be improved using minimality of $\rho_{a,b}(H)$ like the kostochka
% yancey proof
%
%We need a lemma from Kostochka and Yancey \cite{kostochkayancey2012ore}.
%
%\begin{lem}[Kostochka and Yancey \cite{kostochkayancey2012ore}]\label{EdgeAddingLemma}
%Let $S$ be a finite set, $\ell \ge 2$ an integer and $\func{f}{S}{\IN_{\ge 1}}$ such that $\sum_{v \in S} f(v) \ge \ell$.  Then, for any $i \in \irange{\frac{\ell}{2}}$, there is a graph $H$ with $V(H) = S$ and $||H|| = i$ such that for any independent set $I$ in $H$ with $\card{I} \ge 2$, we have \[\sum_{v \in S - M} f(v) \ge i.\]
%\end{lem}
%
%\begin{lem}
%Let $G$ be a $2$-connected $\C_{T,a,b,c}$-mule where $\hat{\rho}_{a,b}(T) \ge c$ and $\tilde{\rho}_{a,b}(T) \le 2\hat{\rho}_{a,b}(T) + 2 - b - c$.  If $H \lhd G$ with $|H| > 1$ and $H \not \inj T$ such that $\card{E(H, G-H)} \ge 2\floor{\frac{\tilde{\rho}_{a,b}(T) - 1}{b}}$, then $\rho_{a,b}(H) > \tilde{\rho}_{a,b}(T)$.
%\end{lem}
%\begin{proof}
%Suppose not and choose $H \lhd G$ with $|H| > 1$ and $H \not \inj T$ minimizing $\rho_{a,b}(H)$. Since $H \lhd G$, Lemma \ref{Criticality} shows that $H \rightarrow T$. Applying Corollary \ref{ArbitraryQInto} gives $\rho_{a,b}(H) > \hat{\rho}_{a,b}(T)$.  The same holds for any induced subgraph of $H$, so $\hat{\rho}_{a,b}(H) > \hat{\rho}_{a,b}(T)$.
%
%Let $H'$ be a graph formed from $H$ by adding $i \DefinedAs \floor{\frac{\hat{\rho}_{a,b}(H) - c}{b}}$ edges (**IN A GOOD WAY**).  Then $\hat{\rho}_{a,b}(H') \ge \hat{\rho}_{a,b}(H) - ib \ge c$.  Since $H' \prec G$, we must have $H' \rightarrow T$ as $G$ is a $\C_{T,a,b,c}$-mule.
%
%Let $h$ be a homomorphism from $H'$ into $T$.  Then $h$ is a homomorphism from $H$ into $T$ with special properties we will use later.   Put $Q \DefinedAs \im(h)$. Now $H \ne Q$ since $H \not \inj T$ and $h$ is an epimorphism from $H$ onto $Q$. Since $G$ is a $\C_{T,a,b,c}$-mule, $G_h$ cannot be a child of $G$.  But we have an epimorphism $h'$ from $G$ onto $G_h$ and $G_h \ne G$ since $H \ne Q$, so it must be that $G_h \not \in \C_{T,a,b,c}$.  Since $G \rightarrow G_h$ and $G \not \rightarrow T$, we must have $G_h \not \rightarrow T$.  Therefore $\hat{\rho}_{a,b}(G_h) < c$.  Pick $W \unlhd G_h$ with $\rho_{a,b}(W) < c$.  Since $W \not \subseteq G$, we must have $V(W) \cap V(Q) \ne \emptyset$.  Hence $\rho_{a,b}\parens{G\brackets{(V(W) - V(Q)) \cup V(H)}} \le \rho_{a,b}(W) - \hat{\rho}_{a,b}(T) + \rho_{a,b}(H) < \rho_{a,b}(H) + c - \hat{\rho}_{a,b}(T) \le \rho_{a,b}(H)$.  This contradicts the minimality of $\rho_{a,b}(H)$ unless $(V(W) - V(Q)) \cup V(H) \ne V(G)$.  Also, if $\card{V(W) \cap V(Q)} > 1$, then we have $\rho_{a,b}\parens{G\brackets{(V(W) - V(Q)) \cup V(H)}} \le \rho_{a,b}(W) - \tilde{\rho}_{a,b}(T) + \rho_{a,b}(H) < c$, a contradiction.
%
%Hence $(V(W) - V(Q)) \cup V(H) = V(G)$ and $V(W) \cap V(Q) = \set{z}$ for some $z$.  When forming $H'$, we added edges to $H$ in a way that guarantees $\card{E(G - H , H - h^{-1}(z))} \ge i$.
%By the discussion before this lemma we get $\rho_{a,b}(G) < c - \hat{\rho}_{a,b}(T) + \rho_{a,b}(H) - b\card{E(G - H , H - h^{-1}(z))} \le c - \hat{\rho}_{a,b}(T) + \rho_{a,b}(H) - ib \le c - \hat{\rho}_{a,b}(T) + \rho_{a,b}(H) - (\hat{\rho}_{a,b}(H) - c - (b-1)) = 2c + b-1 - \hat{\rho}_{a,b}(T) + \rho_{a,b}(H) - \hat{\rho}_{a,b}(H)$.
%\end{proof}

%\begin{lem}\label{VertexTransitive}
%Let $G$ be $\C_{T,a,b,c}$-mule where $T$ is vertex-transitive and $\tilde{\rho}_{a,b}(T) > \hat{\rho}_{a,b}(T) \ge b + c - 1$.  If $H \lhd G$ with $H \not \inj T$, then $\rho_{a,b}(H) > \hat{\rho}_{a,b}(T) + 1$.
%\end{lem}
%\begin{proof}
%Suppose $G$ has a cutvertex $x$.  Then, by Lemma \ref{Criticality}, each lobe has a homomorphism into $T$.  Since $T$ is vertex-transitive, we can match these up on $x$ to get $G \rightarrow T$, a contradiction.  So, $G$ is $2$-connected and we are done by Lemma \ref{UsingT}.
%\end{proof}

\section{Homomorphisms into odd cycles}
For $k \in \IN_{\ge 1}$, put $\H_k \DefinedAs \C_{C_{2k+1}, 4k+1, 4k-1, 4k-2}$ and $\rho_k \DefinedAs \rho_{4k+1, 4k-1}$.  Then $\hat{\rho}_k(C_{2k+1}) = 4k+1$.

\bibliographystyle{plain}
\bibliography{GraphColoring}
\end{document}



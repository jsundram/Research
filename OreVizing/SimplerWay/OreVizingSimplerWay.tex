\documentclass[12pt]{article}
\usepackage{amsmath, amsthm, amssymb}
\usepackage{tkz-graph}
\usepackage{marginnote}
\usepackage{verbatim}
\usepackage[top=1.0in, bottom=1.0in, left=1.0in, right=1.0in]{geometry}
\usepackage{color}
\pagestyle{plain}

\usepackage[backref=page]{hyperref}

\usepackage{sectsty}
\allsectionsfont{\sffamily}

\setcounter{secnumdepth}{5}
\setcounter{tocdepth}{5}

\makeatletter
\newtheorem*{rep@theorem}{\rep@title}
\newcommand{\newreptheorem}[2]{
\newenvironment{rep#1}[1]{
 \def\rep@title{#2 \ref{##1}}
 \begin{rep@theorem}}
 {\end{rep@theorem}}}
\makeatother

\theoremstyle{plain}
\newtheorem{thm}{Theorem}
\newreptheorem{thm}{Theorem}
\newtheorem{prop}[thm]{Proposition}
\newreptheorem{prop}{Proposition}
\newtheorem{lem}[thm]{Lemma}
\newreptheorem{lem}{Lemma}
\newtheorem{conjecture}[thm]{Conjecture}
\newreptheorem{conjecture}{Conjecture}
\newtheorem{cor}[thm]{Corollary}
\newreptheorem{cor}{Corollary}
\newtheorem{prob}[thm]{Problem}

\newtheorem*{KernelLemma}{Kernel Lemma}
\newtheorem*{KernelMagic}{Kernel Magic}
\newtheorem*{TheLemma}{Lemma}
\newtheorem*{MainTheorem}{Main Theorem}
\newtheorem*{BK}{Borodin-Kostochka Conjecture}
\newtheorem*{TheConjecture}{Conjecture}
\newtheorem*{BK2}{Borodin-Kostochka Conjecture (restated)}
\newtheorem*{Reed}{Reed's Conjecture}
\newtheorem*{ClassificationOfd0}{Classification of $d_0$-choosable graphs}


\theoremstyle{definition}
\newtheorem*{TheDefinition}{Definition}
\newtheorem{defn}{Definition}
\theoremstyle{remark}
\newtheorem*{remark}{Remark}
\newtheorem*{problem}{Problem}
\newtheorem{example}{Example}
\newtheorem*{question}{Question}
\newtheorem*{observation}{Observation}

\newcommand{\fancy}[1]{\mathcal{#1}}
\newcommand{\C}[1]{\fancy{C}_{#1}}


\newcommand{\IN}{\mathbb{N}}
\newcommand{\IR}{\mathbb{R}}
\newcommand{\G}{\fancy{G}}
\newcommand{\CC}{\fancy{C}}
\newcommand{\D}{\fancy{D}}
\newcommand{\T}{\fancy{T}}
\newcommand{\B}{\fancy{B}}
\renewcommand{\L}{\fancy{L}}
\newcommand{\HH}{\fancy{H}}

\newcommand{\inj}{\hookrightarrow}
\newcommand{\surj}{\twoheadrightarrow}

\newcommand{\set}[1]{\left\{ #1 \right\}}
\newcommand{\setb}[3]{\left\{ #1 \in #2 : #3 \right\}}
\newcommand{\setbs}[2]{\left\{ #1 : #2 \right\}}
\newcommand{\card}[1]{\left|#1\right|}
\newcommand{\size}[1]{\left\Vert#1\right\Vert}
\newcommand{\ceil}[1]{\left\lceil#1\right\rceil}
\newcommand{\floor}[1]{\left\lfloor#1\right\rfloor}
\newcommand{\func}[3]{#1\colon #2 \rightarrow #3}
\newcommand{\funcinj}[3]{#1\colon #2 \inj #3}
\newcommand{\funcsurj}[3]{#1\colon #2 \surj #3}
\newcommand{\irange}[1]{\left[#1\right]}
\newcommand{\join}[2]{#1 \mbox{\hspace{2 pt}$\ast$\hspace{2 pt}} #2}
\newcommand{\djunion}[2]{#1 \mbox{\hspace{2 pt}$+$\hspace{2 pt}} #2}
\newcommand{\parens}[1]{\left( #1 \right)}
\newcommand{\brackets}[1]{\left[ #1 \right]}
\newcommand{\DefinedAs}{\mathrel{\mathop:}=}

\newcommand{\mic}{\operatorname{mic}}
\newcommand{\AT}{\operatorname{AT}}
\newcommand{\col}{\operatorname{col}}
\newcommand{\ch}{\operatorname{ch}}
\newcommand{\type}{\operatorname{type}}
\newcommand{\nonsep}{\bar{S}}

\def\adj{\leftrightarrow}
\def\nonadj{\not\!\leftrightarrow}

\newcommand\restr[2]{{% make the whole thing an ordinary symbol
  \left.\kern-\nulldelimiterspace % automatically resize the bar with \right
  #1 % the function
  \vphantom{\big|} % pretend it's a little taller at normal size
  \right|_{#2} % this is the delimiter
  }}

\def\D{\fancy{D}}
\def\C{\fancy{C}}
\def\A{\fancy{A}}
\def\chil{{\chi_\ell}}
\def\chiol{\chi_{\rm{OL}}}

\newcommand{\case}[2]{{\bf Case #1.}~{\it #2}~~}
\newcommand{\aside}[1]{\marginnote{\scriptsize{#1}}[0cm]}
\newcommand{\aaside}[2]{\marginnote{\scriptsize{#1}}[#2]}

\title{An improved Ore-type version of Brooks' theorem for list coloring.}
\author{Landon Rabern}

\begin{document}
\maketitle

\section{Scratch}
\begin{TheDefinition} The \emph{maximum independent cover number }of a graph $G$
	is the maximum $\mic(G)$ of $\size{I, V(G) \setminus I}$ over all independent sets $I$
	of $G$. 
\end{TheDefinition}

\begin{KernelMagic}[Kierstead and R. \cite{KernelMagic}]\label{ConsantListMicStrength} 
	Every $k$-list-critical graph $G$ satisfies
	\[2\size{G} \ge (k-2)\card{G} + \mic(G) + 1.\]
\end{KernelMagic}

The connected graphs in which each block is a complete graph
or an odd cycle are called \emph{Gallai trees}.  Gallai \cite{gallai1963kritische} proved that in a $k$-critical graph, the vertices of degree $k-1$ induce a disjoint union of Gallai trees.  
The same is true for $k$-list-critical graphs \cite{borodin1977criterion, erdos1979choosability}.  For a graph $T$ and $k \in \IN$, let $\beta_k(T)$ be the independence number of the subgraph of $T$ 
induced on the vertices of degree $k-1$ in $T$.  When $k$ is defined in the context, put $\beta(T) 
\DefinedAs \beta_k(T)$.  Let $c(T)$ be the number of components of $T$.

\begin{lem}\label{BetaBoundGeneral}
	If $k \ge 4$ and $T \ne K_k$ is a Gallai tree with maximum degree at most $k-1$, then for any $p(k)$ with $\frac{2}{k-2} \le p(k) \le 1$,
	\[2||T|| \le \parens{k-3 + p(k)}|T| + (k-1)(1-p(k)) + (2 + (k-1)(1-p(k)))\beta(T).\]
\end{lem}

Let $G$ be a $k$-list-critical graph with $\Delta(G) = k$ such that $\HH$ is edgeless.  Then
\[k\card{\HH} = \size{\HH, \L} = (k-1)\card{\L} - 2\size{\L},\]
so
\begin{equation}
2\size{\L} = (k-1)\card{\L} - k\card{\HH}.
\end{equation}
Combined with Lemma \ref{BetaBoundGeneral}, this gives
\begin{equation}\label{StraightFromBeta}
\parens{2-p(k)}\card{\L} \le k\card{\HH} + (k-1)(1-p(k))c(\L) + (2 + (k-1)(1-p(k)))\beta(\L)
\end{equation}
Also,
\begin{equation}
2\card{\HH} + \card{\L} = \card{G} + \card{\HH} > \mic(G) \ge k\card{\HH} + (k-1)\beta(\L),
\end{equation}
so with \eqref{StraightFromBeta}, this gives
\begin{equation}\label{MasterP}
\parens{1-p(k)}\card{\L} < 2\card{\HH} + (k-1)(1-p(k))c(\L) + (2 - (k-1)p(k))\beta(\L),
\end{equation}
with $p(k) = 1$, this is
\begin{equation}\label{MasterPWithMax}
(k-3)\beta(\L) < 2\card{\HH}.
\end{equation}
Using \eqref{MasterP} with $p(k) = \frac{2}{k-2}$ gives
\begin{equation}
\frac{k-4}{k-2}\card{\L} < 2\card{\HH} + \frac{(k-1)(k-4)}{k-2}c(\L) - \frac{2}{k-2}\beta(\L),}
\end{equation}
so
\begin{equation}\label{MasterPWithMin}
\card{\L} < \frac{2(k-2)}{k-4}\card{\HH} + (k-1)c(\L) - \frac{2}{k-4}\beta(\L),}
\end{equation}


\bibliographystyle{amsplain}
\bibliography{GraphColoring1}
\end{document} 

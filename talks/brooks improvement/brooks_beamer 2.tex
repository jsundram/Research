\documentclass{beamer}
\usepackage{tkz-graph}
\usetikzlibrary{shapes}
%===============BEGIN GLOBAL BEAMER OPTIONS=====================

%THIS CHANGES THE BACKGROUND COLOUR OF BOXES, AND ROUNDS OFF THEIR EDGES
%\setbeamercolor{block title}{bg=yellow!50}
%\setbeamercolor{block body}{}
%\setbeamertemplate{blocks}[default]

%THIS REMOVES NAVIGATION BAR ON THE BOTTOM
\setbeamertemplate{navigation symbols}{}


%%THIS ADDS FRAME NUMBERS TO THE Right (x/y)
%\newcommand*\oldmacro{}
%\let\oldmacro\insertshorttitle
%\renewcommand*\insertshorttitle{
 %\oldmacro\hfill
 %\insertframenumber\,/\,\inserttotalframenumber}

%+++++++++++++ MODES AND THEMES +++++++++
\mode<presentation>

\usetheme[hideothersubsections]{Hannover}
\addtobeamertemplate{footline}
{
  \usebeamercolor[fg]{author in sidebar}
  \vskip-1cm\hskip2pt
  %\insertpagenumber\,/\,\insertpresentationendpage\kern1em\vskip2pt%
  \insertframenumber\,/\,\inserttotalframenumber\kern1em\vskip2pt%
}

%\usecolortheme{beaver}
%\usecolortheme{crane}
%\usecolortheme{default}
%\usecolortheme{albatross}
%\usecolortheme{beetle} 
%\usecolortheme{dove} 
%\usecolortheme{fly} 
%\usecolortheme{seagull}
%\usecolortheme{rose}
\usecolortheme{orchid}
%\usecolortheme{seahorse}



%\useoutertheme{split}
%\useoutertheme{shadow}

% USE TO CREATE PRINTER FRIENDLY HANDOUT VERSION, TOGETHER WITH [HANDOUT]
%ALSO USE ALTERNATE FIRST SLIDE
%\usecolortheme{dove}

%THIS CHANGES THE DEFAULT TRANSPARENCY GRADE FOR UNCOVERING COMMANDS
%\setbeamercovered{transparent=30}

%\beamertemplateshadingbackground{yellow!70}{black!20}

%====================================

\title[Improving Brooks' theorem]{Improving Brooks' theorem}
\author{Landon Rabern}
\institute{Arizona State University}
\date{October 28, 2011}

\theoremstyle{plain}
\newtheorem{thm}{Theorem}
\newtheorem{prop}[thm]{Proposition}
\newtheorem{lem}[thm]{Lemma}
\newtheorem{cor}[thm]{Corollary}
\newtheorem*{OreBrooks}{Theorem}
\newtheorem*{DeltaTwo}{Theorem}
\newtheorem*{Spectrum}{Theorem}
\newtheorem*{OreBrooksKK}{Theorem}
\newtheorem*{krs1}{Theorem}
\newtheorem*{krs2}{Theorem}
\newtheorem*{BrooksTheorem}{Theorem}
\newtheorem*{MozhansLemma}{Lemma}
\newtheorem*{conjecture}{Conjecture}
\newtheorem*{PartyGame}{A prison problem}
\newtheorem{claim}{Claim}
\theoremstyle{definition}
\newtheorem{defn}{Definition}
\theoremstyle{remark}
\newtheorem*{remark}{Remark}
\newtheorem*{goal}{Goal}
\newtheorem*{question}{Question}
\newtheorem*{observation}{Observation}
\newtheorem*{TemptingThought}{A tempting thought}

\newcommand{\fancy}[1]{\mathcal{#1}}
\newcommand{\C}[1]{\fancy{C}_{#1}}
\newcommand{\IN}{\mathbb{N}}
\newcommand{\IR}{\mathbb{R}}

\newcommand{\inj}{\hookrightarrow}
\newcommand{\surj}{\twoheadrightarrow}

\newcommand{\set}[1]{\left\{ #1 \right\}}
\newcommand{\setb}[3]{\left\{ #1 \in #2 \mid #3 \right\}}
\newcommand{\setbs}[2]{\left\{ #1 \mid #2 \right\}}
\newcommand{\card}[1]{\left|#1\right|}
\newcommand{\size}[1]{\left\Vert#1\right\Vert}
\newcommand{\ceil}[1]{\left\lceil#1\right\rceil}
\newcommand{\floor}[1]{\left\lfloor#1\right\rfloor}
\newcommand{\func}[3]{#1\colon #2 \rightarrow #3}
\newcommand{\funcinj}[3]{#1\colon #2 \inj #3}
\newcommand{\funcsurj}[3]{#1\colon #2 \surj #3}
\newcommand{\irange}[1]{\left[#1\right]}
\newcommand{\join}[2]{#1 \mbox{\hspace{2 pt}$\ast$\hspace{2 pt}} #2}
\newcommand{\djunion}[2]{#1 \mbox{\hspace{2 pt}$+$\hspace{2 pt}} #2}
\newcommand{\parens}[1]{\left( #1 \right)}

\newcommand{\DefinedAs}{\mathrel{\mathop:}=}

%\AtBeginSubsection
%{
%  \begin{frame}<beamer>{Outline}
%    \tableofcontents[currentsection,currentsubsection]
%  \end{frame}
%}

%\AtBeginSection
%{
%  \begin{frame}<beamer>{Outline}
%    \tableofcontents[currentsection,currentsubsection]
%  \end{frame}
%}

\newcommand{\1}{\item<1-> }
\newcommand{\2}{\item<2-> }
\newcommand{\3}{\item<3-> }
\newcommand{\4}{\item<4-> }
\newcommand{\5}{\item<5-> }
\newcommand{\6}{\item<6-> }
\newcommand{\7}{\item<7-> }
\newcommand{\8}{\item<8-> }
\newcommand{\9}{\item<9-> }
\newcommand{\ten}{\item<10-> }
\newcommand{\ele}{\item<11-> }
\newcommand{\twe}{\item<12-> }
\newcommand{\thi}{\item<13-> }
\newcommand{\fou}{\item<14-> }
\newcommand{\fif}{\item<15-> }
\newcommand{\six}{\item<16-> }
\newcommand{\sev}{\item<17-> }
\newcommand{\eig}{\item<18-> }

\begin{document}
\begin{frame}
\titlepage
\end{frame}

\begin{frame}{Outline}
  \tableofcontents
\end{frame}

\section{A prison problem}
\begin{frame}{A prison problem}
\uncover<1->{
\begin{PartyGame}
You are a warden in a prison with five large cells.  You need to put
all the inmates into the cells, but to prevent fighting you cannot put
a pair of inmates that have fought before into the same cell.  Each inmate in
the prison has fought with at most six other inmates and none of the
inmates who have fought with six others have fought with each other.  Under what conditions can you complete your task?
\end{PartyGame}}

\begin{itemize}
\2 plainly, if there is a group of six inmates who have all fought one
another, then you cannot complete your task
\3 is this simple necessary condition sufficient?
\end{itemize}
\end{frame}

\section{Some background}
\begin{frame}[t]{Greedy coloring}
\begin{itemize}
\1 $C \DefinedAs \set{c_1, c_2, c_3, \ldots}$ an infinite set of colors
\2 $G$ has vertices ordered $v_1, v_2, \ldots, v_n$
\3 go through the vertices in order, coloring $v_i$ with the first color not used on a neighbor of $v_i$
\end{itemize}

\uncover<4->{
For example, say $C \DefinedAs \set{\text{red}, \text{green}, \text{blue}, \text{cyan}, \ldots}$ and $G$ is the $5$-cycle:
}

\begin{overprint}
\onslide<4| handout:0>\begin{figure}[h]
\centering
\begin{tikzpicture}[scale = 10]
\tikzstyle{VertexStyle}=[shape = circle,	
								 minimum size = 8pt,
								 inner sep = 1.2pt,
                         draw]
\Vertex[x = 0.3, y = 0.75, L = \tiny {}]{v0}
\Vertex[x = 0.4, y = 0.75, L = \tiny {}]{v1}
\Vertex[x = 0.5, y = 0.75, L = \tiny {}]{v2}
\Vertex[x = 0.6, y = 0.75, L = \tiny {}]{v3}
\Vertex[x = 0.7, y = 0.75, L = \tiny {}]{v4}
\Edge[](v4)(v3)
\Edge[style = {bend left}](v4)(v0)
\Edge[](v0)(v1)
\Edge[](v2)(v1)
\Edge[](v2)(v3)
\end{tikzpicture}
\end{figure}
\bigskip
\onslide<5| handout:0>\begin{figure}[h]
\centering
\begin{tikzpicture}[scale = 10]
\tikzstyle{VertexStyle}=[shape = circle,	
								 minimum size = 8pt,
								 inner sep = 1.2pt,
                         draw]
\Vertex[style = {fill=red}, x = 0.3, y = 0.75, L = \tiny {}]{v0}
\Vertex[x = 0.4, y = 0.75, L = \tiny {}]{v1}
\Vertex[x = 0.5, y = 0.75, L = \tiny {}]{v2}
\Vertex[x = 0.6, y = 0.75, L = \tiny {}]{v3}
\Vertex[x = 0.7, y = 0.75, L = \tiny {}]{v4}
\Edge[](v4)(v3)
\Edge[style = {bend left}](v4)(v0)
\Edge[](v0)(v1)
\Edge[](v2)(v1)
\Edge[](v2)(v3)
\end{tikzpicture}
\end{figure}
\bigskip
\onslide<6| handout:0>\begin{figure}[h]
\centering
\begin{tikzpicture}[scale = 10]
\tikzstyle{VertexStyle}=[shape = circle,	
								 minimum size = 8pt,
								 inner sep = 1.2pt,
                         draw]
\Vertex[style = {fill=red}, x = 0.3, y = 0.75, L = \tiny {}]{v0}
\Vertex[style = {fill=green}, x = 0.4, y = 0.75, L = \tiny {}]{v1}
\Vertex[x = 0.5, y = 0.75, L = \tiny {}]{v2}
\Vertex[x = 0.6, y = 0.75, L = \tiny {}]{v3}
\Vertex[x = 0.7, y = 0.75, L = \tiny {}]{v4}
\Edge[](v4)(v3)
\Edge[style = {bend left}](v4)(v0)
\Edge[](v0)(v1)
\Edge[](v2)(v1)
\Edge[](v2)(v3)
\end{tikzpicture}
\end{figure}
\bigskip
\onslide<7| handout:0>\begin{figure}[h]
\centering
\begin{tikzpicture}[scale = 10]
\tikzstyle{VertexStyle}=[shape = circle,	
								 minimum size = 8pt,
								 inner sep = 1.2pt,
                         draw]
\Vertex[style = {fill=red}, x = 0.3, y = 0.75, L = \tiny {}]{v0}
\Vertex[style = {fill=green},x = 0.4, y = 0.75, L = \tiny {}]{v1}
\Vertex[style = {fill=red},x = 0.5, y = 0.75, L = \tiny {}]{v2}
\Vertex[x = 0.6, y = 0.75, L = \tiny {}]{v3}
\Vertex[x = 0.7, y = 0.75, L = \tiny {}]{v4}
\Edge[](v4)(v3)
\Edge[style = {bend left}](v4)(v0)
\Edge[](v0)(v1)
\Edge[](v2)(v1)
\Edge[](v2)(v3)
\end{tikzpicture}
\end{figure}
\bigskip
\onslide<8| handout:0>\begin{figure}[h]
\centering
\begin{tikzpicture}[scale = 10]
\tikzstyle{VertexStyle}=[shape = circle,	
								 minimum size = 8pt,
								 inner sep = 1.2pt,
                         draw]
\Vertex[style = {fill=red}, x = 0.3, y = 0.75, L = \tiny {}]{v0}
\Vertex[style = {fill=green},x = 0.4, y = 0.75, L = \tiny {}]{v1}
\Vertex[style = {fill=red},x = 0.5, y = 0.75, L = \tiny {}]{v2}
\Vertex[style = {fill=green},x = 0.6, y = 0.75, L = \tiny {}]{v3}
\Vertex[x = 0.7, y = 0.75, L = \tiny {}]{v4}
\Edge[](v4)(v3)
\Edge[style = {bend left}](v4)(v0)
\Edge[](v0)(v1)
\Edge[](v2)(v1)
\Edge[](v2)(v3)
\end{tikzpicture}
\end{figure}
\bigskip
\onslide<9-| handout:1>\begin{figure}[h]
\centering
\begin{tikzpicture}[scale = 10]
\tikzstyle{VertexStyle}=[shape = circle,	
								 minimum size = 8pt,
								 inner sep = 1.2pt,
                         draw]
\Vertex[style = {fill=red}, x = 0.3, y = 0.75, L = \tiny {}]{v0}
\Vertex[style = {fill=green},x = 0.4, y = 0.75, L = \tiny {}]{v1}
\Vertex[style = {fill=red},x = 0.5, y = 0.75, L = \tiny {}]{v2}
\Vertex[style = {fill=green},x = 0.6, y = 0.75, L = \tiny {}]{v3}
\Vertex[style = {fill=blue},x = 0.7, y = 0.75, L = \tiny {}]{v4}
\Edge[](v4)(v3)
\Edge[style = {bend left}](v4)(v0)
\Edge[](v0)(v1)
\Edge[](v2)(v1)
\Edge[](v2)(v3)
\end{tikzpicture}
\end{figure}
\bigskip
\end{overprint}

\begin{itemize}
\ten if $G$ has maximum degree $k$, then $v_i$ has at most $k$ colored neighbors, so greedy coloring uses at most $k+1$ colors
\end{itemize}
\end{frame}
\begin{frame}{Brooks' theorem}
\begin{itemize}
\1 $\chi(G) \DefinedAs $ the minimum number of colors needed to color the vertices of $G$ so that adjacent vertices receive different colors \\
\2 $\omega(G) \DefinedAs $ the number of vertices in a largest complete subgraph of $G$ \\
\3 $\Delta(G) \DefinedAs $ the maximum degree of $G$
\end{itemize}
\uncover<4->{
\bigskip
\begin{BrooksTheorem}[Brooks 1941]
Every graph with $\Delta \geq 3$ satisfies $\chi \leq \max\{\omega, \Delta\}$.
\end{BrooksTheorem}}
\end{frame}
\begin{frame}[t]{Proof sketch}
\uncover<1->{Any incomplete $2$-connected graph with $\Delta \geq 3$ has a spanning tree where the root has two nonadjacent leaves as neighbors.}
\begin{overprint}
\onslide<1-2| handout:1>\begin{figure}[h]
\centering
\begin{tikzpicture}[scale = 10]
\tikzstyle{VertexStyle}=[shape = circle,	
								 minimum size = 12pt,
								 inner sep = 1.2pt,
                         draw]
\Vertex[x = 0.697999894618988, y = 0.750000074505806, L = \tiny {$4$}]{v0}
\Vertex[x = 0.695999979972839, y = 0.591999977827072, L = \tiny {$4$}]{v1}
\Vertex[x = 0.496000051498413, y = 0.59400001168251, L = \tiny {$4$}]{v2}
\Vertex[x = 0.69599986076355, y = 0.840000033378601, L = \tiny {$4$}]{v3}
\Vertex[x = 0.496000111103058, y = 0.526000112295151, L = \tiny {$4$}]{v4}
\Vertex[x = 0.582000017166138, y = 0.636000007390976, L = \tiny {$4$}]{v5}
\Vertex[x = 0.580000042915344, y = 0.405999958515167, L = \tiny {$4$}]{v6}
\Vertex[x = 0.580000102519989, y = 0.510000079870224, L = \tiny {$4$}]{v7}
\Vertex[x = 0.495999962091446, y = 0.407999932765961, L = \tiny {$4$}]{v8}
\Vertex[x = 0.416000008583069, y = 0.539999932050705, L = \tiny {$4$}]{v9}
\Vertex[x = 0.317999958992004, y = 0.541999995708466, L = \tiny {$4$}]{v10}
\Vertex[x = 0.413999974727631, y = 0.442000091075897, L = \tiny {$4$}]{v11}
\Edge[](v1)(v0)
\Edge[](v3)(v2)
\Edge[style = {color=blue, line width=3pt}](v5)(v4)
\Edge[](v7)(v6)
\Edge[style = {color=blue, line width=3pt}](v4)(v1)
\Edge[](v5)(v0)
\Edge[](v3)(v6)
\Edge[](v7)(v2)
\Edge[](v1)(v11)
\Edge[](v7)(v11)
\Edge[](v5)(v8)
\Edge[style = {color=blue, line width=3pt}](v6)(v8)
\Edge[style = {color=blue, line width=3pt}](v2)(v9)
\Edge[style = {color=blue, line width=3pt}](v4)(v9)
\Edge[style = {color=blue, line width=3pt}](v0)(v10)
\Edge[style = {color=blue, line width=3pt}](v3)(v10)
\Edge[style = {color=blue, line width=3pt}](v10)(v11)
\Edge[](v8)(v9)
\Edge[style = {color=blue, line width=3pt}](v8)(v11)
\Edge[style = {color=blue, line width=3pt}](v9)(v10)
\Edge[](v3)(v5)
\Edge[](v0)(v6)
\Edge[](v2)(v1)
\Edge[style = {color=blue, line width=3pt}](v7)(v4)
\end{tikzpicture}
\end{figure}
\bigskip
\onslide<3| handout:0>\begin{figure}[h]
\centering
\begin{tikzpicture}[scale = 10]
\tikzstyle{VertexStyle}=[shape = circle,	
								 minimum size = 12pt,
								 inner sep = 1.2pt,
                         draw]
\Vertex[style={fill=red},x = 0.697999894618988, y = 0.750000074505806, L = \tiny {}]{v0}
\Vertex[x = 0.695999979972839, y = 0.591999977827072, L = \tiny {$3$}]{v1}
\Vertex[x = 0.496000051498413, y = 0.59400001168251, L = \tiny {$3$}]{v2}
\Vertex[style={fill=red},x = 0.69599986076355, y = 0.840000033378601, L = \tiny {}]{v3}
\Vertex[x = 0.496000111103058, y = 0.526000112295151, L = \tiny {$4$}]{v4}
\Vertex[x = 0.582000017166138, y = 0.636000007390976, L = \tiny {$3$}]{v5}
\Vertex[x = 0.580000042915344, y = 0.405999958515167, L = \tiny {$3$}]{v6}
\Vertex[x = 0.580000102519989, y = 0.510000079870224, L = \tiny {$4$}]{v7}
\Vertex[x = 0.495999962091446, y = 0.407999932765961, L = \tiny {$4$}]{v8}
\Vertex[x = 0.416000008583069, y = 0.539999932050705, L = \tiny {$4$}]{v9}
\Vertex[x = 0.317999958992004, y = 0.541999995708466, L = \tiny {$3$}]{v10}
\Vertex[x = 0.413999974727631, y = 0.442000091075897, L = \tiny {$4$}]{v11}
\Edge[](v1)(v0)
\Edge[](v3)(v2)
\Edge[style = {color=blue, line width=3pt}](v5)(v4)
\Edge[](v7)(v6)
\Edge[style = {color=blue, line width=3pt}](v4)(v1)
\Edge[](v5)(v0)
\Edge[](v3)(v6)
\Edge[](v7)(v2)
\Edge[](v1)(v11)
\Edge[](v7)(v11)
\Edge[](v5)(v8)
\Edge[style = {color=blue, line width=3pt}](v6)(v8)
\Edge[style = {color=blue, line width=3pt}](v2)(v9)
\Edge[style = {color=blue, line width=3pt}](v4)(v9)
\Edge[style = {color=blue, line width=3pt}](v0)(v10)
\Edge[style = {color=blue, line width=3pt}](v3)(v10)
\Edge[style = {color=blue, line width=3pt}](v10)(v11)
\Edge[](v8)(v9)
\Edge[style = {color=blue, line width=3pt}](v8)(v11)
\Edge[style = {color=blue, line width=3pt}](v9)(v10)
\Edge[](v3)(v5)
\Edge[](v0)(v6)
\Edge[](v2)(v1)
\Edge[style = {color=blue, line width=3pt}](v7)(v4)
\end{tikzpicture}
\end{figure}
\bigskip
\onslide<4| handout:0>\begin{figure}[h]
\centering
\begin{tikzpicture}[scale = 10]
\tikzstyle{VertexStyle}=[shape = circle,	
								 minimum size = 12pt,
								 inner sep = 1.2pt,
                         draw]
\Vertex[style={fill=red},x = 0.697999894618988, y = 0.750000074505806, L = \tiny {}]{v0}
\Vertex[style={fill=green},x = 0.695999979972839, y = 0.591999977827072, L = \tiny {}]{v1}
\Vertex[x = 0.496000051498413, y = 0.59400001168251, L = \tiny {$2$}]{v2}
\Vertex[style={fill=red},x = 0.69599986076355, y = 0.840000033378601, L = \tiny {}]{v3}
\Vertex[x = 0.496000111103058, y = 0.526000112295151, L = \tiny {$3$}]{v4}
\Vertex[x = 0.582000017166138, y = 0.636000007390976, L = \tiny {$3$}]{v5}
\Vertex[x = 0.580000042915344, y = 0.405999958515167, L = \tiny {$3$}]{v6}
\Vertex[x = 0.580000102519989, y = 0.510000079870224, L = \tiny {$4$}]{v7}
\Vertex[x = 0.495999962091446, y = 0.407999932765961, L = \tiny {$4$}]{v8}
\Vertex[x = 0.416000008583069, y = 0.539999932050705, L = \tiny {$4$}]{v9}
\Vertex[x = 0.317999958992004, y = 0.541999995708466, L = \tiny {$3$}]{v10}
\Vertex[x = 0.413999974727631, y = 0.442000091075897, L = \tiny {$3$}]{v11}
\Edge[](v1)(v0)
\Edge[](v3)(v2)
\Edge[style = {color=blue, line width=3pt}](v5)(v4)
\Edge[](v7)(v6)
\Edge[style = {color=blue, line width=3pt}](v4)(v1)
\Edge[](v5)(v0)
\Edge[](v3)(v6)
\Edge[](v7)(v2)
\Edge[](v1)(v11)
\Edge[](v7)(v11)
\Edge[](v5)(v8)
\Edge[style = {color=blue, line width=3pt}](v6)(v8)
\Edge[style = {color=blue, line width=3pt}](v2)(v9)
\Edge[style = {color=blue, line width=3pt}](v4)(v9)
\Edge[style = {color=blue, line width=3pt}](v0)(v10)
\Edge[style = {color=blue, line width=3pt}](v3)(v10)
\Edge[style = {color=blue, line width=3pt}](v10)(v11)
\Edge[](v8)(v9)
\Edge[style = {color=blue, line width=3pt}](v8)(v11)
\Edge[style = {color=blue, line width=3pt}](v9)(v10)
\Edge[](v3)(v5)
\Edge[](v0)(v6)
\Edge[](v2)(v1)
\Edge[style = {color=blue, line width=3pt}](v7)(v4)
\end{tikzpicture}
\end{figure}
\bigskip
\onslide<5| handout:0>\begin{figure}[h]
\centering
\begin{tikzpicture}[scale = 10]
\tikzstyle{VertexStyle}=[shape = circle,	
								 minimum size = 12pt,
								 inner sep = 1.2pt,
                         draw]
\Vertex[style={fill=red},x = 0.697999894618988, y = 0.750000074505806, L = \tiny {}]{v0}
\Vertex[style={fill=green},x = 0.695999979972839, y = 0.591999977827072, L = \tiny {}]{v1}
\Vertex[x = 0.496000051498413, y = 0.59400001168251, L = \tiny {$2$}]{v2}
\Vertex[style={fill=red},x = 0.69599986076355, y = 0.840000033378601, L = \tiny {}]{v3}
\Vertex[x = 0.496000111103058, y = 0.526000112295151, L = \tiny {$2$}]{v4}
\Vertex[style={fill=green},x = 0.582000017166138, y = 0.636000007390976, L = \tiny {}]{v5}
\Vertex[style={fill=green},x = 0.580000042915344, y = 0.405999958515167, L = \tiny {}]{v6}
\Vertex[style={fill=red},x = 0.580000102519989, y = 0.510000079870224, L = \tiny {}]{v7}
\Vertex[x = 0.495999962091446, y = 0.407999932765961, L = \tiny {$3$}]{v8}
\Vertex[x = 0.416000008583069, y = 0.539999932050705, L = \tiny {$4$}]{v9}
\Vertex[x = 0.317999958992004, y = 0.541999995708466, L = \tiny {$3$}]{v10}
\Vertex[x = 0.413999974727631, y = 0.442000091075897, L = \tiny {$2$}]{v11}
\Edge[](v1)(v0)
\Edge[](v3)(v2)
\Edge[style = {color=blue, line width=3pt}](v5)(v4)
\Edge[](v7)(v6)
\Edge[style = {color=blue, line width=3pt}](v4)(v1)
\Edge[](v5)(v0)
\Edge[](v3)(v6)
\Edge[](v7)(v2)
\Edge[](v1)(v11)
\Edge[](v7)(v11)
\Edge[](v5)(v8)
\Edge[style = {color=blue, line width=3pt}](v6)(v8)
\Edge[style = {color=blue, line width=3pt}](v2)(v9)
\Edge[style = {color=blue, line width=3pt}](v4)(v9)
\Edge[style = {color=blue, line width=3pt}](v0)(v10)
\Edge[style = {color=blue, line width=3pt}](v3)(v10)
\Edge[style = {color=blue, line width=3pt}](v10)(v11)
\Edge[](v8)(v9)
\Edge[style = {color=blue, line width=3pt}](v8)(v11)
\Edge[style = {color=blue, line width=3pt}](v9)(v10)
\Edge[](v3)(v5)
\Edge[](v0)(v6)
\Edge[](v2)(v1)
\Edge[style = {color=blue, line width=3pt}](v7)(v4)
\end{tikzpicture}
\end{figure}
\bigskip
\onslide<6| handout:0>\begin{figure}[h]
\centering
\begin{tikzpicture}[scale = 10]
\tikzstyle{VertexStyle}=[shape = circle,	
								 minimum size = 12pt,
								 inner sep = 1.2pt,
                         draw]
\Vertex[style={fill=red},x = 0.697999894618988, y = 0.750000074505806, L = \tiny {}]{v0}
\Vertex[style={fill=green},x = 0.695999979972839, y = 0.591999977827072, L = \tiny {}]{v1}
\Vertex[style={fill=blue}, x = 0.496000051498413, y = 0.59400001168251, L = \tiny {}]{v2}
\Vertex[style={fill=red},x = 0.69599986076355, y = 0.840000033378601, L = \tiny {}]{v3}
\Vertex[style={fill=blue}, x = 0.496000111103058, y = 0.526000112295151, L = \tiny {}]{v4}
\Vertex[style={fill=green},x = 0.582000017166138, y = 0.636000007390976, L = \tiny {}]{v5}
\Vertex[style={fill=green},x = 0.580000042915344, y = 0.405999958515167, L = \tiny {}]{v6}
\Vertex[style={fill=red},x = 0.580000102519989, y = 0.510000079870224, L = \tiny {}]{v7}
\Vertex[style={fill=red},x = 0.495999962091446, y = 0.407999932765961, L = \tiny {}]{v8}
\Vertex[x = 0.416000008583069, y = 0.539999932050705, L = \tiny {$2$}]{v9}
\Vertex[x = 0.317999958992004, y = 0.541999995708466, L = \tiny {$3$}]{v10}
\Vertex[x = 0.413999974727631, y = 0.442000091075897, L = \tiny {$2$}]{v11}
\Edge[](v1)(v0)
\Edge[](v3)(v2)
\Edge[style = {color=blue, line width=3pt}](v5)(v4)
\Edge[](v7)(v6)
\Edge[style = {color=blue, line width=3pt}](v4)(v1)
\Edge[](v5)(v0)
\Edge[](v3)(v6)
\Edge[](v7)(v2)
\Edge[](v1)(v11)
\Edge[](v7)(v11)
\Edge[](v5)(v8)
\Edge[style = {color=blue, line width=3pt}](v6)(v8)
\Edge[style = {color=blue, line width=3pt}](v2)(v9)
\Edge[style = {color=blue, line width=3pt}](v4)(v9)
\Edge[style = {color=blue, line width=3pt}](v0)(v10)
\Edge[style = {color=blue, line width=3pt}](v3)(v10)
\Edge[style = {color=blue, line width=3pt}](v10)(v11)
\Edge[](v8)(v9)
\Edge[style = {color=blue, line width=3pt}](v8)(v11)
\Edge[style = {color=blue, line width=3pt}](v9)(v10)
\Edge[](v3)(v5)
\Edge[](v0)(v6)
\Edge[](v2)(v1)
\Edge[style = {color=blue, line width=3pt}](v7)(v4)
\end{tikzpicture}
\end{figure}
\bigskip
\onslide<7| handout:0>\begin{figure}[h]
\centering
\begin{tikzpicture}[scale = 10]
\tikzstyle{VertexStyle}=[shape = circle,	
								 minimum size = 12pt,
								 inner sep = 1.2pt,
                         draw]
\Vertex[style={fill=red},x = 0.697999894618988, y = 0.750000074505806, L = \tiny {}]{v0}
\Vertex[style={fill=green},x = 0.695999979972839, y = 0.591999977827072, L = \tiny {}]{v1}
\Vertex[style={fill=blue}, x = 0.496000051498413, y = 0.59400001168251, L = \tiny {}]{v2}
\Vertex[style={fill=red},x = 0.69599986076355, y = 0.840000033378601, L = \tiny {}]{v3}
\Vertex[style={fill=blue}, x = 0.496000111103058, y = 0.526000112295151, L = \tiny {}]{v4}
\Vertex[style={fill=green},x = 0.582000017166138, y = 0.636000007390976, L = \tiny {}]{v5}
\Vertex[style={fill=green},x = 0.580000042915344, y = 0.405999958515167, L = \tiny {}]{v6}
\Vertex[style={fill=red},x = 0.580000102519989, y = 0.510000079870224, L = \tiny {}]{v7}
\Vertex[style={fill=red},x = 0.495999962091446, y = 0.407999932765961, L = \tiny {}]{v8}
\Vertex[style={fill=green},x = 0.416000008583069, y = 0.539999932050705, L = \tiny {}]{v9}
\Vertex[x = 0.317999958992004, y = 0.541999995708466, L = \tiny {$1$}]{v10}
\Vertex[style={fill=blue}, x = 0.413999974727631, y = 0.442000091075897, L = \tiny {}]{v11}
\Edge[](v1)(v0)
\Edge[](v3)(v2)
\Edge[style = {color=blue, line width=3pt}](v5)(v4)
\Edge[](v7)(v6)
\Edge[style = {color=blue, line width=3pt}](v4)(v1)
\Edge[](v5)(v0)
\Edge[](v3)(v6)
\Edge[](v7)(v2)
\Edge[](v1)(v11)
\Edge[](v7)(v11)
\Edge[](v5)(v8)
\Edge[style = {color=blue, line width=3pt}](v6)(v8)
\Edge[style = {color=blue, line width=3pt}](v2)(v9)
\Edge[style = {color=blue, line width=3pt}](v4)(v9)
\Edge[style = {color=blue, line width=3pt}](v0)(v10)
\Edge[style = {color=blue, line width=3pt}](v3)(v10)
\Edge[style = {color=blue, line width=3pt}](v10)(v11)
\Edge[](v8)(v9)
\Edge[style = {color=blue, line width=3pt}](v8)(v11)
\Edge[style = {color=blue, line width=3pt}](v9)(v10)
\Edge[](v3)(v5)
\Edge[](v0)(v6)
\Edge[](v2)(v1)
\Edge[style = {color=blue, line width=3pt}](v7)(v4)
\end{tikzpicture}
\end{figure}
\bigskip
\onslide<8| handout:0>\begin{figure}[h]
\centering
\begin{tikzpicture}[scale = 10]
\tikzstyle{VertexStyle}=[shape = circle,	
								 minimum size = 12pt,
								 inner sep = 1.2pt,
                         draw]
\Vertex[style={fill=red},x = 0.697999894618988, y = 0.750000074505806, L = \tiny {}]{v0}
\Vertex[style={fill=green},x = 0.695999979972839, y = 0.591999977827072, L = \tiny {}]{v1}
\Vertex[style={fill=blue}, x = 0.496000051498413, y = 0.59400001168251, L = \tiny {}]{v2}
\Vertex[style={fill=red},x = 0.69599986076355, y = 0.840000033378601, L = \tiny {}]{v3}
\Vertex[style={fill=blue}, x = 0.496000111103058, y = 0.526000112295151, L = \tiny {}]{v4}
\Vertex[style={fill=green},x = 0.582000017166138, y = 0.636000007390976, L = \tiny {}]{v5}
\Vertex[style={fill=green},x = 0.580000042915344, y = 0.405999958515167, L = \tiny {}]{v6}
\Vertex[style={fill=red},x = 0.580000102519989, y = 0.510000079870224, L = \tiny {}]{v7}
\Vertex[style={fill=red},x = 0.495999962091446, y = 0.407999932765961, L = \tiny {}]{v8}
\Vertex[style={fill=green},x = 0.416000008583069, y = 0.539999932050705, L = \tiny {}]{v9}
\Vertex[style={fill=cyan}, x = 0.317999958992004, y = 0.541999995708466, L = \tiny {}]{v10}
\Vertex[style={fill=blue}, x = 0.413999974727631, y = 0.442000091075897, L = \tiny {}]{v11}
\Edge[](v1)(v0)
\Edge[](v3)(v2)
\Edge[style = {color=blue, line width=3pt}](v5)(v4)
\Edge[](v7)(v6)
\Edge[style = {color=blue, line width=3pt}](v4)(v1)
\Edge[](v5)(v0)
\Edge[](v3)(v6)
\Edge[](v7)(v2)
\Edge[](v1)(v11)
\Edge[](v7)(v11)
\Edge[](v5)(v8)
\Edge[style = {color=blue, line width=3pt}](v6)(v8)
\Edge[style = {color=blue, line width=3pt}](v2)(v9)
\Edge[style = {color=blue, line width=3pt}](v4)(v9)
\Edge[style = {color=blue, line width=3pt}](v0)(v10)
\Edge[style = {color=blue, line width=3pt}](v3)(v10)
\Edge[style = {color=blue, line width=3pt}](v10)(v11)
\Edge[](v8)(v9)
\Edge[style = {color=blue, line width=3pt}](v8)(v11)
\Edge[style = {color=blue, line width=3pt}](v9)(v10)
\Edge[](v3)(v5)
\Edge[](v0)(v6)
\Edge[](v2)(v1)
\Edge[style = {color=blue, line width=3pt}](v7)(v4)
\end{tikzpicture}
\end{figure}
\bigskip
\end{overprint}
\uncover<2->{Greedily coloring in leaf first order proves Brooks' theorem}
\end{frame}


\section{The Ore-degree}
\begin{frame}{The Ore-degree}
\uncover<1->{
\begin{defn}
The \alert{\emph{Ore-degree}} of an edge $xy$ in a graph $G$ is 

\[\theta(xy) \DefinedAs d(x) + d(y).\]

The \alert{\emph{Ore-degree}} of a graph $G$ is 
\[\theta(G) \DefinedAs \max_{xy \in E(G)}\theta(xy).\]
\end{defn}}

\begin{itemize}
\2 every graph satisfies $\floor{\frac{\theta}{2}} \leq \Delta$
\3 greedy coloring (in any order) shows that every graph satisfies 
$\chi \leq \left\lfloor\frac{\theta}{2} \right\rfloor + 1$
\end{itemize}
\end{frame}

\begin{frame}{Kierstead and Kostochka's generalization}
\uncover<1->{
\begin{OreBrooksKK}[Kierstead and Kostochka 2009]
Every graph with $\theta \geq 12$ satisfies $\chi \leq \max \left\{\omega, \left\lfloor\frac{\theta}{2} \right \rfloor\right\}$.
\end{OreBrooksKK}}

\uncover<2->{
Kierstead and Kostochka conjectured that the $12$ could be reduced to $10$.  That this would be best possible can be seen from the following example which has $\theta = 9$, $\omega = 4$ and $\chi = 5$.}

\uncover<3->{
\begin{figure}[h]
\centering
\begin{tikzpicture}[scale = 5]
\tikzstyle{VertexStyle}=[shape = circle,	
								 minimum size = 1pt,
								 inner sep = 1.2pt,
                         draw]
\Vertex[x = 0.270266681909561, y = 0.890800006687641, L = \tiny {4}]{v0}
\Vertex[x = 0.336266696453094, y = 0.962799992412329, L = \tiny {4}]{v1}
\Vertex[x = 0.334666579961777, y = 0.821199983358383, L = \tiny {4}]{v2}
\Vertex[x = 0.56306654214859, y = 0.890800006687641, L = \tiny {5}]{v3}
\Vertex[x = 0.244666695594788, y = 0.731600046157837, L = \tiny {4}]{v4}
\Vertex[x = 0.417866677045822, y = 0.732000052928925, L = \tiny {4}]{v5}
\Vertex[x = 0.243866696953773, y = 0.543200016021729, L = \tiny {4}]{v6}
\Vertex[x = 0.415866762399673, y = 0.542800068855286, L = \tiny {4}]{v7}
\Vertex[x = 0.0926666706800461, y = 0.890000000596046, L = \tiny {5}]{v8}
\tikzstyle{EdgeStyle}=[]
\Edge[](v1)(v0)
\tikzstyle{EdgeStyle}=[]
\Edge[](v2)(v0)
\tikzstyle{EdgeStyle}=[]
\Edge[](v3)(v0)
\tikzstyle{EdgeStyle}=[]
\Edge[](v2)(v1)
\tikzstyle{EdgeStyle}=[]
\Edge[](v3)(v1)
\tikzstyle{EdgeStyle}=[]
\Edge[](v2)(v3)
\tikzstyle{EdgeStyle}=[]
\Edge[](v5)(v4)
\tikzstyle{EdgeStyle}=[]
\Edge[](v6)(v4)
\tikzstyle{EdgeStyle}=[]
\Edge[](v6)(v5)
\tikzstyle{EdgeStyle}=[]
\Edge[](v6)(v7)
\tikzstyle{EdgeStyle}=[]
\Edge[](v7)(v4)
\tikzstyle{EdgeStyle}=[]
\Edge[](v7)(v5)
\tikzstyle{EdgeStyle}=[]
\Edge[](v4)(v8)
\tikzstyle{EdgeStyle}=[]
\Edge[](v6)(v8)
\tikzstyle{EdgeStyle}=[]
\Edge[](v5)(v3)
\tikzstyle{EdgeStyle}=[]
\Edge[](v7)(v3)
\tikzstyle{EdgeStyle}=[]
\Edge[](v0)(v8)
\tikzstyle{EdgeStyle}=[]
\Edge[](v1)(v8)
\tikzstyle{EdgeStyle}=[]
\Edge[](v2)(v8)
\end{tikzpicture}

\caption{$O_5$, a counterexample with $\theta = 9$.}
\end{figure}}
\end{frame}

\section{Rephrasing the problem}
\begin{frame}{Rephrasing the problem}
\uncover<1->{
\begin{defn}
A graph $G$ is called \alert{\emph{vertex critical}} if $\chi(G-v) < \chi(G)$ for each $v \in V(G)$.
\end{defn}}
\uncover<2->{
\begin{defn}
Let $G$ be a vertex critical graph.  The \alert{\emph{low vertex subgraph}} $\mathcal{L}(G)$ is the graph induced on the vertices of degree $\chi(G) - 1$.  The \alert{\emph{high vertex subgraph}} $\mathcal{H}(G)$ is the graph induced on the vertices of degree at least $\chi(G)$.
\end{defn}}

\uncover<3->{	
\begin{problem}
Prove that $K_{\Delta(G) + 1}$ is the only vertex critical graph $G$ with $\chi(G) \geq \Delta(G) \geq 6$ such that $\mathcal{H}(G)$ is edgeless.
\end{problem}}
\end{frame}

\section{Solving the rephrased problem}
\subsection{Kierstead and Kostochka's proof}
\begin{frame}{Kierstead and Kostochka's proof}
\begin{itemize}
\1 the proof is high-tech and clean, it uses both of the following \\
\2 Alon and Tarsi's algebraic list coloring theorem \\
\3 a result of Stiebitz from 1982 proving a conjecture of Gallai stating that $\mathcal{H}(G)$ has at most as many components as $\mathcal{L}(G)$ \\
\4 using these it is basically just a counting argument \\
\5 unfortunately, it only works for $\Delta \geq 7$
\end{itemize}
\end{frame}
\subsection{Problem solved}
\begin{frame}
\uncover<1->{
To get down to $\Delta = 6$, go low-tech and get dirty.}

\bigskip
\uncover<2->{
\begin{OreBrooks}[Rabern 2010]
$K_{\Delta(G) + 1}$ is the only vertex critical graph $G$ with $\chi(G) \geq \Delta(G) \geq 6$ and $\omega(\mathcal{H}(G)) \leq \left \lfloor \frac{\Delta(G)}{2} \right \rfloor - 2$.
\end{OreBrooks}}

\begin{itemize}
\3 setting $\omega(\mathcal{H}(G)) = 1$ proves Kierstead and Kostochka's conjecture \\
\4 equivalently, as long as there is no group of six inmates who have all fought one another, you (the warden) can complete your inmate-cell-assignment task
\end{itemize}
\end{frame}
\subsection{Proof outline}
\begin{frame}{Proof outline}
\begin{itemize}
\1 start with a minimal counterexample $G$ \\
\2 for any induced subgraph $H$, $\Delta - 1$ coloring $G-H$ leaves a list assignment $L$ on $H$ where $\card{L(v)} \geq \deg(v) - 1$
\end{itemize}

\uncover<3->{
\begin{goal}
Construct a subgraph $H$ for which such a list assignment can always be completed. 
\end{goal}}

\begin{itemize}
\4 we need $H$ to have large degrees to get large lists, so $H$ will be ``dense'' \\
\5 first, use minimality of $G$ to exclude some troublesome $H$'s \\
\6 run the following recoloring algorithm to construct $H$
\end{itemize}
\end{frame}
\subsection{Mozhan's lemma}
\begin{frame}{Partitioned colorings}
\begin{defn}
Let $G$ be a vertex critical graph.  Let $a \geq 1$ and $r_1, \ldots, r_a$ be such that $1 + \sum_i r_i = \chi(G)$. By a \alert{\emph{$(r_1, \ldots, r_a)$-partitioned coloring}} of $G$ we mean a proper coloring of $G$ of the form
\[\left\{\{x\}, L_{11}, L_{12}, \ldots, L_{1r_1}, L_{21}, L_{22}, \ldots, L_{2r_2}, \ldots, L_{a1}, L_{a2}, \ldots, L_{ar_a}\right\}.\]

Here $\{x\}$ is a singleton color class and each $L_{ij}$ is a color class.
\end{defn}
\end{frame}

\begin{frame}{Mozhan's Lemma}
\begin{MozhansLemma}[Mozhan 1983]
Let $G$ be a vertex critical graph.   Let $a \geq 1$ and $r_1, \ldots, r_a$ be such that $1 + \sum_i r_i = \chi(G)$. Of all $(r_1, \ldots, r_a)$-partitioned colorings of $G$ pick one minimizing

\[\sum_{i = 1}^a \size{G\left[\bigcup_{j = 1}^{r_i} L_{ij}\right]}.\]

Remember that $\{x\}$ is a singleton color class in the coloring. Put $U_i \DefinedAs \bigcup_{j = 1}^{r_i} L_{ij}$ and let $Z_i(x)$ be the component of $x$ in $G[\{x\} \cup U_i]$.  If $d_{Z_i(x)}(x) = r_i$, then $Z_i(x)$ is complete if $r_i \geq 3$ and $Z_i(x)$ is an odd cycle if $r_i = 2$.
\end{MozhansLemma}
\end{frame}

\subsection{The recoloring algorithm}
\begin{frame}{The recoloring algorithm}
\begin{itemize}
\1 take a  $(\left \lfloor \frac{\Delta - 1}{2} \right \rfloor, \left \lceil \frac{\Delta - 1}{2} \right \rceil)$-partitioned coloring minimizing the above function \\
\2 prove that we may assume that $x$ is a low vertex \\
\3 by Mozhan's lemma, the neighborhood of $x$ in each part induces a clique or an odd cycle
\end{itemize}
\end{frame}
\begin{frame}[t]{The recoloring algorithm}
\begin{overprint}
\onslide<1| handout:0>\begin{figure}[h]
\centering
\begin{tikzpicture}[scale = 10]
\tikzstyle{VertexStyle}=[shape = circle,	
								 minimum size = 12pt,
								 inner sep = 1.2pt,
                         draw]
\Vertex[style = {fill=magenta}, x = 0.5, y = 0.7, L = \tiny {}]{v0}
\Vertex[style = {fill=cyan}, x = 0.45, y = 0.6, L = \tiny {}]{v1}
\Vertex[style = {fill=yellow}, x = 0.55, y = 0.6, L = \tiny {$x_1$}]{v2}
\Vertex[style = {fill=green}, x = 0.35, y = 0.7, L = \tiny {}]{v3}
\Vertex[style = {fill=red}, x = 0.3, y = 0.6, L = \tiny {}]{v4}
\Vertex[style = {fill=blue}, x = 0.4, y = 0.6, L = \tiny {}]{v5}
\Vertex[style = {fill=magenta}, x = 0.5, y = 0.85, L = \tiny {}]{v6}
\Vertex[style = {fill=cyan}, x = 0.45, y = 0.75, L = \tiny {}]{v7}
\Vertex[style = {fill=yellow}, x = 0.55, y = 0.75, L = \tiny {}]{v8}
\Vertex[x = 0.425, y = 0.4, L = \tiny {$x$}]{v9}
\Vertex[style = {fill=green}, x = 0.35, y = 0.85, L = \tiny {}]{v10}
\Vertex[style = {fill=red}, x = 0.3, y = 0.75, L = \tiny {}]{v11}
\Vertex[style = {fill=blue}, x = 0.4, y = 0.75, L = \tiny {}]{v12}
\Edge[](v2)(v0)
\Edge[](v2)(v1)
\Edge[](v1)(v0)
\Edge[](v4)(v3)
\Edge[](v5)(v3)
\Edge[](v5)(v4)
\Edge[](v7)(v6)
\Edge[](v8)(v6)
\Edge[](v8)(v7)
\Edge[](v11)(v10)
\Edge[](v12)(v10)
\Edge[](v12)(v11)
\Edge[](v9)(v4)
\Edge[](v9)(v5)
\Edge[](v9)(v3)
\Edge[](v9)(v1)
\Edge[](v9)(v2)
\Edge[](v9)(v0)
\end{tikzpicture}
\end{figure}
\bigskip
\onslide<2| handout:0>\begin{figure}[h]
\centering
\begin{tikzpicture}[scale = 10]
\tikzstyle{VertexStyle}=[shape = circle,	
								 minimum size = 12pt,
								 inner sep = 1.2pt,
                         draw]
\Vertex[style = {fill=magenta}, x = 0.5, y = 0.7, L = \tiny {}]{v0}
\Vertex[style = {fill=cyan}, x = 0.45, y = 0.6, L = \tiny {}]{v1}
\Vertex[style = {fill=yellow}, x = 0.55, y = 0.6, L = \tiny {$x$}]{v2}
\Vertex[style = {fill=green}, x = 0.35, y = 0.7, L = \tiny {}]{v3}
\Vertex[style = {fill=red}, x = 0.3, y = 0.6, L = \tiny {}]{v4}
\Vertex[style = {fill=blue}, x = 0.4, y = 0.6, L = \tiny {}]{v5}
\Vertex[style = {fill=magenta}, x = 0.5, y = 0.85, L = \tiny {}]{v6}
\Vertex[style = {fill=cyan}, x = 0.45, y = 0.75, L = \tiny {}]{v7}
\Vertex[style = {fill=yellow}, x = 0.55, y = 0.75, L = \tiny {}]{v8}
\Vertex[x = 0.425, y = 0.4, L = \tiny {$x_1$}]{v9}
\Vertex[style = {fill=green}, x = 0.35, y = 0.85, L = \tiny {}]{v10}
\Vertex[style = {fill=red}, x = 0.3, y = 0.75, L = \tiny {$x_2$}]{v11}
\Vertex[style = {fill=blue}, x = 0.4, y = 0.75, L = \tiny {}]{v12}
\Edge[](v2)(v0)
\Edge[](v2)(v1)
\Edge[](v1)(v0)
\Edge[](v4)(v3)
\Edge[](v5)(v3)
\Edge[](v5)(v4)
\Edge[](v7)(v6)
\Edge[](v8)(v6)
\Edge[](v8)(v7)
\Edge[](v11)(v10)
\Edge[](v12)(v10)
\Edge[](v12)(v11)
\Edge[](v9)(v10)
\Edge[](v9)(v11)
\Edge[](v9)(v12)
\Edge[](v9)(v1)
\Edge[](v9)(v2)
\Edge[](v9)(v0)

\Edge[](v2)(v4)
\Edge[](v2)(v5)
\Edge[](v2)(v3)
\end{tikzpicture}
\end{figure}
\bigskip
\onslide<3| handout:0>\begin{figure}[h]
\centering
\begin{tikzpicture}[scale = 10]
\tikzstyle{VertexStyle}=[shape = circle,	
								 minimum size = 12pt,
								 inner sep = 1.2pt,
                         draw]
\Vertex[style = {fill=magenta}, x = 0.5, y = 0.7, L = \tiny {}]{v0}
\Vertex[style = {fill=cyan}, x = 0.45, y = 0.6, L = \tiny {}]{v1}
\Vertex[style = {fill=yellow}, x = 0.55, y = 0.6, L = \tiny {$x$}]{v2}
\Vertex[style = {fill=green}, x = 0.35, y = 0.7, L = \tiny {}]{v3}
\Vertex[style = {fill=red}, x = 0.3, y = 0.6, L = \tiny {}]{v4}
\Vertex[style = {fill=blue}, x = 0.4, y = 0.6, L = \tiny {}]{v5}
\Vertex[style = {fill=magenta}, x = 0.5, y = 0.85, L = \tiny {}]{v6}
\Vertex[style = {fill=cyan}, x = 0.45, y = 0.75, L = \tiny {}]{v7}
\Vertex[style = {fill=yellow}, x = 0.55, y = 0.75, L = \tiny {$x_3$}]{v8}
\Vertex[x = 0.425, y = 0.4, L = \tiny {$x_2$}]{v9}
\Vertex[style = {fill=green}, x = 0.35, y = 0.85, L = \tiny {}]{v10}
\Vertex[style = {fill=red}, x = 0.3, y = 0.75, L = \tiny {$x_1$}]{v11}
\Vertex[style = {fill=blue}, x = 0.4, y = 0.75, L = \tiny {}]{v12}
\Edge[](v2)(v0)
\Edge[](v2)(v1)
\Edge[](v1)(v0)
\Edge[](v4)(v3)
\Edge[](v5)(v3)
\Edge[](v5)(v4)
\Edge[](v7)(v6)
\Edge[](v8)(v6)
\Edge[](v8)(v7)
\Edge[](v11)(v10)
\Edge[](v12)(v10)
\Edge[](v12)(v11)
\Edge[](v9)(v10)
\Edge[](v9)(v11)
\Edge[](v9)(v12)
\Edge[](v9)(v6)
\Edge[](v9)(v7)
\Edge[](v9)(v8)

\Edge[](v2)(v4)
\Edge[](v2)(v5)
\Edge[](v2)(v3)
\end{tikzpicture}
\end{figure}
\bigskip
\onslide<4| handout:0>\begin{figure}[h]
\centering
\begin{tikzpicture}[scale = 10]
\tikzstyle{VertexStyle}=[shape = circle,	
								 minimum size = 12pt,
								 inner sep = 1.2pt,
                         draw]
\Vertex[style = {fill=magenta}, x = 0.5, y = 0.7, L = \tiny {}]{v0}
\Vertex[style = {fill=cyan}, x = 0.45, y = 0.6, L = \tiny {}]{v1}
\Vertex[style = {fill=yellow}, x = 0.55, y = 0.6, L = \tiny {$x$}]{v2}
\Vertex[style = {fill=green}, x = 0.35, y = 0.7, L = \tiny {}]{v3}
\Vertex[style = {fill=red}, x = 0.3, y = 0.6, L = \tiny {$x_4$}]{v4}
\Vertex[style = {fill=blue}, x = 0.4, y = 0.6, L = \tiny {}]{v5}
\Vertex[style = {fill=magenta}, x = 0.5, y = 0.85, L = \tiny {}]{v6}
\Vertex[style = {fill=cyan}, x = 0.45, y = 0.75, L = \tiny {}]{v7}
\Vertex[style = {fill=yellow}, x = 0.55, y = 0.75, L = \tiny {$x_2$}]{v8}
\Vertex[x = 0.425, y = 0.4, L = \tiny {$x_3$}]{v9}
\Vertex[style = {fill=green}, x = 0.35, y = 0.85, L = \tiny {}]{v10}
\Vertex[style = {fill=red}, x = 0.3, y = 0.75, L = \tiny {$x_1$}]{v11}
\Vertex[style = {fill=blue}, x = 0.4, y = 0.75, L = \tiny {}]{v12}
\Edge[](v2)(v0)
\Edge[](v2)(v1)
\Edge[](v1)(v0)
\Edge[](v4)(v3)
\Edge[](v5)(v3)
\Edge[](v5)(v4)
\Edge[](v7)(v6)
\Edge[](v8)(v6)
\Edge[](v8)(v7)
\Edge[](v11)(v10)
\Edge[](v12)(v10)
\Edge[](v12)(v11)
\Edge[](v9)(v3)
\Edge[](v9)(v4)
\Edge[](v9)(v5)
\Edge[](v9)(v6)
\Edge[](v9)(v7)
\Edge[](v9)(v8)

\Edge[](v2)(v5)
\Edge[](v2)(v3)
\Edge[](v2)(v4)
\end{tikzpicture}
\end{figure}
\bigskip
\onslide<5| handout:0>\begin{figure}[h]
\centering
\begin{tikzpicture}[scale = 10]
\tikzstyle{VertexStyle}=[shape = circle,	
								 minimum size = 12pt,
								 inner sep = 1.2pt,
                         draw]
\Vertex[style = {fill=magenta}, x = 0.5, y = 0.7, L = \tiny {}]{v0}
\Vertex[style = {fill=cyan}, x = 0.45, y = 0.6, L = \tiny {}]{v1}
\Vertex[style = {fill=yellow}, x = 0.55, y = 0.6, L = \tiny {$x$}]{v2}
\Vertex[style = {fill=green}, x = 0.35, y = 0.7, L = \tiny {}]{v3}
\Vertex[style = {fill=red}, x = 0.3, y = 0.6, L = \tiny {$x_3$}]{v4}
\Vertex[style = {fill=blue}, x = 0.4, y = 0.6, L = \tiny {}]{v5}
\Vertex[style = {fill=magenta}, x = 0.5, y = 0.85, L = \tiny {}]{v6}
\Vertex[style = {fill=cyan}, x = 0.45, y = 0.75, L = \tiny {}]{v7}
\Vertex[style = {fill=yellow}, x = 0.55, y = 0.75, L = \tiny {$x_2$}]{v8}
\Vertex[x = 0.425, y = 0.4, L = \tiny {$x_4$}]{v9}
\Vertex[style = {fill=green}, x = 0.35, y = 0.85, L = \tiny {}]{v10}
\Vertex[style = {fill=red}, x = 0.3, y = 0.75, L = \tiny {$x_1$}]{v11}
\Vertex[style = {fill=blue}, x = 0.4, y = 0.75, L = \tiny {}]{v12}
\Edge[](v2)(v0)
\Edge[](v2)(v1)
\Edge[](v1)(v0)
\Edge[](v4)(v3)
\Edge[](v5)(v3)
\Edge[](v5)(v4)
\Edge[](v7)(v6)
\Edge[](v8)(v6)
\Edge[](v8)(v7)
\Edge[](v11)(v10)
\Edge[](v12)(v10)
\Edge[](v12)(v11)
\Edge[](v9)(v3)
\Edge[](v9)(v4)
\Edge[](v9)(v5)
\Edge[](v9)(v0)
\Edge[](v9)(v1)
\Edge[](v9)(v2)

\Edge[](v2)(v5)
\Edge[](v2)(v3)
\end{tikzpicture}
\end{figure}
\bigskip
\onslide<6| handout:1>\begin{figure}[h]
\centering
\begin{tikzpicture}[scale = 10]
\tikzstyle{VertexStyle}=[shape = circle,	
								 minimum size = 12pt,
								 inner sep = 1.2pt,
                         draw]
\Vertex[style = {fill=magenta}, x = 0.5, y = 0.7, L = \tiny {}]{v0}
\Vertex[style = {fill=cyan}, x = 0.45, y = 0.6, L = \tiny {}]{v1}
\Vertex[style = {fill=yellow}, x = 0.55, y = 0.6, L = \tiny {$x_4$}]{v2}
\Vertex[style = {fill=green}, x = 0.35, y = 0.7, L = \tiny {}]{v3}
\Vertex[style = {fill=red}, x = 0.3, y = 0.6, L = \tiny {$x_3$}]{v4}
\Vertex[style = {fill=blue}, x = 0.4, y = 0.6, L = \tiny {}]{v5}
\Vertex[style = {fill=magenta}, x = 0.5, y = 0.85, L = \tiny {}]{v6}
\Vertex[style = {fill=cyan}, x = 0.45, y = 0.75, L = \tiny {}]{v7}
\Vertex[style = {fill=yellow}, x = 0.55, y = 0.75, L = \tiny {$x_2$}]{v8}
\Vertex[x = 0.425, y = 0.4, L = \tiny {$x$}]{v9}
\Vertex[style = {fill=green}, x = 0.35, y = 0.85, L = \tiny {}]{v10}
\Vertex[style = {fill=red}, x = 0.3, y = 0.75, L = \tiny {$x_1$}]{v11}
\Vertex[style = {fill=blue}, x = 0.4, y = 0.75, L = \tiny {}]{v12}
\Edge[](v2)(v0)
\Edge[](v2)(v1)
\Edge[](v1)(v0)
\Edge[](v4)(v3)
\Edge[](v5)(v3)
\Edge[](v5)(v4)
\Edge[](v7)(v6)
\Edge[](v8)(v6)
\Edge[](v8)(v7)
\Edge[](v11)(v10)
\Edge[](v12)(v10)
\Edge[](v12)(v11)
\Edge[](v9)(v3)
\Edge[](v9)(v4)
\Edge[](v9)(v5)
\Edge[](v9)(v0)
\Edge[](v9)(v1)
\Edge[](v9)(v2)

\Edge[](v2)(v5)
\Edge[](v2)(v3)
\Edge[](v2)(v4)
\end{tikzpicture}
\end{figure}
\bigskip
\end{overprint}
\begin{itemize}
\1 swap $x$ with a low vertex $x_1$ in the right part \\
\2 swap $x_1$ with a low vertex $x_2$ in the left part \\
\3 continue swapping back and forth until you wrap around \\
\end{itemize}
\end{frame}
\begin{frame}[t]{The recoloring algorithm}
\begin{overprint}
\onslide<1| handout:0>\begin{figure}[h]
\centering
\begin{tikzpicture}[scale = 10]
\tikzstyle{VertexStyle}=[shape = circle,	
								 minimum size = 12pt,
								 inner sep = 1.2pt,
                         draw]
\Vertex[style = {fill=magenta}, x = 0.5, y = 0.7, L = \tiny {}]{v0}
\Vertex[style = {fill=cyan}, x = 0.45, y = 0.6, L = \tiny {}]{v1}
\Vertex[style = {fill=yellow}, x = 0.55, y = 0.6, L = \tiny {$x_4$}]{v2}
\Vertex[style = {fill=green}, x = 0.35, y = 0.7, L = \tiny {}]{v3}
\Vertex[style = {fill=red}, x = 0.3, y = 0.6, L = \tiny {$x_3$}]{v4}
\Vertex[style = {fill=blue}, x = 0.4, y = 0.6, L = \tiny {}]{v5}
\Vertex[style = {fill=magenta}, x = 0.5, y = 0.85, L = \tiny {}]{v6}
\Vertex[style = {fill=cyan}, x = 0.45, y = 0.75, L = \tiny {}]{v7}
\Vertex[style = {fill=yellow}, x = 0.55, y = 0.75, L = \tiny {$x_2$}]{v8}
\Vertex[x = 0.425, y = 0.4, L = \tiny {$x$}]{v9}
\Vertex[style = {fill=green}, x = 0.35, y = 0.85, L = \tiny {}]{v10}
\Vertex[style = {fill=red}, x = 0.3, y = 0.75, L = \tiny {$x_1$}]{v11}
\Vertex[style = {fill=blue}, x = 0.4, y = 0.75, L = \tiny {}]{v12}
\Edge[](v2)(v0)
\Edge[](v2)(v1)
\Edge[](v1)(v0)
\Edge[](v4)(v3)
\Edge[](v5)(v3)
\Edge[](v5)(v4)
\Edge[](v7)(v6)
\Edge[](v8)(v6)
\Edge[](v8)(v7)
\Edge[](v11)(v10)
\Edge[](v12)(v10)
\Edge[](v12)(v11)
\Edge[](v9)(v3)
\Edge[](v9)(v4)
\Edge[](v9)(v5)
\Edge[](v9)(v0)
\Edge[](v9)(v1)
\Edge[](v9)(v2)

\Edge[](v2)(v5)
\Edge[](v2)(v3)
\Edge[](v2)(v4)
\end{tikzpicture}
\end{figure}
\bigskip
\onslide<2-| handout:1>\begin{figure}[h]
\centering
\begin{tikzpicture}[scale = 10]
\tikzstyle{VertexStyle}=[shape = circle,	
								 minimum size = 12pt,
								 inner sep = 1.2pt,
                         draw]
\Vertex[style = {fill=magenta}, x = 0.5, y = 0.7, L = \tiny {}]{v0}
\Vertex[style = {fill=cyan}, x = 0.45, y = 0.6, L = \tiny {}]{v1}
\Vertex[style = {fill=yellow}, x = 0.55, y = 0.6, L = \tiny {$x_4$}]{v2}
\Vertex[style = {fill=green}, x = 0.35, y = 0.7, L = \tiny {}]{v3}
\Vertex[style = {fill=red}, x = 0.3, y = 0.6, L = \tiny {$x_3$}]{v4}
\Vertex[style = {fill=blue}, x = 0.4, y = 0.6, L = \tiny {}]{v5}
\Vertex[style = {fill=magenta, opacity=.2}, x = 0.5, y = 0.85, L = \tiny {}]{v6}
\Vertex[style = {fill=cyan, opacity=.2}, x = 0.45, y = 0.75, L = \tiny {}]{v7}
\Vertex[style = {fill=yellow, opacity=.2}, x = 0.55, y = 0.75, L = \tiny {$x_2$}]{v8}
\Vertex[x = 0.425, y = 0.4, L = \tiny {$x$}]{v9}
\Vertex[style = {fill=green, opacity=.2}, x = 0.35, y = 0.85, L = \tiny {}]{v10}
\Vertex[style = {fill=red, opacity=.2}, x = 0.3, y = 0.75, L = \tiny {$x_1$}]{v11}
\Vertex[style = {fill=blue, opacity=.2}, x = 0.4, y = 0.75, L = \tiny {}]{v12}
\Edge[](v2)(v0)
\Edge[](v2)(v1)
\Edge[](v1)(v0)
\Edge[](v4)(v3)
\Edge[](v5)(v3)
\Edge[](v5)(v4)
\Edge[style={opacity=.2}](v7)(v6)
\Edge[style={opacity=.2}](v8)(v6)
\Edge[style={opacity=.2}](v8)(v7)
\Edge[style={opacity=.2}](v11)(v10)
\Edge[style={opacity=.2}](v12)(v10)
\Edge[style={opacity=.2}](v12)(v11)
\Edge[](v9)(v3)
\Edge[](v9)(v4)
\Edge[](v9)(v5)
\Edge[](v9)(v0)
\Edge[](v9)(v1)
\Edge[](v9)(v2)

\Edge[](v2)(v5)
\Edge[](v2)(v3)
\Edge[](v2)(v3)
\end{tikzpicture}
\end{figure}
\bigskip
\end{overprint}
\begin{itemize}
\1 use the fact that you wrapped around to show that there are many edges between the two cliques \\
\2 we have now constructed the desired large ``dense'' subgraph \\
\end{itemize}
\end{frame}
\section{A spectrum of generalizations}
\subsection{Generalizing maximum degree}
\begin{frame}{Generalizing maximum degree}
\uncover<1->{
\begin{defn}
For $0 \leq \epsilon \leq 1$, define $\Delta_\epsilon(G)$ as

\[\left\lfloor\max_{xy \in E(G)} (1 - \epsilon)\min\{d(x), d(y)\} + \epsilon\max\{d(x), d(y)\}\right\rfloor.\]
\end{defn}
}
\uncover<2->{
Note that $\Delta_1 = \Delta$, $\Delta_{\frac12} = \left\lfloor\frac{\theta}{2} \right\rfloor$.
}
\end{frame}
\subsection{The generalized bound}
\begin{frame}{The generalized bound}
\uncover<1->{
\begin{Spectrum}[Rabern 2010]
For every $0 < \epsilon \leq 1$, there exists $t_\epsilon$ such that every graph with $\Delta_\epsilon \geq t_\epsilon$ satisfies $\chi \leq \max\{\omega, \Delta_\epsilon\}$.
\end{Spectrum}}
\begin{itemize}
\2 the proof uses a recoloring algorithm similar to the above \\
\3 it would be interesting to determine, for each $\epsilon$, the smallest $t_\epsilon$ that works \\
\4 that $t_1 = 3$ is smallest is Brooks' theorem \\
\5 the graph $O_5$ shows that $t_\epsilon = 6$ is smallest for $\frac12 \leq \epsilon < 1$ \\
\6 best known general bounds, $\frac{2}{\epsilon} + 1 \leq t_\epsilon \leq \frac{4}{\epsilon} + 2$
\end{itemize}
\end{frame}
\subsection{The lower bound on $t_\epsilon$}
\begin{frame}{The lower bound on $t_\epsilon$}
\uncover<1->{\begin{figure}[h]
\centering
\begin{tikzpicture}[scale = 1]

\node[circle, minimum width=3cm, thick, draw] (L) at (0,0) {$K_{n-2}$};
\node[ellipse, minimum height=2cm, minimum width=0.5cm, thick, draw] (M) at (2.5,0) {};      
\node[circle split, minimum width=3cm, thick, draw] (R) at (5,0) 
{$K_{\ceil{\frac{n-1}{2}}}$ \nodepart{lower} $K_{\floor{\frac{n-1}{2}}}$};
\node[circle, inner sep =1pt, fill, draw] (P1) at (2.5,-0.5) {};
\node[circle, inner sep =1pt, fill, draw] (P2) at (2.5,0.5) {};

\draw (L) (M) (R) (P1) (P2);
\draw[ultra thick] (1.6,-.25) -- (2.15,-.25);
\draw[ultra thick] (1.6,0) -- (2.15,0);
\draw[ultra thick] (1.6,.25) -- (2.15,.25);

\draw[ultra thick] (2.85,-.7) -- (3.9,-1.15);
\draw[ultra thick] (2.85,-.6) -- (3.7,-.9);
\draw[ultra thick] (2.85,-.5) -- (3.5,-.65);

\draw[ultra thick] (2.85,.7) -- (3.9,1.15);
\draw[ultra thick] (2.85,.6) -- (3.7,.9);
\draw[ultra thick] (2.85,.5) -- (3.5,.65);
\end{tikzpicture}
\caption{The graph $O_n$.}
\end{figure}
}
\begin{itemize}
\2 $\chi(O_n) = n > \omega(O_n)$ and $\Delta(O_n) = \ceil{\frac{n-1}{2}} + n - 2$ \\
\3 $\fancy{H}(O_n)$ is edgeless \\
\4 computing $\Delta_\epsilon$ gives $t_\epsilon \geq \frac{2}{\epsilon} + 1$
\end{itemize}
\end{frame}
\subsection{What about $\Delta_0$?}
\begin{frame}{What about $\Delta_0$?}
\begin{itemize}
\1 the above proofs only work for $\epsilon > 0$ \\
\2 what happens when $\epsilon = 0$? \\
\3 the parameter $\Delta_0$ has already been investigated by Stacho under the name $\Delta_2$ \\
\end{itemize}
\uncover<4->{
\begin{defn}[Stacho 2001]
For a graph $G$ define

\[\Delta_0(G) \DefinedAs \max_{xy \in E(G)} \min\{d(x), d(y)\}.\]
\end{defn}
}
\end{frame}
\begin{frame}{Facts about $\Delta_0$}
\begin{itemize}
\1 greedy coloring (in any order) shows that every graph satisfies $\chi \leq \Delta_0 + 1$ \\
\2 for any fixed $t \geq 3$, the problem of determining whether or not $\chi(G) \leq \Delta_0(G)$ for graphs with $\Delta_0(G) = t$ is \emph{NP}-complete (Stacho 2001)
\end{itemize}
\end{frame}

\begin{frame}{A tempting thought}
\uncover<1->{
\begin{TemptingThought}
There exists $t$ such that every graph with $\Delta_0 \geq t$ satisfies $\chi \leq \max \{\omega, \Delta_0\}$.
\end{TemptingThought}}

\begin{itemize}
\2 since $t_\epsilon \geq \frac{2}{\epsilon} + 1$, we see that $t_\epsilon \rightarrow \infty$ as $\epsilon \rightarrow 0$
\3 thus, $t_0$ does not exist and the tempting thought cannot hold \\
\4 there is a cute algorithmic way to see this assuming \emph{P}$\neq$\emph{NP} \\
\5 we use Lov\'{a}sz's $\vartheta$ parameter which can be appoximated in polynomial time and has the property that $\omega(G) \leq \vartheta(G) \leq \chi(G)$
\end{itemize}
\end{frame}

\begin{frame}{A polynomial-time algorithm}
\begin{itemize}
\1 assume the tempting thought holds for some $t \geq 3$ \\
\2 take any arbitrary graph with $\Delta_0 \geq t$ \\
\3 first, compute $\Delta_0$ in polynomial time \\
\4 second, compute $x$ such that $x - \frac12 < \vartheta < x + \frac12$ in polynomial time \\
\5 if $x \geq \Delta_0 + \frac12$, then $\chi \geq \vartheta > \Delta_0$ and hence $\chi = \Delta_0 + 1$ \\
\6 if $x < \Delta_0 + \frac12$, then $\omega \leq \vartheta < \Delta_0 + 1$, and hence $\omega \leq \Delta_0$ \\
\7  now, $\chi \leq \max \{\omega, \Delta_0\} \leq \Delta_0$ \\
\8 we just gave a polynomial time algorithm to determine whether or not $\chi \leq \Delta_0$ for graphs with $\Delta_0 \geq t$ \\
\9 this is impossible unless \emph{P}=\emph{NP}
\end{itemize}
\end{frame}

\begin{frame}{What we can prove about $\Delta_0$}
\uncover<1->{
\begin{DeltaTwo}[Rabern 2010]
Every graph with $\Delta \geq 3$ satisfies \[\chi \leq \max \left\{\omega, \Delta_0, \frac{5}{6}(\Delta + 1)\right\}.\]
\end{DeltaTwo}}

\begin{itemize}
\2 the proof uses a recoloring algorithm similar to the above \\
\3 actually, all the above results about $\Delta_\epsilon$ follow from this result \\
\end{itemize}
\end{frame}

\section{Further improvements}
\begin{frame}
\uncover<1->{
In joint work with Kostochka and Stiebitz similar techniques were used to improve the bounds further.  Highlights:}

\uncover<2->{
\begin{krs1}[Kostochka, Rabern and Stiebitz 2010]
Every graph with $\theta \geq 8$, except $O_5$, satisfies $\chi \leq \max \left\{\omega, \left\lfloor\frac{\theta}{2} \right \rfloor\right\}$.
\end{krs1}}

\uncover<3->{
\begin{krs2}[Kostochka, Rabern and Stiebitz 2010]
Every graph satisfies \[\chi \leq \max \left\{\omega, \Delta_0, \frac{3}{4}(\Delta + 2)\right\}.\]
\end{krs2}}
\end{frame}

\begin{frame}
\uncover<1->
{
\begin{conjecture}
Every graph satisfies \[\chi \leq \max \left\{\omega, \Delta_0, \frac{2\Delta + 5}{3}\right\}.\]
\end{conjecture}}

\bigskip

The examples $O_n$ above show that this would be tight.
\end{frame}

\nocite{krs_one}
\nocite{gr�tschel1981ellipsoid}
\nocite{stacho2001new}
\nocite{rabern2010a}
\nocite{stiebitz1982proof}
\nocite{kierstead2009ore}
\nocite{stacho2001new}
\nocite{rabern2010b}
\nocite{prison}
\nocite{rabernhitting}
\bibliographystyle{plain}
\tiny
\bibliography{GraphColoring}
\end{document}

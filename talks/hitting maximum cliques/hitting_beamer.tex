\documentclass{beamer}
\usepackage{tkz-graph}
%===============BEGIN GLOBAL BEAMER OPTIONS=====================

%THIS CHANGES THE BACKGROUND COLOUR OF BOXES, AND ROUNDS OFF THEIR EDGES
%\setbeamercolor{block title}{bg=yellow!50}
%\setbeamercolor{block body}{}
%\setbeamertemplate{blocks}[default]

%THIS REMOVES NAVIGATION BAR ON THE BOTTOM
\setbeamertemplate{navigation symbols}{}


%%THIS ADDS FRAME NUMBERS TO THE Right (x/y)
%\newcommand*\oldmacro{}
%\let\oldmacro\insertshorttitle
%\renewcommand*\insertshorttitle{
 %\oldmacro\hfill
 %\insertframenumber\,/\,\inserttotalframenumber}

%+++++++++++++ MODES AND THEMES +++++++++
\mode<presentation>

\usetheme[hideothersubsections]{Hannover}
\addtobeamertemplate{footline}
{
  \usebeamercolor[fg]{author in sidebar}
  \vskip-1cm\hskip2pt
  %\insertpagenumber\,/\,\insertpresentationendpage\kern1em\vskip2pt%
  \insertframenumber\,/\,\inserttotalframenumber\kern1em\vskip2pt%
}

%\usecolortheme{beaver}
%\usecolortheme{crane}
%\usecolortheme{default}
%\usecolortheme{albatross}
%\usecolortheme{beetle} 
%\usecolortheme{dove} 
%\usecolortheme{fly} 
%\usecolortheme{seagull}
%\usecolortheme{rose}
\usecolortheme{orchid}
%\usecolortheme{seahorse}



%\useoutertheme{split}
%\useoutertheme{shadow}

% USE TO CREATE PRINTER FRIENDLY HANDOUT VERSION, TOGETHER WITH [HANDOUT]
%ALSO USE ALTERNATE FIRST SLIDE
%\usecolortheme{dove}

%THIS CHANGES THE DEFAULT TRANSPARENCY GRADE FOR UNCOVERING COMMANDS
%\setbeamercovered{transparent=30}

%\beamertemplateshadingbackground{yellow!70}{black!20}

%====================================

\title[Hitting maximum cliques]{Hitting maximum cliques}
\author{Landon Rabern}
\institute{landon.rabern@gmail.com}
%\date{June 1, 2011}

\theoremstyle{plain}
\newtheorem{thm}{Theorem}
\newtheorem{prop}[thm]{Proposition}
\newtheorem{lem}[thm]{Lemma}
\newtheorem{cor}[thm]{Corollary}
\newtheorem*{conjecture}{Conjecture}
\newtheorem{claim}{Claim}
\theoremstyle{definition}
\newtheorem{defn}{Definition}
\newtheorem*{CliqueGraph}{Clique graph}
\theoremstyle{remark}
\newtheorem*{remark}{Remark}
\newtheorem*{question}{Question}
\newtheorem*{observation}{Observation}

\newcommand{\fancy}[1]{\mathcal{#1}}
\newcommand{\C}[1]{\fancy{C}_{#1}}
\newcommand{\IN}{\mathbb{N}}
\newcommand{\IR}{\mathbb{R}}

\newcommand{\inj}{\hookrightarrow}
\newcommand{\surj}{\twoheadrightarrow}

\newcommand{\set}[1]{\left\{ #1 \right\}}
\newcommand{\setb}[3]{\left\{ #1 \in #2 \mid #3 \right\}}
\newcommand{\setbs}[2]{\left\{ #1 \mid #2 \right\}}
\newcommand{\card}[1]{\left|#1\right|}
\newcommand{\size}[1]{\left\Vert#1\right\Vert}
\newcommand{\ceil}[1]{\left\lceil#1\right\rceil}
\newcommand{\floor}[1]{\left\lfloor#1\right\rfloor}
\newcommand{\func}[3]{#1\colon #2 \rightarrow #3}
\newcommand{\funcinj}[3]{#1\colon #2 \inj #3}
\newcommand{\funcsurj}[3]{#1\colon #2 \surj #3}
\newcommand{\irange}[1]{\left[#1\right]}
\newcommand{\join}[2]{#1 \mbox{\hspace{2 pt}$\ast$\hspace{2 pt}} #2}
\newcommand{\djunion}[2]{#1 \mbox{\hspace{2 pt}$+$\hspace{2 pt}} #2}
\newcommand{\parens}[1]{\left( #1 \right)}

\newcommand{\DefinedAs}{\mathrel{\mathop:}=}

\newcommand{\1}{\item<1-> }
\newcommand{\2}{\item<2-> }
\newcommand{\3}{\item<3-> }
\newcommand{\4}{\item<4-> }
\newcommand{\5}{\item<5-> }
\newcommand{\6}{\item<6-> }
\newcommand{\7}{\item<7-> }
\newcommand{\8}{\item<8-> }
\newcommand{\9}{\item<9-> }
\newcommand{\ten}{\item<10-> }
\newcommand{\ele}{\item<11-> }
\newcommand{\twe}{\item<12-> }
\newcommand{\thi}{\item<13-> }
\newcommand{\fou}{\item<14-> }
\newcommand{\fif}{\item<15-> }
\newcommand{\six}{\item<16-> }
\newcommand{\sev}{\item<17-> }
\newcommand{\eig}{\item<18-> }

\begin{document}
\begin{frame}
\titlepage
\end{frame}

\begin{frame}{Outline}
  \tableofcontents
\end{frame}
\section{Introduction}
\begin{frame}{Introduction}
\uncover<1->{
Finding a stable set hitting every maximum clique in a graph can be very useful for coloring problems.}
\uncover<2->{
\begin{observation}
A graph is perfect iff every induced subgraph has a stable set hitting every maximum clique.
\end{observation}}

\uncover<3->{
Kostochka \cite{kostochkaRussian} gave the following sufficient condition.
\begin{lem}[Kostochka 1980]
A graph satisfying $\omega \geq \Delta + \frac32 - \sqrt{\Delta}$ has a stable set hitting every maximum clique.
\end{lem}}
\end{frame}
\begin{frame}
\uncover<1->{
In \cite{rabernhitting}, we improved this as follows.
\begin{lem}[Rabern 2009]
There exists a positive constant $c < 1$ such that every graph satisfying $\omega > c(\Delta + 1)$ has a stable set hitting every maximum clique.
\end{lem}}
\end{frame}

\subsection{What's it good for?}
\begin{frame}{What's it good for?}
\begin{itemize}
\1 removing a stable set which hits every maximum clique decreases $\omega$
\2 say you are trying to prove Brooks' theorem and start by taking a counterexample minimizing $\Delta$
\3 suppose $\Delta \geq 4$
\4 then $\omega = 4$, hence by the fact that $c = \frac34$ works in the above lemma, we have a stable set hitting every maximum clique
\5 expanding it to a maximal stable set and removing it leaves a graph with $\omega \leq 3$, $\chi = 4$ and $\Delta \leq 3$, contradicting minimality of $\Delta$
\6 thus a countexample to Brooks' theorem minimizing $\Delta$ must have $\Delta = 3$
\end{itemize}
\end{frame}

\begin{frame}{What's it good for?}
\begin{itemize}
\1 Borodin and Kostochka \cite{borodin1977upper} conjecture that graphs with $\Delta \geq 9$ containing no $K_\Delta$ are $(\Delta - 1)$-colorable.  
\2 as shown by Kostochka, almost identically to the case of Brooks' theorem, we may assume that $\Delta = 9$
\3 Reed conjectures that every graph satisfies $\chi \leq \ceil{\frac{\omega + \Delta + 1}{2}}$
\4 when proving this conjecture for a hereditary class of graphs, a minimum counterexample must have $\omega \leq \frac34(\Delta + 1)$
\5 this was used in \cite{rabernhitting} to simplify the proof of Reed's conjecture for line graphs given in \cite{king2007upper}
\end{itemize}
\end{frame}

\subsection{An outline of the proof}
\begin{frame}{An outline of the proof}
\uncover<1->{
PICTURE
}
\begin{itemize}
\2 use lemmas of Hajnal and Kostochka to show that each component of the maximum clique graph has many universal vertices
\3 consider the subgraph induced on these universal vertices and partition it by component
\4 find an independent transversal
\5 this is our desired stable set hitting all maximum cliques
\end{itemize}

\end{frame}
\section{The common lemmas}
\subsection{Hajnal's clique collection lemma}
\begin{frame}{Hajnal's clique collection lemma}
\uncover<1->{
\begin{lem}[Hajnal 1965]
For a collection $\fancy{Q}$ of maximum cliques in a graph $G$ we have
\[\card{\bigcup \fancy{Q}} + \card{\bigcap \fancy{Q}} \geq 2\omega(G).\]
\end{lem}}

\begin{itemize}
\2 let $\mathcal{Q}$ be a counterexample with $\card{\mathcal{Q}}$ minimal
\3 put $r \DefinedAs \card{\mathcal{Q}}$, say $\mathcal{Q} = \set{Q_1, \ldots, Q_r}$
\4 consider $W \DefinedAs (Q_1 \cap \bigcup_{i=2}^r Q_i) \cup \bigcap_{i=2}^r Q_i$
\5 $W$ is a clique, so the following machinations give a contradiction
\end{itemize}
\end{frame}

\begin{frame}{Machinations}
\uncover<1->{

\begin{align*}
\omega(G) &\geq |W| \\
&= \textstyle\card{(Q_1 \cap \bigcup_{i=2}^r Q_i) \cup \bigcap_{i=2}^r Q_i} \\
&= \textstyle\card{Q_1 \cap \bigcup_{i=2}^r Q_i} + \card{\bigcap_{i=2}^r Q_i} - \card{\bigcap_{i=1}^r Q_i \cap \bigcup_{i=2}^r Q_i} \\
&= \textstyle\card{Q_1} +\card{\bigcup_{i=2}^r Q_i} - \card{\bigcup_{i=1}^r Q_i} + \card{\bigcap_{i=2}^r Q_i} - \card{\bigcap_{i=1}^r Q_i} \\
&= \textstyle\omega(G) +\card{\bigcup_{i=2}^r Q_i} + \card{\bigcap_{i=2}^r Q_i} - \card{\bigcup_{i=1}^r Q_i} - \card{\bigcap_{i=1}^r Q_i} \\
&\geq \textstyle\omega(G) + 2\omega(G) - \left(\card{\bigcup_{i=1}^r Q_i} + \card{\bigcap_{i=1}^r Q_i}\right) \\
& > \omega(G).
\end{align*}}
\end{frame}

\subsection{Kostochka's lemma}
\begin{frame}{The clique graph}
\uncover<1->{
In \cite{2009arXiv0907.3705R} we gave the following simple proof of Kostochka's lemma from \cite{kostochkaRussian}.  First a definition.
}

\uncover<2->{
\begin{CliqueGraph}
For a collection of cliques $\mathcal{Q}$ in a graph, let $X_{\mathcal{Q}}$ be the intersection graph of $\mathcal{Q}$; that is, the vertex set of $X_{\mathcal{Q}}$ is $\mathcal{Q}$ and there is an edge between $Q_1 \neq Q_2 \in \mathcal{Q}$ iff $Q_1$ and $Q_2$ intersect.
\end{CliqueGraph}}
\end{frame}

\begin{frame}{Kostochka's lemma}
\uncover<1->{
\begin{lem}[Kostochka 1980]
Let $G$ be a graph satisfying $\omega > \frac{2}{3}(\Delta + 1)$.   If $\mathcal{Q}$ is a collection of maximum cliques in $G$ such that $X_{\mathcal{Q}}$ is connected, then $\cap \mathcal{Q} \neq \emptyset$.
\end{lem}}

\begin{itemize}
\2 let $\mathcal{Q}$ be a counterexample with $\card{\mathcal{Q}}$ minimal
\3 let $A \in \mathcal{Q}$ be a non-cutvertex in $X_{\mathcal{Q}}$ and $B$ a neighbor of $A$
\4 put $\mathcal{Z} \DefinedAs \mathcal{Q} - \set{A}$
\5 then $X_{\mathcal{Z}}$ is connected and hence by minimality of $\card{\mathcal{Q}}$, $\cap \mathcal{Z} \neq \emptyset$
\6 in particular $\card{\cup \mathcal{Z}} \leq \Delta(G) + 1$
\7 thus $\card{\cup Q} \leq \card{A - B} + \card{\cup \mathcal{Z}} \leq 2(\Delta(G) + 1) - \omega(G) < 2\omega(G)$ contradicting Hajnal

\end{itemize}
\end{frame}

\section{Proving the main results}
\subsection{Independent transversals}
\begin{frame}{Independent transversals}
\uncover<1->{
\begin{defn}
An \emph{independent transversal} of a partition $\set{V_1, \ldots, V_r}$ of a the vertex set of a graph $G$ is a stable set $\set{v_1, \ldots, v_r} \subseteq V(G)$ such that $v_i \in V_i$ for each $i$.
\end{defn}}

\uncover<2->{
Alon \cite{alon1988linear} proved a simple sufficient condition probabilistically.
}

\uncover<3->{
\begin{lem}[Alon 1988]
A partition $\set{V_1, \ldots, V_r}$ of the vertex set of a graph $G$ has an independent transversal if $\card{V_i} \geq 2e\Delta(G)$ for each $i$.
\end{lem}}

\end{frame}

\begin{frame}{Alon's proof}
\begin{itemize}
\1 put $\Delta = \Delta(G)$
\2 without loss of generality we may suppose that $\card{V_i} = k \geq 2e\Delta$ for each $i$
\3 randomly select one vertex from each $V_i$
\4 for each edge $e$ of $G$, let $B_e$ be the event that both ends of $e$ get selected
\5 $\fancy{P}(B_e) \leq k^{-2}$
\6 each $B_e$ is independent of all but at most $d \DefinedAs 2k(\Delta - 1) < 2k\Delta - 1 \leq k^2e^{-1} - 1$ other events
\7 since $e\fancy{P}(B_e)(d + 1) < 1$, the Lov{\'a}sz Local Lemma implies that the probability that none of the $B_e$ occur is positive
\8 hence an independent transversal exists
\end{itemize}
\end{frame}

\subsection{Putting it all together}
\begin{frame}{Putting it all together}
\uncover<1->{
\begin{lem}[Rabern 2009]
A graph satisfying $\omega \geq \frac{12}{13}(\Delta + 1)$ has a stable set hitting every maximum clique.
\end{lem}}

\begin{itemize}
\2 let $G$ be a graph satisfying $\omega \geq \frac{12}{13}(\Delta + 1)$
\3 let $\fancy{Q}$ be the maximum cliques in $G$ and $\fancy{Q}_1, \ldots, \fancy{Q}_t$ the vertex sets of the components of $X_{\fancy{Q}}$
\4 put $K_i = \cap \fancy{Q}_i$
\5 by Kostochka's lemma, $K_i \neq \emptyset$ for each $i$
\6 in particular, $\card{\cup \fancy{Q}_i} \leq \Delta(G) + 1$
\end{itemize}
\end{frame}

\begin{frame}{Putting it all together}
\begin{itemize}
\1 put $k \DefinedAs \min_i \card{K_i}$
\2 by Hajnal's lemma, $k \geq 2\omega(G) - (\Delta(G) + 1) \geq \frac{11}{13}(\Delta + 1)$
\3 consider the graph $H$ with vertex set $\cup_i K_i$ and edge set $\setbs{xy \in E(G)}{x \in K_i, y \in K_j, i \neq j}$
\4 $\Delta(H) \leq \Delta(G) + 1 - k \leq \frac{2}{13}(\Delta(G) + 1)$
\5 hence $k \geq \frac{11}{13}(\Delta(G) + 1) \geq \frac{4e}{13}(\Delta(G) + 1) \geq 2e\Delta(H)$
\6 by Alon's lemma, we have an independent transversal through the $K_i$ and this is the desired stable set hitting every maximum clique
\end{itemize}
\end{frame}

\subsection{Improving the constant}
\begin{frame}{Improving the constant}
\uncover<1->{
In \cite{rabernhitting} we proved that $c = \frac34$ works in the same way as above using the following lemma of Haxell \cite{haxell2001note}.}

\uncover<2->{
\begin{lem}[Haxell 2001]
A partition $\set{V_1, \ldots, V_r}$ of the vertex set of a graph $G$ has an independent transversal if $\card{V_i} \geq 2\Delta(G)$ for each $i$.
\end{lem}}

\begin{itemize}
\3 Haxell's proof is elementary and uses some somewhat delicate induction
\4 there are also proofs based on topological connectivity of the independent set complex
\end{itemize}
\end{frame}

\begin{frame}{Improving the constant}
\uncover<1->{
Building on observations by Aharoni, Berger and Ziv \cite{aharoni2007independent} about the proof of Haxell's lemma, King \cite{KingAXiv} proved the following lopsided version of Haxell's lemma.  Using this, he proved that $c = \frac23$ works.}

\uncover<2->{
\begin{lem}[King 2009]
A partition $\set{V_1, \ldots, V_r}$ of the vertex set of a graph $G$ has an independent transversal if there exists a positive integer $k$ such that for 
each $i$ we have $\min \set{k, \card{V_i} - k} \geq \max_{v \in V_i} d(v)$.
\end{lem}}
\end{frame}

\begin{frame}{Tightness of King's lemma}
\begin{itemize}
\1 let $G_r$ be line graph of a $5$-cycle where each edge has multiplicity $r$; that is, $G_r \DefinedAs L(r \cdot C_5)$
\2 then $\omega(G_r) = 2r = \frac23(\Delta(G_r) + 1)$ and no stable set hits every maximum clique
\3 thus King's lemma is tight
\end{itemize}

\uncover<4->{
PICTURE GOES HERE
}
\end{frame}

\nocite{hajnaltheorem}
\nocite{rabernhitting}
\bibliographystyle{amsplain}
\tiny
\bibliography{GraphColoring}
\end{document}

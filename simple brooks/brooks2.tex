\documentclass[12pt]{amsart}
\usepackage{amsmath, amsthm, amssymb}
\usepackage[top=1.25in, bottom=1.25in, left=1.0in, right=1.0in]{geometry}

\pagestyle{headings}

\makeatletter
\newtheorem*{rep@theorem}{\rep@title}
\newcommand{\newreptheorem}[2]{
\newenvironment{rep#1}[1]{
 \def\rep@title{#2 \ref{##1}}
 \begin{rep@theorem}}
 {\end{rep@theorem}}}
\makeatother

\theoremstyle{plain}
\newtheorem{thm}{Theorem}
\newreptheorem{thm}{Theorem}
\newtheorem{prop}[thm]{Proposition}
\newreptheorem{prop}{Proposition}
\newtheorem{lem}[thm]{Lemma}
\newreptheorem{lem}{Lemma}
\newtheorem{conjecture}[thm]{Conjecture}
\newreptheorem{conjecture}{Conjecture}
\newtheorem{cor}[thm]{Corollary}
\newreptheorem{cor}{Corollary}
\newtheorem{prob}[thm]{Problem}
\theoremstyle{definition}
\newtheorem{defn}{Definition}
\theoremstyle{remark}
\newtheorem*{remark}{Remark}
\newtheorem{example}{Example}
\newtheorem*{question}{Question}
\newtheorem*{observation}{Observation}

\newcommand{\fancy}[1]{\mathcal{#1}}
\newcommand{\C}[1]{\fancy{C}_{#1}}
\newcommand{\IN}{\mathbb{N}}
\newcommand{\IR}{\mathbb{R}}
\newcommand{\G}{\fancy{G}}

\newcommand{\inj}{\hookrightarrow}
\newcommand{\surj}{\twoheadrightarrow}

\newcommand{\set}[1]{\left\{ #1 \right\}}
\newcommand{\setb}[3]{\left\{ #1 \in #2 \mid #3 \right\}}
\newcommand{\setbs}[2]{\left\{ #1 \mid #2 \right\}}
\newcommand{\card}[1]{\left|#1\right|}
\newcommand{\size}[1]{\left\Vert#1\right\Vert}
\newcommand{\ceil}[1]{\left\lceil#1\right\rceil}
\newcommand{\floor}[1]{\left\lfloor#1\right\rfloor}
\newcommand{\func}[3]{#1\colon #2 \rightarrow #3}
\newcommand{\funcinj}[3]{#1\colon #2 \inj #3}
\newcommand{\funcsurj}[3]{#1\colon #2 \surj #3}
\newcommand{\irange}[1]{\left[#1\right]}
\newcommand{\join}[2]{#1 \mbox{\hspace{2 pt}$\ast$\hspace{2 pt}} #2}
\newcommand{\djunion}[2]{#1 \mbox{\hspace{2 pt}$+$\hspace{2 pt}} #2}
\newcommand{\parens}[1]{\left( #1 \right)}
\newcommand{\DefinedAs}{\mathrel{\mathop:}=}

\title{}
\begin{document}
\maketitle

\begin{proof}[Proof of Brooks' theorem]
Suppose the theorem is false and choose a counterexample $G$ minimizing $\card{G}$.  Plainly, $G$ is connected. Let $\set{A_1, A_2}$ be a separation of $G$ minimizing $k \DefinedAs \card{A_1 \cap A_2}$.  

First suppose there is an induced $P_3$ $xyz$ in $G$ such that $G - x - z$ is connected.  Then we may order $V(G)$ as $x, z, v_1, \ldots, v_n, y$ so that each $v_i$ has a neighbor to the right.  But this is a contradiction since greedily coloring in this order uses at most $\Delta$ colors as $x$ and $z$ get the same color, each $v_i$ has at most $\Delta - 1$ neighbors to the left and $y$ has two neighbors ($x$ and $z$) colored the same.

Thus $G$ contains no such $P_3$.  In particular, $k \leq 2$.  If there are nonadjacent $u, v \in A_1 \cap A_2$, put $H_i \DefinedAs G[A_i] + uv$ noting that $\Delta(H_i) \leq \Delta(G)$ since both $u$ and $v$ have neighbors on both sides of the separation by minimality of $k$.  Otherwise put $H_i \DefinedAs G[A_i]$.  By minimality of $\card{G}$, each $H_i$ is either $\Delta(G)$-colorable or contains a $K_{\Delta(G) + 1}$.  If both $H_i$ are $\Delta(G)$-colorable, then we have $\Delta(G)$-colorings of $G[A_1]$ and $G[A_2]$ where $A_1 \cap A_2$ receives $k$ colors.  By permuting color names if necessary we can combine these to get a $\Delta(G)$-coloring of $G$, a contradiction.

Otherwise, $k = 2$ and $G$ contains an induced subgraph $H$ which is a $K_{\Delta(G) + 1}$ with one edge missing, call it $xy$.  By minimality of $\card{G}$, we may $\Delta(G)$-color $G-H$, then color $x$ and $y$ the same (as they both have at least $\Delta(G) - 1$ legal colors left and $2(\Delta(G) - 1) > \Delta(G)$) and finally greedily finish the $\Delta(G)$-coloring on the rest of $H$.  This gives a $\Delta(G)$-coloring of $G$, a contradiction.
\end{proof}

\end{document}

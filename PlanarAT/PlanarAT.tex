\documentclass[12pt]{article}
\usepackage{amsmath, amsthm, amssymb}
\usepackage{hyperref}
\usepackage[margin=1cm]{caption}
\usepackage{verbatim}
\usepackage[top=1.0in, bottom=1.0in, left=1.0in, right=1.0in]{geometry}
\usepackage{graphicx}

\pagestyle{plain}

\usepackage{tkz-graph}
\usetikzlibrary{arrows}
\usetikzlibrary{shapes}
\usepackage[position=bottom]{subfig}

\usepackage{longtable}
\usepackage{array}

\usepackage{sectsty}
\allsectionsfont{\sffamily}

\setcounter{secnumdepth}{5}
\setcounter{tocdepth}{5}

\makeatletter
\newtheorem*{rep@theorem}{\rep@title}
\newcommand{\newreptheorem}[2]{
\newenvironment{rep#1}[1]{
 \def\rep@title{#2 \ref{##1}}
 \begin{rep@theorem}}
 {\end{rep@theorem}}}
\makeatother

\theoremstyle{plain}
\newtheorem{thm}{Theorem}[section]
\newreptheorem{thm}{Theorem}
\newtheorem{prop}[thm]{Proposition}
\newreptheorem{prop}{Proposition}
\newtheorem{lem}[thm]{Lemma}
\newreptheorem{lem}{Lemma}
\newtheorem{conjecture}[thm]{Conjecture}
\newreptheorem{conjecture}{Conjecture}
\newtheorem{cor}[thm]{Corollary}
\newreptheorem{cor}{Corollary}
\newtheorem{prob}[thm]{Problem}
\newtheorem{observation}{Observation}
\newtheorem{obs}[observation]{Observation}
\newtheorem*{mainconj}{Main Conjecture}
\newtheorem*{mainthm}{Main Theorem}
\newtheorem{problem}{Problem}
\newtheorem{clm}{Claim}

\theoremstyle{definition}
\newtheorem{defn}{Definition}
\theoremstyle{remark}
\newtheorem*{remark}{Remark}
\newtheorem{example}{Example}
\newtheorem*{question}{Question}


\newcommand{\fancy}[1]{\mathcal{#1}}
\newcommand{\C}[1]{\fancy{C}_{#1}}
\newcommand{\IN}{\mathbb{N}}
\newcommand{\IR}{\mathbb{R}}
\newcommand{\G}{\fancy{G}}
\newcommand{\CC}{\fancy{C}}
\newcommand{\D}{\fancy{D}}

\newcommand{\inj}{\hookrightarrow}
\newcommand{\surj}{\twoheadrightarrow}

\newcommand{\set}[1]{\left\{ #1 \right\}}
\newcommand{\setb}[3]{\left\{ #1 \in #2 \mid #3 \right\}}
\newcommand{\setbs}[2]{\left\{ #1 \mid #2 \right\}}
\newcommand{\card}[1]{\left|#1\right|}
\newcommand{\size}[1]{\left\Vert#1\right\Vert}
\newcommand{\ceil}[1]{\left\lceil#1\right\rceil}
\newcommand{\floor}[1]{\left\lfloor#1\right\rfloor}
\newcommand{\func}[3]{#1\colon #2 \rightarrow #3}
\newcommand{\funcinj}[3]{#1\colon #2 \inj #3}
\newcommand{\funcsurj}[3]{#1\colon #2 \surj #3}
\newcommand{\irange}[1]{\left[#1\right]}
\newcommand{\join}[2]{#1 \mbox{\hspace{2 pt}$\ast$\hspace{2 pt}} #2}
\newcommand{\djunion}[2]{#1 \mbox{\hspace{2 pt}$+$\hspace{2 pt}} #2}
\newcommand{\parens}[1]{\left( #1 \right)}
\newcommand{\brackets}[1]{\left[ #1 \right]}
\newcommand{\nint}[1]{\widetilde{N}\left(#1\right)}
\newcommand{\DefinedAs}{\mathrel{\mathop:}=}
\newcommand{\pot}{\operatorname{pot}}
\newcommand{\mic}{\operatorname{mic}}
\def\adj{\leftrightarrow}
\def\nonadj{\not\!\leftrightarrow}

\def\D{\fancy{D}}
\def\C{\fancy{C}}
\def\Q{\fancy{Q}}
\def\Z{\fancy{Z}}
\def\H{\fancy{H}}
\def\L{\fancy{L}}
\def\L{\fancy{U}}

% any changes to \claim should be mirrored in \claimnonum and \subclaim
\newcommand{\claim}[2]{{\bf Claim #1.}~{\it #2}~~}
\newcommand{\claimnonum}[1]{{\bf Claim.}~{\it #1}~~}
\newcommand{\subclaim}[2]{{\bf Subclaim #1.}~{\it #2}~~}

\newcommand\numberthis{\addtocounter{equation}{1}\tag{\theequation}}

%
%  If the proof ends with a displayed equation, use \aftermath just
%  before \end{proof} to put the halmos in the ``right'' place.  This
%  may not work near page boundaries. 
%
\def\aftermath{\par\vspace{-\belowdisplayskip}\vspace{-\parskip}\vspace{-\baselineskip}}

\def\fr{\frac}
\def\adj{\leftrightarrow}
\def\ch{\textrm{ch}}

\renewcommand{\restriction}{\mathord{\upharpoonright}}
\begin{document}
\title{notes for planar 5-AT}
\author{}
\maketitle

\section{orientation tools}
Let $G$ be a graph and $\le$ a total order on $V(G)$.  
An orientation of $G$ is \emph{even} if the number of directed edges $vw$ with $v \le w$ is even; otherwise, the orientation is $\emph{odd}$.
Let $\func{\alpha}{V(G)}{\IN}$.  An orientation $X$ of $G$ is an \emph{$\alpha$-orientation} if $d_X^+(v) = \alpha(v)$ for all $v \in V$.  Let $D_\alpha(G)$ be the set
of $\alpha$-orientations of $G$.  We partition $D_\alpha(G)$ into even $\alpha$-orientations $DE_\alpha(G)$ and odd $\alpha$-orientations $DO_\alpha(G)$.  
For $X, Y \in D_\alpha(G)$, let $X \oplus Y$ be the spanning subgraph of $X$ with edge set 
\[\setb{x_1x_2}{E(X)}{x_2x_1 \in E(Y)}.\]
Then $X \oplus Y$ is a spanning Eulerian subgraph of $X$.  We say that a spanning Eulerian subgraph of $X$ is \emph{even} if it has an even number of edges and \emph{odd} otherwise.  
Let $EL(X)$ be the set of spanning Eulerian subgraphs of $X$.  We partition $EL(X)$ into even spanning Eulerian subgraphs $EE(X)$ and odd spanning Eulerian subgraphs $EO(X)$.  

\begin{lem}\label{EulerianDifferenceIsEvenOddDifference}
	Let $X \in D_\alpha(G)$.  For each $S \in EL(X)$ there is a unique $X_S \in D_\alpha(G)$ such that $S = X \oplus X_S$.  Moreover, $S$ is odd when $X$ and $X_S$ have opposite parity and even otherwise.
	Therefore, if $X$ is even, then $|EE(X)| = |DE_\alpha(G)|$ and $|EO(X)| = |DO_\alpha(G)|$.  If $X$ is odd, then $|EE(X)| = |DO_\alpha(G)|$ and $|EO(X)| = |DE_\alpha(G)|$.  
	So, up to sign, we always have
	\[|EE(X)| - |EO(X)| = |DE_\alpha(G)| - |DO_\alpha(G)|.\]
\end{lem}

Since Lemma \ref{EulerianDifferenceIsEvenOddDifference} was for any $X \in D_\alpha(G)$, we have the following.
\begin{cor}\label{AlphaOrientationsSameEulerian}
	If $X, Y \in D_\alpha(G)$ then, up to sign, we have	
	\[|EE(X)| - |EO(X)| = |EE(Y)| - |EO(Y)|.\]
\end{cor}

It will be useful to investigate $\alpha$-orientations further.  First, a basic fact about Eulerian graphs.

\begin{lem}\label{EulerianIntoCycles}
	If $D$ is an Eulerian directed graph, then $D$ is an edge-disjoint union of directed cycles.
\end{lem}
\begin{proof}
	If $D$ is not edgeless, it must have a directed cycle, remove it and apply induction.
\end{proof}

One important thing to note about Lemma \ref{EulerianIntoCycles} is there may be multiple different decompositions of $D$ into directed cycles.   Following Felsner \cite{felsner2004lattice}, we say that $vw \in E(G)$ is \emph{$\alpha$-rigid} 
if $vw$ is oriented the same way in every $\alpha$-orientation of $G$.  

\begin{lem}\label{OppCycle}
	If $X, Y \in D_\alpha(G)$ with $x_1x_2 \in E(X)$ and $x_2x_1 \in E(Y)$, then there is a directed cycle $C$ in $X$ containing $x_1x_2$ such that $Y$ contains the directed cycle made from $C$ by reversing all edges.
\end{lem}
\begin{proof}
	Since $X \oplus Y$ is Eulerian, it is an edge-disjoint union of directed cycles.  Let $C$ be the directed cycle containing $x_1x_2$.
\end{proof}

From Lemma \ref{OppCycle} we have the following.

\begin{cor}\label{Rigid}
  An edge $e$ of $G$ is $\alpha$-rigid if and only if no $\alpha$-orientation of $G$ has a directed cycle containing $e$.
\end{cor}

A graph $G$ is \emph{$\alpha$-AT} if there is an $\alpha$-orientation $X$ of $G$ with $EE(X) \ne EO(X)$.  Note that by Lemma \ref{EulerianDifferenceIsEvenOddDifference}, if $G$ is $\alpha$-AT then
$EE(X) \ne EO(X)$ for every $X \in D_\alpha(G)$.  It is useful to see how $\alpha$-AT behaves when we remove edges.

\begin{lem}\label{EdgeRemovalBreakdown}
	For any $\alpha$-orientation of $G$ and $vw \in E(G)$ with $v \le w$, we have
\begin{align*}
|D_\alpha(G)| &= |D_{\alpha - 1_v}(G)| + |D_{\alpha - 1_w}(G)|, \text{ and} \\
|DE_\alpha(G)| &= |DO_{\alpha - 1_v}(G)| + |DE_{\alpha - 1_w}(G)|, \text{ and} \\
|DO_\alpha(G)| &= |DE_{\alpha - 1_v}(G)| + |DO_{\alpha - 1_w}(G)|.
\end{align*}
	
\end{lem}

\begin{lem}\label{EdgeRemoval}
	Suppose $G$ is $\alpha$-AT and $vw \in E(G)$ with $v \le w$.  If $vw$ is $\alpha$-rigid (say always directed from $v$ to $w$), then $G-vw$ is $(\alpha - 1_v)$-AT.  Otherwise, $G-vw$ is either $(\alpha - 1_v)$-AT or $(\alpha - 1_w)$-AT.
\end{lem}
\begin{proof}
	First, suppose $vw$ is $\alpha$-rigid.  Let $X$ be an $\alpha$-orientation of $G$.  Then $vw$ is not contained in any $S \in EL(X)$ and hence removing it does not change parities. So, $G-vw$ is $(\alpha - 1_v)$-AT.
	
	Now, suppose $vw$ is not $\alpha$-rigid.  By Lemma \ref{EdgeRemovalBreakdown}, we have
	\[0 \ne |DE_\alpha(G)| - |DO_\alpha(G)| = |DO_{\alpha - 1_v}(G)| - |DE_{\alpha - 1_v}(G)| + |DE_{\alpha - 1_w}(G)| - |DO_{\alpha - 1_w}(G)|.\]
	Hence either $|DO_{\alpha - 1_v}(G)| - |DE_{\alpha - 1_v}(G)| \ne 0$ or $|DE_{\alpha - 1_w}(G)| - |DO_{\alpha - 1_w}(G)| \ne 0$.  By Lemma \ref{EulerianDifferenceIsEvenOddDifference}, $G-vw$ is either $(\alpha - 1_v)$-AT or $(\alpha - 1_w)$-AT.
\end{proof}


We will use the following to reverse an edge on a triangle cutset when the inductive hypothesis directs the triangle cyclically.
\begin{lem}\label{AtLeastTwoCycles}
	Suppose $G$ is $\alpha$-AT and $X$ is an $\alpha$-orientation of $G$.  If $Z$ is an induced subgraph of $X$ such that $EE(Z) = EO(Z)$, then $X$ has an induced cycle $C \not \subseteq Z$ containing an edge of $Z$.
\end{lem}
\begin{proof}
	Otherwise, every spanning Eulerian subgraph of $X$ is the edge-disjoint union of a spanning Eulerian subgraph of $Z$ and a spanning Eulerian subgraph of $X - E(Z)$.  But then $EE(Z) = EO(Z)$ implies $EE(X) = EO(X)$, a contradiction.
\end{proof}

\section{planar graphs}
We are going to try to prove Thomassen's stronger result about choosability of near-triangulations for AT.  Precisely, our aim is the following.

\begin{conjecture}
  Let $G$ be a plane near-triangulation with outer face $C$.  Then for any $x_1x_2 \in E(C)$, there is an orientation $X$ of $G-x_1x_2$ such that
  \begin{enumerate}
  	\item $d_X^+(x_1) = d_X^+(x_2) = 0$, and
  	\item $d_X^+(v) \le 2$ for all $v \in V(C)$, and
  	\item $d_X^+(v) \le 4$ for all $v \in V(G) \setminus V(C)$, and
  	\item $EE(X) \ne EO(X)$.
  \end{enumerate}
\end{conjecture}

Suppose the conjecture is false and choose a counterexample $G$ minimizing $|G|$ and subject to that, minimizing $|C|$.

\begin{lem}
	$G$ has no clique cutset.
\end{lem}
\begin{proof}
	Let $S \subseteq V(G)$ be a minimal cutset.  Then $|S| \le 4$.  If $|S| \le 2$, we are done immediately by applying minimality to the lobes and patching the orientations together.  The $|S| = 3$ case implies that $G$ contains no $K_4$.  So, all we need to do
	is show there is no triangle cutset.  Say $S = \set{a,b,c}$.  We apply minimality to each $S$-lobe of $G$.  For the lobe containing the interior of $abc$ we use $abc$ as the outer face.  For the other lobe we use $C$.  Let $X$ be the resulting orientation of the outer lobe.  Suppose $X$ does not orient $abc$ cyclically.  Then, by symmetry, we may assume that $ab, ac, bc \in E(X)$.  For the inner lobe, apply minimality with $abc$ as the other face and using edge $bc$, let $Y$ be the resulting orientation of the inner lobe minus $bc$.  Then $ab,ac \in E(Y)$.  Adding $bc$ back in does not change the Eulerian subgraph counts since $c$ is a sink in $Y$.  But now $X$ and $Y$ give the same orientation to $abc$, so we can patch them together to get an orientation $Q$ of $G-x_1x_2$.  Since all edges in $Y$ incident to $a,b,c$ point into $a,b,c$, our patching did not create any new directed cycles.  So the spanning Eulerian subgraphs of $Q$ are all pairings of spanning Eulerian subgraphs in $X$ and in $Y$.  Since $EE(X) \ne EO(X)$ and $EE(Y) \ne EO(Y)$, we conclude $EE(Q) \ne EO(Q)$.  Also, our patching did not change the out-degree of any vertex, so the degree condition is still satisfied.
	
	If $X$ does orient $abc$ cyclically, then we won't be able to match up the two orientations.  But, we can change $X$ to another $\alpha$-orientation (where $\alpha$ is the out-degree sequence of $X$) that does not orient $abc$ cyclically.
	Let $Z$ be the induced subgraph of $X$ on $\set{a,b,c}$.  Then $EE(Z) = EO(Z)$, so by Lemma \ref{AtLeastTwoCycles}, $X$ has an induced cycle $A \not \subseteq Z$ containing an edge of $Z$.  Then $A$ contains at most two edges of $Z$, so reversing all the edges on $A$ produces a new $\alpha$-orientation $X'$ where $abc$ is not oriented cyclically.  By Lemma \ref{AlphaOrientationsSameEulerian}, we have $0 \ne EE(X) - EO(X) = \pm\parens{EE(X') - EO(X')}$.  So we can use $X'$ in place of $X$ in the argument in the previous paragraph to conclude that $G-x_1x_2$ has the desired orientation, a contradiction.
\end{proof}

\begin{lem}
	$|C| \ge 4$.
\end{lem}
\begin{proof}
	The Thomassen-style argument goes through when $|C| = 3$.
\end{proof}

\bibliographystyle{amsplain}
\bibliography{GraphColoring1}

\end{document}

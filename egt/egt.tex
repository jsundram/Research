\documentclass{amsbook}
\usepackage{amsfonts}
\usepackage{marginnote}

\newcommand{\aside}[1]{\marginnote{\scriptsize{#1}}[0cm]}
\newcommand{\aaside}[2]{\marginnote{\scriptsize{#1}}[#2]}

\theoremstyle{plain}
\newtheorem{acknowledgement}{Acknowledgement}
\newtheorem{algorithm}{Algorithm}
\newtheorem{axiom}{Axiom}
\newtheorem{case}{Case}
\newtheorem{claim}{Claim}
\newtheorem{conclusion}{Conclusion}
\newtheorem{condition}{Condition}
\newtheorem{conjecture}{Conjecture}
\newtheorem{corollary}{Corollary}
\newtheorem{criterion}{Criterion}
\newtheorem{definition}{Definition}
\newtheorem{example}{Example}
\newtheorem{exercise}{Exercise}
\newtheorem{lemma}{Lemma}
\newtheorem{notation}{Notation}
\newtheorem{problem}{Problem}
\newtheorem{proposition}{Proposition}
\newtheorem{remark}{Remark}
\newtheorem{solution}{Solution}
\newtheorem{summary}{Summary}
\newtheorem{theorem}{Theorem}
\numberwithin{equation}{chapter}

\newcommand{\set}[1]{\left\{ #1 \right\}}
\newcommand{\setb}[3]{\left\{ #1 \in #2 : #3 \right\}}
\newcommand{\setbs}[2]{\left\{ #1 : #2 \right\}}
\newcommand{\card}[1]{\left|#1\right|}
\newcommand{\size}[1]{\left\Vert#1\right\Vert}
\newcommand{\ceil}[1]{\left\lceil#1\right\rceil}
\newcommand{\floor}[1]{\left\lfloor#1\right\rfloor}
\newcommand{\func}[3]{#1\colon #2 \rightarrow #3}
\newcommand{\funcinj}[3]{#1\colon #2 \inj #3}
\newcommand{\funcsurj}[3]{#1\colon #2 \surj #3}
\newcommand{\irange}[1]{\left[#1\right]}
\newcommand{\join}[2]{#1 \mbox{\hspace{2 pt}$\ast$\hspace{2 pt}} #2}
\newcommand{\djunion}[2]{#1 \mbox{\hspace{2 pt}$+$\hspace{2 pt}} #2}
\newcommand{\parens}[1]{\left( #1 \right)}
\newcommand{\brackets}[1]{\left[ #1 \right]}
\newcommand{\DefinedAs}{\mathrel{\mathop:}=}

\begin{document}
\frontmatter
\title[gct]{embodied graph theory}
\author{}
\maketitle
\tableofcontents

\chapter*{preface}

this comes prior to the face.

\mainmatter

\chapter*{graphs}
This book is about groups of people.  Not about individual people, but about the structure of relationships in the group.  
Each person wears a belt with many loops.  There are ropes with caribiners on both ends for attaching two people by their belt-loops.
We say that two people are \emph{joined}\aaside{joined}{} if they are connected by a rope.  A \emph{graph} is a group of \emph{people} together with some joinings.
If $G$ is a graph, then $P(G)$ is its group of people and $R(G)$ its collection of ropes. \aaside{$P(G)$, $R(G)$}{-.15in}
We write $\card{G}$ for the number of people in $P(G)$ and $\size{G}$ for the number of ropes in $R(G)$. \aaside{$\card{G}$, $\size{G}$}{}
The group of people to which $p$ is joined is her \emph{neighborhood}, written $N(v)$. \aaside{neighborhood}{+0.0in} \aaside{$N(v)$}{+0.15in}
For the size of $v$'s neighborhood $\card{N(v)}$, we write $d(v)$ and call this the \emph{degree} of $v$. \aaside{$d(v)$, degree}{+0.3in}

\chapter*{grouping people}
The entire book concerns one simple task: we want to group the people of a given graph so that joined people are in different groups.
With sufficiently many groups and no preferences about what the groupings should look like, this is easy, we just put each person in her own group.  
Things get interesting when we ask how few different groups we can use.  We are definitely going to need at least zero groups and that will only do for the
graph with no people at all.  Given one group, we can handle all graphs with no joins.  With two groups, we can do
any path and any cycle with an even number of people.  But, we can't handle a triangle or any other cycle with an odd number of people.
In fact, odd cycles are really the only thing that will prevent us from using just two groups. 
A graph $H$ is a \emph{subgraph} of a graph $G$, written $H \subseteq G$ if $P(H) \subseteq P(G)$ and $R(H) \subseteq R(G)$. \aaside{subgraph, $\subseteq$}{}
When $H \subseteq G$, we say that $G$ \emph{contains} $H$. \aaside{contains}{0.12in}  If $p \in P(G)$, then $G-p$ is the graph we get by removing $p$ from the group along with all the ropes attached to her. \aaside{$G-v$}{0.05in}

It will be convenient to have all the members of a group wear the same color shirt.  So, we might have the red shirted group and the blue shirted group, etc.
A graph is $k$-colorable if we can group its people into (at most) $k$ groups such that joined people are in different groups. \aaside{$k$-colorable}{}
A $0$-colorable graph is \emph{empty}, a $1$-colorable graph is \emph{ropeless} and a $2$-colorable graph is \emph{bipartite}. \aaside{empty}{}\aaside{ropeless}{+0.1in}\aaside{bipartite}{+0.2in}
\begin{theorem}\label{TwoColoring}
A graph is $2$-colorable just in case it contains no odd cycle.
\end{theorem}
\begin{proof}
A graph containing an odd cycle clearly can't be $2$-colored.  For the other implication, suppose
there is a graph that is not $2$-colorable and doesn't contain an odd cycle.  Then we may pick such a graph $G$ with $\card{G}$ as small as possible.
Surely, $|G| > 0$, so we may pick $v \in P(G)$.  If $x, y \in N(v)$, then $x$ is not joined to $y$ since then $xyz$ would be an odd cycle.
So we can construct a graph $H$ from $G$ by removing $v$ and identifying all of $N(v)$ to a new person $x_v$.  Any odd cycle
in $H$ would contain $x_v$ and hence give rise to an odd cycle in $G$ passing through $v$.  So $H$ contains no odd cycle. Since $|H| < |G|$, 
applying the theorem to $H$ gives a 2-coloring of $H$, say into the red and blue groups
where $x_v$ is in the red group.  But this gives a 2-coloring of $G$ by putting all people in $N(v)$ in the red group and putting $v$ in the blue group, a contradiction.
\end{proof}

Well, this is embarrassing, coloring appears to be easy.  Fortunately, things get more interesting when we move up to three colors.
\begin{theorem}
3-coloring is hard supposing other things we think are hard are actually hard.
\end{theorem}
\begin{proof}
We need a concise proof of this without having to introduce too much background.  Please submit a pull request on GitHub.
\end{proof}

\section*{basic estimates}
Even though finding the minimum number of colors needed to color a graph is hard in general (supposing it is), we can still
look for lower and upper bounds on this value.  The \emph{chromatic number} $\chi(G)$ of a graph $G$ is the smallest $k$ for which $G$ is $k$-colorable.
\aaside{chromatic number}{+0.0in}\aaside{$\chi(G)$}{+0.15in}
The simplest thing we can do is give each person a different colored shirt.
\begin{theorem}\label{WorstUpperBound}
If $G$ is a graph, then $\chi(G) \le \card{G}$.
\end{theorem}
The only graphs that attain the upper bound in Theorem \ref{WorstUpperBound} are the \emph{complete} graphs; those in which
any two people are joined. \aaside{complete}{}
We can usually do much better by just arbitrarily putting colored shirts on people, reusing colors when we can.  \aaside{maximum degree}{+0.0in}\aaside{$\Delta(G)$}{+0.15in}The \emph{maximum degree} $\Delta(G)$ of a graph $G$ is the largest degree
of any person in $G$; that is 
\[\Delta(G) \DefinedAs \max_{v \in V(G)} d(v).\]
We are going to need smooth language to talk about putting various colored shirts on people.
Let's say that ``to color a person red'' means to put a red shirt on them, etc.
\begin{theorem}\label{SecondWorstUpperBound}
If $G$ is a graph, then $\chi(G) \le \Delta(G) + 1$.
\end{theorem}
\begin{proof}
Suppose there is a graph $G$ that is not $\parens{\Delta(G) + 1}$-colorable.  Then we may pick such a graph $G$ with as few people as possible.
Surely $G$ has at least one person, so we may pick $v \in V(G)$.  Then $\card{G-v} < \card{G}$ and $\Delta(G-v) \le \Delta(G)$, so applying the theorem to $G-v$ gives a $\parens{\Delta(G-v) + 1}$-coloring
of $G-v$.  But $v$ has at most $\Delta(G)$ neighbors, so there is some color, say red, not used on $N(v)$, coloring $v$ red gives a $\parens{\Delta(G) + 1}$-coloring
of $G$, a contradiction.
\end{proof}

Both complete graphs and odd cycles attain the upper bound in Theorem \ref{SecondWorstUpperBound}.  Theorem \ref{TwoColoring} says
we can do better for graphs that don't contain odd cycles.  We can also do better for graphs that don't contain large complete subgraphs.
A group of people $S$ in a graph $G$ is a \emph{clique} if the people in $S$ are pairwise joined.\aaside{clique}{}  
The \emph{clique number} of a graph $G$, written $\omega(G)$, is the number of people in a largest clique in $G$.\aaside{$\omega(G)$}{}

\begin{theorem}\label{OmegaLowerBound}
If $G$ is a graph, then $\chi(G) \ge \omega(G)$.
\end{theorem}

A group of people $S$ in a graph $G$ is \emph{independent} if the people in $S$ are pairwise non-joined.\aaside{independent}{}  
The \emph{independence number} of a graph $G$, written $\alpha(G)$, is the number of people in a largest independent set in $G$.\aaside{$\alpha(G)$}{}

\begin{theorem}
If $G$ is a graph with $\Delta(G) \ge 3$ and $\omega(G) \le \Delta(G)$, then $\chi(G) \le \Delta(G)$.
\label{BrooksTheorem}
\end{theorem}
\begin{proof}
Suppose there is a graph $G$ with $\Delta(G) \ge 3$ and $\omega(G) \le \Delta(G)$ that is not $\Delta(G)$-colorable.  
Then we may pick such a graph $G$ with as few people as possible.  Let $S$ be 
a maximal independent set in $G$.  Since $S$ is maximal, every person in $G-S$ has a neighbor in $S$, so $\Delta(G) > \Delta(G-S)$.
If red is an unused color in a $\chi(G-S)$-coloring of $G-S$, then by coloring all people in $S$ red we get a $\parens{\chi(G-S)+1}$-coloring of $G$.  
So, $\Delta(G) + 1 \le \chi(G) \le \chi(G-S) + 1$. We conclude $\chi(G-S) > \Delta(G - S)$ and thus $\Delta(G) = \chi(G-S) = \Delta(G-S) + 1$ by Theorem \ref{SecondWorstUpperBound}.
Since $\card{G-S} < \card{G}$, applying the theorem to $G-S$ shows that $\Delta(G-S) < 3$ or $\Delta(G -S) < \omega(G - S)$.  
So, either $\chi(G-S) = \Delta(G) = 3$ or $\omega(G-S) \ge \Delta(G)$.  In the former case, let $X$ be the people group of an odd cycle in $G-S$ guaranteed by Theorem \ref{TwoColoring}.  
In the latter case, let $X$ be a $\Delta(G)$-clique in $G-S$.

Since $S$ is maximal and $\omega(G) \le \Delta(G)$, there are $x_1, x_2 \in X$ and $y_1, y_2 \in S$ such that $x_1$ is joined to $y_1$ and $x_2$ is joined to $y_2$.
Construct a graph $H$ from $G-X$ by adding the rope $y_1y_2$.  Since $\card{H} < \card{G}$, applying the theorem to $H$ shows that $\omega(H) > \Delta(G)$ or $\chi(H) \le \Delta(G)$.
Suppose $\chi(H) \le \Delta(G)$.  Then there is a $\Delta(G)$-coloring of $G-X$ where $y_1$ and $y_2$ receive different colors, say red and blue respectively.
Pick the first person $z$ in a shortest path $P$ from $x_1$ to $x_2$ in $X$ that has a blue colored neighbor in $V(H)$. 
Each person in $X$ has $\Delta(G)-1$ neighbors in $X$ and hence at most one neighbor in $V(H)$.  So, $z \ne x_1$ since $x_1$ already has a red colored neighbor in $V(H)$.
Let $w$ be be the person preceding $z$ on $P$. Then $w$ has no blue colored neighbor.  Since $X$ is the person group of a cycle or a 
complete graph, there is a path $Q$ from $w$ to $z$ passing through every person of $X$.  Color $w$ blue and then proceed along $Q$, coloring one person at a time.  
Since each person we encounter before we get to $z$ has at most $\Delta(G) - 1$ colored neighbors, we always have an available color to use.  But, $z$ is joined
to both $w$ and another blue colored person in $V(H)$, so there is an available color for $z$ as well.  This gives a $\Delta(G)$-coloring of $G$, a contradiction.

So, $\omega(H) > \Delta(G)$.  In particular, $y_1$ and $y_2$ each have exactly one neighbor in $X$ and $\Delta(G) - 1$ neighbors in the same $\Delta(G) -1$ clique $A$ in $G - X$.
 Since $S$ is maximal and $\card{X} \ge 3$, there must be joined
$x_3 \in X \setminus \set{x_1,x_2}$ and $y_3 \in S \setminus \set{y_1,y_2}$.  Applying the same argument with $x_3, y_3$ in place of $x_2, y_2$ shows
that $y_1$ and $y_3$ each have exactly one neighbor in $X$ and $\Delta(G) - 1$ neighbors in the same $\Delta(G) -1$ clique $B$ in $G - X$.
Now $\card{A\cap B} = \card{A} + \card{B} - \card{A\cup B} \ge 2(\Delta(G) - 1) - d(y_1) \ge \Delta(G) - 2 > 0$.  But there can't be a person
in $A \cap B$ since she would be joined to $y_1,y_2,y_3$ as well as $\Delta(G) - 2$ people in $A$ and thus have degree greater than $\Delta(G)$, a contradiction. 
\end{proof}
\end{document}
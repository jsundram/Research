\documentclass[12pt]{article}
\usepackage{fullpage, amssymb, amsmath, amsthm, mathabx}

\pagestyle{plain}

\theoremstyle{plain}
\newtheorem{thm}{Theorem}
\newtheorem{prop}[thm]{Proposition}
\newtheorem{lem}[thm]{Lemma}
\newtheorem{cor}[thm]{Corollary}
\newtheorem{prob}[thm]{Problem}
\newtheorem{claim}{Claim}
\newtheorem*{unnumberedClaim}{Claim}

\theoremstyle{definition}
\newtheorem{defn}{Definition}[section]
\newtheorem*{setup}{The setup}
\newtheorem*{atticus}{Atticus' move}
\newtheorem*{lazy_demon}{Lazy Demon}
\newtheorem*{contrary_demon}{Contrary Demon}
\newtheorem*{konig_demon}{K\"onig's Demon}
\newtheorem*{vizing_demon}{Vizing's Demon}
\newtheorem*{winning}{Winning}
\newtheorem*{edgecoloring}{Edge coloring}

\theoremstyle{remark}
\newtheorem*{remark}{Remark}
\newtheorem{example}{Example}
\newtheorem*{question}{Question}
\newtheorem*{observation}{Observation}

\newcommand{\fancy}[1]{\mathcal{#1}}
\newcommand{\C}[1]{\fancy{C}_{#1}}
\newcommand{\IN}{\mathbb{N}}
\newcommand{\IR}{\mathbb{R}}
\newcommand{\G}{\fancy{G}}

\newcommand{\inj}{\hookrightarrow}
\newcommand{\surj}{\twoheadrightarrow}

\newcommand{\set}[1]{\left\{ #1 \right\}}
\newcommand{\setb}[3]{\left\{ #1 \in #2 \mid #3 \right\}}
\newcommand{\setbs}[2]{\left\{ #1 \mid #2 \right\}}
\newcommand{\card}[1]{\left|#1\right|}
\newcommand{\size}[1]{\left\Vert#1\right\Vert}
\newcommand{\ceil}[1]{\left\lceil#1\right\rceil}
\newcommand{\floor}[1]{\left\lfloor#1\right\rfloor}
\newcommand{\func}[3]{#1\colon #2 \rightarrow #3}
\newcommand{\funcinj}[3]{#1\colon #2 \inj #3}
\newcommand{\funcsurj}[3]{#1\colon #2 \surj #3}
\newcommand{\irange}[1]{\left[#1\right]}
\newcommand{\join}[2]{#1 \mbox{\hspace{2 pt}$\ast$\hspace{2 pt}} #2}
\newcommand{\djunion}[2]{#1 \mbox{\hspace{2 pt}$+$\hspace{2 pt}} #2}
\newcommand{\parens}[1]{\left( #1 \right)}
\newcommand{\brackets}[1]{\left[ #1 \right]}
\newcommand{\DefinedAs}{\mathrel{\mathop:}=}

\title{Playing cards with a demon}
\author{Rachel Anderson-Rabern, Atticus Rabern, Brian Rabern, Landon Rabern}

\begin{document}
\maketitle
\begin{abstract}
We analyze a solitaire game in which a demon rearranges some cards after each move.  The graph edge coloring theorems of K\"onig and Vizing follow from the winning strategies developed.
\end{abstract}

\section{The Solitaire Game}
Atticus the solitaire master, plays the solitaire game with a fixed number of stacks of face up cards.  Call a game with $k$ stacks where the $i$-th stack has $n_i$ cards an $(n_1, \ldots, n_k)$-game.  We disallow the two trivial cases where there are no stacks and where a stack has no cards; that is, we require $k \geq 1$ and $n_i \geq 1$ for each $i$.

To set up an $(n_1, \ldots, n_k)$-game, the demon chooses a number $N$ at least $k$ and creates a deck with $k$ $1$-cards, $k$ $2$-cards, \ldots, and $k$ $N$-cards. Then the demon makes $k$ stacks where the $i$-th stack has $n_i$ differently numbered cards from the deck.

With the game set up, Atticus can repeatedly make moves of the following form.

\begin{atticus}
Pick some stack containing an $a$-card but no $b$-card and then swap
the $a$-card for a $b$-card from the deck.
\end{atticus}

Atticus' goal in the game is simply the following.

\begin{winning}
Atticus wins if at the start of his turn he can make a hand of $k$ differently numbered cards by picking one card from each stack.
\end{winning}

To make the game harder for Atticus, the demon will rearrange some cards after each of Atticus' moves.  Whether or not Atticus has a winning strategy will depend on how the demon chooses to rearrange the cards.  Consider for example the following extreme demons.

\begin{lazy_demon}
After Atticus' turn, Lazy Demon does nothing.
\end{lazy_demon}

Can Atticus win against this demon?  Sure he can -- all he has to do is go through the stacks in order, swapping an $i$-card into the $i$-th stack if there is not one there already.  

\begin{contrary_demon}
Say Atticus' swapped an $a$-card out of and a $b$-card into the $i$-th stack.  Contrary Demon undoes what Atticus just did; that is, he swaps a $b$-card out of and an $a$-card into the $i$-th stack.
\end{contrary_demon}

Now Atticus cannot ever change the stacks and so only wins if the demon gave him a winning position to start with.

\section{K\"onig's Demon}

\begin{konig_demon}
Say Atticus' swapped an $a$-card out of and a $b$-card into the $i$-th stack.
K\"onig's Demon either does nothing or picks a stack other than the $i$-th containing a $b$-card but no $a$-card and swaps the $b$-card for an $a$-card from the deck.
\end{konig_demon}

\begin{thm}
Atticus has a winning strategy against K\"onig's Demon for any $\parens{n_1, \ldots, n_k}$-game.
\end{thm}
\begin{proof}
Suppose the demon has set up an $(n_1, \ldots, n_k)$-game.  Atticus should consider the largest hand of differently numbered cards he can make by picking one card from each stack.  If he can make a hand of size $k$, he wins.  Otherwise, there is some stack, say the $i$-th, from which he is not picking a card and some number $b \leq N$ not appearing on any card in his hand.  Since $n_i \geq 1$, there is at least one card in the $i$-th stack, say an $a$-card.  Now Atticus can swap an $a$-card out of and a $b$-card into the $i$-th stack.  Atticus can make a larger hand by picking the $b$-card from the $i$-th stack.  Since the hand uses a $b$-card from only the $i$-th stack, the demon swapping out a $b$-card in another stack cannot decrease the size of Atticus' hand.  Repeating this process, Atticus ends up with a hand of size $k$ and wins.
\end{proof}

\section{Vizing's Demon}

\begin{vizing_demon}
Say Atticus' swapped an $a$-card out of and a $b$-card into the $i$-th stack.
Vizing's Demon either does nothing or picks a stack other than the $i$-th, containing a $b$-card but no $a$-card and swaps the $b$-card for an $a$-card from the deck, or picks a stack other than the $i$-th containing an $a$-card but no $b$-card and swaps the $a$-card for a $b$-card from the deck.
\end{vizing_demon}

\begin{thm}
Atticus has a winning strategy against Vizing's Demon for any $\parens{n_1, \ldots,
n_k}$-game where at most one of the $n_i$ is $1$.
\end{thm}

Atticus' winning strategy will involve reducing a game with $k$ stacks to one with $k - 1$ stacks. We break out the stretegy for reducing the number of stacks into the following lemma.

\begin{lem}\label{Reduction}
For any $\parens{n_1, \ldots, n_k}$-game against Vizing's Demon where at most one of the $n_i$ is $1$, Atticus can either get to a winning position or to a position where, for some $c$, exactly one stack contains a $c$-card.
\end{lem}

Using Lemma \ref{Reduction}, Atticus' strategy is as follows.  Ignore the stack with the $c$-card and play the game on the rest of the stacks never swapping in a $c$-card.  Using Using Lemma \ref{Reduction} again, we have $c_2 \neq c$ that appears on only one stack in this game, ignore that stack as well and play the game on the rest of the stacks never swapping in a $c_2$-card.  Repeating this we eventually get down to a single stack.  Pick a card, say a $c_k$-card, from this stack.  Now Atticus wins by choosing the hand with cards numbered $c, c_2, \ldots, c_k$.  Therefore it just remains to prove the lemma.

\begin{proof}[Proof of Lemma \ref{Reduction}]
\end{proof}

\section{Edge Coloring}

\bibliographystyle{amsplain}
\bibliography{demon}
\end{document}

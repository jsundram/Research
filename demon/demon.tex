\documentclass[12pt]{article}
\usepackage{fullpage, amssymb, amsmath, amsthm, mathabx}

\pagestyle{plain}

\theoremstyle{plain}
\newtheorem{thm}{Theorem}
\newtheorem{prop}[thm]{Proposition}
\newtheorem{lem}[thm]{Lemma}
\newtheorem{cor}[thm]{Corollary}
\newtheorem{prob}[thm]{Problem}
\newtheorem{claim}{Claim}
\newtheorem*{unnumberedClaim}{Claim}
\newtheorem*{beattame}{Beating tame demons}
\newtheorem*{vizing}{Vizing's theorem\footnote{Vizing \cite{vizing1965chromatic}}}

\theoremstyle{definition}
\newtheorem{defn}{Definition}[section]
\newtheorem*{setup}{The setup}
\newtheorem*{atticus}{Atticus' move}
\newtheorem*{demon}{The demon's move}
\newtheorem*{winning}{Winning}
\newtheorem*{tame}{Tame demons}
\newtheorem*{edgecoloring}{Edge coloring}

\theoremstyle{remark}
\newtheorem*{remark}{Remark}
\newtheorem{example}{Example}
\newtheorem*{question}{Question}
\newtheorem*{observation}{Observation}

\newcommand{\fancy}[1]{\mathcal{#1}}
\newcommand{\C}[1]{\fancy{C}_{#1}}
\newcommand{\IN}{\mathbb{N}}
\newcommand{\IR}{\mathbb{R}}

\newcommand{\inj}{\hookrightarrow}
\newcommand{\surj}{\twoheadrightarrow}

\newcommand{\set}[1]{\left\{ #1 \right\}}
\newcommand{\setb}[3]{\left\{ #1 \in #2 \mid #3 \right\}}
\newcommand{\setbs}[2]{\left\{ #1 \mid #2 \right\}}
\newcommand{\card}[1]{\left|#1\right|}
\newcommand{\size}[1]{\left\Vert#1\right\Vert}
\newcommand{\ceil}[1]{\left\lceil#1\right\rceil}
\newcommand{\floor}[1]{\left\lfloor#1\right\rfloor}
\newcommand{\defic}[1]{\text{def}(#1)}
\newcommand{\func}[3]{#1\colon #2 \rightarrow #3}
\newcommand{\irange}[1]{\left[#1\right]}

\title{Playing cards with a demon}
\author{Rachel Anderson-Rabern, Brian Rabern, Landon Rabern}

\begin{document}
\maketitle
\begin{abstract}
We introduce a two-player card game and prove sufficient conditions for the first player to have a winning strategy.  Vizing's edge coloring theorem is a consequence.
\end{abstract}

\section{The card game}
For $k \geq 1$ and each tuple $(n_1, \ldots, n_k) \in \set{1, \ldots, k}^k$ define a two-player card game between our hero Atticus and a demon as follows.

\begin{setup}
There is an unlimited supply of cards each with one of $1, 2, \ldots, k$ on them. In front of the two players, there are $k$
stacks of cards $\set{S_1, \ldots, S_k}$ where $S_i$ contains $n_i$ pairwise distinct cards.
\end{setup}

\begin{atticus}
He picks a stack $S_i$ as well as a number $a$ that appears in $S_i$ and a number $b$ that does not.  Then he discards the $a$-card and draws a $b$-card, putting it in $S_i$.
\end{atticus}

\begin{demon}
Either passes his turn or picks a stack $S_j$ other than $S_i$ and replaces all $a$-cards and $b$-cards in $S_j$ with $b$-cards and $a$-cards respectively.
\end{demon}

\begin{winning}
Atticus wins if at the start of his turn he can form a run $1, 2, \ldots, k$ by choosing one card from each stack.  The demon wins if Atticus never does.
\end{winning}

\section{Beating tame demons}
\begin{tame}
A demon $D$ is \emph{tame} if whenever Atticus has a winning strategy against $D$ for the $(n_1, \ldots, n_k)$-game then he has a winning strategy against $D$ for any $(m_1, \ldots, m_k)$-game where $m_i \geq n_i$.
\end{tame}

\begin{beattame}
Atticus has a winning strategy for the $(n_1, \ldots, n_k)$-game against a tame demon if $n_1 \geq 1$ and $n_i \geq 2$ for $i \geq 2$.
\end{beattame}
\begin{proof}
We prove the theorem by induction on $k$.  For $k=1$, Atticus always wins on his first turn, so we may assume that $k \geq 2$.  Since our demons are tame, we may also assume that $\sum_i n_i = 2k-1$.

For a collection of stacks $P = \set{S_1, \ldots, S_k}$ and $i \in \set{1, \ldots, k}$, let $P_i$ be the stacks of $P$ that contain an $i$-card.  Atticus's strategy will be to maximize $\gamma(P) = \sum_i \card{P_i}^2$.

Say it is Atticus's turn and the current stack collection is $P$.  Assume $\card{P_i} \neq 1$ for all $i$.  Then, as $\sum_i \card{P_i} = 2k-1$, we have $r$ such that $\card{P_r} = 0$.  Also, since $\sum_i \card{P_i}$ is odd we have $t$ such that $\card{P_t} \geq 3$.  Pick $S \in P$ that contains a $t$-card.  Now Atticus will replace the $t$-card with an $r$-card in $S$ to get stacks $P'$.  Since none of the other stacks have an $r$-card, after the demon's move we have $\card{P'_r} \geq 1$ and $\card{P'_t} \leq 2$.  In particular,  $\gamma(P') > \gamma(P)$.  Since the game has only finitely many states, at some point it must be Atticus's turn with stacks $Q$ such that $\card{Q_i} = 1$ for some $i$.

Let $S$ be the stack in $Q$ with the $i$-card and put $Q' = Q - \set{S}$.  Then $Q'$ satisfies hypotheses of the theorem for $k-1$ with card labels $\set{1, \ldots, i - 1, i + 1, \ldots, k}$.  Hence, by induction, Atticus can win the game played on $Q'$.  Doing so and then using the $i$-card from $S$ gives the desired $1, 2, \ldots, k$ run.  Hence Atticus has a winning strategy.\footnote{This proof is essentially the same as Schrijver's proof of Vizing's theorem \cite{schrijver2003combinatorial} which is rooted in the proof of Ehrenfeucht, Faber and Kierstead \cite{Ehrenfeucht1984159}.}
\end{proof}

\section{Edge coloring}
\begin{edgecoloring}
An \emph{edge coloring} of a simple graph is an assignment of colors to its edges such that no pair of incident edges receive the same color.
\end{edgecoloring}

\begin{vizing}
Every simple graph $G$ has an edge coloring using at most $\Delta(G) + 1$ colors.
\end{vizing}
\begin{proof}
Assume not and let $G$ be a counterexample minimizing $\card{G}$. Pick $v \in V(G)$ with degree $\Delta(G)$, say $v_1, \ldots, v_k$ are the neighbors of $v$ in $G$. By minimality of $\card{G}$, we have a $(k + 1)$-edge-coloring of $G - v$.  Let $S_i$ be the colors not incident with $v_i$ in this coloring.  Each $v_i$ has degree at most $k-1$ in $G - v$ and hence $\card{S_i} \geq 2$.  Also, if $a \in S_i$ and $b \not \in S_i$ we may exchange colors on a maximum length path starting at $v_i$ and alternating between colors $b$ and $a$.  But this just gives a Atticus move followed by a demon move.  By adding dummy pendant edges to the $v_i$ if necessary, we see that the demon is tame.  Hence Atticus can win and the won state tells us how to extend the $(k+1)$-edge-coloring to all of $G$.
\end{proof}
\newpage

\bibliographystyle{amsplain}
\bibliography{demon}
\end{document}

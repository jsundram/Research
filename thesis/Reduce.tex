\chapter{REDUCING MAXIMUM DEGREE}\label{ReductionChapter}

\begin{center}
\emph{Some of the material in this chapter appeared in
\cite{rabernhitting}.}
\end{center}

\section{Hitting all maximum cliques}
As part of his proof that every graph with $\chi \geq \Delta$ contains a $K_{\Delta - 28}$, Kostochka proved the following lemma.

\begin{lem}[Kostochka \cite{kostochkaRussian}]\label{KostochkaHitting}
If $G$ is a graph satisfying $\omega \geq \Delta + \frac32 - \sqrt{\Delta}$, then $G$ contains an independent set $I$ such that $\omega(G - I) < \omega(G)$.
\end{lem}

To talk about the proof we first need a definition.

\begin{CliqueGraph}
Let $G$ be a graph. For a collection of cliques $\mathcal{Q}$ in $G$, let $X_{\mathcal{Q}}$ be the intersection graph of $\mathcal{Q}$.  That is, the vertex set of $X_{\mathcal{Q}}$ is $\mathcal{Q}$ and there is an edge between $Q_1 \neq Q_2 \in \mathcal{Q}$ iff $Q_1$ and $Q_2$ intersect.
\end{CliqueGraph}

Kostochka's proof proceeded in two stages.  First show that the vertices in each component of the clique graph have a large intersection.  Then find an independent transversal of these intersections.  Such a transversal is an independent set hitting every maximum clique.  Kostochka used a custom method to find a transversal.  In \cite{rabernhitting}, we applied the following lemma of Haxell \cite{haxell2001note} (proved long after Kostochka's paper) to find the independent transversal.

\begin{lem}\label{HaxellLemma}
Let $H$ be a graph and $V_1 \cup \cdots \cup V_r$ a partition of $V(H)$. Suppose that $\card{V_i} \geq 2\Delta(H)$ for each $i \in \irange{r}$. Then $H$ has an independent set $\set{v_1, \ldots, v_n}$ where $v_i \in V_i$ for each $i \in \irange{r}$.
\end{lem}
Finding the independent transversal using this lemma gives the following.

\begin{lem}[Rabern \cite{rabernhitting}]\label{RabernHitting}
If $G$ is a graph satisfying $\omega \geq \frac{3}{4}\parens{\Delta + 1}$, then $G$ contains an independent set $I$ such that $\omega(G - I) < \omega(G)$.
\end{lem}

Aharoni, Berger and Ziv \cite{aharoni2007independent} showed that Haxell's proof actually gets more than Lemma \ref{HaxellLemma}.  
From their extension, King \cite{KingHitting} proved the following lopsided version of Haxell's lemma.

\begin{lem}[King \cite{KingHitting}]\label{LopsidedISR}
Let $G$ be a graph partitioned into $r$ cliques $V_1, \ldots, V_r$.  If there exists $k \geq 1$ such that for each $i$ every $v \in V_i$ has at most $\min\{k, \card{V_i} - k\}$ neighbors outside $V_i$, then $G$ contains an independent set with $r$ vertices.
\end{lem}

Using this gives the best possible form of the lemma.

\begin{lem}[King \cite{KingHitting}]\label{HittingMaxCliques}
If $G$ is a graph satisfying $\omega > \frac23 (\Delta + 1)$, then $G$ contains
an independent set $I$ such that $\omega(G - I) < \omega(G)$.
\end{lem}

\subsection{A simple proof of Kostochka's first stage}
The proofs for Kostochka's first stage can be made much simpler than the originals and we do so here.

\begin{lem}[Hajnal \cite{HajnalSaturation}]\label{HajnalLemma}
Let $G$ be a graph and $\mathcal{Q}$ a collection of maximum cliques in $G$. Then
\[\card{\bigcup \mathcal{Q}} + \card{\bigcap \mathcal{Q}} \geq 2\omega(G).\]
\end{lem}
\begin{proof}
Suppose the lemma is false and let $\mathcal{Q}$ be a counterexample with $|\mathcal{Q}|$ minimal.  Put $r \DefinedAs \card{\mathcal{Q}}$ and say $\mathcal{Q} = \set{Q_1, \ldots, Q_r}$.  

Consider the set $W \DefinedAs (Q_1 \cap \bigcup_{i=2}^r Q_i) \cup \bigcap_{i=2}^r Q_i$.  Plainly, $W$ is a clique.  Thus we may derive a contradiction as follows.
\begin{align*}
\omega(G) &\geq |W| \\
&= \card{(Q_1 \cap \bigcup_{i=2}^r Q_i) \cup \bigcap_{i=2}^r Q_i} \\
&= \card{Q_1 \cap \bigcup_{i=2}^r Q_i} + \card{\bigcap_{i=2}^r Q_i} - \card{\bigcap_{i=1}^r Q_i \cap \bigcup_{i=2}^r Q_i} \\
&= \card{Q_1} +\card{\bigcup_{i=2}^r Q_i} - \card{\bigcup_{i=1}^r Q_i} + \card{\bigcap_{i=2}^r Q_i} - \card{\bigcap_{i=1}^r Q_i} \\
&= \omega(G) +\card{\bigcup_{i=2}^r Q_i} + \card{\bigcap_{i=2}^r Q_i} - \card{\bigcup_{i=1}^r Q_i} - \card{\bigcap_{i=1}^r Q_i} \\
&\geq \omega(G) + 2\omega(G) - \left(\card{\bigcup_{i=1}^r Q_i} + \card{\bigcap_{i=1}^r Q_i}\right) \\
& > \omega(G).
\end{align*}
\end{proof}

\begin{lem}[Kostochka \cite{kostochkaRussian}]\label{KostochkaCliqueGraph}
If $\fancy{Q}$ is a collection of maximum cliques in a graph $G$ with 
$\omega(G) > \frac23 (\Delta(G) + 1)$ such that $X_{\fancy{Q}}$ is connected, then $\cap \fancy{Q} \neq \emptyset$. 
\end{lem}
\begin{proof}
Suppose not and choose a counterexample $\fancy{Q} \DefinedAs \set{Q_1, \ldots, Q_r}$ minimizing $r$. Plainly, $r \geq 3$. Let $A$ be a noncutvertex in $X_{\fancy{Q}}$ and $B$ a neighbor of $A$. Put $\fancy{Z} \DefinedAs Q - \set{A}$. Then $X_{\fancy{Z}}$ is connected and hence by minimality of $r$, $\cap \fancy{Z} \neq \emptyset$. In particular, $\card{\cup \fancy{Z}} \leq \Delta(G) + 1$. Hence $\card{\cup \fancy{Q}} \leq \card{\cup \fancy{Z}} + \card{A - B} \leq 2(\Delta(G) + 1) - \omega(G) < 2\omega(G)$. This contradicts Hajnal's lemma.
\end{proof}

With a little more work we can prove the following generalization of Kostochka's lemma which has a Helly feel.  
We won't use this result here, but it has some independent interest.  The
following example from King \cite{KingPersonalKostochkaGeneralized} shows that
the condition $\omega > \frac{k+1}{2k+1} (\Delta + 1)$ is tight.  Take
$\join{K_{k+1}}{E_{k+1}}$ and remove a perfect matching between $K_{k+1}$ and
$E_{k+1}$ to get a graph $H$.  Then $\omega(H) = k + 1$ and $\Delta(H) = 2k$ and
thus $\omega(H) = \frac{k+1}{2k+1} (\Delta(H) + 1)$.  Taking $\mathcal{Q}$ to be
all $(k+1)$-cliques containing a vertex in the $E_{k+1}$, we see that $\cap
\mathcal{Q} = \emptyset$ but any $k$ elements of $\mathcal{Q}$ have common
intersection.

\begin{lem}\label{KostochkaGeneralized}
Fix $k \geq 2$. Let $G$ be a graph satisfying $\omega > \frac{k+1}{2k+1} (\Delta + 1)$.  
If $\mathcal{Q}$ is a collection of maximum cliques in $G$ such that any $k$ elements of $\mathcal{Q}$ have common intersection, then $\cap \mathcal{Q} \neq \emptyset$.
\end{lem}
\begin{proof}
Suppose not and choose a counterexample $\fancy{Q} \DefinedAs \set{Q_1, \ldots, Q_r}$ minimizing $r$. Plainly, $r \geq k + 1$.  Put $\mathcal{Z}_i \DefinedAs \mathcal{Q} - \set{Q_i}$.  Then any $k$ elements of $\mathcal{Z}_i$ have common intersection and hence by minimality $\cap \mathcal{Z}_i \neq \emptyset$. In particular $\cup \mathcal{Z}_i$ contains a universal vertex and thus $\card{\cup \mathcal{Z}_i} \leq \Delta(G) + 1$. Now, by Hajnal's Lemma, $\card{\cap \mathcal{Z}_i} \geq 2\omega(G) - (\Delta(G) + 1) > 2\omega(G) - \frac{2k+1}{k+1} \omega(G) = \frac{1}{k+1}\omega(G)$.

Put $m \DefinedAs \min_i \card{Q_i - \cup  \mathcal{Z}_i}$.  Note that the $\cap Z_i$ are pairwise disjoint since $\cap \mathcal{Q} = \emptyset$. Thus $\cup \mathcal{Q}$ contains the disjoint union of the $\cap Z_i$ as well as at least $m$ vertices in each clique outside the rest. In particular,

\[\card{\cup \mathcal{Q}} \geq \frac{1}{k+1}\omega(G) r + mr \geq \omega(G) + (k+1)m.\]

\noindent In addition, 

\[\card{\cup \mathcal{Q}} \leq m + \Delta(G) + 1.\]

\noindent Hence,

\[m \leq \frac{\Delta(G) + 1 - \omega(G)}{k} < \frac{1}{k+1}\omega(G).\]

\noindent Finally,

\[\card{\cup \mathcal{Q}} \leq m + \Delta(G) + 1 < \frac{1}{k+1}\omega(G) + \frac{2k+1}{k+1}\omega(G) = 2\omega(G).\]

\noindent Applying Hajnal's Lemma gives a contradiction.
\end{proof}

\subsection{Independent transversals}
In \cite{haxell2006odd}, Haxell and Szab{\'o} develop a technique for
dealing with independent transversals.  In \cite{haxell2011forming}, Haxell used
this technique to give simpler proof of her Lemma. The proof gives a bit more
as the following lemma shows.  This is just slightly more general than
the extension given in \cite{aharoni2007independent} by Aharoni, Berger and
Ziv and either gives enough to prove King's lopsided version of Haxell's lemma.  We write $\funcsurj{f}{A}{B}$ for a surjective function from $A$ to $B$.  Let $G$ be a graph.  For a $k$-coloring $\funcsurj{\pi}{V(G)}{\irange{k}}$ of $G$ and a subgraph $H$ of $G$ we say that $I \DefinedAs \set{x_1, \ldots, x_k} \subseteq V(H)$ is an $H$-independent transversal of $\pi$ if $I$ is an independent set in $H$ and $\pi(x_i) = i$ for all $i \in \irange{k}$.

\begin{lem}\label{BaseTransversalLemma}
Let $G$ be a graph and $\funcsurj{\pi}{V(G)}{\irange{k}}$ a proper $k$-coloring of
$G$.  Suppose that $\pi$ has no $G$-independent transversal, but for every $e
\in E(G)$, $\pi$ has a $(G-e)$-independent transversal. Then for every $xy \in
E(G)$ there is $J \subseteq \irange{k}$ with $\pi(x), \pi(y) \in J$ and an 
induced matching $M$ of $G\brackets{\pi^{-1}(J)}$ with $xy \in M$ such that
\begin{enumerate}
  \item $\bigcup M$ totally dominates $G\brackets{\pi^{-1}(J)}$,
  \item the multigraph with vertex set $J$ and an edge between $a, b \in J$ for
  each $uv \in M$ with $\pi(u) = a$ and $\pi(v) = b$ is a (simple) tree.  In
  particular $\card{M} = \card{J} - 1$.
\end{enumerate}
\end{lem}
\begin{proof}
Suppose the lemma is false and choose a counterexample $G$ with
$\funcsurj{\pi}{V(G)}{\irange{k}}$ so as to minimize $k$.  Let $xy \in E(G)$.
By assumption $\pi$ has a $(G-xy)$-independent transversal $T$.  Note that we
must have $x,y \in T$ lest $T$ be a $G$-independent transversal of $\pi$.

By symmetry we may assume that $\pi(x) = k-1$ and $\pi(y) = k$. Put $X
\DefinedAs \pi^{-1}(k-1)$, $Y \DefinedAs \pi^{-1}(k)$ and $H \DefinedAs G -
N(\set{x, y}) - E(X,Y)$. Define $\func{\zeta}{V(H)}{\irange{k-1}}$ by $\zeta(v)
\DefinedAs \min\set{\pi(v), k-1}$. Note that since $x,y \in T$, we have
$\card{\zeta^{-1}(i)} \geq 1$ for each $i \in \irange{k-2}$.  Put $Z \DefinedAs
\zeta^{-1}(k-1)$. Then $Z \neq \emptyset$ for otherwise $M \DefinedAs \set{xy}$
totally dominates $G[X \cup Y]$ giving a contradiction.

Suppose $\zeta$ has an $H$-independent transversal $S$.  Then we have $z \in S
\cap Z$ and by symmetry we may assume $z \in X$.  But then $S \cup \set{y}$ is
a $G$-independent transversal of $\pi$, a contradiction.

Let $H' \subseteq H$ be a minimal spanning subgraph such that $\zeta$ has no
$H'$-independent transversal.  Now $d(z) \geq 1$ for each $z \in Z$ for
otherwise $T - \set{x,y} \cup \set{z}$ would be an $H'$-independent transversal
of $\zeta$.  Pick $zw \in E(H')$.  By minimality of $k$, we have $J \subseteq
\irange{k-1}$ with $\zeta(z), \zeta(w) \in J$ and an induced matching $M$ of
$H'\brackets{\zeta^{-1}(J)}$ with $zw \in M$ such that
\begin{enumerate}
  \item $\bigcup M$ totally dominates $H'\brackets{\zeta^{-1}(J)}$,
  \item the multigraph with vertex set $J$ and an edge between $a, b \in J$ for
  each $uv \in M$ with $\zeta(u) = a$ and $\zeta(v) = b$ is a (simple) tree.
\end{enumerate}

Put $M' \DefinedAs M \cup \set{xy}$ and $J' \DefinedAs J \cup \set{k}$.
Since $H'$ is a spanning subgraph of $H$, $\bigcup M$ totally dominates
$H\brackets{\zeta^{-1}(J)}$ and hence $\bigcup M'$ totally dominates
$G\brackets{\pi^{-1}(J')}$.  Moreover, the multigraph in (2) for $M'$ and $J'$
is formed by splitting the vertex $k-1 \in J$ in two vertices and adding an edge
between them and hence it is still a tree.  This final contradiction proves the
lemma.
\end{proof}


\begin{replem}{LopsidedISR}[King \cite{KingHitting}]
Let $H$ be a graph partitioned into $k$ cliques $V_1, \ldots, V_k$.  If there
exists $r \geq 1$ such that for each $i$ every $v \in V_i$ has at most
$\min\set{r, \card{V_i} - r}$ neighbors outside $V_i$, then $G$ contains an
independent set with $k$ vertices.
\end{replem}
\begin{proof}
Suppose not and choose a counterexample $H$ minimizing $\size{H}$.  Remove all
the edges from each $V_i$ to form a graph $G$. Pick $xy \in E(G)$ and apply
Lemma \ref{BaseTransversalLemma} on $xy$ with $\funcsurj{\pi}{V(G)}{\irange{r}}$ given by $\pi(V_i) = i$ to get the guaranteed $J \subseteq \irange{r}$ and induced matching $M$.  Note that by our assumption,
the ends of an edge from $V_i$ to $V_j$ together dominate at most $\min\set{|V_i|, |V_j|}$
vertices.  Let $T$ be the tree with vertex set $J$ and an edge between $a, b \in
J$ for each $uv \in M$ with $\pi(u) = a$ and $\pi(v) = b$.  Choose a root $c$ of
$T$. Traversing $T$ in leaf-first order and for each leaf $a$ with
parent $b$ picking $|V_a|$ from $\min\set{|V_a|, |V_b|}$ we get that the vertices in $M$ together dominate
at most $\sum_{i \in J - c} \card{V_i}$ vertices, a contradiction.
\end{proof}

\subsection{Putting it all together}
Now we are in a position to prove Lemma \ref{HittingMaxCliques}.

\begin{replem}{HittingMaxCliques}[King \cite{KingHitting}]
If $G$ is a graph satisfying $\omega > \frac23 (\Delta + 1)$, then $G$ contains
an independent set $I$ such that $\omega(G - I) < \omega(G)$.
\end{replem}
\begin{proof}
Put $\Delta \DefinedAs \Delta(G)$ and $\omega \DefinedAs \omega(G)$. Let $\fancy{Q}$ be all the maximum cliques in $G$ and $\fancy{Q}_1, \ldots, \fancy{Q}_k$ the vertex sets of the components of $X_{\fancy{Q}}$.  Since the components of $X_{\fancy{Q}}$ satisfy the hypotheses of Lemma \ref{KostochkaCliqueGraph}, we have $F_i \DefinedAs \cap \fancy{Q}_i \neq \emptyset$ for all $i \in \irange{k}$.  Put $D_i \DefinedAs \cup \fancy{Q}_i$. 

Put $r \DefinedAs \frac13(\Delta + 1)$.  Note that the vertices in $F_i$ are universal in $D_i$.  Since $\card{D_i} \geq \omega > \frac23 (\Delta + 1)$, each $v \in F_i$ has at most $r$ neighbors in the rest of the $F_j$. Applying Lemma \ref{HajnalLemma} gives $\card{F_i} + \card{D_i} \geq 2\omega > \frac{4}{3}(\Delta + 1)$.  Thus each $v \in F_i$ has at most $\Delta + 1 - \card{D_i} > \Delta + 1 - (\frac{4}{3}(\Delta + 1) - \card{F_i}) = \card{F_i} - r$ neighbors in the rest of the $F_j$. Applying Lemma \ref{LopsidedISR} gives an independent set intersecting each $F_i$ and hence every maximum clique in $G$.
\end{proof}

\section{Example reductions}
\subsection{The quintessential reduction example}
Reed \cite{reed1998omega} has conjectured that every graph satisfies
\[\chi \leq \ceil{\frac{\omega + \Delta + 1}{2}}.\]

If we could always find an independent set whose removal decreased both $\omega$ and $\Delta$, then the conjecture would follow by simple induction since we can give the independent set a single color and use at most $\ceil{\frac{\omega + \Delta + 1}{2}} - 1$ colors on what remains.  Expanding the independent set given by Lemma \ref{HittingMaxCliques} to a maximal one shows that this sort of argument goes through when $\omega > \frac{2}{3}(\Delta + 1)$. Thus a minimum counterexample to Reed's conjecture satisfies $\omega \leq \frac{2}{3}(\Delta + 1)$.

\subsection{Reducing for Brooks}
The proof in Chapter \ref{BrooksChapter} reduces Brooks'
theorem to the $\Delta=3$ case by an ad hoc argument. The reduction follows
from the general lemmas on hitting maximum cliques as follows. Let $G$ be a
counterexample to Brooks' theorem minimizing $\Delta(G)$. Suppose $\Delta(G) \geq 4$. We may assume $G$ is critical. If $\omega(G) < \Delta(G)$, then removing any maximal independent set from $G$ decreases $\chi(G)$ and $\Delta(G)$ both by one giving a counterexample 
with smaller $\Delta$.  Hence $\omega(G) \geq \Delta(G)$.  
But then $\omega(G) > \frac{2}{3}(\Delta(G) + 1)$ and Lemma \ref{HittingMaxCliques} gives us an independent 
set $I$ such that $\omega(G-I) < \omega(G)$.  Let $M$ be a maximal independent set containing $I$.  
Then $G-M$ is a counterexample with smaller $\Delta$.

\subsection{Reducing for Borodin-Kostochka}
More generally, we can use the facts on hitting maximum cliques to prove the following reduction lemma.

\begin{defn}
For $k, j \in \IN$, let $\C{k, j}$ be the collection of all vertex critical graphs satisfying $\chi = \Delta = k$ and $\omega < k - j$.  Put $\C{k} \DefinedAs \C{k, 0}$. Note that $\C{k, j} \subseteq \C{k, i}$ for $j \geq i$.
\end{defn}

\begin{lem}\label{InductingOnC}
Fix $k, j \in \IN$ with $k \geq 3j + 6$.  If $G \in \C{k, j}$, then there exists $H \in \C{k-1, j}$
such that $H \lhd G$. 
\end{lem}
\begin{proof}
Let $G \in \C{k, j}$. We first show that there exists a maximal independent
set $M$ such that  $\omega(G - M) < k - (j + 1)$.   If $\omega(G)
< k - (j + 1)$, then any maximal independent set will do for $M$.
Otherwise, $\omega(G) = k - (j + 1)$.  Since $k \geq 3j + 6$, we have $\omega(G) = k - (j + 1) > \frac23(k + 1) = \frac23(\Delta(G) + 1)$.  Thus by Lemma
\ref{HittingMaxCliques}, we have an independent set $I$ such that
$\omega(G - I) < \omega(G)$.  Expand $I$ to a maximal independent set
to get $M$.

Now $\chi(G - M) = k - 1 = \Delta(G - M)$, where the last equality
follows from Brooks' theorem and $\omega(G - M) < k - (j + 1) \leq k - 1$.  Since $\omega(G - M) < k - (j + 1)$, for any $(k - 1)$-critical induced subgraph $H \unlhd G - M$ we have $H \in \C{k - 1, j}$.
\end{proof}

As a consequence we get the result of Kostochka that the Borodin-Kostochka conjecture can be reduced to the $k = 9$ case.

\begin{lem}\label{HereditaryReduction}
Let $\fancy{H}$ be a hereditary graph property. For $k \geq 5$, if $\fancy{H} \cap \C{k} = \emptyset$, then $\fancy{H} \cap \C{k+1} = \emptyset$.  In particular, to prove the Borodin-Kostochka conjecture it is enough to show that $\C{9} = \emptyset$.
\end{lem}

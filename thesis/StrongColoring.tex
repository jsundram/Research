\chapter{STRONG COLORING}\label{StrongColoringChapter}

Using ideas developed for strong coloring by Haxell \cite{haxell2004strong} and by Aharoni, Berger and Ziv \cite{aharoni2007independent}, we make explicit a recoloring technique and apply it to the Borodin-Kostochka conjecture.

\section{Strong coloring}
For a positive integer $r$, a graph $G$ with $\card{G} = rk$ is called \emph{strongly $r$-colorable} if for every partition of $V(G)$ into parts of size $r$ there is a proper coloring of $G$ that uses all $r$ colors on each part.  If $\card{G}$ is not a multiple of $r$, then $G$ is strongly $r$-colorable iff the graph formed by adding $r\ceil{\frac{|G|}{r}} - |G|$ isolated vertices to $G$ is strongly $r$-colorable.  The \emph{strong chromatic number} $s\chi(G)$ is the smallest $r$ for which $G$ is strongly $r$-colorable.

Note that a strong $r$-coloring of $G$ with respect to a partition $V_1, \ldots, V_k$ of $V(G)$ with $\card{V_i} = r$ must partition $V(G)$ into $r$ independent transversals of $V_1, \ldots, V_k$. In \cite{szabo2006extremal}, Szab{\'o} and Tardos constructed partitioned graphs with part sizes $2\Delta - 1$ that have no independent transversal.  So we must have $s\chi(G) \geq 2\Delta(G)$.  It is conjectured that this bound is tight.

Haxell \cite{haxell2004strong} proved that $s\chi(G) \leq 3\Delta(G) - 1$.  Aharoni, Berger and Ziv \cite{aharoni2007independent} gave a simple proof that $s\chi(G) \leq 3\Delta(G)$.  It is this latter proof whose recoloring technique we use.  First we need a lemma allowing us to pick an independent transversal when one of the sets has only one element.

\begin{lem}\label{SingletonSetTransversal}
Let $H$ be a graph and $V_1 \cup \cdots \cup V_r$ a partition of $V(H)$. 
Suppose that $\card{V_i} \geq 2\Delta(H)$ for each $i \in \irange{r}$.  If a
graph $G$ is formed by attaching a new vertex $x$ to fewer than $2\Delta(H)$
vertices of $H$,  then $G$ has an independent set $\set{x, v_1, \ldots, v_r}$
where $v_i \in V_i$ for each $i \in \irange{r}$.
\end{lem}
\begin{proof}
Suppose not. Remove $\set{x} \cup N(x)$ from $G$ to form $H'$ with induced
partition $V_1', V_2', \ldots, V_r'$. Then $V_1', V_2', \ldots, V_r'$ has no
independent transversal since we could combine one with $x$ to get our desired
independent set in $G$. Note that $\card{V_i'} \geq 1$. 
Create a graph $Q$ by removing edges from $H'$ until it is edge minimal without
an independent transversal. Pick $yz \in E(Q)$ and apply Lemma
\ref{BaseTransversalLemma} on $yz$ with the induced partition to get the guaranteed 
$J \subseteq \irange{r}$ and the totally dominating induced matching 
$M$ with $\card{M} = \card{J} - 1$. 
Now $\card{\bigcup_{i \in J} V_i'} > 2\Delta(H)\card{J} - 2\Delta(H) =
2(\card{J} - 1)\Delta(H)$ and hence $M$ cannot dominate, a contradiction.
\end{proof}


\begin{thm}\label{StrongColorBound}
Every graph satisfies $s\chi \leq 3\Delta$.
\end{thm}
\begin{proof}
We only need to prove that graphs with $n \DefinedAs 3\Delta k$ vertices have a $3\Delta$ coloring for each $k \geq 1$.  
Suppose not and choose a counterexample $G$ minimizing $\size{G}$.  Put $r \DefinedAs 3\Delta(G)$ and let $V_1, \ldots, V_k$ be a partition 
of $G$ for which there is no acceptable coloring.  Then the $V_i$ are independent by minimality of $\size{G}$. By symmetry we may assume 
there are adjacent vertices $x \in V_1$ and $y \in V_2$. Apply minimality of $\size{G}$ to get an $r$-coloring $\pi$ of $G - xy$ 
with $\pi(V_i) = \irange{r}$ for each $i \in \irange{k}$.  We will modify $\pi$ to get such a coloring of $G$.

By symmetry, we may assume that $\pi(x) = \pi(y) = 1$.  For $2 \leq i \leq k$, let $z_i$ be the unique element of $\pi^{-1}(1) \cap V_i$ and 
put 
\[W_i \DefinedAs V_i - \setb{v}{V_i}{\pi(v) = \pi(w) \text{ for some } w \in N(z_i)}.\]  
Then $\card{W_i} \geq 2\Delta(G)$ and we may apply 
Lemma \ref{SingletonSetTransversal} to get a $G$-independent transversal $w_1, w_2, \ldots, w_k$ of $\set{x}, W_2, W_3, \ldots, W_k$.  
Define a new coloring $\zeta$ of $G$ by 


\begin{equation*}
\zeta(v) \DefinedAs 
\begin{cases}
1 & \text{if $v = w_i$} \\
\pi(w_i) & \text{if $v = z_i$} \\
\pi(v) & \text{otherwise.}
\end{cases}
\end{equation*}

\noindent Then $\zeta$ is a proper coloring of $G$ with $\zeta(V_i) = \irange{r}$ for each $i \in \irange{k}$, a contradiction.
\end{proof}

For our application we will need a lopsided version of Lemma \ref{SingletonSetTransversal} Lemma \ref{LopsidedISR}.

\begin{lem}\label{SingletonSetTransversalLopsided}
Let $H$ be a graph and $V_1 \cup \cdots \cup V_r$ a partition of $V(H)$.  
Suppose there exists $t \geq 1$ such that for each $i \in \irange{r}$ and each $v \in V_i$ we have $d(v) \leq \min\set{t, \card{V_i}-t}$.  For any $S \subseteq V(H)$ with $\card{S} < \min\set{\card{V_1}, \ldots, \card{V_r}}$, there is an independent transversal $I$ of $V_1, \ldots, V_r$ with $I \cap S = \emptyset$.
\end{lem}
\begin{proof}
Suppose the lemma fails for such an $S \subseteq V(H)$.  Put $H' \DefinedAs H - S$ and let $V_1', \ldots, V_r'$ be the induced partition of $H'$. Then there is no independent trasversal of $V_1', \ldots, V_r'$ and $\card{V_i'} \geq 1$ for each $i \in \irange{r}$. Create a graph $Q$ by removing edges from $H'$ until it is edge minimal without an independent transversal. Pick $yz \in E(Q)$ and apply Lemma 
\ref{BaseTransversalLemma} on $yz$ with the induced partition to get the guaranteed 
$J \subseteq \irange{r}$ and the tree $T$ with vertex set $J$ and an edge between $a, b \in
J$ for each $uv \in M$ with $u \in V_a'$ and $v \in V_b'$.  By our condition, for each $uv \in E(V_i, V_j)$, we have $\card{N_H(u) \cup N_H(v)} \leq \min\set{\card{V_i}, \card{V_j}}$.

Choose a root $c$ of $T$. Traversing $T$ in leaf-first order and for each leaf $a$ with parent $b$ picking $|V_a|$ from $\min\set{|V_a|, |V_b|}$ we get that the vertices in $M$ together dominate at most $\sum_{i \in J - c} \card{V_i}$ vertices in $H$.  Since $\card{S} < \card{V_c}$, $M$ cannot totally dominate $\bigcup_{i \in J} V_i'$, a contradiction.
\end{proof}

We note that the condition on $S$ can be weakened slightly.  Suppose we have ordered the $V_i$ so that $\card{V_1} \leq \card{V_2} \leq \cdots \leq \card{V_r}$.  Then for any $S \subseteq V(H)$ with $\card{S} < \card{V_2}$ such that $V_1 \not \subseteq S$, there is an independent transversal $I$ of $V_1, \ldots, V_r$ with $I \cap S = \emptyset$.  The proof is the same except when we choose our root $c$, choose it so as to maximize $\card{V_c}$.  Since $\card{J} \geq 2$, we get $\card{V_c} \geq \card{V_2} > \card{S}$ at the end.

\section{The recoloring technique}\label{recolorsection}
We can extract the idea in the proof of Theorem \ref{StrongColorBound} to get a
general recoloring technique.  Suppose $G$ is a $k$-vertex-critical graph and
pick $x \in V(G)$ and $(k-1)$-coloring $\pi$ of $H \DefinedAs G - x$.  Let $Z$
be a color class of $\pi$, say $Z = \pi^{-1}(1)$.  For each $z \in Z$, let
$O_z$ be the neighbors of $z$ which get a color that no other neighbor of $z$ gets; that is, put
$O_z \DefinedAs \setb{v}{N_H(z)}{\pi(v) \not \in \pi(N_H(z) - v)}$.  Suppose the
$O_z$ are pairwise disjoint.  If we could find an independent transversal
$\set{x} \cup \set{v_z}_{z \in Z}$ of $\set{x}$ together with the $O_z$, then
recoloring each $z \in Z$ with $\pi(v_z)$ and coloring each vertex in $\set{x}
\cup \set{v_z}_{z \in Z}$ with $1$ gives a proper $(k-1)$-coloring of $G$.  This
is exactly what happens in the above proof of the strong coloring result.  To
make this work more generally, we need to find situations where each $G[O_z]$
has high minimum degree.  Also, intuitively, the $O_z$ intersecting each other
should make things easier since recoloring a vertex in the intersection of
$O_{z_1}$ and $O_{z_2}$ works for both $z_1$ and $z_2$.  In our application we
will allow some restricted intersections.

\section{A general decomposition}\label{GeneralDecomposition}
Let $\D_1$ be the collection of graphs without induced $d_1$-choosable
subgraphs.  Plainly, $\D_1$ is hereditary. For a graph $G$ and $t \in \IN$, let
$\CC_t$ be the maximal cliques in $G$ having at least $t$ vertices. We prove the
following decomposition result for graphs in $\D_1$ which generalizes Reed's decomposition in \cite{reed1999strengthening}.

\begin{lem}\label{partition}
Suppose $G \in \D_1$ has $\Delta(G) \geq 8$ and contains no $K_{\Delta(G)}$. If
$\frac{\Delta(G) + 5}{2} \leq t \leq \Delta(G) - 1$, then $\bigcup \CC_t$ can be
partitioned into sets $D_1, \ldots, D_r$ such that for each $i \in \irange{r}$
at least one of the following holds:
\begin{itemize}
  \item $D_i = C_i \in \CC_t$,
  \item $D_i = C_i \cup \set{x_i}$ where $C_i \in \CC_t$ and $\card{N(x_i) \cap
  C_i} \geq t-1$.
\end{itemize}

\noindent Moreover, each $v \in V(G) - D_i$ has at most $t-2$ neighbors in $C_i$  for each $i \in \irange{r}$.
\end{lem}
\begin{proof}
Suppose $\card{C_i} \leq \card{C_j}$ and $C_i \cap C_j \neq \emptyset$. 
Then $\card{C_i \cap C_j} \geq \card{C_i} + \card{C_j} - (\Delta + 1) \geq 4$.  It follows from Corollary
\ref{K_tClassification} that $\card{C_i - C_j} \leq 1$.

Now suppose $C_i$ intersects $C_j$ and $C_k$.  By the above,
$\card{C_i \cap C_j} \geq \frac{\Delta(G) + 3}{2}$ and similarly $\card{C_i \cap
C_k} \geq \frac{\Delta(G) + 3}{2}$.  Hence $\card{C_i \cap C_j \cap C_k} \geq
\Delta(G) + 3 - (\Delta(G) - 1) = 4$.  Put $I \DefinedAs C_i \cap C_j \cap C_k$
and $U \DefinedAs C_i \cup C_j \cup C_k$.  By maximality of $C_i, C_j, C_k$,
$U$ cannot induce an almost complete graph.  Thus, by Corollary
\ref{K_tClassification}, $\card{U} \in \set{4, 5}$ and the graph induced on $U -
I$ is $E_3$.  But then $t \leq 6$ and hence $\Delta(G) \leq 7$, a contradiction.

\smallskip

\noindent The existence of the required partition is immediate. 
\end{proof}

\noindent When $D_i \in \CC_t$, we put $K_i \DefinedAs C_i \DefinedAs D_i$ and
when $D_i = C_i \cup \set{x_i}$, we put $K_i \DefinedAs N(x_i) \cap C_i$.

\section{Borodin-Kostochka when every vertex is in a big clique}
Let $G$ be a graph.  For $v \in V(G)$, we let $\omega(v)$ be the size of a
largest clique in $G$ containing $v$.  The proofs of the results in this section
go more smoothly when we strengthen the induction in terms of the parameter
$\rho(G) \DefinedAs \max_{v \in V(G)} d(v) - \omega(v)$.

\begin{thm}
For $k \geq 9$, every graph satisfying $\Delta \leq k$, $\omega < k$ and $\rho
\leq \frac{k}{3} - 2$ is $(k-1)$-colorable.
\end{thm}
\begin{proof}
Suppose the theorem fails for some $k \geq 9$ and choose a counterexample
$G$ minimizing $\card{G} + \size{G}$. Put $\Delta \DefinedAs \Delta(G)$.  If
$\Delta < k$, then $\Delta = k-1$ and by Brooks' theorem $G$ contains $K_k$, a
contradiction. Thus $\chi(G) = k = \Delta$.  Also, for any $v
\in V(G)$ we have $\rho(G-v) \leq \rho(G)$, applying our minimality condition on $G$ implies that $G$ is
vertex critical.

Therefore $\delta(G) \geq \Delta - 1$ and $G \in
\D_1$.  For any $v \in V(G)$, we have $\Delta - 1 - \omega(v) \leq d(v) -
\omega(v) \leq \frac{\Delta}{3} - 2$ and hence $\omega(v) \geq \frac23\Delta +
1$. Applying Lemma \ref{partition} with $t \DefinedAs \frac23\Delta + 1$ we
get a partition $D_1, \ldots, D_r$ of $\bigcup \CC_t = V(G)$.  Note that for $i
\in \irange{r}$, if $K_i \neq D_i$ then all vertices in $K_i$ are high by Lemma
\ref{E2JoinWithSomeLow}.  Pick $x \in K_1$.  Then $x$ has $\card{C_1} - 1 \leq
\Delta - 2$ neighbors in $D_1$ if $K_i = D_i$ and $\card{C_1} \leq \Delta - 1$
if $K_i \neq D_i$.  Hence, by our note, $x$ has a neighbor $w \in V(G) - D_1$.

We now claim that $xw$ is a critical edge in $G$.  Suppose otherwise that
$\chi(G - xw) = \Delta$.  Then by minimality of $G$ we must have $\rho(G-xw) >
\rho(G)$. Hence there is some vertex $v \in N(x) \cap N(w)$ so that every
largest clique containing $v$ contains $xw$.  But $v$ is in some $D_j$ and all largest cliques containing $v$ are contained in $D_j$ and hence do not contain $xw$, a contradiction.  

Let $\pi$ be a $(\Delta-1)$-coloring of $G - xw$ chosen so that $\pi(x) = 1$
and so as to minimize $\card{\pi^{-1}(1)}$. Consider $\pi$ as a coloring of
$G-x$. One key property of $\pi$ we will use is that since $x$ got $1$ in the
coloring of $G - xw$ and $x \in K_1$, no vertex of $D_1 - x$ gets colored $1$ by $\pi$.

Now put $Z \DefinedAs \pi^{-1}(1)$ and for $z \in Z$, let $O_z$ be as defined
in Section \ref{recolorsection}.  By minimality of $\card{Z}$, each $z \in Z$ has at least one neighbor in every color class of $\pi$.  
Hence $z$ has two or more neighbors in at most $2 + d(z) - \Delta$ of
$\pi$'s color classes. For each $z \in Z$ we have $i(z)$ such that $z \in D_{i(z)}$. For $z \in Z$ such that $i(z) \not \in i(Z - z)$, put $V_z \DefinedAs O_z \cap C_{i(z)}$.  

We have $\card{V_z} \geq \omega(z) - 1 - \parens{2 + d(z) - \Delta}$.  Since $\omega(z) \geq d(z) - \frac13 \Delta + 2$, we have $\card{V_z} \geq \frac23 \Delta - 1$.   Each $y \in V_z$ is adjacent to all of $C_{i(z)} - \set{y}$ and hence has at most $d(y) + 1 - \card{C_{i(z)}}$ neighbors outside $D_{i(z)}$.  Since $\omega(y) \geq d(y) +
2 - \frac13 \Delta$, we conclude that $y$ has at most $d(y) + 1
- (d(y) + 2 - \frac13 \Delta) = \frac13 \Delta - 1$ neighbors outside $D_{i(z)}$.

Now let $Z'$ be the $z \in Z$ with $i(z) \in i(Z - z)$. Then $Z'$ can be partitioned into pairs $\set{z, z'}$ such that $i(z) = i(z')$.  For such a pair, one of $z,z'$ is $x_{i(z)}$ and the other is in $C_{i(z)} - K_{i(z)}$. Put $V_z \DefinedAs O_z \cap O_{z'} \cap K_{i(z)}$ and don't define $V_{z'}$.  We have $\card{V_z} \geq \min\set{\omega(z), \omega(z')} - 1 - \parens{2 + d(z) - \Delta} - \parens{2 + d(z') - \Delta} \geq - \frac13 \Delta + 2 - 1 - 2\parens{2 - \Delta} - \max\set{d(z), d(z')} = \frac53 \Delta - \max\set{d(z), d(z')} - 3 \geq \frac23 \Delta - 3$.  Each $y \in V_z$ is adjacent to all of $D_{i(z)} - \set{y}$ and hence has at most $d(y) + 1 - \card{D_{i(z)}}$ neighbors outside $D_{i(z)}$.  Since $\card{D_{i(z)}} = \omega(y) + 1 \geq d(y) + 3 - \frac13 \Delta$, we conclude that $y$ has at most $\frac13 \Delta - 2$ neighbors outside $D_{i(z)}$.

Let $H$ be the subgraph of $G$ induced on the union of the $V_z$.  Put $S \DefinedAs N(x) \cap V(H)$.  Since $Z \cap D_1 = \emptyset$, $x$ has at least $\card{D_1} - 1$ neighbors in $D_1$ none of which are in $S$.  Hence $\card{S} \leq d(x) + 1 - \card{D_1} \leq d(x) + 1 - \omega(x) \leq \frac{\Delta}{3} - 1 < \card{V_z}$ for all $V_z$ since $\Delta \geq 7$. Hence we may apply Lemma
\ref{SingletonSetTransversalLopsided} on $H$ with $t \DefinedAs \frac13\Delta - 1$ to get an independent set $\set{v_z}_{z\in Z}$ disjoint from $S$ where $v_z \in V_z$. Recoloring each $z \in Z$ with $\pi(z)$ and
coloring $x \cup \set{v_z}_{z \in Z}$ with $1$ gives a $(\Delta - 1)$-coloring
of $G$, a contradiction.
\end{proof}

The following special case is a bit easier to digest.

\begin{cor}\label{TwoThirdsCliqueCor}
Every graph with $\chi \geq \Delta \geq 9$ such that every
vertex is in a clique on $\frac23\Delta + 2$ vertices contains $K_\Delta$.
\end{cor}
\section{Reducing to the irregular case}
It is easy to see that if there are irregular counterexamples to the
Borodin-Kostochka conjecture, then there are regular examples as well: take an
irregular counterexample $G$ clone it add an edge between any vertex with degree
less than $\Delta(G)$ and its clone; repeat until you have a regular graph (from
\cite{molloy2002graph}).

But what about the converse?  If there are regular examples, must there be
(connected) irregular examples?  We'll see that the answer is yes, but we need
to decrease the maximum degree by one.

\begin{thm}\label{IrregularReduction}
Every graph satisfying $\chi \geq \Delta = k \geq 9$ either
contains $K_k$ or contains an irregular critical subgraph satisfying $\chi
= \Delta = k - 1$.
\end{thm}
\begin{proof}
Suppose not and choose a counterexample $G$ minimizing $\card{G}$. Then $G$ is
vertex critical. If every vertex in $G$ were contained in a $(k-1)$-clique, 
then Corollary \ref{TwoThirdsCliqueCor} would give a $K_k$ in $G$, impossible. 
Hence we may pick $v \in V(G)$ not in a $(k-1)$-clique. If $v$ is high, choose a
$(k-1)$-coloring $\pi$ of $G-v$ so that the color class $T$ of $\pi$
where $v$ has two neighbors is as large as possible; if $v$ is low, let $\pi$ be a
$(k-1)$-coloring of $G-v$ where some color class $T$ of $\pi$ is as large as
possible.  By symmetry, we may assume that $\pi(T) = k-1$.  

Now we have a $(k-1)$-coloring $\zeta$ of $H \DefinedAs G-T$ given by $\zeta(x)
= \pi(x)$ for $x \neq v$ and $\zeta(v) = k-1$.  Since $\chi(H) = k - 1$, the
maximality condition on $T$ together with Brooks' theorem gives $\Delta(H) =
k - 1$.  Note that $d_H(v) = k - 2$.  Let $H'$ be a $(k-1)$-critical subgraph of
$H$.  Then $H'$ must contain $v$ and hence is not $K_{k-1}$.  Since $d_{H'}(v) =
k-2$ and $\Delta(H') = k - 1$ (by Brooks' theorem), $H'$ is an irregular
critical subgraph of $G$ satisfying $\chi = \Delta = k - 1$, a contradiction.
\end{proof}

Since the only known critical (or connected even) counterexample to
Borodin-Kostochka for $\Delta = 8$ is regular (see Figure \ref{fig:M_8}) we
might hope that the following strengthened conjecture is true.

\begin{conjecture}\label{EightRegular}
Every critical graph satisfying $\chi \geq \Delta = 8$ is regular.
\end{conjecture}

\section{Dense neighborhoods}
Here we show that the Borodin-Kostochka conjecture holds for graphs where each neighboorhood has ``most'' of its possible edges.  First, we need to convert high average degree in a neighborhood into a large clique in the neighborhood. We need the following extension of a fundamental result of Mader \cite{mader} (see Diestel \cite{diestel2010} for some history of this result).  We will also need $d_1$-choosability results from $\cite{mules}$ as well as some ideas for dealing with average degree in neighborhoods used in \cite{cranstonrabernclaw}.

\begin{lem}\label{MaderLemma}
For $k \geq 1$, every graph $G$ with $d(G) \geq 4k$ has a $(k+1)$-connected induced subgraph $H$ such that $d(H) > d(G) - 2k$.
\end{lem}

\begin{lem}\label{LowVertexHighAverageDegree}
If $B$ is a graph with $d(B) \geq \omega(B) + 2$, then $B$ has an induced subgraph $H$ such that $\join{K_1}{H}$ is $f$-choosable where $f(v) \geq d(v)$ for the $v$ in the $K_1$ and $f(x) \geq d(x) - 1$ for $x \in V(H)$.
\end{lem}
\begin{proof}
Let $B$ be such a graph.  Applying Lemma \ref{MaderLemma} with $k \DefinedAs 1$, we get a $2$-connected subgraph $H$ of $B$ with $d(H) > d(B) - 2 \geq \omega(B)$.  Since $H$ is $2$-connected, if it is not $d_0$-choosable, then it is either an odd cycle or complete.  The former is impossible since $d(H) \geq 3$, hence $H$ would be complete and we'd have the contradiction $\omega(H) > \omega(B)$.  Hence $H$ is $d_0$-choosable.  

Suppose $\join{K_1}{H}$ isn't $f$-choosable and let $L$ be a minimal bad $f$-assignment on $\join{K_1}{B}$.  
By Lemma \ref{LowSinglePair}, no nonadjacent pair in $H$ have intersecting lists and hence we must have $\sum_{v \in V(H)} \card{L(v)} \leq \card{Pot(L)}\omega(H)$.  Since for each $v \in V(H)$ we have $\card{L(v)} \geq d_H(v)$ and by the Small Pot Lemma we have $\card{Pot(L)} \leq \card{H}$, we must have $d(H) \leq \omega(H) \leq \omega(B) < d(H)$, a contradiction.
\end{proof}

\begin{lem}\label{VertexHighAverageDegree}
If $B$ is a graph with $d(B) \geq \omega(B) + 3$, then $B$ has an induced subgraph $H$ such that $\join{K_1}{H}$ is $d_1$-choosable.
\end{lem}
\begin{proof}
Let $B$ be such a graph.  Applying Lemma \ref{MaderLemma} with $k \DefinedAs 1$, we get a $2$-connected subgraph $H$ of $B$ with $d(H) > d(B) - 2 \geq \omega(B) + 1$.  As in the proof of Lemma \ref{LowVertexHighAverageDegree}, we see that $H$ is $d_0$-choosable. Suppose $\join{K_1}{H}$ is not $d_1$-choosable and let $L$ be a minimal bad $d_1$-assignment on $\join{K_1}{H}$.  Combining Lemma \ref{NeighborhoodPotShrink} with the same argument as in the proof of Lemma \ref{LowVertexHighAverageDegree} shows that $\card{Pot(L)} \leq \card{H}-1$.

Now, for $c \in Pot(L)$, we consider how big the color graphs $H_c$ can be.  All of the information comes from Lemma \ref{IntersectionsInB}. We have $\alpha(G_c) \leq 2$ for all $c \in Pot(L)$. First, suppose we have $c \in Pot(L)$ such that $\card{H_c} \geq \omega(H) + 3$.  Then, using Lemma \ref{IntersectionsInB}, we see that $\card{H_{c'}} \leq \omega(H)$ for all $c' \in Pot(L) - c$ and hence $\sum_{\gamma \in Pot(L)} \card{H_\gamma} \leq \card{H} + \parens{\card{Pot(L)} - 1}\omega(H) \leq \card{H}\omega(H) + \card{H} - 2\omega(H)$.  Now suppose we have $c \in Pot(L)$ such that $\card{H_c} = \omega(H) + 2$.  Then, using Lemma \ref{IntersectionsInB} again, we see that $\card{H_{c'}} \leq \omega(H) + 1$ for all $c' \in Pot(L) - c$ and hence $\sum_{\gamma \in Pot(L)} \card{H_\gamma} \leq 1 + \card{Pot(L)}(\omega(H) + 1) \leq \card{H}\omega(H) + \card{H} - \omega(H)$.

Therefore we must have $2\size{H} \leq \card{H}(\omega(H) + 1) - \omega(H)$ and hence $d(H) \leq \omega(H) + 1 < d(H)$, a contradiction.
\end{proof}

\begin{thm}\label{BKdense}
Every graph $G$ with $\omega(G) < \Delta(G)$ such that $d(G_v) \geq \frac23\Delta(G) + 4$ for each $v \in V(G)$ is $(\Delta(G)-1)$-colorable.
\end{thm}
\begin{proof}
Suppose note and let $G$ be a counterexample. Put $\Delta \DefinedAs \Delta(G)$.  Let $H$ be a $\Delta$-vertex-critical induced subgraph of $G$.  Then $\delta(H) \geq \Delta - 1$ and $H$ has no $d_1$-choosable induced subgraphs. By Theorem \ref{TwoThirdsCliqueCor}, we must have $v \in V(H)$ with $\omega(v) < \frac23\Delta + 2$. Suppose $d(H_v) < d(G_v)$. Then $d_H(v) = \Delta - 1$ and $\size{H_v} \geq \size{G_v} - (\Delta - 1)$; therefore, $d(H_v) > d(G_v) - 1 \geq \frac23\Delta + 3$.  Applying Lemma \ref{LowVertexHighAverageDegree} gives $\omega(v) > d(H_v) - 1 \geq \frac23\Delta + 2$, a contradiction.

Hence we must have $d(H_v) = d(G_v) \geq \frac23\Delta + 4$.  Applying Lemma \ref{VertexHighAverageDegree} gives $\omega(v) > d(H_v) - 2 \geq \frac23\Delta + 2$, a contradiction.
\end{proof}

\section{Bounding the order and independence number}
\begin{lem}\label{Onesies}
Let $G$ be a vertex critical graph with $\chi(G) = \Delta(G) + 1 - k$.  For every $v \in V(G)$ there is $H_v \unlhd G_v$ with:
\begin{enumerate}
\item $\card{H_v} \geq \Delta(G) - 2k$; and
\item $\delta(H_v) \geq \card{H_v} - (k+1)(\alpha(G) - 1) - 1$; and
\item $\size{H_v} \geq \card{H_v}\parens{\card{H_v}- (k+2)} - (k+1)\parens{\card{G} + 2k - (\Delta(G) + 1)}$.
\end{enumerate}
\end{lem}
\begin{proof}
Put $\Delta \DefinedAs \Delta(G)$. Pick $v \in V(G)$ and let $\pi$ be a $(\Delta - k)$-coloring of $G-v$.  Let $H_v$ be the subgraph of $G_v$ induced on $\setb{x}{N(v)}{\pi(x) \not \in \pi(N(v) - x)}$.  Plainly, $\card{H_v} \geq \Delta - 2k$.

By the usual Kempe chain argument, any $x, y \in V(H_v)$ must be in the same component of $C_{x,y} \DefinedAs G[\pi^{-1}(\pi(x)) \cup \pi^{-1}(\pi(y))]$.  Thus if $xy \not \in E(G)$, there must be a path of length at least $3$ in $C_{x,y}$ from $x$ to $y$ and hence some vertex of color $\pi(x)$ other than $x$ must have at least two neighbors of color $\pi(y)$ and some vertex of color $\pi(y)$ other than $y$ must have at least two neigbhors of color $\pi(x)$.  We say that such an intermediate vertex \emph{proxies} for $xy$.  Each $xy$ with $y \in V(H_v)$ must have some proxy $z_{xy} \in \pi^{-1}(\pi(x)) - x$ such that $z_{xy}$ proxies for at most $k+1$ total $xw$ with $w \in V(H_v)$, for otherwise we could recolor all of $xy$'s proxies, swap $\pi(x)$ and $\pi(y)$ in $x$'s component of $C_{x,y}$ and then color $v$ with $\pi(x)$ to get a $(\Delta-k)$-coloring of $G$.  We conclude that $x$ has at most $(k+1)(\card{\pi^{-1}(\pi(x))} - 1)$ non-neighbors in $H_v$.  This gives (2) immediately.

For (3), note that $\card{\pi(i)} \geq 2$ for each $i \in \irange{\Delta-k} - \pi(V(H_v))$ and hence $\sum_{j \in \pi(V(H_v))} \card{\pi^{-1}(j)} \leq \card{G} - 1 - 2(\Delta - k - \card{H_v})$.  Since 
\[\size{H_v} \geq \sum_{j  \in \pi(V(H_v))} \parens{\card{H_v} - 1 - (k+1)(\card{\pi^{-1}(j)} - 1)},\] 
\noindent we see that (3) follows.
\end{proof}

\begin{thm}\label{AlphaBound}
Every graph satisfies $\chi \leq \max\set{\omega, \Delta - 1, 4\alpha}$.
\end{thm}
\begin{proof}
Suppose not and choose a counterexample $G$ minimizing $\card{G}$.  Since none of the terms on the right side increase when we remove a vertex, $G$ is vertex critical.  Since the Borodin-Kostochka conjecture holds for graphs with $\alpha = 2$ and $\Delta \geq 9$, we must have $\alpha(G) \geq 3$ and hence $\Delta(G) \geq 13$.  By Lemma \ref{TwoThirdsCliqueCor}, there must be $v \in V(G)$ with $\omega(v) < \frac23 \Delta(G) + 2$.  Applying (2) of Lemma \ref{Onesies}, we get $H_v \unlhd G_v$ with $\card{H_v} \geq \Delta(G) - 2$ and $\delta(H_v) \geq \card{H_v} - 2\alpha(G) + 1$.  Since $\Delta(G) \geq \chi(G) \geq 4\alpha(G) + 1$, we have $\delta(H_v) \geq \card{H_v} - \frac{\Delta(G) - 1}{2} + 1 \geq \frac{\card{H_v} + 1}{2}$.  Applying Lemma \ref{neighborhood} shows that either $H_v = \join{K_3}{E_4}$ or $\omega(H_v) \geq \card{H_v} - 1$.  The former is impossible since $\Delta(G) > 9$. Therefore $\omega(v) \geq \omega(H_v) + 1 \geq \Delta(G) - 2 \geq \frac23 \Delta(G) + 2$ since $\Delta(G) \geq 12$, a contradiction.
\end{proof}

\begin{thm}\label{OrderBound}
Every graph satisfies $\chi \leq \max\set{\omega, \Delta - 1, \ceil{\frac{15 + \sqrt{48n + 73}}{4}}}$.
\end{thm}
\begin{proof}
Suppose not and choose a counterexample $G$ minimizing $\card{G}$.  Put $\Delta \DefinedAs \Delta(G)$ and $n \DefinedAs \card{G}$. Since none of the terms on the right side increase when we remove a vertex, $G$ is vertex critical. By Lemma \ref{TwoThirdsCliqueCor}, there must be $v \in V(G)$ with $\omega(v) < \frac23 \Delta + 2$.  Applying (3) of Lemma \ref{Onesies}, we get $H_v \unlhd G_v$ with with $\card{H_v} \geq \Delta - 2$ and $\size{H_v} \geq \card{H_v}\parens{\card{H_v}- 3} - 2\parens{n + 1 - \Delta}$.  By Lemma \ref{VertexHighAverageDegree}, we must have $d(H_v) < \frac23\Delta + 4$ and hence we have

\begin{align*}
\frac23\Delta + 4 &> 2\parens{\card{H_v}- 3} - \frac{4\parens{n + 1 - \Delta}}{\card{H_v}}\\
&\geq 2\parens{\Delta - 5} - \frac{4\parens{n + 1 - \Delta}}{\Delta - 2}.\\
\end{align*}

Simplifying a bit, we get $6(n-1) > (2\Delta - 15)(\Delta - 2)$.  Since $\Delta \geq \chi(G) \geq \frac{19 + \sqrt{48n + 73}}{4}$, we have $6(n-1) > (\frac{-11 + \sqrt{48n + 73}}{2})(\frac{11 + \sqrt{48n + 73}}{4}) = \frac{48n - 48}{8} = 6(n-1)$, a contradiction.
\end{proof}

\chapter{DOING THE VERTEX SHUFFLE}\label{VertexShuffleChapter}
\begin{center}
\emph{The material in this chapter appeared in \cite{rabern2012partitioning}, \cite{rabern2010destroying} and \cite{partitionnote}.}
\end{center}

Let $\G$ be the collection of all finite simple connected graphs. 
For a graph $G$, $x \in V(G)$ and $D \subseteq V(G)$ we use the notation $N_D(x) \DefinedAs N(x) \cap D$ and $d_D(x) \DefinedAs \card{N_D(x)}$. 
Let $\fancy{C}_G$ be the components of $G$ and $c(G) \DefinedAs
\card{\fancy{C}_G}$. If $\func{h}{\G}{\IN}$, we define $h$ for any graph as
$h(G) \DefinedAs \sum_{D \in \fancy{C}_G} h(D)$.  An
\emph{ordered partition} of $G$ is a sequence $\parens{V_1, V_2, \ldots, V_k}$ where the $V_i$
are pairwise disjoint and cover $V(G)$.  Note that we allow the $V_i$ to be
empty.  When there is no possibility of ambiguity, we call such a sequence a
\emph{partition}.

\section{Coloring when the high vertex subgraph has small
cliques}\label{ShuffleLowsSection}

In \cite{kierstead2009ore} Kierstead and Kostochka investigated the Brooks bound
with the Ore-degree $\theta$ in place of $\Delta$.

\begin{defn}
The \emph{Ore-degree} of an edge $xy$ in a graph $G$ is $\theta(xy) \DefinedAs d(x) + d(y)$.  The \emph{Ore-degree} of a graph $G$ is $\theta(G) \DefinedAs \max_{xy \in E(G)}\theta(xy)$.
\end{defn}

\begin{thm}[Kierstead and Kostochka \cite{kierstead2009ore} 2010]
If $G$ is a graph with $\chi(G) \geq \floor{\frac{\theta(G)}{2}} + 1 \geq 7$ then $G$ contains $K_{\chi(G)}$.
\end{thm}

This statement about Ore-degree is equivalent to the following statement about vertex critical graphs.

\begin{thm}[Kierstead and Kostochka \cite{kierstead2009ore} 2010]
The only vertex critical graph $G$ with $\chi(G) \geq \Delta(G) \geq 7$ such that $\fancy{H}(G)$ is edgeless is $K_{\chi(G)}$.
\end{thm}

In \cite{rabern2010a}, we improved the $7$ to $6$ by proving the following generalization.

\begin{thm}[Rabern 2012 \cite{rabern2010a}]
The only vertex critical graph $G$ with $\chi(G) \geq \Delta(G) \geq 6$ and $\omega(\fancy{H}(G)) \leq \floor{\frac{\Delta(G)}{2}} - 2$ is $K_{\chi(G)}$.
\end{thm}

This result and those in \cite{rabern2010b} were improved by Kostochka, Rabern and Stiebitz in \cite{krs_one}.  In particular, the following was proved.

\begin{thm}[Kostochka, Rabern and Stiebitz \cite{krs_one}
2012]\label{krs_one_main} The only vertex critical graphs $G$ with $\chi(G) \geq
\Delta(G) \geq 5$ such that $\fancy{H}(G)$ is edgeless are $K_{\chi(G)}$ and $O_5$.
\end{thm}

\begin{figure}[h]
\centering
\begin{tikzpicture}[scale = 1]

\node[circle, minimum width=3cm, thick, draw] (L) at (0,0) {$K_{n-2}$};
\node[ellipse, minimum height=2cm, minimum width=0.5cm, thick, draw] (M) at (2.5,0) {};      
\node[circle split, minimum width=3cm, thick, draw] (R) at (5,0) 
{$K_{\ceil{\frac{n-1}{2}}}$ \nodepart{lower} $K_{\floor{\frac{n-1}{2}}}$};
\node[circle, inner sep =1pt, fill, draw] (P1) at (2.5,-0.5) {};
\node[circle, inner sep =1pt, fill, draw] (P2) at (2.5,0.5) {};

\draw (L) (M) (R) (P1) (P2);
\draw[ultra thick] (1.6,-.25) -- (2.15,-.25);
\draw[ultra thick] (1.6,0) -- (2.15,0);
\draw[ultra thick] (1.6,.25) -- (2.15,.25);

\draw[ultra thick] (2.85,-.7) -- (3.9,-1.15);
\draw[ultra thick] (2.85,-.6) -- (3.7,-.9);
\draw[ultra thick] (2.85,-.5) -- (3.5,-.65);

\draw[ultra thick] (2.85,.7) -- (3.9,1.15);
\draw[ultra thick] (2.85,.6) -- (3.7,.9);
\draw[ultra thick] (2.85,.5) -- (3.5,.65);
\end{tikzpicture}
\caption{The graph $O_n$.}
\end{figure}


Here $O_n$ is the graph formed from the disjoint union of $K_n - xy$ and
$K_{n-1}$ by joining $\floor{\frac{n-1}{2}}$ vertices of the $K_{n-1}$ to $x$
and the other $\ceil{\frac{n-1}{2}}$ vertices of the $K_{n-1}$ to $y$ (see
Figure \ref{fig:reducer}). In \cite{rabern2012partitioning} we proved a result that implies all of the results in \cite{krs_one}. The proof replaces an algorithm of Mozhan \cite{mozhan1983} with the original, more general, algorithm of Catlin \cite{CatlinAnotherBound} on which it is based. This allows for a considerable simplification.   Moreover, we prove two preliminary partitioning results that are of independent interest.  
All coloring results follow from the first of these, the second is a generalization of a lemma due to Borodin \cite{borodin1976decomposition} (and independently Bollob\'as and Manvel \cite{bollobasManvel})
about partitioning a graph into degenerate subgraphs.  The following is the main
coloring result in \cite{rabern2012partitioning}.

\begin{cor}\label{MainShuffleColoringResult}
	Let $G$ be a vertex critical graph with $\chi(G) \geq \Delta(G) + 1 - p \geq 4$
	for some $p \in \IN$.  If $\omega(\fancy{H}(G)) \leq \frac{\chi(G) + 1}{p + 1} - 2$,
	then $G = K_{\chi(G)}$ or $G = O_5$.
\end{cor}

In the sections that follow we will prove this Corollary.  First, we give a
non-standard proof of Brooks' theorem to illustrate the technique.

\subsection{A weird proof of Brooks' theorem}
Let $G$ be a graph.  A partition $P \DefinedAs (V_0, V_1)$ of $V(G)$
will be called \emph{normal} if it achieves the minimum value of $(\Delta(G) -
1)\size{V_0} + \size{V_1}$.  Note that if $P$ is a normal partition, then
$\Delta(G[V_0]) \leq 1$ and $\Delta(G[V_1]) \leq \Delta(G) - 1$.  
The \emph{$P$-components} of $G$ are the components of $G[V_i]$ for $i \in
\irange{2}$.  A $P$-component is called an \emph{obstruction} if it is a
$K_2$ in $G[V_0]$ or a $K_{\Delta(G)}$ in $G[V_1]$ or an odd cycle in $G[V_1]$ when
$\Delta(G) = 3$.  A path $x_1x_2\cdots x_k$ is called \emph{$P$-acceptable} if
$x_1$ is contained in an obstruction and for different $i, j \in \irange{k}$, $x_i$ and
$x_j$ are in different $P$-components.  For a subgraph $H$ of $G$ and $x \in V(G)$, 
we put $N_H(x) \DefinedAs N(x) \cap V(H)$.

\begin{lem}\label{ObstructionFree}
Let $G$ be a graph with $\Delta(G) \geq 3$.  If $G$ doesn't contain
$K_{\Delta(G) + 1}$, then $V(G)$ has an obstruction-free normal partition.
\end{lem}
\begin{proof}
Suppose the lemma is false.  Among the normal partitions having the minimum
number of obstructions, choose $P \DefinedAs (V_0, V_1)$ and a
maximal $P$-acceptable path $x_1x_2\cdots x_k$ so as to minimize $k$.

Let $A$ and $B$ be the $P$-components containing $x_1$ and $x_k$ respectively.  
Put $X \DefinedAs N_A(x_k)$. First, suppose $\card{X} = 0$.  Then moving $x_1$ to the
other part of $P$ creates another normal partition $P'$ having the minimum number of
obstructions.  But $x_2x_3\cdots x_k$ is a maximal $P'$-acceptable path,
violating the minimality of $k$.  Hence $\card{X} \geq 1$.

Pick $z \in X$. Moving $z$ to the other part of $P$ destroys the obstruction $A$, so it must create an obstruction containing $x_k$ and hence $B$.  Since obstructions are complete graphs or odd cycles, the only possibility is that $\set{z} \cup V(B)$ induces an
obstruction.  Put $Y \DefinedAs N_B(z)$.  Then, since obstructions are regular, $N_B(x) = Y$ for all $x \in X$ and $\card{Y} = \delta(B) + 1$.  In particular, $X$ is joined to $Y$ in $G$.

Suppose $\card{X} \geq 2$.  Then, similarly to above, switching $z$ and 
$x_k$ in $P$ shows that $\set{x_k} \cup V(A - z)$ induces an obstruction.  Since
obstructions are regular, we must have $\card{N_{A-z}(x_k)} = \Delta(A)$ and hence $\card{X} \geq \Delta(A) + 1$.  
Thus $\card{X \cup Y} = \Delta(A) + \delta(B) + 2 = \Delta(G) + 1$.  Suppose $X$ is not a clique and pick nonadjacent $v_1, v_2 \in X$. It is easily seen that moving $v_1, v_2$ and then $x_k$ to their respective other parts violates normality of $P$.  Hence $X$ is a clique. Suppose $Y$ is not a clique and pick nonadjacent $w_1, w_2 \in Y$.  Pick $z' \in X - \set{z}$. Now moving $z$ and then $w_1, w_2$ and then $z'$ to their respective other parts again violates normality of $P$.  Hence $Y$ is a clique.  But $X$ is joined to $Y$, so $X \cup Y$ induces a $K_{\Delta(G) + 1}$ in $G$, a contradiction.

Hence we must have $\card{X} = 1$.  Suppose $X \neq \set{x_1}$.  First, suppose $A$ is $K_2$.  Then moving $z$ to the other part of $P$ creates another normal partition $Q$ having the minimum number of obstructions.  In $Q$, $x_kx_{k-1}\cdots x_1$ is a maximal $Q$-acceptable path since the $Q$-components containing $x_2$ and $x_k$ contain all of $x_1$'s neighbors in that part.  Running through the above argument using $Q$ gets us to the same point with $A$ not $K_2$.  Hence we may assume $A$ is not $K_2$.

Move each of $x_1, x_2, \ldots, x_k$ in turn to their respective other parts of $P$.  Then the obstruction $A$ was destroyed by moving $x_1$ and for $1 \leq i < k$, the obstruction created by moving $x_i$ was destroyed by moving $x_{i+1}$.  Thus, after the moves, $x_k$ is contained in an obstruction.  By minimality of $k$, it must be that $\set{x_k} \cup V(A - x_1)$ induces an obstruction and hence $\card{X} \geq 2$, a contradiction. 

Therefore $X = \set{x_1}$.  But then moving $x_1$ to the other part of $P$ creates an obstruction containing both $x_2$ and $x_k$.  Hence $k = 2$.
Since $x_1x_2$ is maximal, $x_2$ can have no neighbor in the other part besides
$x_1$.  But now switching $x_1$ and $x_2$ in $P$ creates a partition violating
the normality of $P$.
\end{proof}

\begin{thm}[Brooks 1941]
If a connected graph $G$ is not complete and not an odd cycle, then
$\chi(G) \leq \Delta(G)$.
\end{thm}
\begin{proof}
Suppose not and choose a counterexample $G$ minimizing $\Delta(G)$. 
Plainly, $\Delta(G) \geq 3$.  By Lemma \ref{ObstructionFree}, $V(G)$ has an
obstruction-free normal partition $(V_0, V_1)$.  
Since $G[V_0]$ has maximum degree at most one and contains no
$K_2$'s, we see that $V_0$ is independent.  Since $G[V_1]$ is obstruction-free,
applying minimality of $\Delta(G)$ gives $\chi(G[V_1]) \leq \Delta(G[V_1]) <
\Delta(G)$.  Hence $\chi(G) \leq \Delta(G)$, a contradiction.
\end{proof}

\subsection{The partitioning theorems}
An \emph{ordered partition} of a graph $G$ is a sequence $\parens{V_1, V_2,	\ldots, V_k}$ where the $V_i$ are pairwise disjoint and cover $V(G)$.  
Note that we allow the $V_i$ to be
empty.  When there is no possibility of ambiguity, we call such a sequence a
\emph{partition}.	For a vector $\vec{r} \in \IN^k$ we take the
coordinate labeling $\vec{r} = \parens{r_1, r_2, \ldots, r_k}$ as convention. 
Define the \emph{weight} of a vector $\vec{r} \in \IN^k$ as $\wt{\vec{r}} \DefinedAs \sum_{i \in \irange{k}} r_i$.   
Let $G$ be a graph. An \emph{$\vec{r}$-partition} of $G$ is an ordered partition
$P \DefinedAs \parens{V_1, \ldots, V_k}$ of $V(G)$ minimizing \[f(P) \DefinedAs \sum_{i \in \irange{k}} \parens{\size{G[V_i]} - r_i\card{V_i}}.\]

It is a fundamental result of Lov\'asz \cite{lovasz1966decomposition} that if $P \DefinedAs \parens{V_1, \ldots, V_k}$ is an $\vec{r}$-partition of $G$ with $\wt{\vec{r}} \geq \Delta(G) + 1 - k$, then $\Delta(G[V_i]) \leq r_i$ for each $i \in \irange{k}$.  The proof is simple: if there is a vertex in a part violating the condition, then there is some part it can be moved to that decreases $f(P)$.  As Catlin \cite{CatlinAnotherBound} showed, with the stronger condition $\wt{\vec{r}} \geq \Delta(G) + 2 - k$, a vertex of degree $r_i$ in $G[V_i]$ can always be moved to some other part while maintaining $f(P)$.  Since $G$ is finite, a well-chosen sequence of such moves must always wrap back on itself.  Many authors, including Catlin \cite{CatlinAnotherBound}, Bollob\'as and Manvel \cite{bollobasManvel} and Mozhan \cite{mozhan1983} have used such techniques to prove coloring results. We generalize these techniques by taking into account the degree in $G$ of the vertex to be moved---a vertex of degree less than the maximum needs a weaker condition on $\wt{\vec{r}}$ to be moved.

For $x \in V(G)$ and $D \subseteq V(G)$ we use the notation $N_D(x) \DefinedAs N(x) \cap D$ and $d_D(x) \DefinedAs \card{N_D(x)}$. Let $\fancy{C}(G)$ be the components of $G$ and $c(G) \DefinedAs \card{\fancy{C}(G)}$. For an induced subgraph $H$ of $G$, define $\delta_G(H) \DefinedAs \min_{v \in V(H)} d_G(v)$.  We also need the following notion of a movable subgraph.
		
		\begin{defn}
			Let $G$ be a graph and $H$ an induced subgraph of $G$.  For $d \in \IN$, the
			\emph{$d$-movable subgraph} of $H$ with respect to $G$ is the subgraph
			$\mov{H}{d}$ of $G$ induced on \[\setb{v}{V(H)}{d_G(v) = d \text{ and } H-v
			\text{ is connected}}.\]
		\end{defn}

We prove two partition theorems of similar form.  
All of our coloring results will follow from the first theorem, the second theorem is a 
degeneracy result from which Borodin's result in \cite{borodin1976decomposition} follows.  
For unification purposes, define a \emph{$t$-obstruction} as an odd cycle when
$t=2$ and a $K_{t + 1}$ when $t \geq 3$.

      \begin{thm}\label{PartitionTheorem}
			Let $G$ be a graph, $k,d \in \IN$ with $k \geq 2$ and $\vec{r} \in \IN_{\geq 2}^k$.  If $\wt{\vec{r}} \geq \max\set{\Delta(G) + 1 - k, d}$, then at least one of the following holds:
			\begin{enumerate}
			  \item $\wt{\vec{r}} = d$ and $G$ contains an induced subgraph $Q$ with $\card{Q} = d+1$ that can be partitioned into $k$ cliques $F_1, \ldots, F_k$ where 
					\begin{enumerate}
					\item $\card{F_1} = r_1 + 1$, $\card{F_i} = r_i$ for $i \geq 2$,
					\item $\card{F_1^d} \geq 2$, $\card{F_i^d} \geq 1$ for $i \geq 2$,
					\item for $i \in \irange{k}$, each $v \in V(F_i^d)$ is universal in $Q$;
					\end{enumerate}
			  \item there exists an $\vec{r}$-partition $P \DefinedAs \parens{V_1, \ldots, V_k}$ of 	
$G$ such that if $C$ is an $r_i$-obstruction in $G[V_i]$, then $\delta_G(C) \geq d$ and
			  $\mov{C}{d}$ is edgeless.
			\end{enumerate}
		\end{thm}
\begin{proof}
         For $i \in \irange{k}$, call a connected graph $C$
			\emph{$i$-bad} if $C$ is an $r_i$-obstruction such that $\mov{C}{d}$ has an edge. 
         For a graph $H$ and $i
			\in \irange{k}$, let $b_i(H)$ be the number of $i$-bad components of $H$.
			For an $\vec{r}$-partition $P \DefinedAs \parens{V_1, \ldots, V_k}$ of $G$ let
			\[b(P) \DefinedAs \sum_{i \in \irange{k}} b_i(G[V_i]).\]
			
			\noindent Let $P \DefinedAs \parens{V_1, \ldots, V_k}$ be an $\vec{r}$-partition of
			$V(G)$ minimizing $b(P)$.

			Let $i \in \irange{k}$ and $x \in V_i$ with $d_{V_i}(x) \geq r_i$.  Suppose $d_G(x) = d$.
			Then, since $\wt{\vec{r}} \geq d$, for every $j \neq i$ we have $d_{V_j}(x) \leq
			r_j$. Moving $x$ from $V_i$ to $V_j$ gives a new partition $P^*$ with $f(P^*)
			\leq f(P)$. Note that if $d_{G}(x) < d$ we would have $f(P^*) < f(P)$
			contradicting the minimality of $P$.

			Supppose (2) fails to hold.  Then $b(P) > 0$.  By symmetry, we may assume that there is a
			$1$-bad component $A_1$ of $G[V_{1}]$. Put $P_1 \DefinedAs P$ and $V_{1,i}
			\DefinedAs V_i$ for $i \in \irange{k}$. Since $A_1$ is $1$-bad we have $x_1
			\in V(\mov{A_1}{d})$ which has a neighbor in $V(\mov{A_1}{d})$. By the above we
			can move $x_1$ from $V_{1, 1}$ to $V_{1, 2}$ to get a new partition $P_2
			\DefinedAs \parens{V_{2, 1}, V_{2,2}, \ldots, V_{2,k}}$ where $f(P_2) = f(P_1)$.  
         Since removing $x_1$ from $A_1$ decreased $b_{1}(G[V_{1}])$, minimality of
			$b(P_1)$ implies that $x_1$ is in a $2$-bad component $A_2$ in $V_{2,2}$.			
			Now, we may choose $x_2 \in
			V(\mov{A_2}{d}) - \set{x_1}$ having a neighbor in $\mov{A_2}{d}$ and move
			$x_2$ from $V_{2, 2}$ to $V_{2, 1}$ to get a new partition $P_3
			\DefinedAs \parens{V_{3, 1}, V_{3,2}, \ldots, V_{3,k}}$ where $f(P_3) =
			f(P_1)$.  We continue on this way to construct sequences $A_1, A_2, \ldots$, $P_1, P_2, P_3, \ldots$ and $x_1, x_2, \ldots$.
						
			This process can be defined recursively as follows. For $t \in \IN$, put $j_t \DefinedAs 1$ for odd $t$ and $j_t \DefinedAs 2$ for even $t$. Put $P_1 \DefinedAs P$ and $V_{1,i} \DefinedAs V_i$ for $i \in \irange{k}$. Pick $x_1	\in V(\mov{A_1}{d})$ which has a neighbor in $V(\mov{A_1}{d})$. Move $x_1$ from $V_{1, 1}$ to $V_{1, 2}$ to get a new partition $P_2 \DefinedAs \parens{V_{2, 1}, V_{2,2}, \ldots, V_{2,k}}$ where $f(P_2) = f(P_1)$ and let $A_2$ be the $2$-bad component in $V_{2,2}$ containing $x_1$. Then for $t \geq 2$, pick $x_t \in V(\mov{A_t}{d} - x_{t-1})$ which has a neighbor in $V(\mov{A_t}{d})$. Move $x_t$ from $V_{t, j_t}$ to $V_{t, 3-j_t}$ to get a new partition $P_{t+1} \DefinedAs \parens{V_{t+1, 1}, V_{t+1,2}, \ldots, V_{t+1,k}}$ where $f(P_{t+1}) = f(P_t)$ and let $A_{t+1}$ be the $(3-j_t)$-bad component in $V_{t+1,3-j_t}$ containing $x_t$.

			Since $G$ is finite, at some point we will need to reuse a leftover component; that is, 
			there is a smallest $t$ such that $A_{t + 1} - x_t = A_s - x_s$ for some $s <
			t$.  Let $j \in \irange{2}$ be such that in $V(A_s) \subseteq V_{s, j}$. 
			Then $V(A_t) \subseteq V_{t, 3-j}$.  

			\claim{1}{$N(x_t) \cap V(A_s - x_s) = N(x_s) \cap V(A_s - x_s)$.}  

			This is immediate since $A_s$ is $r_j$-regular.

			\claim{2}{$s = 1$, $t = 2$, both $A_s$ and $A_t$ are complete,
			$\mov{A_s}{d}$ is joined to $A_t - x_{t-1}$ and $\mov{A_t}{d}$ is joined to $A_s - x_s$.}

			\subclaim{2a}{$N(x_s) \cap V(\mov{A_s}{d}) \neq \emptyset$.}
			
			In the construction of the sequence, $x_s$ was chosen such that it had a neighbor in $\mov{A_s}{d}$.

			\subclaim{2b}{For any $z \in N(x_s) \cap V(\mov{A_s}{d})$ we have $N(z) \cap V(A_t - x_{t-1}) = N(x_{t-1}) \cap V(A_t - x_{t-1})$.  Moreover, if $x_s$ is adjacent to $x_t$, then $N(x_s) \cap V(A_t - x_{t-1}) = N(x_{t-1}) \cap V(A_t - x_{t-1})$ and $x_s = x_{t-1}$.}

			In $P_s$, move $z$ to $V_{s, 3-j}$ to get a new partition $P^\gamma \DefinedAs \parens{V_{\gamma,
			1}, V_{\gamma, 2}, \ldots, V_{\gamma, k}}$. Then $z$ must create an
			$r_{3-j}$-obstruction with $A_t - x_{t-1}$ in $V_{\gamma, 3-j}$ since
			$z$ is adjacent to $x_t$ by Claim 1.  
			In particular, $N(z) \cap V(A_t - x_{t-1}) = N(x_{t-1}) \cap V(A_t -
			x_{t-1})$.  If $x_s$ is adjacent to $x_t$, the same argument (with $x_s$ in place of $z$) gives $N(x_s) \cap V(A_t - x_{t-1}) = N(x_{t-1}) \cap V(A_t - x_{t-1})$ and $x_s = x_{t-1}$.

         \subclaim{2c}{$A_s$ is complete and $x_s$	is adjacent to $x_t$.}

			By Subclaim 2a, $N(x_s) \cap V(\mov{A_s}{d}) \neq \emptyset$. Pick $z \in N(x_s) \cap V(\mov{A_s}{d})$ and let $P^\gamma$ be as in Subclaim 2b. In $P^\gamma$, move    $x_t$ to $V_{\gamma, j}$ to get a new partition $P^{\gamma*} \DefinedAs \parens{V_{\gamma*, 1},
			V_{\gamma*, 2}, \ldots, V_{\gamma*, k}}$. Since $x_s$ has at least two neighbors in $A_s$, by Claim 1, $x_t$ has a neighbor in $A_s - z$.  Hence $x_t$ must create an
			$r_{j}$-obstruction with $A_s - z$ in $V_{\gamma*, j}$.  In
			particular, $N(z) \cap V(A_s - z) = N(x_t) \cap V(A_s - z)$.  Thus $x_s$
			is adjacent to $x_t$ and we have $N[z] \cap V(A_s) = N[x_s] \cap
			V(A_s)$.  Thus, if $A_s$ is an odd cycle, it must be a triangle.  
			Hence $A_s$ is complete.  

			\subclaim{2d}{$\mov{A_s}{d}$ is joined to $N(x_{t-1}) \cap V(A_t - x_{t-1})$ and $x_s = x_{t-1}$.}

			Since $A_s$ is complete by Subclaim 2c, we have $N(x_s) \cap V(\mov{A_s}{d}) = V(\mov{A_s}{d} - x_s)$.  Since $x_s$ is adjacent to $x_t$ by Subclaim 2c, applying Subclaim 2b shows that $\mov{A_s}{d}$ is joined to $N(x_{t-1}) \cap V(A_t - x_{t-1})$ and $x_s = x_{t-1}$.  
			
			\subclaim{2e}{$s=1$ and $t=2$.} 

			Suppose $s > 1$.  Then, since $x_{s-1} \in V(\mov{A_s}{d})$, Subclaim 2d shows that $x_{s-1}$ is joined to $N(x_{t-1}) \cap V(A_t - x_{t-1})$ and hence $A_t - x_{t-1} = A_{s - 1} - x_{s-1}$ violating
			minimality of $t$.  Whence, $s = 1$ and $t=2$.
				
			\subclaim{2f}{$A_t$ is complete and $\mov{A_s}{d}$ is joined to $A_t - x_{t-1}$.}

			Pick $z \in N(x_s) \cap V(\mov{A_s}{d})$.  Then $z$ is joined to $A_t-x_t$ by Subclaim 2d.
			In $P_{t+1}$, move $z$ to $V_{t+1, 3-j}$ to
			get a new partition $P^{\beta} \DefinedAs \parens{V_{\beta, 1},
			V_{\beta, 2}, \ldots, V_{\beta, k}}$. Then $z$ must create an
			$r_{3-j}$-obstruction with $A_t - x_t$ in $V_{\beta, 3-j}$.  In
			particular, $V(A_t - x_t) = N(z) \cap V(A_t - x_t) = N(x_t) \cap V(A_t - x_t)$. Thus, if $A_t$ is an odd cycle, it must be a triangle.  	Hence $A_t$ is complete. Now Subclaim 2d gives that $\mov{A_s}{d}$ is joined to $A_t - x_{t-1}$.

			\subclaim{2g}{$\mov{A_t}{d}$ is joined to $A_s - x_s$.}
			
			Since $x_s = x_{t-1}$, the statement is clear for $x_{t-1}$. Pick $y \in V(\mov{A_t}{d} - x_{t-1})$ and $z \in V(\mov{A_s}{d})$. In $P_t$, move $y$ to $V_{t, j}$.  Since $y$ is adjacent to $z$ by Subclaim 2f, $y$ must create an $r_j$-obstruction with $A_s - x_s$ and since $A_s$ is complete, $y$ must be joind to $A_s - x_s$.  Hence $\mov{A_t}{d}$ is joined to $A_s - x_s$.
						
         \claim{3}{(1) holds.}
         
			We can play the same game with $V_1$ and $V_i$ for any
			$3 \leq i \leq k$ as we did with $V_1$ and $V_2$ above.  Let $B_1 \DefinedAs A_1$, $B_2 \DefinedAs A_2$ and for $i \geq 3$, let $B_i$ be the $r_i$-obstruction made by moving
			$x_1$ into $V_i$.  Then $B_i$ is complete for each $i \in \irange{k}$. 
			Applying Claim 2 to all pairs $B_i, B_j$ shows that for any
			distinct $i, j \in \irange{k}$, $\mov{B_i}{d}$ is joined to $B_j - x_1$. 
			Put $F_1 = B_1$ and $F_i = B_i - x_1$ for $i \geq 2$.  
			Let $Q$ be the union of the $F_i$.  Then (a), (b) and (c) of (1) are satisfied.
			Note that $\card{Q} = \wt{\vec{r}}+1$ and since any $v \in B_1^d$ is universal in $Q$,
			$\card{Q} \leq d + 1$. By assumption $\wt{\vec{r}} \geq d$, whence $\wt{\vec{r}}=d$.  
			Hence, (1) holds.
\end{proof}
	
The following result generalizes a lemma due to Borodin \cite{borodin1976decomposition}.  This lemma of Borodin was generalized in another direction in \cite{borodin2000variable}.  The proof that follows is basically the same as that of Theorem \ref{PartitionTheorem}.  For a reader who is only interested in the coloring results, this theorem can be safely skipped.	

		\begin{thm}\label{DegenProp}
			Let $G$ be a graph, $k,d \in \IN$ with $k \geq 2$ and $\vec{r} \in \IN_{\geq 1}^k$ where at most one of the $r_i$ is one.  If $\wt{\vec{r}} \geq \max\set{\Delta(G) + 1 - k, d}$, then at least one of the following holds:
			\begin{enumerate}
			  \item $\wt{\vec{r}} = d$ and $G$ contains a $\join{K_t}{E_{d+1-t}}$ where $t \geq d +
			  1 - k$, for each $v \in V(K_t)$ we have $d_G(v) = d$ and for each $v \in
			  V(E_{d+1-t})$ we have $d_G(v) > d$; or,
			  \item there exists an $\vec{r}$-partition $P \DefinedAs \parens{V_1, \ldots, V_k}$ of 	
$G$ such that if $C$ is an $r_i$-regular component of $G[V_i]$, then $\delta_G(C) \geq d$ and
			  there is at most one $x \in V(\mov{C}{d})$ with $d_{\mov{C}{d}}(x) \geq
			  r_i -1$.  Moreover, $P$ can be chosen so that either: 
			  \begin{enumerate}
			    \item for all $i \in \irange{k}$ and $r_i$-regular component $C$ of
			    $G[V_i]$, we have $\card{\mov{C}{d}} \leq 1$; or,
			  	\item  for some $i \in \irange{k}$ and some $r_i$-regular component $C$ of
			  	$G[V_i]$, there is $x \in V(\mov{C}{d})$ such that
			  	$\setb{y}{N_C(x)}{d_G(y) = d}$ is a clique.
			  \end{enumerate}
			\end{enumerate}
		\end{thm}
		\begin{proof}
			For $i \in \irange{k}$, call a connected graph $C$
			\emph{$i$-bad} if $C$ is $r_i$-regular and there are at least two $x \in
			V(\mov{C}{d})$ with $d_{\mov{C}{d}}(x) \geq r_i - 1$. For a graph $H$ and $i
			\in \irange{k}$, let $b_i(H)$ be the number of $i$-bad components of $H$.
			For an $\vec{r}$-partition $P \DefinedAs \parens{V_1, \ldots, V_k}$ of $G$ let
			\[c(P) \DefinedAs \sum_{i \in \irange{k}} c(G[V_i]),\]
			\[b(P) \DefinedAs \sum_{i\in \irange{k}} b_i(G[V_i]).\]
			
			\noindent Let $P \DefinedAs \parens{V_1, \ldots, V_k}$ be an $\vec{r}$-partition of
			$V(G)$ minimizing $c(P)$ and subject to that $b(P)$.

			Let $i \in \irange{k}$ and $x \in V_i$ with $d_{V_i}(x) \geq r_i$.  Suppose $d_G(x) = d$.
			Then, since $\wt{\vec{r}} \geq d$, for every $j \neq i$ we have $d_{V_j}(x) \leq
			r_j$. Moving $x$ from $V_i$ to $V_j$ gives a new partition $P^*$ with $f(P^*)
			\leq f(P)$. Note that if $d_{G}(x) < d$ we would have $f(P^*) < f(P)$
			contradicting the minimality of $P$.

			Suppose $b(P) > 0$.  By symmetry, we may assume that there is a
			$1$-bad component $A_1$ of $G[V_{1}]$. Put $P_1 \DefinedAs P$ and $V_{1,i}
			\DefinedAs V_i$ for $i \in \irange{k}$. Since $A_1$ is $1$-bad we have $x_1
			\in V(\mov{A_1}{d})$ with $d_{\mov{A_1}{d}}(x) \geq r_1 -1$. By the above we
			can move $x_1$ from $V_{1, 1}$ to $V_{1, 2}$ to get a new partition $P_2
			\DefinedAs \parens{V_{2, 1}, V_{2,2}, \ldots, V_{2,k}}$ where $f(P_2) = f(P_1)$.  
			By the minimality of $c(P_1)$, $x_1$ is adjacent to only one component $C_2$
			in $G[V_{1, 2}]$. Let $A_2 \DefinedAs G[V(C_2) \cup \set{x_1}]$.  
			Since removing $x_1$ from $A_1$ decreased $b_{1}(G[V_{1}])$, minimality of
			$b(P_1)$ implies that $A_2$ is $2$-bad. Now, we may choose $x_2 \in
			V(\mov{A_2}{d}) - \set{x_1}$ with $d_{\mov{A_2}{d}}(x) \geq r_2 -1$ and move
			$x_2$ from $V_{2, 2}$ to $V_{2, 1}$ to get a new partition $P_3
			\DefinedAs \parens{V_{3, 1}, V_{3,2}, \ldots, V_{3,k}}$ where $f(P_3) =
			f(P_1)$.
						
			Continue on this way to construct sequences $A_1, A_2, \ldots$, $P_1, P_2, P_3, \ldots$ and $x_1, x_2, \ldots$.  
			Since $G$ is finite, at some point we will need to reuse a leftover component; that is, 
			there is a smallest $t$ such that $A_{t + 1} - x_t = A_s - x_s$ for some $s <
			t$.  Let $j \in \irange{2}$ be such that in $V(A_s) \subseteq V_{s, j}$. 
			Then $V(A_t) \subseteq V_{t, 3-j}$.  Note that, since $A_s$ is $r_j$-regular,
			$N(x_t) \cap V(A_s - x_s) = N(x_s) \cap V(A_s - x_s)$.
			
			We claim that $s = 1$, $t = 2$, both $A_s$ and $A_t$ are complete,
			$\mov{A_s}{d}$ is joined to $A_t - x_{t-1}$ and $\mov{A_t}{d}$ is joined to $A_s - x_s$.
			
			Put $X \DefinedAs N(x_s) \cap V(\mov{A_s}{d})$.  Since $x_s$ witnesses the
			$j$-badness of $A_s$, $\card{X} \geq \max\set{1, r_j - 1}$. Pick $z \in X$.
			In $P_s$, move $z$ to $V_{s, 3-j}$ to get a new partition $P^\gamma \DefinedAs \parens{V_{\gamma,
			1}, V_{\gamma, 2}, \ldots, V_{\gamma, k}}$. Then $z$ must create an
			$r_{3-j}$-regular component with $A_t - x_{t-1}$ in $V_{\gamma, 3-j}$ since
			$z$ is adjacent to $x_t$.  
			In particular, $N(z) \cap V(A_t - x_{t-1}) = N(x_{t-1}) \cap V(A_t -
			x_{t-1})$. Since $z$ is adjacent to $x_t$, so is $x_{t-1}$. 
			
			Suppose $r_j \geq 2$. In $P^\gamma$, move $x_t$ to $V_{\gamma, j}$ to
			get a new partition $P^{\gamma*} \DefinedAs (V_{\gamma*, 1},
			V_{\gamma*, 2}, \ldots, V_{\gamma*, k})$. Then $x_t$ must create an
			$r_{j}$-regular component with $A_s - z$ in $V_{\gamma*, j}$.  In
			particular, $N(z) \cap V(A_s - z) = N(x_t) \cap V(A_s - z)$.  Thus $x_s$
			is adjacent to $x_t$ and we have $N[z] \cap V(A_s) = N[x_s] \cap
			V(A_s)$. Put $K \DefinedAs X \cup \set{x_s}$.  Then $\card{K} \geq r_j$ and
			$K$ induces a clique.  If $\card{K} > r_j$, then $A_s = K$ is complete. 
			Otherwise, the vertices of $K$ have a common neighbor $y \in V(A_s) - K$ and
			again $A_s$ is complete. Also, since $x_s$ is adjacent to $x_t$, using $x_s$
			in place of $z$ in the previous paragraph, we conclude that $K$
			is joined to $N(x_{t-1}) \cap V(A_t - x_{t-1})$ and $x_s = x_{t-1}$.  
			
			Suppose $s > 1$.  Then $x_{s-1}$ is joined to $N(x_{t-1}) \cap V(A_t -
			x_{t-1})$ and hence $A_t - x_{t-1} = A_{s - 1} - x_{s-1}$ violating
			minimality of $t$.  Whence, if $r_j \geq 2$ then $s = 1$.  
			
			Note that $K = V(\mov{A_s}{d})$ and hence if $r_j \geq 2$ then $A_s$ is
			complete and $\mov{A_s}{d}$ is joined to $N(x_{t-1}) \cap V(A_t - x_{t-1})$.  If $r_{3-j}
			= 1$, then $A_t$ is a $K_2$ and $N(x_{t-1}) \cap V(A_t - x_{t-1}) = V(A_t -
			x_{t-1}) = \set{x_t}$.  We already know that $x_t$ is joined to $A_s - x_s$. 
			Thus the cases when $r_j \geq 2$ and $r_{3-j} = 1$ are taken care of. By
			assumption, at least one of $r_j$ or $r_{3-j}$ is at least two.  Hence it
			remains to handle the cases with $r_{3-j} \geq 2$.

			Suppose $r_{3-j} \geq 2$.  In $P_{t+1}$, move $z$ to $V_{t+1, 3-j}$ to
			get a new partition $P^{\beta} \DefinedAs (V_{\beta, 1},
			V_{\beta, 2}, \ldots, V_{\beta, k})$. Then $z$ must create an
			$r_{3-j}$-regular component with $A_t - x_t$ in $V_{\beta, 3-j}$.  In
			particular, $N(z) \cap V(A_t - x_t) = N(x_t) \cap V(A_t - x_t)$.  Since $N(z)
			\cap V(A_t - x_{t-1}) = N(x_{t-1}) \cap V(A_t - x_{t-1})$, we have
			$N[x_{t-1}] \cap V(A_t) = N(z) \cap V(A_t) = N[x_t] \cap V(A_t)$.  Put $W
			\DefinedAs N[x_t] \cap V(\mov{A_t}{d})$. Each $w \in W$ is adjacent to $z$
			and running through the argument above with $w$ in place of $x_t$ shows
			that $W$ is a clique joined to $z$.  Moreover, since $x_t$ witnesses the
			$(3-j)$-badness of $A_t$, $\card{W} \geq r_{3-j}$.  As with $A_s$ above, we
			conclude that $A_t$ is complete.  Since $x_s \in V_{t+1, 3-j}$ and $x_s$ is
			adjacent to $z$, it must be that $x_s \in V(A_t - x_t)$.  Thence $x_s$ is
			joined to $W$ and $x_s = x_{t-1}$.  
			
			Suppose that $r_j \geq 2$ as well.  We know that $s = 1$, $A_s$ is
			complete and $\mov{A_s}{d}$ is joined to $N(x_{t-1}) \cap V(A_t - x_{t-1})
			= A_t - x_{t-1}$.  Also, we just showed that $A_t$ is complete and
			$\mov{A_t}{d}$ is joined to $A_s - x_s$.
						
			Thus, we must have $r_j = 1$ and $r_{3-j} \geq 2$.  Then,
			since $A_s$ is a $K_2$, by the above, $A_s$ is joined to $W$.  Since $W =
			\mov{A_t}{d}$, it only remains to show that $s=1$. Suppose $s > 1$.  Then
			$x_{s-1}$ is joined to $W$ and hence $A_t - x_{t-1} = A_{s - 1} - x_{s-1}$ violating minimality of $t$.
			
			Therefore $s = 1$, $t = 2$, both $A_s$ and $A_t$ are complete,
			$\mov{A_s}{d}$ is joined to $A_t - x_{t-1}$ and $\mov{A_t}{d}$ is joined to
			$A_s - x_s$.  But we can play the same game with $V_1$ and $V_i$ for any
			$3 \leq i \leq k$ as well.  Let $B_1 \DefinedAs A_1$, $B_2 \DefinedAs A_2$
			and for $i \geq 3$, let $B_i$ be the $r_i$-regular component made by moving
			$x_1$ into $V_i$.  Then $B_i$ is complete for each $i \in \irange{k}$. 
			Applying what we just proved to all pairs $B_i, B_j$ shows that for any
			distinct $i, j \in \irange{k}$, $\mov{B_i}{d}$ is joined to $B_j - x_1$. 
			Since $\card{\mov{B_i}{d}} \geq r_i$ and $x_1 \in V(\mov{B_i}{d})$ for each
			$i$, this gives a $\join{K_t}{E_{\wt{\vec{r}} + 1 - t}}$ in $G$ where $t \geq \wt{\vec{r}} + 1 -
			k$.  Take such a subgraph $Q$ maximizing $t$.  Since all the $B_i$ are
			complete, any vertex of degree $d$ will be in $\mov{B_i}{d}$; therefore, for each $v
			\in V(K_t)$ we have $d_G(v) = d$ and for each $v \in V(E_{\wt{\vec{r}}+1-t})$ we have $d_G(v) > d$.
			Note that $\card{Q} = \wt{\vec{r}}+1$ and since $d_G(v) = d$ for any $v \in V(K_t)$,
			$\card{Q} \leq d + 1$. By assumption $\wt{\vec{r}} \geq d$, whence $\wt{\vec{r}}=d$.  
			Thus if (1) fails, then	the first part of (2) holds.
			
			It remains to prove that we can choose $P$ to satisfy one of (a) or (b).  Suppose that (1) fails and $P$ cannot be chosen to satisfy either (a) or (b).  For $i \in \irange{k}$, call a connected graph $C$
			\emph{$i$-ugly} if $C$ is $r_i$-regular and $\card{\mov{C}{d}} \geq 2$ let $u_i(H)$ be the number of $i$-ugly components of $H$.  Note that if $C$ is $i$-bad, then it is $i$-ugly.  For an $\vec{r}$-partition $P \DefinedAs \parens{V_1, \ldots, V_k}$ of $G$ let
\[u(P) \DefinedAs \sum_{i\in \irange{k}} u_i(G[V_i]).\]

Choose an $\vec{r}$-partition $Q \DefinedAs \parens{V_1, \ldots, V_k}$ of $G$ first minimizing $c(Q)$, then subject to that requiring $b(Q) \leq 1$ and then subject to that minimizing $u(Q)$.  Since $Q$ does not satisfy (a), at least one of $b(Q) = 1$ or $u(Q) \geq 1$ holds.  By symmetry, we may assume that $G[V_1]$ contains a component $D_1$ which is either $1$-bad or $1$-ugly (or both).  If $D_1$ is $1$-bad, pick $w_1 \in V(D_1^d)$ witnessing the $1$-badness of $D_1$; otherwise pick $w_1 \in V(D_1^d)$ arbitrarily. Move $w_1$ to $V_2$, to form a new $\vec{r}$-partition.  This new partition still satisfies all of our conditions on $Q$. As above we construct a sequence of vertex moves that will wrap around on itself. This can be defined recursively as follows.  For $t \geq 2$, if $D_t$ is bad pick $w_t \in V(D_t^d - w_{t-1})$ witnessing the badness of $D_t$; otherwise, if $D_t$ is ugly pick $w_t \in V(D_t^d - w_{t-1})$ arbitrarily.  Now move $w_t$ to the part from which $w_{t-1}$ came to form $D_{t+1}$.  Let $Q_1 \DefinedAs Q, Q_2, Q_3, \ldots$ be the partitions created by a run of this process. Note that the process can never create a component that is not ugly lest we violate the minimality of $u(Q)$.  

Since $G$ is finite, at some point we will need to reuse a leftover component; that is, 
there is a smallest $t$ such that $D_{t + 1} - x_t = D_s - x_s$ for some $s <
t$.  First, suppose $D_s$ is not bad, but merely ugly.  Then $D_{t+1}$ is not bad and hence $b(Q_{t+1}) = 0$ and $u(Q_{t+1}) < u(Q)$, a contradiction.  Hence $D_s$ is bad.  

Suppose $D_t$ is not bad.  
As in the proof of the first part of (2), we can conclude that $x_s = x_{t-1}$.  
Pick $z \in N(x_s) \cap V(\mov{D_s}{d})$. 
Since $z$ is adjacent to $x_t$, by moving $z$ to the part containing $x_t$ in $P_s$ we conclude 
$N(z) \cap V(D_t - x_s) = N(x_s) \cap V(D_t - x_s)$.  
Put $T \DefinedAs \setb{y}{N_{D_t}(x_s)}{d_G(y) = d}$. 
Suppose $T$ is not a clique and let $w_1, w_2 \in T$ be nonadjacent.  
Now, in $P_t$, since $z$ is adjacent to both $w_1$ and $w_2$, swapping $w_1$ and $w_2$ with $z$ contradicts minimality of $f(Q)$.  
Hence $T$ is a clique and (b) holds, a contradiction.

Thus we may assume that $D_t$ is bad as well.  
Now we may apply the same argument as in the proof of the first part of (2) to show that (1) holds.  This final contradiction completes the proof.
			
			
		\end{proof}

		\begin{cor}[Borodin \cite{borodin1976decomposition}]
		Let $G$ be a graph not containing a $K_{\Delta(G) + 1}$. If $r_1, r_2 \in
		\IN_{\geq 1}$ with $r_1 + r_2 \geq \Delta(G) \geq 3$, then $V(G)$ can be
		partitioned into sets $V_1, V_2$ such that $\Delta(G[V_i]) \leq r_i$ and $\text{col}(G[V_i]) \leq r_i$ for $i \in \irange{2}$.
		\end{cor}
		\begin{proof}
		Apply Proposition \ref{DegenProp} with $\vec{r} \DefinedAs \parens{r_1, r_2}$
		and $d = \Delta(G)$.  Since $G$ doesn't contain a $K_{\Delta(G) + 1}$ and no
		vertex in $G$ has degree larger than $d$, (1) cannot hold.  Thus (2) must
		hold.  Let $P \DefinedAs (V_1, V_2)$ be the guaranteed partition and suppose
		that for some $j \in \irange{2}$, $G[V_j]$ contains an $r_j$-regular component
		$H$.  Then every vertex of $H$ has degree $d$ in $G$ and
		hence $\mov{H}{d}$ contains all noncutvertices of $H$.  But $H$ has maximum
		degree $r_j$ and thus contains at least $r_j$ noncutvertices.  If $r_j = 1$, then $H$ is $K_2$ and hence has $2$ noncutvertices. In any case,
		we have $\card{\mov{H}{d}} \geq 2$.  Hence (a) cannot hold for $P$.  Thus, by (b),
		we have $i \in \irange{2}$, an $r_i$-regular component $C$ of $G[V_i]$ and $x
		\in V(C)$ such that $N_C(x)$ is a clique.  But then $C$ is $K_{r_i + 1}$
		violating (2), a contradiction.
		
		Therefore, for $i \in \irange{2}$, each component of $G[V_i]$ contains a
		vertex of degree at most $r_i - 1$.  Whence $\text{col}(G[V_i]) \leq r_i$ for
		$i \in \irange{2}$.
		\end{proof}

	\subsection{The coloring corollaries}
	Using Theorem \ref{PartitionTheorem}, we can prove coloring results for graphs
	with only small cliques among the vertices of high degree. To make this
	precise, for $d \in \IN$ define $\omega_d(G)$ to be the size of the largest
	clique in $G$ containing only vertices of degree larger than $d$; that is, $\omega_d(G)
	\DefinedAs \omega\parens{G\brackets{\setbs{v \in V(G)}{d_G(v) > d}}}$.
	
	\begin{cor}\label{FirstColoringCorollary}
	Let $G$ be a graph, $k,d \in \IN$ with $k \geq 2$ and $\vec{r} \in \IN^k$.  If
	$\wt{\vec{r}} \geq \max\set{\Delta(G) + 1 - k, d}$ and $r_i \geq \omega_d(G)
	+ 1$ for all $i \in \irange{k}$, then at least one of the following holds:
			\begin{enumerate}
           \item $\wt{\vec{r}} = d$ and $G$ contains an induced subgraph $Q$ with $\card{Q} = d+1$ which can be partitioned into $k$ cliques $F_1, \ldots, F_k$ where 
					\begin{enumerate}
					\item $\card{F_1} = r_1 + 1$, $\card{F_i} = r_i$ for $i \geq 2$,
					\item $\card{F_i^d} \geq \card{F_i} - \omega_d(G)$ for $i \in \irange{k}$,
					\item for $i \in \irange{k}$, each $v \in V(F_i^d)$ is universal in $Q$;
					\end{enumerate}
			  \item $\chi(G) \leq \wt{\vec{r}}$. 
			\end{enumerate}
	\end{cor}
	\begin{proof}
		Apply Theorem \ref{PartitionTheorem} to conclude that either (1) holds or there exists an $\vec{r}$-partition $P \DefinedAs \parens{V_1, \ldots, V_k}$ of 	
$G$ such that if $C$ is an $r_i$-obstruction in $G[V_i]$, then $\delta_G(C) \geq
d$ and $\mov{C}{d}$ is edgeless.  Since $\Delta(G[V_i]) \leq r_i$ for all $i
\in \irange{k}$, it will be enough to show that no $G[V_i]$ contains an
$r_i$-obstruction.  Suppose otherwise that we have an $r_i$-obstruction $C$ in
some $G[V_i]$.  First, if $r_i \geq 3$, then $C$ is $K_{r_i + 1}$ and hence $C$
contains a $K_{\omega_d(G) + 2}$.  But $\mov{C}{d}$ is edgeless, so
$\omega_d(G) > \omega_d(G) + 1$, a contradiction.  Thus $r_i = 2$ and $C$ is an
odd cycle.  Since $\mov{C}{d}$ is edgeless, the vertices of $C$ are $2$-colored
by the properties `degree is $d$' and `degree is greater than $d$',
impossible.
	\end{proof}

	For a vertex critical graph $G$, call $v \in V(G)$	$\emph{low}$ if $d(v) = \chi(G) - 1$ and $\emph{high}$ otherwise. 
	Let $\fancy{H}(G)$ be the subgraph of $G$ induced on the high vertices of $G$.
	
	\begin{cor}\label{SecondColoring}
	Let $G$ be a vertex critical graph with $\chi(G) = \Delta(G) + 2 - k$ for some
	$k \geq 2$.  If $k \leq \frac{\chi(G) - 1}{\omega(\fancy{H}(G)) + 1}$,
	then $G$ contains an induced subgraph $Q$ with $\card{Q} = \chi(G)$ which can be partitioned into $k$ cliques $F_1, \ldots, F_k$ where 
					\begin{enumerate}
					\item $\card{F_1} = \chi(G) - (k-1)(\omega(\fancy{H}(G)) + 1)$, $\card{F_i} = \omega(\fancy{H}(G)) + 1$ for $i \geq 2$;
					\item for each $i \in \irange{k}$, $F_i$ contains at least $\card{F_i} - \omega(\fancy{H}(G))$ low vertices that are all universal in $Q$.
					\end{enumerate}
	\end{cor}
	\begin{proof}
		Suppose $k \leq \frac{\chi(G) - 1}{\omega(\fancy{H}(G)) + 1}$.  Put
		$r_i \DefinedAs \omega(\fancy{H}(G)) + 1$ for $i \in \irange{k} - \set{1}$ and $r_1
		\DefinedAs \chi(G) - 1 - (k-1)(\omega(\fancy{H}(G)) + 1)$.  Set $\vec{r}
		\DefinedAs \parens{r_1, r_2, \ldots, r_k}$.  Then $\wt{\vec{r}} = \chi(G) - 1
		= \Delta(G) + 1 - k$.  Now applying Corollary \ref{FirstColoringCorollary}
		with $d \DefinedAs \chi(G) - 1$ proves the corollary.
	\end{proof}
	
	\begin{cor}\label{ThirdColoring}
	Let $G$ be a vertex critical graph with $\chi(G) \geq \Delta(G) + 1 - p \geq 4$
	for some $p \in \IN$.  If $\omega(\fancy{H}(G)) \leq \frac{\chi(G) + 1}{p + 1} - 2$,
	then $G = K_{\chi(G)}$ or $G = O_5$.
	\end{cor}
	\begin{proof}
	Suppose not and choose a counterexample $G$ minimizing $\card{G}$.  Put
	$\chi \DefinedAs \chi(G)$, $\Delta \DefinedAs \Delta(G)$ and $h \DefinedAs
	\omega(\fancy{H}(G))$. Then $p \geq 1$ and $h \geq 1$ by Brooks' theorem. Hence
	$\chi \geq 5$. By assumption, we have $h \leq \frac{\chi + 1}{p+1} - 2 =
	\frac{\chi - 2p - 1}{p + 1} \leq \frac{\chi - p - 2}{p + 1}$ since $p \geq 1$. 
	Thus $p + 1 \leq \frac{\chi - 1}{h + 1}$ and we may apply Corollary
	\ref{ThirdColoring} with $k \DefinedAs p + 1$ to get an induced subgraph $Q$ of
	$G$ with $\card{Q} = \chi$ which can be partitioned into $p + 1$ cliques $F_1,
	\ldots, F_{p + 1}$ where
			\begin{enumerate}
					\item $\card{F_1} = \chi - p(h + 1)$, $\card{F_i} = h	+ 1$ for $i \geq 2$;
					\item for each $i \in \irange{p+1}$, $F_i$ contains at least $\card{F_i} -
					h$ low vertices that are all universal in $Q$.
			\end{enumerate}
Let $T$ be the low vertices in $Q$, put $H \DefinedAs Q - T$ and $t
\DefinedAs \card{T}$.  Then $Q = \join{K_t}{H}$ and $t \geq \chi - p(h + 1) +
p(h + 1) - (p + 1)h = \chi - (p + 1)h$.  

Take any $(\chi - 1)$-coloring of $G-Q$ and let $L$ be the resulting list
assignment on $Q$.  Then $\card{L(v)} = d_Q(v)$ for each $v \in T$ and
$\card{L(v)} \geq d_Q(v) - p$ for each $v \in V(H)$.  Since $t \geq \chi - (p +
1)h \geq 2p + 1 \geq p + 1$, if there are nonadjacent $x,y \in V(H)$ and $c \in
L(x) \cap L(y)$, then we may color $x$ and $y$ both with $c$ and then greedily
complete the coloring to the rest of $H$ and then to all of $Q$, a
contradiction.  Hence any nonadjacent pair in $H$ have disjoint lists.

Let $I$ be a maximal independent set in $H$. If there is an induced $P_3$
in $H$ with ends in $I$, set $o_I \DefinedAs 1$, otherwise set $o_I
\DefinedAs 0$. Since each pair of vertices in $I$ have disjoint lists, we must
have

\begin{align*}
	\chi - 1 &\geq \sum_{v \in I} \card{L(v)} \\
	&\geq \sum_{v \in I} t + d_H(v) - p \\
	&= (t-p)\card{I} + \sum_{v \in I} d_H(v) \\
	&\geq (t-p)\card{I} + \card{H} - \card{I} + o_I \\
	&= (t - (p + 1))\card{I} + \chi - t + o_I. 
\end{align*}
Hence $\card{I} \leq \frac{t-1 - o_I}{t - (p + 1)} = 1 +
\frac{p-o_I}{t-(p+1)} \leq 1 + \frac{p-o_I}{2p + 1 - (p+1)} \leq 2$ as $t \geq
2p + 1$.  Since $G$ is not $K_\chi$, we must
have $\card{I} = 2$ and thus $t = 2p + 1$ and $o_I = 0$.  Thence $H$ is the
disjoint union of two complete subgraphs.  We then have $\frac{\chi - 2p - 1}{p
+ 1} \geq h \geq \frac{\card{H}}{2} = \frac{\chi - 2p - 1}{2}$.  Hence $p =
1$, $h = \frac{\chi - 3}{2}$ and $Q = \join{K_3}{2K_h}$.

Let $x,y \in V(H)$ be nonadjacent.  Then $d_Q(x) + d_Q(y) = \chi + 1$.  Let $A$
be the subgraph of $G$ induced on $V(G - Q) \cup \set{x,y}$.  Then $d_A(x) + d_A(y) \leq 2\Delta - (\chi + 1) = \chi - 1$.  Let
$A'$ be the graph obtained by collapsing $\set{x, y}$ to a single vertex
$v_{xy}$. If $\chi(A') \leq \chi - 1$, then we have a $(\chi - 1)$-coloring of
$A$ in which $x$ and $y$ receive the same color.  This is impossible as then we could
complete the $(\chi - 1)$-coloring to all of $G$ greedily as above.  Hence
$\chi(A') = \chi$ and thus we have a vertex critical subgraph $Z$ of $A'$ with
$\chi(Z) = \chi$.  We must have $v_{xy} \in V(Z)$ and since $d_A(x) + d_A(y)
\leq \chi - 1$, $v_{xy}$ is low.  Hence, by minimality of $\card{G}$, $Z =
K_\chi$ or $Z = O_5$.

First, suppose $\chi \geq 6$.  Then $h \geq 2$ and thus we have $z \in V(H) - \set{x, y}$ nonadjacent to $x$.  
Apply the previous paragraph to both pairs $\set{x, y}$ and $\set{x, z}$.  
The case $Z = O_5$ cannot happen, for then we would have $\chi = \chi(Z) = 5$, a contradiction.  
Put $X_1 \DefinedAs N(x) \cap V(G - Q)$, $X_2 \DefinedAs N(y) \cap V(G - Q)$, $X_3 \DefinedAs N(z) \cap V(G - Q)$.  
Then $\card{X_i} = \frac{\chi - 1}{2}$ for $i \in \irange{3}$ and $X_1$ is joined to both $X_2$ and $X_3$.  
Since $\card{X_i} - h > 0$, each $X_i$ contains a low vertex $v_i$.  But then
$N(v_1) = X_1 \cup X_2 \cup \set{x}$ and we must have $X_3 = X_2$. Whence $N(v_2) = X_1 \cup X_2 \cup \set{y, z}$ giving $d(v_2) \geq \chi$, a contradiction.

Therefore $\chi = 5$, $h = 1$ and $V(H) = \set{x, y}$.  If $Z = K_5$, then $N[x] \cup N[y]$ induces an $O_5$ in $G$ and hence $G = O_5$, a contradiction.  
Thus $Z = O_5$.  But $h = 1$, so all of the neighbors of both $x$ and $y$ are
low and hence all of the neighbors of $v_{xy}$ in $Z$ are low. But $O_5$ has no such low vertex $v_{xy}$ with all low neighbors, so this is impossible.
	\end{proof}

\begin{question}
The condition on $k$ needed in Corollary \ref{SecondColoring} is weaker than that in Corollary \ref{ThirdColoring}.  What do the intermediate cases look like?  What are the extremal examples?
\end{question}

\section{Destroying incomplete components in vertex partitions}\label{ShuffleHeightSection}
In \cite{KostochkaTriangleFree} Kostochka modified an algorithm of Catlin \cite{CatlinAnotherBound} to show that every triangle-free graph $G$ can be colored with at most $\frac23 \Delta(G) + 2$ colors.  
In fact, his modification proves that the vertex set of any triangle-free graph $G$ can be partitioned into $\left \lceil \frac{\Delta(G) + 2}{3} \right \rceil$ sets, 
each of which induces a disjoint union of paths. In \cite{rabern2010destroying} we generalized this as follows.

\begin{lem}[Rabern \cite{rabern2010destroying}]\label{DestroyLemma}
Let $G$ be a graph and $r_1, \ldots, r_k \in \IN$ such that $\sum_{i=1}^k r_i \geq \Delta(G) + 2 - k$. Then $V(G)$ can be partitioned into sets $V_1, \ldots, V_k$ such that $\Delta(G[V_i]) \leq r_i$ and $G[V_i]$ contains no incomplete $r_i$-regular components for each $i \in \irange{k}$.
\end{lem}

Setting $k = \left \lceil \frac{\Delta(G) + 2}{3} \right \rceil$ and $r_i = 2$ for each $i$ gives a slightly more general form of Kostochka's theorem.

\begin{cor}[Rabern \cite{rabern2010destroying}]\label{TrianglesAndPaths}
The vertex set of any graph $G$ can be partitioned into $\left \lceil \frac{\Delta(G) + 2}{3} \right \rceil$ sets, each of which induces a disjoint union of triangles and paths.
\end{cor}

For coloring, this actually gives the bound $\chi(G) \leq 2  \left \lceil \frac{\Delta(G) + 2}{3} \right \rceil$ for triangle free graphs.  
To get $\frac23 \Delta(G) + 2$, just use $r_k = 0$ when $\Delta \equiv 2 (\text{mod } 3)$. 
Similarly, for any $r \geq 2$, setting $k = \left \lceil \frac{\Delta(G) + 2}{r + 1} \right \rceil$ and $r_i = r$ for each $i$ gives the following.
\begin{cor}[Rabern \cite{rabern2010destroying}]\label{NoKrPlusOne}
Fix $r \geq 2$.  The vertex set of any $K_{r + 1}$-free graph $G$ can be partitioned into $\left \lceil \frac{\Delta(G) + 2}{r + 1} \right \rceil$ sets each inducing an $(r-1)$-degenerate subgraph with maximum degree at most $r$.
\end{cor}

\noindent For the purposes of coloring it is more economical to split off $\Delta + 2 - (r+1)\left \lfloor \frac{\Delta + 2}{r + 1} \right \rfloor$ parts with $r_j = 0$.

\begin{cor}[Rabern \cite{rabern2010destroying}]
Fix $r \geq 2$.  The vertex set of any $K_{r + 1}$-free graph $G$ can be partitioned into $\left \lfloor \frac{\Delta(G) + 2}{r + 1} \right \rfloor$ sets each inducing an $(r-1)$-degenerate subgraph with maximum degree at most $r$ and $\Delta(G) + 2 - (r+1)\left \lfloor \frac{\Delta(G) + 2}{r + 1} \right \rfloor$ independent sets.  In particular, $\chi(G) \leq \Delta(G) + 2 - \left \lfloor \frac{\Delta(G) + 2}{r + 1} \right \rfloor$.
\end{cor}

For $r \geq 3$, the bound on the chromatic number is only interesting in that its proof does not rely on Brooks' Theorem.  
Lemma \ref{DestroyLemma} is of the same form as Lov\'{a}sz's Lemma
\ref{LovaszDecomposition}, but it gives a more restrictive partition at the cost
of replacing $\Delta(G) + 1$ with $\Delta(G) + 2$.  For $r \geq 3$, combining Lov\'{a}sz's Lemma
\ref{LovaszDecomposition} with Brooks' theorem gives the following better bound for a $K_{r + 1}$-free graph $G$ (first proved in \cite{borodin1977upper}, \cite{catlin1978bound} and \cite{lawrence1978covering}):

\[\chi(G) \leq \Delta(G) + 1 - \left \lfloor \frac{\Delta(G) + 1}{r + 1} \right \rfloor.\]

\bigskip

\subsection{A generalization}
Here we prove a generalization of Lemma \ref{DestroyLemma}.  

\begin{defn}
For $\func{h}{\G}{\IN}$ and $G \in \G$, a vertex $x \in V(G)$ is called \emph{$h$-critical} in $G$ if $G - x \in \G$ and $h(G-x) < h(G)$.
\end{defn}

\begin{defn}
For $\func{h}{\G}{\IN}$ and $G \in \G$, a pair of vertices $\set{x,y} \subseteq V(G)$ is called an \emph{$h$-critical pair} in $G$ if $G - \set{x,y} \in \G$ and $x$ is $h$-critical in $G-y$ and $y$ is $h$-critical in $G-x$.
\end{defn}

\begin{defn}
For $r \in \IN$ a function $\func{h}{\G}{\IN}$ is called an \emph{$r$-height function} if it has each of the following properties:
\begin{enumerate}
\item if $h(G) > 0$, then $G$ contains an $h$-critical vertex $x$ with $d(x) \geq r$;
\item if $G \in \G$ and $x \in V(G)$ is $h$-critical with $d(x) \geq r$, then $h(G-x) = h(G) - 1$;
\item if $G \in \G$ and $x \in V(G)$ is $h$-critical with $d(x) \geq r$, then $G$ contains an $h$-critical vertex $y \not \in \set{x} \cup N(x)$ with $d(y) \geq r$;
\item if $G \in \G$ and $\set{x, y} \subseteq V(G)$ is an $h$-critical pair in $G$ with $d_{G-y}(x) \geq r$ and $d_{G-x}(y) \geq r$, then there exists $z \in N(x) \cap N(y)$ with $d(z) \geq r + 1$.
\end{enumerate}
\end{defn}

\begin{lem}\label{HeightFunctionLemma}
Let $G$ be a graph and $r_1, \ldots, r_k \in \IN$ such that $\sum_{i=1}^k r_i \geq \Delta(G) + 2 - k$. If $h_i$ is an $r_i$-height function for each $i \in \irange{k}$, then $V(G)$ can be partitioned into sets $V_1, \ldots, V_k$ such that for each $i \in \irange{k}$, $\Delta(G[V_i]) \leq r_i$ and $h_i(D) = 0$ for each component $D$ of $G[V_i]$.
\end{lem}

\noindent For each $r \in \IN$, it is easy to see that the function $\func{h_r}{\G}{\IN}$ defined as follows is an $r$-height function:

\[h_r(G) \DefinedAs 
\begin{cases}
1 & \text{$G$ is incomplete and $r$-regular;}\\
0 & \text{otherwise.}
\end{cases}\]

Applying Lemma \ref{HeightFunctionLemma} with these height functions proves Lemma \ref{DestroyLemma}.  Other height functions exist, but we don't yet have a sense of their ubiquity or lack thereof.


\begin{proof}[Proof of Lemma \ref{HeightFunctionLemma}]
For a partition $P \DefinedAs \parens{V_1, \ldots, V_k}$ of $V(G)$ let

\[f(P) \DefinedAs \sum_{i=1}^k \parens{\size{G[V_i]} - r_i\card{V_i}},\]
\[c(P) \DefinedAs \sum_{i=1}^k c(G[V_i]),\]
\[h(P) \DefinedAs \sum_{i=1}^k h_i(G[V_i]).\]

\noindent Let $P \DefinedAs \parens{V_1, \ldots, V_k}$ be a partition of $V(G)$ minimizing $f(P)$, and subject to that $c(P)$, and subject to that $h(P)$.

Let $i \in \irange{k}$ and $x \in V_i$ with $d_{V_i}(x) \geq r_i$.  Since $\sum_{i=1}^k r_i \geq \Delta(G) + 2 - k$ there is some $j \neq i$ such that $d_{V_j}(x) \leq r_j$.  Moving $x$ from $V_i$ to $V_j$ gives a new partition $P^*$ with $f(P^*) \leq f(P)$.  Note that if $d_{V_i}(x) > r_i$ we would have $f(P^*) < f(P)$ contradicting the minimality of $P$. This proves that $\Delta(G[V_i]) \leq r_i$ for each $i \in \irange{k}$.

Now suppose that for some $i_1$ there is a component $A_1$ of $G[V_{i_1}]$ with
$h_{i_1}(A_1) > 0$. Put $P_1 \DefinedAs P$ and $V_{1,i} \DefinedAs V_i$ for $i \in \irange{k}$. By property 1 of height functions, we have an $h_{i_1}$-critical vertex $x_1 \in V(A_1)$ with $d_{A_1}(x_1) \geq r_{i_1}$.  By the above we have $i_2 \neq i_1$ such that moving $x_1$ from $V_{1, i_1}$ to $V_{1, i_2}$ gives a new partition $P_2 \DefinedAs \parens{V_{2, 1}, V_{2,2}, \ldots, V_{2,k}}$ where $f(P_2) = f(P_1)$.  By the minimality of $c(P_1)$, $x_1$ is adjacent to only one component $C_2$ in $G[V_{1, i_2}]$. Let $A_2 \DefinedAs G[V(C_2) \cup \set{x_1}]$.  Since $x_1$ is $h_{i_1}$-critical, by the minimality of $h(P_1)$, it must be that $h_{i_2}(A_2) > h_{i_2}(C_2)$.  By property 2 of height functions we must have $h_{i_2}(A_2) = h_{i_2}(C_2) + 1$.  Hence $h(P_2)$ is still minimum.  Now, by property 3 of height functions, we have an $h_{i_2}$-critical vertex $x_2 \in V(A_2) - \parens{\set{x_1} \cup N_{A_2}(x_1)}$ with $d_{A_2}(x_2) \geq r_{i_2}$.

Continue on this way to construct sequences $i_1, i_2, \ldots$, $A_1, A_2, \ldots$, $P_1, P_2, P_3, \ldots$ and $x_1, x_2, \ldots$.  Since $G$ is finite, at some point we will need to reuse a leftover component; that is, there is a smallest $t$ such that $A_{t + 1} - x_t = A_s - x_s$ for some $s < t$.  In particular, $\set{x_s, x_{t+1}}$ is an $h_{i_s}$-critical pair in  $Q \DefinedAs G\left[\set{x_{t+1}} \cup V(A_s)\right]$ where $d_{Q-x_{t+1}}(x_s) \geq r_{i_s}$ and $d_{Q-x_s}(x_{t+1}) \geq r_{i_s}$.  Thus, by property 4 of height functions, we have $z \in N_Q(x_s) \cap N_Q(x_{t+1})$ with $d_Q(z) \geq r_{i_s} + 1$.

We now modify $P_s$ to contradict the minimality of $f(P)$.  At step $t+1$,  $x_t$ was adjacent to exactly $r_{i_s}$ vertices in $V_{t+1, i_s}$. This is what allowed us to move $x_t$ into $V_{t+1, i_s}$.  Our goal is to modify $P_s$ so that we can move $x_t$ into the $i_s$ part without moving $x_s$ out. Since $z$ is adjacent to both $x_s$ and $x_t$, moving $z$ out of the $i_s$ part will then give us our desired contradiction.  

So, consider the set $X$ of vertices that could have been moved out of $V_{s, i_s}$ between step $s$ and step $t+1$; that is, $X \DefinedAs \set{x_{s+1}, x_{s+2}, \ldots, x_{t-1}} \cap V_{s, i_s}$.  For $x_j \in X$, since $d_{A_j}(x_j) \geq r_{i_s}$ and $x_j$ is not adjacent to $x_{j-1}$ we see that $d_{V_{s, i_s}}(x_j) \geq r_{i_s}$.  Similarly, $d_{V_{s, i_t}}(x_t) \geq r_{i_t}$. Also, by the minimality of $t$, $X$ is an independent set in $G$.  Thus we may move all elements of $X$ out of $V_{s, i_s}$ to get a new partition $P^* \DefinedAs \parens{V_{*, 1}, \ldots, V_{*, k}}$ with $f(P^*) = f(P)$. 

Since $x_t$ is adjacent to exactly $r_{i_s}$ vertices in $V_{t+1, i_s}$ and the only possible neighbors of $x_t$ that were moved out of $V_{s, i_s}$ between steps $s$ and $t+1$ are the elements of $X$, we see that $d_{V_{*, i_s}}(x_t) = r_{i_s}$.  Since $d_{V_{*, i_t}}(x_t) \geq r_{i_t}$ we can move $x_t$ from $V_{*, i_t}$ to $V_{*, i_s}$ to get a new partition $P^{**} \DefinedAs \parens{V_{**, 1}, \ldots, V_{**, k}}$ with $f(P^{**}) = f(P^*)$.  Now, recall that $z \in V_{**, i_s}$.  Since $z$ is adjacent to $x_t$ we have $d_{V_{**, i_s}}(z) \geq r_{i_s} + 1$.  Thus we may move $z$ out of $V_{**, i_s}$ to get a new partition $P^{***}$ with $f(P^{***}) < f(P^{**}) = f(P)$.  This contradicts the minimality of $f(P)$.
\end{proof}

\documentclass[12pt]{article}
\usepackage{amsmath, amsthm, amssymb}
\usepackage{tkz-graph}
\usepackage{marginnote}
\usepackage{verbatim}
\usepackage[top=1.0in, bottom=1.0in, left=1.0in, right=1.0in]{geometry}
\usepackage{color}
\pagestyle{plain}

\usepackage[backref=page]{hyperref}

\usepackage{sectsty}
\allsectionsfont{\sffamily}

\setcounter{secnumdepth}{5}
\setcounter{tocdepth}{5}

\makeatletter
\newtheorem*{rep@theorem}{\rep@title}
\newcommand{\newreptheorem}[2]{
\newenvironment{rep#1}[1]{
 \def\rep@title{#2 \ref{##1}}
 \begin{rep@theorem}}
 {\end{rep@theorem}}}
\makeatother

\theoremstyle{plain}
\newtheorem{thm}{Theorem}
\newreptheorem{thm}{Theorem}
\newtheorem{prop}[thm]{Proposition}
\newreptheorem{prop}{Proposition}
\newtheorem{lem}[thm]{Lemma}
\newreptheorem{lem}{Lemma}
\newtheorem{conjecture}[thm]{Conjecture}
\newreptheorem{conjecture}{Conjecture}
\newtheorem{cor}[thm]{Corollary}
\newreptheorem{cor}{Corollary}
\newtheorem{prob}[thm]{Problem}

\newtheorem*{KernelLemma}{Kernel Lemma}
\newtheorem*{MainTheorem}{Main Theorem}
\newtheorem*{BK}{Borodin-Kostochka Conjecture}
\newtheorem*{BK2}{Borodin-Kostochka Conjecture (restated)}
\newtheorem*{Reed}{Reed's Conjecture}
\newtheorem*{ClassificationOfd0}{Classification of $d_0$-choosable graphs}


\theoremstyle{definition}
\newtheorem{defn}{Definition}
\theoremstyle{remark}
\newtheorem*{remark}{Remark}
\newtheorem*{problem}{Problem}
\newtheorem{example}{Example}
\newtheorem*{question}{Question}
\newtheorem*{observation}{Observation}

\newcommand{\fancy}[1]{\mathcal{#1}}
\newcommand{\C}[1]{\fancy{C}_{#1}}


\newcommand{\IN}{\mathbb{N}}
\newcommand{\IR}{\mathbb{R}}
\newcommand{\G}{\fancy{G}}
\newcommand{\CC}{\fancy{C}}
\newcommand{\D}{\fancy{D}}
\newcommand{\T}{\fancy{T}}
\newcommand{\B}{\fancy{B}}
\renewcommand{\L}{\fancy{L}}
\newcommand{\HH}{\fancy{H}}

\newcommand{\inj}{\hookrightarrow}
\newcommand{\surj}{\twoheadrightarrow}

\newcommand{\set}[1]{\left\{ #1 \right\}}
\newcommand{\setb}[3]{\left\{ #1 \in #2 : #3 \right\}}
\newcommand{\setbs}[2]{\left\{ #1 : #2 \right\}}
\newcommand{\card}[1]{\left|#1\right|}
\newcommand{\size}[1]{\left\Vert#1\right\Vert}
\newcommand{\ceil}[1]{\left\lceil#1\right\rceil}
\newcommand{\floor}[1]{\left\lfloor#1\right\rfloor}
\newcommand{\func}[3]{#1\colon #2 \rightarrow #3}
\newcommand{\funcinj}[3]{#1\colon #2 \inj #3}
\newcommand{\funcsurj}[3]{#1\colon #2 \surj #3}
\newcommand{\irange}[1]{\left[#1\right]}
\newcommand{\join}[2]{#1 \mbox{\hspace{2 pt}$\ast$\hspace{2 pt}} #2}
\newcommand{\djunion}[2]{#1 \mbox{\hspace{2 pt}$+$\hspace{2 pt}} #2}
\newcommand{\parens}[1]{\left( #1 \right)}
\newcommand{\brackets}[1]{\left[ #1 \right]}
\newcommand{\DefinedAs}{\mathrel{\mathop:}=}

\newcommand{\mic}{\operatorname{mic}}
\newcommand{\AT}{\operatorname{AT}}
\newcommand{\col}{\operatorname{col}}
\newcommand{\ch}{\operatorname{ch}}
\newcommand{\type}{\operatorname{type}}
\newcommand{\nonsep}{\bar{S}}

\def\adj{\leftrightarrow}
\def\nonadj{\not\!\leftrightarrow}

\newcommand\restr[2]{{% we make the whole thing an ordinary symbol
  \left.\kern-\nulldelimiterspace % automatically resize the bar with \right
  #1 % the function
  \vphantom{\big|} % pretend it's a little taller at normal size
  \right|_{#2} % this is the delimiter
  }}

\def\D{\fancy{D}}
\def\C{\fancy{C}}
\def\A{\fancy{A}}
\def\chil{{\chi_\ell}}
\def\chiol{\chi_{\rm{OL}}}

\newcommand{\case}[2]{{\bf Case #1.}~{\it #2}~~}
\newcommand{\aside}[1]{\marginnote{\scriptsize{#1}}[0cm]}
\newcommand{\aaside}[2]{\marginnote{\scriptsize{#1}}[#2]}

\newcommand\numberthis{\addtocounter{equation}{1}\tag{\theequation}}

\title{}
\author{}

\begin{document}
\maketitle

For a graph $G$, let $\beta_k(G)$ be the independence number of the subgraph of $G$ induced on the vertices of degree $k-1$.  
When $k$ is defined in context, we just write $\beta(G)$.  Let $\HH(G)$ be the subgraph of $G$ induced on vertices of degree greater than $\delta(G)$.
Let $\L(G)$ be the subgraph of $G$ induced on vertices of degree $\delta(G)$.
\begin{defn} The \emph{maximum independent cover number }of a graph $G$
	is the maximum $\mic(G)$ of $\size{I, V(G) \setminus I}$ over all independent sets $I$
	of $G$. 
\end{defn}

\begin{defn} A graph $G$ is \emph{OC-reducible} to $H$ if $H$ is a nonempty induced
subgraph of $G$ which is online $f_{H}$-choosable where $f_{H}(v)\DefinedAs\delta(G)+d_{H}(v)-d_{G}(v)$
for all $v\in V(H)$. If $G$ is not OC-reducible to any nonempty induced subgraph,
then it is \emph{OC-irreducible}. 
\end{defn}

\begin{lem}\label{ConsantListMicStrength} 
	Every OC-irreducible graph $G$ satisfies
	\[2\size{G} > (\delta(G) - 1)\card{G} + \mic(G).\]
\end{lem}

\begin{lem}\label{LBound}
If $G$ is an OC-irreducible graph where $\HH(G)$ is edgeless, $\Delta \DefinedAs \Delta(G) = \delta(G) + 1$ and $\L\DefinedAs\L(G)$, then
\[2\size{\L} > \parens{\Delta-2 - \frac{2}{\Delta-2}}\card{\L} + \frac{\Delta(\Delta-1)}{\Delta-2}\beta_{\Delta}(\L).\]
\end{lem}
\begin{proof}
Let $G$ be such a graph. Put $\HH \DefinedAs \HH(G)$ and $\L \DefinedAs \L(G)$.  Since
$\HH$ is edgeless,
\begin{align*}
\Delta\card{\HH}&= \size{\HH, \L}\\
&=(\Delta-1)\card{\L} - 2\size{\L},\numberthis \label{ha}\\
\end{align*}
so, by Lemma \ref{ConsantListMicStrength},

\begin{align*}
(\Delta-1)\card{\L} + \Delta\card{\HH} &= 2\size{G}\\
&> (\Delta-2)\card{G} + \mic(G) \\
&\ge (\Delta-2)\card{G} + \Delta\card{\HH} + (\Delta-1)\beta_{\Delta}(\L)\\
&= (\Delta-2)\card{\L} + (2\Delta-2)\card{\HH} + (\Delta-1)\beta_{\Delta}(\L),\\
\end{align*}
so simplifying and using \eqref{ha} again gives
\begin{align*}
\card{\L} &> (\Delta-2)\card{\HH} + (\Delta-1)\beta_{\Delta}(\L)\\
&= \frac{\Delta-2}{\Delta}\parens{(\Delta-1)\card{\L} - 2\size{\L}} + (\Delta-1)\beta_{\Delta}(\L),
\end{align*}
now some mild manipulation yields the desired bound. 
\end{proof}

\begin{defn}
A quadruple $\parens{p,h,z,f}$ of functions from $\IN$ to $\IR$ is \emph{$r$-Gallai} if for every $k \ge r$ and Gallai tree $T \ne K_k$ with $\Delta(T) \le k-1$,
the following hold:
\begin{itemize}
\item if $K_{k-1} \subseteq T$, then $2\size{T} \le \parens{k-3 + p(k)}\card{T} + h(k)q(T) + z(k)\beta(T) + f(k)$; and
\item if $K_{k-1} \not\subseteq T$, then $2\size{T} \le \parens{k-3 + p(k)}\card{T} + z(k)\beta(T)$.
\end{itemize}
\end{defn}

\begin{lem}
If $\func{z}{\IN}{\IR}$ is such that $z(k) = 0$ or $2 \le z(k) \le \frac{k(k-3)}{k-2}$  for all $k \in \IN$, then 
$(p,h,z,f)$ is $5$-Gallai, where
\[h(k) \DefinedAs \frac{k(k-3) - (k-2)z(k)}{k^2-4k+5},\]
\[p(k) \DefinedAs \frac{2 + h(k)}{k-2},\]
\[f(k) \DefinedAs (k-1)(1 - h(k) - p(k)).\]
\end{lem}

\begin{cor}\label{GBound}
$\parens{\frac{2}{k-2}, 0, \frac{k(k-3)}{k-2}, \frac{(k-1)(k-4)}{k-2}}$ is $5$-Gallai.
\end{cor}

\begin{thm}\label{LBoundT}

\end{thm}
\begin{proof}
Combining Lemma \ref{LBound} and Corollary \ref{GBound} gives
\begin{align*}
&\parens{\Delta-2 - \frac{2}{\Delta-2}}\card{\L} + \frac{\Delta(\Delta-1)}{\Delta-2}\beta_{\Delta}(\L) <\\
&\parens{\Delta - 3 + \frac{2}{\Delta-2}}\card{\L} + \frac{\Delta(\Delta-3)}{\Delta-2}\beta_{\Delta}(\L) + \frac{(\Delta-1)(\Delta-4)}{\Delta-2}c_0(\L),\\
\end{align*}
so
\[(\Delta-6)\card{\L} + 2\Delta\beta_{\Delta}(\L) < (\Delta-1)(\Delta-4)c_0(\L).\]
\end{proof}
\end{document} 
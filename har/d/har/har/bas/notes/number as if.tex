\documentclass[12pt]{article}
\usepackage{amsmath, amssymb}

\newcommand{\fancy}[1]{\mathcal{#1}}
\newcommand{\C}[1]{\fancy{C}_{#1}}


\newcommand{\IN}{\mathbb{N}}
\newcommand{\IZ}{\mathbb{Z}}
\newcommand{\IR}{\mathbb{R}}
\newcommand{\G}{\fancy{G}}
\newcommand{\CC}{\fancy{C}}
\newcommand{\D}{\fancy{D}}
\newcommand{\T}{\fancy{T}}
\newcommand{\B}{\fancy{B}}
\renewcommand{\L}{\fancy{L}}
\newcommand{\HH}{\fancy{H}}

\newcommand{\inj}{\hookrightarrow}
\newcommand{\surj}{\twoheadrightarrow}

\newcommand{\set}[1]{\left\{ #1 \right\}}
\newcommand{\setb}[3]{\left\{ #1 \in #2 : #3 \right\}}
\newcommand{\setbs}[2]{\left\{ #1 : #2 \right\}}
\newcommand{\card}[1]{\left|#1\right|}
\newcommand{\size}[1]{\left\Vert#1\right\Vert}
\newcommand{\ceil}[1]{\left\lceil#1\right\rceil}
\newcommand{\floor}[1]{\left\lfloor#1\right\rfloor}
\newcommand{\func}[3]{#1\colon #2 \rightarrow #3}
\newcommand{\funcinj}[3]{#1\colon #2 \inj #3}
\newcommand{\funcsurj}[3]{#1\colon #2 \surj #3}
\newcommand{\irange}[1]{\left[#1\right]}
\newcommand{\join}[2]{#1 \mbox{\hspace{2 pt}$\ast$\hspace{2 pt}} #2}
\newcommand{\djunion}[2]{#1 \mbox{\hspace{2 pt}$+$\hspace{2 pt}} #2}
\newcommand{\parens}[1]{\left( #1 \right)}
\newcommand{\brackets}[1]{\left[ #1 \right]}
\newcommand{\DefinedAs}{\mathrel{\mathop:}=}

\newcommand{\mic}{\operatorname{mic}}
\newcommand{\AT}{\operatorname{AT}}
\newcommand{\col}{\operatorname{col}}
\newcommand{\ch}{\operatorname{ch}}
\newcommand{\type}{\operatorname{type}}
\newcommand{\nonsep}{\bar{S}}
\newcommand{\type}{\operatorname{type}}
\def\adj{\leftrightarrow}
\def\nonadj{\not\!\leftrightarrow}
\newcommand{\gcd}{\operatorname{gcd}}

\newcommand\restr[2]{{% we make the whole thing an ordinary symbol
  \left.\kern-\nulldelimiterspace % automatically resize the bar with \right
  #1 % the function
  \vphantom{\big|} % pretend it's a little taller at normal size
  \right|_{#2} % this is the delimiter
  }}

\def\D{\fancy{D}}
\def\C{\fancy{C}}
\def\A{\fancy{A}}

\newcommand{\claim}[2]{{\bf Claim #1.}~{\it #2}~~}
\newcommand{\case}[2]{{\bf Case #1.}~{\it #2}~~}
\newcommand\numberthis{\addtocounter{equation}{1}\tag{\theequation}}

\def\gcd{\bigtriangledown}
\def\lcm{\bigtriangleup}
\def\no{\natural}

\title{number as if}
\begin{document}
\maketitle

The inhabitants of the land of Nog look, act and sound just like us.  But they differ in number,
not in their count, but in their manner. In our mind a number is a quantity, a stack of like items, but in Nog
a number is a crystal of facets in infinite dimensional space. Workers of Nog regularly combine numbers in crystal form
to create new numbers.  Outsiders have been able to replicate a few of these techniques.   For example, given numbers $n$ and $m$ in crystal form, the
number $n\star m$ is made by letting each dimension aggregate individually. The number $n\gcd m$ is made by taking the
smallest in each dimension.  The number $n\lcm m$ is made by taking the
largest in each dimension.  There is a technique $n\dagger m$ practiced only by the great masters of Nog.  The $\dagger$ technique is a powerful
tool in the hands of these masters.  I have personally witnessed a master create extraordinarily intricate crystal structure with just one application of $\dagger$, starting from simple
numbers in crystal form.  I have spent the better part of my life here, as a wanderer in the great library of Nog.  In many directions my pursuit of knowledge has led me back to $\dagger$
in one form or another.  My contention is that in $\dagger$ the masters have tapped into a process that the universe performs but does not explain. If we can replicate their method
we will achieve computational power beyond what appears possible.

The Nogling masters claim to be able to visualize these infinite dimensional crystals.  Such visualization is probably a good guide to their intuition when dealing with $\dagger$ technique.
We lack this visualization, so we must proceed in a more pedestrian, more formal manner.
The basic properties of $\dagger$ are well known.  I will go through these quickly before we get on the the more difficult topics.  

\begin{figure}
\centering
\begin{tabular}{l|l}
$\dagger \text{ is associative}$ & $a\dagger(b\dagger c) = (a\dagger b) \dagger c$\\
$\dagger \text{ is commutative}$ & $a \dagger b = b\dagger a$\\
$\dagger \text{ is distributive}$ & $a\star (b \dagger c) = (a\star b)\dagger(a\star c)$\\
\end{tabular}
\end{figure}

It will be useful to have names for the basic crystal forms.  The number with no facets is $\no$.
The numbers with one facet are $\mho_0,\mho_1,\mho_2, \ldots$.  To get a feel for $\star$ and $\dagger$, here are a few known constructions.
\begin{align*}
\no \star a &= a\\
a \dagger a &= \mho_0 \star a\\
\no \dagger \no &= \mho_0\\
\no \dagger \no \dagger \no &= \mho_1\\
\no \dagger \no \dagger \no \dagger \no&= \mho_0 \star \mho_0\\
\no \dagger \no \dagger \no \dagger \no \dagger \no&= \mho_2\\
\no \dagger \no \dagger \no \dagger \no \dagger \no \dagger \no&= \mho_0 \star \mho_1\\
\no \dagger \no \dagger \no \dagger \no \dagger \no \dagger \no \dagger \no&= \mho_3\\
\end{align*}
\end{document}


\documentclass[12pt]{article}
\usepackage{amsmath, amssymb}

\newtheorem{acknowledgement}{Acknowledgement}
\newtheorem{algorithm}{Algorithm}
\newtheorem{axiom}{Axiom}
\newtheorem{case}{Case}
\newtheorem{claim}{Claim}
\newtheorem{conclusion}{Conclusion}
\newtheorem{condition}{Condition}
\newtheorem{conjecture}{Conjecture}
\newtheorem{corollary}{Corollary}
\newtheorem{criterion}{Criterion}
\newtheorem{definition}{Definition}
\newtheorem{example}{Example}
\newtheorem{exercise}{Exercise}
\newtheorem{lemma}{Lemma}
\newtheorem{notation}{Notation}
\newtheorem{problem}{Problem}
\newtheorem{proposition}{Proposition}
\newtheorem{remark}{Remark}
\newtheorem{solution}{Solution}
\newtheorem{summary}{Summary}
\newtheorem{theorem}{Theorem}


\newcommand{\fancy}[1]{\mathcal{#1}}
\newcommand{\C}[1]{\fancy{C}_{#1}}


\newcommand{\IN}{\mathbb{N}}
\newcommand{\IZ}{\mathbb{Z}}
\newcommand{\IR}{\mathbb{R}}
\newcommand{\G}{\fancy{G}}
\newcommand{\CC}{\fancy{C}}
\newcommand{\D}{\fancy{D}}
\newcommand{\T}{\fancy{T}}
\newcommand{\B}{\fancy{B}}
\renewcommand{\L}{\fancy{L}}
\newcommand{\HH}{\fancy{H}}

\newcommand{\inj}{\hookrightarrow}
\newcommand{\surj}{\twoheadrightarrow}

\newcommand{\set}[1]{\left\{ #1 \right\}}
\newcommand{\setb}[3]{\left\{ #1 \in #2 : #3 \right\}}
\newcommand{\setbs}[2]{\left\{ #1 : #2 \right\}}
\newcommand{\card}[1]{\left|#1\right|}
\newcommand{\size}[1]{\left\Vert#1\right\Vert}
\newcommand{\ceil}[1]{\left\lceil#1\right\rceil}
\newcommand{\floor}[1]{\left\lfloor#1\right\rfloor}
\newcommand{\func}[3]{#1\colon #2 \rightarrow #3}
\newcommand{\funcinj}[3]{#1\colon #2 \inj #3}
\newcommand{\funcsurj}[3]{#1\colon #2 \surj #3}
\newcommand{\irange}[1]{\left[#1\right]}
\newcommand{\join}[2]{#1 \mbox{\hspace{2 pt}$\ast$\hspace{2 pt}} #2}
\newcommand{\djunion}[2]{#1 \mbox{\hspace{2 pt}$+$\hspace{2 pt}} #2}
\newcommand{\parens}[1]{\left( #1 \right)}
\newcommand{\brackets}[1]{\left[ #1 \right]}
\newcommand{\DefinedAs}{\mathrel{\mathop:}=}

\newcommand{\mic}{\operatorname{mic}}
\newcommand{\AT}{\operatorname{AT}}
\newcommand{\col}{\operatorname{col}}
\newcommand{\ch}{\operatorname{ch}}
\newcommand{\type}{\operatorname{type}}
\newcommand{\nonsep}{\bar{S}}
\newcommand{\type}{\operatorname{type}}
\def\adj{\leftrightarrow}
\def\nonadj{\not\!\leftrightarrow}
\newcommand{\gcd}{\operatorname{gcd}}

\newcommand\restr[2]{{% we make the whole thing an ordinary symbol
  \left.\kern-\nulldelimiterspace % automatically resize the bar with \right
  #1 % the function
  \vphantom{\big|} % pretend it's a little taller at normal size
  \right|_{#2} % this is the delimiter
  }}

\def\D{\fancy{D}}
\def\C{\fancy{C}}
\def\A{\fancy{A}}

\newcommand{\claim}[2]{{\bf Claim #1.}~{\it #2}~~}
\newcommand{\case}[2]{{\bf Case #1.}~{\it #2}~~}
\newcommand\numberthis{\addtocounter{equation}{1}\tag{\theequation}}

\def\gcd{\bigtriangledown}
\def\lcm{\bigtriangleup}
\def\no{\natural}


\usepackage{tikz}
\usetikzlibrary{calc}

\pgfdeclarelayer{background}
\pgfsetlayers{background,main}
\newcommand{\Bond}[6]%
% start, end, thickness, incolor, outcolor, iterations
{ \begin{pgfonlayer}{background}
        \colorlet{InColor}{#4}
        \colorlet{OutColor}{#5}
        \foreach \I in {#6,...,1}
        {   \pgfmathsetlengthmacro{\r}{#3/#6*\I}
            \pgfmathsetmacro{\C}{sqrt(1-\r*\r/#3/#3)*100}
            \draw[InColor!\C!OutColor, line width=\r] (#1.center) -- (#2.center);
        }
    \end{pgfonlayer}
}

\newcommand{\BlackBond}[2]%
% start, end
{   \Bond{#1}{#2}{0.7071mm}{black!25}{black!25!black}{10}
}


\begin{document}

\section{Early encounters}
\begin{figure}
\centering
\begin{tikzpicture}
[   AA/.style={circle, ball color=gray!75, minimum size=16mm, inner sep=0},
]
   \node[AA] (A1) at (0,0) {};
\end{tikzpicture}
\caption{The white ball.}
\end{figure}

I saw objects sort of like molecules, balls of different colors joined by bars.   Pairs of objects were placed in
some kind of machine with four buttons, labeled $\star, \lcm, \gcd$ and $\dagger$.  After the pressing one of the buttons the original
objects were gone, replaced by a new object.  He showed me many demonstrations of this, pressing different buttons, with differently colored balls coming out
connected in various ways.  It was clear that he we trying to show me what this machine did, the look on his face said ``you see, yes?''  I tried to write down as much
as I could afterwards, hoping that he would show me more next time.  The colors definitely have meaning, there is a structure there.  Two white balls
make a red ball when you push $\dagger$ and a red and white ball make a blue ball when you push $\dagger$ (whether you blue white in the first or second slot
does not appear to matter).  He showed me a few with the $\star$ button as well.  A red and blue ball with $\star$ makes just joins the balls by a black bar.  
\begin{figure}
\centering
\begin{tikzpicture}
[   AA/.style={circle, ball color=gray!75, minimum size=16mm, inner sep=0},
	BB/.style={circle, ball color=red!75, minimum size=16mm, inner sep=0},
]
    \node[AA] (A1) at (0,0) {};
	\node at (1.5,0) {\huge $\dagger$};
	\node[AA] (A2) at (3,0) {};
	\node at (4.5,0) {\huge $=$};
	\node[BB] (B1) at (6,0) {};
\end{tikzpicture}
\caption{Two white balls in the machine produced a red ball when the $\dagger$ button was pressed.}
\end{figure}

\begin{figure}
\centering
\begin{tikzpicture}
[   AA/.style={circle, ball color=gray!75, minimum size=16mm, inner sep=0},
	BB/.style={circle, ball color=red!75, minimum size=16mm, inner sep=0},
	CC/.style={circle, ball color=blue!75, minimum size=16mm, inner sep=0},
]
    \node[AA] (A1) at (0,0) {};
	\node at (1.5,0) {\huge $\dagger$};
	\node[BB] (A2) at (3,0) {};
	\node at (4.5,0) {\huge $=$};
	\node[CC] (B1) at (6,0) {};
\end{tikzpicture}
\caption{A white and a red ball in the machine produced a blue ball when the $\dagger$ button was pressed.}
\end{figure}

\begin{figure}
\centering
\begin{tikzpicture}
[   AA/.style={circle, ball color=red!75, minimum size=10mm, inner sep=0},
BB/.style={circle, ball color=blue!75, minimum size=10mm, inner sep=0},
CC/.style={circle, ball color=gray!75, minimum size=10mm, inner sep=0}
]
    \node[AA] (A1) at (0,0) {};
	\node at (1,0) {\huge $\star$};
	\node[BB] (A2) at (2,0) {};
	\node at (3,0) {\huge $=$};
	\node[AA] (A3) at (4,0) {};
	\node[BB] (A4) at (6,0) {};
	\BlackBond{A3}{A4}
	
	\begin{scope}[shift = {(8,0)}]
	\node[AA] (A1) at (0,0) {};
	\node at (1,0) {\huge $\star$};
	\node[CC] (A2) at (2,0) {};
	\node at (3,0) {\huge $=$};
	\node[AA] (A3) at (4,0) {};
	\end{scope}
	
	\begin{scope}[shift = {(8,-3)}]
	\node[BB] (A1) at (0,0) {};
	\node at (1,0) {\huge $\star$};
	\node[CC] (A2) at (2,0) {};
	\node at (3,0) {\huge $=$};
	\node[BB] (A3) at (4,0) {};
	\end{scope}
	
	\begin{scope}[shift = {(0,-3)}]
	 \node[AA] (A1) at (0,0) {};
	\node at (1,0) {\huge $\star$};
	\node[AA] (A2) at (2,0) {};
	\node at (3,0) {\huge $=$};
	\node[AA] (A3) at (4,0) {};
	\node[AA] (A4) at (6,0) {};
	\BlackBond{A3}{A4}
\end{scope}
\end{tikzpicture}
\caption{}
\end{figure}

I thought I was starting to see the pattern, but then he showed me that a white and blue ball with $\star$ gives just a blue ball.  Then things
started to get really weird, with new ball colors appearing with no immediately apparent pattern.  He showed my more images than I readily
reproduce without it becoming overwhelming.  

\begin{figure}
\centering
\begin{tikzpicture}
[   AA/.style={circle, ball color=red!75, minimum size=8mm, inner sep=0},
BB/.style={circle, ball color=blue!75, minimum size=8mm, inner sep=0},
CC/.style={circle, ball color=green!75, minimum size=8mm, inner sep=0}
]

\begin{scope}[shift = {(0,0)}]
	\node[AA] (A3) at (0,0) {};
	\node[AA] (A4) at (0,2) {};
	\BlackBond{A3}{A4}
\end{scope}

\node at (1,1) {\huge $\dagger$};

  \begin{scope}[shift = {(2,0)}]
	\node[AA] (A3) at (0,0) {};
	\node[BB] (A4) at (0,2) {};
	\BlackBond{A3}{A4}
\end{scope}

\node at (3,1) {\huge $=$};
	
	  \begin{scope}[shift = {(4,0)}]
	\node[AA] (A3) at (0,0) {};
	\node[CC] (A4) at (0,2) {};
	\BlackBond{A3}{A4}
\end{scope}
\end{tikzpicture}
\caption{}
\end{figure}

\begin{figure}
\centering
\begin{tikzpicture}
[   
Z1/.style={circle, ball color=gray!75, minimum size=8mm, inner sep=0},
Z2/.style={circle, ball color=red!75, minimum size=8mm, inner sep=0},
Z3/.style={circle, ball color=blue!75, minimum size=8mm, inner sep=0},
Z5/.style={circle, ball color=green!75, minimum size=8mm, inner sep=0},
Z7/.style={circle, ball color=yellow!75, minimum size=8mm, inner sep=0},
Z11/.style={circle, ball color=cyan!75, minimum size=8mm, inner sep=0},
Z13/.style={circle, ball color=magenta!75, minimum size=8mm, inner sep=0}
]

	  \begin{scope}[shift = {(0,0)}]
	\node[Z2] (A3) at (0,0) {};
	\node[Z5] (A4) at (0,2) {};
	\BlackBond{A3}{A4}
\end{scope}
\node at (1,1) {\huge $\dagger$};

  \begin{scope}[shift = {(2,0)}]
	\node[Z1] (A3) at (0,1) {};

\end{scope}

\node at (3,1) {\huge $=$};

  \begin{scope}[shift = {(4,0)}]
	\node[Z11] (A3) at (0,1) {};
\end{scope}

\begin{scope}[shift = {(6,0)}]
	  \begin{scope}[shift = {(0,0)}]
	\node[Z5] (A4) at (0,1) {};
\end{scope}
\node at (1,1) {\huge $\dagger$};

  \begin{scope}[shift = {(2,0)}]
	\node[Z11] (A3) at (0,1) {};

\end{scope}

\node at (3,1) {\huge $=$};

  \begin{scope}[shift = {(4,0)}]
	\node[Z2] (A1) at (0,0) {};
	\node[Z2] (A2) at (0,2) {};
	\node[Z2] (A3) at (2,0) {};
	\node[Z2] (A4) at (2,2) {};
	\BlackBond{A1}{A3}
	\BlackBond{A1}{A2}
	\BlackBond{A4}{A2}
	\BlackBond{A3}{A4}
\end{scope}
\end{scope}
\end{tikzpicture}
\end{figure}

\begin{figure}
\centering
\begin{tikzpicture}
[   
Z1/.style={circle, ball color=gray!75, minimum size=8mm, inner sep=0},
Z2/.style={circle, ball color=red!75, minimum size=8mm, inner sep=0},
Z3/.style={circle, ball color=blue!75, minimum size=8mm, inner sep=0},
Z5/.style={circle, ball color=green!75, minimum size=8mm, inner sep=0},
Z7/.style={circle, ball color=yellow!75, minimum size=8mm, inner sep=0},
Z11/.style={circle, ball color=cyan!75, minimum size=8mm, inner sep=0},
Z13/.style={circle, ball color=magenta!75, minimum size=8mm, inner sep=0}
]
\end{tikzpicture}
\caption{}
\end{figure}

\section{Possible patterns?}

After four or five more encounters with this being, I was starting to have some guesses at some basic rules he was trying to communicate through all these examples.
Pressing the buttons appears to start some sort of reaction inside the machine that combines the two objects.  So, it makes sense to view $\star$ and $\dagger$ as binary operations.
In all the examples I have seen, these operation have nice properties in that they are both associative and commutative.  Actually based on the complete structures, they are not quite commutative, but if we consider
two objects the same if they have the same number of balls of each color, then the operations are commutative.  There is further structure in how the balls are connected by bars that
does not behave commutatively, but I don't yet have enough information to even guess at how the operations work with this structure.  For now, I am going to call two objects the same if they have the same number of balls of each color.

\begin{figure}
\centering
\begin{tabular}{l|l}
$\dagger \text{ is associative}$ & $a\dagger(b\dagger c) = (a\dagger b) \dagger c$\\
$\dagger \text{ is commutative}$ & $a \dagger b = b\dagger a$\\
$\dagger \text{ is distributive}$ & $a\star (b \dagger c) = (a\star b)\dagger(a\star c)$\\
$\star \text{ is associative}$ & $a\star(b\star c) = (a\star b) \star c$\\
$\star \text{ is commutative}$ & $a \star b = b\star a$\\
\end{tabular}
\caption{Some plausible rules I will assume until proven wrong.}
\end{figure} 


\begin{definition}
A \emph{ball sequence} is a sequence of natural numbers.  Each slot gives the number of balls of a given color in order 
(\textcolor{gray}{white},\textcolor{red}{red},\textcolor{blue}{blue},\textcolor{green}{green},\textcolor{yellow}{yellow},\textcolor{cyan}{cyan},\textcolor{magenta}{magenta}, ...).
Of course, these are just approximations of the colors I saw and there were many more.  We will add to the list as needed.
\end{definition}

\begin{conjecture}
The only ball sequence with a white ball is $(1,0,0,0,\ldots)$.
\end{conjecture}
\begin{conjecture}
The ball sequence $(1,0,0,0,\ldots)$ has no effect when using $\star$, that is $(1,0,0,0,\ldots) \star b = b$ for all ball sequences $b$.
In the absence of white balls, $\star$ button just adds the values in each slot.  That is,
\[(\textcolor{gray}{0},\textcolor{red}{a_2},\textcolor{blue}{a_3},\ldots) \star
(\textcolor{gray}{0},\textcolor{red}{b_2},\textcolor{blue}{b_3}, \ldots) = (\textcolor{gray}{0},\textcolor{red}{a_2+b_2},\textcolor{blue}{a_3+b_3}, \ldots).\]  In particular, $\star$ never creates new colors.
\end{conjecture}

\section{The $\dagger$ button is sort of like nuclear fission}
I have seen applications of $\dagger$ produce many balls from just a few.  Figure \ref{ManyFromFew} shows an example of this.  It is almost as if the red balls are
the simplest element and $\dagger$ breaks down cyan and yellow balls into many more red balls somehow.

\begin{figure}\label{ManyFromFew}
\centering
\begin{tikzpicture}
[   
Z1/.style={circle, ball color=gray!75, minimum size=8mm, inner sep=0},
Z2/.style={circle, ball color=red!75, minimum size=8mm, inner sep=0},
Z3/.style={circle, ball color=blue!75, minimum size=8mm, inner sep=0},
Z5/.style={circle, ball color=green!75, minimum size=8mm, inner sep=0},
Z7/.style={circle, ball color=yellow!75, minimum size=8mm, inner sep=0},
Z11/.style={circle, ball color=cyan!75, minimum size=8mm, inner sep=0},
Z13/.style={circle, ball color=magenta!75, minimum size=8mm, inner sep=0}
]

	  \begin{scope}[shift = {(0,0)}]
	\node[Z2] (A3) at (0,0) {};
	\node[Z11] (A4) at (0,2) {};
	\node[Z11] (A5) at (2,2) {};
	\BlackBond{A3}{A4}
	\BlackBond{A4}{A5}
\end{scope}
\node at (3,1) {\huge $\dagger$};

  \begin{scope}[shift = {(5,0)}]
	\node[Z2] (A3) at (0,0) {};
	\node[Z7] (A4) at (0,2) {};
	\BlackBond{A3}{A4}

\end{scope}

\node at (7,1) {\huge $=$};

  \begin{scope}[shift = {(9,0)}]
	\foreach \c in {1,...,4}
    {  
	    \node[Z2] (C\c) at ($(1,1)+((\c*90 - 45:1.5)$) {};
		\node[Z2] (CW\c) at ($(1,1)+(\c*90:1.5)$) {};
		\BlackBond{C\c}{CW\c}
    }
	\BlackBond{CW1}{C2}
	\BlackBond{CW2}{C3}
	\BlackBond{CW3}{C4}
	\BlackBond{CW4}{C1}
\end{scope}

\end{tikzpicture}
\caption{Some sort of splitting?}
\end{figure}

\end{document}
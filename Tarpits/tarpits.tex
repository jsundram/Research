\documentclass[12pt]{article}
\usepackage{amsmath, amsthm, amssymb}
\usepackage{hyperref}
\usepackage[margin=1cm]{caption}
\usepackage{verbatim}
\usepackage[top=1.0in, bottom=1.0in, left=1.0in, right=1.0in]{geometry}
\usepackage{graphicx}

\pagestyle{plain}

\usepackage{tkz-graph}
\usetikzlibrary{arrows}
\usetikzlibrary{shapes}
\usepackage[position=bottom]{subfig}

\usepackage{longtable}
\usepackage{array}

\usepackage{sectsty}
\allsectionsfont{\sffamily}

\setcounter{secnumdepth}{5}
\setcounter{tocdepth}{5}

\makeatletter
\newtheorem*{rep@theorem}{\rep@title}
\newcommand{\newreptheorem}[2]{
\newenvironment{rep#1}[1]{
 \def\rep@title{#2 \ref{##1}}
 \begin{rep@theorem}}
 {\end{rep@theorem}}}
\makeatother

\theoremstyle{plain}
\newtheorem{thm}{Theorem}[section]
\newreptheorem{thm}{Theorem}
\newtheorem{prop}[thm]{Proposition}
\newreptheorem{prop}{Proposition}
\newtheorem{lem}[thm]{Lemma}
\newreptheorem{lem}{Lemma}
\newtheorem{conjecture}[thm]{Conjecture}
\newreptheorem{conjecture}{Conjecture}
\newtheorem{cor}[thm]{Corollary}
\newreptheorem{cor}{Corollary}
\newtheorem{prob}[thm]{Problem}
\newtheorem{observation}{Observation}
\newtheorem{obs}[observation]{Observation}
\newtheorem*{mainconj}{Main Conjecture}
\newtheorem*{mainthm}{Main Theorem}
\newtheorem{problem}{Problem}
\newtheorem{clm}{Claim}

\theoremstyle{definition}
\newtheorem{defn}{Definition}
\theoremstyle{remark}
\newtheorem*{remark}{Remark}
\newtheorem*{Player1Move}{Player 1 Move}
\newtheorem*{Player2Move}{Player 2 Move}
\newtheorem{example}{Example}
\newtheorem*{question}{Question}


\newcommand{\fancy}[1]{\mathcal{#1}}
\newcommand{\C}[1]{\fancy{C}_{#1}}
\newcommand{\IN}{\mathbb{N}}
\newcommand{\IR}{\mathbb{R}}
\newcommand{\G}{\fancy{G}}
\newcommand{\CC}{\fancy{C}}
\newcommand{\D}{\fancy{D}}

\newcommand{\inj}{\hookrightarrow}
\newcommand{\surj}{\twoheadrightarrow}

\newcommand{\set}[1]{\left\{ #1 \right\}}
\newcommand{\setb}[3]{\left\{ #1 \in #2 \mid #3 \right\}}
\newcommand{\setbs}[2]{\left\{ #1 \mid #2 \right\}}
\newcommand{\card}[1]{\left|#1\right|}
\newcommand{\size}[1]{\left\Vert#1\right\Vert}
\newcommand{\ceil}[1]{\left\lceil#1\right\rceil}
\newcommand{\floor}[1]{\left\lfloor#1\right\rfloor}
\newcommand{\func}[3]{#1\colon #2 \rightarrow #3}
\newcommand{\funcinj}[3]{#1\colon #2 \inj #3}
\newcommand{\funcsurj}[3]{#1\colon #2 \surj #3}
\newcommand{\irange}[1]{\left[#1\right]}
\newcommand{\join}[2]{#1 \mbox{\hspace{2 pt}$\ast$\hspace{2 pt}} #2}
\newcommand{\djunion}[2]{#1 \mbox{\hspace{2 pt}$+$\hspace{2 pt}} #2}
\newcommand{\parens}[1]{\left( #1 \right)}
\newcommand{\brackets}[1]{\left[ #1 \right]}
\newcommand{\nint}[1]{\widetilde{N}\left(#1\right)}
\newcommand{\DefinedAs}{\mathrel{\mathop:}=}
\newcommand{\pot}{\operatorname{pot}}

\def\adj{\leftrightarrow}
\def\nonadj{\not\!\leftrightarrow}

\def\D{\fancy{D}}
\def\C{\fancy{C}}
\def\Q{\fancy{Q}}
\def\Z{\fancy{Z}}
\def\H{\fancy{H}}
\def\L{\fancy{L}}
\def\B{\fancy{B}}
\def\W{\fancy{W}}
\def\I{\fancy{I}}
\def\P{\fancy{P}}
\def\T{\fancy{T}}

\newcommand{\claim}[2]{{\bf Claim #1.}~{\it #2}~~}
\newcommand{\claimnonum}[1]{{\bf Claim.}~{\it #1}~~}
\newcommand{\subclaim}[2]{{\bf Subclaim #1.}~{\it #2}~~}

\newcommand\numberthis{\addtocounter{equation}{1}\tag{\theequation}}

\def\aftermath{\par\vspace{-\belowdisplayskip}\vspace{-\parskip}\vspace{-\baselineskip}}

\def\fr{\frac}
\def\adj{\leftrightarrow}
\def\ch{\textrm{ch}}

\renewcommand{\restriction}{\mathord{\upharpoonright}}
\begin{document}
\title{tarpit notes}
\author{}
\maketitle
\section{The game}
Let $\Q_n$ be the words of length $n$ in the alphabet $\set{A, B, C}$.  For $w \in Q_n$, let $w_i$ be the $i$-th letter of $w$; that is, $w = w_1w_2\cdots w_n$. For $w \in \Q_n$ and $x \in \set{A,B,C}$, let $\I_x(w) = \setbs{i}{w_i=x}$.  A \emph{2-partition} of a set $S$ is a partition of $S$ into sets of size two and at most one set of size one.

A game is specified by a tuple $(\W, n)$ where $\W \subseteq \Q_n$ is the set of \emph{won} positions.  The \emph{$(\W, n)$-game} is played as follows.  At the start of the game, Player 1 picks an initial $w \in \Q_n$.  After that, the players alternate turns (with Player 1 going first) until $w \in \W$ at the start of Player 1's turn.  Player 2 wins if the game ends and Player 1 wins otherwise.

\begin{Player1Move}
	For each $x \in \set{A, B, C}$, choose a 2-partition $\P_x$ of $\I_x(w)$.
\end{Player1Move}

\begin{Player2Move}
	Pick $x \in \set{A, B, C}$ as well as $p \in \P_x$ and for each $i \in p$, change $w_i$ to the letter in $\set{A, B, C} \setminus \set{x, w_i}$.
\end{Player2Move}

\begin{question}
	For what pairs $(\W,n)$ does Player 2 have a winning strategy?
\end{question}

\section{Tarpits}
We say that $T \subseteq \Q_n$ is a \emph{tarpit} if for each $w \in T$, Player 1 can move such that all of Player 2's possible modifications of $w$ still lie in $T$.  Tarpits are great for Player 1, if $T$ is a tarpit in $\Q_n$ disjoint from $\W$, then Player 1 can just pick the initial $w$ to be in $T$ to guarantee a win.  In fact, if we know the minimal tarpits in $\Q_n$ then we know exactly which pairs $(\W,n)$ Player 2 wins on.  For $n \in \IN$, let $\T_n$ be the set of minimal tarpits in $\Q_n$.

\begin{lem}\label{TarpitEquiv}
	Player 2 wins the $(\W,n)$-game if and only if $\W \cap T \ne \emptyset$ for every $T \in \T_n$.
\end{lem}
\begin{proof}
	If there is $T \in \T_n$ for which $\W \cap T = \emptyset$, then Player 1 wins by definition.  For the other direction, suppose $\W \cap T \ne \emptyset$ for every $T \in \T_n$.  Then Player 2 should play by the strategy: choose $x \in \set{A, B, C}$ and $p \in \P_x$ such that, after modification, $w$ is the word seen least recently (with words that have never been seen being top choice, breaking ties arbitrarily).  Suppose there is a game where Player 2 plays by this strategy, but Player 1 wins.  Say the sequence of words Player 1 encounters at the start of his turns is $w^1, w^2, w^3, \ldots$.   Since $|Q_n|$ is finite, there is $k$ such that each word appearing in $w^k, w^{k+1}, w^{k+2}, \ldots$ appears infinitely many times.
	Let $S = \set{w^k, w^{k+1}, w^{k+2},\ldots}$.   Going out far enough in the sequence, the words in $S$ have been seen more recently than any other word in $Q_n$.  So, by Player 2's strategy, he would escape $S$ if he could.  Since each word in $S$ is encountered infinitely many times, Player 2 has the opportunity to escape $S$ from any word in $S$.  That means that $S$ must be a tarpit.  But, $S \cap \W = \emptyset$ since Player 2 does not win and hence $S$ contains a minimal tarpit disjoint from $\W$, contradicting our assumption.
\end{proof}

The strategy for Player 2 in the proof of Lemma \ref{TarpitEquiv} has the interesting property that it is a winning strategy whenever a winning strategy exists.

\section{Transversal hypergraphs}
A \emph{transversal} of $\T_n$ is a subset $S$ of $\Q_n$ such that $S \cap T \ne \emptyset$ for all $T \in \T_n$.  Let $\H_n$ be the hypergraph with vertex set $\Q_n$ and edge set the minimal transversals of $\T_n$.  As an immediate consequence of Lemma \ref{TarpitEquiv}, we have the following.

\begin{lem}\label{HypergraphEquiv}
		Player 2 wins the $(\W,n)$-game if and only if $\W$ contains an edge of $\H_n$.
\end{lem}

\begin{conjecture}
	If $n \in \IN$, then $\H_n$ is $k$-uniform for some $k$.
\end{conjecture}
\bibliographystyle{amsplain}
\bibliography{GraphColoring1}

\end{document}



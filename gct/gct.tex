\documentclass{amsbook}
\usepackage{amsfonts}
\usepackage{marginnote}

\newcommand{\aside}[1]{\marginnote{\scriptsize{#1}}[0cm]}
\newcommand{\aaside}[2]{\marginnote{\scriptsize{#1}}[#2]}

\theoremstyle{plain}
\newtheorem{acknowledgement}{Acknowledgement}
\newtheorem{algorithm}{Algorithm}
\newtheorem{axiom}{Axiom}
\newtheorem{case}{Case}
\newtheorem{claim}{Claim}
\newtheorem{conclusion}{Conclusion}
\newtheorem{condition}{Condition}
\newtheorem{conjecture}{Conjecture}
\newtheorem{corollary}{Corollary}
\newtheorem{criterion}{Criterion}
\newtheorem{definition}{Definition}
\newtheorem{example}{Example}
\newtheorem{exercise}{Exercise}
\newtheorem{lemma}{Lemma}
\newtheorem{notation}{Notation}
\newtheorem{problem}{Problem}
\newtheorem{proposition}{Proposition}
\newtheorem{remark}{Remark}
\newtheorem{solution}{Solution}
\newtheorem{summary}{Summary}
\newtheorem{theorem}{Theorem}
\numberwithin{equation}{chapter}

\newcommand{\set}[1]{\left\{ #1 \right\}}
\newcommand{\setb}[3]{\left\{ #1 \in #2 : #3 \right\}}
\newcommand{\setbs}[2]{\left\{ #1 : #2 \right\}}
\newcommand{\card}[1]{\left|#1\right|}
\newcommand{\size}[1]{\left\Vert#1\right\Vert}
\newcommand{\ceil}[1]{\left\lceil#1\right\rceil}
\newcommand{\floor}[1]{\left\lfloor#1\right\rfloor}
\newcommand{\func}[3]{#1\colon #2 \rightarrow #3}
\newcommand{\funcinj}[3]{#1\colon #2 \inj #3}
\newcommand{\funcsurj}[3]{#1\colon #2 \surj #3}
\newcommand{\irange}[1]{\left[#1\right]}
\newcommand{\join}[2]{#1 \mbox{\hspace{2 pt}$\ast$\hspace{2 pt}} #2}
\newcommand{\djunion}[2]{#1 \mbox{\hspace{2 pt}$+$\hspace{2 pt}} #2}
\newcommand{\parens}[1]{\left( #1 \right)}
\newcommand{\brackets}[1]{\left[ #1 \right]}
\newcommand{\DefinedAs}{\mathrel{\mathop:}=}

\begin{document}
\frontmatter
\title[gct]{graph coloring tools}
\author{landon rabern}
\maketitle
\tableofcontents

\chapter*{Preface}

This is the preface.

\mainmatter

\chapter*{graphs}
A \emph{graph} is a collection of dots we call \emph{vertices} \aaside{vertices}{-.0in} some of which are connected by curves we call \emph{edges}. \aaside{edges}{-.0in}
The relative location of the dots and the shape of the curves are not relevant, we are only concerned with whether or not a given
pair of dots is connected by a curve.  Initially, we forbid edges from a vertex to itself and multiple edges between two vertices.
If $G$ is a graph, then $V(G)$ is its set of vertices and $E(G)$ its set of edges. \aaside{$V(G)$, $E(G)$}{-.15in}
We write $\card{G}$ for the number of vertices in $V(G)$ and $\size{G}$ for the number of edges in $E(G)$. \aaside{$\card{G}$, $\size{G}$}{}
Two vertices
are \emph{adjacent} \aaside{adjacent}{+0.15in} if they are connected by an edge.  The set
of vertices to which $v$ is adjacent is its \emph{neighborhood}, written $N(v)$. \aaside{neighborhood}{+0.0in} \aaside{$N(v)$}{+0.15in}
For the size of $v$'s neighborhood $\card{N(v)}$, we write $d(v)$ and call this the \emph{degree} of $v$. \aaside{$d(v)$, degree}{+0.3in}

[ADD PICTURES]

\chapter*{coloring vertices}
The entire book concerns one simple task: we want to color the vertices of a given graph so that adjacent vertices receive different colors.
With sufficiently many crayons and no preferences about what the coloring should look like, this is easy, we just use a different crayon for each vertex.  
Things get interesting when we ask how few different crayons we can use.  We are definitely going to need an empty box of crayons and that will only do for the
graph with no vertices at all.  Given one crayon, we can handle all graphs with no edges.  With two crayons, we can do
any path and any cycle with an even number of vertices [PICTURE].  But, we can't handle a triangle or any other cycle with an odd number of vertices [PICTURE].
In fact, odd cycles are really the only thing that will prevent us from using just two crayons. 
A graph $H$ is a \emph{subgraph} of a graph $G$, written $H \subseteq G$ if $V(H) \subseteq V(G)$ and $E(H) \subseteq E(G)$. \aaside{subgraph, $\subseteq$}{}
When $H \subseteq G$, we say that $G$ \emph{contains} $H$. \aaside{contains}{0.12in}  If $v \in V(G)$, then $G-v$ is the graph we get by removing $v$ and all edges incident to $v$ from $G$. \aaside{$G-v$}{0.05in}
A graph is $k$-colorable if we can color its vertices with (at most) $k$ colors such that adjacent vertices receive different colors. \aaside{$k$-colorable}{}
A $0$-colorable graph is \emph{empty}, a $1$-colorable graph is \emph{edgeless} and a $2$-colorable graph is \emph{bipartite}. \aaside{empty}{}\aaside{edgeless}{+0.1in}\aaside{bipartite}{+0.2in}
\begin{theorem}\label{TwoColoring}
A graph is $2$-colorable just in case it contains no odd cycle.
\end{theorem}
\begin{proof}
A graph containing an odd cycle clearly can't be $2$-colored.  For the other implication, suppose
there is a graph that is not $2$-colorable and doesn't contain an odd cycle.  Then we may pick such a graph $G$ with $\card{G}$ as small as possible.
Surely, $|G| > 0$, so we may pick $v \in V(G)$.  If $x, y \in N(v)$, then $x$ is not adjacent to $y$ since then $xyz$ would be an odd cycle.
So we can construct a graph $H$ from $G$ by removing $v$ and identifying all of $N(v)$ to a new vertex $x_v$.  Any odd cycle
in $H$ would contain $x_v$ and hence give rise to an odd cycle in $G$ passing through $v$.  So $H$ contains no odd cycle. Since $|H| < |G|$, appplying the theorem to $H$ gives a 2-coloring of $H$, say with red and blue
where $x_v$ gets colored red.  But this gives a 2-coloring of $G$ by coloring all vertices in $N(v)$ red and $v$ blue, a contradiction.
\end{proof}

Well, this is embarrassing, coloring appears to be easy.  Fortunately, things get more interesting when we move up to three colors.
\begin{theorem}
3-coloring is hard supposing other things we think are hard are actually hard.
\end{theorem}
\begin{proof}
reduce 3-SAT to 3-coloring.
\end{proof}

\section*{basic estimates}
Even though finding the minimum number of colors needed to color a graph is hard in general (supposing it is), we can still
look for lower and upper bounds on this value.  The \emph{chromatic number} $\chi(G)$ of a graph $G$ is the smallest $k$ for which $G$ is $k$-colorable.
\aaside{chromatic number}{+0.0in}\aaside{$\chi(G)$}{+0.15in}
The simplest thing we can do is give each vertex a different color.
\begin{theorem}\label{WorstUpperBound}
If $G$ is a graph, then $\chi(G) \le \card{G}$.
\end{theorem}
The only graphs that attain the upper bound in Theorem \ref{WorstUpperBound} are the \emph{complete} graphs; those in which
any two vertices are adjacent. \aaside{complete}{}
We can usually do much better by just arbitrarily coloring vertices, reusing colors when we can.  \aaside{maximum degree}{+0.0in}\aaside{$\Delta(G)$}{+0.15in}The \emph{maximum degree} $\Delta(G)$ of a graph $G$ is the largest degree
of any vertex in $G$; that is 
\[\Delta(G) \DefinedAs \max_{v \in V(G)} d(v).\]

\begin{theorem}\label{SecondWorstUpperBound}
If $G$ is a graph, then $\chi(G) \le \Delta(G) + 1$.
\end{theorem}
\begin{proof}
Suppose there is a graph $G$ that is not $\parens{\Delta(G) + 1}$-colorable.  Then we may pick such a graph $G$ with $\card{G}$ as small as possible.
Surely, $\card{G} > 0$, so we may pick $v \in V(G)$.  Then $\card{G-v} < \card{G}$ and $\Delta(G-v) \le \Delta(G)$, so applying the theorem to $G-v$ gives a $\parens{\Delta(G-v) + 1}$-coloring
of $G-v$.  But $v$ has at most $\Delta(G)$ neighbors, so there is some color, say red, not used on $N(v)$, coloring $v$ red gives a $\parens{\Delta(G) + 1}$-coloring
of $G$, a contradiction.
\end{proof}

Both complete graphs and odd cycles attain the upper bound in Theorem \ref{SecondWorstUpperBound}.  Theorem \ref{TwoColoring} says
we can do better for graphs that don't contain odd cycles.  We can also do better for graphs that don't contain large complete subgraphs.
A set of vertices $S$ in a graph $G$ is a \emph{clique} if the vertices in $S$ are pairwise adjacent.\aaside{clique}{}  
The \emph{clique number} of a graph $G$, written $\omega(G)$, is the number of vertices in a largest clique in $G$.\aaside{$\omega(G)$}{}

\begin{theorem}\label{OmegaLowerBound}
If $G$ is a graph, then $\chi(G) \ge \omega(G)$.
\end{theorem}

A set of vertices $S$ in a graph $G$ is \emph{independent} if the vertices in $S$ are pairwise non-adjacent.\aaside{independent}{}  
The \emph{independence number} of a graph $G$, written $\alpha(G)$, is the number of vertices in a largest independent set in $G$.\aaside{$\alpha(G)$}{}

\begin{theorem}[Brooks' theorem]
If $G$ is a graph with $\Delta(G) \ge 3$ and $\omega(G) \le \Delta(G)$, then $\chi(G) \le \Delta(G)$.
\label{BrooksTheorem}
\end{theorem}
\begin{proof}
Suppose there is a graph $G$  with $\Delta(G) \ge 3$ and $\omega(G) \le \Delta(G)$ that is not $\Delta(G)$-colorable.  
Then we may pick such a graph $G$ with $\card{G}$ as small as possible.  Let $S$ be 
a maximal independent set in $G$.  Since $S$ is maximal, every vertex in $G-S$ has a neighbor in $S$, so $\Delta(G) > \Delta(G-S)$.
If red is an unused color in a $\chi(G-S)$-coloring of $G-S$, then by coloring all vertices in $S$ red we get a $\parens{\chi(G-S)+1}$-coloring of $G$.  
So, $\Delta(G) + 1 \le \chi(G) \le \chi(G-S) + 1$. We conclude $\chi(G-S) > \Delta(G - S)$ and thus $\Delta(G) = \chi(G-S) = \Delta(G-S) + 1$ by Theorem \ref{SecondWorstUpperBound}.
Since $\card{G-S} < \card{G}$, applying the theorem to $G-S$ shows that $\Delta(G-S) < 3$ or $\Delta(G -S) < \omega(G - S)$.  
So, either $\chi(G-S) = \Delta(G) = 3$ or $\omega(G-S) \ge \Delta(G)$.  In the former case, let $X$ be the vertex set of an odd cycle in $G-S$ guaranteed by Theorem \ref{TwoColoring}.  
In the latter case, let $X$ be a $\Delta(G)$-clique in $G-S$.

Since $S$ is maximal and $\omega(G) \le \Delta(G)$, there are $x_1, x_2 \in X$ and $y_1, y_2 \in S$ such that $x_1$ is adjacent to $y_1$ and $x_2$ is adjacent to $y_2$.
Construct a graph $H$ from $G-X$ by adding the edge $y_1y_2$.  Since $\card{H} < \card{G}$, applying the theorem to $H$ shows that $\omega(H) > \Delta(G)$ or $\chi(H) \le \Delta(G)$.
Suppose $\chi(H) \le \Delta(G)$.  Then there is a $\Delta(G)$-coloring of $G-X$ where $y_1$ and $y_2$ receive different colors, say red and blue respectively.
Pick the first vertex $z$ in a shortest path $P$ from $x_1$ to $x_2$ in $X$ that has a blue colored neighbor in $V(H)$. 
Each vertex in $X$ has $\Delta(G)-1$ neighbors in $X$ and hence at most one neighbor in $V(H)$.  So, $z \ne x_1$ since $x_1$ already has a red colored neighbor in $V(H)$.
Let $w$ be be the vertex preceding $z$ on $P$. Then $w$ has no blue colored neighbor.  Since $X$ is the vertex set of a cycle or a 
complete graph, there is a path $Q$ from $w$ to $z$ passing through every vertex of $X$.  Color $w$ blue and then proceed along $Q$, coloring one vertex at a time.  
Since each vertex we encounter before we get to $z$ has at most $\Delta(G) - 1$ colored neighbors, we always have an available color to use.  But, $z$ is adjacent
to both $w$ and another blue colored vertex in $V(H)$, so there is an available color for $z$ as well.  This gives a $\Delta(G)$-coloring of $G$, a contradiction.

So, $\omega(H) > \Delta(G)$.  In particular, $y_1$ and $y_2$ each have exactly one neighbor in $X$ and $\Delta(G) - 1$ neighbors in the same $\Delta(G) -1$ clique $A$ in $G - X$.
 Since $S$ is maximal and $\card{X} \ge 3$, there must be adjacent
$x_3 \in X \setminus \set{x_1,x_2}$ and $y_3 \in S \setminus \set{y_1,y_2}$.  Applying the same argument with $x_3, y_3$ in place of $x_2, y_2$ shows
that $y_1$ and $y_3$ each have exactly one neighbor in $X$ and $\Delta(G) - 1$ neighbors in the same $\Delta(G) -1$ clique $B$ in $G - X$.
Now $\card{A\cap B} = \card{A} + \card{B} - \card{A\cup B} \ge 2(\Delta(G) - 1) - d(y_1) \ge \Delta(G) - 2 > 0$.  But there can't be a vertex
in $A \cap B$ since it would be adjacent to $y_1,y_2,y_3$ as well as $\Delta(G) - 2$ vertices in $A$ and thus have degree greater than $\Delta(G)$, a contradiction. 

\chapter*{edge coloring}
It is also useful to consider coloring the edges of a graph so that incident edges receive different colors.  This
appears to be at odds with our previous claim that this book was only about coloring vertices of graph; fortunately, edge coloring
is just a special case of vertex coloring.  If $G$ is a graph, the \emph{line graph} of $G$, written
$L(G)$ is the graph with vertex set $E(G)$ where two edges of $G$ are adjacent in $L(G)$ if they are incident in $G$.  Coloring
the edges of $G$ is equivalent to coloring the vertices of $L(G)$.

For graphs with maximum degree zero (that is, no edges at all), we can get by with zero colors.  
With just one color we can edge color any graph with maximum degree at most one.  We will definitely always need at least $\Delta(G)$ colors
to edge color a graph $G$.  Could we be so fortunate that the pattern continues
and we can edge color any graph $G$ with only $\Delta(G)$-colors? Not quite, but we can do so for bipartite ($2$-colorable) graphs.
A graph is $k$-edge-colorable if we can color its edges with (at most) $k$ colors such that incident edges receive different colors.
Before proving this result, we develop a tool that will useful throughout the book.

\section*{Hall's marriage theorem}
Suppose $G$ is a bipartite graph, say it has a $2$-coloring with colors red and blue.  Let $R \subseteq V(G)$ be the vertices colored red and
$B \subseteq V(G)$ the vertices colored red.  Then $R$ and $B$ are independent sets in $G$.  This situation is so common, we need special notation
to introduce bipartite graphs with the sets $R$ and $B$ already labeled, like so: let $G = (R,B)$ be a bipartite graph.\aaside{$G = (R,B)$}{}
A graph $G$ is $k$-regular if every vertex in $G$ has degree $k$. \aaside{$k$-regular}{} A $1$-regular graph is a \emph{matching}. \aaside{matching}{0.15in} If $H \subseteq G$ is a matching, then
$v \in V(G)$ is \emph{matched} in $H$ if $v$ is the endpoint of some edge in $H$.\aaside{matched}{0.1in}

\begin{theorem}\label{HallsMarriageTheorem}
Let $G = (R,B)$ be a bipartite graph.  If $\card{N_G(X)} \ge \card{X}$ for every $X \subseteq R$, then $G$ contains a matching $H$ that matches
every vertex of $R$.
\end{theorem}
\begin{proof}
Suppose the theorem does not hold.  Then we may pick a counterexample $G = (R,B)$ with $\size{G}$ as small as possible.  Surely, $\size{G} > 0$.
For any $e \in E(G)$, since a matching in $G-e$ is also a matching in $G$, any matching in $G-e$ must fail to match some vertex $v_e \in R$.
As $\size{G-e} < \size{G}$, we may apply the theorem to $G-e$ to conclude that there is $X_e \subseteq R$ such that $\card{N_{G-e}(X_e)} < \card{X_e}$.
\end{proof}

\section*{edge coloring bipartite graphs}

\begin{theorem}\label{DeltaEdgeColoring}
If $G$ is a bipartite graph, then $G$ is $\Delta(G)$-edge-colorable.
\end{theorem}
\begin{proof}
\end{proof}

\section*{a game version of Hall's marriage theorem}
It may come as a surpise that even though we might need more than $\Delta(G)$ colors to edge color a graph $G$, we will only ever need at most one extra color.
To prove this result, we develop a dynamic version of Hall's marriage theorem that can be conveniently phrased as a game.


\section*{edge coloring general graphs}

\begin{theorem}\label{VizingsTheorem}
If $G$ is a graph, then $G$ is $\parens{\Delta(G) + 1}$-edge-colorable.
\end{theorem}
\begin{proof}
\end{proof}

\section*{hardness}
We now know that every graph $G$ can be edge colored with either $\Delta(G)$ or $\Delta(G) + 1$ colors.  So, edge coloring is basically trivial, right?
Furtunately, no it isn't, the collection of graphs requiring $\Delta(G) + 1$ colors is very rich.  Another way to say this, is that it is a hard problem
to decide whether or not edge coloring a given graph $G$ requires $\Delta(G) + 1$ colors.

\begin{theorem}\label{HolyerEdgeColoringDecisionIsHard}
Deciding whether or not edge coloring a given graph $G$ requires $\Delta(G) + 1$ colors is hard supposing other things we think are hard are actually hard.
\end{theorem}
\begin{proof}
\end{proof}

\chapter*{vertex coloring, again}
\section*{list coloring}
\section*{online list coloring}
\section*{kernel tools}
\subsection*{Brooks again}
\subsection*{maximum independent covers}
\section*{polynomial tools}
\subsection*{combinatorial nullstellensatz}
\subsection*{coefficient formulae}
\subsection*{a combinatorial interpretation}

\chapter*{edge coloring, again}
\section*{fans as a greedy strategy}
\section*{Kierstead paths}
\section*{Tashkinov trees}
\section*{edge list coloring}
\subsection*{more kernel method}
\subsection*{2-edge-coloring}
\subsection*{improved Brooks' theorem}
\subsection*{a hint of quasiline and claw-free graphs}

\chapter*{Lov\'{a}sz-Catlin shuffle tool}
\section*{destroying non-complete}
\section*{Mozhan basics}
\section*{a hint of Kostochka's technique}

\chapter*{independent transversalls}
\section*{randomly}
\section*{Haxell's tool}
\subsection*{a hint of Sperner's lemma}
\section*{Hajnal and Kostochka maximum clique tool}
\section*{King's lemma}

\chapter*{vertex transitive graphs}
\section*{strong coloring}
\section*{medium clique implies big clique}

\chapter*{Kostochka-Yancey potential tool}
\end{document}
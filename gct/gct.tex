\documentclass{amsbook}
\usepackage{amsfonts}
\usepackage{marginnote}

\newcommand{\aside}[1]{\marginnote{\scriptsize{#1}}[0cm]}
\newcommand{\aaside}[2]{\marginnote{\scriptsize{#1}}[#2]}

\theoremstyle{plain}
\newtheorem{acknowledgement}{Acknowledgement}
\newtheorem{algorithm}{Algorithm}
\newtheorem{axiom}{Axiom}
\newtheorem{case}{Case}
\newtheorem{claim}{Claim}
\newtheorem{conclusion}{Conclusion}
\newtheorem{condition}{Condition}
\newtheorem{conjecture}{Conjecture}
\newtheorem{corollary}{Corollary}
\newtheorem{criterion}{Criterion}
\newtheorem{definition}{Definition}
\newtheorem{example}{Example}
\newtheorem{exercise}{Exercise}
\newtheorem{lemma}{Lemma}
\newtheorem{notation}{Notation}
\newtheorem{problem}{Problem}
\newtheorem{proposition}{Proposition}
\newtheorem{remark}{Remark}
\newtheorem{solution}{Solution}
\newtheorem{summary}{Summary}
\newtheorem{theorem}{Theorem}
\numberwithin{equation}{chapter}

\newcommand{\set}[1]{\left\{ #1 \right\}}
\newcommand{\setb}[3]{\left\{ #1 \in #2 : #3 \right\}}
\newcommand{\setbs}[2]{\left\{ #1 : #2 \right\}}
\newcommand{\card}[1]{\left|#1\right|}
\newcommand{\size}[1]{\left\Vert#1\right\Vert}
\newcommand{\ceil}[1]{\left\lceil#1\right\rceil}
\newcommand{\floor}[1]{\left\lfloor#1\right\rfloor}
\newcommand{\func}[3]{#1\colon #2 \rightarrow #3}
\newcommand{\funcinj}[3]{#1\colon #2 \inj #3}
\newcommand{\funcsurj}[3]{#1\colon #2 \surj #3}
\newcommand{\irange}[1]{\left[#1\right]}
\newcommand{\join}[2]{#1 \mbox{\hspace{2 pt}$\ast$\hspace{2 pt}} #2}
\newcommand{\djunion}[2]{#1 \mbox{\hspace{2 pt}$+$\hspace{2 pt}} #2}
\newcommand{\parens}[1]{\left( #1 \right)}
\newcommand{\brackets}[1]{\left[ #1 \right]}
\newcommand{\DefinedAs}{\mathrel{\mathop:}=}

\begin{document}
\frontmatter
\title[gct]{graph coloring tools}
\author{landon rabern}
\maketitle
\tableofcontents

\chapter*{Preface}

This is the preface.

\mainmatter
\part{basics}
\chapter*{graphs}
A \emph{graph} is a collection of dots we call \emph{vertices} \aaside{vertices}{-.0in} some of which are connected by curves we call \emph{edges}. \aaside{edges}{-.0in}
The relative location of the dots and the shape of the curves are not relevant, we are only concerned with whether or not a given
pair of dots is connected by a curve.  Initially, we forbid edges from a vertex to itself and multiple edges between two vertices.
If $G$ is a graph, then $V(G)$ is its set of vertices and $E(G)$ its set of edges. \aaside{$V(G)$, $E(G)$}{-.15in}
We write $\card{G}$ for the number of vertices in $V(G)$ and $\size{G}$ for the number of edges in $E(G)$. \aaside{$\card{G}$, $\size{G}$}{}
Two vertices
are \emph{adjacent} \aaside{adjacent}{+0.15in} if they are connected by an edge.  The set
of vertices to which $v$ is adjacent is its \emph{neighborhood}, written $N(v)$. \aaside{neighborhood}{+0.0in} \aaside{$N(v)$}{+0.15in}
For the size of $v$'s neighborhood $\card{N(v)}$, we write $d(v)$ and call this the \emph{degree} of $v$. \aaside{$d(v)$, degree}{+0.3in}

[ADD PICTURES]

\chapter*{coloring vertices}
The entire book concerns one simple task: we want to color the vertices of a given graph so that adjacent vertices receive different colors.
With no preferences about what the coloring should look like, this is easy, we just give each vertex a different color.  Things get
interesting when we ask how few colors we can use.  We are definitely going to need at least zero colors and that will only do for the
graph with no vertices at all.  Given one color, we can handle all graphs with no edges.  With two colors, we can do
any path and any cycle with an even number of vertices [PICTURE].  But, we can't handle a triangle or any other cycle with an odd number of vertices [PICTURE].
In fact, odd cycles are really the only thing that will prevent us from using two colors. 
A graph $H$ is a \emph{subgraph} of a graph $G$, written $H \subseteq G$ if $V(H) \subseteq V(G)$ and $E(H) \subseteq E(G)$. \aaside{subgraph, $\subseteq$}{}
When $H \subseteq G$, we say that $G$ \emph{contains} $H$. \aaside{contains}{}  If $v \in V(G)$, then $G-v$ is the graph we get by removing $v$ and all edges incident to $v$ from $G$. \aaside{$G-v$}{}
A graph is $k$-colorable if we can color its vertices with (at most) $k$ colors such that adjacent vertices receive different colors. \aaside{$k$-colorable}{}
\begin{theorem}
A graph is $2$-colorable just in case it contains no odd cycle.
\end{theorem}
\begin{proof}
A graph containing an odd cycle clearly can't be $2$-colored.  For the other implication, suppose
there is a graph that is not $2$-colorable and doesn't contain an odd cycle.  Then we may pick such a graph $G$ with $\card{G}$ as small as possible.
Surely, $|G| > 0$, so we may pick $v \in V(G)$.  If $x, y \in N(v)$, then $x$ is not adjacent to $y$ since then $xyz$ would be an odd cycle.
So we can construct a graph $H$ from $G$ by removing $v$ and identifying all of $N(v)$ to a new vertex $x_v$.  Any odd cycle
in $H$ would contain $x_v$ and hence give rise to an odd cycle in $G$.  So $H$ contains no odd cycle. Since $|H| < |G|$, we can 2-color $H$, say with red and blue
where $x_v$ gets colored red.  But this gives a 2-coloring of $G$ by coloring all vertices in $N(v)$ red and $v$ blue, a contradiction.
\end{proof}

Well, this is embarrassing, coloring appears to be easy.  Fortunately, things get more interesting when we move up to three colors.
\begin{theorem}
3-coloring is hard supposing other things we think are hard are actually hard.
\end{theorem}
\begin{proof}
reduce 3-SAT to 3-coloring.
\end{proof}

\section*{basic estimates}
Even though finding the minimum number of colors needed to color a graph is hard in general (supposing it is), we can still
look for lower and upper bounds on this value.  The \emph{chromatic number} $\chi(G)$ of a graph $G$ is the smallest $k$ for which $G$ is $k$-colorable.
\aaside{chromatic number}{+0.0in}\aaside{$\chi(G)$}{+0.15in}
The simplest thing we can do is give each vertex a different color.
\begin{theorem}
For every graph $G$, we have $\chi(G) \le \card{G}$.
\end{theorem}
We can usually do much better by just arbitrarily coloring vertices, reusing colors when we can.  \aaside{maximum degree}{+0.0in}\aaside{$\Delta(G)$}{+0.15in}The \emph{maximum degree} $\Delta(G)$ of a graph $G$ is the largest degree
of any vertex in $G$; that is 
\[\Delta(G) \DefinedAs \max_{v \in V(G)} d(v).\]

\begin{theorem}
For every graph $G$, we have $\chi(G) \le \Delta(G) + 1$.
\end{theorem}
\begin{proof}
Suppose there is a graph $G$ that is not $\parens{\Delta(G) + 1}$-colorable.  Then we may pick such a graph $G$ with $\card{G}$ as small as possible.
Surely, $\card{G} > 0$, so we may pick $v \in V(G)$.  Then $\card{G-v} < \card{G}$ and $\Delta(G-v) \le \Delta(G)$, so we have a $\parens{\Delta(G) + 1}$-coloring
of $G-v$.  But $v$ has at most $\Delta(G)$ neighbors, so there is some color, say red, not used on $N(v)$, coloring $v$ red gives a $\parens{\Delta(G) + 1}$-coloring
of $G$, a contradiction.
\end{proof}
\end{document}
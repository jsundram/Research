%% LyX 2.1.4 created this file.  For more info, see http://www.lyx.org/.
%% Do not edit unless you really know what you are doing.
\documentclass[12pt]{article}
\usepackage[latin9]{inputenc}
\usepackage{geometry}
\geometry{verbose}
\pagestyle{plain}
\setcounter{secnumdepth}{5}
\setcounter{tocdepth}{5}
\usepackage{verbatim}
\usepackage{amsmath}
\usepackage{amsthm}
\usepackage{amssymb}
\usepackage[unicode=true,
 bookmarks=false,
 breaklinks=false,pdfborder={0 0 1},backref=section,colorlinks=false]
 {hyperref}

\makeatletter
%%%%%%%%%%%%%%%%%%%%%%%%%%%%%% User specified LaTeX commands.
%\documentclass[12pt]{article}
\usepackage{amsmath, amsthm, amssymb}
\usepackage{hyperref}
\usepackage{verbatim}
%\usepackage[top=1.0in, bottom=1.0in, left=1.0in, right=1.0in]{geometry}
\usepackage{pdfsync}
\pagestyle{plain}
\usepackage{fullpage}

\usepackage{sectsty}
\allsectionsfont{\sffamily}

\setcounter{secnumdepth}{5}
\setcounter{tocdepth}{5}

\makeatletter
\newtheorem*{rep@theorem}{\rep@title}
\newcommand{\newreptheorem}[2]{
\newenvironment{rep#1}[1]{
 \def\rep@title{#2 \ref{##1}}
 \begin{rep@theorem}}
 {\end{rep@theorem}}}
\makeatother

\theoremstyle{plain}
\newtheorem{thm}{Theorem}[section]
\newreptheorem{thm}{Theorem}
\newtheorem{prop}[thm]{Proposition}
\newreptheorem{prop}{Proposition}
\newtheorem{lem}[thm]{Lemma}
\newreptheorem{lem}{Lemma}
\newtheorem{conjecture}[thm]{Conjecture}
\newreptheorem{conjecture}{Conjecture}
\newtheorem{cor}[thm]{Corollary}
\newreptheorem{cor}{Corollary}
\newtheorem{prob}[thm]{Problem}

\newtheorem*{KernelLemma}{Kernel Lemma}
\newtheorem*{MainLemma}{Main Lemma}
\newtheorem*{BK}{Borodin-Kostochka Conjecture}
\newtheorem*{BK2}{Borodin-Kostochka Conjecture (restated)}
\newtheorem*{Reed}{Reed's Conjecture}
\newtheorem*{ClassificationOfd0}{Classification of $d_0$-choosable graphs}
\newtheorem*{BrooksTheoremAlpha}{Brooks' Theorem for Independence Number}

\theoremstyle{definition}
\newtheorem{defn}{Definition}
\theoremstyle{remark}
\newtheorem*{remark}{Remark}
\newtheorem*{problem}{Problem}
\newtheorem{example}{Example}
\newtheorem*{question}{Question}
\newtheorem*{observation}{Observation}

\newcommand{\fancy}[1]{\mathcal{#1}}
%\newcommand{\C}[1]{\fancy{C}_{#1}}
\newcommand{\IN}{\mathbb{N}}
\newcommand{\IR}{\mathbb{R}}
%\newcommand{\G}{\fancy{G}}
\newcommand{\CC}{\fancy{C}}
\newcommand{\D}{\fancy{D}}
\newcommand{\T}{\fancy{T}}
\newcommand{\B}{\fancy{B}}
\renewcommand{\L}{\fancy{L}}
\newcommand{\HH}{\fancy{H}}

\newcommand{\inj}{\hookrightarrow}
\newcommand{\surj}{\twoheadrightarrow}

\newcommand{\set}[1]{\left\{ #1 \right\}}
\newcommand{\setb}[3]{\left\{ #1 \in #2 : #3 \right\}}
\newcommand{\setbs}[2]{\left\{ #1 : #2 \right\}}
\newcommand{\card}[1]{\left|#1\right|}
\newcommand{\size}[1]{\left\Vert#1\right\Vert}
\newcommand{\ceil}[1]{\left\lceil#1\right\rceil}
\newcommand{\floor}[1]{\left\lfloor#1\right\rfloor}
\newcommand{\func}[3]{#1\colon #2 \rightarrow #3}
\newcommand{\funcinj}[3]{#1\colon #2 \inj #3}
\newcommand{\funcsurj}[3]{#1\colon #2 \surj #3}
\newcommand{\irange}[1]{\left[#1\right]}
\newcommand{\join}[2]{#1 \mbox{\hspace{2 pt}$\ast$\hspace{2 pt}} #2}
\newcommand{\djunion}[2]{#1 \mbox{\hspace{2 pt}$+$\hspace{2 pt}} #2}
\newcommand{\parens}[1]{\left( #1 \right)}
\newcommand{\brackets}[1]{\left[ #1 \right]}
\newcommand{\DefinedAs}{\mathrel{\mathop:}=}

\newcommand{\mic}{\operatorname{mic}}
\newcommand{\AT}{\operatorname{AT}}
\newcommand{\col}{\operatorname{col}}
\newcommand{\ch}{\operatorname{ch}}
\newcommand{\type}{\operatorname{type}}
\newcommand{\nonsep}{\bar{S}}

\newcommand{\sm}{\smallsetminus}

\def\adj{\leftrightarrow}
\def\nonadj{\not\!\leftrightarrow}

\newcommand\restr[2]{{% we make the whole thing an ordinary symbol
  \left.\kern-\nulldelimiterspace % automatically resize the bar with \right
  #1 % the function
  \vphantom{\big|} % pretend it's a little taller at normal size
  \right|_{#2} % this is the delimiter
  }}

\def\D{\fancy{D}}
%\def\C{\fancy{C}}
\def\A{\fancy{A}}

\newcommand{\case}[2]{{\bf Case #1.}~{\it #2}~~}

\makeatother

\begin{document}
\title{hypergraph kernel magic notes}
 \maketitle
\section{Hypergraph orientations}
Let $H = (V, E)$ be a hypergraph. An \emph{orientation} of $H$ is a function $q$ that assigns to each $e \in E$, a subset of $e$, that is $q(e) \subseteq e$. Given an orientation $q$ of $H$, the \emph{out-degree} of $v \in V$ is $d_q^+(v) \DefinedAs \card{\setb{e}{E}{v \in q(e)}}$.
We say that and orientation $q$ of $H$ is \emph{kernel-perfect} if for all induced subhypergraphs $H' = (V', E')$ of $H$, there is $S \subseteq V'$ such that $S$ is independent and for each $v \in V' \setminus S$, there is $e \in E'$ with $e \cap S \ne \emptyset$ and $v \in q(e)$.  Such an $S$ is a \emph{kernel}.
 

\begin{lem}\label{HypergraphKernelLemma}
	Let $H = (V, E)$ be a hypergraph and $\func{f}{V}{\IN}$.
	If $H$ has a kernel-perfect orientation $q$ such that $f(v) > d_q^+(v)$ for all $v \in V$, then $H$ is $f$-paintable.
\end{lem}
\begin{proof}
	Suppose not and choose a counterexample $H = (V, E)$ with $f$ so as to minimize $|V|$.  Let $q$ be a kernel-perfect orientation of $H$ such that $f(v) > d_q^+(v)$ for all $v \in V$. 
	Since $H$ is not $f$-paintable, Lister has a winning move, say he chooses $A \subseteq V$ as the vertices that have blue available.  Painter should pick a kernel $S \subseteq A$ and color all vertices in $S$ blue.
	Define a function $f'$ on $H-S$ by $f'(v) = f(v)$ for $v \in V \setminus A$ and $f'(v) = f(v) - 1$ for all $v \in S\setminus A$.  Since $S$ is a kernel, the out-degree of each vertex in $S \setminus A$ went down by at least one.  Now Painter can win on $H-S$ with $f'$ by minimality of $|V|$, contradicting our choice of $A$.
\end{proof}

\begin{lem}\label{IndependentCutKernelPerfection}
	Let $H = (V, E)$ be a hypergraph and $S \subseteq V$ an independent set.  If $q$ is an orientation of $H$ such that $q(e) \ge 1$ for all $e \in E$ and $q(e) \ge 2$ for all $e \subseteq E\setminus S$, then $q$ is kernel-perfect.
\end{lem}
\begin{proof}
	Suppose not and choose a counterexample $H = (V,E)$ with $q$ so as to minimize $|V|$.  Then every proper induced subhypergraph of $H$ has a kernel by minimality of $|V|$.  So, it must be that $H$ has no kernel.  In particular, $S$ is not a kernel of $H$ with $q$.  So, there is $v \in V \setminus S$ such that $v \not \in q(e)$ for every $e \in E$ with $v \in e$ and $e \cap S \ne \emptyset$.  Since $q(e) \ge 1$ for all $e \in E$ and $q(e) \ge 2$ for all $e \subseteq E\setminus S$, for each $e \in E$ with $v \in e$, we can choose $x_e \in q(e) \setminus \set{v}$.  Let $H' = H - \parens{\set{v} \cup \setbs{x_e}{e \in E \text{ with } v\in e}}$.  Then $H'$ has a kernel $A'$ by minimality of $|V|$.  We claim that $A \DefinedAs A' \cup \set{v}$ is a kernel in $H$.  	If $A$ was not independent, then there would be $e \in E$ with $v \in E$ and $e \subseteq A$.  But then $x_e \in A$ which is impossible since $x_e$ is not in $H'$.  So, $A$ in independent.  So, $A$ is a kernel since $v \in A$ and $x_e \in q(e)$ for all $e$ containing $v$.  This contradiction completes the proof.
\end{proof}
 
\end{document}

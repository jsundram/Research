\documentclass[12pt]{article}
\usepackage{amsmath, amsthm, amssymb}
\usepackage{hyperref}
\usepackage{verbatim}
\usepackage[top=1.0in, bottom=1.0in, left=1.0in, right=1.0in]{geometry}

\pagestyle{plain}

\usepackage{tkz-graph}
\usetikzlibrary{arrows}
\usetikzlibrary{shapes}
\usepackage[position=bottom]{subfig}

\usepackage{longtable}
\usepackage{array}

\usepackage{sectsty}
\allsectionsfont{\sffamily}

\setcounter{secnumdepth}{5}
\setcounter{tocdepth}{5}

\makeatletter
\newtheorem*{rep@theorem}{\rep@title}
\newcommand{\newreptheorem}[2]{
\newenvironment{rep#1}[1]{
 \def\rep@title{#2 \ref{##1}}
 \begin{rep@theorem}}
 {\end{rep@theorem}}}
\makeatother

\theoremstyle{plain}
\newtheorem{thm}{Theorem}[section]
\newreptheorem{thm}{Theorem}
\newtheorem{prop}[thm]{Proposition}
\newreptheorem{prop}{Proposition}
\newtheorem{lem}[thm]{Lemma}
\newreptheorem{lem}{Lemma}
\newtheorem{conjecture}[thm]{Conjecture}
\newreptheorem{conjecture}{Conjecture}
\newtheorem{cor}[thm]{Corollary}
\newreptheorem{cor}{Corollary}
\newtheorem{prob}[thm]{Problem}
\newtheorem{observation}{Observation}
\newtheorem*{mainconj}{Main Conjecture}
\newtheorem*{mainthm}{Main Theorem}

\newtheorem*{SmallPotLemma}{Small Pot Lemma}
\newtheorem*{BK}{Borodin-Kostochka Conjecture}
\newtheorem*{BK2}{Borodin-Kostochka Conjecture (restated)}
\newtheorem*{Reed}{Reed's Conjecture}
\newtheorem*{ClassificationOfd0}{Classification of $d_0$-choosable graphs}


\theoremstyle{definition}
\newtheorem{defn}{Definition}
\theoremstyle{remark}
\newtheorem*{remark}{Remark}
\newtheorem*{problem}{Problem}
\newtheorem{example}{Example}
\newtheorem*{question}{Question}


\newcommand{\fancy}[1]{\mathcal{#1}}
\newcommand{\C}[1]{\fancy{C}_{#1}}
\newcommand{\IN}{\mathbb{N}}
\newcommand{\IR}{\mathbb{R}}
\newcommand{\G}{\fancy{G}}
\newcommand{\CC}{\fancy{C}}
\newcommand{\D}{\fancy{D}}

\newcommand{\inj}{\hookrightarrow}
\newcommand{\surj}{\twoheadrightarrow}

\newcommand{\set}[1]{\left\{ #1 \right\}}
\newcommand{\setb}[3]{\left\{ #1 \in #2 \mid #3 \right\}}
\newcommand{\setbs}[2]{\left\{ #1 \mid #2 \right\}}
\newcommand{\card}[1]{\left|#1\right|}
\newcommand{\size}[1]{\left\Vert#1\right\Vert}
\newcommand{\ceil}[1]{\left\lceil#1\right\rceil}
\newcommand{\floor}[1]{\left\lfloor#1\right\rfloor}
\newcommand{\func}[3]{#1\colon #2 \rightarrow #3}
\newcommand{\funcinj}[3]{#1\colon #2 \inj #3}
\newcommand{\funcsurj}[3]{#1\colon #2 \surj #3}
\newcommand{\irange}[1]{\left[#1\right]}
\newcommand{\join}[2]{#1 \mbox{\hspace{2 pt}$\ast$\hspace{2 pt}} #2}
\newcommand{\djunion}[2]{#1 \mbox{\hspace{2 pt}$+$\hspace{2 pt}} #2}
\newcommand{\parens}[1]{\left( #1 \right)}
\newcommand{\brackets}[1]{\left[ #1 \right]}
\newcommand{\nint}[1]{\widetilde{N}\left(#1\right)}
\newcommand{\DefinedAs}{\mathrel{\mathop:}=}

\def\adj{\leftrightarrow}
\def\nonadj{\not\!\leftrightarrow}

\def\D{\fancy{D}}
\def\C{\fancy{C}}
\def\Q{\fancy{Q}}
\def\Z{\fancy{Z}}

\newcommand{\bigclique}[1]{\frac{2}{3}\Delta(#1) + 5}
\newcommand{\bigcliqueraw}{\frac{2}{3}\Delta + 5}
\newcommand{\cliqueparts}{\frac{2}{3}\Delta}

% any changes to \claim should be mirrored in \claimnonum and \subclaim
\newcommand{\claim}[2]{{\bf Claim #1.}~{\it #2}~~}
\newcommand{\claimnonum}[1]{{\bf Claim.}~{\it #1}~~}
\newcommand{\subclaim}[2]{{\bf Subclaim #1.}~{\it #2}~~}


\title{notes on coloring cayley graphs}
\begin{document}
\maketitle
\section{Basics}

\begin{defn}
For a group $G$ and $A \subseteq G$, the \emph{cayley graph} of $G$ with respect to $A$ is the directed graph with vertex set $G$ and an edge from $x$ to $xa$ for each $x \in G$ and $a \in A$.
Write $\C(G,A)$ for this digraph.
\end{defn}

We are concerned with coloring undirected graphs without loops, so we want $A$ to not contain the identity element of $G$ and $\frac{1}{A} = A$, where
\[\frac{1}{A} = \setbs{a^{-1}}{a \in A}.\]
Given this, $\C(G,A)$ has all edges directed both ways.  Let $G_A$ be the undirected graph with the structure of $\C(G,A)$.  We call such $G_A$ a \emph{standard cayley graph}.

\begin{remark}
$a$ and $b$ are adjacent in a standard cayley graph $G_A$ just in case $ab^{-1} \in A$.
\end{remark}


\begin{conjecture}
Let $G$ be an abelian group and $G_A$ a standard cayley graph.  If $\Delta(G) \ge 9$ and $\omega(G) < \Delta(G)$, then $\chi(G) < \Delta(G)$.
\end{conjecture}
i am trying to make the $\Delta=8$ example as a cayley graph of $C_5 \times C_3$, with the standard generators, its missing some edges though so need to throw more into $A$.

\begin{lem}\label{IndependentSetShiftingLemma}
If $a$ and $b$ are adjacent in a standard cayley graph $G_A$, then for any independent set $X$ in $G_A$
\[\frac{1}{X}a \bigcap \frac{1}{X}b = \emptyset.\]
\end{lem}
\begin{proof}
Suppose there is $c \in \frac{1}{X}a \bigcap \frac{1}{X}b$.  Then $c = x^{-1}a$ and $c = y^{-1}b$ for some $x,y \in X$. So $yx^{-1} = ba^{-1} \in A$, so $x$ and $y$ are adjacent, but they can't be since 
both are in the independent set $X$.
\end{proof}
Since we are just working with abelian groups now, we can use a nicer form of Lemma \ref{IndependentSetShiftingLemma}.

\begin{lem}\label{IndependentSetShiftingLemmaAbelian}
Let $G$ be an abelian group and $G_A$ a standard cayley graph.   Then for any clique $K$ and independent set $X$ in $G_A$,
\begin{enumerate}
\item $\card{XK} = \card{X}\card{K}$, and\\
\item $\card{\frac{1}{X}K} = \card{X}\card{K}$
\end{enumerate}
\end{lem}
\begin{proof}
Part (2) is immediate from Lemma \ref{IndependentSetShiftingLemma} since the sets $\setbs{\frac{1}{X}a}{a \in K}$ are pairwise disjoint and $\card{\frac{1}{X}} = \card{X}$.

For (1), suppose $a,b \in K$ are different vertices such that $Xa \cap Xb \ne \emptyset$.  Then for $c \in Xa \cap Xb$, we have $c = xa = yb$ for some $x,y \in X$.  But then
$ab^{-1} = x^{-1}y = yx^{-1}$.  But $a$ and $b$ are adjacent, so $ab^{-1} \in A$, so $yx^{-1} \in A$, so $x$ and $y$ are adjacent, a contradiction.  Now (1) follows in the same way as (2).
\end{proof}
Using this, we can get our first bound on the chromatic number.
\begin{thm}\label{OmegaTilingColoring}
Let $G$ be an abelian group and $G_A$ a standard cayley graph.  Then 
\[\chi(G_A) \le \omega(G_A) + |G| - \omega(G_A)\alpha(G_A).\]
\end{thm}
\begin{proof}
Take a maximum independent set $X$ in $G_A$ and maximum clique $K$ in $G_A$.  By Lemma \ref{IndependentSetShiftingLemmaAbelian} $\setbs{Xa}{a \in K}$ is 
a collection of pairwise disjoint maximum independent sets in $G_A$.  Using one color for each of those and then one color for each vertex in $G_A - XK$ gives the bound.
\end{proof}

Generally, that is a terrible bound, but we have a lot of room for improvement in coloring the leftover bit $G_A - XK$.  The case where $G_A - XK$ is empty matches up nicely
with $\chi(G_A) = \omega(G_A)$ in the $\frac56$-bound.  We want to show that when there is some of the leftover bit $G_A - XK$, we can color it with something like $\frac56\Delta(G_A) - \omega(G_A)$ colors.
There is a lot to play with here. For example, we can swap a vertex in $G_A - XK$ that has only one neighbor in $X$ for its neighbor to get $X'$.  Now we get a new coloring by looking at $X'K$ which has a lot in common
with our previous coloring.  Right now i am trying to see what sorts of information we can get out of this.
\end{document}

\documentclass[12pt]{amsart}
\usepackage{amsmath, amsthm, amssymb}
\usepackage[top=1.25in, bottom=1.25in, left=1.0in, right=1.0in]{geometry}
\usepackage{hyperref}
\usepackage{color}
\usepackage{verbatim}
\usepackage{tikz,tkz-graph}

\makeatletter
\newtheorem*{rep@theorem}{\rep@title}
\newcommand{\newreptheorem}[2]{
\newenvironment{rep#1}[1]{
 \def\rep@title{#2 \ref{##1}}
 \begin{rep@theorem}}
 {\end{rep@theorem}}}
\makeatother

\theoremstyle{plain}
\newtheorem{thm}{Theorem}
\newreptheorem{thm}{Theorem}
\newtheorem{prop}[thm]{Proposition}
\newreptheorem{prop}{Proposition}
\newtheorem{lem}[thm]{Lemma}
\newreptheorem{lem}{Lemma}
\newtheorem{lemma}[thm]{Lemma}
\newtheorem*{lemmaA}{Tashkinov's Lemma}
\newreptheorem{lemma}{Lemma}
\newtheorem{conj}[thm]{Conjecture}
\newreptheorem{conj}{Conjecture}
\newtheorem{cor}[thm]{Corollary}
\newreptheorem{cor}{Corollary}
\newtheorem{prob}[thm]{Problem}
\theoremstyle{definition}
\newtheorem{defn}{Definition}
\newtheorem{clm}{Claim}
\newtheorem{obs}[thm]{Observation}
\theoremstyle{remark}
\newtheorem*{remark}{Remark}
\newtheorem{example}{Example}
\newtheorem*{question}{Question}

\newcommand{\fancy}[1]{\mathcal{#1}}
%\newcommand{\C}[1]{\fancy{C}_{#1}}
\newcommand{\C}{\fancy{C}}
\newcommand{\F}{\fancy{F}}
\newcommand{\W}{\fancy{W}}
\newcommand{\IN}{\mathbb{N}}
\newcommand{\IR}{\mathbb{R}}
\newcommand{\G}{\fancy{G}}
\newcommand{\LB}{\mathcal{L}_B}
\newcommand{\col}{{\textrm{col}}}
\newcommand{\ch}{{\textrm{ch}}}
\newcommand{\chil}{{\chi_{\ell}}}
\newcommand{\chiol}{{\chi_{OL}}}
\newcommand{\T}{\fancy{T}}

\newcommand{\inj}{\hookrightarrow}
\newcommand{\surj}{\twoheadrightarrow}

\newcommand{\set}[1]{\left\{ #1 \right\}}
\newcommand{\setb}[3]{\left\{ #1 \in #2 : #3 \right\}}
\newcommand{\setbs}[2]{\left\{ #1 : #2 \right\}}
\newcommand{\card}[1]{\left|#1\right|}
\newcommand{\size}[1]{\left\Vert#1\right\Vert}
\newcommand{\ceil}[1]{\left\lceil#1\right\rceil}
\newcommand{\floor}[1]{\left\lfloor#1\right\rfloor}
\newcommand{\func}[3]{#1\colon #2 \rightarrow #3}
\newcommand{\funcinj}[3]{#1\colon #2 \inj #3}
\newcommand{\funcsurj}[3]{#1\colon #2 \surj #3}
\newcommand{\irange}[1]{\left[#1\right]}
\newcommand{\join}[2]{#1 \mbox{\hspace{2 pt}$\ast$\hspace{2 pt}} #2}
\newcommand{\djunion}[2]{#1 \mbox{\hspace{2 pt}$+$\hspace{2 pt}} #2}
\newcommand{\parens}[1]{\left( #1 \right)}
\newcommand{\brackets}[1]{\left[ #1 \right]}
\newcommand{\DefinedAs}{\mathrel{\mathop:}=}

\newcommand{\mic}{\operatorname{mic}}
\newcommand{\AT}{\operatorname{AT}}
\renewcommand{\col}{\operatorname{col}}
\renewcommand{\ch}{\operatorname{ch}}
\newcommand{\type}{\operatorname{type}}
\newcommand{\nonsep}{\bar{S}}
\newcommand{\dclaw}[1]{d_{\text{claw}}\left( #1 \right)}

\def\adj{\leftrightarrow}
\def\nonadj{\not\!\leftrightarrow}

\newcommand{\vph}{\varphi}
\newcommand{\vphn}{\overline{\varphi}}

\newcommand{\claim}[2]{{\noindent\bf Claim #1.}~{\it #2}~~}
\newenvironment{claimproof}[1]{\par\noindent\underline{Proof:}\space#1}{\leavevmode\unskip\penalty9999
\hbox{}\nobreak\hfill\quad\hbox{$\qed$}}

\begin{document}
\section{Overview}

For a multigraph $G$, we clearly have $\chi'(G)\ge
\ceil{|E(G)|/\floor{|V(G)|/2}}$.  Likewise, the same bound holds for any
subgraph $H$.  Thus, 
The biggest open problem in edge-coloring is the Goldberg--Seymour conjecture.
Over the past two decades, the main tool for attacking this problem has become
Tashkinov trees, a vast generalization of Vizing fans and Kierstead paths.
The second author proved that if $G$ is a line graph, then $\chi(G)\le
\max\{\omega(G),\frac{7\Delta(G)+10}{8}\}$.  In the same paper, he conjectured
that $\chi(G)\le \max\{\omega(G),\frac{5\Delta(G)+8}{6}\}$. This conjecture is best
possible, as shown by replacing each edge in a 5-cycle by $k$ parallel edges,
and taking the line graph.
We call the latter inequality the $\frac56$-Conjecture, and in this paper we prove
it.  Along the way, we develop more general techniques and results that will
likely be of independent interest, due to their use in approaching the
Goldberg--Seymour conjecture.

A graph $G$ is \emph{elementary} if $\chi'(G)=\W(G)$; such graphs satisfy the
Goldberg--Seymour Conjecture.  
%(We begin by proving that every minimal counterexample to the
%$\frac56$-Conjecture is elementary.  In Section 3, we conclude
%by also proving that every elementary graph satisfies the $\frac56$-conjecture.)  
A
\emph{defective color}
for a Tashkinov tree $T$ is a color used on more than one edge from $V(T)$ to
$V(G)-V(T)$; a Tashkinov tree is \emph{strongly closed} if it has no defective
color.  Andersen~[] and Goldberg~[] showed that if $G$ is critical, then $G$ is
elementary if there exists $e\in E(G)$ and $X\subseteq V(G)$ and a
$k$-edge-coloring $\vph$ of $G-e$ such that $X$ contains the endpoints of $e$
and $X$ is elementary and strongly closed w.r.t.~$\vph$.  Thus, to show that $G$
is elementary, it suffices to show that if $G$ is $(k+1)$-critical, then there
exists an edge $e\in E(G)$ and a $k$-coloring $\vph$ of $G-e$ such that some
maximal Tashkinov tree containing $e$ is strongly closed.  The following
definition is useful.  
A vertex $v \in V(G)$ is \emph{special} if every Vizing fan rooted at $v$ (taken over all
$k$-colorings of $G-e$, over all edges $e$ incident to $v$) has at most 3 vertices,
including $v$.
As a warmup, in Section 2 we prove that if $\chi'(G)\ge \Delta(G)+2$ and every
vertex of $G$ is special, then $G$ is elementary, i.e., $\chi'(G)=\W(G)$.  Next,
we push our methods further, allowing our maximal Tashkinov tree to have at most
3 non-special vertices.

In Section 3, we consider the %show that 
$\frac56$-Conjecture.  As a consequence of results in Section 2,
if $G$ is a minimal counterexample, then every non-special vertex $v$ has
$d_G(v)<\frac34\Delta(G)$.  Since every maximal Tashkinov tree $T$ is
elementary, and every non-special vertex misses more than $\frac{k}4$ colors, we
conclude that $T$ has at most 3 non-special vertices.  Thus our results from
Section 2 apply.  As a consequence, every minimal countexample to the
$\frac56$-Conjecture is elementary.  To complete the proof of the $\frac56$-conjecture,
%we prove that it follows from the Goldberg--Seymour Conjecture.  More precisely,
we show for each graph $G$ that if $\chi'(G)=\W(G)$, then
$\chi'(G)\le\max\{\omega(G),\frac{5\Delta(G)+8}6\}$.


Graphs can have multiple edges.

\section{Tashkinov Trees}
Recall that a graph $G$ is \emph{elementary} if $\chi'(G)=\W(G)$, as defined
above.  We also use the following notation.  
Let $[k]$ denote $\{1,\ldots,k\}$.
A graph $G$ is \emph{critical} if $\chi'(G-e) < \chi'(G)$ for all $e \in E(G)$. 
For a graph $G$ and a partial $k$-edge-coloring $\varphi$, for each vertex $v\in
V(G)$, let $\varphi(v)$ denote the set of colors used in $\varphi$ on edges
incident to $v$.  Let $\vphn(v)=[k]\setminus\varphi(v)$.  A color $c$ is
\emph{seen} by a vertex $v$ if $c\in \varphi(v)$ and $c$ is \emph{missed} by $v$
if $c\in\vphn(v)$.
Given a partial $k$-edge-coloring $\varphi$, a set $W\subseteq V(G)$ is
\emph{elementary} (w.r.t.~$\varphi$) if each color in $[k]$ is
missed by at most one vertex of $W$.  More formally, $\vphn(u)\cap
\vphn(v)=\emptyset$ for all distinct $u,v\in W$.
A \emph{defective color} for a set $X\subseteq V(G)$ (w.r.t.~$\varphi$) is a color
used on more than one edge from $X$ to $V(G) \setminus X$.  
A set $X$ is \emph{strongly closed} w.r.t.~$\varphi$ if $X$ has no 
defective color.
Elementary and strongly closed sets are of particular interest because of the
following theorem, proved implicitly by Andersen~\cite{} and Goldberg~\cite{}.
%As we will see shortly, there is a strong relationship between elementary sets
%and elementary graphs.
% 

\begin{thm}
\label{elementary}
Let $G$ be a graph with $\chi'(G)=k+1$ for some integer $k\ge \Delta(G)$.  If
$G$ is critical, then $G$ is elementary if and only if there exists $uv\in E(G)$,
a $k$-edge-coloring $\vph$ of $G-uv$, and a set $X$ with $u,v\in X$ such
that $X$ is both elementary and strongly closed w.r.t.~$\varphi$.
\end{thm}

A \emph{Tashkinov tree} w.r.t.~$\varphi$ is a sequence $v_0, e_1, v_1,
e_2,\ldots, v_{t-1},e_t,v_t$ such that all $v_i$ are distinct, $e_i=v_jv_i$ and
$\vph(e_i)\in \vphn(v_\ell)$ for some $j$ and $\ell$ with $0\le j< i$ and $0\le
\ell < i$.  Tashkinov trees are of interest because of the following lemma. 

\begin{lemmaA}%[Tashkinov]
Let $G$ be a graph with $\chi'(G)=k+1$, for some integer $k\ge \Delta(G)+1$ and
choose $e\in E(G)$ such that $\chi'(G-e)<\chi'(G)$.  Let $\varphi$ be a
$k$-edge-coloring of $G-e$.  If $T$ is a Tashkinov tree w.r.t.~$\varphi$ and
$e$, then $V(T)$ is elementary w.r.t.~$\varphi$.
\end{lemmaA}

In view of Theorem~1 and Tashkinov's Lemma, to prove that a graph $G$ is elementary,
it suffices to find an edge $e$, a $k$-edge-coloring $\vph$ of $G-e$, and a
Tashkinov tree $T$ containing $e$ such that $V(T)$ is strongly closed.
This motivates our next two lemmas.  But first, we need a few more definitions.

Let $t(G)$ be the maximum number of vertices in a Tashkinov tree over all $e \in E(G)$
and all $k$-edge-colorings $\vph$ of $G - e$.  Let $\T(G)$ be the set of all triples $(T,e,\vph)$ such that $e \in E(G)$, $\vph$ is a $k$-edge-coloring of $G-e$ and
$T$ is a Tashkinov tree with respect to $e$ and $\vph$ with $|T| = t(G)$.  Notice that, by definition, we have $\T(G) \ne \emptyset$.
%
For a $k$-edge-coloring $\vph$ of $G-e$, a maximal Tashkinov tree
starting with $e$ may not be unique.  However, if $T_1$ and $T_2$ are both such
trees, then it is easy to show that $V(T_1)\subseteq V(T_2)$; by symmetry, also
$V(T_2)\subseteq V(T_1)$, so $V(T_1)=V(T_2)$.
%
Let $G$ be a critical graph with $\chi'(G) = k+1$ for an integer $k \ge \Delta(G) + 1$. 
Let $\varphi$ be a $k$-edge-coloring of $G - e_0$ for some $e_0 \in E(G)$.  
%For each vertex $v\in V(G)$, let $\varphi(v)$ be the set of colors used in
%$\vph$ on edges incident to $v$ and let $\vphn(v)=[k]\setminus \vph(v)$. 
For $v \in V(G)$ and colors $\alpha, \beta$, let $P_v(\alpha, \beta)$ be the
maximal connected subgraph of $G$ that contains $v$ and is induced by edges with color
$\alpha$ or $\beta$.  So $P_v(\alpha, \beta)$ is a path or a cycle.

%TODO: NEED DEFINITION OF ELEMENTARY SET, ELEMENTARY GRAPH.  NEED TO STATE LEMMA THAT TASHKINOV TREES ARE ELEMENTARY, ALSO THAT MAXIMUM SIZE TASHKINOV TREE WITHOUT DEFECTIVE COLORS IMPLIES G IS ELEMENTARY.

\begin{lem}\label{FreeColorsLemma}
Let $G$ be a non-elementary critical graph with $\chi'(G) = k+1$ for an integer
$k \ge \Delta(G) + 1$.  For every $v_0v_1 \in E(G)$, $k$-edge-coloring $\vph$ of $G-v_0v_1$ and for all $\alpha \in \vphn(v_0)$ and $\beta \in
\vphn(v_1)$ we have $|P_{v_1}(\alpha, \beta)| < t(G)$.
\end{lem}
\begin{proof}
Suppose the lemma is false and choose $v_0v_1 \in E(G)$, a $k$-edge-coloring $\vph$ of $G-v_0v_1$, $\alpha \in \vphn(v_0)$ and $\beta \in
\vphn(v_1)$ such that $|P_{v_1}(\alpha, \beta)| \ge t(G)$.  Put $P = P_{v_1}(\alpha, \beta)$.  
Clearly $P$ must end at $v_0$ (or we can swap colors $\alpha$ and $\beta$ on $P$ and color $v_0v_1$), so let
$v_1,\ldots,v_r,v_0$ denote the vertices of $P$ in order. Let $(T, v_0v_1, \vph)$ be a Tashkinov tree that begins with edges
$v_0v_1, v_1v_2, \ldots, v_{r-1}v_r$.  Then $V(T)=V(P)$ since $t(G) \ge |T| \ge |P| \ge t(G)$.
Since $G$ is non-elementary, Theorem~\ref{elementary} implies that $V(T)$ is not
strongly closed, so $T$ has a defective color $\delta$ with respect to $\vph$.
Choose $\tau\in \vphn(v_2)$. Let $Q = P_{v_2}(\tau, \delta)$.
Since $T$ is maximal, $\delta$ is not missing at any vertex of $T$;
since $V(T)$ is elementary, $\tau$ is not missing at any vertex of $T$ other 
than $v_2$.  As a result, $Q$ ends outside $V(T)$.  Now $Q$ could leave
$V(T)$ and re-enter it repeatedly, but $Q$ ends outside $V(T)$, so there is a
last vertex $w \in V(Q) \cap V(T)$;
say $Q$ ends at $z \in V(G)\setminus V(T)$.  Let $\pi \notin \{\alpha, \beta\}$ be a
color missing at $w$.  Since
$|T| = t(G)$, no edge colored $\tau$ or $\pi$ leaves $V(T)$.  So, we can
swap $\tau$ and $\pi$ on every edge in $G - V(T)$ without changing the fact that $T$ is a Tashkinov tree with
$|T| = t(G)$.  Now swap $\delta$ and $\pi$ on the
subpath of $Q$ from $w$ to $z$;
since $\pi$ is missing at $w$, the $\delta-\pi$ path does end at $w$.  Now
$\delta$ is missing at $w$, but $\delta$ was defective in $\vph$, so some other
edge $e$ colored $\delta$ still leaves $V(T)$, adding $e$ gets a larger
Tashkinov tree, a contradiction.
\end{proof}


\section{Special vertices}
Recall that a vertex $v \in V(G)$ is \emph{special} if every Vizing fan rooted
at $v$ (taken over all $k$-colorings of $G-e$, over all edges $e$ incident to
$v$) has at most 3 vertices, including $v$.
%TODO: DEFINE SPECIAL.
Let $\nu(T)$ be the number of non-special vertices in $T$.
\begin{lem}\label{SpecialPath}
Let $G$ be a critical graph with $\chi'(G) = k+1$ for an integer $k \ge \Delta(G) + 1$.
Let $\vph$ be a $k$-edge-coloring of $G-v_0v_1$.  Suppose $\alpha \in \vphn(v_0)$ and $\beta \in \vphn(v_1)$.  
Let $P = v_1v_2\cdots v_r$ be an $\alpha-\beta$ path
with edges $e_i = v_iv_{i+1}$ for $1 \le i \le r-1$.  If $v_i$ is special for
all odd $i$, then for any $\tau \in \vphn(v_0)$ there are edges $f_i =
v_iv_{i+1}$ for $1 \le i \le r-1$ such that $f_i = e_i$ for $i$ even and
$\vph(f_i) = \tau$ for $i$ odd.
\end{lem}
\begin{proof}
Suppose not and choose a counterexample minimizing $r$.  By minimality of
$r$, we have $\vph(v_{r-1}v_r) = \alpha$ and we have $f_i = v_iv_{i+1}$ for
$1 \le i \le r-2$ such that $f_i = e_i$ for $i$ even and $\vph(f_i) = \tau$ for
$i$ odd.  Swap $\alpha$ and $\beta$ on $e_i$ for $1 \le i \le r-3$ and then
color $v_0v_1$ (call this edge $e_0$) with $\alpha$ and uncolor $e_{r-2}$.  Let
$\vph'$ be the resulting coloring.  Since $k \ge \Delta(G) + 1$, some color
other than $\alpha$ is missing at $v_{r-2}$; let $\gamma$ be such a color.  Now 
$v_{r-1}$ is special since $r-1$ is odd (since $P$ starts and ends with
$\alpha$), so there is an edge $e = v_{r-1}v_r$ with $\vph'(e) = \gamma$.  
Swap $\tau$ and $\alpha$ on $e_i$ for $0 \le i \le r-3$ to get a new coloring
$\vph^*$.  Now $\gamma$ and $\tau$ are both missing at $v_{r-2}$ in $\vph^*$.
Since $v_{r-1}$ is special, the fan with $v_{r-2}, v_{r-1}, v_r$ and $e$
implies that there is an edge $f_{r-1} = v_{r-1}v_r$ with $\vph^*(f_{r-1}) =
\tau$.  But we have never recolored $f_{r-1}$, so $\vph(f_{r-1})=\tau$, a
contradiction.
\end{proof}

\begin{lem}\label{ZeroNonSpecial}
Let $G$ be a non-elementary critical graph with $\chi'(G) = k+1$ for an integer $k \ge \Delta(G) + 1$.
Let $(T, v_0v_1, \vph) \in \T(G)$ for some $v_0v_1 \in E(G)$.  Let $\alpha \in \vphn(v_0)$ and $\beta \in \vphn(v_1)$ and put $P = P_{v_1}(\alpha, \beta)$.  The $P$ contains a non-special vertex. 
In particular, $\nu(T) \ge 1$.
\label{lem2}
\end{lem}
\begin{proof}
Suppose every vertex of $P$ is special.  Applying Lemma \ref{SpecialPath} to $P$ shows that every $\tau \in \vphn(v_0)$, 
there is a $\tau$-edge in $T$ incident to every $v \in V(P - v_0)$. By symmetry, the same is true of every $v \in V(P)$.  Hence $V(P) = V(T)$ contradicting Lemma \ref{FreeColorsLemma}.
\end{proof}

\begin{thm}\label{AllSpecialImpliesElementary}
If $G$ is a critical graph in which every vertex is special, then
\[\chi'(G) \le \max \set{\ceil{\chi'_f(G)}, \Delta(G) + 1}.\]
\end{thm}
\begin{proof}
Suppose $G$ is a critical graph in which every vertex is special and put $k = \chi'(G) - 1$.  Then $k \ge \Delta(G) + 1$.
Since $\T(G) \ne \emptyset$, applying Lemma \ref{ZeroNonSpecial}, we conclude that $G$ is elementary.  Hence $\chi'(G) = \ceil{\chi'_f(G)}$, a contradiction.
\end{proof}

\section{The easy bound}
Let $G$ be a graph.  The \emph{claw-degree} of $x \in V(G)$ is 
\[\dclaw{x} \DefinedAs \max_{\substack{S \subseteq N(x) \\ \card{S} = 3}}\frac14 \parens{d(x) + \sum_{v \in S} d(v)}.\]
The \emph{claw-degree} of $G$ is 
\[\dclaw{G} \DefinedAs \max_{x \in V(G)} \dclaw{x}.\]
\begin{thm}\label{EasyBound}
If $G$ is a graph, then
\[\chi'(G) \le \max\set{\ceil{\chi'_f(G)}, \Delta(G) + 1, \ceil{\frac43\dclaw{G}}}.\]
\end{thm}
\begin{proof}
Suppose not and choose a counterexample $G$ minimizing $\size{G}$; note that $G$ critical. 
Let $k=\chi'(G)-1$, so $k \ge \ceil{\frac43\dclaw{G}}$. 
By Theorem \ref{AllSpecialImpliesElementary}, $G$ has a non-special vertex $x$.
Choose $xy_1 \in E(G)$ and a $k$-edge-coloring $\vph$ of $G - xy_1$ such that
$\vph$ has a fan $F$ of length $3$ rooted at $x$ with leaves $y_1, y_2, y_3$.  
Since $V(F)$ is elementary, 
\[2 + k - d(x) + \sum_{i \in \irange{3}} k-d(y_i) \le k,\]
and hence
\[\dclaw{x} \ge \frac14\parens{d(x) + \sum_{i \in \irange{3}} d(y_i)} \ge \frac{3k+2}{4}.\]
This gives the contradiction
\[\ceil{\frac43\dclaw{G}} \le k \le \frac43\dclaw{G} - \frac23.\]
\end{proof}

TODO: ADD REED, LOCAL REED AND SUPERLOCAL REED CONSEQUENCES.

\section{Properties of non-special vertices}
For a path $Q$ and $x,y \in V(Q)$, let $d_Q(x,y) = \ell(Q)$.

\begin{lem}\label{NonSpecialsInThinAreAtEvenDistance}
Let $G$ be a critical graph with $\chi'(G) = k+1$ for an integer $k \ge \Delta(G) + 1$.
Let $\vph$ be a $k$-edge-coloring of $G-v_0v_1$. Suppose $\alpha \in \vphn(v_0)$ and $\beta \in \vphn(v_1)$ and let $C = P_{v_1}(\alpha, \beta) + v_0v_1$.
If $\tau \in \vphn(x)$ for some $x \in V(C)$ and there is a $\tau$-colored edge from $y \in V(C)$ to $w \in V(G) \setminus V(C)$, then $C$ has a 
subpath $Q$ with $x \in V(Q)$, $y \not \in V(Q)$ and non-special endpoints $z_1,z_2$ such that $d_Q(x, z_i)$ is odd for $i \in \irange{2}$.  
Moreover, there are no $\tau$-colored edges between $z_i$ and $N_C(z_i)$ for $i \in \irange{2}$.
\end{lem}
\begin{proof}
TODO: FILL THIS SPACE WITH PROOF.
\end{proof}

\section{Thin graphs}
Let $G$ be a critical graph with $\chi'(G) = k+1$ for an integer $k \ge \Delta(G) + 1$.
For vertices $x,y \in V(G)$, we say that $x$ is \emph{$y$-special} if every Vizing fan rooted at $x$, with respect to any $k$-edge-coloring of $G-xy$, has at most 3 vertices.
We say that $G$ is \emph{$k$-thin} if $\mu(G) < 2k - d(x) - d(y)$ for all non-special $x,y \in V(G)$.

\begin{lem}\label{NonSpecialsInThinAreAtEvenDistance}
Let $G$ be a $k$-thin, critical graph with $\chi'(G) = k+1$ for an integer $k \ge \Delta(G) + 1$.
Let $\vph$ be a $k$-edge-coloring of $G-v_0v_1$. Suppose $\alpha \in \vphn(v_0)$ and $\beta \in \vphn(v_1)$ and let $C = P_{v_1}(\alpha, \beta) + v_0v_1$.
If $Q$ is a subpath of $C$ with non-special end vertices and all special internal vertices such that $2 \le \ell(Q) \le \ell(C) - 2$, then $\ell(Q)$ is even.
\end{lem}
\begin{proof}
Suppose to the contrary that we have a subpath $Q$ of $C$ with non-special end vertices and all special internal vertices, such that $\ell(Q) \le \ell(C) - 2$ and $\ell(Q)$ is odd.  Let $x$ and $y$ be the end vertices of $Q$.
Say $C = v_1v_2\cdots v_rv_0v_1$.  By rotating the $\alpha-\beta$ coloring of $C$, we may assume that $x = v_1$ and $y = v_a$ where $a \ge 4$ is even.

Apply Lemma \ref{SpecialPath} twice, to show that $\mu(v_2v_3) \ge 2k - d(v_1) - d(v_a)$, violating $k$-thinness.
\end{proof}

\begin{lem}\label{ThreeNonSpecialOnCycle}
Let $G$ be a $k$-thin, critical graph with $\chi'(G) = k+1$ for an integer $k \ge \Delta(G) + 1$.
Let $\vph$ be a $k$-edge-coloring of $G-v_0v_1$. Suppose $\alpha \in \vphn(v_0)$ and $\beta \in \vphn(v_1)$ and let $C = P_{v_1}(\alpha, \beta) + v_0v_1$.  If $C$ contains exactly 3 non-special vertices,
then $C = xyAzBx$ where $A$ and $B$ are paths of even length and $x,y,z$ are all non-special.  Moreover, $x$ is $y$-non-special and $y$ is $x$-non-special.
\end{lem}
\begin{proof}
Immediate from Lemma \ref{NonSpecialsInThinAreAtEvenDistance} and Lemma \ref{SpecialPath}.
\end{proof}

\begin{lem}\label{ConsecutiveNonSpecials}
Let $G$ be a non-elementary, $k$-thin, critical graph with $\chi'(G) = k+1$ for an integer $k \ge \Delta(G) + 1$.
Let $(T, v_0v_1, \vph) \in \T(G)$. Suppose $\alpha \in \vphn(v_0)$ and $\beta \in \vphn(v_1)$.  Then there are consecutive non-special vertices on $P_{v_1}(\alpha, \beta) + v_0v_1$.
\end{lem}
\begin{proof}
Put $C = P_{v_1}(\alpha, \beta) + v_0v_1$.  By Lemma \ref{FreeColorsLemma}, there is $x \in V(C)$ and $\tau \in \vphn(x)$ such that there is a $\tau$-colored edge from $y \in V(C)$ to $w \in V(T) \setminus V(C)$.
By Lemma \ref{NonSpecialsInThinAreAtEvenDistance}, $C$ has a subpath $Q$ with $x \in V(Q)$, $y \not \in V(Q)$ and non-special endpoints $z_1,z_2$ such that $d_Q(x, z_i)$ is odd for $i \in \irange{2}$.  
Let $Q'$ be the subpath of $Q$ with endpoints $z_1$ and $z_2$ that contains $y$. Since $C$ is an odd cycle, $\ell(Q')$ is odd.  Let $Q^*$ be a minimum length subpath of $Q'$ with non-special ends.  
Then $\ell(Q^*) = 1$ by Lemma \ref{NonSpecialsInThinAreAtEvenDistance}, as desired.
\end{proof}


\begin{lem}\label{MasterHelper}
Let $G$ be a non-elementary, $k$-thin, critical graph with $\chi'(G) = k+1$ for an integer $k \ge \Delta(G) + 1$.
Let $(T, v_0v_1, \vph) \in \T(G)$ with $\nu(T) \le 3$. Choose $\alpha \in \vphn(v_0)$ and $\beta \in \vphn(v_1)$ so that $C = P_{v_1}(\alpha, \beta) + v_0v_1$ contains as many non-special vertices as possible.  
Then $C$ contains non-special vertices $z_1,z_2,z_3$ such that $z_i$ is $z_{3-i}$-non-special and $z_{i+1}$ is $z_{4-i}$-non-special for $i \in \irange{2}$.
\end{lem}
\begin{proof}
By Lemma \ref{FreeColorsLemma}, there is $x \in V(C)$ and $\tau \in \vphn(x)$ such that there is a $\tau$-colored edge from $y \in V(C)$ to $w \in V(T) \setminus V(C)$.

First, suppose $C$ contains only 2 non-special vertices, $z_1$ and $z_2$.  By Lemma \ref{ConsecutiveNonSpecials}, $z_1$ and $z_2$ are consecutive on $C$.
By Lemma \ref{NonSpecialsInThinAreAtEvenDistance}, $C$ has a subpath $Q$ with $x \in V(Q)$ and $y \not \in V(Q)$ with endpoints $z_1,z_2$ and there are no 
$\tau$-colored edges between $z_i$ and $N_C(z_i)$ for $i \in \irange{2}$.  

By rotating the $\alpha-\beta$ coloring of $C$, we may assume that $x = v_0$. Consider $C' = P_{v_1}(\tau, \beta) + v_0v_1$.  Since $z_1$ and $z_2$ are not consecutive on $C'$ and
$C'$ contains no other non-special vertices by the maximality condition on $C$, Lemma \ref{ConsecutiveNonSpecials} gives a contradiction.

So, $C$ contains exactly 3 non-special vertices, $z_1$, $z_2$ and $z_3$.  By Lemma \ref{ThreeNonSpecialOnCycle}, $C = z_1z_2Az_3Bz_1$ where $A$ and $B$ are paths of even length.  Also,
$z_1$ is $z_2$-special and $z_2$ is $z_1$-special.  

By Lemma \ref{NonSpecialsInThinAreAtEvenDistance}, $C$ has a subpath $Q$ with $x \in V(Q)$ and $y \not \in V(Q)$ with endpoints $z_1,z_3$ and there are no 
$\tau$-colored edges between $z_i$ and $N_C(z_i)$ for $i \in \set{1,3}$ (it could happen that $z_3$ is bypassed and the endpoints are $z_1, z_2$, but then we get a contradiction as in the previous case).
By rotating the $\alpha-\beta$ coloring of $C$, we may assume that $x = v_0$.  
Consider $C' = P_{v_1}(\tau, \beta) + v_0v_1$.  We know that $C'$ contains $z_1$ and $z_3$ and that $z_1$ and $z_2$ are not consecutive on $C'$.  By Lemma \ref{ConsecutiveNonSpecials}, either $z_1$ and $z_3$ are consecutive on 
$C'$ or $z_2$ and $z_3$ are consecutive on $Q'$.

Suppose $z_2$ and $z_3$ are consecutive on $C'$ and connected by a $\tau$ edge.  Then applying Lemma \ref{NonSpecialsInThinAreAtEvenDistance} shows that $z_2$ is $z_3$-special and $z_3$ is $z_2$-special, so we win.

So it must be that $z_1$ and $z_3$ are consecutive on $C'$ and connected by a $\tau$ edge.  TODO: FIGURE OUT WHAT GOOD THING WE GET IN THIS CASE.
\end{proof}

\begin{thm}[from strengthening Brooks paper]\label{CriticalMuBound}
If $Q$ is the line graph of a graph $G$ and $Q$ is vertex critical, then
\[\chi(Q) \leq \max\left\{\omega(Q), \Delta(Q) + 1 - \frac{\mu(G) - 1}{2}\right\}.\]
\end{thm}

\begin{thm}
If $Q$ is a line graph, then
\[\chi(Q) \le \max\set{\ceil{\chi_f(Q)}, \ceil{\epsilon(\Delta(Q) + 1)}}.\]
\end{thm}
\begin{proof}
Suppose the theorem is false and choose a counterexample minimizing $\card{Q}$.
Let $k = \max\set{\ceil{\chi_f(Q)}, \ceil{\epsilon(\Delta(Q) + 1)}}$. Say $Q =
L(G)$ for a graph $G$. The minimality of $Q$ implies that $G$ is $k$-edge-critical.

\claim{0}{Let $F$ be a fan rooted at $x$ with respect to a $k$-edge-coloring of $G - xy$.  If $|F| = 4$, then
\[d(x) < \frac{1-\epsilon}{2\epsilon -1}\sum_{v \in V(F-x)} d(v).\]}
\begin{claimproof}
Since $F$ is elementary, we have
\[2 + k-d(x) + \sum_{v \in V(F-x)} k - d(v) \le k,\]
so
\[2 + (|F| - 1)k \le d(x) + \sum_{v \in V(F-x)} d(v).\]
Using $k \ge \epsilon(\Delta(Q) + 1) \ge \epsilon(d(x) + d(v) - \mu(xv))$ for each $v \in V(F-x)$, we get
\[2 + \sum_{v \in V(F-x)}\epsilon(d(x) + d(v) - \mu(xv)) \le d(x) + \sum_{v \in V(F-x)} d(v),\]
so
\[2 + \parens{\epsilon|F| - 1 - \epsilon}d(x) \le \sum_{v \in V(F-x)} \epsilon\mu(xv) + \sum_{v \in V(F-x)} (1-\epsilon)d(v).\]
Now $\sum_{v \in V(F-x)} \mu(xv) \le d(x)$, so this implies
\[2 + \parens{\epsilon|F| - 1 - 2\epsilon}d(x) \le \sum_{v \in V(F-x)} (1-\epsilon)d(v).\]
Using $|F| = 4$ gives
\[d(x) < \frac{1-\epsilon}{2\epsilon -1}\sum_{v \in V(F-x)} d(v).\]
\end{claimproof}
\bigskip

\claim{1}{If $x \in V(G)$ with $d(x) \ge \frac{3(1-\epsilon)}{2\epsilon -1}\Delta(G)$, then $x$ is special.}

\begin{claimproof}
This is immediate from Claim 0, since $d(v)\le \Delta(G)$ for all $v\in V(F-x)$.
\end{claimproof}
\bigskip

Note that when $\epsilon=\frac56$, every non-special vertex $x$ has
$d(x)<\frac34\Delta(G)$.  Since $k\ge \Delta(G)+1$, every non-special vertex $x$ misses
at least $k-d(x)> k-\frac34\Delta(G)> \frac{k}4$ colors.
Since $T$ is elementary, $T$ has at most 3 non-special vertices.
This observation is not needed for the present proof, but suggests how we will
prove the $\frac56$-conjecture in the next section.
\bigskip

\claim{2}{If $x_1x_2 \in E(G)$ with \[d(x_i) \ge \frac{2(1-\epsilon)}{3\epsilon - 2}\Delta(G),\] for at least one $i \in \irange{2}$, then $x_1$ is $x_2$-special or $x_2$ is $x_1$-special.}

\begin{claimproof}
Suppose $x_1$ is not $x_2$-special and $x_2$ is not $x_1$-special.  By Claim 0,
for each $i \in \irange{2}$,
\[d(x_i) < \frac{1-\epsilon}{2\epsilon -1}\sum_{v \in V(F-x)} d(v) \le
\frac{1-\epsilon}{2\epsilon -1}\parens{d(x_{3-i}) + 2\Delta(G)},\]
Substituting the bound on $d(x_{3-i})$ into that on $d(x_i)$ and simplifying
gives for each $i \in \irange{2}$,
\[d(x_i) < \frac{2(1-\epsilon)}{3\epsilon - 2}\Delta(G).\]
\end{claimproof}
\bigskip

\claim{3}{The theorem is true.}

\begin{claimproof}
Let $(T, v_0v_1, \vph) \in \T(G)$. By Lemma \ref{MasterHelper}, one of the following holds:
\begin{enumerate}
\item $G$ is elementary; or
\item $G$ is not thin; or
\item $\nu(T) = 3$ and $E(T)$ contains non-special $x_1,x_2,x_3 \in V(T)$ such that $x_1$ is $x_2$-non-special, $x_2$ is $x_1$-non-special, $x_2$ is $x_3$-non-special and $x_3$ is $x_2$-non-special,; or
\item $V(T)$ contains four non-special vertices $x_1, x_2, x_3, x_4$.
\end{enumerate}

If (1) holds, then $k + 1 = \ceil{\chi_f(Q)} \le k$, a contradiction.

If (2) holds, then by Claim 1 we have $\mu(G) \ge 2k - 2\frac{3(1-\epsilon)}{2\epsilon -1}\Delta(G)$.  Hence Theorem \ref{CriticalMuBound} gives
\[k + 1 \le \Delta(Q) + 1 - k + \frac{3(1-\epsilon)}{2\epsilon -1}\Delta(G) + \frac12,\]
so
\[2(k + 1) \le \Delta(Q) + \frac52 + \frac{3(1-\epsilon)}{2\epsilon -1}\Delta(G).\]

Since $k \ge \Delta(G) + 1$, this gives

\[k + 1 < \frac{\Delta(Q) + \frac52}{2 - \frac{3(1-\epsilon)}{2\epsilon -1}},\]
which is a contradiction when $\epsilon > \frac45$.

Suppose (3) holds.  So
\[2 + \sum_{i \in \irange{3}} k - d(x_i) \le k,\]
using Claim 2, this gives
\[3\parens{\frac{2(1-\epsilon)}{3\epsilon - 2}}\Delta(G) \ge 2k+2,\]
which is a contradiction when $\epsilon \ge \frac56$.

So (4) must hold.  But then
\[2 + \sum_{i \in \irange{4}} k - d(x_i) \le k,\]
using Claim 1 gives
\[\frac{12(1-\epsilon)}{2\epsilon -1}\Delta(G) \ge 3k+2,\]
which is a contradiction when $\epsilon \ge \frac{5}{6}$.
\end{claimproof}

\end{proof}

\newpage

\section{The $\frac56$-Conjecture}
\begin{lem}
If $H$ is a connected multigraph and $G = L(H)$, then $\W(H) \le
\max\{\omega(G), \frac56(\Delta(G) + 1) + \frac36\}$.
\end{lem}
\begin{proof}
%$M = |N(x)|$,
Let $d = d_H(x)$,  $\Delta = \Delta(H)$, and $h = |H|$. 
Also, let $p = \sum_{v \in N(x)} d_H(v)$ and
%Here's a way that seems a little simpler to me.  
let $t = \Delta h-2||H||$.
Note that $0 < t \le \Delta$.  Also $p \ge Md - t$.  Now summing over $N_H(x)$
gives

\begin{align*}
|N(x)|(\Delta h-t)/(h-1) > 5/6((|N(x)|-1)d + |N(x)|\Delta - t) + |N(x)|/2
\end{align*}

Solving for $|N(x)|$ gives

\begin{align*}
|N(x)| < (5d+5t)/(3+5d+5\Delta-6(\Delta h-t)/(h-1)).
\end{align*}

Since the numerator and denominator are linear in $t$, the right side is
maximized at one end of the interval $1 \le t \le D$.  Letting $t = D$,
gives $|N(x)| < (5d+5\Delta)/(3+5d-\Delta)$, like you had originally.  Letting $t = 1$,
gives $|N(x)| < (5d+5)/(3+5d+5\Delta-6(\Delta h-1)/(h-1))$, which requires a little more
analysis, akin to what you wrote in your most recent email.

Does that look right to you?



I did the analysis a little differently, but I got to the same
conclusion: Substituting $d \ge 4D/5$ gives that if $M \ge 3$, then we
must have $h \le 4$, which implies $h \le 3$, which contradicts $M \ge 3$.

So, I think I believe it.  I also agree there must be an easier way.
One thing that seems a little magical is that when $5/6 - M/(h-1) \ge 0$
all of the h's go away.

$w(H)$ really has a ceiling in its definition, not sure how much that changes
things.   without, it is the fractional chromatic index.

i think we get some gain as well from the $\Delta(H) + 2$ in place of $\Delta(H)$
we get as i wrote in the previous emails.   Maybe this helps with the ceiling.

We can use $|H|$ odd to get a bit better on the ceiling in what you wrote since the top is even (divide both by two before doing ceiling approximation).

Thinking about your comment that we can assume H is critical, we can,  but not how i was setting it up.   Probably you are already thinking something like this:

Assume Goldberg.   Take minimum counterexample to 5/6 conjecture, say 
$G = L(H)$.   The $H$ is critical.  From the argument like in strengthening of
Brooks,  we get $\chi(G) \ge \Delta(H) + 2$.  By Goldberg this implies

\begin{align*}
\chi(G) = \max_{Q \subseteq H \mbox{ s.t.~$|Q|\ge 3$ and odd
}}\ceil{\frac{2||Q||}{|Q| - 1}}
\end{align*}

If the max is achieved at a proper subgraph of $H$, then there is an edge we
can remove without decreasing the max, but this decreases the chromatic number
by criticality and the max is a lower bound, so impossible. Therefore, $|H|$ is odd and

\begin{align*}
\chi(G)  =  \ceil{\frac{2||H||}{|H| - 1}}
\end{align*}

so,

\begin{align*}
\ceil{\frac{2||H||}{|H| - 1}} \ge \Delta(H) + 2
\end{align*}

\begin{align*}
2||H|| / (|H| - 1) \ge \Delta(H) + 1
\end{align*}
using 
\begin{align*}
\Delta(H)|H| \ge 2||H||,
\end{align*}

using $\Delta(H)|H| \ge 2||H||$, I get

\begin{align*}
\Delta(H) \ge |H| - 1,
\end{align*}
\end{proof}

%About  the partial results.   Supposing we have the bound on w(H) to be good enough for Golberg (or at least the partial result we want to use),
%
%partial results are like
%
%\chi'(H) \le max(\Delta(H) + sqrt(\Delta(H) / 2),  ceiling(w(H)))
%
%so we can use s = sqrt(\Delta(H) / 2) in the more general bound we get from strengthening brooks (first email).


\bigskip
\bigskip

I think we should be able to prove that the conjecture follows from
Goldberg--Seymour.
That lemma you proved is pretty useful.  We can assume that H is
critical, which implies that $|N(x)| \ge 2$ for all $x$ in $H$.  Now let $J$
be the simple graph underlying $H$.  We know that $\delta(J) \ge 2$.  Let
$B = \{ x \in H s.t. d_J(x) \ge 3\}$.  That lemma implies that $|B| \le 4$.
Further, if $|B| = 4$, then each vertex of $B$ has degree 3 in $J$.  If
$|B|=3$, then two vertices of $B$ have degree 3 in $J$ and one has degree 4
in $J$.  Otherwise $|B| \le 2$.  Now if $J$ has a vertex $x$ of degree at
least 5, and $|B|=2$, then the other vertex in $B$ has degree 3 in $J$.  Now
$x$ must be a cut-vertex (since $J$ is formed by identifying one vertex in
multiple disjoint cycles, exactly one of which has a chord).  But a
cut-vertex in $J$ is also a cut-vertex in $H$, which is a contradiction.
Thus, we only need consider the cases when $|B|=3$ and $|B|=4$, which have
degree sequences $3,3,3,3,2,\ldots2$. and $4,3,3,2,\ldots,2$.
$|B|=4$ is a subdivided $K_4$ or a subdivision of a 4-cycle where one
matching has multiplicity 2.
$|B|=3$ is a subdivision of a triangulated 5-cycle.  I haven't worked
out those cases, but I don't think they should be too hard.
\newpage

\begin{lem}
\label{lem-A}
Suppose $G=L(H)$ and $G$ is a minimal counterexample to the
$\frac56$-Conjecture.  Let $k = \frac56(\Delta(G)+1)$.  If $T$ is a Tashkinov
tree w.r.t.~a $k$-edge-coloring $\vph$ of $H-e$, then
\begin{align*}
\sum_{v\in V(T)}d_H(v)(5d_T(v)-6) &\le -12 + 5\sum_{e\in E(T)}\mu_H(e)
\end{align*}
\end{lem}
\begin{proof}
Since $T$ is elementary, the sets of colors missing at vertices of $T$ are
disjoint, so $2+\sum_{v\in V(T)}(k-d_H(v))\le k$.  Rewriting this gives
$k(|V(T)|-1)\le -2 + \sum_{v\in V(T)}d_H(v)$.  
For each edge $xy\in E(T)$, we
have $k = \frac56(\Delta(G)+1)\ge \frac56(d_H(x)+d_H(y)-\mu_H(xy))$.
Summing over all $|T|-1$ edges gives
\begin{align*}
-2 + \sum_{v\in V(T)} d_H(v) &\ge k(|V(T)|-1) \\
&\ge \frac56(\Delta(G)+1)(|T|-1) \\
&\ge \frac56 \sum_{uv\in E(T)}(d_H(u)+d_H(v)-\mu_H(uv)) \\
& = \frac56\sum_{v\in V(T)}d_H(v)d_T(v)-\frac56\sum_{uv\in E(T)}\mu_H(uv)
\end{align*}

To prove the lemma, we take the first and last expressions in the inequality
chain, multiply by 6, then rearrange terms.
\end{proof}

\begin{cor}
If $G=L(H)$ and $G$ is a minimal counterexample to the
$\frac56$-Conjecture, then each $x\in V(H)$ is special if $d_H(x) >
\frac34\Delta(H)-3$.  
\end{cor}
\begin{proof}
Suppose that $x$ is a non-special vertex.  Choose $e$ incident to $x$ and a
$k$-edge-coloring $\vph$ of $G-e$ such that there exists a Vizing fan $T$
rooted at $x$ with $|T|\ge 4$.  Since every edge in $F$ is incident to $x$, we
have $\sum_{e\in E(T)}\mu_H(e)\le d_H(x)$.
From Lemma~\ref{lem-A}, we have 
\begin{align*}
-12+5d_H(x) &\ge -12+5\sum_{e\in E(T)}\mu_H(e) \\
& \ge \sum_{v\in T}(5d_T(v)-6)d_H(v) \\
& \ge (5d_T(x)-6)d_H(x)+\sum_{v\in T-x}(5d_T(v)-6)d_H(v) \\
& = (5(|T|-1)-6)d_H(x)-\sum_{v\in V(T-x)}d_H(v),
\end{align*}
where the final equality holds because each vertex $v\in T-x$ is a leaf.
Now rearranging terms gives
\begin{align*}
-12 + \sum_{v\in V(T-x)}d_H(v) & \ge (5(|T|-1)-11)d_H(x)\\
-12 + (|T|-1)\Delta(H) & \ge (5(|T|-16)d_H(x)\\
d_H(x) & \le \frac{-12+(|T|-1)\Delta(H)}{5|T|-16}\\
d_H(x) & \le \frac{-12+3\Delta(H)}{4} = \frac34\Delta(H)-3,
\end{align*}
where the final inequality holds because $|T|\ge 4$ and the right side decreases
as a function of $|T|$.
\end{proof}

\end{document}

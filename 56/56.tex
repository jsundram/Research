\documentclass[12pt]{amsart}
\usepackage{amsmath, amsthm, amssymb}
\usepackage[top=1.25in, bottom=1.25in, left=1.0in, right=1.0in]{geometry}
\usepackage{hyperref}
\usepackage{color}
\usepackage{verbatim}
\usepackage{tikz,tkz-graph}

\makeatletter
\newtheorem*{rep@theorem}{\rep@title}
\newcommand{\newreptheorem}[2]{
\newenvironment{rep#1}[1]{
 \def\rep@title{#2 \ref{##1}}
 \begin{rep@theorem}}
 {\end{rep@theorem}}}
\makeatother

\theoremstyle{plain}
\newtheorem{thm}{Theorem}
\newreptheorem{thm}{Theorem}
\newtheorem{prop}[thm]{Proposition}
\newreptheorem{prop}{Proposition}
\newtheorem{lem}[thm]{Lemma}
\newreptheorem{lem}{Lemma}
\newtheorem{lemma}[thm]{Lemma}
\newreptheorem{lemma}{Lemma}
\newtheorem{conj}[thm]{Conjecture}
\newreptheorem{conj}{Conjecture}
\newtheorem{cor}[thm]{Corollary}
\newreptheorem{cor}{Corollary}
\newtheorem{prob}[thm]{Problem}
\theoremstyle{definition}
\newtheorem{defn}{Definition}
\newtheorem{clm}{Claim}
\newtheorem{obs}[thm]{Observation}
\theoremstyle{remark}
\newtheorem*{remark}{Remark}
\newtheorem{example}{Example}
\newtheorem*{question}{Question}

\newcommand{\fancy}[1]{\mathcal{#1}}
%\newcommand{\C}[1]{\fancy{C}_{#1}}
\newcommand{\C}{\fancy{C}}
\newcommand{\F}{\fancy{F}}
\newcommand{\W}{\fancy{W}}
\newcommand{\IN}{\mathbb{N}}
\newcommand{\IR}{\mathbb{R}}
\newcommand{\G}{\fancy{G}}
\newcommand{\LB}{\mathcal{L}_B}
\newcommand{\col}{{\textrm{col}}}
\newcommand{\ch}{{\textrm{ch}}}
\newcommand{\chil}{{\chi_{\ell}}}
\newcommand{\chiol}{{\chi_{OL}}}

\newcommand{\inj}{\hookrightarrow}
\newcommand{\surj}{\twoheadrightarrow}

\newcommand{\set}[1]{\left\{ #1 \right\}}
\newcommand{\setb}[3]{\left\{ #1 \in #2 : #3 \right\}}
\newcommand{\setbs}[2]{\left\{ #1 : #2 \right\}}
\newcommand{\card}[1]{\left|#1\right|}
\newcommand{\size}[1]{\left\Vert#1\right\Vert}
\newcommand{\ceil}[1]{\left\lceil#1\right\rceil}
\newcommand{\floor}[1]{\left\lfloor#1\right\rfloor}
\newcommand{\func}[3]{#1\colon #2 \rightarrow #3}
\newcommand{\funcinj}[3]{#1\colon #2 \inj #3}
\newcommand{\funcsurj}[3]{#1\colon #2 \surj #3}
\newcommand{\irange}[1]{\left[#1\right]}
\newcommand{\join}[2]{#1 \mbox{\hspace{2 pt}$\ast$\hspace{2 pt}} #2}
\newcommand{\djunion}[2]{#1 \mbox{\hspace{2 pt}$+$\hspace{2 pt}} #2}
\newcommand{\parens}[1]{\left( #1 \right)}
\newcommand{\brackets}[1]{\left[ #1 \right]}
\newcommand{\DefinedAs}{\mathrel{\mathop:}=}

\newcommand{\mic}{\operatorname{mic}}
\newcommand{\AT}{\operatorname{AT}}
\renewcommand{\col}{\operatorname{col}}
\renewcommand{\ch}{\operatorname{ch}}
\newcommand{\type}{\operatorname{type}}
\newcommand{\nonsep}{\bar{S}}
\newcommand{\dclaw}[1]{d_{\text{claw}}\left( #1 \right)}

\def\adj{\leftrightarrow}
\def\nonadj{\not\!\leftrightarrow}

\newcommand{\vph}{\varphi}
\newcommand{\vphn}{\overline{\varphi}}

\newcommand{\claim}[2]{{\noindent\bf Claim #1.}~{\it #2}~~}

\begin{document}
\section{Overview}

The biggest open problem in edge-coloring is the Goldberg--Seymour conjecture.
Over the past two decades, the main tool for attacking this problem has become
Tashkinov trees, a vast generalization of Vizing fans and Kierstead paths.
The second author proved that if $G$ is a line graph, then $\chi(G)\le
\max\{\omega(G),\frac{7\Delta(G)+10}{8}\}$.  In the same paper, he conjectured
that $\chi(G)\le \max\{\omega(G),\frac{5\Delta(G)+8}{6}\}$, which is best possible.
We call the latter inequality the $\frac56$-Conjecture, and in this paper we prove
it.  Along the way, we develop more general techniques and results that will
likely be of independent interest, due to their use in approaching the
Goldberg--Seymour conjecture.

A graph $G$ is \emph{elementary} if $\chi'(G)=\W(G)$; such graphs satisfy the
Goldberg--Seymour Conjecture.  (We begin by proving that every minimal
counterexample to the $\frac56$-Conjecture is elementary.  In Section 3, we conclude
by also proving that every elementary graph satisfies the $\frac56$-conjecture.)  A
\emph{defective color}
for a Tashkinov tree $T$ is a color used on more than one edge from $V(T)$ to
$V(G)-V(T)$; a Tashkinov tree is \emph{strongly closed} if it has no defective
color.  Andersen~[] and Goldberg~[] showed that if $G$ is critical, then $G$ is
elementary if there exists $e\in E(G)$ and $X\subseteq V(G)$ and a
$k$-edge-coloring $\vph$ of $G-e$ such that $X$ contains the endpoints of $e$
and $X$ is elementary and strongly closed w.r.t.~$\vph$.  Thus, to show that $G$
is elementary, it suffices to show that if $G$ is $(k+1)$-critical, then there
exists an edge $e\in E(G)$ and a $k$-coloring $\vph$ of $G-e$ such that some
maximal Tashkinov tree containing $e$ is strongly closed.  The following
definition is useful.  
A vertex $v \in V(G)$ is \emph{special} if every Vizing fan rooted at $v$ (taken over all
$k$-colorings of $G-e$, over all edges $e$ incident to $v$) has at most 3 vertices,
including $v$.
As a warmup, in Section 2 we prove that if $\chi'(G)\ge \Delta(G)+2$ and every
vertex of $G$ is special, then $G$ is elementary, i.e., $\chi'(G)=\W(G)$.  Next,
we push our methods further, allowing our maximal Tashkinov tree to have at most
3 non-special vertices.

In Section 3, we show that if $G$ is a minimal counterexample to the
$\frac56$-Conjecture, then every non-special vertex $v$ has
$d_G(v)<\frac34\Delta(G)$.  Since every maximal Tashkinov tree $T$ is
elementary, and every non-special vertex misses more than $k/4$ colors, we
conclude that $T$ has at most 3 non-special vertices.  Thus our results from
Section 2 apply.  As a consequence, every minimal countexample to the
$\frac56$-Conjecture is elementary.  To complete the proof of the $\frac56$-conjecture,
we prove that it follows from the Goldberg--Seymour Conjecture.  More precisely,
we show for each graph $G$ that if $\chi'(G)=\W(G)$, then
$\chi'(G)\le\max\{\omega(G),\frac{5\Delta(G)+8}6\}$.

%Let $G$ be a graph with $\chi'(G)=k+1$, for some $k\ge \Delta(G)$.  Choose an
%edge $e$ and a $k$-coloring $\vph$ of $G-e$.

\section{Useful lemmas}
%Let $G$ be $(k+1)$-edge-critical for some $k \ge \Delta(G) + 1$.  

Throughout this section, let $G$ be a $(k+1)$-edge-critical multigraph for some
$k \ge \Delta(G) + 1$.  We use the following notation.  Let $\varphi$ be a partial
$k$-edge-coloring of $G$.  For each vertex $v\in V(G)$, let $\varphi(v)$ be the
set of colors used in $\vph$ on edges incident to $v$ and let
$\vphn(v)=[k]\setminus \vph(v)$. For an uncolored edge $e_0$, a \emph{Tashkinov
tree} is a sequence $v_0, e_1, v_1, e_2,\ldots, v_{t-1},e_t,v_t$ such that all
$v_i$ are distinct and $e_i=v_jv_i$ for some $j$ and $\ell$ with $0\le j< i$
and $0\le \ell < i$ such that $\vph(e_i)\in \vphn(v_\ell)$.

\setcounter{thm}{-1}
\begin{lem}\label{SpecialPath}
Let $\vph$ be a $k$-edge-coloring of $G-v_0v_1$.  Suppose $\alpha \in \vphn(v_0)$ and $\beta \in \vphn(v_1)$.  
Let $P = v_1v_2\cdots v_r$ be an $\alpha-\beta$ path
with edges $e_i = v_iv_{i+1}$ for $1 \le i \le r-1$.  If $v_i$ is special for
all odd $i$, then for any $\tau \in \vphn(v_0)$ there are edges $f_i =
v_iv_{i+1}$ for $1 \le i \le r-1$ such that $f_i = e_i$ for $i$ even and
$\vph(f_i) = \tau$ for $i$ odd.
\end{lem}
\begin{proof}
Suppose not and choose a counterexample minimizing $r$.  By minimality of
$r$, we have $\vph(v_{r-1}v_r) = \alpha$ and we have $f_i = v_iv_{i+1}$ for
$1 \le i \le r-2$ such that $f_i = e_i$ for $i$ even and $\vph(f_i) = \tau$ for
$i$ odd.  Swap $\alpha$ and $\beta$ on $e_i$ for $1 \le i \le r-3$ and then
color $v_0v_1$ (call this edge $e_0$) with $\alpha$ and uncolor $e_{r-2}$.  Let
$\vph'$ be the resulting coloring.  Since $k \ge \Delta(G) + 1$, some color
other than $\alpha$ is missing at $v_{r-2}$; let $\gamma$ be such a color.  Now 
$v_{r-1}$ is special since $r-1$ is odd (since $P$ starts and ends with
$\alpha$), so there is an edge $e = v_{r-1}v_r$ with $\vph'(e) = \gamma$.  
Swap $\tau$ and $\alpha$ on $e_i$ for $0 \le i \le r-3$ to get a new coloring
$\vph^*$.  Now $\gamma$ and $\tau$ are both missing at $v_{r-2}$ in $\vph^*$.
Since $v_{r-1}$ is special, the fan with $v_{r-2}, v_{r-1}, v_r$ and $e$
implies that there is an edge $f_{r-1} = v_{r-1}v_r$ with $\vph^*(f_{r-1}) =
\tau$.  
But we have never recolored $f_{r-1}$, so $\vph(f_{r-1})=\tau$, a
contradiction.
%Now swap $\alpha$ and $\tau$ back on $e_i$ for $0 \le i \le r-3$ and
%then shift the $\alpha-\beta$ coloring one to the right to get back to $\vph$. 
%We have all the desired $f_i$, a contradiction.
\end{proof}

Recall that for a $k$-edge-coloring $\vph$ of $G-e$, a maximal Tashkinov tree
starting with $e$ may not be unique.  However, if $T_1$ and $T_2$ are both such
trees, then it is easy to show that $V(T_1)\subseteq V(T_2)$; by symmetry, also
$V(T_2)\subseteq V(T_1)$, so $V(T_1)=V(T_2)$.

\begin{lem}
\label{lem1}
Let $T$ be a maximal Tashkinov tree with respect to a $k$-edge-coloring $\vph$
of $G-xy$.  If at most one $v \in V(T)$ is non-special, then, for all $\alpha \in \vphn(x)$
and $\beta \in \vphn(y)$, the $\alpha-\beta$ path $P$ from $y$
to $x$ has $V(P) = V(T)$.
\end{lem}
\begin{proof}
Let $\vph$ be a $k$-edge-coloring of $G-xy$ and let $T$ be a maximal Tashkinov
tree containing $xy$.  Choose $\alpha \in \vphn(x)$ and $\beta \in \vphn(y)$.
Note that the $\alpha-\beta$ path $P$ starting from $y$ must end at $x$ (or
else we can perform a Kempe swap on $P$ and color edge $xy$ with color $\alpha$).
We show that $P$ is a maximal Tashkinov tree; hence $V(P) = V(T)$.  Say $P =
v_0\cdots v_r$, where $v_0=y$ and $v_r = x$. 
%and $v_r$ is the vertex right before $x$.  
Suppose $P$ is not maximal; so some color $\tau$ is missing on
$P$ and some edge colored $\tau$ leaves $P$.  
%Since at most one vertex of $P$ is non-special, 
Note that $E(P)\cup\{xy\}$ is an odd cycle.
We can relabel $V(P)$ so that $\tau$ is missing at $v_0$, the edge 
colored $\tau$ leaving $V(P)$ is incident to $v_i$ and all vertices $v_j$ with
$0<j<i$ are special.  Further, if $v_i$ is non-special, then we
can choose $i$ to be even (by possibly going around the cycle the other way).
Also, we can recolor $E(P)\cup\{xy\}$ with $\alpha$ and $\beta$ so that only
$v_0v_1$ is uncolored and $\alpha$ and $\beta$ are missing at $v_0$ and $v_1$,
respectively.
%We have a 2-colored cycle, so by symmetry we may assume that $i < j$.  
Now by Lemma \ref{SpecialPath}, %we can walk from $i$ to $j$ showing
each of these edges colored $\alpha$ has a parallel edge colored $\tau$.  
In particular, the edge colored $\tau$ incident to $v_i$ ends in $P$, a
contradiction.
\end{proof}

A \emph{defective color} for a
Tashkinov tree $T$  in a critical graph $G$ is a color used on more than one edge
from $V(T)$ to $V(G) - V(T)$.  Let $t(G)$ be the maximum size of a Tashkinov tree over all $e \in E(G)$
and all $k$-edge-colorings $\vph$ of $G - e$.  If a Tashkinov tree $T$ has $|T| = t(G)$, then $T$ is \emph{maximum}.

\begin{lem}\label{AtMostOneNonSpecial}
Let $T$ be a Tashkinov tree with respect to a $k$-edge-coloring $\vph$ of $G - v_0v_1$.  If $T$ is maximum and at most one $v \in V(T)$ is
non-special, then $V(T)$ has no defective colors.
\label{lem2}
\end{lem}
\begin{proof}
Use Lemma~\ref{lem1} to get an $\alpha-\beta$ path $P = v_1\ldots v_rv_0$ with
$V(P) = V(T)$; recall that $P$ is a maximum size Tashkinov tree.  Suppose that
$P$ has a defective color $\delta$ with respect to $\vph$. Let $\tau$ be
missing at $v_2$. Consider a maximal $\tau-\delta$ path $Q$, starting at $v_2$.  
Since $P$ is maximal, $\delta$ is not missing at any vertex of $P$;
since $V(P)$ is elementary, $\tau$ is not missing at any vertex of $P$ other 
than $v_2$.  As a result, $Q$ ends outside $V(P)$.  Now $Q$ could leave
$V(T)$ and re-enter it repeatedly, %and bounce around inside a bunch,
%(in fact $Q$ must contain
%every $\delta$-colored edge leaving $V(T)$, but we don't need that), 
but $Q$ ends outside $V(P)$, so there is a last vertex $w \in V(Q) \cap V(P)$;
%(this is what the Stiebitz book calls an "exit vertex"), 
say $Q$ ends at $z \in V(G) - V(P)$.  Let $\pi \notin \{\alpha, \beta\}$ be a
color missing at $w$.  Since
$P$ is maximal, no edge colored $\tau$ or $\pi$ leaves $V(P)$.  So, we can
swap $\tau$ and $\pi$ on every edge in $G - V(P)$ without changing the fact that
$P$ is a maximum size Tashkinov tree.  Now swap $\delta$ and $\pi$ on the
subpath of $Q$ from $w$ to $z$;
since $\pi$ is missing at $w$, the $\delta-\pi$ path does end at $w$.  Now
$\delta$ is missing at $w$, but $\delta$ was defective in $\vph$, so there are
still edges colored $\delta$ leaving $V(P)$, adding such an edge gets a larger
Tashkinov tree, a contradiction.
\end{proof}

\begin{thm}\label{AllSpecialImpliesElementary}
Let $G$ be a multigraph.
If at most one $v \in V(G)$ is non-special, then
$\chi'(G) \le \max \set{\ceil{\chi'_f(G)}, \Delta(G) + 1}$.
\end{thm}
\begin{proof}
We may assume that $G$ is critical.
If $\chi'(G)\le \Delta(G)+1$, then we are done.
So instead, assume that $\chi'(G)\ge \Delta(G)+2$ and let $k=\chi'(G)-1$.
Choose $e\in E(G)$.  Now by Lemma~\ref{lem2}, $G-e$ has a $k$-edge-coloring
$\vph$ and a maximal Tashkinov tree $T$ such that $V(T)$ has no defective color, i.e.,
$V(T)$ is strongly closed.  By a Theorem of Andersen and Goldberg (see also
[Stiebitz, p.~8--9]), this implies that $\chi'(G)=\ceil{\chi'_f(G)}$.
\end{proof}

\section{The easy bound}
Let $G$ be a multigraph.  The \emph{claw-degree} of $x \in V(G)$ is 
\[\dclaw{x} \DefinedAs \max_{\substack{S \subseteq N(x) \\ \card{S} = 3}}\frac14 \parens{d(x) + \sum_{v \in S} d(v)}.\]
The \emph{claw-degree} of $G$ is 
\[\dclaw{G} \DefinedAs \max_{x \in V(G)} \dclaw{x}.\]
\begin{thm}\label{EasyBound}
If $G$ is a multigraph, then
\[\chi'(G) \le \max\set{\ceil{\chi'_f(G)}, \Delta(G) + 1, \ceil{\frac43\dclaw{G}}}.\]
\end{thm}
\begin{proof}
Suppose not and choose a counterexample $G$ minimizing $\size{G}$; note that $G$ is
edge-critical. Let $k=\chi'(G)-1$, so $k \ge \ceil{\frac43\dclaw{G}}$. 
By Theorem \ref{AllSpecialImpliesElementary}, $G$ has a non-special vertex $x$.
Choose $xy_1 \in E(G)$ and a $k$-edge-coloring $\vph$ of $G - xy_1$ such that
$\vph$ has a fan $F$ of length $3$ rooted at $x$ with leaves $y_1, y_2, y_3$.  
Since $V(F)$ is elementary, 
\[2 + k - d(x) + \sum_{i \in \irange{3}} k-d(y_i) \le k,\]
and hence
\[\dclaw{x} \ge \frac14\parens{d(x) + \sum_{i \in \irange{3}} d(y_i)} \ge \frac{3k+2}{4}.\]
This gives the contradiction
\[\ceil{\frac43\dclaw{G}} \le k \le \frac43\dclaw{G} - \frac23.\]
\end{proof}

\section{A stronger bound}
An edge $x_1x_2 \in E(G)$ is \emph{special} if for at least one $i \in \irange{2}$, every Vizing fan containing $x_1x_2$, rooted at $x_i$ (taken over all $k$-colorings of $G-x_1x_2$), has at most 3 vertices.
 First, some helper lemmas.

\begin{lem}\label{HelperOne}
Let $\vph$ be a $k$-edge-coloring of $G-v_0v_1$.  Suppose $\alpha \in \vphn(v_0)$ and $\beta \in \vphn(v_1)$.  
Let $P = v_1v_2\cdots v_r$ be an $\alpha-\beta$ path.  Let $i,j \in \irange{r}$ with $i + 3 \le j$ such that $j-i$ is odd.  If $v_t$ is special for all $i < t < j$, then
$\mu(G) \ge 2k - d(v_i) - d(v_j)$.
\end{lem}

\begin{cor}\label{HelperTwo}
Let $\vph$ be a $k$-edge-coloring of $G-v_0v_1$.  Suppose $\alpha \in \vphn(v_0)$ and $\beta \in \vphn(v_1)$.  
Let $P = v_1v_2\cdots v_r$ be an $\alpha-\beta$ path.  Let $i,a,j \in \irange{r}$ with $i + 2 \le a \le j - 2$ such that $j-i$ is odd.  If $v_t$ is special for all $i < t < a$ and $a < t < j$, then
$\mu(G) \ge 2k - d(v_i) - d(v_a)$ or $\mu(G) \ge 2k - d(v_a) - d(v_j)$.
\end{cor}

\begin{lem}\label{AtMostThreeNonSpecial}
Let $T$ be a Tashkinov tree with respect to a $k$-edge-coloring
$\vph$ of $G - v_0v_1$ in $G$ such that $|T| = t(G)$.  If all but at most three $v \in V(T)$ is special, then 
\begin{enumerate}
\item $V(T)$ has no defective colors; or
\item there are non-special vertices $x_1, x_2 \in V(T)$ such that $\mu(G) \ge 2k - d(x_1) - d(x_2)$; or
\item $E(T)$ contains a non-special edge and $T$ has three non-special vertices.
\end{enumerate}
\end{lem}
\begin{proof}
Choose $\alpha \in \vphn(v_0)$ and $\beta \in \vphn(v_1)$ so that the length of the $\alpha-\beta$ path $P = v_1v_2\cdots v_rv_0$ is maximized.  Suppose both (2) and (3) do not hold.  To prove (1), it will
suffice to show that $P$ is a maximal Tashkinov tree.  Suppose $P$ is not maximal. By Lemma \ref{AtMostOneNonSpecial} there must be non-special vertices $v_i, v_j$, where $i < j$.  Without loss of generality, 
there is $\tau \in \vphn(v_0)$ a $\tau$-colored edge leaving $P$ from $v_b$.  Then, by Lemma \ref{SpecialPath}, $i$ is odd, $j$ is even and $i \le b \le j$.  By shifting the coloring, we may assume $i = 1$.

Suppose $v_t$ is special for $1 < t < j$.  Then, since (2) does not hold, Lemma \ref{HelperOne} implies $j = 2$.  By symmetry, we may assume $b = 1$.  Consider the $\tau-\beta$ path $Q$ starting at $v_1$.  
Plainly, $Q$ must end at $v_0$. Since $P$ is an $\alpha-\beta$ path, $Q$ must re-enter $P$ along a $\tau$-edge.
But we just showed that $\tau$ edges can only leave at $v_1$ or $v_2$.  So, the $\tau-\beta$ path re-enters $P$ at $v_2$.  But then we replaced a single edge $v_1v_2$ with a path of length at least three, 
so the $\tau-\beta$ path is longer than the $\alpha-\beta$ path, contradicting our maximality condition on $P$.

Hence there is $1 < a < j$ such that $v_a$ is non-special.  Since (2) does not hold and we are assuming there are at most three non-special vertices, Lemma \ref{HelperTwo} implies that $a = 2$ or $a = j-1$.  
By symmetry, we may assume $a=2$.  Uncolor $v_1v_2$ and color $v_0v_1$ with $\alpha$.  Let $F_i$ be a maximal fan rooted at $v_i$ for $i \in \irange{2}$.  If $|F_i| \le 3$, then all the colors in $\vphn(v_{3-i)$
must appear on edges between $v_i$ and $v_{3(i-1)}$.  But then (2) is violated.  Hence $|F_i| \ge 4$ for $i \in \irange{2}$.  So, $v_1v_2$ is a non-special edge, violating (3).
\end{proof}

\begin{thm}[from strengthening Brooks paper]\label{CriticalMuBound}
If $Q$ is the line graph of a multigraph $G$ and $Q$ is vertex critical, then
\[\chi(Q) \leq \max\left\{\omega(Q), \Delta(Q) + 1 - \frac{\mu(G) - 1}{2}\right\}.\]
\end{thm}

\begin{thm}
If $Q$ is the line graph of a multigraph, then
\[\chi(Q) \le \max\set{\ceil{\chi_f(Q)}, \ceil{\frac{5\Delta(Q) + 3}{6}}}.\]
\end{thm}
\begin{proof}
Suppose the theorem is false and choose a counterexample minimizing $\card{Q}$.  Put $k = \max\set{\ceil{\chi_f(Q)}, \ceil{\frac{5\Delta(Q) + 3}{6}}}}$. Say $Q = L(G)$ for a multigraph $G$. Then $G$ is $k$-edge-critical.

\claim{0}{Let $F$ be a fan rooted at $x$ with respect to a $k$-edge-coloring of $G - xy$.  If $|F| = 4$, then
\[d(x) \le \frac{-6 + \sum_{v \in V(F-x)} d(v)}{4}.\]}
Since $F$ is elementary, we have
\[2 + k-d(x) + \sum_{v \in V(F-x)} k - d(v) \le k,\]
so
\[2 + (|F| - 1)k \le d(x) + \sum_{v \in V(F-x)} d(v).\]
Using $k \ge \frac56(\Delta(Q) + 1) - \frac13 \ge \frac56(d(x) + d(v) - \mu(xv)) - \frac13$ for each $v \in V(F-x)$, we get
\[2 + \sum_{v \in V(F-x)}\parens{\frac56(d(x) + d(v) - \mu(xv)) - \frac13} \le d(x) + \sum_{v \in V(F-x)} d(v),\]
so
\[12 + \parens{5|F| - 11}d(x) - 2(|F| - 1) \le 5\sum_{v \in V(F-x)} \mu(xv) + \sum_{v \in V(F-x)} d(v).\]
Now $\sum_{v \in V(F-x)} \mu(xv) \le d(x)$, so this becomes
\[12 + \parens{5|F| - 16}d(x) -2(|F| - 1) \le \sum_{v \in V(F-x)} d(v).\]
Using $|F| = 4$ gives
\[d(x) \le \frac{-6 + \sum_{v \in V(F-x)} d(v)}{4}.\]

\claim{1}{If $x \in V(G)$ with $d(x) > \frac34 \Delta(G) - \frac32$, then $x$ is special.}

Immediate from Claim 0.

\claim{2}{If $x_1x_2 \in E(G)$ with $d(x_i) > \frac23 \Delta(G) - 2$ for at least one $i \in \irange{2}$, then $x_1x_2$ is special.}
Suppose $x_1x_2$ is not special.  Then by Claim 0, we have for $i \in \irange{2}$,
\[d(x_i) \le \frac{-6 + \sum_{v \in V(F-x_i)} d(v)}{4} \le \frac{-6 + d(x_{3-i}) + 2\Delta(G)}{4},\]
so
\[4d(x_i) - d(x_{3-i}) \le 2\Delta(G) - 6.\]
Solving the system gives for $i \in \irange{2}$,
\[d(x_i) \le \frac23 \Delta(G) - 2.\]

\claim{3}{The theorem is true.}

Let $T$ be a Tashkinov tree with respect to a $k$-edge-coloring $\vph$ of $G - v_0v_1$ in $G$ such that $|T| = t(G)$.  By Lemma \ref{AtMostThreeNonSpecial} one of the following holds:
\begin{enumerate}
\item $V(T)$ has no defective colors; or
\item there are non-special vertices $x_1, x_2 \in V(T)$ such that $\mu(G) \ge 2k - d(x_1) - d(x_2)$; or
\item $E(T)$ contains a non-special edge $x_1x_2$ and $T$ has three non-special vertices; or
\item $V(T)$ contains four non-special vertices $x_1, x_2, x_3, x_4$.
\end{enumerate}

If (1) holds, then $T$ is strongly closed and hence $\ceil{\chi_f(Q)} < k+1 \le \max\set{\ceil{\chi_f(Q)}, \Delta(G) + 1}$ which is impossible.

If (2) holds, then by Claim 1 we have $\mu(G) \ge 2k + 3 - \frac32\Delta(G) \ge \frac{k}{2} + 3$.  Hence Theorem \ref{CriticalMuBound} gives
\[k + 1 \le \Delta(Q) + 1 - \frac{k + 1}{4},\]
so
\[\ceil{\frac{5\Delta(Q) + 3}{6}}} < k + 1 \le \frac45(\Delta(Q) + 1),\]
a contradiction.

Suppose (3) holds.  Let $x_3$ be a special vertex in $V(T) \setminus \set{x_1,x_2}$. Then
\[2 + \sum_{i \in \irange{3}} k - d(x_i) \le k,\]
using Claim 1 and Claim 2, this gives
\[2(\frac23\Delta(G) - 2) + \frac34\Delta(G) - \frac32 \ge 2k+2,\]
so $\Delta(G) > k$, a contradiction.

So, (4) must hold.  But then
\[2 + \sum_{i \in \irange{4}} k - d(x_i) \le k,\]
using Claim 1 gives
\[4(\frac34\Delta(G) - \frac32) \ge 3k+2,\]
so $\Delta(G) > k$, a contradiction.
\end{proof}

\section{Complicated generalization}
The \emph{claw-degree} of $x_1x_2 \in E(G)$ is 
\[\dclaw{x_1x_2} \DefinedAs \frac14 \min_{i \in \irange{2}}\max_{\substack{S \subseteq N(x_i) \\ \card{S} = 3 \\ S \ni x_{3-i}}}\parens{d(x_i) + \sum_{v \in S} d(v)}.\]

For $q \in \IN$, put $G_q \DefinedAs \setb{v}{V(G)}{d(v) \ge q}$.  Put
\[d_q(G) \DefinedAs \max_{x \in G_q} \dclaw{x},\]
and
\[d^e_q(G) \DefinedAs \max_{\substack{xy \in E(G) \\ \set{x,y} \cap G_q \ne \emptyset}} \dclaw{xy},\]


\begin{thm}\label{StrongerBound}
If $G$ is a multigraph and $q \in \IN$, then 
\[\chi'(G) \le \max\set{\ceil{\chi'_f(G)}, \Delta(G) + 1, \ceil{\frac43d_q(G)}, \ceil{\frac43d^e_{\ceil{\frac23 q}}(G)}, \ceil{\frac43 q}, q + \ceil{\frac{\mu(G)}{2}}}.\]
\end{thm}
\begin{proof}
Let $G$ be a minimal counterexample.  Then $G$ is edge-critical.   Let $T$ be a maximum size Tashkinov tree.  We are good if $T$ has one or fewer non-special vertices.  Let $x$ be a non-special vertex in $T$.
As in the proof of Theorem \ref{EasyBound}, we get $\dclaw{x} \ge \frac{3k+2}{4}$.  Hence if any vertex in $G_q$ is non-special we are done as in Theorem \ref{EasyBound}.  Hence every non-special vertex $x$ has $d(x) \le q-1$.
If $T$ has four or more non-special vertices, then $4(q-1) \ge 3k + 2$ .  But then $k > \ceil{\frac43 q} - 1 \ge k + 1$, a contradiction.  
Hence $T$ has two or three non-special vertices.  If Lemma \ref{AtMostThreeNonSpecial}(2) holds, we have $\mu(G) \ge 2k - 2(q-1)$ which gives $q \ge k + 1 - \frac12 \mu(G)$.  
Hence $k > q + \frac12 \mu(G) - 1 \ge k$, a contradiction.  So, by Lemma \ref{AtMostThreeNonSpecial}(3), $E(T)$ contains a non-special edge $x_1x_2$ there is a non-special $x_3 \in V(T) \setminus \set{x_1,x_2}$.

If $\set{x_1, x_2} \cap G_{\ceil{\frac23 q}} = \emptyset$, then
\[2 + k - d(x_1) + k - d(x_2) + k-d(x_3) \le k,\]
so
\[\ceil{\frac23 q} - 1 + 2(q - 1) \le 2k + 2,\]
so $k >  \ceil{\frac43 q} - 1 \ge k + 1$, a contradiction.  

Hence $\set{x_1, x_2} \cap G_{\ceil{\frac23 q}} \ne \emptyset$.  That implies $d^e_{\ceil{\frac23 q}}(G) \ge \dclaw{x_1x_2}$.  Since $x_1x_2$ is non-special, we have for $i \in \irange{2}$
a fan $F_i$ rooted at $x_i$ such that $x_{3-i} \in V(F_i)$ and $|F_i| = 4$.  Hence
\[2 + k - d(x_i) + k - d(x_{3-i}) + \sum_{v \in F_i \setminus \set{x_i, x_{3-i}}} d(v) \le k,\]
so,
\[d(x_i) + d(x_{3-i}) + \sum_{v \in F_i \setminus \set{x_i, x_{3-i}}} d(v) \ge 3k + 2,\]
so,
$d^e_{\ceil{\frac23 q}}(G) \ge \dclaw{x_1x_2} \ge \frac34 k + \frac12,\]
a contradiction.
\end{proof}

\begin{cor}
If $Q$ is the line graph of a multigraph, then
\[\chi(Q) \le \max\set{\omega(Q), \frac56\parens{\Delta(Q) + 1} + \frac12}.\]
\end{cor}
\begin{proof}[sketch]
Assume the result holds fractionally.  Suppose the result is false.  Say $Q = L(G)$.  Apply Theorem \ref{StrongerBound} with $q = \frac34 \Delta(G)$.  
Then at least one of the bounds in the max exceed $\frac56\parens{\Delta(Q) + 1} + \frac12}$.  Suppose $\frac43 d_q(G) > \frac56\parens{\Delta(Q) + 1} + \frac12$,  we have
$x \in G_q$ with neighbors $v_1, v_2, v_3$ such that

\[(d(x) + \sum_i d(v_i)) / 3 > \frac56\parens{\Delta(Q) + 1} + \frac12,\]
this is now just our previous computation showing that vertices of degree at least $\frac34\Delta(G)$ are special, so not possible.

Now suppose $q + \frac{\mu(G)}{2} > (\frac56\parens{\Delta(Q) + 1} + \frac12$.
Then $ \frac{\mu(G) - 1}{2} \ge \frac56\parens{\Delta(Q) + 1}  - q$, plugging into brooks strengthening, we get
\[k+1 \le (\Delta(G) + 1) - \frac56\parens{\Delta(Q) + 1}  - q) = \frac16\parens{\Delta(Q) + 1} + q = \frac16\parens{\Delta(Q) + 1} + \frac34\Delta(G).\]
Hence $\Delta(G) > \frac89(\Delta(Q) + 1) > k + 1$, a contradiction.

So, we must have $\frac43d^e_{\ceil{\frac23 q}}(G) > \frac56\parens{\Delta(Q) + 1} + \frac12$. We have $\ceil{\frac23 q} \ge \frac12 \Delta(G)$.  So, there is $x_1x_2 \in E(G)$ where 
$d(x_1) \ge \frac12 \Delta(G)$ (say, we know it for at least one of them) and there is $S_i \subseteq N(x_i)$ with $|S| = 3$ and $x_{3-i} \in S$ such that
\[\parens{\frac43}\parens{\frac14}\parens{d(x_i) + \sum_{v \in S} d(v)} \ge \frac56\parens{\Delta(Q) + 1} + \frac12,\]
so
\[d(x_i) + d(x_{3-i}) + \sum_{v \in S \setminus \set{x_{3 -i}}} d(v)\ge \frac52\parens{\Delta(Q) + 1} + \frac32,\]
\end{proof}
\newpage

\section{The $\frac56$-Conjecture}
\begin{lem}
If $H$ is a connected multigraph and $G = L(H)$, then $\W(H) \le
\max\{\omega(G), 5(\Delta(G) + 1) / 6 + 3/6\}$.
\end{lem}
\begin{proof}
%$M = |N(x)|$,
Let $d = d_H(x)$,  $\Delta = \Delta(H)$, and $h = |H|$. 
Also, let $p = \sum_{v \in N(x)} d_H(v)$ and
%Here's a way that seems a little simpler to me.  
let $t = \Delta h-2||H||$.
Note that $0 < t \le \Delta$.  Also $p \ge Md - t$.  Now summing over $N_H(x)$
gives

\begin{align*}
|N(x)|(\Delta h-t)/(h-1) > 5/6((|N(x)|-1)d + |N(x)|\Delta - t) + |N(x)|/2
\end{align*}

Solving for $|N(x)|$ gives

\begin{align*}
|N(x)| < (5d+5t)/(3+5d+5\Delta-6(\Delta h-t)/(h-1)).
\end{align*}

Since the numerator and denominator are linear in $t$, the right side is
maximized at one end of the interval $1 \le t \le D$.  Letting $t = D$,
gives $|N(x)| < (5d+5\Delta)/(3+5d-\Delta)$, like you had originally.  Letting $t = 1$,
gives $|N(x)| < (5d+5)/(3+5d+5\Delta-6(\Delta h-1)/(h-1))$, which requires a little more
analysis, akin to what you wrote in your most recent email.

Does that look right to you?



I did the analysis a little differently, but I got to the same
conclusion: Substituting $d \ge 4D/5$ gives that if $M \ge 3$, then we
must have $h \le 4$, which implies $h \le 3$, which contradicts $M \ge 3$.

So, I think I believe it.  I also agree there must be an easier way.
One thing that seems a little magical is that when $5/6 - M/(h-1) \ge 0$
all of the h's go away.

$w(H)$ really has a ceiling in its definition, not sure how much that changes
things.   without, it is the fractional chromatic index.

i think we get some gain as well from the $\Delta(H) + 2$ in place of $\Delta(H)$
we get as i wrote in the previous emails.   Maybe this helps with the ceiling.

We can use $|H|$ odd to get a bit better on the ceiling in what you wrote since the top is even (divide both by two before doing ceiling approximation).

Thinking about your comment that we can assume H is critical, we can,  but not how i was setting it up.   Probably you are already thinking something like this:

Assume Goldberg.   Take minimum counterexample to 5/6 conjecture, say 
$G = L(H)$.   The $H$ is critical.  From the argument like in strengthening of
Brooks,  we get $\chi(G) \ge \Delta(H) + 2$.  By Goldberg this implies

\begin{align*}
\chi(G) = \max_{Q \subseteq H \mbox{ s.t.~$|Q|\ge 3$ and odd
}}\ceil{\frac{2||Q||}{|Q| - 1}}
\end{align*}

If the max is achieved at a proper subgraph of $H$, then there is an edge we
can remove without decreasing the max, but this decreases the chromatic number
by criticality and the max is a lower bound, so impossible. Therefore, $|H|$ is odd and

\begin{align*}
\chi(G)  =  \ceil{\frac{2||H||}{|H| - 1}}
\end{align*}

so,

\begin{align*}
\ceil{\frac{2||H||}{|H| - 1}} \ge \Delta(H) + 2
\end{align*}

\begin{align*}
2||H|| / (|H| - 1) \ge \Delta(H) + 1
\end{align*}
using 
\begin{align*}
\Delta(H)|H| \ge 2||H||,
\end{align*}

using $\Delta(H)|H| \ge 2||H||$, I get

\begin{align*}
\Delta(H) \ge |H| - 1,
\end{align*}
\end{proof}

%About  the partial results.   Supposing we have the bound on w(H) to be good enough for Golberg (or at least the partial result we want to use),
%
%partial results are like
%
%\chi'(H) \le max(\Delta(H) + sqrt(\Delta(H) / 2),  ceiling(w(H)))
%
%so we can use s = sqrt(\Delta(H) / 2) in the more general bound we get from strengthening brooks (first email).


\bigskip
\bigskip

I think we should be able to prove that the conjecture follows from
Goldberg--Seymour.
That lemma you proved is pretty useful.  We can assume that H is
critical, which implies that $|N(x)| \ge 2$ for all $x$ in $H$.  Now let $J$
be the simple graph underlying $H$.  We know that $\delta(J) \ge 2$.  Let
$B = \{ x \in H s.t. d_J(x) \ge 3\}$.  That lemma implies that $|B| \le 4$.
Further, if $|B| = 4$, then each vertex of $B$ has degree 3 in $J$.  If
$|B|=3$, then two vertices of $B$ have degree 3 in $J$ and one has degree 4
in $J$.  Otherwise $|B| \le 2$.  Now if $J$ has a vertex $x$ of degree at
least 5, and $|B|=2$, then the other vertex in $B$ has degree 3 in $J$.  Now
$x$ must be a cut-vertex (since $J$ is formed by identifying one vertex in
multiple disjoint cycles, exactly one of which has a chord).  But a
cut-vertex in $J$ is also a cut-vertex in $H$, which is a contradiction.
Thus, we only need consider the cases when $|B|=3$ and $|B|=4$, which have
degree sequences $3,3,3,3,2,\ldots2$. and $4,3,3,2,\ldots,2$.
$|B|=4$ is a subdivided $K_4$ or a subdivision of a 4-cycle where one
matching has multiplicity 2.
$|B|=3$ is a subdivision of a triangulated 5-cycle.  I haven't worked
out those cases, but I don't think they should be too hard.
\newpage

\begin{lem}
\label{lem-A}
Suppose $G=L(H)$ and $G$ is a minimal counterexample to the
$\frac56$-Conjecture.  Let $k = \frac56(\Delta(G)+1)$.  If $T$ is a Tashkinov
tree w.r.t.~a $k$-edge-coloring $\vph$ of $H-e$, then
\begin{align*}
\sum_{v\in V(T)}d_H(v)(5d_T(v)-6) &\le -12 + 5\sum_{e\in E(T)}\mu_H(e)
\end{align*}
\end{lem}
\begin{proof}
Since $T$ is elementary, the sets of colors missing at vertices of $T$ are
disjoint, so $2+\sum_{v\in V(T)}(k-d_H(v))\le k$.  Rewriting this gives
$k(|V(T)|-1)\le -2 + \sum_{v\in V(T)}d_H(v)$.  
For each edge $xy\in E(T)$, we
have $k = \frac56(\Delta(G)+1)\ge \frac56(d_H(x)+d_H(y)-\mu_H(xy))$.
Summing over all $|T|-1$ edges gives
\begin{align*}
-2 + \sum_{v\in V(T)} d_H(v) &\ge k(|V(T)|-1) \\
&\ge \frac56(\Delta(G)+1)(|T|-1) \\
&\ge \frac56 \sum_{uv\in E(T)}(d_H(u)+d_H(v)-\mu_H(uv)) \\
& = \frac56\sum_{v\in V(T)}d_H(v)d_T(v)-\frac56\sum_{uv\in E(T)}\mu_H(uv)
\end{align*}

To prove the lemma, we take the first and last expressions in the inequality
chain, multiply by 6, then rearrange terms.
\end{proof}

\begin{cor}
If $G=L(H)$ and $G$ is a minimal counterexample to the
$\frac56$-Conjecture, then each $x\in V(H)$ is special if $d_H(x) >
\frac34\Delta(H)-3$.  
\end{cor}
\begin{proof}
Suppose that $x$ is a non-special vertex.  Choose $e$ incident to $x$ and a
$k$-edge-coloring $\vph$ of $G-e$ such that there exists a Vizing fan $T$
rooted at $x$ with $|T|\ge 4$.  Since every edge in $F$ is incident to $x$, we
have $\sum_{e\in E(T)}\mu_H(e)\le d_H(x)$.
From Lemma~\ref{lem-A}, we have 
\begin{align*}
-12+5d_H(x) &\ge -12+5\sum_{e\in E(T)}\mu_H(e) \\
& \ge \sum_{v\in T}(5d_T(v)-6)d_H(v) \\
& \ge (5d_T(x)-6)d_H(x)+\sum_{v\in T-x}(5d_T(v)-6)d_H(v) \\
& = (5(|T|-1)-6)d_H(x)-\sum_{v\in V(T-x)}d_H(v),
\end{align*}
where the final equality holds because each vertex $v\in T-x$ is a leaf.
Now rearranging terms gives
\begin{align*}
-12 + \sum_{v\in V(T-x)}d_H(v) & \ge (5(|T|-1)-11)d_H(x)\\
-12 + (|T|-1)\Delta(H) & \ge (5(|T|-16)d_H(x)\\
d_H(x) & \le \frac{-12+(|T|-1)\Delta(H)}{5|T|-16}\\
d_H(x) & \le \frac{-12+3\Delta(H)}{4} = \frac34\Delta(H)-3,
\end{align*}
where the final inequality holds because $|T|\ge 4$ and the right side decreases
as a function of $|T|$.
\end{proof}

\end{document}

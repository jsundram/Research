\documentclass[12pt]{amsart}
\usepackage{amsmath, amsthm, amssymb}
\usepackage[top=1.25in, bottom=1.25in, left=1.0in, right=1.0in]{geometry}
\usepackage{hyperref}
\usepackage{color}
\usepackage{verbatim}
\usepackage{tikz,tkz-graph}

\makeatletter
\newtheorem*{rep@theorem}{\rep@title}
\newcommand{\newreptheorem}[2]{
\newenvironment{rep#1}[1]{
 \def\rep@title{#2 \ref{##1}}
 \begin{rep@theorem}}
 {\end{rep@theorem}}}
\makeatother

\theoremstyle{plain}
\newtheorem{thm}{Theorem}
\newreptheorem{thm}{Theorem}
\newtheorem{prop}[thm]{Proposition}
\newreptheorem{prop}{Proposition}
\newtheorem{lem}[thm]{Lemma}
\newreptheorem{lem}{Lemma}
\newtheorem{lemma}[thm]{Lemma}
\newtheorem*{lemmaA}{Tashkinov's Lemma}
\newtheorem*{PEL}{Parallel Edge Lemma}
\newtheorem*{GS}{Goldberg--Seymour Conjecture}
\newtheorem*{main}{Theorem 16}
\newtheorem*{main2}{Theorem 13}
\newreptheorem{lemma}{Lemma}
\newtheorem{conj}[thm]{Conjecture}
\newreptheorem{conj}{Conjecture}
\newtheorem{cor}[thm]{Corollary}
\newreptheorem{cor}{Corollary}
\newtheorem{prob}[thm]{Problem}
\theoremstyle{definition}
\newtheorem{defn}{Definition}
\newtheorem{clm}{Claim}
\newtheorem{obs}[thm]{Observation}
\theoremstyle{remark}
\newtheorem*{remark}{Remark}
\newtheorem{example}{Example}
\newtheorem*{question}{Question}

\newcommand{\fancy}[1]{\mathcal{#1}}
%\newcommand{\C}[1]{\fancy{C}_{#1}}
\newcommand{\C}{\fancy{C}}
\newcommand{\F}{\fancy{F}}
\newcommand{\W}{\fancy{W}}
\newcommand{\IN}{\mathbb{N}}
\newcommand{\IR}{\mathbb{R}}
\newcommand{\G}{\fancy{G}}
\newcommand{\B}{\fancy{B}}
\newcommand{\LB}{\mathcal{L}_B}
\newcommand{\col}{{\textrm{col}}}
\newcommand{\ch}{{\textrm{ch}}}
\newcommand{\chil}{{\chi_{\ell}}}
\newcommand{\chiol}{{\chi_{OL}}}
\newcommand{\T}{\fancy{T}}

\newcommand{\inj}{\hookrightarrow}
\newcommand{\surj}{\twoheadrightarrow}

\newcommand{\set}[1]{\left\{ #1 \right\}}
\newcommand{\setb}[3]{\left\{ #1 \in #2 : #3 \right\}}
\newcommand{\setbs}[2]{\left\{ #1 : #2 \right\}}
\newcommand{\card}[1]{\left|#1\right|}
\newcommand{\size}[1]{\left\Vert#1\right\Vert}
\newcommand{\ceil}[1]{\left\lceil#1\right\rceil}
\newcommand{\floor}[1]{\left\lfloor#1\right\rfloor}
\newcommand{\func}[3]{#1\colon #2 \rightarrow #3}
\newcommand{\funcinj}[3]{#1\colon #2 \inj #3}
\newcommand{\funcsurj}[3]{#1\colon #2 \surj #3}
\newcommand{\irange}[1]{\left[#1\right]}
\newcommand{\join}[2]{#1 \mbox{\hspace{2 pt}$\ast$\hspace{2 pt}} #2}
\newcommand{\djunion}[2]{#1 \mbox{\hspace{2 pt}$+$\hspace{2 pt}} #2}
\newcommand{\parens}[1]{\left( #1 \right)}
\newcommand{\brackets}[1]{\left[ #1 \right]}
\newcommand{\DefinedAs}{\mathrel{\mathop:}=}

\newcommand{\mic}{\operatorname{mic}}
\newcommand{\AT}{\operatorname{AT}}
\renewcommand{\col}{\operatorname{col}}
\renewcommand{\ch}{\operatorname{ch}}
\newcommand{\type}{\operatorname{type}}
\newcommand{\nonsep}{\bar{S}}
\newcommand{\dclaw}[1]{d_{\text{claw}}\left( #1 \right)}

\def\adj{\leftrightarrow}
\def\nonadj{\not\!\leftrightarrow}

\newcommand{\vph}{\varphi}
\newcommand{\vphn}{\overline{\varphi}}

\newcommand{\claim}[2]{{\noindent\bf Claim #1.}~{\it #2}~~}
\newenvironment{claimproof}[1]{\par\noindent\underline{Proof:}\space#1}{\leavevmode\unskip\penalty9999
\hbox{}\nobreak\hfill\quad\hbox{$\qed$}}
%
%  If the proof ends with a displayed equation, use \aftermath just
%  before \end{proof} to put the halmos in the ``right'' place.  This
%  may not work near page boundaries. 
%
\def\aftermath{\par\vspace{-\belowdisplayskip}\vspace{-\parskip}\vspace{-\baselineskip}}

\begin{document}
\section{Overview}

For every multigraph $G$, we have $\chi'(G)\ge
\ceil{\frac{|E(G)|}{\floor{|V(G)|/2}}}$, since each color class has size at most
$\floor{\frac{|V(G)|}2}$.  Likewise, the same bound holds for any subgraph $H$.  
Thus, let $\W(G)=\max_{H\subseteq G}\ceil{\frac{|E(H)|}{\floor{|V(H)|/2}}}$
(over all subgraphs $H$ with at least two vertices). Now 
clearly $\chi'(G)\ge \W(G)$ for every multigraph $G$.  Goldberg~\cite{} and
Seymour~\cite{} each conjectured that this lower bound holds with equality,
whenever $\chi'(G)>\Delta(G)+1$.
\begin{GS}
When $\W(G)$ %\max_{H\subseteq G}\ceil{\frac{|E(H)|}{\floor{|V(H)|/2}}}$ 
is as above, 
every multigraph $G$ satisfies
\[
\chi'(G)\le\max\{\W(G), \Delta(G)+1\}.
\]
\end{GS}
The Goldberg--Seymour conjecture is the major open problem in the area of
edge-coloring multigraphs.
%The biggest open problem in edge-coloring is 
%Over the past two decades, the main tool for attacking this problem has become
%Tashkinov trees, a vast generalization of Vizing fans and Kierstead paths.
The second author showed~\cite{rabern2011strengthening} that $\chi(G)\le
\max\{\omega(G), \frac{7\Delta(G)+10}{8}\}$  for every line graph $G$.
In the same paper, he conjectured
that $\chi(G)\le \max\{\omega(G),\frac{5\Delta(G)+8}{6}\}$. This conjecture is best
possible, as shown by replacing each edge in a 5-cycle by $k$ parallel edges,
and taking the line graph.
%We call the latter inequality the $\frac56$-Conjecture, and 
In this paper we prove the latter inequality.  Along the way, we develop
more general techniques and results that will likely be of independent
interest, due to their use in approaching the Goldberg--Seymour conjecture.

The main result of this paper is the following theorem.
\begin{main}[$\frac56$-Theorem]
If $Q$ the line graph of a multigraph $G$, then 
we have $\chi(Q)\le \max\{\omega(Q),\frac{5\Delta(Q)+8}{6}\}$. 
\end{main}

Most of our work goes toward proving the following
intermediate result, in Section~\ref{sec:thin}.
\begin{main2}[Weak $\frac56$-Theorem]
If $Q$ the line graph of a multigraph $G$, then 
$\chi(Q)\le \max\{\W(G),\Delta(G)+1,\frac{5\Delta(Q)+8}{6}\}$. 
\end{main2}
Finally, in Section~\ref{sec:56} we show that the Weak $\frac56$-Theorem does indeed imply
the $\frac56$-Theorem.


%A graph $G$ is \emph{elementary} if $\chi'(G)=\W(G)$; such graphs satisfy the
%Goldberg--Seymour Conjecture.  
%%(We begin by proving that every minimal counterexample to the
%%$\frac56$-Conjecture is elementary.  In Section 3, we conclude
%%by also proving that every elementary graph satisfies the $\frac56$-conjecture.)  
%A
%\emph{defective color}
%for a Tashkinov tree $T$ is a color used on more than one edge from $V(T)$ to
%$V(G)-V(T)$; a Tashkinov tree is \emph{strongly closed} if it has no defective
%color.  Andersen~[] and Goldberg~[] showed that if $G$ is critical, then $G$ is
%elementary if there exists $e\in E(G)$ and $X\subseteq V(G)$ and a
%$k$-edge-coloring $\vph$ of $G-e$ such that $X$ contains the endpoints of $e$
%and $X$ is elementary and strongly closed w.r.t.~$\vph$.  Thus, to show that $G$
%is elementary, it suffices to show that if $G$ is $(k+1)$-critical, then there
%exists an edge $e\in E(G)$ and a $k$-coloring $\vph$ of $G-e$ such that some
%maximal Tashkinov tree containing $e$ is strongly closed.  The following
%definition is useful.  
%
%A vertex $v \in V(G)$ is \emph{short} if every Vizing fan rooted at $v$ (taken over all
%$k$-colorings of $G-e$, over all edges $e$ incident to $v$) has at most 3 vertices,
%including $v$.  Otherwise, $v$ is \emph{long}.
%As a warmup, in Section 2 we prove that if $\chi'(G)\ge \Delta(G)+2$ and every
%vertex of $G$ is short, then $G$ is elementary, i.e., $\chi'(G)=\W(G)$.  Next,
%we push our methods further, allowing our maximal Tashkinov tree to have at most
%3 long vertices.
%
%In Section 3, we consider the %show that 
%$\frac56$-Conjecture.  As a consequence of results in Section 2,
%if $G$ is a minimal counterexample, then every long vertex $v$ has
%$d_G(v)<\frac34\Delta(G)$.  Since every maximal Tashkinov tree $T$ is
%elementary, and every long vertex misses more than $\frac{k}4$ colors, we
%conclude that $T$ has at most 3 long vertices.  Thus our results from
%Section 2 apply.  As a consequence, every minimal countexample to the
%$\frac56$-Conjecture is elementary.  To complete the proof of the $\frac56$-conjecture,
%%we prove that it follows from the Goldberg--Seymour Conjecture.  More precisely,
%we show for each graph $G$ that if $\chi'(G)=\W(G)$, then
%$\chi'(G)\le\max\{\omega(G),\frac{5\Delta(G)+8}6\}$.
%
%

\section{Tashkinov Trees}
Throughout this paper, graphs can have multiple edges unless stated otherwise.
A graph $G$ is \emph{elementary} if $\chi'(G)=\W(G)$. 
%as defined above.  
%We also use the following notation.  
Let $[k]$ denote $\{1,\ldots,k\}$.
For a path or cycle $Q$, let \emph{$\ell(Q)$} denote the length of $Q$.
A graph $G$ is \emph{critical} if $\chi'(G-e) < \chi'(G)$ for all $e \in E(G)$. 
For a graph $G$ and a partial $k$-edge-coloring $\varphi$, for each vertex $v\in
V(G)$, let $\varphi(v)$ denote the set of colors used in $\varphi$ on edges
incident to $v$.  Let $\vphn(v)=[k]\setminus\varphi(v)$.  A color $c$ is
\emph{seen} by a vertex $v$ if $c\in \varphi(v)$ and $c$ is \emph{missed} by $v$
if $c\in\vphn(v)$.
Given a partial $k$-edge-coloring $\varphi$, a set $W\subseteq V(G)$ is
\emph{elementary} with respect to $\varphi$ (henceforth,
\emph{w.r.t.~$\varphi$}) if each color in $[k]$ is
missed by at most one vertex of $W$.  More formally, $\vphn(u)\cap
\vphn(v)=\emptyset$ for all distinct $u,v\in W$.
A \emph{defective color} for a set $X\subseteq V(G)$ (w.r.t.~$\varphi$) is a color
used on more than one edge from $X$ to $V(G) \setminus X$.  
A set $X$ is \emph{strongly closed} w.r.t.~$\varphi$ if $X$ has no 
defective color.
Elementary and strongly closed sets are of particular interest because of the
following theorem, proved implicitly by Andersen~\cite{} and Goldberg~\cite{};
see also~\cite[Theorem 1.4]{SSTF}.
%As we will see shortly, there is a strong relationship between elementary sets
%and elementary graphs.
% 

\begin{thm}
\label{elementary}
Let $G$ be a graph with $\chi'(G)=k+1$ for some integer $k\ge \Delta(G)$.  If
$G$ is critical, then $G$ is elementary if and only if there exists $uv\in E(G)$,
a $k$-edge-coloring $\vph$ of $G-uv$, and a set $X$ with $u,v\in X$ such
that $X$ is both elementary and strongly closed w.r.t.~$\varphi$.
\end{thm}

A \emph{Tashkinov tree} w.r.t.~$\varphi$ is a sequence $v_0, e_1, v_1,
e_2,\ldots, v_{t-1},e_t,v_t$ such that all $v_i$ are distinct, $e_i=v_jv_i$ and
$\vph(e_i)\in \vphn(v_\ell)$ for some $j$ and $\ell$ with $0\le j< i$ and $0\le
\ell < i$.  
A \emph{Vizing fan} (or simply \emph{fan}) is a Tashkinov tree that induces a
star.  Tashkinov trees are of interest because of the following lemma. 

\begin{lemmaA}%[Tashkinov]
Let $G$ be a graph with $\chi'(G)=k+1$, for some integer $k\ge \Delta(G)+1$ and
choose $e\in E(G)$ such that $\chi'(G-e)<\chi'(G)$.  Let $\varphi$ be a
$k$-edge-coloring of $G-e$.  If $T$ is a Tashkinov tree w.r.t.~$\varphi$ and
$e$, then $V(T)$ is elementary w.r.t.~$\varphi$.
\end{lemmaA}

In view of Theorem~1 and Tashkinov's Lemma, to prove that a graph $G$ is elementary,
it suffices to find an edge $e$, a $k$-edge-coloring $\vph$ of $G-e$, and a
Tashkinov tree $T$ containing $e$ such that $V(T)$ is strongly closed.
This motivates our next two lemmas.  But first, we need a few more definitions.

Let $t(G)$ be the maximum number of vertices in a Tashkinov tree over all $e \in E(G)$
and all $k$-edge-colorings $\vph$ of $G - e$.  Let $\T(G)$ be the set of all triples $(T,e,\vph)$ such that $e \in E(G)$, $\vph$ is a $k$-edge-coloring of $G-e$ and
$T$ is a Tashkinov tree with respect to $e$ and $\vph$ with $|T| = t(G)$.  Notice that, by definition, we have $\T(G) \ne \emptyset$.
%
For a $k$-edge-coloring $\vph$ of $G-e$, a maximal Tashkinov tree
starting with $e$ may not be unique.  However, if $T_1$ and $T_2$ are both such
trees, then it is easy to show that $V(T_1)\subseteq V(T_2)$; by symmetry, also
$V(T_2)\subseteq V(T_1)$, so $V(T_1)=V(T_2)$.
%
Let $G$ be a critical graph with $\chi'(G) = k+1$ for some integer $k \ge \Delta(G) + 1$. 
Let $\varphi$ be a $k$-edge-coloring of $G - e_0$ for some $e_0 \in E(G)$.  
%For each vertex $v\in V(G)$, let $\varphi(v)$ be the set of colors used in
%$\vph$ on edges incident to $v$ and let $\vphn(v)=[k]\setminus \vph(v)$. 
For $v \in V(G)$ and colors $\alpha, \beta$, let $P_v(\alpha, \beta)$ be the
maximal connected subgraph of $G$ that contains $v$ and is induced by edges with color
$\alpha$ or $\beta$.  So $P_v(\alpha, \beta)$ is a path or a cycle.
For a $k$-edge-coloring $\vph$ of $G-v_0v_1$, we often let
$P=P_{v_1}(\alpha,\beta)$ for some $\alpha\in\vphn(v_0)$ and
$\beta\in\vphn(v_1)$.  
Clearly $P$ must end at $v_0$ (or we can swap colors $\alpha$ and $\beta$ on
$P$ and color $v_0v_1$ with $\alpha$), so let $v_1,\ldots,v_r,v_0$ denote the
vertices of $P$ in order. 
To \emph{rotate the $\alpha,\beta$ coloring on $P\cup\{v_0v_1\}$ by one}, we
uncolor $v_1v_2$ and use its color on $v_0v_1$.  To \emph{rotate the
$\alpha,\beta$ coloring on $P\cup\{v_0v_1\}$ by $j$}, we rotate the
$\alpha,\beta$ coloring by one $j$ times in succession.
(When we do not specify $j$, we allow $j$ to take any value from $1$ to
$r$.)  

%TODO: NEED DEFINITION OF ELEMENTARY SET, ELEMENTARY GRAPH.  NEED TO STATE LEMMA THAT TASHKINOV TREES ARE ELEMENTARY, ALSO THAT MAXIMUM SIZE TASHKINOV TREE WITHOUT DEFECTIVE COLORS IMPLIES G IS ELEMENTARY.

\begin{lem}\label{FreeColorsLemma}
Let $G$ be a non-elementary critical graph with $\chi'(G) = k+1$ for some integer
$k \ge \Delta(G) + 1$.  For every $v_0v_1 \in E(G)$, $k$-edge-coloring $\vph$
of $G-v_0v_1$, $\alpha \in \vphn(v_0)$, and $\beta \in
\vphn(v_1)$, we have $|P_{v_1}(\alpha, \beta)| < t(G)$.
\end{lem}
\begin{proof}
Suppose the lemma is false and choose $v_0v_1 \in E(G)$, a $k$-edge-coloring
$\vph$ of $G-v_0v_1$, $\alpha \in \vphn(v_0)$, and $\beta \in \vphn(v_1)$, such
that $|P_{v_1}(\alpha, \beta)| \ge t(G)$.  Let $P = P_{v_1}(\alpha, \beta)$.  
Let $(T, v_0v_1, \vph)$ be a Tashkinov tree that
begins with edges $v_0v_1, v_1v_2, \ldots, v_{r-1}v_r$.  Now $V(T)=V(P)$ since
$t(G) \ge |T| \ge |P| \ge t(G)$.
By hypothesis $G$ is non-elementary, so Theorem~\ref{elementary} implies that
$V(T)$ is not strongly closed; thus, $T$ has a defective color $\delta$ with
respect to $\vph$.  Choose $\tau\in \vphn(v_2)$. Let $Q = P_{v_2}(\tau, \delta)$.
Since $T$ is maximal, $\delta$ is not missing at any vertex of $T$, and
since $V(T)$ is elementary, $\tau$ is not missing at any vertex of $T$ other 
than $v_2$.  As a result, $Q$ ends outside $V(T)$.  Now $Q$ could leave
$V(T)$ and re-enter it repeatedly, but $Q$ ends outside $V(T)$, so there is a
last vertex $w \in V(Q) \cap V(T)$; say $Q$ ends at $z \in V(G)\setminus V(T)$.
 Let $\pi \notin \{\alpha, \beta\}$ be a color missing at $w$.  
Since $\tau\in\vphn(v_2)$ and $\pi\in\vphn(w)$ and $|T| = t(G)$, no edge
colored $\tau$ or $\pi$ leaves $V(T)$.  So we can swap $\tau$ and $\pi$ on
every edge in $G - V(T)$ without changing the fact that $T$ is a Tashkinov tree
with $|T| = t(G)$.  After swapping $\tau$ and $\pi$, we swap $\delta$ and $\pi$
on the subpath of $Q$ from $w$ to $z$. Since $\pi$ is missing at $w$, the
$\delta-\pi$ path starting at $z$ must end at $w$.  Now $\delta$ is missing at
$w$, but $\delta$ was defective in $\vph$, so some other edge $e$ colored
$\delta$ still leaves $V(T)$. Adding $e$ gets a larger Tashkinov tree, which is
a contradiction.
\end{proof}


\section{Short vertices}
\label{sec:short}
Recall that a vertex $v \in V(G)$ is \emph{short} if every Vizing fan rooted
at $v$ (taken over all $k$-colorings of $G-e$, over all edges $e$ incident to
$v$) has at most 3 vertices, including $v$.  Otherwise, $v$ is \emph{long}.
Let $\nu(T)$ be the number of long vertices in a Tashkinov tree $T$.

Now we can outline our proof of the $\frac56$-Conjecture.
We will show in Section~\ref{sec:56} that the $\frac56$-Conjecture is implied
by the Goldberg--Seymour Conjecture.  More precisely, if $G$ is a multigraph
such that $\chi'(G)\le\max\{\ceil{\W(G)},\Delta(G)+1\}$, then also $\chi'(G)\le
\frac{5\Delta(G)+8}6$.  So here it suffices to show that 
$\chi'(G)\le\max\{\ceil{\W(G)},\Delta(G)+1,\frac{5\Delta(G)+8}6\}$.  We consider
cases based on $\nu(T)$, for some Tashkinov tree $T\in \T(G)$.

In the present section, we show that if $G$ has a maximum Tashkinov tree $T$ that contains
no short vertices, i.e., $\nu(T)=0$, then $G$ is elementary.  In fact, Lemma~7
implies that the same is true when $\nu(T)=1$.  In the proof of
Theorem~\ref{mainhelper}, we show that if $G$ is a minimal counterexample to
the $\frac56$-Conjecture, then every long vertex $v$ has
$d(v)<\frac34\Delta(G)$.  This implies that $\nu(T)< 4$, since otherwise
the number of colors missing at vertices of $T$ is more than
$4(k-\frac34\Delta(G))>k$, which contradicts that $V(T)$ is elementary.
%$2|E(G)|\le (|V(G)|-4)\Delta(G)+3\Delta(G)=\Delta(G)(|V(G)|-1)$, so
%$\ceil{\frac{|E(G)|}{\floor{\frac{|V(G)|}2}}}\le
%\ceil{\frac{|E(G)|}{\frac{|V(G)|}2-\frac12}}=
%\ceil{\frac{2|E(G)|}{|V(G)|-1}}\le
%\ceil{\frac{\Delta(G)(|V(G)|-1)}{|V(G)|-1}}=\Delta(G)$, and thus
%$G$ has too few edges to be a minimal counterexample.  
So it remains to consider the case $\nu(T)\in\{2,3\}$.

In Section~6, we introduce the notion of \emph{$k$-thin graphs}, which are
essentially those for which $\mu(G)$ is not too large.  Using a lemma
from~\cite{rabern2011strengthening}, we show that every minimal counterexample to the
$\frac56$-Conjecture must be $k$-thin.  We then extend the ideas of the present
section to show handle the case when $\nu(T)\in\{2,3\}$.  Much like when
$\nu(T)\ge 4$, we show that $T$ has too many colors missing at its vertices to
be elementary.  More precisely, $\sum_{v\in V(T)}|\vphn(v)|>k$, which is a
contradiction.

Short vertices were introduced in~\cite{CKPS}, where they were motivated by a
version of the following lemma in the context of proving a strengthening of
Reed's Conjecture for line graphs.  

\begin{lem}\label{SpecialPath}
Let $G$ be a critical graph with $\chi'(G) = k+1$ for some integer $k \ge
\Delta(G) + 1$.  Let $\vph$ be a $k$-edge-coloring of $G-v_0v_1$.  Choose
$\alpha \in \vphn(v_0)$ and $\beta \in \vphn(v_1)$.  Let $P = v_1v_2\cdots v_r$
be an $\alpha,\beta$ path with edges $e_i = v_iv_{i+1}$ for all $i\in[r-1]$. 
If $v_i$ is short for all odd $i$, then for each $\tau \in \vphn(v_0)$ there
are edges $f_i = v_iv_{i+1}$ for all $i\in[r-1]$ such that $f_i = e_i$ for $i$
even and $\vph(f_i) = \tau$ for $i$ odd.
\end{lem}
\begin{proof}
Suppose not and choose a counterexample minimizing $r$.  By minimality of
$r$, we have $\vph(v_{r-1}v_r) = \alpha$ and we have $f_i = v_iv_{i+1}$ for
all $i\in[r-2]$ such that $f_i = e_i$ for $i$ even and $\vph(f_i) = \tau$ for
$i$ odd.  Swap $\alpha$ and $\beta$ on $e_i$ for all $i\in [r-3]$ and then
color $v_0v_1$ (call this edge $e_0$) with $\alpha$ and uncolor $e_{r-2}$.  Let
$\vph'$ be the resulting coloring.  Since $k \ge \Delta(G) + 1$, some color
other than $\alpha$ is missing at $v_{r-2}$; let $\gamma$ be such a color.  Now 
$v_{r-1}$ is short since $r-1$ is odd (since $P$ starts and ends with
$\alpha$), so there is an edge $e = v_{r-1}v_r$ with $\vph'(e) = \gamma$.  
Swap $\tau$ and $\alpha$ on $e_i$ for all $i$ with $0 \le i \le r-3$ to get a
new coloring $\vph^*$.  Now $\gamma$ and $\tau$ are both missing at $v_{r-2}$
in $\vph^*$.  Since $v_{r-1}$ is short, the fan with $v_{r-2}, v_{r-1}, v_r$
and $e$ implies that there is an edge $f_{r-1} = v_{r-1}v_r$ with
$\vph^*(f_{r-1}) = \tau$.  But we have never recolored $f_{r-1}$, so
$\vph(f_{r-1})=\tau$, which is a contradiction.
\end{proof}

\begin{lem}\label{ZeroNonSpecial}
Let $G$ be a non-elementary critical graph with $\chi'(G) = k+1$ for some
integer $k \ge \Delta(G) + 1$.  Choose $(T, v_0v_1, \vph) \in \T(G)$ for some
$v_0v_1 \in E(G)$.  Choose $\alpha \in \vphn(v_0)$ and $\beta \in \vphn(v_1)$ and
let $P = P_{v_1}(\alpha, \beta)$.  Now $P$ contains a long vertex. 
In particular, $\nu(T) \ge 1$.
\label{lem2}
\end{lem}
\begin{proof}
Suppose every vertex of $P$ is short.  Applying Lemma \ref{SpecialPath} to $P$
shows that for every $\tau \in \vphn(v_0)$, there is an edge in $T$ colored
$\tau$ incident to every $v \in V(P - v_0)$.  The same is also true of every $v
\in V(P)$; to see this, we rotate the $\alpha,\beta$ coloring of
$P\cup\{v_0v_1\}$ and repeat the same argument.  Hence $V(P) = V(T)$, which
contradicts Lemma \ref{FreeColorsLemma}.
\end{proof}

\begin{thm}\label{AllSpecialImpliesElementary}
If $G$ is a critical graph in which every vertex is short, then
\[\chi'(G) \le \max \set{\W(G), \Delta(G) + 1}.\]
\end{thm}
\begin{proof}
Suppose not and let $G$ be a counterexample. % with the fewest edges.
%Suppose $G$ is a critical graph in which every vertex is short and 
Let $k = \chi'(G) - 1$, and note that $k \ge \Delta(G) + 1$.  
Since $\T(G) \ne \emptyset$, by applying Lemma \ref{ZeroNonSpecial} we conclude
that $G$ is elementary.  Hence $\chi'(G) = \W(G)$, which is a
contradiction.
\end{proof}

\section{An easy bound}
In this section, we apply the results of Section~\ref{sec:short} to prove an
easy bound on $\chi'(G)$.  We also show how those results imply Reed's Conjecture, as
well as Local and Superlocal strengthenings of Reed's Conjecture, for the class
of line graphs.

Let $G$ be a graph.  The \emph{claw-degree} of $x \in V(G)$ is 
\[\dclaw{x} \DefinedAs \max_{\substack{S \subseteq N(x) \\ \card{S} = 3}}\frac14 \parens{d(x) + \sum_{v \in S} d(v)}.\]
The \emph{claw-degree} of $G$ is 
\[\dclaw{G} \DefinedAs \max_{x \in V(G)} \dclaw{x}.\]
\begin{thm}
\label{EasyBound}
If $G$ is a graph, then
\[\chi'(G) \le \max\set{\W(G), \Delta(G) + 1, \ceil{\frac43\dclaw{G}}}.\]
\end{thm}
\begin{proof}
Suppose not and choose a counterexample $G$ with the fewest edges; note that $G$
is critical. 
Let $k=\chi'(G)-1$, so $k \ge \ceil{\frac43\dclaw{G}}$. 
By Theorem \ref{AllSpecialImpliesElementary}, $G$ has a long vertex $x$.
Choose $xy_1 \in E(G)$ and a $k$-edge-coloring $\vph$ of $G - xy_1$ such that
$\vph$ has a fan $F$ of length $3$ rooted at $x$ with leaves $y_1, y_2, y_3$.  
Since $V(F)$ is elementary, 
\[2 + k - d(x) + \sum_{i \in \irange{3}} k-d(y_i) \le k,\]
and hence
\[\dclaw{x} \ge \frac14\parens{d(x) + \sum_{i \in \irange{3}} d(y_i)} \ge \frac{3k+2}{4}.\]
This gives the contradiction
\[\ceil{\frac43\dclaw{G}} \le k \le \frac43\dclaw{G} - \frac23.\]
\aftermath
\end{proof}

%TODO: ADD REED, LOCAL REED AND SUPERLOCAL REED CONSEQUENCES.
Reed~\cite{Reed1998omega} conjectured that $\chi(G)\le
\ceil{\frac{\omega(G)+\Delta+1}2}$ for every
graph $G$.  This is the average of a trivial lower bound $\omega(G)$ and a
trivial upper bound $\Delta(G)+1$.  King~\cite{King} conjecture the stronger
bound $\chi(G)\le \max_{v\in V(G)}\ceil{\frac{\omega(v)+d(v)+1}2}$, where
$\omega(v)$ is the size of the largest clique containing $v$, which is now known
to hold for many classes of graphs, including line graphs~\cite{CKPS}.  Here we
show that for line graphs this bound is an easy consequence of our more general
lemmas from Section~\ref{sec:short}.  The following is essentially Lemma~10 from
\cite{CKPS}.

\begin{cor}
\label{SuperlocalReedCor}
Let $G$ be a graph.  For $uv\in E(G)$, let
$f(uv)=\max\{d(u)+\frac12(d(v)-\mu(uv)),d(v)+\frac12(d(u)-\mu(uv)\}$.  Let
$f(G)=\max_{uv,vw\in E(G)}\ceil{\frac12(f(uv)+f(uw))}$.  Now 
\[\chi'(G) \le \max\set{\W(G), \Delta(G) + 1, f(G)}.\]
In particular, the Superlocal version of Reed's Conjecture holds for every line
graph.
\end{cor}
\begin{proof}
The first statement follows directly from Theorem~\ref{EasyBound}, by showing that
$\ceil{\frac43d_{claw}(G)}\le f(G)$.  Choose $x\in V(G)$ and $S\in N(x)$ such
that $x$ and $S$ achieve maximality in the definition of $d_{claw}(G)$.
Now 
\begin{align*}
\ceil{\frac43\frac14(d(x)+d(v_1)+d(v_2)+d(v_3))} \le&
\left\lceil\frac13\left( 
d(v_1)+\frac12(d(x)-\mu(xv_1))\right.\right.\\
&+d(v_2)+\frac12(d(x)-\mu(xv_2))\\
&\left.\left.+d(v_3)+\frac12(d(x)-\mu(xv_3))\right)\right\rceil\\
&\le \ceil{\frac13(f(xv_1)+f(xv_2)+f(xv_3))}\\
&\le f(G).
\end{align*}
This proves the first statement.  For the second statement, we show that
$\W(G)\le f(G)$, as follows.  For each vertex $v$, let $v_1, v_2,\ldots$ denote
the neighbors of $v$ (with subscripts modulo $|N(v)|$).  Also, let
$\overline{d}=\frac{2|E(H)|}{|V(H)|}$.
\begin{align*}
f(G)2|E(H)| &\ge \sum_{v\in
V}\sum_{i=1}^{|N(v)|}\frac12(d(v)+\frac12(d(v_i)-\mu(vv_i))+d(v)+\frac12(d(v_{i+1})-\mu(vv_{i+1})))\\
 &= \sum_{v\in
V}\sum_{i=1}^{|N(v)|}d(v)+\frac12(d(v_i)-\mu(vv_i))\\
& = \sum_{uv\in E(H)}\frac32d(u)+\frac32d(v)-\mu(uv) \\
& = \sum_{v\in V(H)}\frac32d(v)^2 - |E(H)|\\
& \ge \frac32\overline{d}^2|V(H)|-|E(H)|\\
& = 6\frac{|E(H)|^2}{|V(H)|}-|E(H)|
\end{align*}
Thus $f(G)\ge \frac{3|E(H)|}{|V(H)|}-\frac12$.
Since $\W(G)=\ceil{\frac{2|E(H)|}{|V(H)|-1}}\le
\frac{2|E(H)|+|V(H)|-3}{|V(H)|-1}$, 
it suffices to have 
$\frac{3|E(H)|}{|V(H)|}-\frac12 \ge \frac{2|E(H)|+|V(H)|-3}{|V(H)|-1}$. 
Now solving for $|E(H)|$
gives $|E(H)|\ge \frac32|V(H)|\frac{|V(H)|-\frac73}{|V(H)|-3}$.  Taking
$|V(H)|\ge 5$, it suffices to have $2|V(H)|\le |E(H)|$.  Suppose, to the
contrary, that we have $|E(H)|<2|V(H)|$.  It will suffice to show that
$\frac{2|E(H)|}{|V(H)|-1}\le \Delta(H)$.  Now solving
$\frac{4|V(H)|}{|V(H)|-1}\le \Delta(H)$ (using $|V(H)|\ge 5$), shows that it
suffices to have $\Delta(H)\ge 5$.\\
TODO: HANDLE $\Delta(H)\in\{3,4\}$.
\end{proof}

%\begin{lem}
%\label{ReedHelper}
%Let $G$ be a mulitgraph with a long vertex $v_0$.  
%If $k=\max_{uv\in E(G)}d(u)+\frac12(d(v)-\mu(uv))$, then $\chi'(G)\le k$.
%\end{lem}
%\begin{proof}
%Suppose the lemma is false and let $G$ be a minimal counterexample.  So
%$\chi'(G)=k+1$ and $\chi'(G-e)=k$ for all $e\in E(G)$.
%Choose $e$ incident to $v_0$ such that $G-e$ has a $k$-coloring $\phi$ and some
%fan $F$ rooted at $v_0$ and containing $e$ has length at least 3.  Subject to
%this, let $F$ be maximal, and let $v_1,\ldots,v_\ell$ denote the leaves of $F$.
%Since $F$ is a Tashkinov tree, $V(F)$ must be elementary.  Further, since $F$ is
%maximal, each edge incident to $v_0$ with other endpoint outside of $F$ must use
%a color that is not missing at every vertex of $F$.  This gives
%\begin{align*}
%2+\sum_{i=0}^\ell k-d(v_i) + d(v_0)-\sum_{i=1}^\ell\mu(v_0v_i) &\le k\\
%2+\sum_{i=1}^\ell k-d(v_i) &\le \sum_{i=1}^\ell\mu(v_0v_i) 
%\end{align*}
%For each edge $v_0v_i$ we have $d(v_i)+\frac12(d(v_0)-\mu(v_0v_i))\le k$.
%Substituting this gives
%\begin{align*}
%2+\sum_{i=1}^\ell d(v_i)+\frac12(d(v_0)-\mu(v_0v_i))-d(v_i) &\le
%\sum_{i=1}^\ell\mu(v_0v_i) \\
%2+\sum_{i=1}^\ell \frac12d(v_0) &\le \frac32\sum_{i=1}^\ell\mu(v_0v_i) \\
%\frac{l}2d(v_0)< \frac32d(v_0),
%\end{align*}
%since $\ell\ge 3$.

%\begin{cor}
%The Local Strengthening of Reed's Conjecture holds for line graphs.
%\end{cor}
%\begin{proof}
%Suppose the corollary is false and let $G$ be a counterexample with as few edges
%as possible.  Note that $G$ is critical.
%If $G$ contains a long vertex $x$, then the bound follows from
%Lemma~\ref{ReedHelper}.  So suppose $G$ has no long vertex.  By
%Lemma~\ref{AllSpecialImpliesElementary},
%now we have $\chi'(G) \le \max \set{\W(G), \Delta(G) + 1}.$  TODO: SHOW THIS
%IMPLIES LOCAL REED'S.
%\end{proof}

\section{Properties of long vertices}
For a path $Q$, recall that $\ell(Q)$ denotes the length of $Q$.
For $x,y \in V(Q)$, let $xQy$ denote the subpath of $Q$ with
endvertices $x$ and $y$, and let $d_Q(x,y) = \ell(xQy)$, i.e., the distance
from $x$ to $y$ along $Q$.

\begin{lem}\label{TauEscape}
Let $G$ be a critical graph with $\chi'(G) = k+1$ for some integer $k \ge \Delta(G) + 1$.
Let $\vph$ be a $k$-edge-coloring of $G-v_0v_1$. Choose $\alpha \in \vphn(v_0)$
and $\beta \in \vphn(v_1)$ and let $C = P_{v_1}(\alpha, \beta) + v_0v_1$.  If
$\tau \in \vphn(x)$ for some $x \in V(C)$ and there is a $\tau$-colored edge
from $y \in V(C)$ to $w \in V(G) \setminus V(C)$, then $C$ has a subpath $Q$
with long endpoints $z_1,z_2$ such that $x \in V(Q)$, $y \not \in V(Q-z_1-z_2)$
and the distance from $x$ to $z_i$ along $Q$ is odd for each $i \in \irange{2}$. 
Moreover, for each $i \in \irange{2}$, there are no $\tau$-colored edges
between $z_i$ and its neighbors along $C$.
\end{lem}
\begin{proof}
Let $G$, $\alpha$, $\beta$, $\tau$, $x$, and $y$ be as in the statement of the
lemma.  Choose $z_1$ (resp. $z_2$) to be the first vertex at an odd distance
from $x$ along $C$ in the clockwise (resp. counterclockwise) direction with no
incident $\tau$-colored edge parallel to some edge of $C$.  
Let $Q$ be the subpath of $C$ with endpoints $z_1$ and $z_2$ that contains $x$.
By the choice of $z_1$ each vertex $w$ between $x$ and $z_1$ with $d_Q(xw)$ odd
has a $\tau$-colored edge parallel to some edge of $C$.  The presence of these
edges implies the same for each $w$ for which $d_Q(xw)$ is even.  By the proof
of the Parallel Edge Lemma, $z_1$ must be long, since otherwise it would have an
incident $\tau$-colored edge parallel to some edge of $C$.  The same argument
applies to $z_2$.
%By rotating the $\alpha,\beta$ coloring, we can assume that $x=v_0$.
%By the Parallel Edge Lemma, we must not reach $y$ before we reach $z_1$ (resp.
%$z_2$).
%each vertex between $x$ and $z_1$ (exclusive)
%must be incident to a $\tau$-colored edge that is parallel to some edge of $C$.
%The same is true for each vertex between $x$ and $z_2$, as shown by making the
%same argument going around the cycle in the other direction.
%FILL THIS SPACE WITH PROOF.
%
%TODO: EXPLAIN WHY $y\notin\{z_1,z_2\}$ (OR CHANGE STATEMENT).
\end{proof}

\section{Thin graphs}
\label{sec:thin}
Let $G$ be a critical graph with $\chi'(G) = k+1$ for some integer $k \ge \Delta(G) + 1$.
For vertices $x \in V(G)$ and $S \subseteq V(G) \setminus \set{x}$, we say that $x$ is \emph{$S$-short} if 
every Vizing fan $F$ rooted at $x$ with $S \subseteq V(F)$, has $|F| \le 3$ (with respect to any $k$-edge-coloring of $G-xy$).
Otherwise, $x$ is \emph{$S$-long}.  For brevity, when $S = \set{y}$, we may write $y$-short instead of $\set{y}$-short.
It is worth noting that in Lemma~\ref{SpecialPath} we can weaken the hypothesis
that $v_i$ is short for all odd $i$ to require only that $v_i$ is
$v_{i-1}$-short for all odd $i$, since this is what we use in the proof.  

A graph $G$ is \emph{$k$-thin} if $\mu(G) < 2k - d(x) - d(y)$ for all
long $x,y \in V(G)$.  In the proof of Theorem~\ref{mainhelper}, we will show that
every counterexample to the $\frac56$-Conjecture must be $k$-thin.

\begin{lem}\label{NonSpecialsInThinAreAtEvenDistance}
Let $G$ be a $k$-thin, critical graph with $\chi'(G) = k+1$ for some integer $k \ge \Delta(G) + 1$.
Let $\vph$ be a $k$-edge-coloring of $G-v_0v_1$. Choose $\alpha \in \vphn(v_0)$
and $\beta \in \vphn(v_1)$ and let $C = P_{v_1}(\alpha, \beta) + v_0v_1$.
Let $Q$ be a subpath of $C$ with long end vertices.  If all internal vertices
of $Q$ are short and $2 \le \ell(Q) \le \ell(C) - 2$, then $\ell(Q)$ is even.
\end{lem}
\begin{proof}
Suppose to the contrary that we have a subpath $Q$ of $C$ with end vertices
long, all internal vertices short, $2\le \ell(Q) \le \ell(C) - 2$,
and $\ell(Q)$ odd.  Let $x$ and $y$ be the end vertices of $Q$.
Say $C = v_1v_2\cdots v_rv_0v_1$.  By rotating the $\alpha,\beta$ coloring of
$C$, we may assume that $x = v_0$ and $y = v_a$, where $a \ge 3$ is odd.

We now apply Lemma \ref{SpecialPath} twice, to show that $\mu(v_1v_2) \ge 2k -
d(v_0) - d(v_a)$, which contradicts that $G$ is $k$-thin.  More specifically,
assume that the edges $v_0v_1,v_1v_2,\ldots$ go clockwise around $C$.  We apply
Lemma~\ref{SpecialPath} once going clockwise starting from $v_0$ and once going
counterclockwise starting from $v_a$.  The first application implies that every
color in $\vphn(v_0)$ appears on some edge parallel to $v_1v_2$; the second
implies the same for every color in $\vphn(v_a)$.  Since $|\vphn(v_i)|=k-d(v_i)$
for each $i\in\{0,a\}$ and $\vphn(v_0)\cap \vphn(v_a)=\emptyset$, the conclusion
follows.
\end{proof}

\begin{lem}\label{ThreeNonSpecialOnCycle}
Let $G$ be a $k$-thin, critical graph with $\chi'(G) = k+1$ for some integer $k
\ge \Delta(G) + 1$.  Let $\vph$ be a $k$-edge-coloring of $G-v_0v_1$. Suppose
$\alpha \in \vphn(v_0)$ and $\beta \in \vphn(v_1)$ and let $C = P_{v_1}(\alpha,
\beta) + v_0v_1$.  If $C$ contains exactly 3 long vertices, then $C = xyAzBx$
where $A$ and $B$ are paths of even length and $x,y,z$ are all long.  Moreover,
$x$ is $y$-long and $y$ is $x$-long.
\end{lem}
\begin{proof}
Let $G$ be a graph satisfying the hypotheses, and let $x$, $y$, $z$ be the
three long vertices.
The three subpaths of $C$ with endpoints $x$,
$y$, and $z$ either (i) all have odd length or (ii) include two paths of even
length and one of odd length.  
First assume that $\ell(C)\ge 5$.  
%Since $\ell(C)\ge 5$, 
If we are in (i), then the longest of these three subpaths
violates Lemma~\ref{NonSpecialsInThinAreAtEvenDistance}; so we are in (ii), and
also the path of odd length is simply an edge.  This proves the first statement.
For the second statement, assume to the contrary that $x$ is $y$-short.
By rotating the $\alpha,\beta$ coloring, we can assume that $y=v_0$ and $x=v_1$.
As in the previous lemma, we use Lemma~\ref{SpecialPath} (and the comment in the
first paragraph of Section~\ref{sec:thin}) to conclude that $\mu(v_1v_2)\ge
2k-d(v_0)-d(z)$.  As above, this contradicts that $G$ is $k$-thin; this
contradiction proves the second statement.
\end{proof}

\begin{lem}\label{ConsecutiveNonSpecials}
Let $G$ be a non-elementary, $k$-thin, critical graph with $\chi'(G) = k+1$ for
some integer $k \ge \Delta(G) + 1$.  Choose $(T, v_0v_1, \vph) \in \T(G)$. If
$\alpha \in \vphn(v_0)$ and $\beta \in \vphn(v_1)$, then $P_{v_1}(\alpha,
\beta) + v_0v_1$ contains consecutive long vertices.
\end{lem}
\begin{proof}
Let $C = P_{v_1}(\alpha, \beta) + v_0v_1$.  By Lemma \ref{FreeColorsLemma},
there is $x \in V(C)$ and $\tau \in \vphn(x)$ such that there is a
$\tau$-colored edge from $y \in V(C)$ to $w \in V(T) \setminus V(C)$.
Lemma \ref{TauEscape} implies that $C$ has a subpath $Q$ with $x \in V(Q)$
 and long endpoints $z_1,z_2$ such that the distance from $x$ to
$z_i$ along $Q$ is odd for each $i \in \irange{2}$.  
Let $Q'$ be the subpath of $C$ with endpoints $z_1$ and $z_2$ that does not contain
$x$. Since $C$ is an odd cycle, $\ell(Q')$ is odd.  Let $Q^*$ be a minimum
length subpath of $Q'$ with long ends.  Now $\ell(Q^*) = 1$ by Lemma
\ref{NonSpecialsInThinAreAtEvenDistance}, as desired.
\end{proof}

\begin{lem}\label{MasterHelper}
Let $G$ be a non-elementary, $k$-thin, critical graph with $\chi'(G) = k+1$ for
some integer $k \ge \Delta(G) + 1$.  If $(T, v_0v_1, \vph) \in \T(G)$ and
$\nu(T) \le 3$, then $T$ contains long vertices $z_1,z_2,z_3$ such that either 
\begin{enumerate}
\item $z_1$ is $\set{z_2,z_3}$-long and $z_2$ is $z_1$-long; or
%\item $z_i$ is $z_{3-i}$-long and $z_{i+1}$ is $z_{4-i}$-long for $i \in \irange{2}$.
\item $z_i$ is $z_j$-long and $z_j$ is $z_i$-long for each %unordered pair
$(i,j)\in\{(1,2),(2,3)\}$.
\end{enumerate}
\end{lem}
\begin{proof}
Choose $\alpha \in \vphn(v_0)$ and $\beta \in \vphn(v_1)$ so that
$P_{v_1}(\alpha, \beta)$ contains as many long vertices as possible;
let $C=P_{v_1}(\alpha,\beta)+v_0v_1$.
By Lemma \ref{FreeColorsLemma}, there is $x \in V(C)$ and $\tau \in \vphn(x)$
such that there is a $\tau$-colored edge from $y \in V(C)$ to $w \in V(T)
\setminus V(C)$.  By Lemma~\ref{ConsecutiveNonSpecials}, $C$ has at least two
long vertices.

First suppose that $C$ contains only 2 long vertices, $z_1$ and $z_2$.  By
Lemma \ref{ConsecutiveNonSpecials}, $z_1$ and $z_2$ are consecutive on $C$.
Lemma \ref{TauEscape} implies that $C$ has a subpath $Q$ 
with endpoints $z_1,z_2$ such that $x \in V(Q)$ and $y \not \in V(Q-z_1-z_2)$ 
and for each $i \in \irange{2}$ 
there are no $\tau$-colored edges between $z_i$ and its neighbors on $C$.
%
By rotating the $\alpha,\beta$ coloring of $C$, we can assume that $x = v_0$ and
$\alpha,\tau\in \vphn(v_0)$ and $\beta\in\vphn(v_1)$. 
Note that $P_{v_1}(\tau,\beta)$ must end at $v_0$ (since otherwise we can
recolor the Kempe chain and color $v_0v_1$ with $\tau$).  
Let $C' = P_{v_1}(\tau, \beta) + v_0v_1$.  Note that $C'$ must include $v_1Qz_1$
and also $v_0Qz_2$ (the $\beta$-colored edges are present by definition and the
$\tau$-colored edges are present by the Parallel Edge Lemma).  Thus, $z_1,z_2\in
V(C')$.  Since $z_1$ and $z_2$ are not consecutive on $C'$
and $C'$ contains no other long vertices by the maximality condition on $C$,
Lemma \ref{ConsecutiveNonSpecials} gives a contradiction.

So instead $C$ contains exactly 3 long vertices, $z_1$, $z_2$, and $z_3$.  By Lemma
\ref{ThreeNonSpecialOnCycle}, $C = z_1z_2Az_3Bz_1$ where $A$ and $B$ are paths
of even length.  Also, $z_1$ is $z_2$-long and $z_2$ is $z_1$-long.  

By Lemma \ref{TauEscape}, $C$ has a subpath $Q$ 
with endpoints $z_1,z_3$ and with $x \in V(Q)$ and $y \not
\in V(Q-z_1-z_3)$ 
such that there are no $\tau$-colored edges
between $z_i$ and its neighbors along $C$ for each $i \in \set{1,3}$ (it could
happen that $z_3$ has a $\tau$-colored edge parallel to an edge of $C$,
so the endpoints of $Q$ are $z_1, z_2$, but now we get a contradiction
as in the previous case, by letting $C'=P_{v_1}(\tau,\beta)+v_0v_1$).  By
rotating the $\alpha,\beta$ coloring of $C$, we
may assume that $x = v_0$.  Again, let $C' = P_{v_1}(\tau, \beta) + v_0v_1$.  We
know that $C'$ contains $z_1$ and $z_3$ and that $z_1$ and $z_2$ are not
consecutive on $C'$.  Note also that all long vertices in $V(C')$ must be among
$z_1,z_2,z_3$, since otherwise $\nu(T)\ge 4$, contradicting our hypothesis.
So by Lemma~\ref{ConsecutiveNonSpecials}, either $z_1$ and
$z_3$ are consecutive on $C'$ or $z_2$ and $z_3$ are consecutive on $C'$.

Suppose that $z_2$ and $z_3$ are consecutive on $C'$, and thus connected by a
$\tau$-colored edge.  Now applying Lemma \ref{ThreeNonSpecialOnCycle} shows that $z_2$
is $z_3$-long and $z_3$ is $z_2$-long, so we satisfy (2) in the conclusion of
the lemma (by swapping the names of $z_1$ and $z_2$).

So instead $z_1$ and $z_3$ must be consecutive on $C'$, and thus connected by a
$\tau$-colored edge.  If $z_1 = v_1$, then we have a fan with an
$\alpha$-colored edge from $z_1$ to $z_2$ and a $\tau$-colored edge from $z_1$
to $z_3$, so $z_1$ is $\set{z_2,z_3}$-long. 

Now assume that $z_1\ne v_1$.
Let $z_1'$ be the predecessor of $z_1$ on the path from $v_0$ (through $v_1$) to
$z_1$.  We can shift the coloring so that $z_1'z_1$ is uncolored and $z_1z_2$
is colored $\alpha$ (as in the proof of the Parallel Edge Lemma).  In fact, 
we can shift either the $\alpha,\beta$ edges or the $\tau,\beta$ edges.  This
gives the options that either $\alpha\in \vphn(z_1')$ or $\tau\in \vphn(z_1')$,
whichever we prefer.  Suppose we shift the $\tau,\beta$ edges.
Now choose $\gamma\in \vphn(z_1')-\alpha-\tau$.  Consider
the $\gamma$-colored edge $e$ incident to $z_1$.  If $e$ goes to $z_2$, then we
$z_1$ is $\{z_2,z_3\}$-long, by colors $\gamma$ and $\tau$; so we satisfy (1) in
the conclusion of the lemma.
If instead $e$ goes to $z_3$, then instead of shifting the $\tau,\beta$ edges we
shift the $\alpha,\beta$ edges; note that this recoloring preserves the fact
that $\gamma$ is missing at $z_1'$.  Now again $z_1$ is $\{z_2,z_3\}$-long, this
time by colors $\alpha$ and $\gamma$; so we again satisfy (1) in the conclusion
of the lemma.

Finally, assume that the $\gamma$-colored edge incident to $z_1$ goes to some
vertex other than $z_2$ and $z_3$.  Now let $C''=P_{z_1}(\gamma,\beta)+z_1z_1'$.
Since $V(C'')\subseteq V(T)$, Lemmas~\ref{ConsecutiveNonSpecials} and
\ref{ThreeNonSpecialOnCycle} imply that $z_2$ and $z_3$ are
adjacent on $C''$ and furthermore $z_2$ is $z_3$-long and $z_3$ is $z_2$-long;
thus, we satisfy (2) in the conclusion of the lemma.
%can win by shifting the $\tau,\gamma$ edges.  Similarly, if $e$ goes to $z_3$, then
%we can win by shifting the $\alpha,\gamma$ edges.  
%The question is why do we know that e goes to either z_2 or z_3?
%When $z_1 \ne v_1$, we get the same thing, but have to shift the coloring over
%to $z_1$ like in the proof of Lemma \ref{SpecialPath} using another missing
%color $\gamma$ at $z_1$.
\end{proof}

We need the following result from~\cite{rabern2011strengthening}, which we use
to handle the case when $G$ is not $k$-thin.

\begin{thm}[\cite{rabern2011strengthening}]\label{CriticalMuBound}
If $Q$ is the line graph of a graph $G$ and $Q$ is vertex critical, then
\[\chi(Q) \leq \max\left\{\omega(Q), \Delta(Q) + 1 - \frac{\mu(G) - 1}{2}\right\}.\]
\end{thm}

Now we prove the main result of this section.

\begin{thm}
If $Q$ is the line graph of $G$, then
\[\chi(Q) \le \max\set{\ceil{\chi_f(Q)}, \Delta(G) + 1, \ceil{\frac{5\Delta(Q) + 3}{6}}}.\]
\label{mainhelper}
\end{thm}
\begin{proof}
Suppose the theorem is false and choose a counterexample minimizing $\card{Q}$.
Let $k = \max\set{\ceil{\chi_f(Q)}, \Delta(G) + 1, \ceil{\frac{5\Delta(Q) +
3}{6}}}$. Say $Q = L(G)$ for a graph $G$. The minimality of $Q$ implies that
$G$ is critical and $\chi(Q) = k+1$, for some $k \ge \Delta(G) + 1$.

The heart of the proof is Claim~1, which roughly says that if $x$ is long, then
$d(x)<\frac34\Delta(G)$. Moreover, we can improve this bound further if $x$ is
the root of a long fan $F$ such that either (i) $F$ has length more than 3 or (ii)
some of the other vertices in $F$ have degree less than $\Delta(G)$.  The claims
thereafter are all essentially applications of Claim~1.
\bigskip

\claim{1}{Let $F$ be a fan rooted at $x$ with respect to a $k$-edge-coloring of
$G - xy$.  If $S\subseteq V(F)-x$ and $|S| \ge 3$, then
\[d(x) \le \frac1{5|S|-11}\parens{2|S|-12 + \sum_{v \in S} d(v)}.\]
In particular, if $|S|=3$, then $d(x) \le \frac1{4}\parens{-6 + \sum_{v \in S}
d(v)}.$}
\begin{claimproof}
Since $F$ is elementary, we have
\[2 + k-d(x) + \sum_{v \in S} k - d(v) \le k,\]
so
\[2 + |S|k \le d(x) + \sum_{v \in S} d(v).\]
Using $k \ge \frac56(\Delta(Q) + 1) - \frac13 \ge \frac56(d(x) + d(v) - \mu(xv))
- \frac13$ for each $v
\in S$, we get
\[2 + \sum_{v \in S}\parens{\frac56(d(x) + d(v) - \mu(xv)) -\frac13} \le d(x) +
\sum_{v \in S} d(v),\]
so multiplying by 6 and rearranging terms gives
\[12 + \parens{5|S| - 6}d(x) - 2|S| \le \sum_{v \in S} 5\mu(xv) + \sum_{v \in S} d(v).\]
Now $\sum_{v \in S} \mu(xv) \le d(x)$, so this implies
\[12 + \parens{5|S| - 11}d(x) - 2|S| \le \sum_{v \in S} d(v).\]
Solving for $d(x)$ gives
\[d(x) \le \frac1{5|S|-11}\parens{2|S|-12 + \sum_{v \in S} d(v)},\]
and when $|S| = 3$, we get
$d(x) \le \frac14\parens{-6 + \sum_{v \in S} d(v)}.$
\end{claimproof}
\bigskip

\claim{2}{If $x \in V(G)$ is long, then $d(x) \le \frac34\Delta(G) - 1$.}

\begin{claimproof}
This is immediate from Claim 1, since $d(v)\le \Delta(G)$ for all $v\in S$.
\end{claimproof}
\bigskip

\claim{3}{If $x_1x_2 \in E(G)$ such that $x_1$ is $x_2$-long and $x_2$ is
$x_1$-long, % for all $i \in \irange{2}$, 
then
\[d(x_i) \le \frac23\Delta(G) -2 \text{ for all $i \in \irange{2}$.}\]}

\begin{claimproof}
By Claim 1, for each $i \in \irange{2}$,
\[d(x_i) \le \frac14\parens{-6 + \sum_{v \in S} d(v)} \le \frac14\parens{-6 + d(x_{3-i}) + 2\Delta(G)},\]
Substituting the bound on $d(x_{3-i})$ into that on $d(x_i)$ and simplifying
gives for each $i \in \irange{2}$,
\[d(x_i) \le -2 + \frac23\Delta(G).\]
\end{claimproof}

\bigskip

\claim{4}{If $x_1x_2, x_1x_3 \in E(G)$ such that $x_1$ is $\set{x_2,x_3}$-long, $x_2$ is $x_1$-long and $x_3$ is long, then 
\[d(x_1) \le -\frac85 + \frac35\Delta(G),\]
\[d(x_2) \le -\frac75 + \frac{13}{20}\Delta(G).\]}

\begin{claimproof}
By Claim 1, we have
\[d(x_1) \le \frac14\parens{-6 + \sum_{v \in S} d(v)} \le \frac14\parens{-6 + d(x_2) + d(x_3) + \Delta(G)},\]
\[d(x_2) \le \frac14\parens{-6 + \sum_{v \in S} d(v)} \le \frac14\parens{-6 + d(x_1) + 2\Delta(G)}.\]
By the same calculation as in Claim~3, these together imply
\[d(x_1) \le -2 + \frac25\Delta(G) + \frac{4}{15}d(x_3).\]
Since $x_3$ is long, using Claim 2, we get
%\[d(x_1) \le -\frac85 + \frac35\Delta(G),\]
\[d(x_1) \le -\frac{34}{15} + \frac35\Delta(G),\]
and hence
%\[d(x_2) \le -\frac75 + \frac{13}{20}\Delta(G).\]
\[d(x_2) \le -\frac{61}{15} + \frac{13}{20}\Delta(G).\]
\end{claimproof}
\bigskip

\claim{5}{The theorem is true.}

\begin{claimproof}
Let $(T, v_0v_1, \vph) \in \T(G)$. By Lemma \ref{MasterHelper}, one of the following holds:
\begin{enumerate}
\item $G$ is elementary; or
\item $G$ is not $k$-thin; or
\item $\nu(T) = 3$ and $V(T)$ contains vertices $x_1,x_2,x_3$ such that $x_1$
is $x_2$-long, $x_2$ is $x_1$-long, $x_2$ is $x_3$-long, and $x_3$ is $x_2$-long; or
\item $\nu(T) = 3$ and $V(T)$ contains vertices $x_1,x_2,x_3$ such that $x_1$ is
$\set{x_2,x_3}$-long, $x_2$ is $x_1$-long, and $x_3$ is long; or
\item $V(T)$ contains four long vertices $x_1, x_2, x_3, x_4$.
\end{enumerate}

%If (1) holds, then $k + 1 = \ceil{\chi_f(Q)} \le k$, a contradiction.
If (1) holds, then $\chi(Q) = \ceil{\chi_f(Q)}$, which contradicts our choice of
$Q$ as a counterexample.

If (2) holds, then Claim 2 implies that $\mu(G) \ge 2k - \frac32\Delta(G) + 2$. 
Now Theorem \ref{CriticalMuBound} gives
\begin{align*}
k + 1 &\le \Delta(Q)+1-\frac{2k-\frac32\Delta(G)+2}2\\
&=\Delta(Q) + 1 - k +
\frac34\Delta(G) - 1,
\end{align*}
so
\[2(k + 1) \le \Delta(Q) + 1 + \frac34\Delta(G).\]

Substituting $\Delta(G) \le k$ and solving for $k$ gives

\[k  \le \frac45\Delta(Q) - \frac45 < \frac56\Delta(Q)+\frac12 \le k,\]
which is a contradiction.

Suppose (3) holds.  
Now \[2 + \sum_{i \in \irange{3}} k - d(x_i) \le k,\]
so Claim 3 implies
\[3\parens{\frac23\Delta(G) -2} \ge 2k+2,\]
which is a contradiction, since $\Delta(G) \le k$.

Suppose (4) holds.  Now
\[2 + \sum_{i \in \irange{3}} k - d(x_i) \le k,\]
so Claims 2 and 4 give
\[ \parens{\frac35 + \frac{13}{20} + \frac34}\Delta(G)-\parens{\frac{34}{15} + \frac{16}{15} + 1}\ge 2k+2,\]
which is
\[2\Delta(G) -\frac{13}3\ge 2k+2,\]
again a contradiction, since $\Delta(G) \le k$.


So (5) must hold.  But now
\[2 + \sum_{i \in \irange{4}} k - d(x_i) \le k,\]
so using Claim 2 gives
\[4\parens{\frac34\Delta(G) - 1} \ge 3k+2,\]
a contradiction since $\Delta(G) \le k$.
\end{claimproof}

This finishes the final case of Claim~5, which proves the theorem.
\end{proof}

In the previous theorem, we showed that 
$\chi(Q) \le \max\set{\ceil{\chi_f(Q)}, \Delta(G) + 1, \ceil{\frac{5\Delta(Q) +
3}{6}}}$.  Now we show that if the maximum is attained by the second argument,
then $G$ satisfies the $\frac56$-Conjecture.
We use the following lemma, which is implicit in~\cite{rabern2011strengthening}.

\begin{lem}\label{CriticalMuBoundOtherWay}
If $Q$ is the line graph of a graph $G$ and $Q$ is vertex critical, then
\[\chi(Q) \leq \max\left\{\Delta(G), \Delta(Q) + 1 + 2\mu(G) - \Delta(G)\right\}.\]
\end{lem}
\begin{proof}
The fan equation implies this (see the proof in strengthening Brooks paper).
\end{proof}

%To take care of the $k = \Delta(G)$ case. 

\begin{cor}
If $Q$ is the line graph of $G$, then
\[\chi(Q) \le \max\set{\ceil{\chi_f(Q)}, \frac56\Delta(Q)+ \frac43}.\]
\label{mainCor}
\end{cor}
\begin{proof}
Since $\ceil{\frac56\Delta+\frac36}\le\frac56\Delta+\frac86$, the bound follows
directly from Theorem~\ref{mainhelper} unless we have $k +1=\chi'(G)=
\Delta(G)+1$.  So assume this is true.
Now Lemma \ref{CriticalMuBoundOtherWay} gives
\[k + 1 = \chi(Q) \le \Delta(Q) + 1 + 2\mu(G) - k,\]
so solving for $\mu(G)$ gives
\[\mu(G) \ge k - \frac{\Delta(Q)}{2}.\]
Applying Theorem \ref{CriticalMuBound} gives
\[k+1 = \chi(Q) \le \Delta(Q) + 1 - \frac{k - \frac{\Delta(Q)}{2} - 1}{2},\]
and solving for $k+1$ yields
\[k+1 \le \frac56\Delta(Q) + \frac43.\]
\end{proof}

\newpage

\section{With slack variables}
\begin{lem}
Let $G$ be a critical, elementary graph with $\chi'(G) = k + 1$ where $k \ge \Delta(G) + 1$.  Put $Q \DefinedAs L(G)$.  
If $k = \epsilon\parens{\Delta(Q) + 1} + \beta$, then for all $x \in V(G)$,
\[\card{N(x)} = \frac{\epsilon\parens{|G| - \Delta(G) - d_G(x) - 1 + S_1 + S_2 + S_3\parens{\card{G} - 1}}}{(1-\epsilon)\Delta(G) - \epsilon d_G(x) + 1 - \beta + S_3},\]
where 
\[S_1 \DefinedAs \sum_{v \in N(x)} \Delta(Q) - d_Q(xv),\]
\[S_2 \DefinedAs 2 + \sum_{v \in V(G) \setminus N(x)} \Delta(G) - d_G(v),\]
\[S_3 = k - (\Delta(G) + 1).\]
\end{lem}
\begin{proof}
Since $G$ is critical and elementary, $\card{G}$ is odd and
\begin{equation}\label{eq1}
k = \frac{2(\size{G} - 1)}{\card{G} - 1}.
\end{equation}
Let $x \in V(G)$, put $M \DefinedAs \card{N(x)}$ and 
\[P \DefinedAs \sum_{v \in N(x)} d_G(v).\] 
Then
\begin{equation}\label{eq2}
2(\size{G} - 1) = \Delta(G)(|G| - M) - S_2 + P.
\end{equation}
Since 
\[\frac{2(\size{G} - 1)}{\card{G} - 1} = k = \Delta(G) + 1 + S_3,\]
using \eqref{eq2}, we get
\[P = (|G| - 1)(\Delta(G) + 1 + S_3) - \Delta(G)(|G| - M) + S_2,\]
which is
\begin{equation}\label{eq3}
P = \Delta(G)(M-1) + |G| - 1 + S_2 + S_3(|G| - 1).
\end{equation}
Also, using $k = \epsilon\parens{\Delta(Q) + 1} + \beta$, we get
\[kM = \beta M + \epsilon S_1 + \epsilon\sum_{v \in N(x)} d_G(x) + d_G(v) - \mu(xv),\]
Since $\sum_{v \in N(x)} \mu(xv) = d_G(x)$,we have
\begin{equation}\label{eq4}
kM = \beta M + \epsilon S_1 + \epsilon d_G(x)(M - 1) + \epsilon P.
\end{equation}
Plugging \eqref{eq3} into \eqref{eq4} and solving for $M$ gives
\[M= \frac{\epsilon\parens{|G| - \Delta(G) - d_G(x) - 1 + S_1 + S_2 + S_3\parens{\card{G} - 1}}}{(1-\epsilon)\Delta(G) - \epsilon d_G(x) + 1 - \beta + S_3},\]
as desired.
\end{proof}

Using $\epsilon = \frac56$, we get the following.

\begin{lem}\label{Slacked56}
Let $G$ be a critical, elementary graph with $\chi'(G) = k + 1$ where $k \ge \Delta(G) + 1$.  Put $Q \DefinedAs L(G)$. 
If $k = \frac56\parens{\Delta(Q) + 1} + \beta$, then for all $x \in V(G)$,
\[\card{N(x)} = \frac{5\parens{|G| - \Delta(G) - d_G(x) - 1 + S_1 + S_2 + S_3\parens{\card{G} - 1}}}{\Delta(G) - 5 d_G(x) + 6(1 - \beta + S_3)},\]
where 
\[S_1 \DefinedAs \sum_{v \in N(x)} \Delta(Q) - d_Q(xv),\]
\[S_2 \DefinedAs 2 + \sum_{v \in V(G) \setminus N(x)} \Delta(G) - d_G(v),\]
\[S_3 = k - (\Delta(G) + 1).\]
\end{lem}


\begin{lem}\label{DegreeBoundedForMiddling}
Let $G$ be a critical, elementary graph with $\chi'(G) = k + 1$ where $k \ge \Delta(G) + 1$.  Put $Q \DefinedAs L(G)$. 
If $k = \frac56\parens{\Delta(Q) + 1} + \beta$ where $\beta \ge -\frac13$, then for all $x \in V(G)$ with $\card{N(x)} \ge 3$,
\[d_G(x) \le \frac{\card{N(x)}}{5\parens{\card{N(x)} - 2}}\Delta(G) - \frac{1}{\card{N(x)} - 2}\sum_{v \in V(G)\setminus N[x]} \Delta(G) - d_G(v).\]
\end{lem}
\begin{proof}
Say $|N(x)| = 2 + S_4$ for some $S_4 \ge 1$.  Applying Lemma \ref{Slacked56} and simplifying using $S_1 \ge 0$ and $\beta \ge -\frac13$ gives
\begin{equation}\label{longeq}
(5+5S_4)d_G(x) \le (7 + S_4)\Delta(G) - 5|G| + 21 + S_3(-5|G| + 17 + 6S_4) + 8S_4 - 5S_2.
\end{equation}
Put 
\[t \DefinedAs \sum_{v \in V(G) \setminus N[x]} \Delta(G) - d_G(v).\]
Then $S_2 = t + 2 + \Delta(G) - d_G(x)$.  Using this in \eqref{longeq}, we get
\begin{equation}\label{longeq2}
5S_4d_G(x) \le (2 + S_4)\Delta(G) - 5|G| + 11 + S_3(-5|G| + 17 + 6S_4) + 8S_4 - 5t.
\end{equation}
The desired bound follows when $S_4 \le \frac58\card{G} - 2$, since then
\[- 5|G| + 11 + S_3(-5|G| + 17 + 6S_4) + 8S_4 \le 0.\]


So, suppose $S_4 > \frac58\card{G} - 2$. Rearranging, we get
\begin{equation}\label{longeq2}
5S_4d_G(x) \le 3S_4\Delta(G) - 5|G| + 11 + S_3(-5|G| + 17 + 6S_4) + 8S_4 - (2S_4-2)\Delta(G)
\end{equation}
We know that
\[- 5|G| + 10 + 5S_4 +S_3(-5|G| + 10 + 5S_4) \le 0,\]
so
\begin{equation}\label{longeq2}
5S_4d_G(x) \le 3S_4\Delta(G) + 1 + 7S_3 + (S_3 + 3)S_4 - (2S_4-2)\Delta(G)
\end{equation}
Since $d_G(x) \ge |N(x)| \ge \frac58|G| \ge \frac58\Delta(G)$, we have a contradiction unless
\[1 + 7S_3 + (S_3 + 3)S_4 - (2S_4-2)\Delta(G) > 0\]
By Shannon's bound $S_3 \le \frac{\Delta(G)}{2}$, so 
\[1 + \parens{\frac72 + 2}\Delta(G) + \frac{\Delta(G)+6}{2}S_4 - 2S_4\Delta(G) > 0,\]
which is
\[1 + \parens{\frac72 + 2}\Delta(G) > \frac{3\Delta(G)-6}{2}S_4,\]
so
\[\frac58|G| - 2 < S_4 < \frac{11\Delta(G)+2}{3\Delta(G)-6}.\]

\end{proof}

\begin{cor}
Let $G$ be a critical, elementary graph with $\chi'(G) = k + 1$ where $k \ge \Delta(G) + 1$.  Put $Q \DefinedAs L(G)$. 
If $k = \frac56\parens{\Delta(Q) + 1} + \beta$ where $\beta \ge -\frac13$, then there are at most two $x \in V(G)$ with $\card{N(x)} \ge 3$
and if there are two $x_1, x_2$, then $\card{N(x_1)} = \card{N(x_2)} = 3$ and $x_1 \adj x_2$.
\end{cor}
\begin{proof}
Since $G$ is critical and elementary, $\card{G}$ is odd and
\[\frac{2(\size{G} - 1)}{\card{G} - 1} = k \ge \Delta(G) + 1,\]
so
\[2\size{G} \ge \Delta(G)\card{G} + \card{G} - \Delta(G) + 1.\]
In particular,
\[\sum_{v \in V(G)} \Delta(G) - d_G(v) \le \Delta(G) - 1 - \card{G}.\]
By Lemma \ref{DegreeBoundedForMiddling}, every $x \in V(G)$ with $3 \le \card{N(x)} \le \frac58\card{G}$ has $d_G(x) \le \frac35\Delta(G)$, so there are at most two such $x$ since
$\frac25 + \frac25 + \frac25 > 1$.

Suppose there are two such $x_1, x_2$ with $x_1 \nonadj x_2$.  If $\card{N(x_1)} \ge 4$, then Lemma \ref{DegreeBoundedForMiddling} gives $d_G(x_1) \le \frac25\Delta(G)$ which is impossible because $d_G(x_2) \le \frac35\Delta(G)$.  So, we must have $\card{N(x_1)} = \card{N(x_2)} = 3$.  Since $x_1 \nonadj x_2$, Lemma \ref{DegreeBoundedForMiddling} gives for $i \in \irange{2}$,
\[d_G(x_i) \le \frac35\Delta(G) - (\Delta(G) - d_G(x_{3-i})),\]
so
\[d_G(x_i) - d_G(x_{3-i}) \le -\frac25\Delta(G),\]
so $d_G(x_1) < d_G(x_2) < d_G(x_1)$, a contradiction.
\end{proof}

\section{The $\frac56$-Conjecture}
\label{sec:56}
%\begin{comment}
\begin{lem}
\label{56helper}
Let $H$ be an elementary multigraph and let $J$ be the underlying simple graph.
If $H$ is a minimal counterexample to the $\frac56$-Conjecture, then
$J$ is the a subdivision of (i) a cycle with a chord, 
(ii) a copy of $K_4$,
(iii) a cycle with two
chords with disjoint sets of endpoints, or 
(iv) a cycle with two chords with a common endpoint.
\end{lem}
We assume that $H$ is elementary, and that $H$ is a minimal counterexample to
the $\frac56$-Conjecture.  We will show that if a vertex $v\in V(H)$ has more
than two neighbors, then $d(v)$ is small.  In particular, if $|N(v)|\ge 3$, then
$d(v)<$ and if $N(v)|\ge 4$, then $d(v)<$.  An immediate consequence of the
first inequality is that $H$ has at most four vertices $v$ with $|N(v)|\ge 3$.
%We are attempting to prove Goldberg implies $\frac56$ bound. 
%
%Assume Goldberg.

Let $H$ be a minimal counterexample to $\frac56$-conjecture. Then $\chi'(H) \ge
\Delta(H) + 2$, so by Goldberg we have $\chi'(H) = \ceil{\W(H)} >
\ceil{\frac{5\Delta(L(H)) + 3}6}$.  Suppose $Q \subseteq V(H)$ achieves max in
$\ceil{\W(H)}$.  If there is $e \in E(H) \setminus E(Q)$, then removing it we get
$\chi'(H-e) < \chi'(H)$, but $\ceil{\W(H-e)} = \ceil{\W(H)}$, so  $\chi'(H) >
\chi'(G-e) \ge \ceil{\W(H-e)} =  \ceil{\W(H)} = \chi'(H)$, a contradiction. Hence $Q
= H$.

Choose $x \in H$, and let $M = |N(x)|$ and $p = \sum_{v \in N(x)} d_H(v)$.

\begin{align}
\W(H) & = \ceil{\frac{2||H||}{|H| - 1}} \\
& \le \frac{2||H|| + |H| - 3}{|H|-1}  \nonumber \\
& \le \frac{\Delta(H)(|H| - M) + p + |H| - 3}{|H|-1} \nonumber
\end{align}

Also, summing $\W(H) > \frac56(\Delta(G) + 1)$ over all $v \in N(x)$, and using
$\sum_{y\in N(x)}\mu(xy)\le d(x)$, gives

\begin{align}
\W(H)M & >  \frac56\sum_{y\in N(x)}d(x)+d(y)-\mu(xy) \nonumber \\
& \ge \frac56((M-1)d(x) + p).
\end{align}

So, with (1),

\begin{align*}
 \frac56((M-1)d(x) + p) <  M\frac{\Delta(H)(|H| - M) + p + |H| - 3}{|H|-1}.
\end{align*}

Thus

\begin{align}
\frac{(\Delta(H)(|H| - M) + |H| - 3)M}{|H| - 1} - \frac56(M-1)d(x) > \left(\frac56 -
\frac{M}{|H| - 1}\right)p
\end{align}

Since,
$\ceil{\W(H)} = \chi'(H) \ge \Delta(H) + 2$, we have
$\frac{2||H||}{|H| - 1} + 1 \ge \Delta(H) + 2$ (here the `+1' on the left comes
from the ceiling).
%actually only need a bit less)
So,

\begin{align*}
2||H|| \ge |H|\Delta(H) + |H| - 1 - \Delta(H),
\end{align*}
which implies that

\begin{align}
p \ge (M-1)\Delta(H) + |H| - 1.
\end{align}

Now substituting (4) into (3) gives

\begin{align*}
\frac{(\Delta(H)(|H| - M) + |H| - 3)M}{|H| - 1} - \frac56(M-1)d(x) > \left(\frac56 -
\frac{M}{|H| - 1}\right)((M-1)\Delta(H) + |H| - 1)
\end{align*}

\begin{align*}
\frac{(\Delta(H)(|H| - M) + |H| - 3)M}{|H| - 1} + M\frac{(M-1)\Delta(H) %+ |H| - 1
}{|H| - 1}+M > \frac56(M-1)(d(x) + \Delta(H)) + \frac56(|H| - 1)
\end{align*}

\begin{align*}
M(\Delta(H) + 2) - \frac{2M}{|H| - 1} &>  \frac56(M-1)(d(x) + \Delta(H)) + \frac56(|H|
- 1) \\
&= \frac56M(d(x) + \Delta(H)) - \frac56(d(x) + \Delta(H)) + \frac56(|H| - 1)\\
\left(\frac{\Delta(H)}6 - \frac56d(x) + 2\right)M &> \frac{2M}{|H| - 1} + 
\frac56(|H| - 1) - \frac56(d(x) + \Delta(H))\\
(\Delta(H) - 5d(x) + 12)M &> \frac{12M}{|H| - 1} +  5(|H| - 1) - 5(d(x) +
\Delta(H))\\ & > 5(|H| - 1) - 5(d(x) + \Delta(H))\\
M &< 5\frac{d(x) + \Delta(H) + 1 - |H|}{5d(x) - \Delta(H) - 12}
\end{align*}
The final inequality follows from the previous one because
$\Delta(H)-5d(x)+12<0$, since: WHY???\\
We want to find bounds on $d(x)$ that ensure $M$ is small.  To this end, we
write $M<2+Y$, for some expression $Y$ and solve $Y<1$ to get the desired bounds
on $d(x)$.

\begin{align*}
M &< \frac{10d(x) - 2\Delta(H) - 24}{5d(x) - \Delta(H) - 12} + \frac{7\Delta(H) -
5d(x) + 29 - 5|H|}{5d(x) - \Delta(H) - 12} \\
&= 2 + \frac{7\Delta(H) - 5d(x) + 29 - 5|H|}{5d(x) - \Delta(H) - 12}.
\end{align*}
So $M\le 2$ when

\begin{align*}
7\Delta(H) - 5d(x) + 29 - 5|H| < 5d(x) - \Delta(H) - 12
\end{align*}
which simplifies to

\begin{align*}
d(x) > \frac{8\Delta(H) + 41 - 5|H|}{10}.
\end{align*}



\bigskip
\bigskip

I think we should be able to prove that the conjecture follows from
Goldberg--Seymour.
That lemma you proved is pretty useful.  We can assume that H is
critical, which implies that $|N(x)| \ge 2$ for all $x$ in $H$.  Now let $J$
be the simple graph underlying $H$.  We know that $\delta(J) \ge 2$.  Let
$B = \{ x \in H s.t. d_J(x) \ge 3\}$.  That lemma implies that $|B| \le 4$.
Further, if $|B| = 4$, then each vertex of $B$ has degree 3 in $J$.  If
$|B|=3$, then two vertices of $B$ have degree 3 in $J$ and one has degree 4
in $J$.  Otherwise $|B| \le 2$.  Now if $J$ has a vertex $x$ of degree at
least 5, and $|B|=2$, then the other vertex in $B$ has degree 3 in $J$.  Now
$x$ must be a cut-vertex (since $J$ is formed by identifying one vertex in
multiple disjoint cycles, exactly one of which has a chord).  But a
cut-vertex in $J$ is also a cut-vertex in $H$, which is a contradiction.
Thus, we only need consider the cases when $|B|=3$ and $|B|=4$, which have
degree sequences $3,3,3,3,2,\ldots2$. and $4,3,3,2,\ldots,2$.
$|B|=4$ is a subdivided $K_4$ or a subdivision of a 4-cycle where one
matching has multiplicity 2.
$|B|=3$ is a subdivision of a triangulated 5-cycle.  I haven't worked
out those cases, but I don't think they should be too hard.

\begin{lem}
If $H$ %is an elementary graph that 
is a minimal counterexample to the $\frac56$-Conjecture, then $H$ has no
path of three or more vertices of degree two.
\label{no3path}
\end{lem}
\begin{proof}[Proof Ideas]
%So a natural thing to try is to rule out long paths of 2-vertices.  
That lemma from strengthening Brooks is
helpful because it shows that $\mu(G) \le \Delta(G)/3$.  Using that I
showed that $G$ can't have any path of 3 or more 2-vertices.  For 4 or
more, I think you can just delete those edges, then extend the
coloring.  For exactly 3, it looks like we can contract two of those
edges on the path (then possibly reduce the multiplicity of the
remaining edges on the path, so we don't increase $\Delta(G)$). 
From that coloring, it seems like we can extend fairly easily
(after possibly uncoloring some of them).
\end{proof}

\begin{lem}
If $H$ is an elementary multigraph that is critical, then $|V(H)|$ is odd.
\label{elem-odd}
\end{lem}
\begin{proof}
Proof is from Scheinerman and Ullman, Section 4.2.  The basic idea is to show
that if $|V(H)|$ is odd, and $v\in V(H)$ with $d(v)=\delta(H)$, then
$\chi'(H-v)\ge \chi'(H)$, so $H$ is not critical.
\end{proof}

\begin{thm}
The $\frac56$-Conjecture is true.  That is, if $Q$ is the line graph of a
multigraph, then $\chi(Q)\le \max\{\omega(Q),\frac{5\Delta(Q)+8}6\}$.
\end{thm}
\begin{proof}
Suppose the theorem is false and let $H$ be a minimal counterexample.  By
Corollary~\ref{mainCor}, we know that $H$ is elementary.  By
Lemma~\ref{no3path}, we also know that the underlying simple graph $J$ of $H$
has no path of three or more consecutive 2-vertices.
By Lemma~\ref{56helper}, we know that $J$ is a subdivision of 
(i) a cycle with a chord, (ii) a copy of $K_4$, (iii) a cycle with two chords
with disjoint sets of endpoints, or (iv) a cycle with two chords with a common
endpoint.  We consider these four possibilities in succession.  In each case we
find a small dominating set $S$ in the line graph $Q$ of $H$ and sum
$\frac56(1+d_Q(v))$ over all $v\in S$ to show that
$\chi(Q)=\ceil{\frac{2|E(H)|}{|V(H)|-1}}\le \ceil{\frac{5\Delta(Q)+3}6}$.

(i) Suppose that $H$ is a subdivision of a cycle with a chord, i.e., $J$
consists of two 3-vertices joined by three internally disjoint paths, each
with length between 1 and 3 (by Lemma~\ref{no3path}).  
%Call these 3-vertices $v_1$ and $v_2$, and 
Denote these three paths by $A$, $B$, and $C$.  By
symmetry, assume that $|A|\le |B|\le |C|$.  A priori, we have ${5\choose 3}$
possibilities for the lengths of $A$, $B$, and $C$.  However, by
Lemma~\ref{elem-odd}, we know that $|V(H)|$ is odd, so $|A|+|B|+|C|$ is even.  
Thus, $(|A|,|B|,|C|)\in\{(1,1,2),(1,2,3),(2,2,2),(2,3,3)\}$.
The first case is trivial, since $Q$ is a clique.  So assume $|A|=1$, $|B|=2$,
and $|C|=3$.  By considering an initial edge $e_1$ and a final edge $e_2$ of
$C$, we get $2(\frac56(\Delta(Q)+1))\ge
\frac56((d_Q(e_1)+1)+(d_Q(e_2)+1))\ge\frac56(|E(H)|+2)$. Now
$\frac56(\Delta(Q)+1)\ge\frac5{12}(|E(H)|+2)$, which gives
$\frac{5\Delta(Q)+3}6>\frac{|E(H)|}2$, so taking ceilings gives the desired
bound on $\chi'(H)$.  When $|A|=|B|=|C|=2$, essentially the same argument works,
but now we consider an initial edge on $A$ and a final edge on $B$.
Finally, suppose $|A|=2$ and $|B|=|C|=3$.  Now we take an initial edge on $B$
and a final edge on $C$.  This gives $2(\frac56(\Delta(Q)+1))\ge\frac56|E(H)|$, so
$\frac{5\Delta(Q)+3}6\ge\frac5{12}|E(H)|-\frac13\ge\frac13|E(H)|$, since
$|E(H)|\ge 4$.  Again, taking ceilings gives the desired bound on $\chi'(H)$.


Before considering further possibilities for $J$ we prove the
following claim, which generalizes the approach we took in (i) above.
We use this claim repeatedly in the rest of the proof.

\begin{clm}
Let $S$ be a subset of $j$ vertices in $Q$ such that each vertex $w$ of $Q$ appears in
the closed neighborhood of at least $k$ vertices in $S$ 
%i.e. $\sum_{v\in S}|\{w\}\cap N[v]|\ge k$ 
and such that $\sum_{v\in S}|N[v]|\ge k|V(Q)|+\ell$, for some positive integers
$j$, $k$, $\ell$.  Now $H$ satisfies the $\frac56$-Conjecture when $\frac{5k}{6j}\ge
\frac2{|V(H)|-1}$ and $|E(H)|(\frac{5k}{6j} - \frac2{|V(H)|-1}) \ge
\frac56\left(\frac25 - \frac{l}j\right)$.  In particular, this is true when
$\frac{5k}{6j}\ge \frac2{|V(H)|-1}$ and $\frac{\ell}{j}\ge\frac25$. 
\end{clm}
\begin{claimproof}
The second statement follows from the first, since the left side is nonnegative
and the right side is nonpositive.  Now we prove the first statement.
We have 
\begin{align*}
j\frac56(\Delta(Q)+1) &\ge \sum_{v \in S}\frac56(d_Q(v)+1)\\ 
& = \frac56(k|V(Q)|+\ell) \\ 
&= \frac{5k}6|E(H)|+\frac{5\ell}6\\
\frac{5\Delta(Q)+3}6 &\ge
\frac{5k}{6j}|E(H)|+\frac{5}6\left(\frac{\ell}{j}-\frac25\right).
\end{align*}
To show that $H$ satisfies the $\frac56$-Conjecture,
it suffices to show that the final expression is at least
$\frac{2|E(H)|}{|V(H)|-1}$, and this follows immediately from the hypothesis.
\end{claimproof}


(ii) and (iii) Suppose that $J$ is a subdivision of $K_4$ or a subdivision
of a cycle with two chords with no common endpoint; for brevity, we write these
two cases as $K'_4$ and $C'_4$.  Now we have
$|V(H)|\in\{15,13,11,9,7,5\}$.  Note that always $|E(J)|=|V(J)|+2$.  Let $T$
denote the four vertices of degree 3 in $J$.
It is useful to observe that if at most four of the paths in $J$ with endpoints in
$T$ have length 3, then $Q$ has a dominating set of size four.  For reference,
we call this Fact 1.  The idea of the proof is to
choose an edge incident to each vertex in $S$ such that we choose one edge from
each path of length 3.  It is straightforward to check that we can do this (up
to symmetry, when $J$ is $K'_4$ we have only two possibilities for the paths of
length 3 and when $J$ is $C'_4$ we have four possibilities).

Suppose $|V(H)|=15$.  Choose an arbitrary middle edge $e$ of some path of length
3 and contract it. By Fact~1, $Q-e$ has a dominating set $S$ of size 4, so
$S\cup\{e\}$ is a dominating set of size 5 with $\sum_{v\in
S\cup\{e\}}|N_Q(v)|\ge |V(Q)|+2$.  Thus, we are done by Claim 1, with $j=5$,
$k=1$, and $\ell=2$, since $\frac2{|V(H)|-1}=\frac17$.

Suppose $|V(H)|=13$.  By Fact~1, $Q$ has a dominating set of size 4.
We use Claim~1 with $j=4$, $k=1$, and $\ell\ge 0$.  Cleary, $\frac5{24}\ge
\frac16$.  Also $|E(H)|\ge \frac13/(\frac{5}{24}-\frac16)=8$, so we are
done.

Suppose $|V(H)|=11$. Again, by Fact~1, $Q$ has a dominating set of size 4.
%we get $4(\frac56(\Delta(Q)+1)\ge\frac56(|E(H)|+3)$, which is
Now we use Claim~1 with $j=4$, $k=1$, and $\ell\ge 3$.
(That $\ell\ge3$ comes from the fact that $|E(J)|=13$ but the sum of sizes
of closed neighborhoods in $Q$ is at least $4(4)=16$.)  

Suppose $|V(H)|=9$.
Now $|E(H)|=9+2=11$. So among paths in $J$ joining vertices of $T$, the number
with length 1 is one more than that with length 3.  In particular, at least one
such path has length 1 and at most two have length 3.
We use these observations to find a dominating set $S$ in $Q$ of size 3.
We choose one edge $e$ on a path of length 1, say with endpoints $v_1$ and $v_2$,
where $T=\{v_1,v_2,v_3,v_4\}$.  Now we must also choose edges incident to $v_3$
and $v_4$ such that we choose one edge from each path of length 3.  To succeed,
we must choose $e$ that (a) does not form a triangle with two paths of length 3,
when $J=K'_4$, (b) does not share both endpoints with a path of length 3, when
$J=C'_4$, and (c) does not share a common endpoint $v_i$ with two paths of
length 3, when $J=C'_4$.  That we can choose such an $e$ follows from the fact
that $J$ has more paths of length 1 than of length 3.  Now we apply Claim~1 with
$j=3$, $k=1$, and $\ell\ge 2$.  (Note that $\ell\ge 5+4+4-|E(H)|=2$, since the
size of each closed neighborhood in $Q$ of an edge in $S$ is at least 4 and that
of $e$ is 5.)

Suppose $|V(H)|=7$.  Now the lengths of the paths, with multiplicity, are either
(a) $2,2,2,1,1,1$ or (b) $3,2,1,1,1,1$.  First assume that $J=C'_4$.  
Note that no two paths of length 1 can share both endpoints, since $J$ would be
non-simple, which is a contradiction.  
Suppose that we are in (a).  Now there
exist two paths of length 1 that do not share any
endpoints, so the edges of these paths form a dominating set in $Q$; we
apply Claim~1 with $j=2$, $k=1$, and $\ell=0$ (using that $|E(H)|\ge 4$).
%If no such two paths of length 1 exist, then two paths of 
%length 2 share both endpoints and two paths of 
%length 1 share both endpoints, so $J$ is not simple, which is a contradiction.
So now suppose that we are in (b).  Now $J$ contains a 7-cycle, which consists
of the edges of all paths except for two of length 1.  Now taking $S$ to be the
edges of the 7-cycle, we apply Claim~1 with $j=7$, $k=3$, and $\ell\ge 2$
(since each path of length 1 not in the 7-cycle has four incident edges on that
cycle).  This concludes the case $J=C'_4$.

Now assume that $J=K'_4$.  
%Suppose that the path lengths are as in (a).
%So instead the path lengths are as in (b).  
First suppose that $J$ contains a 7-cycle such
that the two edges excluded from it each have four incident edges on the cycle.
In this case we, we apply Claim~1 with $j=7$, $k=3$, and $\ell=2$.  This covers
all of (b), as well as the case of (a) when the three paths of length 2 together
form a path of length 6.  So assume instead that we are in (a) and either the
three paths of length 2 have a common endpoint or they form a 6-cycle.
In the former case, let $S$ consist of the three edges of these paths not
incident to their common endpoint.
%In the former case, let $S$ consist of three independent edges of this 6-cycle.
In the latter case, let $S$ be the edges of the three paths of length 1.
In each case we apply Claim~1 with $j=3$, $k=1$, and $\ell=3$.

Finally, suppose that $|V(H)|=5$.  As above, no two paths of length 1 can share
both endpoints, since $J$ would be non-simple, which is a contradiction.  
Thus, we must have $J=K'_4$.  Let $e$ be the edge of $J$ with no endpoint in
common with the path of length 2, and let $S=E(J)-e$.  Now we apply Claim~1 with
$j=6$, $k=4$, and $\ell=0$ (and using $|E(H)|\ge 2$).

(iv) Suppose that $H$ is a subdivision of a cycle with two chords with a common
endpoint.
In $J$, let $u_1$, $u_2$, $u_3$ denote the vertices of degree greater than two,
and assume $d(u_1)=4$ and $d(u_2)=d(u_3)=3$.  Let $A$ and $B$ denote paths from
$u_1$ to $u_2$, let $C$ and $D$ denote paths from $u_1$ to $u_3$, and let $E$
denote a path from $u_2$ to $u_3$.  Recall that each path has length at most 3.
Trivially, we can assume that $|V(H)|>3$, since otherwise the line graph of $H$
is a clique.  Thus, by Lemma~\ref{elem-odd}, we know that
$|V(H)|\in\{13,11,9,7,5\}$.
Let $v_{a_1}$ denote a vertex of $Q$ corresponding to a first edge on $A$ in
$J$; define $v_{b_i},\ldots,v_{e_i}$ analogously.  

Suppose $|V(H)|=13$.
%By symmetry, we assume that $|D|=3$, $|C|\ge 2$, and $|B|\ge 2$.
Let $S=\{v_{a_1},v_{b_2},v_{c_2},v_{d_3},v_{e_1}\}$ and note that $S$ 
is a dominating set in $Q$.  Furthermore, 
%when $|V(H)|=15$,  
both $v_{b_3}$ and $v_{c_3}$ have two neighbors in $S$, so $\sum_{v\in
S}|N[v]|\ge |V(Q)|+2=|E(H)|+2$.  
%The same bound holds when $|V(H)|=13$.  
Thus,
$5(\frac56(\Delta(Q)+1)\ge\frac56(|E(H)|+2)$, so $\frac{5\Delta(Q)+3}6 \ge
\frac{|E(H)|}6$, which implies $\ceil{\frac{5\Delta(Q)+3}6}\ge
\ceil{\frac{|E(H)|}6}\ge\chi(Q)$, as desired.

Suppose $|V(H)|\in\{11,9\}$.  First assume that $J$ has four paths of length
3; since $|V(H)|\le 11$, the fifth path has length 1.
By considering an initial and a final edge on each path of length 3, we have
$8(\frac56(\Delta(Q)+1))\ge \frac56(2|E(H)|+2)$, so
$\frac{5\Delta(Q)+3}6\ge\frac5{24}|E(H)|-\frac3{24}\ge\frac{|E(H)|}5$, since
$|E(H)|\ge 15$.  So assume instead that $J$ has at most three paths of length 3.

Now we show that $Q$ has a dominating set of size 3.  Form a bipartite
graph $\B$ with $u_1$, $u_2$, $u_3$ as one part and the paths of length three as
the other part.  It is easy to check that $\B$ has a matching saturating the
paths of length three.  This matching corresponds to a subset of $Q$.  If the
subset has size less than 3, then extend it to include one edge of $H$ incident to
each of $u_1$, $u_2$, $u_3$.  Now, $3(\frac56(\Delta(Q)+1))\ge \frac56|E(H)|$,
so $\frac{5\Delta(Q)+3}6\ge \frac5{18}|E(H)|-\frac13\ge\frac14|E(H)|$, so taking
ceilings yields the desired bound (the final inequality holds since $|E(H)|\ge
12$, because $\frac{|E(H)|}{\floor{\frac{|V(H)|}2}}>\Delta(H)\ge d(u_1)=4$).
%So now we assume that $|V(H)|=7$.
\end{proof}

%I think I see mostly how to finish the proof.  I just worked out most
%of the details for subdivisions of cycle+chord and cycle+2chords.
%Basically, the idea is to find small dominating sets in the line
%graph.  For the cycle+2chords, in the line graph we can always find a
%dominating set of size at most 5.  This uses the fact that in the
%underlying simple graph, we never have a path of 3 or more successive
%2-vertices, like I wrote earlier.  In the case where H actually has
%13 vertices (the most possible), we end up with $5(5/6)(\Delta+1) \ge
%(5/6)|E(H)|$.  Then canceling the 5, and taking the ceiling, we get the
%desired bound (actually, we need to argue that we can lower the left
%side by like 1/3 before taking the ceiling, but that is not hard).
%The case of cycle+chord should be even easier.  Still need to think
%about the case of $K_4$ with a $2K_2$ of multiplicity 2.

%Using the idea in the proof of Theorem~12 (actually of Lemma~6 in ``Strengthening
%Brooks''), we can show that $J$ cannot have adjacent 2-vertices.  I also found a
%paper of Eggan and Plantholt (JCTB '86) that proves Goldberg--Seymour for graphs
%$G$ that have a $v$ such that $G-v$ is bipartite, which would cut down our
%search space.  But I don't think that's actually helpful, since we already are
%assuming Goldberg--Seymour.

\bibliographystyle{plain}
\bibliography{GraphColoring}

\end{document}

\documentclass[12pt]{article}
\usepackage{amsmath, amsthm, amssymb}
\usepackage{hyperref}
\usepackage{verbatim}
\usepackage[top=1.0in, bottom=1.0in, left=1.0in, right=1.0in]{geometry}
\usepackage{color}
\pagestyle{plain}

\usepackage{sectsty}
\allsectionsfont{\sffamily}

\setcounter{secnumdepth}{5}
\setcounter{tocdepth}{5}

\makeatletter
\newtheorem*{rep@theorem}{\rep@title}
\newcommand{\newreptheorem}[2]{
\newenvironment{rep#1}[1]{
 \def\rep@title{#2 \ref{##1}}
 \begin{rep@theorem}}
 {\end{rep@theorem}}}
\makeatother

\theoremstyle{plain}
\newtheorem{thm}{Theorem}[section]
\newreptheorem{thm}{Theorem}
\newtheorem{prop}[thm]{Proposition}
\newreptheorem{prop}{Proposition}
\newtheorem{lem}[thm]{Lemma}
\newreptheorem{lem}{Lemma}
\newtheorem{conjecture}[thm]{Conjecture}
\newreptheorem{conjecture}{Conjecture}
\newtheorem{cor}[thm]{Corollary}
\newreptheorem{cor}{Corollary}
\newtheorem{prob}[thm]{Problem}

\newtheorem*{KernelLemma}{Kernel Lemma}
\newtheorem*{BK}{Borodin-Kostochka Conjecture}
\newtheorem*{BK2}{Borodin-Kostochka Conjecture (restated)}
\newtheorem*{Reed}{Reed's Conjecture}
\newtheorem*{ClassificationOfd0}{Classification of $d_0$-choosable graphs}


\theoremstyle{definition}
\newtheorem{defn}{Definition}
\theoremstyle{remark}
\newtheorem*{remark}{Remark}
\newtheorem*{problem}{Problem}
\newtheorem{example}{Example}
\newtheorem*{question}{Question}
\newtheorem*{observation}{Observation}

\newcommand{\fancy}[1]{\mathcal{#1}}
\newcommand{\C}[1]{\fancy{C}_{#1}}


\newcommand{\IN}{\mathbb{N}}
\newcommand{\IR}{\mathbb{R}}
\newcommand{\G}{\fancy{G}}
\newcommand{\CC}{\fancy{C}}
\newcommand{\D}{\fancy{D}}
\newcommand{\T}{\fancy{T}}
\newcommand{\B}{\fancy{B}}
\renewcommand{\L}{\fancy{L}}
\newcommand{\HH}{\fancy{H}}

\newcommand{\inj}{\hookrightarrow}
\newcommand{\surj}{\twoheadrightarrow}

\newcommand{\set}[1]{\left\{ #1 \right\}}
\newcommand{\setb}[3]{\left\{ #1 \in #2 : #3 \right\}}
\newcommand{\setbs}[2]{\left\{ #1 : #2 \right\}}
\newcommand{\card}[1]{\left|#1\right|}
\newcommand{\size}[1]{\left\Vert#1\right\Vert}
\newcommand{\ceil}[1]{\left\lceil#1\right\rceil}
\newcommand{\floor}[1]{\left\lfloor#1\right\rfloor}
\newcommand{\func}[3]{#1\colon #2 \rightarrow #3}
\newcommand{\funcinj}[3]{#1\colon #2 \inj #3}
\newcommand{\funcsurj}[3]{#1\colon #2 \surj #3}
\newcommand{\irange}[1]{\left[#1\right]}
\newcommand{\join}[2]{#1 \mbox{\hspace{2 pt}$\ast$\hspace{2 pt}} #2}
\newcommand{\djunion}[2]{#1 \mbox{\hspace{2 pt}$+$\hspace{2 pt}} #2}
\newcommand{\parens}[1]{\left( #1 \right)}
\newcommand{\brackets}[1]{\left[ #1 \right]}
\newcommand{\DefinedAs}{\mathrel{\mathop:}=}

\newcommand{\mic}{\operatorname{mic}}
\newcommand{\AT}{\operatorname{AT}}
\newcommand{\col}{\operatorname{col}}
\newcommand{\ch}{\operatorname{ch}}
\newcommand{\type}{\operatorname{type}}
\newcommand{\nonsep}{\bar{S}}

\def\adj{\leftrightarrow}
\def\nonadj{\not\!\leftrightarrow}

\newcommand\restr[2]{{% we make the whole thing an ordinary symbol
  \left.\kern-\nulldelimiterspace % automatically resize the bar with \right
  #1 % the function
  \vphantom{\big|} % pretend it's a little taller at normal size
  \right|_{#2} % this is the delimiter
  }}

\def\D{\fancy{D}}
\def\C{\fancy{C}}
\def\A{\fancy{A}}

\newcommand{\case}[2]{{\bf Case #1.}~{\it #2}~~}

\title{Edge lower bounds via discharging notes}

\begin{document}
\maketitle

\section{Introduction}
For a graph $G$, let $d(G)$ be the average degree of $G$. Let $\T_k$ be the Gallai trees with maximum degree at most $k-1$, excepting $K_k$. 

\section{Gallai's bound via discharging}

\begin{thm}[Gallai]
	For $k \ge 4$ and $G \ne K_k$ a $k$-AT-critical graph, we have
	\[d(G) > k-1 + \frac{k-3}{k^2-3}.\]
\end{thm}
\begin{proof}
	Start with initial charge function $\ch(v) = d_G(v)$.  Have each $k^+$-vertex give charge $\frac{k-1}{k^2-3}$ to each of its $(k-1)$-neighbors.  Then let the vertices in each component of the low vertex subgraph share their total charge equally.  Let $\ch^*(v)$ be the resulting charge function.  We finish the proof by showing that $\ch^*(v) \ge k-1 + \frac{k-3}{k^2-3}$ for all $v \in V(G)$.
	
	If $v$ is a $k^+$-vertex, then $ch^*(v) \ge d_G(v) - \frac{k-1}{k^2-3}d_G(v) = \parens{1- \frac{k-1}{k^2-3}}d_G(v) \ge \parens{1- \frac{k-1}{k^2-3}}k = k-1 + \frac{k-3}{k^2-3}$ as desired.

	Let $T$ be a component of the low vertex subgraph.  Then the vertices in $T$ receive total charge
	\[\frac{k-1}{k^2-3}\sum_{v \in V(T)} k-1 - d_G(v) = \frac{k-1}{k^2-3}\parens{(k-1)|T| - 2\size{T}}.\]
	So, after distributing this charge out equally, each vertex in $T$ receives charge
	\[\frac{1}{|T|}\frac{k-1}{k^2-3}((k-1)|T| - 2\size{T}) = \frac{k-1}{k^2-3}\parens{(k-1) - d(T)}.\]
	By Lemma \ref{BasicGallaiTreeBound}, this is at least
	\[\frac{k-1}{k^2-3}\parens{(k-1) - \parens{k-2 + \frac{2}{k-1}}} = \frac{k-1}{k^2-3}\parens{\frac{k-3}{k-1}} = \frac{k-3}{k^2-3}.\]
	Hence each low vertex ends with charge at least $k-1 + \frac{k-3}{k^2-3}$ as desired.
\end{proof}


\begin{lem}[Gallai]\label{BasicGallaiTreeBound}
	For $k \ge 4$ and $T \in \T_k$, we have $d(T) < k-2 + \frac{2}{k-1}$.
\end{lem}
\begin{proof}
	Suppose not and choose a counterexample $T$ minimizing $|T|$.  Then $T$ has at least two blocks.  Let $B$ be an endblock of $T$.  If $B$ is $K_t$ for $2 \le t \le k-2$, then remove the non-separating vertices of $B$ from $T$ to get $T'$.  By minimality of $|T|$, we have 
	\[2\size{T} - t(t-1) = 2\size{T'} < \parens{k-2 + \frac{2}{k-1}}|T'| = \parens{k-2 + \frac{2}{k-1}}|T| - \parens{k-2 + \frac{2}{k-1}}(t-1).\]
	Hence we have the contradiction
	\[2\size{T} < \parens{k-2 + \frac{2}{k-1}}|T| + (t+2 -k - \frac{2}{k-1})(t-1) \le \parens{k-2 + \frac{2}{k-1}}|T|.\]
	
	The case when $B$ is an odd cycle is the same as the above, a longer cycle just makes things better.  Finally, if $B = K_{k-1}$, remove all vertices of $B$ from $T$ to get $T'$. By minimality of $|T|$, we have 
	\begin{align*}
	  2\size{T} - (k-1)(k-2) - 2 &= 2\size{T'}\\
	  &< \parens{k-2 + \frac{2}{k-1}}|T'|\\
	  &= \parens{k-2 + \frac{2}{k-1}}|T| - \parens{k-2 + \frac{2}{k-1}}(k-1).
	\end{align*}

	Hence $2\size{T} < \parens{k-2 + \frac{2}{k-1}}|T|$, a contradiction.
\end{proof}

\section{An initial improved bound}
Lemma \ref{BasicGallaiTreeBound} is best possible as can be seen by the family of graphs with blocks on a path alternating $K_{k-1}$ and $K_2$.  But we have reducible configurations (see the last section for the precise statements) that place restrictions on $K_{k-1}$ blocks. To state these restrictions, we need the following auxiliary bipartite graph. 

For a $k$-AT-critical graph $G$, let $\L(G)$ be the subgraph of $G$ induced on the $(k-1)$-vertices and $\HH(G)$ the subgraph of $G$ induced on the $k$-vertices.   For $T \in \T_k$, let $W^k(T)$ be the set of vertices of $T$ that are contained in some $K_{k-1}$ in $T$.  Let $\B_k(G)$ be the bipartite graph with one part $V(\HH(G))$ and the other part the components of $\L(G)$.  Put an edge between $y \in V(\HH(G))$ and a component $T$ of $\L(G)$ if and only if $N(y) \cap W^k(T) \ne \emptyset$.  Then Lemma \ref{MultipleHighConfigurationEuler} says that $\B_k(G)$ is $2$-degenerate.

We can use this fact to refine our discharging argument.  Let $\epsilon$ and $\gamma$ be parameters that we will determine where $\epsilon \le \gamma < 2\epsilon$. Start with initial charge function $\ch(v) = d_G(v)$.  
\begin{enumerate}
	\item Each $k^+$-vertex gives charge $\epsilon$ to each of its $(k-1)$-neighbors not in a $K_{k-1}$,
	\item Each $(k+1)^+$-vertex give charge $\gamma$ to each of its $(k-1)$-neighbors in a $K_{k-1}$,
	\item Let $Q = \B_k(G)$.  Repeat the following steps until $Q$ is empty.
	  \begin{enumerate}
	  	\item For each component $T$ of $\L(G)$ in $Q$ that has degree at most two in $Q$ do the following:
	  	    \begin{enumerate}
	  	    	\item For each $v \in V(\HH(G)) \cap V(Q)$ such that $\card{N_G(v) \cap W^k(T)} = 2$, pick one $x \in N_G(v) \cap W^k(T)$ and send charge $\gamma$ from $v$ to $x$,
	  	    	\item Remove $T$ from $Q$.
	  	    \end{enumerate}
	  	\item Pick $v \in V(\HH(G)) \cap V(Q)$ with degree at most two in $Q$.  Send charge $\gamma$ from $v$ to each $x \in N_G(v) \cap W^k(T)$ for each component $T$ of $\L(G)$ where $vT \in E(Q)$.
	  	\item Remove $v$ from $Q$.
	  \end{enumerate}
	\item Have the vertices in each component of $\L(G)$ share their total charge equally.
\end{enumerate}
Let $\ch^*(v)$ be the resulting charge function.  Here is some intuition for why this might be a useful refinement.  In (3b), $v$ sends charge to at most two different $T$ and so, by Lemma \ref{ConfigurationTypeOneEuler} (or our `beyond degree choosability' classification), $v$ loses charge at most $3\gamma$.  On the other hand, from (3a) each component $T$ of $\L(G)$ receives charge $\gamma$ for all but at most two non-separating vertices in a $K_{k-1}$ (the at most two is coming from Lemma \ref{ConfigurationTypeOneEuler} and the fact that we leave $T$ in $Q$ until it has degree at most two and when it does, we send up to two extra $\gamma$ to $T$ in (3ai) as needed). Note that (3ai) doesn't cause any $v \in V(\HH(G))$ to lose more than $3\gamma$, because it only gets enacted when the component $T$ is about to be removed, after that $v$ does not have two neighbors in another component. So, we can get each $T$ almost as much charge as we could hope for without losing too much from the $k$-vertices.  We don't have the same control over $(k+1)^+$-vertices, but it won't matter since they have extra charge to start with and sending $\gamma$ to every $(k-1)$-neighbor will leave enough charge (we'll use $\gamma < 2\epsilon$ here).

To analyze this discharging procedure we need a bound like Lemma \ref{BasicGallaiTreeBound}, but taking into account the number of edges in $\B_k(G)$.  We can do this by taking into account the number of non-separating vertices in $K_{k-1}$'s in $T$.  To this end, for $T \in \T_k$, let $q(T)$ be the number of non-separating vertices in a $K_{k-1}$ in $T$.  We give a family of such bounds.  Without more reducible configurations we can't hope to do better than average degree $k-3$ because of $K_{k-2}$ components, that is why the bound below has $(k-3 + p(k))|T|$, a slight worsening of average degree $k-3$.

\begin{lem}\label{BoundFamily}
	Let $K \ge 7$ and $\func{p}{\IN}{\IR}$, $\func{f}{\IN}{\IR}$, $\func{h}{\IN}{\IR}$ be such that for all $k \ge K$ we have
	\begin{enumerate}
		\item $f(k) \ge t(t+2-k-p(k))$ for all $t \in \irange{k-2}$; and
    	\item $f(k) \ge (5-k-p(k))s$ for all $s \ge 5$; and
		\item $f(k) \ge (k-1)(1- p(k) - h(k))$; and	
		\item $p(k) \ge h(k) + 5 - k$; and
		\item $p(k) \ge \frac{3}{k-2}$; and
		\item $p(k) \ge \frac{2+h(k)}{k-2}$; and
		\item $(k-1)p(k) + (k-3)h(k) \ge k+1$.
	\end{enumerate}
	Then for $k \ge K$ and $T \in \T_k$, we have
	\[2\size{T} \le (k-3 + p(k))\card{T} + f(k) + h(k)q(T).\]
\end{lem}
\begin{proof}
	Suppose not and choose a counterexample $T$ minimizing $|T|$.  First, suppose $T$ is $K_t$ for $t \in \irange{k-2}$.  Then $t(t-1) > (k-3 + p(k))t + f(k)$ contradicting (1).  If $T$ is $C_{2r+1}$ for $r \ge 2$, then $2(2r+1) > (k-3 + p(k))(2r+1) + f(k)$ and hence $f(k) < (5-k-p(k))(2r+1)$ contradicting (2).  If $T$ is $K_{k-1}$, then $(k-1)(k-2) > (k-3 + p(k))(k-1) + f(k) + h(k)(k-1)$ contradicting (3).
	
	Hence $T$ has at least two blocks.  Let $B$ be an endblock of $T$ and $x_B$ the cutvertex of $T$ contained in $B$. Let $T' = T - \parens{V(B) \setminus \set{x_B}}$. Then, by minimality of $|T|$, we have
	\[2\size{T'} \le (k-3 + p(k))\card{T'} + f(k) + h(k)q(T').\]
	Hence
		\[2\size{T} - 2\size{B} \le (k-3 + p(k))\parens{\card{T} - (\card{B} - 1)} + f(k) + h(k)q(T').\]
    Since $T$ is a counterexample, this gives
    \begin{equation}
	    2\size{B} > (k-3 + p(k))(\card{B} - 1) + h(k)\parens{q(T) - q(T')}.\tag{*}\label{eqn:*}
    \end{equation}
    
    Suppose $B$ is $K_t$ for $3 \le t \le k-3$ or $B$ is an odd cycle. Then $q(T') = q(T)$, $2\size{B} \le \card{B}(\card{B}-1)$ and $2\size{B} = 2\card{B}$ if $\card{B} > k-3$.  Since $p(k) \ge \frac{4}{k-2}$ by (5), this contradicts \ref{eqn:*}.
    
    If $B$ is $K_2$, then $q(T') \le q(T) + 1$ and \ref{eqn:*} gives $2 > k-3 + p(k) - h(k)$ contradicting (4).
    
	To handle the cases when $B$ is $K_{k-2}$ or $K_{k-1}$ we need to remove $x_B$ from $T$ as well.  Let $T^* = T - V(B)$. Then, by minimality of $|T|$, we have
	\[2\size{T^*} \le (k-3 + p(k))\card{T^*} + f(k) + h(k)q(T^*).\]
	Hence
		\[2\size{T} - 2\size{B} - 2(d_T(x_B) - d_B(x_B)) \le (k-3 + p(k))\parens{\card{T} - \card{B}} + f(k) + h(k)q(T^*).\]
		
	Since $T$ is a counterexample and $B$ is complete, this gives
	\[2\size{B} > (k-3 + p(k))\card{B} -2(d_T(x_B) + 1 - \card{B}) + h(k)\parens{q(T) - q(T^*)},\]
	which is
		 \begin{equation}
		 2\size{B} > (k-1 + p(k))\card{B} - 2d_T(x_B) - 2 + h(k)\parens{q(T) - q(T^*)}.\tag{**}\label{eqn:**}
		 \end{equation}
		 
		 	Suppose $B$ is $K_{k-1}$.  Then $d_T(x_B) = k - 1$ and $q(T^*) \le q(T) - (k-2) + 1 = q(T) - (k-3)$.  From \ref{eqn:**}, we have
		 	\begin{align*}
		 		(k-1)(k-2) &> (k-1 + p(k))(k-1) - 2(k-1) - 2 + h(k)(k-3)\\
		 		&= (k-1)(k-2) + p(k)(k-1) - (k+1) + h(k)(k-3),
		 	\end{align*}
		 	contradicting (7).
		 	
		 Finally, suppose $B$ is $K_{k-2}$.  Then $d_T(x_B) = k - 1$ or $d_T(x_B) = k-2$.  In the former case, $q(T) = q(T^*)$ and in the latter $q(T^*) \le q(T) + 1$.  If $d_T(x_B) = k - 2$, we have
		 \[(k-2)(k-3) > (k-1 + p(k))(k-2) - 2(k-2) - 2 - h(k) = (k-2)(k-3) - 2 + (k-2)p(k) - h(k),\]
		 contradicting (6).

		 Now we need to handle the remaining case when $B$ is $K_{k-2}$ and $d_T(x_B) = k - 1$.  All of the above cases were for when $B$ was any endblock of $T$, 
		 so we may assume that every endblock of $T$ is a $K_{k-2}$ that shares a vertex with an odd cycle.  Choose an endblock $B$ that is the end of a longest path in the block-tree of $T$.  
		 Let $C$ be the odd cycle sharing a vertex $x_B$ with $B$.  We claim that all but one vertex of $C$ is in an endblock.  Since $B$ is the end of a longest path, $C$ cannot have two non-cut-vertices that are both not in endblocks, for then we could
		 get a longer path.  So, to prove our claim, it will suffice to show that every vertex of $C$ is a cut-vertex.  Suppose $v \in V(C)$ is not a cut-vertex.  Then $d_T(v) = 2$ and hence by minimality of $\card{T}$
		 \[2\size{T} - 4 \le (k-3 + p(k))\parens{\card{T} - 1} + f(k) + h(k)q(T - v),\]
		 Since $q(T-v) = q(T)$, the fact that $T$ is a counterexample implies
		 \[4 > k-3 + p(k),\]
		 a contradiction since $k \ge K \ge 7$ and $p(k) > 0$.  So, we have shown that all but one vertex of $C$ is in an endblock.  Hence there are endblocks $A$ and $B$ such that $x_A, x_B \in V(C)$ and $x_A$ is adjacent to $x_B$.  let $\hat{T} = T - \parens{V(A) \cup V(B)}$.  Then $q(\hat{T}) = q(T)$.  Since the edge $x_Ax_B$ is shared, by minimality of $\card{T}$, we have
		\[2\size{T} - 2\size{A} - 2\size{B} - 6 \le (k-3 + p(k))\parens{\card{T} - \card{A} - \card{B}} + f(k) + h(k)q(T).\]
 		Since $T$ is a counterexample, this gives
	 	\[2\size{A} + 2\size{B} + 6 > (k-3 + p(k))(\card{A} + \card{B}),\]
		 which is
		 \[2(k-2)(k-3) + 6 > 2(k-3 + p(k))(k-2),\]
		 giving
		 \[3 > (k-2)p(k),\]
		 which contradicts (5).
 \end{proof}

\begin{lem}\label{BoundFamilyWithKKMinusOne}
	Let $K \ge 7$ and $\func{p}{\IN}{\IR}$, $\func{f}{\IN}{\IR}$, $\func{h}{\IN}{\IR}$ be such that for all $k \ge K$ we have
	\begin{enumerate}
		\item $f(k) \ge (k-1)(1- p(k) - h(k))$; and	
		\item $p(k) \ge h(k) + 5 - k$; and
		\item $p(k) \ge \frac{3}{k-2}$; and
		\item $p(k) \ge \frac{2+h(k)}{k-2}$; and
		\item $(k-1)p(k) + (k-3)h(k) \ge k+1$.
	\end{enumerate}
	Then for $k \ge K$ and $T \in \T_k$ with $K_{k-1} \subseteq T$, we have
	\[2\size{T} \le (k-3 + p(k))\card{T} + f(k) + h(k)q(T).\]
\end{lem}
\begin{proof}
	Suppose not and choose a counterexample $T$ minimizing $|T|$.  If $T$ has only one block, then $T=K_{k-1}$ and hence $(k-1)(k-2) > (k-3 + p(k))(k-1) + f(k) + h(k)(k-1)$ contradicting (1). 
	
	Hence $T$ has at least two blocks.  Suppose $T$ has an endblock $B$ that is an odd cycle or $K_t$ for $3 \le t \le k-3$. Let $x_B$ the cutvertex of $T$ contained in $B$. 
	Let $T' = T - \parens{V(B) \setminus \set{x_B}}$. Then $K_{k-1} \subseteq T'$, so by minimality of $|T|$, we have
	
	\[2\size{T'} \le (k-3 + p(k))\card{T'} + f(k) + h(k)q(T').\]
	Hence
	\[2\size{T} - 2\size{B} \le (k-3 + p(k))\parens{\card{T} - (\card{B} - 1)} + f(k) + h(k)q(T').\]
	Since $T$ is a counterexample, this gives
	\begin{equation}
		2\size{B} > (k-3 + p(k))(\card{B} - 1) + h(k)\parens{q(T) - q(T')}.\tag{*}\label{eqn:*}
	\end{equation}
	
	Then $q(T') = q(T)$, $2\size{B} \le \card{B}(\card{B}-1)$ and $2\size{B} = 2\card{B}$ if $\card{B} > k-3$.  Since $p(k) \ge \frac{3}{k-2}$ by (3), this contradicts \ref{eqn:*}.
	
	If $B$ is $K_2$, then $q(T') \le q(T) + 1$ and \ref{eqn:*} gives $2 > k-3 + p(k) - h(k)$ contradicting (2).
	
	Hence every endblock of $B$ is $K_{k-2}$ or $K_{k-1}$. To handle these cases, we will need to remove $x_B$ from $T$ as well.  Let $\set{B_1, \ldots, B_r}$ be the endblocks $B_i$ of $T$ such that $K_{k-1} \subseteq T - V(B_i)$.
	Then $r \ge 1$ since two $K_{k-1}$ blocks cannot share a vertex.  For $i \in \irange{r}$, let $T_i = T - V(B_i)$. 
	
	Then, by minimality of $|T|$, we have
	\[2\size{T_i} \le (k-3 + p(k))\card{T_i} + f(k) + h(k)q(T_i).\]
	Hence
	\[2\size{T} - 2\size{B_i} - 2(d_T(x_{B_i}) - d_{B_i}(x_{B_i})) \le (k-3 + p(k))\parens{\card{T} - \card{B_i}} + f(k) + h(k)q(T_i).\]
	
	Since $T$ is a counterexample and $B_i$ is complete, this gives
	\[2\size{B_i} > (k-3 + p(k))\card{B_i} -2(d_T(x_{B_i}) + 1 - \card{B_i}) + h(k)\parens{q(T) - q(T_i)},\]
	which is
	\begin{equation}
		2\size{B_i} > (k-1 + p(k))\card{B_i} - 2d_T(x_{B_i}) - 2 + h(k)\parens{q(T) - q(T_i)}.\tag{**}\label{eqn:**}
	\end{equation}
	
	If $B_i$ is $K_{k-1}$, we have $d_T(x_{B_i}) = k - 1$ and $q(T_i) \le q(T) - (k-2) + 1 = q(T) - (k-3)$.  From \ref{eqn:**}, we have
	\begin{align*}
		(k-1)(k-2) &> (k-1 + p(k))(k-1) - 2(k-1) - 2 + h(k)(k-3)\\
		&= (k-1)(k-2) + p(k)(k-1) - (k+1) + h(k)(k-3),
	\end{align*}
	contradicting (5).
	
	Hence $B_i = K_{k-2}$ for all $i \in \irange{r}$. So $d_T(x_{B_i}) = k - 1$ or $d_T(x_{B_i}) = k-2$.  In the former case, $q(T) = q(T_i)$ and in the latter $q(T_i) \le q(T) + 1$.  If $d_T(x_{B_i}) = k - 2$, we have
	\[(k-2)(k-3) > (k-1 + p(k))(k-2) - 2(k-2) - 2 - h(k) = (k-2)(k-3) - 2 + (k-2)p(k) - h(k),\]
	contradicting (4).
	
	Hence $d_T(x_{B_i}) = k - 1$ for all $i \in \irange{r}$. If $B$ and $B'$ are blocks that are the ends of a longest path in the block-tree of $T$, then $\set{B,B'} \cap \set{B_1, \ldots, B_r} \ne \emptyset$.  By symmetry, we may assume that $B = B_1$.
	Let $C$ be the odd cycle sharing a vertex $x_B$ with $B$.  We claim that all but one vertex of $C$ is in an endblock.  Since $B$ is the end of a longest path, $C$ cannot have two non-cut-vertices that are both not in endblocks, for then we could
	get a longer path.  So, to prove our claim, it will suffice to show that every vertex of $C$ is a cut-vertex.  Suppose $v \in V(C)$ is not a cut-vertex.  Then $d_T(v) = 2$ and hence by minimality of $\card{T}$
	\[2\size{T} - 4 \le (k-3 + p(k))\parens{\card{T} - 1} + f(k) + h(k)q(T - v),\]
	Since $q(T-v) = q(T)$, the fact that $T$ is a counterexample implies
	\[4 > k-3 + p(k),\]
	a contradiction since $k \ge K \ge 7$ and $p(k) > 0$ by (3).  So, we have shown that all but one vertex of $C$ is in an endblock.  Hence there are endblocks $A$ and $B$ such that $x_A, x_B \in V(C)$ and $x_A$ is adjacent to $x_B$.  let $\hat{T} = T - \parens{V(A) \cup V(B)}$.  Then $K_{k-1} \subseteq \hat{T}$ and $q(\hat{T}) = q(T)$.  Since the edge $x_Ax_B$ is shared, by minimality of $\card{T}$, we have
	\[2\size{T} - 2\size{A} - 2\size{B} - 6 \le (k-3 + p(k))\parens{\card{T} - \card{A} - \card{B}} + f(k) + h(k)q(T).\]
	Since $T$ is a counterexample, this gives
	\[2\size{A} + 2\size{B} + 6 > (k-3 + p(k))(\card{A} + \card{B}),\]
	which is
	\[2(k-2)(k-3) + 6 > 2(k-3 + p(k))(k-2),\]
	giving
	\[3 > (k-2)p(k),\]
	which contradicts (3).
\end{proof}

\begin{lem}\label{BoundFamilyWithoutKKMinusOne}
	Let $K \ge 7$ and $\func{p}{\IN}{\IR}$, $\func{f}{\IN}{\IR}$ be such that for all $k \ge K$ we have
	\begin{enumerate}
		\item $p(k) \ge \frac{-f(k)}{k-2}$; and
		\item $p(k) \ge \frac{-f(k)}{5} + 5 - k$; and
		\item $p(k) \ge \frac{3}{k-2}$.
	\end{enumerate}
	Then for $k \ge K$ and $T \in \T_k$ with $K_{k-1} \not \subseteq T$, we have
	\[2\size{T} \le (k-3 + p(k))\card{T} + f(k).\]
\end{lem}
\begin{proof}
	Suppose not and choose a counterexample $T$ minimizing $|T|$.  First, suppose $T$ is $K_t$ for $t \in \irange{k-2}$.  Then $t(t-1) > (k-3 + p(k))t + f(k)$ contradicting (1).  If $T$ is $C_{2r+1}$ for $r \ge 2$, then $2(2r+1) > (k-3 + p(k))(2r+1) + f(k)$ and hence $f(k) < (5-k-p(k))(2r+1)$ contradicting (2).
	
	Hence $T$ has at least two blocks.  Let $B$ be an endblock of $T$ and $x_B$ the cutvertex of $T$ contained in $B$. Let $T' = T - \parens{V(B) \setminus \set{x_B}}$. Then, by minimality of $|T|$, we have
	\[2\size{T'} \le (k-3 + p(k))\card{T'} + f(k).\]
	Hence
	\[2\size{T} - 2\size{B} \le (k-3 + p(k))\parens{\card{T} - (\card{B} - 1)} + f(k).\]
	Since $T$ is a counterexample, this gives
	\begin{equation}
	2\size{B} > (k-3 + p(k))(\card{B} - 1).\tag{*}\label{eqn:*}
	\end{equation}
	
	Suppose $B$ is $K_t$ for $3 \le t \le k-3$ or $B$ is an odd cycle. Then $2\size{B} \le \card{B}(\card{B}-1)$ and $2\size{B} = 2\card{B}$ if $\card{B} > k-3$.  Since $p(k) \ge \frac{3}{k-2}$ by (3), this contradicts \ref{eqn:*}.
	
	If $B$ is $K_2$, then \ref{eqn:*} gives $2 > k-3 + p(k)$, a contradiction since $k \ge 5$ and $p(k) > 0$ by (3).
	
	To handle the case when $B$ is $K_{k-2}$ we need to remove $x_B$ from $T$ as well.  Let $T^* = T - V(B)$. Then, by minimality of $|T|$, we have
	\[2\size{T^*} \le (k-3 + p(k))\card{T^*} + f(k).\]
	Hence
	\[2\size{T} - 2\size{B} - 2(d_T(x_B) - d_B(x_B)) \le (k-3 + p(k))\parens{\card{T} - \card{B}} + f(k).\]
	
	Since $T$ is a counterexample and $B$ is complete, this gives
	\[2\size{B} > (k-3 + p(k))\card{B} -2(d_T(x_B) + 1 - \card{B}),\]
	which is
	\begin{equation}
	2\size{B} > (k-1 + p(k))\card{B} - 2d_T(x_B) - 2.\tag{**}\label{eqn:**}
	\end{equation}

	Since $B=K_{k-2}$, we have either $d_T(x_B) = k - 1$ or $d_T(x_B) = k-2$. If $d_T(x_B) = k - 2$, we have
	\[(k-2)(k-3) > (k-1 + p(k))(k-2) - 2(k-2) - 2 = (k-2)(k-3) - 2 + (k-2)p(k),\]
	contradicting (3).
	
	Now we need to handle the remaining case when $B$ is $K_{k-2}$ and $d_T(x_B) = k - 1$.  All of the above cases were for when $B$ was any endblock of $T$, 
	so we may assume that every endblock of $T$ is a $K_{k-2}$ that shares a vertex with an odd cycle.  Choose an endblock $B$ that is the end of a longest path in the block-tree of $T$.  
	Let $C$ be the odd cycle sharing a vertex $x_B$ with $B$.  We claim that all but one vertex of $C$ is in an endblock.  Since $B$ is the end of a longest path, $C$ cannot have two non-cut-vertices that are both not in endblocks, for then we could
	get a longer path.  So, to prove our claim, it will suffice to show that every vertex of $C$ is a cut-vertex.  Suppose $v \in V(C)$ is not a cut-vertex.  Then $d_T(v) = 2$ and hence by minimality of $\card{T}$
	\[2\size{T} - 4 \le (k-3 + p(k))\parens{\card{T} - 1} + f(k),\]
	The fact that $T$ is a counterexample implies
	\[4 > k-3 + p(k),\]
	a contradiction since $k \ge K \ge 7$ and $p(k) > 0$.  So, we have shown that all but one vertex of $C$ is in an endblock.  Hence there are endblocks $A$ and $B$ such that $x_A, x_B \in V(C)$ and $x_A$ is adjacent to $x_B$.  let $\hat{T} = T - \parens{V(A) \cup V(B)}$.  Since the edge $x_Ax_B$ is shared, by minimality of $\card{T}$, we have
	\[2\size{T} - 2\size{A} - 2\size{B} - 6 \le (k-3 + p(k))\parens{\card{T} - \card{A} - \card{B}} + f(k).\]
	Since $T$ is a counterexample, this gives
	\[2\size{A} + 2\size{B} + 6 > (k-3 + p(k))(\card{A} + \card{B}),\]
	which is
	\[2(k-2)(k-3) + 6 > 2(k-3 + p(k))(k-2),\]
	giving
	\[3 > (k-2)p(k),\]
	which contradicts (3).
\end{proof}

Lemma \ref{BoundFamilyWithoutKKMinusOne} works with $p(k) = \frac{3}{k-2}$ and $f(k) = -3$.  We probably get (2) for free, clean up later.  In the discharging it will be convenient to apply Lemma \ref{BoundFamilyWithoutKKMinusOne} with a larger $p(k)$ to match the one we get when we have $K_{k-1}$ blocks (we can even be wasteful with the charge and use $f(k) = 0$).

Now some examples of using Lemma \ref{BoundFamily} and Lemma \ref{BoundFamilyWithKKMinusOne}.  What happens if we take $h(k) = 0$ in Lemma \ref{BoundFamily}?  Then, by (7), we need $(k-1)p(k) \ge k + 1$ and hence $p(k) \ge 1 + \frac{2}{k-1}$.  Taking $p(k) = 1 + \frac{2}{k-1}$, (3) requires $f(k) \ge -2$.  Using $f(k) = -2$, all of the other conditions are satisfied and we conclude $2\size{T} \le \parens{k-2 + \frac{2}{k-1}}\card{T} - 2$ for every $T \in \T_k$ when $k \ge 4$.  This is a slight refinement of Gallai's Lemma \ref{BasicGallaiTreeBound}.

Instead, let's make $p(k)$ as small as Lemma \ref{BoundFamilyWithKKMinusOne} will let us. By (4), $h(k) \le (k-2)p(k) - 2$, plugging this in to (5) and solving we get $p(k) \ge \frac{3k-5}{k^2 - 4k + 5}$.  Now $\frac{3k-5}{k^2 - 4k + 5} \ge \frac{3}{k-2}$ for $k \ge 7$, so $p(k) = \frac{3k-5}{k^2 - 4k + 5}$ satisfies (3).  With $h(k) = \frac{k(k-3)}{k^2 - 4k + 5}$, (4) and (5) are also satisfied. Now with $f(k) = -\frac{2(k-1)(2k-5)}{k^2 - 4k + 5}$, condition (1) is satisfied and hence by Lemma we have the following.

\begin{cor}\label{SmallP}
	For $k \ge 7$ and $T \in \T_k$ with $K_{k-1} \subseteq T$, we have
	\[2\size{T} \le \parens{k-3 + \frac{3k-5}{k^2 - 4k + 5}}\card{T} - \frac{2(k-1)(2k-5)}{k^2 - 4k + 5} + \frac{k(k-3)}{k^2 - 4k + 5}q(T).\]
\end{cor}

If we put the Kostochka-Stiebitz bound on $\sigma(T)$ into this form we get the following.

\begin{lem}[Kostochka-Stiebitz]
		For $k \ge 7$ and $T \in \T_k$, we have
		\[2\size{T} \le \parens{k-3 + \frac{4(k-1)}{k^2 - 3k + 4}}\card{T} - \frac{4(k^2-3k+2)}{k^2-3k+4} + \frac{k^2 - 3k}{k^2-3k+4}q(T).\]
\end{lem}

Note that $\frac{3k-5}{(k-5)(k-1)} < \frac{4(k-1)}{k^2 - 3k + 4}$ for $k \ge 7$.

\subsection{Analyzing the discharging}
Our discharging procedure gives charge $\epsilon$ to a component $T$ for every incident edge not ending in a $K_{k-1}$.  The number of such edges is exactly
\[A(T) \DefinedAs -q(T) + \sum_{v \in V(T)} k-1 - d_T(v) = (k-1)\card{T} - 2\size{T} - q(T).\]
Suppose we have a bound when $K_{k-1} \subseteq T$ of the form
\[2\size{T} \le (k-3 + p(k))\card{T} + f(k) + h(k)q(T).\]
So, when $K_{k-1} \subseteq T$ we get
\[A(T) \ge (2-p(k))\card{T} - f(k) - (h(k) + 1)q(T).\]

We will use $\gamma = (h(k) + 1)\epsilon$ in order to make the $q(T)$ term cancel.  That happens because $T$ receives charge on all but at most two of its non-separating vertices in a $K_{k-1}$; that is, in discharging steps 2 and 3, $T$ receives charge at least $\gamma\max\set{0, q(G) - 2}$.   Hence in total $T$ receives charge at least
\[\epsilon A(T) + \gamma(q(G) - 2) = \epsilon\parens{2-p(k)}\card{T} - \epsilon \parens{f(k) + 2(h(k) + 1)}.\]

To simplify things, let's impose the requirement $f(k) + 2(h(k) + 1) \le 0$.  Then $T$ receives charge at least

\[\epsilon\parens{2-p(k)}\card{T}.\]
We want the $k$-vertices to end with enough charge, the worst case is when
\[1 - (3\gamma + (k-3)\epsilon) = \epsilon\parens{2-p(k)},\]
and thus
\[\epsilon = \frac{1}{k+2 + 3h(k) - p(k)},\]
\[\gamma = \frac{h(k)+1}{k+2 + 3h(k) - p(k)}.\]

When $K_{k-1} \not \subseteq T$, we have $q(T) = 0$.  By Lemma \ref{BoundFamilyWithoutKKMinusOne} with $f(k) = 0$, we get 
\[2\size{T} \le (k-3 + p(k))\card{T}\]
and hence
\[A(T) \ge (2-p(k))\card{T},\]
which is sufficient charge.

It remains to check that the $(k+1)^+$-vertices don't give away too much charge.  Let $v$ be a $(k+1)^+$-vertex, then $v$ ends with charge at least
\[d(v) - \gamma d(v) = (1-\gamma)d(v) \ge (1-\gamma)(k+1) = (k+1)\frac{k+1 + 2h(k) - p(k)}{k+2 + 3h(k) - p(k)},\]
so we need
\[(k+1)\frac{k+1 + 2h(k) - p(k)}{k+2 + 3h(k) - p(k)} \ge k-1 + \frac{2-p(k)}{k+2 + 3h(k) - p(k)},\]
simplifying, we get that we need
\[p(k) + (k-5)h(k) \le k+1.\]
Let's just add this as another requirement, it will be easily satisfied by the functions we want to use.  We have proved the following.

\begin{thm}\label{UberTheorem}
	Let $K \ge 7$ and $\func{p}{\IN}{\IR}$, $\func{f}{\IN}{\IR}$, $\func{h}{\IN}{\IR}$ be functions satisfying
	\begin{itemize}
		\item $f(k) + 2(h(k) + 1) \le 0$; and
		\item $p(k) + (k-5)h(k) \le k+1$.
	\end{itemize}
    If for all $k \ge K$ and $T \in \T_k$ we have
	\[2\size{T} \le (k-3 + p(k))\card{T} + f(k) + h(k)q(T),\]
	then for $k \ge K$ and $G \ne K_k$ a $k$-AT-critical graph, we have
	\[d(G) \ge k-1 + \frac{2-p(k)}{k+2 + 3h(k) - p(k)}.\]
\end{thm}

As a first test, suppose $p(k) = 1 - \frac{2}{k-1}$, $f(k) = -2$ and $h(k) = 0$.  Then the hypotheses of Theorem \ref{UberTheorem} are satisfied with $K=7$ and we get Gallai's bound $d(G) \ge k-1 + \frac{k-3}{k^2-3}$. 

Now, let's try the Kostochka-Stiebitz bound, that is, $p(k) = \frac{4(k-1)}{k^2 - 3k + 4}$, $f(k) = -\frac{4(k^2-3k+2)}{k^2-3k+4}$ and $h(k) = \frac{k^2 - 3k}{k^2-3k+4}$.  Again, the hypotheses of Theorem \ref{UberTheorem} are satisfied with $K=7$ and we get
\[d(G) \ge k-1 + \frac{2(k-2)(k-3)}{(k-1)(k^2 + 3k - 12)}.\]

This is exactly equal to the bound in the paper with Hal!  

Now, let's try our bound in
Lemma \ref{SmallP}, that is, $p(k) = \frac{3k-5}{k^2 - 4k + 5}$, $f(k) = -\frac{2(k-1)(2k-5)}{k^2 - 4k + 5}$ and $h(k) = \frac{k(k-3)}{k^2 - 4k + 5}$.  The hypotheses of Theorem \ref{UberTheorem} are satisfied with $K=7$ and we get
 \[d(G) \ge k-1 + \frac{(k-3)(2k-5)}{k^3 + k^2 - 15k + 15}.\]
 
This is better than the bound with Hal for $k \ge 7$.  



Possible improvements:

\begin{enumerate}
	\item Use a better bound on average degree of Gallai trees.  i would like to find the best possible family in the form here.  How does this bound compare to the hand waiving one in the other document?
	\item In the discharging, the $k$-vertices lost $3\gamma$ even though they had degree two in $Q$ because of the possibility of two edges into one component.  Can we get this to $2\gamma$ somehow, like maybe we can order our picking so that no vertex is picked before the component where it has two edges has been removed.   
	\item Related to the previous item, improved reducible configurations, a less restrictive condition in Lemma \ref{MultipleHighConfigurationEuler} taking into account the two edges to a component issue.
\end{enumerate}

\section{Reducible Configurations}
\begin{defn}
	A graph $G$ is \emph{AT-reducible} to $H$ if $H$ is a nonempty induced subgraph of $G$ which is $f_H$-AT where $f_H(v) \DefinedAs \delta(G) + d_H(v) - d_G(v)$ for all $v \in V(H)$.  
	If $G$ is not AT-reducible to any nonempty induced subgraph, then it is \emph{AT-irreducible}.
\end{defn}

This first lemma tells us how a single high vertex can interact with the low vertex subgraph.  This is the version Hal and i used, it (and more) follows from the classification in ``mostlow''.

\begin{lem}\label{ConfigurationTypeOneEuler}
Let $k \ge 5$ and let $G$ be a graph with $x \in V(G)$ such that:
\begin{enumerate}
\item $K_k \not \subseteq G$; and
\item $G-x$ has $t$ components $H_1, H_2, \ldots, H_t$, and all are in $\T_k$; and
\item $d_G(v) \leq k - 1$ for all $v \in V(G-x)$; and
\item $\card{N(x) \cap W^k(H_i)} \ge 1$ for $i \in \irange{t}$; and
\item $d_G(x) \ge t+2$.
\end{enumerate}

\noindent Then $G$ is $f$-AT where $f(x) = d_G(x) - 1$ and $f(v) = d_G(v)$ for all $v \in V(G - x)$.
\end{lem}

To deal with more than one high vertex we need the following auxiliary bipartite graph.  For a graph $G$, $\set{X, Y}$ a partition of $V(G)$ and $k \ge 4$, let $\B_k(X, Y)$ be the bipartite graph with one part $Y$ and the other part the components of $G[X]$.  Put an edge between $y \in Y$ and a component $T$ of $G[X]$ if and only if $N(y) \cap W^k(T) \ne \emptyset$.   The next lemma tells us that we have a reducible configuration if this bipartite graph has minimum degree at least three.  

\begin{lem}
	\label{MultipleHighConfigurationEuler} Let $k\ge7$ and let $G$ be a graph with
	$Y\subseteq V(G)$ such that: 
	\begin{enumerate}
		\item $K_{k}\not\subseteq G$; and 
		\item the components of $G-Y$ are in $\T_{k}$; and 
		\item $d_{G}(v)\leq k-1$ for all $v\in V(G-Y)$; and 
		\item with $\B\DefinedAs\B_{k}(V(G-Y),Y)$ we have $\delta(\B)\ge3$. 
	\end{enumerate}
	\noindent Then $G$ has an induced subgraph $G'$ that is $f$-AT where $f(y)=d_{G'}(y)-1$
	for $y\in Y$ and $f(v)=d_{G'}(v)$ for all $v\in V(G'-Y)$.\end{lem}

We also have the following version with asymmetric degree condition on $\B$.  The point here is that this works for $k \ge 5$.  As we'll see in the next section, the consequence is that we trade a bit in our size bound for the proof to go through with $k \in \set{5,6}$.

\begin{lem}
	\label{MultipleHighConfigurationEulerLopsided} Let $k \ge 5$ and let $G$ be a graph with
	$Y\subseteq V(G)$ such that: 
	\begin{enumerate}
		\item $K_{k}\not\subseteq G$; and 
		\item the components of $G-Y$ are in $\T_{k}$; and 
		\item $d_{G}(v)\leq k-1$ for all $v\in V(G-Y)$; and 
		\item with $\B \DefinedAs \B_k(V(G-Y), Y)$ we have $d_{\B}(y) \ge 4$ for all $y \in Y$ and $d_{\B}(T) \ge 2$ for all components $T$ of $G-Y$.
	\end{enumerate}
	\noindent Then $G$ has an induced subgraph $G'$ that is $f$-AT where $f(y)=d_{G'}(y)-1$
	for $y\in Y$ and $f(v)=d_{G'}(v)$ for all $v\in V(G'-Y)$.\end{lem}



\bibliographystyle{amsplain}
\bibliography{GraphColoring1}
\end{document}

 

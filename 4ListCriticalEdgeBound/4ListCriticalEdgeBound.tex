\documentclass[12pt]{article}
\usepackage{amsmath, amsthm, amssymb}
\usepackage{tkz-graph}
\usepackage{marginnote}
\usepackage{verbatim}
\usepackage[top=1.0in, bottom=1.0in, left=1.0in, right=1.0in]{geometry}
\usepackage{color}
\pagestyle{plain}

\usepackage[backref=page]{hyperref}

\usepackage{sectsty}
\allsectionsfont{\sffamily}

\setcounter{secnumdepth}{5}
\setcounter{tocdepth}{5}

\makeatletter
\newtheorem*{rep@theorem}{\rep@title}
\newcommand{\newreptheorem}[2]{
\newenvironment{rep#1}[1]{
 \def\rep@title{#2 \ref{##1}}
 \begin{rep@theorem}}
 {\end{rep@theorem}}}
\makeatother

\theoremstyle{plain}
\newtheorem{thm}{Theorem}
\newreptheorem{thm}{Theorem}
\newtheorem{prop}[thm]{Proposition}
\newreptheorem{prop}{Proposition}
\newtheorem{lem}[thm]{Lemma}
\newreptheorem{lem}{Lemma}
\newtheorem{conjecture}[thm]{Conjecture}
\newreptheorem{conjecture}{Conjecture}
\newtheorem{cor}[thm]{Corollary}
\newreptheorem{cor}{Corollary}
\newtheorem{prob}[thm]{Problem}

\newtheorem*{KernelLemma}{Kernel Lemma}
\newtheorem*{MainTheorem}{Main Theorem}
\newtheorem*{BK}{Borodin-Kostochka Conjecture}
\newtheorem*{BK2}{Borodin-Kostochka Conjecture (restated)}
\newtheorem*{Reed}{Reed's Conjecture}
\newtheorem*{ClassificationOfd0}{Classification of $d_0$-choosable graphs}


\theoremstyle{definition}
\newtheorem{defn}{Definition}
\theoremstyle{remark}
\newtheorem*{remark}{Remark}
\newtheorem*{problem}{Problem}
\newtheorem{example}{Example}
\newtheorem*{question}{Question}
\newtheorem*{observation}{Observation}

\newcommand{\fancy}[1]{\mathcal{#1}}
\newcommand{\C}[1]{\fancy{C}_{#1}}


\newcommand{\IN}{\mathbb{N}}
\newcommand{\IR}{\mathbb{R}}
\newcommand{\G}{\fancy{G}}
\newcommand{\CC}{\fancy{C}}
\newcommand{\D}{\fancy{D}}
\newcommand{\T}{\fancy{T}}
\newcommand{\B}{\fancy{B}}
\renewcommand{\L}{\fancy{L}}
\newcommand{\HH}{\fancy{H}}

\newcommand{\inj}{\hookrightarrow}
\newcommand{\surj}{\twoheadrightarrow}

\newcommand{\set}[1]{\left\{ #1 \right\}}
\newcommand{\setb}[3]{\left\{ #1 \in #2 : #3 \right\}}
\newcommand{\setbs}[2]{\left\{ #1 : #2 \right\}}
\newcommand{\card}[1]{\left|#1\right|}
\newcommand{\size}[1]{\left\Vert#1\right\Vert}
\newcommand{\ceil}[1]{\left\lceil#1\right\rceil}
\newcommand{\floor}[1]{\left\lfloor#1\right\rfloor}
\newcommand{\func}[3]{#1\colon #2 \rightarrow #3}
\newcommand{\funcinj}[3]{#1\colon #2 \inj #3}
\newcommand{\funcsurj}[3]{#1\colon #2 \surj #3}
\newcommand{\irange}[1]{\left[#1\right]}
\newcommand{\join}[2]{#1 \mbox{\hspace{2 pt}$\ast$\hspace{2 pt}} #2}
\newcommand{\djunion}[2]{#1 \mbox{\hspace{2 pt}$+$\hspace{2 pt}} #2}
\newcommand{\parens}[1]{\left( #1 \right)}
\newcommand{\brackets}[1]{\left[ #1 \right]}
\newcommand{\DefinedAs}{\mathrel{\mathop:}=}

\newcommand{\mic}{\operatorname{mic}}
\newcommand{\AT}{\operatorname{AT}}
\newcommand{\col}{\operatorname{col}}
\newcommand{\ch}{\operatorname{ch}}
\newcommand{\type}{\operatorname{type}}
\newcommand{\nonsep}{\bar{S}}

\def\adj{\leftrightarrow}
\def\nonadj{\not\!\leftrightarrow}

\newcommand\restr[2]{{% we make the whole thing an ordinary symbol
  \left.\kern-\nulldelimiterspace % automatically resize the bar with \right
  #1 % the function
  \vphantom{\big|} % pretend it's a little taller at normal size
  \right|_{#2} % this is the delimiter
  }}

\def\D{\fancy{D}}
\def\C{\fancy{C}}
\def\A{\fancy{A}}
\def\chil{{\chi_\ell}}
\def\chiol{\chi_{\rm{OL}}}

\newcommand{\case}[2]{{\bf Case #1.}~{\it #2}~~}
\newcommand{\aside}[1]{\marginnote{\scriptsize{#1}}[0cm]}
\newcommand{\aaside}[2]{\marginnote{\scriptsize{#1}}[#2]}

\title{A better lower bound on average degree of 4-list-critical graphs}
\author{Landon Rabern}

\begin{document}
\maketitle
\begin{abstract}
		We show that for $k \ge 4$, every incomplete $k$-list-critical graph has average degree at least $k-1 + \frac{k-3}{k^2-2k+2}$.  This improves the best known bound for $k = 4,5,6$.
		The same bound holds for online $k$-list-critical graphs.
\end{abstract}

\section{Introduction}
A graph $G$ is \emph{$k$-list-critical} if $G$ is not $(k-1)$-choosable, but every
proper subgraph of $G$ is $(k-1)$-choosable.  For further definitions and notation, see \cite{OreVizing, DischargingLowerBound}. 
Table \ref{TheTable} shows some history of lower bounds on the average degree of $k$-list-critical graphs.

\begin{MainTheorem}
	For $k \ge 4$, every incomplete $k$-list-critical graph has average degree at least $k-1 + \frac{k-3}{k^2-2k+2}$.
\end{MainTheorem}

Main Theorem gives a lower bound of $3 + \frac{1}{10}$ for $4$-list-critical graphs. This appears to be the first improvement over Gallai's bound of $3 + \frac{1}{13}$. 
The same proof shows that Main Theorem holds for online $k$-list-critical graphs as well.  The proof does not work for k-Alon-Tarsi-critical graphs since we use the Kernel Lemma.

\begin{table}
	\begin{center}
		\begin{tabular}{|c|c|c|c|c|c|c|c|c|}
			\hline
			&\multicolumn{4}{ |c| }{$k$-Critical
				$G$}&\multicolumn{4}{|c|}{$k$-List Critical $G$}\\
			\hline
			& Gallai \cite{gallai1963kritische}
			& Kriv \cite{krivelevich1997minimal}
			& KS \cite{kostochkastiebitzedgesincriticalgraph}
			& KY \cite{kostochkayancey2012ore}
			& KS \cite{kostochkastiebitzedgesincriticalgraph} 
			& KR \cite{OreVizing}
			& CR \cite{DischargingLowerBound}
			& Here \\
			$k$ & $d(G) \ge$ & $d(G) \ge$ & $d(G) \ge$ & $d(G) \ge$ & $d(G) \ge$ & $d(G) \ge$ & $d(G) \ge$ & $d(G) \ge$\\
			\hline 
			4 & 3.0769 &3.1429&---&3.3333& --- & --- & --- & \bf{3.1}\\
			5 & 4.0909 &4.1429&---&4.5000& --- & 4.0984 & 4.1000 & \bf{4.1176}\\
			6 & 5.0909 &5.1304&5.0976&5.6000& --- & 5.1053 & 5.1076 & \bf{5.1153}\\
			7 & 6.0870 &6.1176&6.0990&6.6667& --- & 6.1149 & \bf{6.1192} & 6.1081\\
			8 & 7.0820 &7.1064&7.0980&7.7143& --- & 7.1128 & \bf{7.1167} & 7.1\\
			9 & 8.0769 &8.0968&8.0959&8.7500& 8.0838 & 8.1094 & \bf{8.1130} & 8.0923\\
			10 & 9.0722 &9.0886&9.0932&9.7778& 9.0793 & 9.1055 & \bf{9.1088} & 9.0853\\
			15 & 14.0541 &14.0618&14.0785&14.8571& 14.0610 & 14.0864 & \bf{14.0884} & 14.0609\\
			20 & 19.0428 &19.0474&19.0666&19.8947& 19.0490 & 19.0719 & \bf{19.0733} & 19.0469 \\
			\hline
		\end{tabular}
	\end{center}
	\caption{History of lower bounds on the average degree $d(G)$ of $k$-critical and $k$-list-critical graphs $G$.}
	\label{TheTable}
\end{table}

\section{The Proof}
The connected graphs in which each block is a complete graph
or an odd cycle are called \emph{Gallai trees}.  Gallai \cite{gallai1963kritische} proved that in a $k$-critical graph, the vertices of degree $k-1$ induce a disjoint union of Gallai trees.  The same is true for $k$-list-critical graphs (\cite{borodin1977criterion, erdos1979choosability}).  For a graph $T$ and $k \in \IN$, let $\beta_k(T)$ be the independence number of the subgraph of $T$ induced on the vertices of degree $k-1$.  When $k$ is defined in the context, put $\beta(T) 
\DefinedAs \beta_k(T)$.  

\begin{lem}\label{SimpleGallaiBetaBound}
	If $k \ge 4$ and $T \ne K_k$ is a Gallai tree with maximum degree at most $k-1$, then
	\[2||T|| \le (k-2)|T| + 2\beta(T).\]
\end{lem}
\begin{proof}
	Suppose the lemma is false and choose a counterexample $T$ minimizing $\card{T}$.  Plainly, $T$ has more than one block.  Let $A$ be an endblock of $T$ and let $x$ be the unique cutvertex of $T$ with $x \in V(A)$.
	Consider $T' \DefinedAs T - (V(A)\setminus\set{x})$.  By minimality of $\card{T}$,
	\begin{equation*}
		2\size{T} - 2\size{A} \le (k-2)(\card{T} + 1 - \card{A}) + 2\beta(T').
	\end{equation*}
		Since $T$ is a counterexample, $2\size{A} > (k-2)(\card{A} - 1)$.  So, if $k > 4$, then $A = K_{k-1}$ and if $k=4$, then $A$ is an odd cycle.  So, $d_G(x) = k-1$.
	Consider $T^* \DefinedAs T - V(A)$.  By minimality of $\card{T}$,
	\begin{equation*}
	2\size{T} - 2\size{A} - 2 \le (k-2)(\card{T} - \card{A}) + 2\beta(T^*).
	\end{equation*}
	Since $T$ is a counterexample, $2\size{A} + 2 > (k-2)\card{A} + 2(\beta(T) - \beta(T^*))$.  In $T^*$, all of $x$'s neighbors have degree at most $k-2$.
	But $d_G(x) = k-1$, so some vertex in $\set{x} \cup N(x)$ is in a maximum independent set of degree $k-1$ vertices in $T$.  Hence $\beta(T^*) \le \beta(T) - 1$, which gives
	\begin{equation*}
	 2\size{A} > (k-2)\card{A},
	\end{equation*}
	a contradiction since $k \ge 4$.
\end{proof}

\begin{defn} The \emph{maximum independent cover number }of a graph $G$
	is the maximum $\mic(G)$ of $\size{I, V(G) \setminus I}$ over all independent sets $I$
	of $G$. 
\end{defn}

\begin{thm}[Kierstead and R. \cite{KernelMagic}]\label{ConsantListMicStrength} 
	Every $k$-list-critical graph $G$ satisfies
	\[2\size{G} \ge (k-2)\card{G} + \mic(G) + 1.\]
\end{thm}

\begin{MainTheorem}
	For $k \ge 4$, every incomplete $k$-list-critical graph has average degree at least $k-1 + \frac{k-3}{k^2-2k+2}$.
\end{MainTheorem}
\begin{proof}
	Let $G \ne K_k$ be a $k$-list-critical graph.  Let $\L \subseteq V(G)$ be the vertices with degree $k-1$ and let $\HH = V(G) \setminus \L$.  Put $\size{\L} \DefinedAs \size{G[\L]}$ and $\size{\HH} \DefinedAs \size{G[\HH]}$.  
	Then
	\begin{equation}
		\size{\HH, \L} = (k-1)\card{\L} - 2\size{\L}.
		\label{FirstHHL}
	\end{equation}
	By Lemma \ref{SimpleGallaiBetaBound},
		\begin{equation}
		2\size{\L} \le (k-2)|\L| + 2\beta(\L)
		\label{BetaBound}
		\end{equation}
		Combining \ref{FirstHHL} and \ref{BetaBound} gives
			\begin{equation}
			\size{\HH, \L} \ge \card{\L} - 2\beta(\L).
			\label{SecondHHL}
			\end{equation}
			Also,
			\begin{align*}
			\size{\HH, \L} &= -2\size{\HH} + \sum_{v \in \HH} d_G(v) \\
			&= (k-1)\card{\HH} - 2\size{\HH} + \sum_{v \in \HH} \parens{d_G(v) - (k-1)} \\
			&= (k-1)\card{\HH} - 2\size{\HH} + \sum_{v \in V(G)} \parens{d_G(v) - (k-1)}\\
			&=(k-1)\card{\HH} - 2\size{\HH} + 2\size{G} - (k-1)\card{G},\\
			\end{align*}
			that is
			\begin{equation}
			\size{\HH, \L} = (k-1)\card{\HH} - 2\size{\HH} + 2\size{G} - (k-1)\card{G}.
			\label{ThirdHHL}
			\end{equation}
			Combining \ref{SecondHHL} with \ref{ThirdHHL} gives
			\begin{equation*}
			  2\size{G} \ge (k-1)\card{G} + \card{\L} + 2\size{\HH} - (k-1)\card{\HH} - 2\beta(\L).
			\end{equation*}
			Since $\card{G} = \card{\L} + \card{\HH}$, this is
			\begin{equation}
			2\size{G} \ge k\card{G} + 2\size{\HH} - k\card{\HH} - 2\beta(\L).
			\label{FirstTwoG}
			\end{equation}
			Let $M$ be the maximum of $\size{I, V(G) \setminus I}$ over all independent sets $I$ of $G$ with $I \subseteq \HH$. Then
			\begin{equation*}
				\mic(G) \ge M + (k-1)\beta(\L).
			\end{equation*}
			Applying Lemma \ref{ConsantListMicStrength} gives
			\begin{equation}
			2\size{G} \ge (k-2)\card{G} + M + (k-1)\beta(\L) + 1.
			\label{SecondTwoG}
			\end{equation}				
			Adding twice \ref{SecondTwoG} to $k-1$ times \ref{FirstTwoG} gives
			\begin{equation*}
				(k+1)(2\size{G}) \ge (k(k-1) + 2(k-2))\card{G} + 2M + 2 + 2(k-1)\size{\HH} - k(k-1)\card{\HH}.
			\end{equation*}
			Hence
			\begin{equation}
			2\size{G} \ge \frac{k^2 + k -4}{k+1}\card{G} + \frac{2(M + (k-1)\size{\HH} + 1) - k(k-1)\card{\HH}}{k+1}.
			\label{ThirdTwoG}
			\end{equation}
			Let $\C$ be the components of $G[\HH]$.  Then $\alpha(C) \ge \frac{\card{C}}{\chi(C)}$ for all $C \in \C$.  Whence
			\begin{equation}
			  M + (k-1)\size{\HH} \ge \sum_{C \in \C} k\frac{\card{C}}{\chi(C)} + (k-1)\size{C}.
			  \label{Mbound}
			\end{equation}
			
			If $\L = \emptyset$, then $G$ has average degree at least $k \ge k-1 + \frac{k-3}{(k-1)^2}$.  So, assume $\L \ne \emptyset$.  Then $G[\HH]$ is $(k-1)$-colorable by $k$-list-criticality of $G$. In particular, $\chi(C) \le k-1$ for every $C \in \C$.
			We claim that for every $C \in \C$,
			\begin{equation}
			 k\frac{\card{C}}{\chi(C)} + (k-1)\size{C} \ge (k - \frac12)\card{C}.
			 \label{KFC}
			\end{equation}
			If $C \in \C$ is not a tree, then $\size{C} \ge \card{C}$ and hence $k\frac{\card{C}}{\chi(C)} + (k-1)\size{C} \ge (k - \frac12)\card{C}$.  If $C$ is a tree, then $\chi(C) \le 2$ and hence 
			$k\frac{\card{C}}{\chi(C)} + (k-1)\size{C} \ge k\frac{\card{C}}{2} + (k-1)(\card{C} - 1) \ge (k-\frac12)\card{C}$ unless $\card{C} = 1$.  This proves \ref{KFC} since the bound is trivially satisfied when $\card{C} = 1$.
			
			Now combining \ref{ThirdTwoG}, \ref{Mbound} and \ref{KFC} gives 
			\begin{equation}
			2\size{G} \ge \frac{k^2 + k -4}{k+1}\card{G} - \frac{(k^2 - 3k + 1)\card{\HH} - 2}{k+1}.
			\label{FourthTwoG}
			\end{equation}
			Since,
			\begin{equation*}
			  \card{\HH} \le 2\size{G} - (k-1)\card{G},
			\end{equation*}
			after some algebra, \ref{FourthTwoG} implies
			\begin{equation*}
				2\size{G} \ge \parens{k-1 + \frac{k-3}{k^2 -2k + 2}}\card{G} + \frac{2}{k^2 -2k + 2}.
			\end{equation*}
			That proves the theorem.
\end{proof}

\begin{problem}
The right side of equation (9) in the above proof can be improved to $k\card{C}$ unless $C$ is a $K_2$ where both vertices have degree $k$ in $G$.  If these $K_2$'s could be handled, the average degree bound would improve to $k-1 + \frac{k-3}{(k-1)^2}$.  Handle the $K_2$'s.
\end{problem}

\bibliographystyle{amsplain}
\bibliography{GraphColoring1}
\end{document} 
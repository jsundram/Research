\documentclass[12pt]{article}
\usepackage{amsmath, amsthm, amssymb}
\usepackage{tkz-graph}
\usepackage{marginnote}
\usepackage{verbatim}
\usepackage[top=1.0in, bottom=1.0in, left=1.0in, right=1.0in]{geometry}
\usepackage{color}
\pagestyle{plain}

\usepackage[backref=page]{hyperref}

\usepackage{sectsty}
\allsectionsfont{\sffamily}

\setcounter{secnumdepth}{5}
\setcounter{tocdepth}{5}

\makeatletter
\newtheorem*{rep@theorem}{\rep@title}
\newcommand{\newreptheorem}[2]{
\newenvironment{rep#1}[1]{
 \def\rep@title{#2 \ref{##1}}
 \begin{rep@theorem}}
 {\end{rep@theorem}}}
\makeatother

\theoremstyle{plain}
\newtheorem{thm}{Theorem}[section]
\newreptheorem{thm}{Theorem}
\newtheorem{prop}[thm]{Proposition}
\newreptheorem{prop}{Proposition}
\newtheorem{lem}[thm]{Lemma}
\newreptheorem{lem}{Lemma}
\newtheorem{conjecture}[thm]{Conjecture}
\newreptheorem{conjecture}{Conjecture}
\newtheorem{cor}[thm]{Corollary}
\newreptheorem{cor}{Corollary}
\newtheorem{prob}[thm]{Problem}

\newtheorem*{KernelLemma}{Kernel Lemma}
\newtheorem*{BK}{Borodin-Kostochka Conjecture}
\newtheorem*{BK2}{Borodin-Kostochka Conjecture (restated)}
\newtheorem*{Reed}{Reed's Conjecture}
\newtheorem*{ClassificationOfd0}{Classification of $d_0$-choosable graphs}


\theoremstyle{definition}
\newtheorem{defn}{Definition}
\theoremstyle{remark}
\newtheorem*{remark}{Remark}
\newtheorem*{problem}{Problem}
\newtheorem{example}{Example}
\newtheorem*{question}{Question}
\newtheorem*{observation}{Observation}

\newcommand{\fancy}[1]{\mathcal{#1}}
\newcommand{\C}[1]{\fancy{C}_{#1}}


\newcommand{\IN}{\mathbb{N}}
\newcommand{\IR}{\mathbb{R}}
\newcommand{\G}{\fancy{G}}
\newcommand{\CC}{\fancy{C}}
\newcommand{\D}{\fancy{D}}
\newcommand{\T}{\fancy{T}}
\newcommand{\B}{\fancy{B}}
\renewcommand{\L}{\fancy{L}}
\newcommand{\HH}{\fancy{H}}

\newcommand{\inj}{\hookrightarrow}
\newcommand{\surj}{\twoheadrightarrow}

\newcommand{\set}[1]{\left\{ #1 \right\}}
\newcommand{\setb}[3]{\left\{ #1 \in #2 : #3 \right\}}
\newcommand{\setbs}[2]{\left\{ #1 : #2 \right\}}
\newcommand{\card}[1]{\left|#1\right|}
\newcommand{\size}[1]{\left\Vert#1\right\Vert}
\newcommand{\ceil}[1]{\left\lceil#1\right\rceil}
\newcommand{\floor}[1]{\left\lfloor#1\right\rfloor}
\newcommand{\func}[3]{#1\colon #2 \rightarrow #3}
\newcommand{\funcinj}[3]{#1\colon #2 \inj #3}
\newcommand{\funcsurj}[3]{#1\colon #2 \surj #3}
\newcommand{\irange}[1]{\left[#1\right]}
\newcommand{\join}[2]{#1 \mbox{\hspace{2 pt}$\ast$\hspace{2 pt}} #2}
\newcommand{\djunion}[2]{#1 \mbox{\hspace{2 pt}$+$\hspace{2 pt}} #2}
\newcommand{\parens}[1]{\left( #1 \right)}
\newcommand{\brackets}[1]{\left[ #1 \right]}
\newcommand{\DefinedAs}{\mathrel{\mathop:}=}

\newcommand{\mic}{\operatorname{mic}}
\newcommand{\AT}{\operatorname{AT}}
\newcommand{\col}{\operatorname{col}}
\newcommand{\ch}{\operatorname{ch}}
\newcommand{\type}{\operatorname{type}}
\newcommand{\nonsep}{\bar{S}}

\def\adj{\leftrightarrow}
\def\nonadj{\not\!\leftrightarrow}

\newcommand\restr[2]{{% we make the whole thing an ordinary symbol
  \left.\kern-\nulldelimiterspace % automatically resize the bar with \right
  #1 % the function
  \vphantom{\big|} % pretend it's a little taller at normal size
  \right|_{#2} % this is the delimiter
  }}

\def\D{\fancy{D}}
\def\C{\fancy{C}}
\def\A{\fancy{A}}
\def\chil{{\chi_\ell}}
\def\chiol{\chi_{\rm{OL}}}

\newcommand{\case}[2]{{\bf Case #1.}~{\it #2}~~}
\newcommand{\aside}[1]{\marginnote{\scriptsize{#1}}[0cm]}
\newcommand{\aaside}[2]{\marginnote{\scriptsize{#1}}[#2]}

\title{better bound for edges in 4-list-critical graphs}

\begin{document}
\maketitle
\begin{abstract}
\end{abstract}

\section{Introduction}
For a graph $G$ and disjoint $A, B \subseteq V(G)$, let $\size{A,B}$ be the number of edges between $A$ and $B$.

\begin{defn} The \emph{maximum independent cover number }of a graph $G$
	is the maximum $\mic(G)$ of $\size{I, V(G) \setminus I}$ over all independent sets $I$
	of $G$. 
\end{defn}

\begin{thm}\label{ConsantListMicStrength} 
	Every OC-irreducible graph $G$ satisfies
	$$\mic(G)\leq2\size{G}-(\delta(G)-1)\card{G}-1.$$
\end{thm}

\section{initial improvement}
Let $G$ be OC-irreducible.  Let $\L$ be the subgraph of $G$ induced on the vertices of degree $\delta \DefinedAs \delta(G)$. Let $\HH$ be $G - V(\L)$.  Let $\beta$ be the maximum size of an independent set $A \subseteq V(\L)$ such that each $v \in A$ has no neighbors in $V(\HH)$.  Let $\mic_G(\HH)$ be the maximum of $\size{I, V(G) \setminus I}$ over all independent sets $I$ fo $G$ with $I \subseteq V(G) \setminus \L$.  Then

\begin{observation}
	$\mic(G) \ge \mic_G(\HH) + \delta\beta$.
\end{observation}

We need a couple bounds on $\size{\HH, \L}$.  

\begin{observation}
	$\size{\HH, \L} = \delta\card{\L} - 2\size{L}$.
\end{observation}

\begin{lem}
	$\size{\HH, \L} = \delta\card{\HH} - 2\size{\HH} + 2\size{G} - \delta\card{G}$.
\end{lem}
\begin{proof}
	$\size{\HH, \L} = -2\size{\HH} + \sum_{v \in V(\HH)} d_G(v) = \delta\card{\HH} - 2\size{\HH} + \sum_{v \in V(\HH)} \parens{d_G(v) - \delta} = \delta\card{\HH} - 2\size{\HH} + \sum_{v \in V(G)} \parens{d_G(v) - \delta}$.
\end{proof}

\begin{lem}
	 If $T$ is a Gallai tree with max degree $\delta$, not equal to $K_\delta$, then
	 \[2||T|| \le (\delta - 1)|T| + 2\beta(T).\]
\end{lem}

\begin{lem}
	$\size{\HH, \L} \ge \card{\L} - 2\beta$.
\end{lem}

\begin{lem}
	\[2\size{G} \ge \delta\card{G} + \card{\L} + 2\size{\HH} - \delta\card{\HH} - 2\beta.\]
\end{lem}

\begin{lem}
	\[2\size{G} \ge (\delta-1)\card{G} + \mic_G(\HH) + \delta\beta + 1.\]
\end{lem}

\begin{lem}
	\[(2 + \delta)(2\size{G}) \ge (\delta^2 + 3\delta - 2)\card{G} + 2\mic_G(\HH) + 2 + 2\delta\size{\HH} - \delta(\delta+1)\card{\HH}.\]
\end{lem}

\begin{lem}
	$\mic_G(\HH) \ge \frac{\delta + 1}{\delta}\card{\HH}$.
\end{lem}

\begin{lem}
	$\mic_G(\HH) + \delta\size{\HH} \ge (\delta + 1)\card{\HH}$.
\end{lem}

\begin{lem}
	\[(2 + \delta)(2\size{G}) \ge (\delta^2 + 3\delta - 2)\card{G} + 2 - (\delta - 2)(\delta + 1)\card{\HH}.\]
\end{lem}

\begin{lem}
	$2\size{G} \ge \delta\card{G} + \card{\HH}$.
\end{lem}

\begin{lem}
	\[(\delta + 2 + (\delta - 2)(\delta+1))(2\size{G}) \ge (\delta^2 + 3\delta - 2 + \delta(\delta - 2)(\delta+1))\card{G} + 2\]
\end{lem}


\begin{lem}
	\[d(G) > \delta + \frac{1}{\delta} - \frac{2}{\delta^2}\]
\end{lem}



\bibliographystyle{amsplain}
\bibliography{GraphColoring1}
\end{document} 
\documentclass[12pt]{article}
\usepackage{amsmath, amsthm, amssymb}
\usepackage{hyperref}
\usepackage{verbatim}
\usepackage[top=1.0in, bottom=1.0in, left=1.0in, right=1.0in]{geometry}

\pagestyle{plain}

\usepackage{sectsty}
\allsectionsfont{\sffamily}

\setcounter{secnumdepth}{5}
\setcounter{tocdepth}{5}

\makeatletter
\newtheorem*{rep@theorem}{\rep@title}
\newcommand{\newreptheorem}[2]{
\newenvironment{rep#1}[1]{
 \def\rep@title{#2 \ref{##1}}
 \begin{rep@theorem}}
 {\end{rep@theorem}}}
\makeatother

\theoremstyle{plain}
\newtheorem{thm}{Theorem}[section]
\newreptheorem{thm}{Theorem}
\newtheorem{prop}[thm]{Proposition}
\newreptheorem{prop}{Proposition}
\newtheorem{lem}[thm]{Lemma}
\newreptheorem{lem}{Lemma}
\newtheorem{conjecture}[thm]{Conjecture}
\newreptheorem{conjecture}{Conjecture}
\newtheorem{cor}[thm]{Corollary}
\newreptheorem{cor}{Corollary}
\newtheorem{prob}[thm]{Problem}

\newtheorem*{KernelLemma}{Kernel Lemma}
\newtheorem*{Theorem}{Theorem}
\newtheorem*{Lemma}{Lemma}
\newtheorem*{BK2}{Borodin-Kostochka Conjecture (restated)}
\newtheorem*{Reed}{Reed's Conjecture}
\newtheorem*{ClassificationOfd0}{Classification of $d_0$-choosable graphs}


\theoremstyle{definition}
\newtheorem{defn}{Definition}
\theoremstyle{remark}
\newtheorem*{remark}{Remark}
\newtheorem*{problem}{Problem}
\newtheorem{example}{Example}
\newtheorem*{question}{Question}
\newtheorem*{observation}{Observation}

\newcommand{\fancy}[1]{\mathcal{#1}}
\newcommand{\C}[1]{\fancy{C}_{#1}}
\newcommand{\IN}{\mathbb{N}}
\newcommand{\IR}{\mathbb{R}}
\newcommand{\G}{\fancy{G}}
\newcommand{\CC}{\fancy{C}}
\newcommand{\D}{\fancy{D}}
\newcommand{\T}{\fancy{T}}
\newcommand{\B}{\fancy{B}}
\renewcommand{\L}{\fancy{L}}
\newcommand{\HH}{\fancy{H}}

\newcommand{\inj}{\hookrightarrow}
\newcommand{\surj}{\twoheadrightarrow}

\newcommand{\set}[1]{\left\{ #1 \right\}}
\newcommand{\setb}[3]{\left\{ #1 \in #2 \mid #3 \right\}}
\newcommand{\setbs}[2]{\left\{ #1 \mid #2 \right\}}
\newcommand{\card}[1]{\left|#1\right|}
\newcommand{\size}[1]{\left\Vert#1\right\Vert}
\newcommand{\ceil}[1]{\left\lceil#1\right\rceil}
\newcommand{\floor}[1]{\left\lfloor#1\right\rfloor}
\newcommand{\func}[3]{#1\colon #2 \rightarrow #3}
\newcommand{\funcinj}[3]{#1\colon #2 \inj #3}
\newcommand{\funcsurj}[3]{#1\colon #2 \surj #3}
\newcommand{\irange}[1]{\left[#1\right]}
\newcommand{\join}[2]{#1 \mbox{\hspace{2 pt}$\ast$\hspace{2 pt}} #2}
\newcommand{\djunion}[2]{#1 \mbox{\hspace{2 pt}$+$\hspace{2 pt}} #2}
\newcommand{\parens}[1]{\left( #1 \right)}
\newcommand{\brackets}[1]{\left[ #1 \right]}
\newcommand{\DefinedAs}{\mathrel{\mathop:}=}

\newcommand{\mic}{\operatorname{mic}}
\newcommand{\AT}{\operatorname{AT}}
\newcommand{\col}{\operatorname{col}}
\newcommand{\ch}{\operatorname{ch}}

\def\adj{\leftrightarrow}
\def\nonadj{\not\!\leftrightarrow}

\newcommand\restr[2]{{% we make the whole thing an ordinary symbol
  \left.\kern-\nulldelimiterspace % automatically resize the bar with \right
  #1 % the function
  \vphantom{\big|} % pretend it's a little taller at normal size
  \right|_{#2} % this is the delimiter
  }}

\def\D{\fancy{D}}
\def\C{\fancy{C}}
\def\A{\fancy{A}}
\def\L{\fancy{L}}
\def\H{\fancy{H}}

\newcommand{\case}[2]{{\bf Case #1.}~{\it #2}~~}
\newcommand{\claim}[2]{{\bf Claim #1.}~{\it #2}~~}
\newcommand{\subclaim}[2]{{\bf Subclaim #1.}~{\it #2}~~}

\title{graph theory notes\thanks{clarifications, errors, simplifications $\Rightarrow$ \texttt{landon.rabern@gmail.com}}\\ \bigskip
Stiebitz's proof of Gallai's conjecture on the number of components in the high and low vertex subgraphs of critical graphs}
\date{}
\begin{document}
\maketitle

Tibor Gallai conjectured the following in 1963 \cite{gallai1963kritische, gallai1963kritische2} and Michael Stiebitz proved it in 1982 \cite{stiebitz1982proof}.  For a graph $G$,
let $\L(G)$ be the subgraph of $G$ induced on the vertices of degree $\delta(G)$ and let $\H(G)$ be the subgraph of $G$ induced on the vertices of degree larger than $\delta(G)$.

\begin{Theorem}[Stiebitz]
If $G$ is a color-critical graph with $\delta(G) = \chi(G) - 1$, then $\H(G)$ has at most as many components as $\L(G)$.
\end{Theorem}

\begin{Lemma}
Let $G$ be a connected graph and $\emptyset \ne X \subseteq V(G)$ such that
\begin{itemize}
\item $d_G(x) \le k - 1$ for all $x \in X$; and
\item for each component $C$ of $G-X$, we have $\chi(G - V(C)) \le k - 1$; and
\item $G[X]$ has $\ell$ components and $G-X$ has at least $\ell + 1$ components.
\end{itemize}
If $G-X$ is the disjoint union of (possibly not connected) graphs $M_1, \ldots, M_{\ell + 1}$ and $f_i$ is a $(k-1)$-coloring of $M_i$ for each $i \in \irange{\ell + 1}$, 
then there are permutations $\pi_1, \ldots, \pi_{\ell + 1}$ of $\irange{k-1}$ such that the $(k-1)$-coloring of $G-X$ given by $(\pi_1 \circ f_1) \cup \cdots \cup (\pi_{\ell + 1} \circ f_{\ell + 1})$ extends to a $(k - 1)$-coloring of $G$.
\end{Lemma}
\begin{proof}
Suppose the lemma is false and choose a counterexample $G$ and nonempty $X \subseteq V(G)$ so that $|X|$ is as small as possible.  So, $G-X$ is the disjoint union of graphs $M_1, \ldots, M_{\ell + 1}$ and we have $(k-1)$-colorings $f_i$ of $M_i$ for each $i \in \irange{\ell + 1}$ so that no permutations allow us to extend to a $(k - 1)$-coloring of $G$.

\claim{1}{Each component of $G[X]$ has edges to at least two of the $M_i$.}
Suppose to the contrary that we have a component $C$ of $G[X]$ that has edges to at most one of the $M_i$.  Then, since $G$ is connected, we must have $\ell \ge 2$. But now the hypotheses of the lemma are satisfied with $X' = X \setminus V(C)$ in place of $X$, so by minimality of $|X|$ we get permutations that allow us to extend to a $(k - 1)$-coloring of $G$, a contradiction.

\claim{2}{Each non-separating vertex in $G[X]$ has neighbors in at least two of the $M_i$.}
Suppose to the contrary that we have a component $C$ of $G[X]$ and $x \in V(C)$ a non-separating vertex that has neighbors in at most one of the $M_i$.  Then, by Claim 1, we must have $|C| \ge 2$.  But then $x$ has at most $k-2$ neighbors in $G-X$, so we can greedily complete any $(k-1)$-coloring of $G-X$ to $G-X'$ where $X' = X \setminus \set{x}$.  So, the hypotheses of the lemma are satisfied with $X'$ in place of $X$.  Again, by minimality of $|X|$, we get permutations that allow us to extend to a $(k - 1)$-coloring of $G$, a contradiction.

\claim{3}{The lemma is true.}
Pick a component $C$ in $G[X]$ and a non-separating vertex $x \in V(C)$.  By Claim 2 and symmetry, we may assume that $x$ has neighbors $y_1, y_2$ in $M_1, M_2$ respectively.  Let $G' = G - V(C)$ and $X' = X \setminus V(C)$.  Then $G'$ is the disjoint union of the $\ell$ graphs $M_1 \cup M_2, M_3, \ldots, M_{\ell + 1}$.  Let $\tau$ be a permutation of $\irange{k-1}$ such that $(\tau \circ f_2)(y_2) = f_1(y_1)$ and let $f_* = f_1 \cup (\tau \circ f_2)$. WHY $G'$ CONNECTED? By minimality of $|X|$, we can apply the lemma to $G'$ with $M_1 \cup M_2, M_3, \ldots, M_{\ell + 1}$ and colorings $f_*, f_3, \ldots, f_{\ell +1}$ to get permutations $\pi_*, \pi_3, \ldots, \pi_{\ell+1}$ such that the $(k-1)$-coloring of $G'-X'$ given by $(\pi_* \circ f_*) \cup (\pi_3 \circ f_3) \cup \cdots \cup (\pi_{\ell + 1} \circ f_{\ell + 1})$ extends to a $(k - 1)$-coloring of $G'$.  But this is the same as the $(k-1)$-coloring $(\pi_* \circ f_1) \cup (\pi_* \circ \tau \circ f_2) \cup (\pi_3 \circ f_3) \cup \cdots \cup (\pi_{\ell + 1} \circ f_{\ell + 1})$, so using the permutations $\pi_*, \pi_* \circ \tau, \pi_3, \ldots, \pi_{\ell + 1}$ we get a coloring of $G - X$ that extends to $G - V(C)$.  But in this coloring, $y_1$ and $y_2$ receive the same color. This means that $x$ has $k - 1 - (d_G(x) - d_C(x)) + 1 \ge d_C(x) + 1$ colors available and each other vertex $v$ in $C$ has $k - 1 - (d_G(v) - d_C(v)) + 1 \ge d_C(v) \ge d_C(v)$ colors available.  So, coloring $C$ greedily in order of decreasing distance from $x$ gives an extension to a $(k - 1)$-coloring of $G$, a contradiction.
\end{proof}

\bibliographystyle{amsplain}
\bibliography{GraphColoring}
\end{document}

 

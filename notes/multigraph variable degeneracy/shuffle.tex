\documentclass[12pt]{amsart}
\usepackage{amsmath, amsthm, amssymb}
\usepackage[top=1.25in, bottom=1.25in, left=1.0in, right=1.0in]{geometry}
\usepackage{hyperref}
\usepackage{color}
\usepackage{verbatim}

\makeatletter
\newtheorem*{rep@theorem}{\rep@title}
\newcommand{\newreptheorem}[2]{
\newenvironment{rep#1}[1]{
 \def\rep@title{#2 \ref{##1}}
 \begin{rep@theorem}}
 {\end{rep@theorem}}}
\makeatother

\theoremstyle{plain}
\newtheorem{thm}{Theorem}
\newreptheorem{thm}{Theorem}
\newtheorem{prop}[thm]{Proposition}
\newreptheorem{prop}{Proposition}
\newtheorem{lem}[thm]{Lemma}
\newreptheorem{lem}{Lemma}
\newtheorem{conj}[thm]{Conjecture}
\newreptheorem{conj}{Conjecture}
\newtheorem{cor}[thm]{Corollary}
\newreptheorem{cor}{Corollary}
\newtheorem{prob}[thm]{Problem}
\theoremstyle{definition}
\newtheorem{defn}{Definition}
\theoremstyle{remark}
\newtheorem*{remark}{Remark}
\newtheorem{example}{Example}
\newtheorem*{question}{Question}
\newtheorem*{observation}{Observation}

\title{Shuffle}
\author{}

\newcommand{\fancy}[1]{\mathcal{#1}}
\newcommand{\C}{\fancy{C}}
\newcommand{\IN}{\mathbb{N}}
\newcommand{\IR}{\mathbb{R}}
\newcommand{\G}{\fancy{G}}
\newcommand{\LB}{\mathcal{L}_B}
\newcommand{\col}{{\textrm{col}}}
\newcommand{\chil}{{\chi_{\ell}}}
\newcommand{\chiol}{{\chi_{OL}}}

\newcommand{\inj}{\hookrightarrow}
\newcommand{\surj}{\twoheadrightarrow}

\newcommand{\set}[1]{\left\{ #1 \right\}}
\newcommand{\setb}[3]{\left\{ #1 \in #2 \mid #3 \right\}}
\newcommand{\setbs}[2]{\left\{ #1 \mid #2 \right\}}
\newcommand{\card}[1]{\left|#1\right|}
\newcommand{\size}[1]{\left\Vert#1\right\Vert}
\newcommand{\ceil}[1]{\left\lceil#1\right\rceil}
\newcommand{\floor}[1]{\left\lfloor#1\right\rfloor}
\newcommand{\func}[3]{#1\colon #2 \rightarrow #3}
\newcommand{\funcinj}[3]{#1\colon #2 \inj #3}
\newcommand{\funcsurj}[3]{#1\colon #2 \surj #3}
\newcommand{\irange}[1]{\left[#1\right]}
\newcommand{\join}[2]{#1 \mbox{\hspace{2 pt}$\ast$\hspace{2 pt}} #2}
\newcommand{\djunion}[2]{#1 \mbox{\hspace{2 pt}$+$\hspace{2 pt}} #2}
\newcommand{\parens}[1]{\left( #1 \right)}
\newcommand{\brackets}[1]{\left[ #1 \right]}
\newcommand{\DefinedAs}{\mathrel{\mathop:}=}
\newcommand{\im}{\operatorname{im}}
\newcommand{\mic}{\operatorname{mic}}
\newcommand{\pot}{\operatorname{Pot}}

\begin{document}
\maketitle

Call multigraph \emph{standard} if it is loopless with maximum multiplicity at most $2$.

\begin{defn}
For $k \ge 1$, a standard multigraph is $k$-bad if
\begin{itemize}
\item $k = 1$ and $G$ is a simple cycle; or
\item $k = 2$ and $G$ is an odd cycle with all multiplicities equal to $2$; or
\item $G$ is a $K_{k+1}$ with all multiplicities $2$; or
\item $G$ is $K_{\Delta(G) + 1}$.
\end{itemize}
\end{defn}

\begin{lem}
Let $G$ be a connected standard multigraph. If $\Delta(G) \le 2k$ and $G$ is not $k$-bad, then $V(G)$ can be partitioned into $k$ (possibly empty) sets $V_1, \ldots, V_k$ such that $G[V_i]$ induces a disjoint union of simple paths.
\end{lem}
\begin{proof}
Choose a partition $P = \{P_1, \ldots, P_k\}$ of $V(G)$ 
\begin{enumerate}
\item minimizing $\sum_i \size{G[P_i]}$,
\item subject to (1), minimizing the total number of cycles within parts.
\end{enumerate}

For all $i \in \irange{k}$, we have $\Delta(G[P_i]) \le 2$ since if $v \in P_i$ has degree $>2$ in $P_i$ then it has degree $<2$ in some other $P_j$, moving $v$ to $P_j$ violates (1).

We are done unless some $G[P_i]$ contains a cycle, so by symmetry we may assume that $G[P_1]$ contains a cycle $C_0$.  Put $P^0 = P$ and choose $v_0$ on $C_0$. Then, by the minimization in (1), $v_0$ has degree $2$ is every other part.  So, we can move $v_0$ to any other part and preserve (1). Let $P^1$ be the partition formed by moving $v_0$ to $P^0_1$.  Since moving $v_0$ destroys $C_0$, it must be that $v_0$ is in a cycle component in $G[P^1_1]$, call this $C_1$.  Now pick $v_1 \in P^1_1$ adjacent to $v_0$ and move it to $P^1_0$ to form $P^2$.  Let $C_2$ be the created cycle.  Since $G$ is finite, continuing this way, there is a least $t$ such that for some $s < t$ we have $C_s - v_s = C_t - v_{t-1}$.

By symmetry, we may assume that $C_s \subseteq G[P^s_0]$.  Let $v_sx_s$ and $v_sy_s$ be edges in $G[P^s_0]$ (where we could have $x_s = y_s$ for a digon). Then, from the cycle in $G[P^t_0]$, there must also be edges $v_{t-1}x_s$ and $v_{t-1}y_s$.  Let $v_{t-2}v_{t-1}$ and $v_{t-2}x_{t-1}$ be edges in $G[P^{t-1}_1]$. In $P^s$, consider moving $x_s$ to $P^s_1$.  Then, there must be edges $x_sv_{t-1}$ and $x_sx_{t-1}$.

Suppose $C_{t-1}$ is a digon.  Then $v_{t-1} = x_{t-1}$.  But then the edge $v_{t-1}x_s$ is a double edge and hence $x_s = y_s$ showing that $C_s$ is a digon as well.  Now consider the fact that the sequence $P^s, \ldots, P^t$ is cyclical, so we can start it from any point and even run it in reverse.  This shows that either all or none of $C_s, \ldots, C_t$ are digons.  When they are all digons, we have an odd cycle with all edges doubled.  In particular, $s = 0$.

Now, suppose $C_{t-1}$ is not a digon.  Further, suppose $C_{t-1}$ is not a triangle and let $y_{t-1}$ be a neighbor of $x_{t-1}$ different from $v_{t-2}$ and $v_{t-1}$. In $P^t$, move $x_s$ to $P^t_1$.  Since we have the edge $x_sx_{t-1}$, we see that $x_{t-1}$ has different neighbors $v_{t-2}, y_{t-1}$ and $x_s$, so there is some part we can move $x_{t-1}$ to violate (1).  Hence $C_{t-1}$ is a triangle.  As before, we can start the sequence $P^s, \ldots, P^t$ anywhere to conclude that each of $C_s, \ldots, C_t$ is a triangle. Since $x_sx_{t-1}$ is an edge, moving $x_s$ to $P^t_1$ shows that $x_sv_{t-2}$ is an edge.  Similarly, $y_sv_{t-2}$ is an edge.  But the only way for that to happen is for $v_s$ to be $v_{t-2}$.  So $t = s + 2$.  Suppose $s > 0$, then by symmetry, we may assume that $v_{s-1} = x_s$.  But then $x_s$ has neighbors in two different components in $P^s_1$, so moving $x_s$ to $P^s_1$ does not create a cycle, violating (2).  Hence $s = 0$, $C_0$ is a triangle joined to $C_1 - v_0$ which is a $K_2$.

So, the possible sequences come in two types, digon odd cycles or length one joins.  Suppose $C_0$ is not a digon.  Then applying the above argument to each part $P_1, \ldots, P_k$, get that $C_0$ is a triangle joined to a $K_2$ in each of these parts.  But we can move a vertex from $C_0$ to $P_1$ to form a triangle there and run the same argument to conclude that the $K_2$ in $P_1$ is joined to the rest of the $K_2$'s, playing this game for all parts, we conclude that all the $K_2$'s are joined to each other giving a $K_{\Delta(G) + 1}$.

Instead, suppose $C_0$ is a digon and let $v$ be on $C_0$.  Then, by the above argument, moving $v$ to any other part, we must create a digon.  But, we can repeat this argument after moving $v$ to any other part with $v$'s neighbors and all their neighbors.  We conclude that $v$'s component has all doubled edges.  Since $G$ is connected, $G$ has all doubled edges.  Since $\Delta(G) \le 2k$, Brooks' theorem gives a $k$-coloring of $G$ unless $G$ is a $K_{k+1}$ with all multiplicities $2$.
\end{proof}


\bibliographystyle{plain}
\bibliography{GraphColoring}
\end{document}



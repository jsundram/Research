\documentclass[12pt]{amsart}
\usepackage{amsmath, amsthm, amssymb}
\usepackage[top=1.25in, bottom=1.25in, left=1.0in, right=1.0in]{geometry}

\allowdisplaybreaks
\pagestyle{headings}

\makeatletter
\newtheorem*{rep@theorem}{\rep@title}
\newcommand{\newreptheorem}[2]{
\newenvironment{rep#1}[1]{
 \def\rep@title{#2 \ref{##1}}
 \begin{rep@theorem}}
 {\end{rep@theorem}}}
\makeatother

\theoremstyle{plain}
\newtheorem{thm}{Theorem}
\newreptheorem{thm}{Theorem}
\newtheorem{prop}[thm]{Proposition}
\newreptheorem{prop}{Proposition}
\newtheorem{lem}[thm]{Lemma}
\newreptheorem{lem}{Lemma}
\newtheorem{conjecture}[thm]{Conjecture}
\newreptheorem{conjecture}{Conjecture}
\newtheorem{cor}[thm]{Corollary}
\newreptheorem{cor}{Corollary}
\newtheorem{prob}[thm]{Problem}
\theoremstyle{definition}
\newtheorem{defn}{Definition}
\theoremstyle{remark}
\newtheorem*{remark}{Remark}
\newtheorem{example}{Example}
\newtheorem*{question}{Question}
\newtheorem*{observation}{Observation}

\newcommand{\fancy}[1]{\mathcal{#1}}
\newcommand{\C}[1]{\fancy{C}_{#1}}
\newcommand{\IN}{\mathbb{N}}
\newcommand{\IR}{\mathbb{R}}
\newcommand{\G}{\fancy{G}}

\newcommand{\inj}{\hookrightarrow}
\newcommand{\surj}{\twoheadrightarrow}

\newcommand{\set}[1]{\left\{ #1 \right\}}
\newcommand{\setb}[3]{\left\{ #1 \in #2 \mid #3 \right\}}
\newcommand{\setbs}[2]{\left\{ #1 \mid #2 \right\}}
\newcommand{\card}[1]{\left|#1\right|}
\newcommand{\size}[1]{\left\Vert#1\right\Vert}
\newcommand{\ceil}[1]{\left\lceil#1\right\rceil}
\newcommand{\floor}[1]{\left\lfloor#1\right\rfloor}
\newcommand{\func}[3]{#1\colon #2 \rightarrow #3}
\newcommand{\funcinj}[3]{#1\colon #2 \inj #3}
\newcommand{\funcsurj}[3]{#1\colon #2 \surj #3}
\newcommand{\irange}[1]{\left[#1\right]}
\newcommand{\join}[2]{#1 \mbox{\hspace{2 pt}$\ast$\hspace{2 pt}} #2}
\newcommand{\djunion}[2]{#1 \mbox{\hspace{2 pt}$+$\hspace{2 pt}} #2}
\newcommand{\parens}[1]{\left( #1 \right)}
\newcommand{\brackets}[1]{\left[ #1 \right]}
\newcommand{\DefinedAs}{\mathrel{\mathop:}=}
\newcommand{\im}{\operatorname{im}}
\newcommand{\ex}{\operatorname{E}}

\title{List Borodin-Kostochka for large $\Delta$}
%\author{Landon Rabern}
\begin{document}

\maketitle
\section{The setup}
\begin{thm}
There exists $\Delta_0$ such that every graph $G$ with $\chi_l(G) \geq \Delta(G)
\geq \Delta_0$ contains a $K_{\Delta(G)}$.
\end{thm}

Suppose the theorem is false and choose a counterexample $G$ minimizing
$\card{G}$.  Put $\Delta \DefinedAs \Delta(G)$ and let $L$ be a bad
$(\Delta - 1)$-assignment on $G$.  Then, by minimality of $\card{G}$, any proper
induced subgraph of $G$ is $L$-colorable.  In particular, every vertex has
degree either $\Delta$ or $\Delta-1$, we call these \emph{high} and \emph{low}
vertices respectively. We need the following concept.

\begin{defn}
A graph $H$ is called \emph{$d_1$-choosable} if $H$ is $f$-choosable where $f(v)
\DefinedAs d(v) - 1$.
\end{defn}

If $G$ had an induced $d_1$-choosable subgraph $H$, then we could
$L$-color $G - H$ by minimality and then complete the $L$-coloring to all of
$G$.  So, $G$ has no $d_1$-choosable induced subgraphs.  We will use results
from \cite{mules} where many graphs were shown to be $d_1$-choosable.

The proof strategy is the same as Reed's \cite{reed1999strengthening} for
chromatic number, except some more care must be taken when lists have small
intersection and $K_{\Delta-1}$'s require special attention.

\section{The decomposition}
For a vertex $v$, put $G_v \DefinedAs G[N(v)]$.

\begin{defn}
A clique with at least $\frac34 \Delta + 1$ vertices is called \emph{big}.
\end{defn}

\begin{defn}
A vertex $v$ is called \emph{sparse} if $\size{G_v} < \frac25 \Delta^2$.
\end{defn}

We'll use the following easy consequence of $d_1$-choosability lemmas in
\cite{mules} to show that nonsparse vertices are in a big clique.

\begin{lem}\label{neighborhood}
If $B$ is a graph with $\delta(B) \geq \frac{\card{B} + 1}{2}$ such that
$\join{K_1}{B}$ is not $d_1$-choosable, then $\omega(B) \geq \card{B} - 1$ or
$B = \join{E_3}{K_4}$.
\end{lem}

Note that the neighborhoods we will be looking at are huge, so the $B =
\join{E_3}{K_4}$ case will never happen here.

\begin{lem}
Any nonsparse vertex is contained in a big clique.
\end{lem}
\begin{proof}
Suppose we have a vertex $x$ which is not contained in a big clique. 
By applying Lemma \ref{neighborhood} repeatedly, we get a sequence $y_1,
\ldots, y_{\floor{\frac{\Delta}{4}}} \in N(x)$ such that 
\[\card{N(y_i) \cap (N(x) - \set{y_1, \ldots, y_{i-1}})} \leq \frac12 (\Delta +
1 - i).\]

\noindent Hence $x$ is sparse since

\[\size{G_x} \leq {\Delta \choose 2} -
\frac12\sum_{i=1}^{\floor{\frac{\Delta}{4}}} (\Delta - i) < \frac25 \Delta^2.\]
\end{proof}

We need another $d_1$-choosability lemma from \cite{mules}.

\begin{lem}\label{K6join}
If $B$ is a graph such that $\join{K_6}{B}$ is not $d_1$-choosable, then
$\omega(B) \geq \card{B} - 1$.
\end{lem}

Let $C_1, \ldots, C_t$ be the maximal big cliques in $G$.

\begin{lem}\label{MaxBig1}
If $C_i \cap C_j \neq \emptyset$ and $\card{C_i} \leq \card{C_j}$, then
$\card{C_i - C_j} \leq 1$.
\end{lem}
\begin{proof}
We have $\card{C_i \cap C_j} \geq \card{C_i} + \card{C_j} - (\Delta + 1) \geq
\frac{\Delta}{2} \geq 6$.  Now the lemma follows from Lemma \ref{K6join}. 
\end{proof}

\begin{lem}\label{MaxBig2}
No $C_i$ intersects two others.
\end{lem}
\begin{proof}
Say $C_i$ intersects $C_j$ and $C_k$.  By Lemma \ref{MaxBig1}, $\card{C_i \cap
C_j} \geq \frac34 \Delta$ and similarly $\card{C_i \cap
C_k} \geq \frac34 \Delta$.  Hence $\card{C_i \cap C_j \cap C_k} \geq \frac12
\Delta \geq 6$ and Lemma \ref{K6join} gives a contradiction.
\end{proof}

Putting these lemmas together we can partition the nonsparse vertices of $G$
into $D_1, \ldots, D_r$ where each $D_i$ is either a maximal big clique $C_i$ or
$C_i \cup \set{x_i}$ where $x_i$ has at least $\frac34 \Delta$ neighbors in
$C_i$. Put $K_i \DefinedAs N(x_i) \cap C_i$ in this latter case and $K_i
\DefinedAs C_i$ in the former.  Let $S$ be the set of all sparse vertices.

\section{The random procedure}
For each vertex $v$, pick $c \in L(v)$ at random to get a possibly improper
coloring $\zeta$ of $G$ from $L$.  Put $U \DefinedAs \setb{x}{V(G)}{\zeta(x) =
\zeta(y) \text{ for some $y \in N(x)$}}$.  Put $H \DefinedAs G - U$, $F
\DefinedAs G[U]$ and let $\pi$ be $\zeta$ restricted to $V(H)$.  We refer
to $V(H)$ as the \emph{colored} vertices and $V(F)$ as the \emph{uncolored}
vertices.  Also, let $J$ be the resulting list assignment on $F$; that is, $J(x)
\DefinedAs L(x) - \bigcup_{y \in N(x) \cap V(H)} \pi(y)$ for $x \in V(F)$.

\begin{defn}
A vertex in $v \in V(G)$ is called \emph{safe} if it is colored or $\card{J(v)}
\geq d_F(v) + 1$.
\end{defn}

Note that if every vertex is safe, then we can easily complete the $L$-coloring
to all of $G$.  Our goal will be to show that the random procedure will, with
positive probability, produce a partial coloring where every sparse vertex is
safe and the uncolored nonsparse vertices satisfy conditions that will allow the
coloring to be completed.  Now we make this precise.  Consider the following
events:

\begin{itemize}
  \item $S_v$, for $v \in S$: the event that $v$ is not safe.
  \item $E_i$, for $i \in \irange{r}$ where $\card{C_i} \leq \Delta - 2$: the event that $C_i$ does not contain two uncolored safe vertices.
  \item $Q_i$, for $i \in \irange{r}$ where $\card{C_i} \leq \Delta - 2$: the event that $K_i$ does not contain two uncolored vertices.
  \item $F_i$, for $i \in \irange{r}$ where $\card{C_i} = \Delta - 1$, every $x
  \in G - C_i$ has $\card{N(x) \cap C_i} \leq \sqrt{\Delta}\log(\Delta)$ and
  there are at most $\log^2(\Delta)$ vertices $x \in G - C_i$ with $\card{N(x)
  \cap C_i} > \frac{\sqrt{\Delta}}{\log(\Delta)}$: the event that
  $C_i$ does not contain two uncolored safe vertices.
  \item $P_i$, for $i \in \irange{r}$ where $\card{C_i} = \Delta - 1$ and either some $x \in G
  - C_i$ has $\card{N(x) \cap C_i} > \sqrt{\Delta}\log(\Delta)$ or more than
  $\log^2(\Delta)$ vertices $x \in G - C_i$ have $\card{N(x) \cap C_i} >
  \frac{\sqrt{\Delta}}{\log(\Delta)}$: the event that every $x \in G - C_i$ has
  at most two ``good clumps'' in $K_i$.
\end{itemize}

It remains to define ``good clumps''.  To do so we need a lemma.

\begin{lem}\label{clumping}
Let $K$ be a $\Delta-1$ clique in $G$ and $x \in G - K$ with $\card{N(x) \cap
K} \geq 4$.  Then every vertex in $\card{N(x) \cap
K}$ is high and there is a partition $\set{Z_1, \ldots, Z_m}$
of $N(x) \cap K$ such that for each $i \in \irange{m}$ we have $\card{Z_i} \leq
5$ and $L(u) = L(v)$ for all $u, v \in Z_i$.  Moreover, $\card{L(v) - L(w)}
\leq 1$ for all $v, w \in N(x) \cap K$.
\end{lem}
\begin{proof}
Put $A \DefinedAs N(x) \cap K$ and $Q \DefinedAs G[\set{x} \cup K]$.  For any
$L$-coloring $\gamma$ of $G - Q$, let $L_\gamma$ be the resulting list
assignment on $Q$.

First, suppose there is an $L$-coloring $\gamma$ of $G - Q$ such that
$L_\gamma(u) \neq L_\gamma(v)$ for some $u, v \in A$.  Pick $y \in K - A$.  If
$L_\gamma(x) \cap L_\gamma(y) \neq \emptyset$, then coloring $x$ and $y$ the
same leaves a list assignment on $K - y$  which is completable by Hall's theorem.
Hence we must have $L_\gamma(x) \cap L_\gamma(y) = \emptyset$.  Thus $\card{L_\gamma(x) \cup L_\gamma(y)} \geq \Delta$.  Put $Pot(T) \DefinedAs \bigcup_{v \in T} L_\gamma(v)$ for $T \subseteq A$.  If there is $c \in (L_\gamma(x) \cup L_\gamma(y)) - Pot(A)$, then coloring $x$ and $y$ so that $c$ is used leaves a list assignment on $K-y$ which is completable by Hall's theorem.  In particular, we must have $\card{Pot(A)} \geq \Delta$. Now, if we color $x$ and $y$ arbitrarily we can complete the coloring unless there exists $T \subseteq A$ with $\card{T} = \card{A}-1$ and $\card{Pot(T)} \leq \Delta - 2$.  Thus we can pick a color in $L_\gamma(x) \cup L_\gamma(y)$ which is not in any of $T$'s lists giving a coloring that is again easily completable.

Therefore $L_\gamma(u) = L_\gamma(v)$ for all $u, v \in A$ for every
$L$-coloring $\gamma$ of $G - Q$.  In particular, no vertex of $A$ is
low and $\card{L(v) - L(w)} \leq 1$ for all $v, w \in A$.  Suppose there exists
$Z \subseteq A$ with $\card{Z} \geq 6$ such that $L(u) = L(v)$ for all $u, v \in Z$.  Then every $v
\in Z$ has exactly one neighbor $z_v$ in $G-Q$.  Put $N \DefinedAs
\setbs{z_v}{v \in Z}$.  If $\card{N} = 1$, then $G$ contains $\join{K_6}{E_3}$
violating Lemma \ref{K6join}.  If some $L$-coloring $\gamma$ of $G-Q$ assigned
two vertices of $N$ different colors, then $L_\gamma$ would give different lists
for two vertices of $A$, a contradiction.  Hence $N$ is an independent set and
adding an edge between two vertices of $N$ in $G-Q$ must create a $K_\Delta$ by
minimality of $\card{G}$.  By counting degrees this is plainly impossible for
$\card{N} \geq 3$.  For $\card{N} = 2$, both vertices have $\Delta-2$ neighbors
in $G-Q$ and one has at least $3$ vertices in $Z$, impossible.

Now taking maximal subsets of $A$ of vertices all having the same list gives the
desired partition.
\end{proof}

The $Z_i$ in the partition in Lemma \ref{clumping} are called \emph{clumps} of $x$ in $K$.
Note that there exists $Y$ such that for any $i \neq j$ we have $L(v) \cap L(z)
= Y$ for $v \in Z_i$ and $w \in Z_j$.  For $i \in \irange{m}$ and $v \in Z_i$ we
let $\alpha_i$ be the unique element of $L(v) - Y$.  We call $\alpha_i$ the
\emph{special} color for $Z_i$.

Now let $i \in \irange{r}$ where $\card{C_i} = \Delta - 1$ and some $x \in G -
C_i$ has $\card{N(x) \cap C_i} \geq 4$. A clump $Z_j \subseteq
N(x) \cap C_i$ is \emph{good} if there is uncolored $z_i \in Z_i$ such that
$\alpha_i$ is not used on any neighbor of $z_i$ and the unique $y$ in $N(z_i) - C_i - \set{x}$ is colored with a color that is
either not in $L(z_i)$ or is used on $C_i$.

Now suppose we have a partial coloring $\pi$ where none of the bad events $S_v$,
$E_i$, $Q_i$, $F_i$ and $P_i$ occur.  We color the $D_i$ corresponding to $P_i$
events first.  Suppose $x \in G - C_i$ has $3$ good clumps $Z_1, Z_2, Z_3$ in $K_i$ with
corresponding vertices $z_1, z_2, z_3$.  Since $\alpha_1 \not \in L(z_2),
L(z_3)$, coloring $z_1$ with $\alpha_1$ leaves a list assignment we can complete
greedily by coloring $z_2$ and $z_3$ last.  However, we need to be careful
to not break the other such $D_i$ in the process.  So, we first color the
respective $z_1$ in each such $D_i$.  After all of those have been colored, we
greedily color the rest of each $D_i$.  
It still needs to be checked that when we color $z_1$ with $\alpha_1$ we don't lose the ability to do the same 
with $\alpha_1$ on some other $D_j$. To see this, note that $x$ has at least $3$ neighbors in $C_i$ and thus is 
contain in no other $C_j$ with $\card{C_j} = \Delta-1$.  Moreover, $z_1$'s only
possible other neighbor $y$ outside $C_i$ is already colored by assumption. 
Now consider the $D_i$ that have two safe uncolored vertices in $C_i$.  If $C_i
\neq K_i$, then since $Q_i$ doesn't happen $x_i$ has two uncolored neighbors,
color it first.  Now color $C_i$ greedily saving the two safe uncolored vertices
in $C_i$ for last. Now we can finish the coloring on the sparse vertices greedily. 
Therefore if we can prevent all the bad events from happening we get our desired contradiction.

It is easy to see that any given event depends on less than $3\Delta^5$ others,
so the result will follow by showing that $\Pr(S_v), \Pr(E_i), \Pr(Q_i),
\Pr(F_i),\Pr(P_i) \leq \Delta^{-6}$.  The following sections prove these bounds.

\section{$\Pr(S_v) \leq \Delta^{-6}$}
We know $\size{G_v} < \frac25 \Delta^2$.  Put $A \DefinedAs
\setb{x}{N(v)}{\card{L(x) \cap L(v)} \geq \frac23 \Delta}$ and $B \DefinedAs
N(v) - A$.  Note that for $x, y \in A$ we have $\card{L(x) \cap L(y)} \geq
\frac13 \Delta$ and for $x \in B$ we have $\card{L(x) - L(v)} \geq \frac13
\Delta$.

Let $A_v$ be the random variable that counts the number of nonadjacent pairs
$x, y \in A$ such that, $\zeta(x) = \zeta(y)$ and $\zeta(z) \neq \zeta(x)$ for
all $z \in N(v) - \set{x, y} \cup N(x) \cup N(y)$.

Let $B_v$ be the random variable that counts the number of $x \in B$ such that
$\zeta(x) \not \in L(v)$ and $\zeta(z) \neq \zeta(x)$ for all $z \in N(v) -
\set{x} \cup N(x)$.

Put $Z_v \DefinedAs A_v + B_v$.  Then $\ex(Z_v) = \ex(A_v) + \ex(B_v)$.  We
prove the bound $\ex(Z_v) \geq \frac{\Delta}{1000}$ and then use Azuma's
inequality to prove that $\Pr(\card{Z_v - \ex(Z_v)} > \frac{\Delta}{1000} - 2)
\leq \Delta^{-6}$. The conclusion $\Pr(S_v) \leq \Delta^{-6}$ is then immediate.

We know that $G_v$ has at least ${\Delta - 1 \choose 2} - \frac25 \Delta^2 \geq
\frac{\Delta^2}{12}$ nonadjacent pairs.  Let $b$ be the number of nonadjacent pairs in $G_v$ that intersect $B$. 
Plainly, $G[A]$ contains at least $\frac{\Delta^2}{12} - b$ nonadjacent pairs
and $b \leq \card{B}\Delta$.

First let's consider $\ex(A_v)$.  Let $x, y \in A$ be nonadjacent.  Since $\card{L(x) \cap L(y)} \geq
\frac13 \Delta$, the probability that $x$ and $y$ get the same color and this
color is not used on any of the rest of $N(v) \cup N(x) \cup N(y)$ is at least
$(3\Delta)^{-1} (1-(\Delta-1)^{-1})^{3\Delta - 3} \geq (3\Delta)^{-1}3^{-3}$. 
Thus $\ex(A_v) \geq (\frac{\Delta^2}{12} - b)\Delta^{-1}3^{-4} \geq
\frac{\Delta}{1000} - \frac{b}{81\Delta}$.

Now consider $\ex(B_v)$.  Let $x \in B$.  Since $\card{L(x) - L(v)} \geq \frac13
\Delta$, the probability that $x$ gets a color not in $L(v)$ and this color is
not used on any of the rest of $N(v) \cup N(x)$ is at least $\frac13
(1-(\Delta-1)^{-1})^{2\Delta - 2} \geq 3^{-4}$.  Hence $\ex(B_v) \geq
\frac{\card{B}}{81} \geq \frac{b}{81\Delta}$.  Therefore $\ex(Z_v) \geq
\frac{\Delta}{1000}$.

Now we need Azuma's inequality.  The concentration analysis is
identical to the coloring case in Reed's proof.  We reproduce it here for
completeness.

\begin{lem}[Azuma]
Let $X$ be a random variable determined by n trials $T_1, \ldots, T_n$ such that
for each $i$ and any two possible sequences of outcomes $t_1, \ldots, t_i$ and
$t_1, \ldots, t_{i-1}, t_i'$:

\[\card{\ex(X \mid T_1 = t_1, \ldots, T_i = t_i) - \ex(X \mid T_1 = t_1,
\ldots, T_i = t_i')} \leq c_i,\]

\noindent then $\Pr(\card{X - \ex(X)} > t) \leq 2e^{\frac{-t^2}{2\sum c_i^2}}$.
\end{lem}

Since we colored the vertices of $G$ independently, we can apply Azuma using any
ordering.  Order $V(G)$ as $w_1, \ldots, w_n$ so that $N(v)$ comes last and let
$w_s$ be the last vertex not in $N(v)$.  Changing $\zeta(w_i)$ from $\beta$ to
$\tau$ only affects the vertices using $\beta$ or $\tau$ and thus changes the
conditional expected value by at most $2$.  For $w_i \not \in N(v)$, the
probability that changing $w_i$'s color will affect $Z_v$ is at most the probability that one of
$w_i$'s two colors is also assigned to one of its neighbors in $N(v)$.  Say
$w_i$ has $d_i$ neighbors in $N(v)$.  Then the most changing $w_i$ can change
$\ex(Z_v)$ is $c_i \DefinedAs 2\frac{2d_i}{\Delta-1} = \frac{4d_i}{\Delta-1}$. 
Now $\sum_{i=1}^s d_i \leq \Delta^2$ and thus $\sum_{i=1}^s c_i \leq 4\Delta + 4
\frac{\Delta}{\Delta-1} \leq 4\Delta + 5$.  As each $c_i \leq 5$, we have
$\sum_{i=1}^s c_i^2 \leq 21\Delta$ and hence $\sum_i c_i^2 \leq 25\Delta$.  Now
using $t \DefinedAs \frac{\Delta}{1000} - 2$ in Azuma gives $\Pr(Z_v < 2) <
2e^{\frac{-(\frac{\Delta}{1000} - 2)^2}{50\Delta}} \leq \Delta^{-6}$ for large
enough $\Delta$.

\section{$\Pr(E_i) \leq \Delta^{-6}$}
The following is an immediate consequence of the $d_1$-choosability lemmas in
\cite{mules}.

\begin{lem}\label{AtMostFourIn}
Let $C$ be a maximal big clique.  Each $v \in C$ has at most one neighbor with
more than $4$ neighbors in $C$ and no such neighbor if $v$ is low.
\end{lem}

\begin{lem}\label{triples}
Let $C$ be a maximal big clique with $\card{C} \leq \Delta - 2$.  There are at
least $\frac{3}{28}\Delta$ disjoint $P_3$'s $xyz$ with $y \in C$ and $x, z \not
\in C$ such that $x$ and $z$ each have at most $4$ neighbors in $C$.
\end{lem}
\begin{proof}
Consider a maximal such set of $P_3$'s. Let $A$ be all the central vertices of
these $P_3$'s and $B$ all the ends.  Then each $v \in B$ has at most $3$
neighbors in $C - A$ and by Lemma \ref{AtMostFourIn} and maximality, each $v \in
C - A$ has at most $2$ neighbors in $G-C-B$ and at most $1$ if $v$ is low.  Thus
$6\card{A} = 3\card{B} \geq \size{C-A, B} \geq (\Delta - \card{C} -
1)\card{C-A} \geq \card{C} - \card{A}$.  Hence $\card{A} \geq
\frac{3}{28}\Delta$.
\end{proof}

We need to force safe uncolored vertices in $C_i$.  If the lists have small
intersections this might not happen with high probability.  We handle this case
using minimality of $\card{G}$ instead.

\begin{lem}\label{LargeIntersections}
There exists $C_i' \subset C_i$ with $\card{C_i'} = \card{C_i} - 1$ such that
for $x, y \in C_i'$ we have $\card{L(x) \cap L(y)} \geq \frac23 \Delta$.
\end{lem}
\begin{proof}
Suppose not and consider an $L$-coloring of $G - C_i$.  Let $L'$ be the
resulting list assignment on $C_i$.  Then $\card{L'(v)} \geq \card{C_i} - 2$ for
all $v \in C_i$.  By assumption, for each $v \in C_i$ we have $x, y \in C_i -
\set{v}$ with $\card{L(x) \cap L(y)} < \frac23 \Delta$.  But then $\card{L'(x) \cup
L'(y)} \geq 2(\Delta - 1 - (\Delta + 1 - \card{C_i})) - \frac23 \Delta \geq
\card{C_i}$.  Hence we can complete the $L$-coloring to $C_i$ by Hall's theorem,
a contradiction.
\end{proof}

We will find the desired uncolored safe vertices in $C_i'$.  By Lemma
\ref{triples}, there are at least $\frac{\Delta}{10}$ paths $acb$ where $c \in
C_i'$ and $a,b \not \in C_i'$ such that $a$ and $b$ each have at most $4$
neighbors in $C$.  Let $T_i$ be the union of all the vertices in these paths. 
For some such fixed path we want to bound the probability that $c$ is uncolored and safe and the colors used on $a$ and $b$ are used on
none of the rest of $T_i$.  To do so, we distinguish three cases.

{\bf Case~1.} {\it $\card{L(a) \cap L(c)} < \frac23 \Delta$ and $\card{L(b) \cap
L(c)} < \frac23 \Delta$}

For $\alpha \in L(a) - L(c)$, $\beta \in L(b) - L(c)$, $z \in C_i' - T_i$ and
$\gamma \in L(c) \cap L(z)$ where $\alpha, \beta, \gamma$ are all different, let
$A_{\alpha, \beta, \gamma, z}$ be the event that all of the following hold:

\begin{enumerate}
  \item $\alpha$ is assigned to $a$ and none of the rest of $T_i \cup N(a)$,
  \item $\beta$ is assigned to $b$ and none of the rest of $T_i \cup N(b)$,
  \item $\gamma$ is assigned to $c$ and $z$ and none of the rest of $T_i$.
\end{enumerate}

Then $\Pr(A_{\alpha, \beta, \gamma, z}) \geq (\Delta-1)^{-1}(1-
(\Delta-1)^{-1})^{\card{T_i \cup N(a)}}(\Delta-1)^{-1}(1-
(\Delta-1)^{-1})^{\card{T_i \cup N(b)}}(\Delta-1)^{-2}(1-
(\Delta-1)^{-1})^{\card{T_i}} \geq (\Delta-1)^{-4}3^{-5}$.

Plainly, the $A_{\alpha, \beta, \gamma, z}$ are disjoint for different sets of
indices.  Since $\card{L(a) - L(c)} \geq \frac{\Delta}{3}$, we have
$\frac{\Delta}{3}$ choices for $\alpha$.  Similarly we then have
$\frac{\Delta}{3} - 1$ choices for $\beta$.  For $z$ we have at least $\frac34
\Delta - \frac{1}{10}\Delta \geq \frac{\Delta}{3}$ choices.  Since $\card{L(z)
\cap L(c)} \geq \frac23 \Delta$, we then have at least $\frac23\Delta - 2$
choices for $\gamma$ for each $z$.  In total we have at least $\Delta^4 3^{-4}$
choices and thus the probability that $A_{\alpha, \beta, \gamma, z}$ holds for
some choice of indices is at least $3^{-9}$.

{\bf Case~2.} {\it $\card{L(a) \cap L(c)} < \frac23 \Delta$ and $\card{L(b) \cap
L(c)} \geq \frac23 \Delta$}

For $y \in C_i' - T_i - N(b)$, $z \in C_i' - T_i$, $\alpha \in L(a) - L(c)$,
$\beta \in L(b) \cap L(y)$ and $\gamma \in L(c) \cap L(z)$ where $\alpha, \beta,
\gamma$ are all different, let $A_{\alpha, \beta, \gamma, y, z}$ be the event
that all of the following hold:

\begin{enumerate}
  \item $\alpha$ is assigned to $a$ and none of the rest of $T_i \cup N(a)$,
  \item $\beta$ is assigned to $b$ and $y$ and none of the rest of $T_i \cup
  N(b) \cup N(y)$,
  \item $\gamma$ is assigned to $c$ and $z$ and none of the rest of $T_i$.
\end{enumerate}

Then $\Pr(A_{\alpha, \beta, \gamma, y, z}) \geq (\Delta-1)^{-1}(1-
(\Delta-1)^{-1})^{\card{T_i \cup N(a)}}(\Delta-1)^{-2}(1-
(\Delta-1)^{-1})^{\card{T_i \cup N(b) \cup N(y)}}(\Delta-1)^{-2}(1-
(\Delta-1)^{-1})^{\card{T_i}} \geq (\Delta-1)^{-5}3^{-6}$.

Again the $A_{\alpha, \beta, \gamma, y, z}$ are disjoint for different sets of
indices.  For $y$ we have at least $\card{C_i'} - \card{T_i \cap C_i'} -
\card{N(b) \cap C_i} \geq \frac34 \Delta - 1 - \frac{\Delta}{10} - 4 \geq
\frac{\Delta}{9}$ choices.  For each $y$ we have at least $\frac23 \Delta$ choices for $\beta$. 
The rest are similar to above and in total we have at least $\Delta^53^{-6}$
choices and thus the probability that $A_{\alpha, \beta, \gamma, y, z}$ holds
for some choice of indices is at least $3^{-12}$.

{\bf Case~3.} {\it $\card{L(a) \cap L(c)} \geq \frac23 \Delta$ and $\card{L(b)
\cap L(c)} \geq \frac23 \Delta$}

For $x \in C_i' - T_i - N(a)$, $y \in C_i' - T_i - N(b)$, $z \in C_i' - T_i$,
$\alpha \in L(a) \cap L(c)$, $\beta \in L(b) \cap L(y)$ and $\gamma \in L(c)
\cap L(z)$ where $\alpha, \beta, \gamma$ are all different, let $A_{\alpha,
\beta, \gamma, x, y, z}$ be the event that all of the following hold:

\begin{enumerate}
  \item $\alpha$ is assigned to $a$ and $y$ and none of the rest of $T_i \cup
  N(a) \cup N(x)$,
  \item $\beta$ is assigned to $b$ and $y$ and none of the rest of $T_i \cup
  N(b) \cup N(y)$,
  \item $\gamma$ is assigned to $c$ and $z$ and none of the rest of $T_i$.
\end{enumerate}

Then $\Pr(A_{\alpha, \beta, \gamma, x, y, z}) \geq (\Delta-1)^{-2}(1-
(\Delta-1)^{-1})^{\card{T_i \cup N(a) \cup N(x)}}(\Delta-1)^{-2}(1-
(\Delta-1)^{-1})^{\card{T_i \cup N(b) \cup N(y)}}(\Delta-1)^{-2}(1-
(\Delta-1)^{-1})^{\card{T_i}} \geq (\Delta-1)^{-6}3^{-7}$.

Again the $A_{\alpha, \beta, \gamma, x, y, z}$ are disjoint for different sets
of indices. In total we get at least $\Delta^6 3^{-8}$
choices and thus the probability that $A_{\alpha, \beta, \gamma, x, y, z}$ holds
for some choice of indices is at least $3^{-15}$.

\bigskip

Now we have at least $\frac{\Delta}{10}$ such triples.  So if $M_i$ counts the
number of uncolored safe vertices in $C_i$ we have $\ex(M_i) \geq
10^{-9}\Delta$.  The concentration details are identical to Reed's proof and we
conclude $Pr(M_i < 2) < \Delta^{-6}$.

\section{$\Pr(Q_i) \leq \Delta^{-6}$}
If $\zeta(x) = \zeta(y)$ for different $x,y \in K_i$, then $x$ and $y$ will be
uncolored and $Q_i$ cannot hold.  Thus it is enough to show that all vertices of
$K_i$ getting different colors is unlikely.  Just like Lemma
\ref{LargeIntersections}, we can find $K_i' \subset K_i$ with $\card{K_i'} =
\card{K_i} - 1$ such that for $x,y \in K_i'$ we have $\card{L(x) \cap L(y)}
\geq \frac23 \Delta$.  

Let $x, y \in K_i'$.  The probability that $x$ and $y$ get the same color and
this color is used on none of the rest of $N(x) \cup N(y)$ is at least
$\frac{2}{3\Delta}(1 - (\Delta-1)^{-1})^{2\Delta - 2} \geq \frac{2}{3^3\Delta}$.
Since there are at least $\frac12 (\frac23 \Delta)^2$ such pairs, the expected
number of pairs getting the same color is at least $3^{-4}\Delta$.  An
application of Azuma's inequality very similar to the sparse case now proves
$\Pr(Q_i) \leq \Delta^{-6}$.

\section{$\Pr(F_i) \leq \Delta^{-6}$}
In this case we must have $C_i = K_i$ since no vertex outside $C_i$ has
$\frac34 \Delta$ neighbors in $C_i$.  Since low vertices don't make things
harder, we will assume there are no low vertices in $C_i$.  In particular, for a
low vertex, we don't need a triple as in the follow lemma, but just one good neighbor outside because we only need to save one color on a low
vertex's neighborhood to make it safe.

\begin{lem}\label{triplesAgain}
There are at least $\frac14 \sqrt{\Delta}\log{\Delta}$ disjoint $P_3$'s $xyz$
with $y \in C_i$ and $x, z \not \in C_i$ such that $x$ and $z$ each have at most
$\frac{\sqrt{\Delta}}{\log(\Delta)}$ neighbors in $C_i$.
\end{lem}
\begin{proof}
Since there are at most $\log^2(\Delta)$ vertices outside $C_i$ which have more
than $\frac{\sqrt{\Delta}}{\log(\Delta)}$ neighbors in $C_i$ and all of these
vertices have at most $\sqrt{\Delta}\log(\Delta)$ neighbors in $C_i$, removing
all their neighbors from $C_i$ we are left with a set $A$ of of vertices all of
whose neighbors outside $C_i$ have at most $\frac{\sqrt{\Delta}}{\log(\Delta)}$
neighbors in $C_i$.  Now $\card{A} \geq \Delta - 1 -
\log^2(\Delta)\sqrt{\Delta}\log(\Delta) \geq \frac{\Delta}{2}$.  Now pick
$P_3$'s $xyz$ with $y \in A$ in turn removing the neighbors of $x$ and $z$ each
time.  We get at least $\frac{\card{A}}{2\frac{\sqrt{\Delta}}{\log(\Delta)}}
\geq \frac14 \sqrt{\Delta}\log{\Delta}$ disjoint $P_3$'s.

Now the proof of the expected value is the same as the proof of $\Pr(E_i) \leq
\Delta^{-6}$, except that we have fewer $P_3$'s to multiply by at the end.  So,
if $M_i$ counts the number of uncolored safe vertices in $C_i$, we have
$\ex(M_i) \geq 3^{-15} (\frac14) \sqrt{\Delta}\log{\Delta} \geq 10^{-9}
\sqrt{\Delta}\log{\Delta}$.

Now, the application of Azuma is the same as in the $\Pr(E_i) \leq
\Delta^{-6}$ case, except we use $t \DefinedAs 10^{-9}
\sqrt{\Delta}\log{\Delta} - 2$, which gives gives $\Pr(M_i < 2) <
2e^{\frac{-(10^{-9} \sqrt{\Delta}\log{\Delta} - 2)^2}{\Delta}} \leq
\Delta^{-6}$ for large enough $\Delta$.
\end{proof}

\section{$\Pr(P_i) \leq \Delta^{-6}$}
{\bf Case~1.} {\it Some $x \in G  - C_i$ has $\card{N(x) \cap C_i} >
\sqrt{\Delta}\log(\Delta)$.} 

If $C_i \neq K_i$, then take $x$ to be $x_i$. Let $Z_1, \ldots, Z_m$ be the
clumps of $x$ in $K_i$ and for $j \in \irange{m}$, let $\alpha_j$ be the color
the $Z_j$ clump has that the others do not.  By Lemma \ref{LargeIntersections}, we may as well assume that $\card{L(x) \cap
L(y)} \geq \frac23 \Delta$ for all $x,y \in K_i$ (the cost is one vertex which
changes nothing).

Pick $z_j \in Z_j$ arbitrarily.  By Lemma \ref{AtMostFourIn}, any neighbor of
$z_j$ in $G - C_i - x$ (of which there is at most one) has at most $4$ neighbors
in $C_i$.  Thus, by symmetry, for each $j \in \irange{\frac{m}{4}}$ we can pick
$y_j \in G - C_i - x$ such that $y_jz_j \in E(G)$ and the $y_j$ are all
different.  Put $A \DefinedAs N(x) \cap K_i$.  Then $\frac{m}{4} \geq
(\frac14)(\frac15)\card{A} \geq \frac{1}{20}\sqrt{\Delta}\log(\Delta)$.

Now, for fixed $j \in \irange{\frac{m}{4}}$, we bound the probability that $z_j$
is uncolored, $\alpha_j$ is not used on any neighbor of $z_j$ and $y_j$ is colored
with a color that is either not in $L(z_j)$ or is used on $C_i$.  Let $T_i$ be
the union of all the $y_j$'s and $z_j$'s. We distinguish two cases.

{\bf Subcase~1a.} {\it $\card{L(y_j) \cap L(z_j)} < \frac23 \Delta$}

For $\beta \in L(y_j) - L(z_j)$, $w \in C_i - T_i$ and $\gamma \in L(z_j) \cap
L(w)$ where $\beta, \gamma \neq \alpha_j$ and $\beta \neq \gamma$, let
$F_{\beta, \gamma, w}$ be the event that all of the following hold:

\begin{enumerate}
  \item $\beta$ is assigned to $y_j$ and none of the rest of $T_i \cup N(y_j)$,
  \item $\gamma$ is assigned to $z_j$ and $w$ and none of the rest of $T_i$,
  \item $\alpha_j$ is assigned to no neighbor of $z_j$.
\end{enumerate}

The probability of (3) is
at least $\parens{\frac{\Delta - 2}{\Delta - 1}}^{\Delta} = (1 -
(1-\Delta)^{-1})^\Delta \geq \frac13$. Hence $\Pr(F_{\beta, \gamma, w}) \geq
\frac13 (\Delta-1)^{-1}(1-(\Delta-1)^{-1})^{\card{T_i \cup
N(y_j)}}(\Delta-1)^{-2}(1 - (\Delta-1)^{-1})^{\card{T_i}} \geq
(\Delta-1)^{-3}3^{-4}.$

Now we have at least $\frac{\Delta}{3}$ choices for $\beta$, $\frac{\Delta}{2}$
choices for $w$ and $\frac 23 \Delta$ choices for $\gamma$ for each $w$.  Thus
the probability that $F_{\beta, \gamma, w}$ holds for some choice of indices is
at least $3^{-6}$.

{\bf Subcase~1b.} {\it $\card{L(y_j) \cap L(z_j)} \geq \frac23 \Delta$}
For $y \in C_i - T_i - N(y_j)$, $\beta \in L(y_j) \cap L(y)$, $w \in C_i - T_i$
and $\gamma \in L(z_j) \cap L(w)$ where $\beta, \gamma \neq \alpha_j$ and $\beta \neq \gamma$, let
$F_{\beta, \gamma, y, w}$ be the event that all of the following hold:

\begin{enumerate}
  \item $\beta$ is assigned to $y_j$ and $y$ and none of the rest of $T_i \cup
  N(y_j) \cup N(y)$,
  \item $\gamma$ is assigned to $z_j$ and $w$ and none of the rest of $T_i$,
  \item $\alpha_j$ is assigned to no neighbor of $z_j$.
\end{enumerate}

We have $\Pr(F_{\beta, \gamma, y, w}) \geq
\frac13 (\Delta-1)^{-2}(1-(\Delta-1)^{-1})^{\card{T_i \cup
N(y_j) \cup N(y)}}(\Delta-1)^{-2}(1 - (\Delta-1)^{-1})^{\card{T_i}} \geq
(\Delta-1)^{-4}3^{-5}.$

Now for $y$ we have at least $\card{K_i} - \card{C_i \cap T_i} - \card{N(y_j)
\cap C_i} \geq \frac{\Delta}{2}$ choices and for each $y$ we have at least
$\frac23 \Delta$ choices for $\beta$. Thus the probability that $F_{\beta,
\gamma, y, w}$ holds for some choice of indices is at least $3^{-7}$.

\bigskip

Therefore the expected number of good clumps is at least $3^{-7}
(\frac{1}{20})\sqrt{\Delta}\log(\Delta) \geq 10^{-5} \sqrt{\Delta}\log(\Delta)$.   
Changing any color will affect the conditional expectations by at most $2$ and a
similar computation for Azuma shows that $\Pr(F_i) \leq \Delta^{-6}$.  The key
here is that $(\sqrt{\Delta}\log(\Delta))^2$ grows faster that $\Delta$.

{\bf Case~2.} {\it More than $\log^2(\Delta)$ vertices $x \in G  - C_i$ have
$\card{N(x) \cap C_i} > \frac{\sqrt{\Delta}}{\log(\Delta)}$.}

We must have $C_i = K_i$.  Let $x_1, \ldots, x_k$ be $k \DefinedAs
\ceil{\log^2(\Delta)}$ different vertices in $G - C_i$ which have $\card{N(x_j)
\cap C_i} > \frac{\sqrt{\Delta}}{\log(\Delta)}$ for each $j \in \irange{k}$.  

The computation for the expected number of good clumps for each $x_j$ is the
same as Case 1 and so we expect at least
$10^{-5}\frac{\sqrt{\Delta}}{\log(\Delta)}$ good clumps for each $x_j$.  Thus in
total we expect $10^{-5}\sqrt{\Delta}\log(\Delta)$ good clumps over the
$\log^2(\Delta)$ sets.  Let $X$ count this total number of good clumps.  We show
that $\Pr(X < 3\log^2(\Delta)) \leq \Delta^{-6}$ and hence at least one $x_j$
has at least $3$ good clumps with high enough probability.

If we applied Azuma with the information we have now we'd be in trouble because
many of the $x_j$'s could use the same special color and hence changing a vertex
to that color would change the conditional expectation by a lot.  We need one
further structural lemma that guarantees at most $4$ of the $x_j$'s use any
given special color.

\begin{lem}\label{LastStructure}
Let $K$ be a $\Delta-1$ clique in $G$ and $x_1, x_2, x_3, x_4, x_5 \in G - K$
with $\card{N(x_j) \cap K} \geq 5$ such that the $N(x_j) \cap K$ are pairwise
disjoint. Then no color is special for all the $x_j$.
\end{lem}
\begin{proof}
Suppose otherwise that some color $\alpha$ is special for all the $x_j$. Put
$A_j \DefinedAs N(x_j) \cap K$. Just like in the proof of Lemma \ref{clumping},
any $L$-coloring of $G - (K \cup \set{x_1, \ldots, x_5})$ must not leave
$\alpha$ available on any of the vertices in $A_j$ for any $j \in \irange{5}$.
Pick $z_j \in A_j$ for each $j$ and let $y_j$ be the neighbor of $z_j$ in $G -
(K \cup \set{x_1, \ldots, x_5})$. Put $N \DefinedAs \set{y_1, \ldots, y_5}$. By
Lemma \ref{AtMostFourIn}, $\card{N} \geq 2$.  Now just like in Lemma
\ref{clumping}, by using minimality of $\card{G}$ we see that adding any edge
between vertices in $N$ must create a $K_\Delta$ and then counting degrees gives
a contradiction.
\end{proof}

Now when we change a color we change the conditional expectation by at most $8$.
A similar computation to before bounds $\sum_j c_j^2 \leq 500\Delta$.  Applying
Azuma with $t = 10^{-5}\sqrt{\Delta}\log(\Delta) - 3\log^2(\Delta)$ gives $\Pr(X
< 3\log^2(\Delta)) < 2e^{\frac{-(10^{-5}\sqrt{\Delta}\log(\Delta) -
3\log^2(\Delta))^2}{500\Delta}} \leq \Delta^{-6}$ for large $\Delta$.


\bibliographystyle{amsplain}
\bibliography{GraphColoring}
\end{document}

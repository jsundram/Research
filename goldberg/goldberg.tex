\documentclass[12pt]{amsart}
\usepackage{amsmath, amsthm, amssymb}
\usepackage[top=1.25in, bottom=1.25in, left=1.0in, right=1.0in]{geometry}
\usepackage{hyperref}
\usepackage{color}
\usepackage{verbatim}

\makeatletter
\newtheorem*{rep@theorem}{\rep@title}
\newcommand{\newreptheorem}[2]{
\newenvironment{rep#1}[1]{
 \def\rep@title{#2 \ref{##1}}
 \begin{rep@theorem}}
 {\end{rep@theorem}}}
\makeatother

\theoremstyle{plain}
\newtheorem{thm}{Theorem}
\newreptheorem{thm}{Theorem}
\newtheorem*{Brooks}{Brooks' Theorem}
\newtheorem*{KernelLemma}{Kernel Lemma}
\newtheorem{prop}[thm]{Proposition}
\newreptheorem{prop}{Proposition}
\newtheorem{lem}[thm]{Lemma}
\newreptheorem{lem}{Lemma}
\newtheorem{conj}[thm]{Conjecture}
\newreptheorem{conj}{Conjecture}
\newtheorem{cor}[thm]{Corollary}
\newreptheorem{cor}{Corollary}
\newtheorem{prob}[thm]{Problem}
\theoremstyle{definition}
\newtheorem{defn}{Definition}
\theoremstyle{remark}
\newtheorem*{remark}{Remark}
\newtheorem{example}{Example}
\newtheorem*{question}{Question}
\newtheorem*{observation}{Observation}
\newtheorem*{FixerMove}{\bf {Fixer's turn}}
\newtheorem*{BreakerMove}{\bf {Breaker's turn}}

\title{Goldberg's Conjecture from a game}
\author{Landon Rabern}

\newcommand{\fancy}[1]{\mathcal{#1}}
\newcommand{\C}[1]{\fancy{C}_{#1}}
\newcommand{\IN}{\mathbb{N}}
\newcommand{\IR}{\mathbb{R}}
\newcommand{\G}{\fancy{G}}
\newcommand{\LB}{\mathcal{L}_B}
\newcommand{\col}{{\textrm{col}}}
\newcommand{\chil}{{\chi_{\ell}}}
\newcommand{\chiol}{{\chi_{OL}}}

\newcommand{\inj}{\hookrightarrow}
\newcommand{\surj}{\twoheadrightarrow}

\newcommand{\set}[1]{\left\{ #1 \right\}}
\newcommand{\setb}[3]{\left\{ #1 \in #2 \mid #3 \right\}}
\newcommand{\setbs}[2]{\left\{ #1 \mid #2 \right\}}
\newcommand{\card}[1]{\left|#1\right|}
\newcommand{\size}[1]{\left\Vert#1\right\Vert}
\newcommand{\ceil}[1]{\left\lceil#1\right\rceil}
\newcommand{\floor}[1]{\left\lfloor#1\right\rfloor}
\newcommand{\func}[3]{#1\colon #2 \rightarrow #3}
\newcommand{\funcinj}[3]{#1\colon #2 \inj #3}
\newcommand{\funcsurj}[3]{#1\colon #2 \surj #3}
\newcommand{\irange}[1]{\left[#1\right]}
\newcommand{\join}[2]{#1 \mbox{\hspace{2 pt}$\ast$\hspace{2 pt}} #2}
\newcommand{\djunion}[2]{#1 \mbox{\hspace{2 pt}$+$\hspace{2 pt}} #2}
\newcommand{\parens}[1]{\left( #1 \right)}
\newcommand{\brackets}[1]{\left[ #1 \right]}
\newcommand{\DefinedAs}{\mathrel{\mathop:}=}
\newcommand{\im}{\operatorname{im}}
\newcommand{\mic}{\operatorname{mic}}
\newcommand{\pot}{\operatorname{Pot}}

\begin{document}
\maketitle
\begin{abstract}
We characterize the initial positions from which the first player has a winning
strategy in a certain two-player game.  Goldberg's Conjecture on edge-coloring follows from a special case.
\end{abstract}

\section{Introduction}
The game is played on a multigraph $G$ by \emph{Fixer} and
\emph{Breaker}.  To setup, we assign a list of colors $L(v)$ to each
$v \in V(G)$. Put  $\pot(L) \DefinedAs \bigcup_{v \in V(G)} L(v)$.
Fixer always gets the first move and he wins if and only if on some turn, before moving, he can construct a proper edge coloring of $G$, say $\func{\pi}{E(G)}{\pot(L)}$ such that
$\pi(xy) \in L(x) \cap L(y)$ for each $xy \in E(G)$.

\begin{FixerMove}
Pick $\alpha \in \pot(L)$ and $v \in V(G)$ with $\alpha \not \in L(v)$ and set $L(v)
\DefinedAs L(v) \cup \set{\alpha} \smallsetminus \set{\beta}$ for some $\beta
\in L(v)$.
\end{FixerMove}

\begin{BreakerMove}
If Fixer modified $L(v)$ by inserting $\alpha$ and removing $\beta$, Breaker can
pick at most one vertex in $V(G) \smallsetminus \set{v}$ and modify its list by
swapping $\alpha$ for $\beta$ or $\beta$ for $\alpha$.
\end{BreakerMove}

For $\alpha \in \pot(L)$ and $H \subseteq G$, put $d_H(\alpha) \DefinedAs \card{\setb{v}{V(H)}{\alpha \in L(v)}}$.

\begin{conj}\label{MainTheorem}
If $|L(v)| \ge d(v) + 1$ for all $v \in V(G)$, then Fixer has a winning strategy against Breaker if and only if for each $H \subseteq G$,
\[\sum_{\alpha \in \pot(L)} \floor{\frac{d_H(\alpha)}{2}} \ge \size{H}.\]
\end{conj}

Suppose $G$ is not $k$-edge-colorable for some $k \ge \Delta(G) + 1$. Let $L(v) = \irange{k}$ for all $v \in V(G)$.  Then Fixer has no moves, so Fixer has a winning strategy if and only if $G$ is $k$-edge-colorable.  By Conjecture \ref{MainTheorem}, there must be $H \subseteq G$ with 
\[k\floor{\frac{|H|}{2}} = \sum_{\alpha \in \pot(L)} \floor{\frac{|H|}{2}} = \sum_{\alpha \in \pot(L)} \floor{\frac{d_H(\alpha)}{2}} < \size{H}.\] 

\noindent That is, $k \le \floor{\frac{\size{H} - 1}{\floor{\frac{|H|}{2}}}}$.  Hence $G$ is $w(G)$-edge-colorable where $w(G) \DefinedAs \max_{\substack{H \subseteq G\\|H| \ge 2}}\ceil{\frac{\size{H}}{\floor{\frac{|H|}{2}}}}$ if it is not $(\Delta(G) + 1)$-edge-colorable.  This is Goldberg's Conjecture.

\bibliographystyle{plain}
\bibliography{GraphColoring}
\end{document}



\documentclass[12pt]{amsart}
\usepackage{amsmath, amsthm, amssymb}
\usepackage[top=1.25in, bottom=1.25in, left=1.0in, right=1.0in]{geometry}
\usepackage{hyperref}
\usepackage{color}
\usepackage{verbatim}
\usepackage{hyperref}

\makeatletter
\newtheorem*{rep@theorem}{\rep@title}
\newcommand{\newreptheorem}[2]{
\newenvironment{rep#1}[1]{
 \def\rep@title{#2 \ref{##1}}
 \begin{rep@theorem}}
 {\end{rep@theorem}}}
\makeatother

\theoremstyle{plain}
\newtheorem{thm}{Theorem}
\newreptheorem{thm}{Theorem}
\newtheorem*{Brooks}{Brooks' Theorem}
\newtheorem*{KernelLemma}{Kernel Lemma}
\newtheorem{prop}[thm]{Proposition}
\newreptheorem{prop}{Proposition}
\newtheorem{lem}[thm]{Lemma}
\newreptheorem{lem}{Lemma}
\newtheorem{conj}[thm]{Conjecture}
\newreptheorem{conj}{Conjecture}
\newtheorem{cor}[thm]{Corollary}
\newreptheorem{cor}{Corollary}
\newtheorem{prob}[thm]{Problem}
\theoremstyle{definition}
\newtheorem{defn}{Definition}
\theoremstyle{remark}
\newtheorem*{remark}{Remark}
\newtheorem{example}{Example}
\newtheorem*{question}{Question}
\newtheorem*{observation}{Observation}
\newtheorem*{FixerMove}{\bf {Fixer's turn}}
\newtheorem*{BreakerMove}{\bf {Breaker's turn}}
\newtheorem*{ChronicleUpdate}{\bf {Chronicle update}}

%\title{Fixer-Breaker and Short Tashkinov Trees }
%\author{Daniel W. Cranston \and Landon Rabern}

\newcommand{\fancy}[1]{\mathcal{#1}}
\newcommand{\C}[1]{\fancy{C}_{#1}}
\newcommand{\IN}{\mathbb{N}}
\newcommand{\IR}{\mathbb{R}}
\newcommand{\G}{\fancy{G}}
\newcommand{\LB}{\mathcal{L}_B}
\newcommand{\col}{{\textrm{col}}}
\newcommand{\chil}{{\chi_{\ell}}}
\newcommand{\chiol}{{\chi_{OL}}}

\newcommand{\inj}{\hookrightarrow}
\newcommand{\surj}{\twoheadrightarrow}

\newcommand{\set}[1]{\left\{ #1 \right\}}
\newcommand{\setb}[3]{\left\{ #1 \in #2 \mid #3 \right\}}
\newcommand{\setbs}[2]{\left\{ #1 \mid #2 \right\}}
\newcommand{\card}[1]{\left|#1\right|}
\newcommand{\size}[1]{\left\Vert#1\right\Vert}
\newcommand{\ceil}[1]{\left\lceil#1\right\rceil}
\newcommand{\floor}[1]{\left\lfloor#1\right\rfloor}
\newcommand{\func}[3]{#1\colon #2 \rightarrow #3}
\newcommand{\funcinj}[3]{#1\colon #2 \inj #3}
\newcommand{\funcsurj}[3]{#1\colon #2 \surj #3}
\newcommand{\irange}[1]{\left[#1\right]}
\newcommand{\join}[2]{#1 \mbox{\hspace{2 pt}$\ast$\hspace{2 pt}} #2}
\newcommand{\djunion}[2]{#1 \mbox{\hspace{2 pt}$+$\hspace{2 pt}} #2}
\newcommand{\parens}[1]{\left( #1 \right)}
\newcommand{\brackets}[1]{\left[ #1 \right]}
\newcommand{\DefinedAs}{\mathrel{\mathop:}=}
\newcommand{\im}{\operatorname{im}}
\newcommand{\mic}{\operatorname{mic}}
\newcommand{\pot}{\operatorname{Pot}}

\renewcommand{\S}{\fancy{S}}
\newcommand{\W}{\fancy{W}}
\newcommand{\T}{\fancy{T}}
\renewcommand{\P}{\fancy{P}}
\renewcommand{\C}{\fancy{C}}

\begin{document}

\begin{lem}\label{chair}
Fixer has a winning strategy against Breaker in the chronicled game on the chair with superabundant list assignment $L$ where $|L(v_1)| = 2$, $|L(v_2)| = 3$, $|L(v_3)| = 3$, $|L(v_4)| = 2$ and $|L(v_5)| = 2$.
\end{lem}
\begin{proof}
We show that for each possible such list assignment $L$ on $G$, Fixer has a winning strategy.
Up to symmetry, the following cases cover all the possible list assignments that are not an immediate win for Fixer.

\noindent\textbf{Case 1.  }\textit{$L(v_1) = \{0, 1\}$, $L(v_2) = \{0, 1, 2\}$, $L(v_3) = \{0, 1, 2\}$, $L(v_4) = \{0, 3\}$ and $L(v_5) = \{0, 3\}$.}

Fixer gets a winning strategy by coloring $v_1v_2$ with $0$ and applying Lemma \ref{CanColorAndPlayOnRest}.

\noindent\textbf{Case 2.  }\textit{$L(v_1) = \{0, 1\}$, $L(v_2) = \{0, 1, 2\}$, $L(v_3) = \{0, 1, 2\}$, $L(v_4) = \{2, 3\}$ and $L(v_5) = \{2, 3\}$.}

Fixer gets a winning strategy by coloring $v_1v_2$ with $0$ and applying Lemma \ref{CanColorAndPlayOnRest}.

\noindent\textbf{Case 3.  }\textit{$L(v_1) = \{0, 1\}$, $L(v_2) = \{0, 1, 2\}$, $L(v_3) = \{0, 1, 3\}$, $L(v_4) = \{0, 1\}$ and $L(v_5) = \{0, 1\}$.}

Fixer gets a winning strategy by coloring $v_1v_2$ with $0$ and applying Lemma \ref{CanColorAndPlayOnRest}.

\noindent\textbf{Case 4.  }\textit{$L(v_1) = \{0, 1\}$, $L(v_2) = \{0, 1, 2\}$, $L(v_3) = \{0, 1, 3\}$, $L(v_4) = \{0, 1\}$ and $L(v_5) = \{0, 2\}$.}

Fixer gets a winning strategy by coloring $v_1v_2$ with $0$ and applying Lemma \ref{CanColorAndPlayOnRest}.

\noindent\textbf{Case 5.  }\textit{$L(v_1) = \{0, 1\}$, $L(v_2) = \{0, 1, 2\}$, $L(v_3) = \{0, 1, 3\}$, $L(v_4) = \{0, 2\}$ and $L(v_5) = \{0, 1\}$.}

Fixer gets a winning strategy by coloring $v_1v_2$ with $0$ and applying Lemma \ref{CanColorAndPlayOnRest}.

\noindent\textbf{Case 6.  }\textit{$L(v_1) = \{0, 1\}$, $L(v_2) = \{0, 1, 2\}$, $L(v_3) = \{0, 1, 3\}$, $L(v_4) = \{0, 2\}$ and $L(v_5) = \{0, 2\}$.}

Fixer gets a winning strategy by coloring $v_1v_2$ with $0$ and applying Lemma \ref{CanColorAndPlayOnRest}.

\noindent\textbf{Case 7.  }\textit{$L(v_1) = \{0, 1\}$, $L(v_2) = \{0, 1, 2\}$, $L(v_3) = \{0, 1, 3\}$, $L(v_4) = \{0, 2\}$ and $L(v_5) = \{1, 2\}$.}

Fixer gets a winning strategy by coloring $v_1v_2$ with $0$ and applying Lemma \ref{CanColorAndPlayOnRest}.

\noindent\textbf{Case 8.  }\textit{$L(v_1) = \{0, 1\}$, $L(v_2) = \{0, 1, 2\}$, $L(v_3) = \{0, 1, 3\}$, $L(v_4) = \{2, 3\}$ and $L(v_5) = \{2, 3\}$.}

Fixer gets a winning strategy by coloring $v_1v_2$ with $0$ and applying Lemma \ref{CanColorAndPlayOnRest}.

\noindent\textbf{Case 9.  }\textit{$L(v_1) = \{0, 1\}$, $L(v_2) = \{0, 1, 2\}$, $L(v_3) = \{0, 2, 3\}$, $L(v_4) = \{0, 1\}$ and $L(v_5) = \{0, 1\}$.}

Fixer gets a winning strategy by coloring $v_1v_2$ with $0$ and applying Lemma \ref{CanColorAndPlayOnRest}.

\noindent\textbf{Case 10.  }\textit{$L(v_1) = \{0, 1\}$, $L(v_2) = \{0, 1, 2\}$, $L(v_3) = \{0, 2, 3\}$, $L(v_4) = \{0, 1\}$ and $L(v_5) = \{0, 2\}$.}

Fixer gets a winning strategy by coloring $v_1v_2$ with $0$ and applying Lemma \ref{CanColorAndPlayOnRest}.

\noindent\textbf{Case 11.  }\textit{$L(v_1) = \{0, 1\}$, $L(v_2) = \{0, 1, 2\}$, $L(v_3) = \{0, 2, 3\}$, $L(v_4) = \{0, 1\}$ and $L(v_5) = \{1, 2\}$.}

Fixer gets a winning strategy by coloring $v_1v_2$ with $0$ and applying Lemma \ref{CanColorAndPlayOnRest}.

\noindent\textbf{Case 12.  }\textit{$L(v_1) = \{0, 1\}$, $L(v_2) = \{0, 1, 2\}$, $L(v_3) = \{0, 2, 3\}$, $L(v_4) = \{0, 2\}$ and $L(v_5) = \{0, 1\}$.}

Fixer gets a winning strategy by coloring $v_1v_2$ with $0$ and applying Lemma \ref{CanColorAndPlayOnRest}.

\noindent\textbf{Case 13.  }\textit{$L(v_1) = \{0, 1\}$, $L(v_2) = \{0, 1, 2\}$, $L(v_3) = \{0, 2, 3\}$, $L(v_4) = \{0, 2\}$ and $L(v_5) = \{0, 2\}$.}

Fixer gets a winning strategy by coloring $v_1v_2$ with $0$ and applying Lemma \ref{CanColorAndPlayOnRest}.

\noindent\textbf{Case 14.  }\textit{$L(v_1) = \{0, 1\}$, $L(v_2) = \{0, 1, 2\}$, $L(v_3) = \{0, 2, 3\}$, $L(v_4) = \{0, 2\}$ and $L(v_5) = \{1, 2\}$.}

Fixer gets a winning strategy by coloring $v_1v_2$ with $0$ and applying Lemma \ref{CanColorAndPlayOnRest}.

\noindent\textbf{Case 15.  }\textit{$L(v_1) = \{0, 1\}$, $L(v_2) = \{0, 1, 2\}$, $L(v_3) = \{0, 2, 3\}$, $L(v_4) = \{1, 2\}$ and $L(v_5) = \{0, 1\}$.}

Fixer gets a winning strategy by coloring $v_1v_2$ with $0$ and applying Lemma \ref{CanColorAndPlayOnRest}.

\noindent\textbf{Case 16.  }\textit{$L(v_1) = \{0, 1\}$, $L(v_2) = \{0, 1, 2\}$, $L(v_3) = \{0, 2, 3\}$, $L(v_4) = \{1, 2\}$ and $L(v_5) = \{0, 2\}$.}

Fixer gets a winning strategy by coloring $v_1v_2$ with $0$ and applying Lemma \ref{CanColorAndPlayOnRest}.

\noindent\textbf{Case 17.  }\textit{$L(v_1) = \{0, 1\}$, $L(v_2) = \{0, 1, 2\}$, $L(v_3) = \{0, 2, 3\}$, $L(v_4) = \{1, 2\}$ and $L(v_5) = \{1, 2\}$.}

Fixer gets a winning strategy by coloring $v_1v_2$ with $0$ and applying Lemma \ref{CanColorAndPlayOnRest}.

\noindent\textbf{Case 18.  }\textit{$L(v_1) = \{0, 1\}$, $L(v_2) = \{0, 1, 2\}$, $L(v_3) = \{0, 2, 3\}$, $L(v_4) = \{1, 3\}$ and $L(v_5) = \{1, 3\}$.}

Fixer gets a winning strategy by coloring $v_1v_2$ with $0$ and applying Lemma \ref{CanColorAndPlayOnRest}.

\noindent\textbf{Case 19.  }\textit{$L(v_1) = \{0, 1\}$, $L(v_2) = \{0, 2, 3\}$, $L(v_3) = \{0, 1, 2\}$, $L(v_4) = \{0, 1\}$ and $L(v_5) = \{2, 3\}$.}

Fixer gets a winning strategy by coloring $v_1v_2$ with $0$ and applying Lemma \ref{CanColorAndPlayOnRest}.

\noindent\textbf{Case 20.  }\textit{$L(v_1) = \{0, 1\}$, $L(v_2) = \{0, 2, 3\}$, $L(v_3) = \{0, 1, 2\}$, $L(v_4) = \{0, 2\}$ and $L(v_5) = \{0, 2\}$.}

Fixer gets a winning strategy by coloring $v_1v_2$ with $0$ and applying Lemma \ref{CanColorAndPlayOnRest}.

\noindent\textbf{Case 21.  }\textit{$L(v_1) = \{0, 1\}$, $L(v_2) = \{0, 2, 3\}$, $L(v_3) = \{0, 1, 2\}$, $L(v_4) = \{0, 2\}$ and $L(v_5) = \{0, 3\}$.}

Fixer gets a winning strategy by coloring $v_1v_2$ with $0$ and applying Lemma \ref{CanColorAndPlayOnRest}.

\noindent\textbf{Case 22.  }\textit{$L(v_1) = \{0, 1\}$, $L(v_2) = \{0, 2, 3\}$, $L(v_3) = \{0, 1, 2\}$, $L(v_4) = \{0, 2\}$ and $L(v_5) = \{2, 3\}$.}

Fixer gets a winning strategy by coloring $v_1v_2$ with $0$ and applying Lemma \ref{CanColorAndPlayOnRest}.

\noindent\textbf{Case 23.  }\textit{$L(v_1) = \{0, 1\}$, $L(v_2) = \{0, 2, 3\}$, $L(v_3) = \{0, 1, 2\}$, $L(v_4) = \{1, 2\}$ and $L(v_5) = \{1, 2\}$.}

Fixer gets a winning strategy by coloring $v_1v_2$ with $0$ and applying Lemma \ref{CanColorAndPlayOnRest}.

\noindent\textbf{Case 24.  }\textit{$L(v_1) = \{0, 1\}$, $L(v_2) = \{0, 2, 3\}$, $L(v_3) = \{0, 1, 2\}$, $L(v_4) = \{1, 2\}$ and $L(v_5) = \{1, 3\}$.}

Fixer gets a winning strategy by coloring $v_1v_2$ with $0$ and applying Lemma \ref{CanColorAndPlayOnRest}.

\noindent\textbf{Case 25.  }\textit{$L(v_1) = \{0, 1\}$, $L(v_2) = \{0, 2, 3\}$, $L(v_3) = \{0, 1, 2\}$, $L(v_4) = \{1, 2\}$ and $L(v_5) = \{2, 3\}$.}

Fixer gets a winning strategy by coloring $v_1v_2$ with $0$ and applying Lemma \ref{CanColorAndPlayOnRest}.

\noindent\textbf{Case 26.  }\textit{$L(v_1) = \{0, 1\}$, $L(v_2) = \{0, 2, 3\}$, $L(v_3) = \{0, 1, 2\}$, $L(v_4) = \{0, 3\}$ and $L(v_5) = \{0, 2\}$.}

Fixer gets a winning strategy by coloring $v_1v_2$ with $0$ and applying Lemma \ref{CanColorAndPlayOnRest}.

\noindent\textbf{Case 27.  }\textit{$L(v_1) = \{0, 1\}$, $L(v_2) = \{0, 2, 3\}$, $L(v_3) = \{0, 1, 2\}$, $L(v_4) = \{0, 3\}$ and $L(v_5) = \{0, 3\}$.}

Fixer gets a winning strategy by coloring $v_1v_2$ with $0$ and applying Lemma \ref{CanColorAndPlayOnRest}.

\noindent\textbf{Case 28.  }\textit{$L(v_1) = \{0, 1\}$, $L(v_2) = \{0, 2, 3\}$, $L(v_3) = \{0, 1, 2\}$, $L(v_4) = \{0, 3\}$ and $L(v_5) = \{2, 3\}$.}

Fixer gets a winning strategy by coloring $v_1v_2$ with $0$ and applying Lemma \ref{CanColorAndPlayOnRest}.

\noindent\textbf{Case 29.  }\textit{$L(v_1) = \{0, 1\}$, $L(v_2) = \{0, 2, 3\}$, $L(v_3) = \{0, 1, 2\}$, $L(v_4) = \{1, 3\}$ and $L(v_5) = \{1, 2\}$.}

Fixer gets a winning strategy by coloring $v_1v_2$ with $0$ and applying Lemma \ref{CanColorAndPlayOnRest}.

\noindent\textbf{Case 30.  }\textit{$L(v_1) = \{0, 1\}$, $L(v_2) = \{0, 2, 3\}$, $L(v_3) = \{0, 1, 2\}$, $L(v_4) = \{1, 3\}$ and $L(v_5) = \{1, 3\}$.}

Fixer gets a winning strategy by coloring $v_1v_2$ with $0$ and applying Lemma \ref{CanColorAndPlayOnRest}.

\noindent\textbf{Case 31.  }\textit{$L(v_1) = \{0, 1\}$, $L(v_2) = \{0, 2, 3\}$, $L(v_3) = \{0, 1, 2\}$, $L(v_4) = \{1, 3\}$ and $L(v_5) = \{2, 3\}$.}

Fixer gets a winning strategy by coloring $v_1v_2$ with $0$ and applying Lemma \ref{CanColorAndPlayOnRest}.

\noindent\textbf{Case 32.  }\textit{$L(v_1) = \{0, 1\}$, $L(v_2) = \{0, 2, 3\}$, $L(v_3) = \{0, 1, 2\}$, $L(v_4) = \{2, 3\}$ and $L(v_5) = \{0, 1\}$.}

Fixer gets a winning strategy by coloring $v_1v_2$ with $0$ and applying Lemma \ref{CanColorAndPlayOnRest}.

\noindent\textbf{Case 33.  }\textit{$L(v_1) = \{0, 1\}$, $L(v_2) = \{0, 2, 3\}$, $L(v_3) = \{0, 1, 2\}$, $L(v_4) = \{2, 3\}$ and $L(v_5) = \{0, 2\}$.}

Fixer gets a winning strategy by coloring $v_1v_2$ with $0$ and applying Lemma \ref{CanColorAndPlayOnRest}.

\noindent\textbf{Case 34.  }\textit{$L(v_1) = \{0, 1\}$, $L(v_2) = \{0, 2, 3\}$, $L(v_3) = \{0, 1, 2\}$, $L(v_4) = \{2, 3\}$ and $L(v_5) = \{1, 2\}$.}

Fixer gets a winning strategy by coloring $v_1v_2$ with $0$ and applying Lemma \ref{CanColorAndPlayOnRest}.

\noindent\textbf{Case 35.  }\textit{$L(v_1) = \{0, 1\}$, $L(v_2) = \{0, 2, 3\}$, $L(v_3) = \{0, 1, 2\}$, $L(v_4) = \{2, 3\}$ and $L(v_5) = \{0, 3\}$.}

Fixer gets a winning strategy by coloring $v_1v_2$ with $0$ and applying Lemma \ref{CanColorAndPlayOnRest}.

\noindent\textbf{Case 36.  }\textit{$L(v_1) = \{0, 1\}$, $L(v_2) = \{0, 2, 3\}$, $L(v_3) = \{0, 1, 2\}$, $L(v_4) = \{2, 3\}$ and $L(v_5) = \{1, 3\}$.}

Fixer gets a winning strategy by coloring $v_1v_2$ with $0$ and applying Lemma \ref{CanColorAndPlayOnRest}.

\noindent\textbf{Case 37.  }\textit{$L(v_1) = \{0, 1\}$, $L(v_2) = \{0, 2, 3\}$, $L(v_3) = \{0, 1, 2\}$, $L(v_4) = \{2, 3\}$ and $L(v_5) = \{2, 3\}$.}

Fixer gets a winning strategy by coloring $v_1v_2$ with $0$ and applying Lemma \ref{CanColorAndPlayOnRest}.

\noindent\textbf{Case 38.  }\textit{$L(v_1) = \{0, 1\}$, $L(v_2) = \{0, 2, 3\}$, $L(v_3) = \{0, 2, 3\}$, $L(v_4) = \{0, 1\}$ and $L(v_5) = \{0, 1\}$.}

Fixer gets a winning strategy by coloring $v_1v_2$ with $0$ and applying Lemma \ref{CanColorAndPlayOnRest}.

\noindent\textbf{Case 39.  }\textit{$L(v_1) = \{0, 1\}$, $L(v_2) = \{0, 2, 3\}$, $L(v_3) = \{0, 2, 3\}$, $L(v_4) = \{1, 2\}$ and $L(v_5) = \{1, 2\}$.}

Fixer gets a winning strategy by coloring $v_1v_2$ with $0$ and applying Lemma \ref{CanColorAndPlayOnRest}.

\noindent\textbf{Case 40.  }\textit{$L(v_1) = \{0, 1\}$, $L(v_2) = \{0, 2, 3\}$, $L(v_3) = \{0, 2, 3\}$, $L(v_4) = \{1, 2\}$ and $L(v_5) = \{1, 3\}$.}

Fixer gets a winning strategy by coloring $v_1v_2$ with $0$ and applying Lemma \ref{CanColorAndPlayOnRest}.

\noindent\textbf{Case 41.  }\textit{$L(v_1) = \{0, 1\}$, $L(v_2) = \{0, 2, 3\}$, $L(v_3) = \{0, 2, 3\}$, $L(v_4) = \{1, 2\}$ and $L(v_5) = \{2, 3\}$.}

Fixer gets a winning strategy by coloring $v_1v_2$ with $0$ and applying Lemma \ref{CanColorAndPlayOnRest}.

\noindent\textbf{Case 42.  }\textit{$L(v_1) = \{0, 1\}$, $L(v_2) = \{0, 2, 3\}$, $L(v_3) = \{0, 2, 3\}$, $L(v_4) = \{2, 3\}$ and $L(v_5) = \{1, 2\}$.}

Fixer gets a winning strategy by coloring $v_1v_2$ with $0$ and applying Lemma \ref{CanColorAndPlayOnRest}.

\noindent\textbf{Case 43.  }\textit{$L(v_1) = \{0, 1\}$, $L(v_2) = \{0, 2, 3\}$, $L(v_3) = \{0, 2, 3\}$, $L(v_4) = \{2, 3\}$ and $L(v_5) = \{2, 3\}$.}

Fixer gets a winning strategy by coloring $v_1v_2$ with $0$ and applying Lemma \ref{CanColorAndPlayOnRest}.

\noindent\textbf{Case 44.  }\textit{$L(v_1) = \{0, 1\}$, $L(v_2) = \{0, 2, 3\}$, $L(v_3) = \{1, 2, 3\}$, $L(v_4) = \{0, 1\}$ and $L(v_5) = \{0, 1\}$.}

Fixer gets a winning strategy by coloring $v_1v_2$ with $0$ and applying Lemma \ref{CanColorAndPlayOnRest}.

\noindent\textbf{Case 45.  }\textit{$L(v_1) = \{0, 1\}$, $L(v_2) = \{0, 2, 3\}$, $L(v_3) = \{1, 2, 3\}$, $L(v_4) = \{0, 2\}$ and $L(v_5) = \{0, 2\}$.}

Fixer gets a winning strategy by coloring $v_1v_2$ with $0$ and applying Lemma \ref{CanColorAndPlayOnRest}.

\noindent\textbf{Case 46.  }\textit{$L(v_1) = \{0, 1\}$, $L(v_2) = \{0, 2, 3\}$, $L(v_3) = \{1, 2, 3\}$, $L(v_4) = \{0, 2\}$ and $L(v_5) = \{0, 3\}$.}

Fixer gets a winning strategy by coloring $v_1v_2$ with $0$ and applying Lemma \ref{CanColorAndPlayOnRest}.

\noindent\textbf{Case 47.  }\textit{$L(v_1) = \{0, 1\}$, $L(v_2) = \{0, 2, 3\}$, $L(v_3) = \{1, 2, 3\}$, $L(v_4) = \{0, 2\}$ and $L(v_5) = \{2, 3\}$.}

Fixer gets a winning strategy by coloring $v_1v_2$ with $0$ and applying Lemma \ref{CanColorAndPlayOnRest}.

\noindent\textbf{Case 48.  }\textit{$L(v_1) = \{0, 1\}$, $L(v_2) = \{0, 2, 3\}$, $L(v_3) = \{1, 2, 3\}$, $L(v_4) = \{2, 3\}$ and $L(v_5) = \{0, 2\}$.}

Fixer gets a winning strategy by coloring $v_1v_2$ with $0$ and applying Lemma \ref{CanColorAndPlayOnRest}.

\noindent\textbf{Case 49.  }\textit{$L(v_1) = \{0, 1\}$, $L(v_2) = \{0, 2, 3\}$, $L(v_3) = \{1, 2, 3\}$, $L(v_4) = \{2, 3\}$ and $L(v_5) = \{2, 3\}$.}

Fixer gets a winning strategy by coloring $v_1v_2$ with $0$ and applying Lemma \ref{CanColorAndPlayOnRest}.

\noindent\textbf{Case 50.  }\textit{$L(v_1) = \{0, 1\}$, $L(v_2) = \{0, 1, 2\}$, $L(v_3) = \{0, 1, 3\}$, $L(v_4) = \{0, 1\}$ and $L(v_5) = \{0, 4\}$.}

Fixer gets a winning strategy by coloring $v_5v_3$ with $0$ and applying Lemma \ref{CanColorAndPlayOnRest}.

\noindent\textbf{Case 51.  }\textit{$L(v_1) = \{0, 1\}$, $L(v_2) = \{0, 1, 2\}$, $L(v_3) = \{0, 1, 3\}$, $L(v_4) = \{0, 2\}$ and $L(v_5) = \{1, 4\}$.}

Fixer gets a winning strategy by coloring $v_1v_2$ with $1$ and applying Lemma \ref{CanColorAndPlayOnRest}.

\noindent\textbf{Case 52.  }\textit{$L(v_1) = \{0, 1\}$, $L(v_2) = \{0, 1, 2\}$, $L(v_3) = \{0, 1, 3\}$, $L(v_4) = \{0, 4\}$ and $L(v_5) = \{0, 1\}$.}

Fixer gets a winning strategy by coloring $v_4v_3$ with $0$ and applying Lemma \ref{CanColorAndPlayOnRest}.

\noindent\textbf{Case 53.  }\textit{$L(v_1) = \{0, 1\}$, $L(v_2) = \{0, 1, 2\}$, $L(v_3) = \{0, 1, 3\}$, $L(v_4) = \{0, 4\}$ and $L(v_5) = \{1, 2\}$.}

Fixer gets a winning strategy by coloring $v_1v_2$ with $0$ and applying Lemma \ref{CanColorAndPlayOnRest}.

\noindent\textbf{Case 54.  }\textit{$L(v_1) = \{0, 1\}$, $L(v_2) = \{0, 1, 2\}$, $L(v_3) = \{0, 1, 3\}$, $L(v_4) = \{0, 4\}$ and $L(v_5) = \{0, 4\}$.}

Fixer gets a winning strategy by coloring $v_1v_2$ with $0$ and applying Lemma \ref{CanColorAndPlayOnRest}.

\noindent\textbf{Case 55.  }\textit{$L(v_1) = \{0, 1\}$, $L(v_2) = \{0, 1, 2\}$, $L(v_3) = \{0, 1, 3\}$, $L(v_4) = \{3, 4\}$ and $L(v_5) = \{3, 4\}$.}

Fixer gets a winning strategy by coloring $v_1v_2$ with $0$ and applying Lemma \ref{CanColorAndPlayOnRest}.

\noindent\textbf{Case 56.  }\textit{$L(v_1) = \{0, 1\}$, $L(v_2) = \{0, 1, 2\}$, $L(v_3) = \{0, 2, 3\}$, $L(v_4) = \{0, 1\}$ and $L(v_5) = \{2, 4\}$.}

Fixer gets a winning strategy by coloring $v_5v_3$ with $2$ and applying Lemma \ref{CanColorAndPlayOnRest}.

\noindent\textbf{Case 57.  }\textit{$L(v_1) = \{0, 1\}$, $L(v_2) = \{0, 1, 2\}$, $L(v_3) = \{0, 2, 3\}$, $L(v_4) = \{0, 2\}$ and $L(v_5) = \{0, 4\}$.}

Fixer gets a winning strategy by coloring $v_1v_2$ with $1$ and applying Lemma \ref{CanColorAndPlayOnRest}.

\noindent\textbf{Case 58.  }\textit{$L(v_1) = \{0, 1\}$, $L(v_2) = \{0, 1, 2\}$, $L(v_3) = \{0, 2, 3\}$, $L(v_4) = \{0, 2\}$ and $L(v_5) = \{2, 4\}$.}

Fixer gets a winning strategy by coloring $v_1v_2$ with $1$ and applying Lemma \ref{CanColorAndPlayOnRest}.

\noindent\textbf{Case 59.  }\textit{$L(v_1) = \{0, 1\}$, $L(v_2) = \{0, 1, 2\}$, $L(v_3) = \{0, 2, 3\}$, $L(v_4) = \{1, 2\}$ and $L(v_5) = \{0, 4\}$.}

Fixer gets a winning strategy by coloring $v_1v_2$ with $0$ and applying Lemma \ref{CanColorAndPlayOnRest}.

\noindent\textbf{Case 60.  }\textit{$L(v_1) = \{0, 1\}$, $L(v_2) = \{0, 1, 2\}$, $L(v_3) = \{0, 2, 3\}$, $L(v_4) = \{0, 4\}$ and $L(v_5) = \{0, 2\}$.}

Fixer gets a winning strategy by coloring $v_1v_2$ with $1$ and applying Lemma \ref{CanColorAndPlayOnRest}.

\noindent\textbf{Case 61.  }\textit{$L(v_1) = \{0, 1\}$, $L(v_2) = \{0, 1, 2\}$, $L(v_3) = \{0, 2, 3\}$, $L(v_4) = \{0, 4\}$ and $L(v_5) = \{1, 2\}$.}

Fixer gets a winning strategy by coloring $v_1v_2$ with $0$ and applying Lemma \ref{CanColorAndPlayOnRest}.

\noindent\textbf{Case 62.  }\textit{$L(v_1) = \{0, 1\}$, $L(v_2) = \{0, 1, 2\}$, $L(v_3) = \{0, 2, 3\}$, $L(v_4) = \{0, 4\}$ and $L(v_5) = \{0, 4\}$.}

Fixer gets a winning strategy by coloring $v_1v_2$ with $0$ and applying Lemma \ref{CanColorAndPlayOnRest}.

\noindent\textbf{Case 63.  }\textit{$L(v_1) = \{0, 1\}$, $L(v_2) = \{0, 1, 2\}$, $L(v_3) = \{0, 2, 3\}$, $L(v_4) = \{0, 4\}$ and $L(v_5) = \{2, 4\}$.}

Fixer gets a winning strategy by coloring $v_1v_2$ with $1$ and applying Lemma \ref{CanColorAndPlayOnRest}.

\noindent\textbf{Case 64.  }\textit{$L(v_1) = \{0, 1\}$, $L(v_2) = \{0, 1, 2\}$, $L(v_3) = \{0, 2, 3\}$, $L(v_4) = \{2, 4\}$ and $L(v_5) = \{0, 1\}$.}

Fixer gets a winning strategy by coloring $v_4v_3$ with $2$ and applying Lemma \ref{CanColorAndPlayOnRest}.

\noindent\textbf{Case 65.  }\textit{$L(v_1) = \{0, 1\}$, $L(v_2) = \{0, 1, 2\}$, $L(v_3) = \{0, 2, 3\}$, $L(v_4) = \{2, 4\}$ and $L(v_5) = \{0, 2\}$.}

Fixer gets a winning strategy by coloring $v_1v_2$ with $1$ and applying Lemma \ref{CanColorAndPlayOnRest}.

\noindent\textbf{Case 66.  }\textit{$L(v_1) = \{0, 1\}$, $L(v_2) = \{0, 1, 2\}$, $L(v_3) = \{0, 2, 3\}$, $L(v_4) = \{2, 4\}$ and $L(v_5) = \{0, 4\}$.}

Fixer gets a winning strategy by coloring $v_1v_2$ with $1$ and applying Lemma \ref{CanColorAndPlayOnRest}.

\noindent\textbf{Case 67.  }\textit{$L(v_1) = \{0, 1\}$, $L(v_2) = \{0, 1, 2\}$, $L(v_3) = \{0, 2, 3\}$, $L(v_4) = \{2, 4\}$ and $L(v_5) = \{2, 4\}$.}

Fixer gets a winning strategy by coloring $v_1v_2$ with $1$ and applying Lemma \ref{CanColorAndPlayOnRest}.

\noindent\textbf{Case 68.  }\textit{$L(v_1) = \{0, 1\}$, $L(v_2) = \{0, 1, 2\}$, $L(v_3) = \{0, 2, 3\}$, $L(v_4) = \{3, 4\}$ and $L(v_5) = \{3, 4\}$.}

Fixer gets a winning strategy by coloring $v_1v_2$ with $0$ and applying Lemma \ref{CanColorAndPlayOnRest}.

\noindent\textbf{Case 69.  }\textit{$L(v_1) = \{0, 1\}$, $L(v_2) = \{0, 1, 2\}$, $L(v_3) = \{0, 3, 4\}$, $L(v_4) = \{0, 1\}$ and $L(v_5) = \{0, 3\}$.}

Fixer gets a winning strategy by coloring $v_5v_3$ with $3$ and applying Lemma \ref{CanColorAndPlayOnRest}.

\noindent\textbf{Case 70.  }\textit{$L(v_1) = \{0, 1\}$, $L(v_2) = \{0, 1, 2\}$, $L(v_3) = \{0, 3, 4\}$, $L(v_4) = \{0, 1\}$ and $L(v_5) = \{1, 3\}$.}

Fixer gets a winning strategy by coloring $v_5v_3$ with $3$ and applying Lemma \ref{CanColorAndPlayOnRest}.

\noindent\textbf{Case 71.  }\textit{$L(v_1) = \{0, 1\}$, $L(v_2) = \{0, 1, 2\}$, $L(v_3) = \{0, 3, 4\}$, $L(v_4) = \{0, 1\}$ and $L(v_5) = \{2, 3\}$.}

Fixer gets a winning strategy by coloring $v_5v_3$ with $3$ and applying Lemma \ref{CanColorAndPlayOnRest}.

\noindent\textbf{Case 72.  }\textit{$L(v_1) = \{0, 1\}$, $L(v_2) = \{0, 1, 2\}$, $L(v_3) = \{0, 3, 4\}$, $L(v_4) = \{0, 1\}$ and $L(v_5) = \{3, 4\}$.}

Fixer gets a winning strategy by coloring $v_5v_3$ with $3$ and applying Lemma \ref{CanColorAndPlayOnRest}.

\noindent\textbf{Case 73.  }\textit{$L(v_1) = \{0, 1\}$, $L(v_2) = \{0, 1, 2\}$, $L(v_3) = \{0, 3, 4\}$, $L(v_4) = \{0, 2\}$ and $L(v_5) = \{0, 2\}$.}

Fixer gets a winning strategy by coloring $v_1v_2$ with $1$ and applying Lemma \ref{CanColorAndPlayOnRest}.

\noindent\textbf{Case 74.  }\textit{$L(v_1) = \{0, 1\}$, $L(v_2) = \{0, 1, 2\}$, $L(v_3) = \{0, 3, 4\}$, $L(v_4) = \{0, 2\}$ and $L(v_5) = \{0, 3\}$.}

Fixer gets a winning strategy by coloring $v_1v_2$ with $1$ and applying Lemma \ref{CanColorAndPlayOnRest}.

\noindent\textbf{Case 75.  }\textit{$L(v_1) = \{0, 1\}$, $L(v_2) = \{0, 1, 2\}$, $L(v_3) = \{0, 3, 4\}$, $L(v_4) = \{0, 2\}$ and $L(v_5) = \{1, 3\}$.}

Fixer gets a winning strategy by coloring $v_1v_2$ with $1$ and applying Lemma \ref{CanColorAndPlayOnRest}.

\noindent\textbf{Case 76.  }\textit{$L(v_1) = \{0, 1\}$, $L(v_2) = \{0, 1, 2\}$, $L(v_3) = \{0, 3, 4\}$, $L(v_4) = \{0, 2\}$ and $L(v_5) = \{2, 3\}$.}

Fixer gets a winning strategy by coloring $v_1v_2$ with $1$ and applying Lemma \ref{CanColorAndPlayOnRest}.

\noindent\textbf{Case 77.  }\textit{$L(v_1) = \{0, 1\}$, $L(v_2) = \{0, 1, 2\}$, $L(v_3) = \{0, 3, 4\}$, $L(v_4) = \{0, 2\}$ and $L(v_5) = \{3, 4\}$.}

Fixer gets a winning strategy by coloring $v_1v_2$ with $1$ and applying Lemma \ref{CanColorAndPlayOnRest}.

\noindent\textbf{Case 78.  }\textit{$L(v_1) = \{0, 1\}$, $L(v_2) = \{0, 1, 2\}$, $L(v_3) = \{0, 3, 4\}$, $L(v_4) = \{0, 3\}$ and $L(v_5) = \{0, 1\}$.}

Fixer gets a winning strategy by coloring $v_4v_3$ with $3$ and applying Lemma \ref{CanColorAndPlayOnRest}.

\noindent\textbf{Case 79.  }\textit{$L(v_1) = \{0, 1\}$, $L(v_2) = \{0, 1, 2\}$, $L(v_3) = \{0, 3, 4\}$, $L(v_4) = \{0, 3\}$ and $L(v_5) = \{0, 2\}$.}

Fixer gets a winning strategy by coloring $v_1v_2$ with $1$ and applying Lemma \ref{CanColorAndPlayOnRest}.

\noindent\textbf{Case 80.  }\textit{$L(v_1) = \{0, 1\}$, $L(v_2) = \{0, 1, 2\}$, $L(v_3) = \{0, 3, 4\}$, $L(v_4) = \{0, 3\}$ and $L(v_5) = \{0, 3\}$.}

Fixer gets a winning strategy by coloring $v_1v_2$ with $1$ and applying Lemma \ref{CanColorAndPlayOnRest}.

\noindent\textbf{Case 81.  }\textit{$L(v_1) = \{0, 1\}$, $L(v_2) = \{0, 1, 2\}$, $L(v_3) = \{0, 3, 4\}$, $L(v_4) = \{0, 3\}$ and $L(v_5) = \{2, 3\}$.}

Fixer gets a winning strategy by coloring $v_1v_2$ with $1$ and applying Lemma \ref{CanColorAndPlayOnRest}.

\noindent\textbf{Case 82.  }\textit{$L(v_1) = \{0, 1\}$, $L(v_2) = \{0, 1, 2\}$, $L(v_3) = \{0, 3, 4\}$, $L(v_4) = \{1, 3\}$ and $L(v_5) = \{0, 1\}$.}

Fixer gets a winning strategy by coloring $v_4v_3$ with $3$ and applying Lemma \ref{CanColorAndPlayOnRest}.

\noindent\textbf{Case 83.  }\textit{$L(v_1) = \{0, 1\}$, $L(v_2) = \{0, 1, 2\}$, $L(v_3) = \{0, 3, 4\}$, $L(v_4) = \{1, 3\}$ and $L(v_5) = \{0, 2\}$.}

Fixer gets a winning strategy by coloring $v_1v_2$ with $1$ and applying Lemma \ref{CanColorAndPlayOnRest}.

\noindent\textbf{Case 84.  }\textit{$L(v_1) = \{0, 1\}$, $L(v_2) = \{0, 1, 2\}$, $L(v_3) = \{0, 3, 4\}$, $L(v_4) = \{1, 3\}$ and $L(v_5) = \{1, 3\}$.}

Fixer gets a winning strategy by coloring $v_1v_2$ with $1$ and applying Lemma \ref{CanColorAndPlayOnRest}.

\noindent\textbf{Case 85.  }\textit{$L(v_1) = \{0, 1\}$, $L(v_2) = \{0, 1, 2\}$, $L(v_3) = \{0, 3, 4\}$, $L(v_4) = \{2, 3\}$ and $L(v_5) = \{0, 1\}$.}

Fixer gets a winning strategy by coloring $v_4v_3$ with $3$ and applying Lemma \ref{CanColorAndPlayOnRest}.

\noindent\textbf{Case 86.  }\textit{$L(v_1) = \{0, 1\}$, $L(v_2) = \{0, 1, 2\}$, $L(v_3) = \{0, 3, 4\}$, $L(v_4) = \{2, 3\}$ and $L(v_5) = \{0, 2\}$.}

Fixer gets a winning strategy by coloring $v_1v_2$ with $1$ and applying Lemma \ref{CanColorAndPlayOnRest}.

\noindent\textbf{Case 87.  }\textit{$L(v_1) = \{0, 1\}$, $L(v_2) = \{0, 1, 2\}$, $L(v_3) = \{0, 3, 4\}$, $L(v_4) = \{2, 3\}$ and $L(v_5) = \{0, 3\}$.}

Fixer gets a winning strategy by coloring $v_1v_2$ with $1$ and applying Lemma \ref{CanColorAndPlayOnRest}.

\noindent\textbf{Case 88.  }\textit{$L(v_1) = \{0, 1\}$, $L(v_2) = \{0, 1, 2\}$, $L(v_3) = \{0, 3, 4\}$, $L(v_4) = \{2, 3\}$ and $L(v_5) = \{2, 3\}$.}

Fixer gets a winning strategy by coloring $v_1v_2$ with $1$ and applying Lemma \ref{CanColorAndPlayOnRest}.

\noindent\textbf{Case 89.  }\textit{$L(v_1) = \{0, 1\}$, $L(v_2) = \{0, 1, 2\}$, $L(v_3) = \{0, 3, 4\}$, $L(v_4) = \{3, 4\}$ and $L(v_5) = \{0, 1\}$.}

Fixer gets a winning strategy by coloring $v_4v_3$ with $3$ and applying Lemma \ref{CanColorAndPlayOnRest}.

\noindent\textbf{Case 90.  }\textit{$L(v_1) = \{0, 1\}$, $L(v_2) = \{0, 1, 2\}$, $L(v_3) = \{0, 3, 4\}$, $L(v_4) = \{3, 4\}$ and $L(v_5) = \{0, 2\}$.}

Fixer gets a winning strategy by coloring $v_1v_2$ with $1$ and applying Lemma \ref{CanColorAndPlayOnRest}.

\noindent\textbf{Case 91.  }\textit{$L(v_1) = \{0, 1\}$, $L(v_2) = \{0, 1, 2\}$, $L(v_3) = \{2, 3, 4\}$, $L(v_4) = \{0, 2\}$ and $L(v_5) = \{0, 2\}$.}

Fixer gets a winning strategy by coloring $v_1v_2$ with $1$ and applying Lemma \ref{CanColorAndPlayOnRest}.

\noindent\textbf{Case 92.  }\textit{$L(v_1) = \{0, 1\}$, $L(v_2) = \{0, 1, 2\}$, $L(v_3) = \{2, 3, 4\}$, $L(v_4) = \{0, 2\}$ and $L(v_5) = \{0, 3\}$.}

Fixer gets a winning strategy by coloring $v_1v_2$ with $1$ and applying Lemma \ref{CanColorAndPlayOnRest}.

\noindent\textbf{Case 93.  }\textit{$L(v_1) = \{0, 1\}$, $L(v_2) = \{0, 1, 2\}$, $L(v_3) = \{2, 3, 4\}$, $L(v_4) = \{0, 2\}$ and $L(v_5) = \{1, 3\}$.}

Fixer gets a winning strategy by coloring $v_1v_2$ with $1$ and applying Lemma \ref{CanColorAndPlayOnRest}.

\noindent\textbf{Case 94.  }\textit{$L(v_1) = \{0, 1\}$, $L(v_2) = \{0, 1, 2\}$, $L(v_3) = \{2, 3, 4\}$, $L(v_4) = \{0, 2\}$ and $L(v_5) = \{2, 3\}$.}

Fixer gets a winning strategy by coloring $v_1v_2$ with $1$ and applying Lemma \ref{CanColorAndPlayOnRest}.

\noindent\textbf{Case 95.  }\textit{$L(v_1) = \{0, 1\}$, $L(v_2) = \{0, 1, 2\}$, $L(v_3) = \{2, 3, 4\}$, $L(v_4) = \{0, 2\}$ and $L(v_5) = \{3, 4\}$.}

Fixer gets a winning strategy by coloring $v_1v_2$ with $1$ and applying Lemma \ref{CanColorAndPlayOnRest}.

\noindent\textbf{Case 96.  }\textit{$L(v_1) = \{0, 1\}$, $L(v_2) = \{0, 1, 2\}$, $L(v_3) = \{2, 3, 4\}$, $L(v_4) = \{0, 3\}$ and $L(v_5) = \{0, 2\}$.}

Fixer gets a winning strategy by coloring $v_1v_2$ with $1$ and applying Lemma \ref{CanColorAndPlayOnRest}.

\noindent\textbf{Case 97.  }\textit{$L(v_1) = \{0, 1\}$, $L(v_2) = \{0, 1, 2\}$, $L(v_3) = \{2, 3, 4\}$, $L(v_4) = \{0, 3\}$ and $L(v_5) = \{1, 2\}$.}

Fixer gets a winning strategy by coloring $v_1v_2$ with $0$ and applying Lemma \ref{CanColorAndPlayOnRest}.

\noindent\textbf{Case 98.  }\textit{$L(v_1) = \{0, 1\}$, $L(v_2) = \{0, 1, 2\}$, $L(v_3) = \{2, 3, 4\}$, $L(v_4) = \{0, 3\}$ and $L(v_5) = \{0, 3\}$.}

Fixer gets a winning strategy by coloring $v_1v_2$ with $0$ and applying Lemma \ref{CanColorAndPlayOnRest}.

\noindent\textbf{Case 99.  }\textit{$L(v_1) = \{0, 1\}$, $L(v_2) = \{0, 1, 2\}$, $L(v_3) = \{2, 3, 4\}$, $L(v_4) = \{0, 3\}$ and $L(v_5) = \{2, 3\}$.}

Fixer gets a winning strategy by coloring $v_1v_2$ with $1$ and applying Lemma \ref{CanColorAndPlayOnRest}.

\noindent\textbf{Case 100.  }\textit{$L(v_1) = \{0, 1\}$, $L(v_2) = \{0, 1, 2\}$, $L(v_3) = \{2, 3, 4\}$, $L(v_4) = \{2, 3\}$ and $L(v_5) = \{0, 2\}$.}

Fixer gets a winning strategy by coloring $v_1v_2$ with $1$ and applying Lemma \ref{CanColorAndPlayOnRest}.

\noindent\textbf{Case 101.  }\textit{$L(v_1) = \{0, 1\}$, $L(v_2) = \{0, 1, 2\}$, $L(v_3) = \{2, 3, 4\}$, $L(v_4) = \{2, 3\}$ and $L(v_5) = \{0, 3\}$.}

Fixer gets a winning strategy by coloring $v_1v_2$ with $1$ and applying Lemma \ref{CanColorAndPlayOnRest}.

\noindent\textbf{Case 102.  }\textit{$L(v_1) = \{0, 1\}$, $L(v_2) = \{0, 1, 2\}$, $L(v_3) = \{2, 3, 4\}$, $L(v_4) = \{2, 3\}$ and $L(v_5) = \{2, 3\}$.}

Fixer gets a winning strategy by coloring $v_1v_2$ with $0$ and applying Lemma \ref{CanColorAndPlayOnRest}.

\noindent\textbf{Case 103.  }\textit{$L(v_1) = \{0, 1\}$, $L(v_2) = \{0, 1, 2\}$, $L(v_3) = \{2, 3, 4\}$, $L(v_4) = \{3, 4\}$ and $L(v_5) = \{0, 2\}$.}

Fixer gets a winning strategy by coloring $v_1v_2$ with $1$ and applying Lemma \ref{CanColorAndPlayOnRest}.

\noindent\textbf{Case 104.  }\textit{$L(v_1) = \{0, 1\}$, $L(v_2) = \{0, 2, 3\}$, $L(v_3) = \{0, 1, 2\}$, $L(v_4) = \{0, 1\}$ and $L(v_5) = \{2, 4\}$.}

Fixer gets a winning strategy by coloring $v_5v_3$ with $2$ and applying Lemma \ref{CanColorAndPlayOnRest}.

\noindent\textbf{Case 105.  }\textit{$L(v_1) = \{0, 1\}$, $L(v_2) = \{0, 2, 3\}$, $L(v_3) = \{0, 1, 2\}$, $L(v_4) = \{0, 2\}$ and $L(v_5) = \{0, 4\}$.}

Fixer gets a winning strategy by coloring $v_5v_3$ with $0$ and applying Lemma \ref{CanColorAndPlayOnRest}.

\noindent\textbf{Case 106.  }\textit{$L(v_1) = \{0, 1\}$, $L(v_2) = \{0, 2, 3\}$, $L(v_3) = \{0, 1, 2\}$, $L(v_4) = \{0, 2\}$ and $L(v_5) = \{2, 4\}$.}

Fixer gets a winning strategy by coloring $v_5v_3$ with $2$ and applying Lemma \ref{CanColorAndPlayOnRest}.

\noindent\textbf{Case 107.  }\textit{$L(v_1) = \{0, 1\}$, $L(v_2) = \{0, 2, 3\}$, $L(v_3) = \{0, 1, 2\}$, $L(v_4) = \{1, 2\}$ and $L(v_5) = \{1, 4\}$.}

Fixer gets a winning strategy by coloring $v_5v_3$ with $1$ and applying Lemma \ref{CanColorAndPlayOnRest}.

\noindent\textbf{Case 108.  }\textit{$L(v_1) = \{0, 1\}$, $L(v_2) = \{0, 2, 3\}$, $L(v_3) = \{0, 1, 2\}$, $L(v_4) = \{1, 2\}$ and $L(v_5) = \{2, 4\}$.}

Fixer gets a winning strategy by coloring $v_5v_3$ with $2$ and applying Lemma \ref{CanColorAndPlayOnRest}.

\noindent\textbf{Case 109.  }\textit{$L(v_1) = \{0, 1\}$, $L(v_2) = \{0, 2, 3\}$, $L(v_3) = \{0, 1, 2\}$, $L(v_4) = \{0, 3\}$ and $L(v_5) = \{2, 4\}$.}

Fixer gets a winning strategy by coloring $v_5v_3$ with $2$ and applying Lemma \ref{CanColorAndPlayOnRest}.

\noindent\textbf{Case 110.  }\textit{$L(v_1) = \{0, 1\}$, $L(v_2) = \{0, 2, 3\}$, $L(v_3) = \{0, 1, 2\}$, $L(v_4) = \{1, 3\}$ and $L(v_5) = \{2, 4\}$.}

Fixer gets a winning strategy by coloring $v_5v_3$ with $2$ and applying Lemma \ref{CanColorAndPlayOnRest}.

\noindent\textbf{Case 111.  }\textit{$L(v_1) = \{0, 1\}$, $L(v_2) = \{0, 2, 3\}$, $L(v_3) = \{0, 1, 2\}$, $L(v_4) = \{2, 3\}$ and $L(v_5) = \{0, 4\}$.}

Fixer gets a winning strategy by coloring $v_1v_2$ with $0$ and applying Lemma \ref{CanColorAndPlayOnRest}.

\noindent\textbf{Case 112.  }\textit{$L(v_1) = \{0, 1\}$, $L(v_2) = \{0, 2, 3\}$, $L(v_3) = \{0, 1, 2\}$, $L(v_4) = \{2, 3\}$ and $L(v_5) = \{1, 4\}$.}

Fixer gets a winning strategy by coloring $v_1v_2$ with $0$ and applying Lemma \ref{CanColorAndPlayOnRest}.

\noindent\textbf{Case 113.  }\textit{$L(v_1) = \{0, 1\}$, $L(v_2) = \{0, 2, 3\}$, $L(v_3) = \{0, 1, 2\}$, $L(v_4) = \{0, 4\}$ and $L(v_5) = \{0, 2\}$.}

Fixer gets a winning strategy by coloring $v_4v_3$ with $0$ and applying Lemma \ref{CanColorAndPlayOnRest}.

\noindent\textbf{Case 114.  }\textit{$L(v_1) = \{0, 1\}$, $L(v_2) = \{0, 2, 3\}$, $L(v_3) = \{0, 1, 2\}$, $L(v_4) = \{0, 4\}$ and $L(v_5) = \{2, 3\}$.}

Fixer gets a winning strategy by coloring $v_1v_2$ with $0$ and applying Lemma \ref{CanColorAndPlayOnRest}.

\noindent\textbf{Case 115.  }\textit{$L(v_1) = \{0, 1\}$, $L(v_2) = \{0, 2, 3\}$, $L(v_3) = \{0, 1, 2\}$, $L(v_4) = \{0, 4\}$ and $L(v_5) = \{0, 4\}$.}

Fixer gets a winning strategy by coloring $v_1v_2$ with $0$ and applying Lemma \ref{CanColorAndPlayOnRest}.

\noindent\textbf{Case 116.  }\textit{$L(v_1) = \{0, 1\}$, $L(v_2) = \{0, 2, 3\}$, $L(v_3) = \{0, 1, 2\}$, $L(v_4) = \{1, 4\}$ and $L(v_5) = \{1, 2\}$.}

Fixer gets a winning strategy by coloring $v_4v_3$ with $1$ and applying Lemma \ref{CanColorAndPlayOnRest}.

\noindent\textbf{Case 117.  }\textit{$L(v_1) = \{0, 1\}$, $L(v_2) = \{0, 2, 3\}$, $L(v_3) = \{0, 1, 2\}$, $L(v_4) = \{1, 4\}$ and $L(v_5) = \{2, 3\}$.}

Fixer gets a winning strategy by coloring $v_1v_2$ with $0$ and applying Lemma \ref{CanColorAndPlayOnRest}.

\noindent\textbf{Case 118.  }\textit{$L(v_1) = \{0, 1\}$, $L(v_2) = \{0, 2, 3\}$, $L(v_3) = \{0, 1, 2\}$, $L(v_4) = \{1, 4\}$ and $L(v_5) = \{1, 4\}$.}

Fixer gets a winning strategy by coloring $v_1v_2$ with $0$ and applying Lemma \ref{CanColorAndPlayOnRest}.

\noindent\textbf{Case 119.  }\textit{$L(v_1) = \{0, 1\}$, $L(v_2) = \{0, 2, 3\}$, $L(v_3) = \{0, 1, 2\}$, $L(v_4) = \{2, 4\}$ and $L(v_5) = \{0, 1\}$.}

Fixer gets a winning strategy by coloring $v_4v_3$ with $2$ and applying Lemma \ref{CanColorAndPlayOnRest}.

\noindent\textbf{Case 120.  }\textit{$L(v_1) = \{0, 1\}$, $L(v_2) = \{0, 2, 3\}$, $L(v_3) = \{0, 1, 2\}$, $L(v_4) = \{2, 4\}$ and $L(v_5) = \{0, 2\}$.}

Fixer gets a winning strategy by coloring $v_4v_3$ with $2$ and applying Lemma \ref{CanColorAndPlayOnRest}.

\noindent\textbf{Case 121.  }\textit{$L(v_1) = \{0, 1\}$, $L(v_2) = \{0, 2, 3\}$, $L(v_3) = \{0, 1, 2\}$, $L(v_4) = \{2, 4\}$ and $L(v_5) = \{1, 2\}$.}

Fixer gets a winning strategy by coloring $v_4v_3$ with $2$ and applying Lemma \ref{CanColorAndPlayOnRest}.

\noindent\textbf{Case 122.  }\textit{$L(v_1) = \{0, 1\}$, $L(v_2) = \{0, 2, 3\}$, $L(v_3) = \{0, 1, 2\}$, $L(v_4) = \{2, 4\}$ and $L(v_5) = \{0, 3\}$.}

Fixer gets a winning strategy by coloring $v_4v_3$ with $2$ and applying Lemma \ref{CanColorAndPlayOnRest}.

\noindent\textbf{Case 123.  }\textit{$L(v_1) = \{0, 1\}$, $L(v_2) = \{0, 2, 3\}$, $L(v_3) = \{0, 1, 2\}$, $L(v_4) = \{2, 4\}$ and $L(v_5) = \{1, 3\}$.}

Fixer gets a winning strategy by coloring $v_4v_3$ with $2$ and applying Lemma \ref{CanColorAndPlayOnRest}.

\noindent\textbf{Case 124.  }\textit{$L(v_1) = \{0, 1\}$, $L(v_2) = \{0, 2, 3\}$, $L(v_3) = \{0, 2, 3\}$, $L(v_4) = \{1, 2\}$ and $L(v_5) = \{3, 4\}$.}

Fixer gets a winning strategy by coloring $v_5v_3$ with $3$ and applying Lemma \ref{CanColorAndPlayOnRest}.

\noindent\textbf{Case 125.  }\textit{$L(v_1) = \{0, 1\}$, $L(v_2) = \{0, 2, 3\}$, $L(v_3) = \{0, 2, 3\}$, $L(v_4) = \{2, 3\}$ and $L(v_5) = \{2, 4\}$.}

Fixer gets a winning strategy by coloring $v_1v_2$ with $0$ and applying Lemma \ref{CanColorAndPlayOnRest}.

\noindent\textbf{Case 126.  }\textit{$L(v_1) = \{0, 1\}$, $L(v_2) = \{0, 2, 3\}$, $L(v_3) = \{0, 2, 3\}$, $L(v_4) = \{0, 4\}$ and $L(v_5) = \{0, 4\}$.}

Fixer gets a winning strategy by coloring $v_1v_2$ with $0$ and applying Lemma \ref{CanColorAndPlayOnRest}.

\noindent\textbf{Case 127.  }\textit{$L(v_1) = \{0, 1\}$, $L(v_2) = \{0, 2, 3\}$, $L(v_3) = \{0, 2, 3\}$, $L(v_4) = \{2, 4\}$ and $L(v_5) = \{1, 3\}$.}

Fixer gets a winning strategy by coloring $v_4v_3$ with $2$ and applying Lemma \ref{CanColorAndPlayOnRest}.

\noindent\textbf{Case 128.  }\textit{$L(v_1) = \{0, 1\}$, $L(v_2) = \{0, 2, 3\}$, $L(v_3) = \{0, 2, 3\}$, $L(v_4) = \{2, 4\}$ and $L(v_5) = \{2, 3\}$.}

Fixer gets a winning strategy by coloring $v_1v_2$ with $0$ and applying Lemma \ref{CanColorAndPlayOnRest}.

\noindent\textbf{Case 129.  }\textit{$L(v_1) = \{0, 1\}$, $L(v_2) = \{0, 2, 3\}$, $L(v_3) = \{0, 2, 3\}$, $L(v_4) = \{2, 4\}$ and $L(v_5) = \{2, 4\}$.}

Fixer gets a winning strategy by coloring $v_1v_2$ with $0$ and applying Lemma \ref{CanColorAndPlayOnRest}.

\noindent\textbf{Case 130.  }\textit{$L(v_1) = \{0, 1\}$, $L(v_2) = \{0, 2, 3\}$, $L(v_3) = \{0, 2, 3\}$, $L(v_4) = \{2, 4\}$ and $L(v_5) = \{3, 4\}$.}

Fixer gets a winning strategy by coloring $v_1v_2$ with $0$ and applying Lemma \ref{CanColorAndPlayOnRest}.

\noindent\textbf{Case 131.  }\textit{$L(v_1) = \{0, 1\}$, $L(v_2) = \{0, 2, 3\}$, $L(v_3) = \{1, 2, 3\}$, $L(v_4) = \{0, 2\}$ and $L(v_5) = \{3, 4\}$.}

Fixer gets a winning strategy by coloring $v_5v_3$ with $3$ and applying Lemma \ref{CanColorAndPlayOnRest}.

\noindent\textbf{Case 132.  }\textit{$L(v_1) = \{0, 1\}$, $L(v_2) = \{0, 2, 3\}$, $L(v_3) = \{1, 2, 3\}$, $L(v_4) = \{2, 3\}$ and $L(v_5) = \{2, 4\}$.}

Fixer gets a winning strategy by coloring $v_1v_2$ with $0$ and applying Lemma \ref{CanColorAndPlayOnRest}.

\noindent\textbf{Case 133.  }\textit{$L(v_1) = \{0, 1\}$, $L(v_2) = \{0, 2, 3\}$, $L(v_3) = \{1, 2, 3\}$, $L(v_4) = \{1, 4\}$ and $L(v_5) = \{1, 4\}$.}

Fixer gets a winning strategy by coloring $v_1v_2$ with $0$ and applying Lemma \ref{CanColorAndPlayOnRest}.

\noindent\textbf{Case 134.  }\textit{$L(v_1) = \{0, 1\}$, $L(v_2) = \{0, 2, 3\}$, $L(v_3) = \{1, 2, 3\}$, $L(v_4) = \{2, 4\}$ and $L(v_5) = \{0, 3\}$.}

Fixer gets a winning strategy by coloring $v_4v_3$ with $2$ and applying Lemma \ref{CanColorAndPlayOnRest}.

\noindent\textbf{Case 135.  }\textit{$L(v_1) = \{0, 1\}$, $L(v_2) = \{0, 2, 3\}$, $L(v_3) = \{1, 2, 3\}$, $L(v_4) = \{2, 4\}$ and $L(v_5) = \{2, 3\}$.}

Fixer gets a winning strategy by coloring $v_1v_2$ with $0$ and applying Lemma \ref{CanColorAndPlayOnRest}.

\noindent\textbf{Case 136.  }\textit{$L(v_1) = \{0, 1\}$, $L(v_2) = \{0, 2, 3\}$, $L(v_3) = \{1, 2, 3\}$, $L(v_4) = \{2, 4\}$ and $L(v_5) = \{2, 4\}$.}

Fixer gets a winning strategy by coloring $v_1v_2$ with $0$ and applying Lemma \ref{CanColorAndPlayOnRest}.

\noindent\textbf{Case 137.  }\textit{$L(v_1) = \{0, 1\}$, $L(v_2) = \{0, 2, 3\}$, $L(v_3) = \{1, 2, 3\}$, $L(v_4) = \{2, 4\}$ and $L(v_5) = \{3, 4\}$.}

Fixer gets a winning strategy by coloring $v_1v_2$ with $0$ and applying Lemma \ref{CanColorAndPlayOnRest}.

\noindent\textbf{Case 138.  }\textit{$L(v_1) = \{0, 1\}$, $L(v_2) = \{0, 2, 3\}$, $L(v_3) = \{0, 1, 4\}$, $L(v_4) = \{0, 1\}$ and $L(v_5) = \{0, 4\}$.}

Fixer gets a winning strategy by coloring $v_5v_3$ with $4$ and applying Lemma \ref{CanColorAndPlayOnRest}.

\noindent\textbf{Case 139.  }\textit{$L(v_1) = \{0, 1\}$, $L(v_2) = \{0, 2, 3\}$, $L(v_3) = \{0, 1, 4\}$, $L(v_4) = \{0, 1\}$ and $L(v_5) = \{1, 4\}$.}

Fixer gets a winning strategy by coloring $v_5v_3$ with $4$ and applying Lemma \ref{CanColorAndPlayOnRest}.

\noindent\textbf{Case 140.  }\textit{$L(v_1) = \{0, 1\}$, $L(v_2) = \{0, 2, 3\}$, $L(v_3) = \{0, 1, 4\}$, $L(v_4) = \{0, 1\}$ and $L(v_5) = \{2, 4\}$.}

Fixer gets a winning strategy by coloring $v_5v_3$ with $4$ and applying Lemma \ref{CanColorAndPlayOnRest}.

\noindent\textbf{Case 141.  }\textit{$L(v_1) = \{0, 1\}$, $L(v_2) = \{0, 2, 3\}$, $L(v_3) = \{0, 1, 4\}$, $L(v_4) = \{0, 2\}$ and $L(v_5) = \{0, 4\}$.}

Fixer gets a winning strategy by coloring $v_5v_3$ with $4$ and applying Lemma \ref{CanColorAndPlayOnRest}.

\noindent\textbf{Case 142.  }\textit{$L(v_1) = \{0, 1\}$, $L(v_2) = \{0, 2, 3\}$, $L(v_3) = \{0, 1, 4\}$, $L(v_4) = \{0, 2\}$ and $L(v_5) = \{1, 4\}$.}

Fixer gets a winning strategy by coloring $v_5v_3$ with $4$ and applying Lemma \ref{CanColorAndPlayOnRest}.

\noindent\textbf{Case 143.  }\textit{$L(v_1) = \{0, 1\}$, $L(v_2) = \{0, 2, 3\}$, $L(v_3) = \{0, 1, 4\}$, $L(v_4) = \{0, 2\}$ and $L(v_5) = \{2, 4\}$.}

Fixer gets a winning strategy by coloring $v_5v_3$ with $4$ and applying Lemma \ref{CanColorAndPlayOnRest}.

\noindent\textbf{Case 144.  }\textit{$L(v_1) = \{0, 1\}$, $L(v_2) = \{0, 2, 3\}$, $L(v_3) = \{0, 1, 4\}$, $L(v_4) = \{0, 2\}$ and $L(v_5) = \{3, 4\}$.}

Fixer gets a winning strategy by coloring $v_5v_3$ with $4$ and applying Lemma \ref{CanColorAndPlayOnRest}.

\noindent\textbf{Case 145.  }\textit{$L(v_1) = \{0, 1\}$, $L(v_2) = \{0, 2, 3\}$, $L(v_3) = \{0, 1, 4\}$, $L(v_4) = \{1, 2\}$ and $L(v_5) = \{0, 4\}$.}

Fixer gets a winning strategy by coloring $v_5v_3$ with $4$ and applying Lemma \ref{CanColorAndPlayOnRest}.

\noindent\textbf{Case 146.  }\textit{$L(v_1) = \{0, 1\}$, $L(v_2) = \{0, 2, 3\}$, $L(v_3) = \{0, 1, 4\}$, $L(v_4) = \{1, 2\}$ and $L(v_5) = \{1, 4\}$.}

Fixer gets a winning strategy by coloring $v_5v_3$ with $4$ and applying Lemma \ref{CanColorAndPlayOnRest}.

\noindent\textbf{Case 147.  }\textit{$L(v_1) = \{0, 1\}$, $L(v_2) = \{0, 2, 3\}$, $L(v_3) = \{0, 1, 4\}$, $L(v_4) = \{1, 2\}$ and $L(v_5) = \{2, 4\}$.}

Fixer gets a winning strategy by coloring $v_5v_3$ with $4$ and applying Lemma \ref{CanColorAndPlayOnRest}.

\noindent\textbf{Case 148.  }\textit{$L(v_1) = \{0, 1\}$, $L(v_2) = \{0, 2, 3\}$, $L(v_3) = \{0, 1, 4\}$, $L(v_4) = \{1, 2\}$ and $L(v_5) = \{3, 4\}$.}

Fixer gets a winning strategy by coloring $v_5v_3$ with $4$ and applying Lemma \ref{CanColorAndPlayOnRest}.

\noindent\textbf{Case 149.  }\textit{$L(v_1) = \{0, 1\}$, $L(v_2) = \{0, 2, 3\}$, $L(v_3) = \{0, 1, 4\}$, $L(v_4) = \{0, 4\}$ and $L(v_5) = \{0, 1\}$.}

Fixer gets a winning strategy by coloring $v_4v_3$ with $4$ and applying Lemma \ref{CanColorAndPlayOnRest}.

\noindent\textbf{Case 150.  }\textit{$L(v_1) = \{0, 1\}$, $L(v_2) = \{0, 2, 3\}$, $L(v_3) = \{0, 1, 4\}$, $L(v_4) = \{0, 4\}$ and $L(v_5) = \{0, 2\}$.}

Fixer gets a winning strategy by coloring $v_4v_3$ with $4$ and applying Lemma \ref{CanColorAndPlayOnRest}.

\noindent\textbf{Case 151.  }\textit{$L(v_1) = \{0, 1\}$, $L(v_2) = \{0, 2, 3\}$, $L(v_3) = \{0, 1, 4\}$, $L(v_4) = \{0, 4\}$ and $L(v_5) = \{1, 2\}$.}

Fixer gets a winning strategy by coloring $v_4v_3$ with $4$ and applying Lemma \ref{CanColorAndPlayOnRest}.

\noindent\textbf{Case 152.  }\textit{$L(v_1) = \{0, 1\}$, $L(v_2) = \{0, 2, 3\}$, $L(v_3) = \{0, 1, 4\}$, $L(v_4) = \{1, 4\}$ and $L(v_5) = \{0, 1\}$.}

Fixer gets a winning strategy by coloring $v_4v_3$ with $4$ and applying Lemma \ref{CanColorAndPlayOnRest}.

\noindent\textbf{Case 153.  }\textit{$L(v_1) = \{0, 1\}$, $L(v_2) = \{0, 2, 3\}$, $L(v_3) = \{0, 1, 4\}$, $L(v_4) = \{1, 4\}$ and $L(v_5) = \{0, 2\}$.}

Fixer gets a winning strategy by coloring $v_4v_3$ with $4$ and applying Lemma \ref{CanColorAndPlayOnRest}.

\noindent\textbf{Case 154.  }\textit{$L(v_1) = \{0, 1\}$, $L(v_2) = \{0, 2, 3\}$, $L(v_3) = \{0, 1, 4\}$, $L(v_4) = \{1, 4\}$ and $L(v_5) = \{1, 2\}$.}

Fixer gets a winning strategy by coloring $v_4v_3$ with $4$ and applying Lemma \ref{CanColorAndPlayOnRest}.

\noindent\textbf{Case 155.  }\textit{$L(v_1) = \{0, 1\}$, $L(v_2) = \{0, 2, 3\}$, $L(v_3) = \{0, 1, 4\}$, $L(v_4) = \{2, 4\}$ and $L(v_5) = \{0, 1\}$.}

Fixer gets a winning strategy by coloring $v_4v_3$ with $4$ and applying Lemma \ref{CanColorAndPlayOnRest}.

\noindent\textbf{Case 156.  }\textit{$L(v_1) = \{0, 1\}$, $L(v_2) = \{0, 2, 3\}$, $L(v_3) = \{0, 1, 4\}$, $L(v_4) = \{2, 4\}$ and $L(v_5) = \{0, 2\}$.}

Fixer gets a winning strategy by coloring $v_4v_3$ with $4$ and applying Lemma \ref{CanColorAndPlayOnRest}.

\noindent\textbf{Case 157.  }\textit{$L(v_1) = \{0, 1\}$, $L(v_2) = \{0, 2, 3\}$, $L(v_3) = \{0, 1, 4\}$, $L(v_4) = \{2, 4\}$ and $L(v_5) = \{1, 2\}$.}

Fixer gets a winning strategy by coloring $v_4v_3$ with $4$ and applying Lemma \ref{CanColorAndPlayOnRest}.

\noindent\textbf{Case 158.  }\textit{$L(v_1) = \{0, 1\}$, $L(v_2) = \{0, 2, 3\}$, $L(v_3) = \{0, 1, 4\}$, $L(v_4) = \{2, 4\}$ and $L(v_5) = \{0, 3\}$.}

Fixer gets a winning strategy by coloring $v_4v_3$ with $4$ and applying Lemma \ref{CanColorAndPlayOnRest}.

\noindent\textbf{Case 159.  }\textit{$L(v_1) = \{0, 1\}$, $L(v_2) = \{0, 2, 3\}$, $L(v_3) = \{0, 1, 4\}$, $L(v_4) = \{2, 4\}$ and $L(v_5) = \{1, 3\}$.}

Fixer gets a winning strategy by coloring $v_4v_3$ with $4$ and applying Lemma \ref{CanColorAndPlayOnRest}.

\noindent\textbf{Case 160.  }\textit{$L(v_1) = \{0, 1\}$, $L(v_2) = \{0, 2, 3\}$, $L(v_3) = \{0, 2, 4\}$, $L(v_4) = \{0, 1\}$ and $L(v_5) = \{0, 1\}$.}

Fixer gets a winning strategy by coloring $v_1v_2$ with $0$ and applying Lemma \ref{CanColorAndPlayOnRest}.

\noindent\textbf{Case 161.  }\textit{$L(v_1) = \{0, 1\}$, $L(v_2) = \{0, 2, 3\}$, $L(v_3) = \{0, 2, 4\}$, $L(v_4) = \{0, 1\}$ and $L(v_5) = \{2, 3\}$.}

Fixer gets a winning strategy by coloring $v_1v_2$ with $0$ and applying Lemma \ref{CanColorAndPlayOnRest}.

\noindent\textbf{Case 162.  }\textit{$L(v_1) = \{0, 1\}$, $L(v_2) = \{0, 2, 3\}$, $L(v_3) = \{0, 2, 4\}$, $L(v_4) = \{0, 2\}$ and $L(v_5) = \{0, 2\}$.}

Fixer gets a winning strategy by coloring $v_1v_2$ with $0$ and applying Lemma \ref{CanColorAndPlayOnRest}.

\noindent\textbf{Case 163.  }\textit{$L(v_1) = \{0, 1\}$, $L(v_2) = \{0, 2, 3\}$, $L(v_3) = \{0, 2, 4\}$, $L(v_4) = \{0, 2\}$ and $L(v_5) = \{0, 3\}$.}

Fixer gets a winning strategy by coloring $v_1v_2$ with $0$ and applying Lemma \ref{CanColorAndPlayOnRest}.

\noindent\textbf{Case 164.  }\textit{$L(v_1) = \{0, 1\}$, $L(v_2) = \{0, 2, 3\}$, $L(v_3) = \{0, 2, 4\}$, $L(v_4) = \{0, 2\}$ and $L(v_5) = \{2, 3\}$.}

Fixer gets a winning strategy by coloring $v_1v_2$ with $0$ and applying Lemma \ref{CanColorAndPlayOnRest}.

\noindent\textbf{Case 165.  }\textit{$L(v_1) = \{0, 1\}$, $L(v_2) = \{0, 2, 3\}$, $L(v_3) = \{0, 2, 4\}$, $L(v_4) = \{1, 2\}$ and $L(v_5) = \{0, 4\}$.}

Fixer gets a winning strategy by coloring $v_5v_3$ with $4$ and applying Lemma \ref{CanColorAndPlayOnRest}.

\noindent\textbf{Case 166.  }\textit{$L(v_1) = \{0, 1\}$, $L(v_2) = \{0, 2, 3\}$, $L(v_3) = \{0, 2, 4\}$, $L(v_4) = \{1, 2\}$ and $L(v_5) = \{1, 4\}$.}

Fixer gets a winning strategy by coloring $v_5v_3$ with $4$ and applying Lemma \ref{CanColorAndPlayOnRest}.

\noindent\textbf{Case 167.  }\textit{$L(v_1) = \{0, 1\}$, $L(v_2) = \{0, 2, 3\}$, $L(v_3) = \{0, 2, 4\}$, $L(v_4) = \{1, 2\}$ and $L(v_5) = \{2, 4\}$.}

Fixer gets a winning strategy by coloring $v_5v_3$ with $4$ and applying Lemma \ref{CanColorAndPlayOnRest}.

\noindent\textbf{Case 168.  }\textit{$L(v_1) = \{0, 1\}$, $L(v_2) = \{0, 2, 3\}$, $L(v_3) = \{0, 2, 4\}$, $L(v_4) = \{1, 2\}$ and $L(v_5) = \{3, 4\}$.}

Fixer gets a winning strategy by coloring $v_5v_3$ with $4$ and applying Lemma \ref{CanColorAndPlayOnRest}.

\noindent\textbf{Case 169.  }\textit{$L(v_1) = \{0, 1\}$, $L(v_2) = \{0, 2, 3\}$, $L(v_3) = \{0, 2, 4\}$, $L(v_4) = \{0, 3\}$ and $L(v_5) = \{0, 2\}$.}

Fixer gets a winning strategy by coloring $v_1v_2$ with $0$ and applying Lemma \ref{CanColorAndPlayOnRest}.

\noindent\textbf{Case 170.  }\textit{$L(v_1) = \{0, 1\}$, $L(v_2) = \{0, 2, 3\}$, $L(v_3) = \{0, 2, 4\}$, $L(v_4) = \{0, 3\}$ and $L(v_5) = \{0, 3\}$.}

Fixer gets a winning strategy by coloring $v_1v_2$ with $0$ and applying Lemma \ref{CanColorAndPlayOnRest}.

\noindent\textbf{Case 171.  }\textit{$L(v_1) = \{0, 1\}$, $L(v_2) = \{0, 2, 3\}$, $L(v_3) = \{0, 2, 4\}$, $L(v_4) = \{0, 3\}$ and $L(v_5) = \{2, 3\}$.}

Fixer gets a winning strategy by coloring $v_1v_2$ with $0$ and applying Lemma \ref{CanColorAndPlayOnRest}.

\noindent\textbf{Case 172.  }\textit{$L(v_1) = \{0, 1\}$, $L(v_2) = \{0, 2, 3\}$, $L(v_3) = \{0, 2, 4\}$, $L(v_4) = \{2, 3\}$ and $L(v_5) = \{0, 1\}$.}

Fixer gets a winning strategy by coloring $v_1v_2$ with $0$ and applying Lemma \ref{CanColorAndPlayOnRest}.

\noindent\textbf{Case 173.  }\textit{$L(v_1) = \{0, 1\}$, $L(v_2) = \{0, 2, 3\}$, $L(v_3) = \{0, 2, 4\}$, $L(v_4) = \{2, 3\}$ and $L(v_5) = \{0, 2\}$.}

Fixer gets a winning strategy by coloring $v_1v_2$ with $0$ and applying Lemma \ref{CanColorAndPlayOnRest}.

\noindent\textbf{Case 174.  }\textit{$L(v_1) = \{0, 1\}$, $L(v_2) = \{0, 2, 3\}$, $L(v_3) = \{0, 2, 4\}$, $L(v_4) = \{2, 3\}$ and $L(v_5) = \{0, 3\}$.}

Fixer gets a winning strategy by coloring $v_1v_2$ with $0$ and applying Lemma \ref{CanColorAndPlayOnRest}.

\noindent\textbf{Case 175.  }\textit{$L(v_1) = \{0, 1\}$, $L(v_2) = \{0, 2, 3\}$, $L(v_3) = \{0, 2, 4\}$, $L(v_4) = \{2, 3\}$ and $L(v_5) = \{2, 3\}$.}

Fixer gets a winning strategy by coloring $v_1v_2$ with $0$ and applying Lemma \ref{CanColorAndPlayOnRest}.

\noindent\textbf{Case 176.  }\textit{$L(v_1) = \{0, 1\}$, $L(v_2) = \{0, 2, 3\}$, $L(v_3) = \{0, 2, 4\}$, $L(v_4) = \{2, 3\}$ and $L(v_5) = \{0, 4\}$.}

Fixer gets a winning strategy by coloring $v_1v_2$ with $0$ and applying Lemma \ref{CanColorAndPlayOnRest}.

\noindent\textbf{Case 177.  }\textit{$L(v_1) = \{0, 1\}$, $L(v_2) = \{0, 2, 3\}$, $L(v_3) = \{0, 2, 4\}$, $L(v_4) = \{2, 3\}$ and $L(v_5) = \{1, 4\}$.}

Fixer gets a winning strategy by coloring $v_1v_2$ with $0$ and applying Lemma \ref{CanColorAndPlayOnRest}.

\noindent\textbf{Case 178.  }\textit{$L(v_1) = \{0, 1\}$, $L(v_2) = \{0, 2, 3\}$, $L(v_3) = \{0, 2, 4\}$, $L(v_4) = \{2, 3\}$ and $L(v_5) = \{2, 4\}$.}

Fixer gets a winning strategy by coloring $v_1v_2$ with $0$ and applying Lemma \ref{CanColorAndPlayOnRest}.

\noindent\textbf{Case 179.  }\textit{$L(v_1) = \{0, 1\}$, $L(v_2) = \{0, 2, 3\}$, $L(v_3) = \{0, 2, 4\}$, $L(v_4) = \{2, 3\}$ and $L(v_5) = \{3, 4\}$.}

Fixer gets a winning strategy by coloring $v_1v_2$ with $0$ and applying Lemma \ref{CanColorAndPlayOnRest}.

\noindent\textbf{Case 180.  }\textit{$L(v_1) = \{0, 1\}$, $L(v_2) = \{0, 2, 3\}$, $L(v_3) = \{0, 2, 4\}$, $L(v_4) = \{0, 4\}$ and $L(v_5) = \{1, 2\}$.}

Fixer gets a winning strategy by coloring $v_4v_3$ with $4$ and applying Lemma \ref{CanColorAndPlayOnRest}.

\noindent\textbf{Case 181.  }\textit{$L(v_1) = \{0, 1\}$, $L(v_2) = \{0, 2, 3\}$, $L(v_3) = \{0, 2, 4\}$, $L(v_4) = \{0, 4\}$ and $L(v_5) = \{2, 3\}$.}

Fixer gets a winning strategy by coloring $v_1v_2$ with $0$ and applying Lemma \ref{CanColorAndPlayOnRest}.

\noindent\textbf{Case 182.  }\textit{$L(v_1) = \{0, 1\}$, $L(v_2) = \{0, 2, 3\}$, $L(v_3) = \{0, 2, 4\}$, $L(v_4) = \{1, 4\}$ and $L(v_5) = \{1, 2\}$.}

Fixer gets a winning strategy by coloring $v_4v_3$ with $4$ and applying Lemma \ref{CanColorAndPlayOnRest}.

\noindent\textbf{Case 183.  }\textit{$L(v_1) = \{0, 1\}$, $L(v_2) = \{0, 2, 3\}$, $L(v_3) = \{0, 2, 4\}$, $L(v_4) = \{1, 4\}$ and $L(v_5) = \{2, 3\}$.}

Fixer gets a winning strategy by coloring $v_1v_2$ with $0$ and applying Lemma \ref{CanColorAndPlayOnRest}.

\noindent\textbf{Case 184.  }\textit{$L(v_1) = \{0, 1\}$, $L(v_2) = \{0, 2, 3\}$, $L(v_3) = \{0, 2, 4\}$, $L(v_4) = \{1, 4\}$ and $L(v_5) = \{1, 4\}$.}

Fixer gets a winning strategy by coloring $v_1v_2$ with $0$ and applying Lemma \ref{CanColorAndPlayOnRest}.

\noindent\textbf{Case 185.  }\textit{$L(v_1) = \{0, 1\}$, $L(v_2) = \{0, 2, 3\}$, $L(v_3) = \{0, 2, 4\}$, $L(v_4) = \{2, 4\}$ and $L(v_5) = \{1, 2\}$.}

Fixer gets a winning strategy by coloring $v_4v_3$ with $4$ and applying Lemma \ref{CanColorAndPlayOnRest}.

\noindent\textbf{Case 186.  }\textit{$L(v_1) = \{0, 1\}$, $L(v_2) = \{0, 2, 3\}$, $L(v_3) = \{0, 2, 4\}$, $L(v_4) = \{2, 4\}$ and $L(v_5) = \{2, 3\}$.}

Fixer gets a winning strategy by coloring $v_1v_2$ with $0$ and applying Lemma \ref{CanColorAndPlayOnRest}.

\noindent\textbf{Case 187.  }\textit{$L(v_1) = \{0, 1\}$, $L(v_2) = \{0, 2, 3\}$, $L(v_3) = \{0, 2, 4\}$, $L(v_4) = \{2, 4\}$ and $L(v_5) = \{2, 4\}$.}

Fixer gets a winning strategy by coloring $v_1v_2$ with $0$ and applying Lemma \ref{CanColorAndPlayOnRest}.

\noindent\textbf{Case 188.  }\textit{$L(v_1) = \{0, 1\}$, $L(v_2) = \{0, 2, 3\}$, $L(v_3) = \{0, 2, 4\}$, $L(v_4) = \{2, 4\}$ and $L(v_5) = \{3, 4\}$.}

Fixer gets a winning strategy by coloring $v_1v_2$ with $0$ and applying Lemma \ref{CanColorAndPlayOnRest}.

\noindent\textbf{Case 189.  }\textit{$L(v_1) = \{0, 1\}$, $L(v_2) = \{0, 2, 3\}$, $L(v_3) = \{0, 2, 4\}$, $L(v_4) = \{3, 4\}$ and $L(v_5) = \{1, 2\}$.}

Fixer gets a winning strategy by coloring $v_4v_3$ with $4$ and applying Lemma \ref{CanColorAndPlayOnRest}.

\noindent\textbf{Case 190.  }\textit{$L(v_1) = \{0, 1\}$, $L(v_2) = \{0, 2, 3\}$, $L(v_3) = \{0, 2, 4\}$, $L(v_4) = \{3, 4\}$ and $L(v_5) = \{2, 3\}$.}

Fixer gets a winning strategy by coloring $v_1v_2$ with $0$ and applying Lemma \ref{CanColorAndPlayOnRest}.

\noindent\textbf{Case 191.  }\textit{$L(v_1) = \{0, 1\}$, $L(v_2) = \{0, 2, 3\}$, $L(v_3) = \{0, 2, 4\}$, $L(v_4) = \{3, 4\}$ and $L(v_5) = \{2, 4\}$.}

Fixer gets a winning strategy by coloring $v_1v_2$ with $0$ and applying Lemma \ref{CanColorAndPlayOnRest}.

\noindent\textbf{Case 192.  }\textit{$L(v_1) = \{0, 1\}$, $L(v_2) = \{0, 2, 3\}$, $L(v_3) = \{0, 2, 4\}$, $L(v_4) = \{3, 4\}$ and $L(v_5) = \{3, 4\}$.}

Fixer gets a winning strategy by coloring $v_1v_2$ with $0$ and applying Lemma \ref{CanColorAndPlayOnRest}.

\noindent\textbf{Case 193.  }\textit{$L(v_1) = \{0, 1\}$, $L(v_2) = \{0, 2, 3\}$, $L(v_3) = \{1, 2, 4\}$, $L(v_4) = \{0, 1\}$ and $L(v_5) = \{0, 1\}$.}

Fixer gets a winning strategy by coloring $v_1v_2$ with $0$ and applying Lemma \ref{CanColorAndPlayOnRest}.

\noindent\textbf{Case 194.  }\textit{$L(v_1) = \{0, 1\}$, $L(v_2) = \{0, 2, 3\}$, $L(v_3) = \{1, 2, 4\}$, $L(v_4) = \{0, 1\}$ and $L(v_5) = \{2, 3\}$.}

Fixer gets a winning strategy by coloring $v_1v_2$ with $0$ and applying Lemma \ref{CanColorAndPlayOnRest}.

\noindent\textbf{Case 195.  }\textit{$L(v_1) = \{0, 1\}$, $L(v_2) = \{0, 2, 3\}$, $L(v_3) = \{1, 2, 4\}$, $L(v_4) = \{0, 2\}$ and $L(v_5) = \{0, 4\}$.}

Fixer gets a winning strategy by coloring $v_5v_3$ with $4$ and applying Lemma \ref{CanColorAndPlayOnRest}.

\noindent\textbf{Case 196.  }\textit{$L(v_1) = \{0, 1\}$, $L(v_2) = \{0, 2, 3\}$, $L(v_3) = \{1, 2, 4\}$, $L(v_4) = \{0, 2\}$ and $L(v_5) = \{1, 4\}$.}

Fixer gets a winning strategy by coloring $v_5v_3$ with $4$ and applying Lemma \ref{CanColorAndPlayOnRest}.

\noindent\textbf{Case 197.  }\textit{$L(v_1) = \{0, 1\}$, $L(v_2) = \{0, 2, 3\}$, $L(v_3) = \{1, 2, 4\}$, $L(v_4) = \{0, 2\}$ and $L(v_5) = \{2, 4\}$.}

Fixer gets a winning strategy by coloring $v_5v_3$ with $4$ and applying Lemma \ref{CanColorAndPlayOnRest}.

\noindent\textbf{Case 198.  }\textit{$L(v_1) = \{0, 1\}$, $L(v_2) = \{0, 2, 3\}$, $L(v_3) = \{1, 2, 4\}$, $L(v_4) = \{0, 2\}$ and $L(v_5) = \{3, 4\}$.}

Fixer gets a winning strategy by coloring $v_5v_3$ with $4$ and applying Lemma \ref{CanColorAndPlayOnRest}.

\noindent\textbf{Case 199.  }\textit{$L(v_1) = \{0, 1\}$, $L(v_2) = \{0, 2, 3\}$, $L(v_3) = \{1, 2, 4\}$, $L(v_4) = \{1, 2\}$ and $L(v_5) = \{1, 2\}$.}

Fixer gets a winning strategy by coloring $v_1v_2$ with $0$ and applying Lemma \ref{CanColorAndPlayOnRest}.

\noindent\textbf{Case 200.  }\textit{$L(v_1) = \{0, 1\}$, $L(v_2) = \{0, 2, 3\}$, $L(v_3) = \{1, 2, 4\}$, $L(v_4) = \{1, 2\}$ and $L(v_5) = \{1, 3\}$.}

Fixer gets a winning strategy by coloring $v_1v_2$ with $0$ and applying Lemma \ref{CanColorAndPlayOnRest}.

\noindent\textbf{Case 201.  }\textit{$L(v_1) = \{0, 1\}$, $L(v_2) = \{0, 2, 3\}$, $L(v_3) = \{1, 2, 4\}$, $L(v_4) = \{1, 2\}$ and $L(v_5) = \{2, 3\}$.}

Fixer gets a winning strategy by coloring $v_1v_2$ with $0$ and applying Lemma \ref{CanColorAndPlayOnRest}.

\noindent\textbf{Case 202.  }\textit{$L(v_1) = \{0, 1\}$, $L(v_2) = \{0, 2, 3\}$, $L(v_3) = \{1, 2, 4\}$, $L(v_4) = \{1, 3\}$ and $L(v_5) = \{1, 2\}$.}

Fixer gets a winning strategy by coloring $v_1v_2$ with $0$ and applying Lemma \ref{CanColorAndPlayOnRest}.

\noindent\textbf{Case 203.  }\textit{$L(v_1) = \{0, 1\}$, $L(v_2) = \{0, 2, 3\}$, $L(v_3) = \{1, 2, 4\}$, $L(v_4) = \{1, 3\}$ and $L(v_5) = \{1, 3\}$.}

Fixer gets a winning strategy by coloring $v_1v_2$ with $0$ and applying Lemma \ref{CanColorAndPlayOnRest}.

\noindent\textbf{Case 204.  }\textit{$L(v_1) = \{0, 1\}$, $L(v_2) = \{0, 2, 3\}$, $L(v_3) = \{1, 2, 4\}$, $L(v_4) = \{1, 3\}$ and $L(v_5) = \{2, 3\}$.}

Fixer gets a winning strategy by coloring $v_1v_2$ with $0$ and applying Lemma \ref{CanColorAndPlayOnRest}.

\noindent\textbf{Case 205.  }\textit{$L(v_1) = \{0, 1\}$, $L(v_2) = \{0, 2, 3\}$, $L(v_3) = \{1, 2, 4\}$, $L(v_4) = \{2, 3\}$ and $L(v_5) = \{0, 1\}$.}

Fixer gets a winning strategy by coloring $v_1v_2$ with $0$ and applying Lemma \ref{CanColorAndPlayOnRest}.

\noindent\textbf{Case 206.  }\textit{$L(v_1) = \{0, 1\}$, $L(v_2) = \{0, 2, 3\}$, $L(v_3) = \{1, 2, 4\}$, $L(v_4) = \{2, 3\}$ and $L(v_5) = \{1, 2\}$.}

Fixer gets a winning strategy by coloring $v_1v_2$ with $0$ and applying Lemma \ref{CanColorAndPlayOnRest}.

\noindent\textbf{Case 207.  }\textit{$L(v_1) = \{0, 1\}$, $L(v_2) = \{0, 2, 3\}$, $L(v_3) = \{1, 2, 4\}$, $L(v_4) = \{2, 3\}$ and $L(v_5) = \{1, 3\}$.}

Fixer gets a winning strategy by coloring $v_1v_2$ with $0$ and applying Lemma \ref{CanColorAndPlayOnRest}.

\noindent\textbf{Case 208.  }\textit{$L(v_1) = \{0, 1\}$, $L(v_2) = \{0, 2, 3\}$, $L(v_3) = \{1, 2, 4\}$, $L(v_4) = \{2, 3\}$ and $L(v_5) = \{2, 3\}$.}

Fixer gets a winning strategy by coloring $v_1v_2$ with $0$ and applying Lemma \ref{CanColorAndPlayOnRest}.

\noindent\textbf{Case 209.  }\textit{$L(v_1) = \{0, 1\}$, $L(v_2) = \{0, 2, 3\}$, $L(v_3) = \{1, 2, 4\}$, $L(v_4) = \{2, 3\}$ and $L(v_5) = \{0, 4\}$.}

Fixer gets a winning strategy by coloring $v_1v_2$ with $0$ and applying Lemma \ref{CanColorAndPlayOnRest}.

\noindent\textbf{Case 210.  }\textit{$L(v_1) = \{0, 1\}$, $L(v_2) = \{0, 2, 3\}$, $L(v_3) = \{1, 2, 4\}$, $L(v_4) = \{2, 3\}$ and $L(v_5) = \{1, 4\}$.}

Fixer gets a winning strategy by coloring $v_1v_2$ with $0$ and applying Lemma \ref{CanColorAndPlayOnRest}.

\noindent\textbf{Case 211.  }\textit{$L(v_1) = \{0, 1\}$, $L(v_2) = \{0, 2, 3\}$, $L(v_3) = \{1, 2, 4\}$, $L(v_4) = \{2, 3\}$ and $L(v_5) = \{2, 4\}$.}

Fixer gets a winning strategy by coloring $v_1v_2$ with $0$ and applying Lemma \ref{CanColorAndPlayOnRest}.

\noindent\textbf{Case 212.  }\textit{$L(v_1) = \{0, 1\}$, $L(v_2) = \{0, 2, 3\}$, $L(v_3) = \{1, 2, 4\}$, $L(v_4) = \{2, 3\}$ and $L(v_5) = \{3, 4\}$.}

Fixer gets a winning strategy by coloring $v_1v_2$ with $0$ and applying Lemma \ref{CanColorAndPlayOnRest}.

\noindent\textbf{Case 213.  }\textit{$L(v_1) = \{0, 1\}$, $L(v_2) = \{0, 2, 3\}$, $L(v_3) = \{1, 2, 4\}$, $L(v_4) = \{0, 4\}$ and $L(v_5) = \{0, 2\}$.}

Fixer gets a winning strategy by coloring $v_4v_3$ with $4$ and applying Lemma \ref{CanColorAndPlayOnRest}.

\noindent\textbf{Case 214.  }\textit{$L(v_1) = \{0, 1\}$, $L(v_2) = \{0, 2, 3\}$, $L(v_3) = \{1, 2, 4\}$, $L(v_4) = \{0, 4\}$ and $L(v_5) = \{2, 3\}$.}

Fixer gets a winning strategy by coloring $v_1v_2$ with $0$ and applying Lemma \ref{CanColorAndPlayOnRest}.

\noindent\textbf{Case 215.  }\textit{$L(v_1) = \{0, 1\}$, $L(v_2) = \{0, 2, 3\}$, $L(v_3) = \{1, 2, 4\}$, $L(v_4) = \{0, 4\}$ and $L(v_5) = \{0, 4\}$.}

Fixer gets a winning strategy by coloring $v_1v_2$ with $0$ and applying Lemma \ref{CanColorAndPlayOnRest}.

\noindent\textbf{Case 216.  }\textit{$L(v_1) = \{0, 1\}$, $L(v_2) = \{0, 2, 3\}$, $L(v_3) = \{1, 2, 4\}$, $L(v_4) = \{1, 4\}$ and $L(v_5) = \{0, 2\}$.}

Fixer gets a winning strategy by coloring $v_4v_3$ with $4$ and applying Lemma \ref{CanColorAndPlayOnRest}.

\noindent\textbf{Case 217.  }\textit{$L(v_1) = \{0, 1\}$, $L(v_2) = \{0, 2, 3\}$, $L(v_3) = \{1, 2, 4\}$, $L(v_4) = \{1, 4\}$ and $L(v_5) = \{2, 3\}$.}

Fixer gets a winning strategy by coloring $v_1v_2$ with $0$ and applying Lemma \ref{CanColorAndPlayOnRest}.

\noindent\textbf{Case 218.  }\textit{$L(v_1) = \{0, 1\}$, $L(v_2) = \{0, 2, 3\}$, $L(v_3) = \{1, 2, 4\}$, $L(v_4) = \{2, 4\}$ and $L(v_5) = \{0, 2\}$.}

Fixer gets a winning strategy by coloring $v_4v_3$ with $4$ and applying Lemma \ref{CanColorAndPlayOnRest}.

\noindent\textbf{Case 219.  }\textit{$L(v_1) = \{0, 1\}$, $L(v_2) = \{0, 2, 3\}$, $L(v_3) = \{1, 2, 4\}$, $L(v_4) = \{2, 4\}$ and $L(v_5) = \{2, 3\}$.}

Fixer gets a winning strategy by coloring $v_1v_2$ with $0$ and applying Lemma \ref{CanColorAndPlayOnRest}.

\noindent\textbf{Case 220.  }\textit{$L(v_1) = \{0, 1\}$, $L(v_2) = \{0, 2, 3\}$, $L(v_3) = \{1, 2, 4\}$, $L(v_4) = \{2, 4\}$ and $L(v_5) = \{2, 4\}$.}

Fixer gets a winning strategy by coloring $v_1v_2$ with $0$ and applying Lemma \ref{CanColorAndPlayOnRest}.

\noindent\textbf{Case 221.  }\textit{$L(v_1) = \{0, 1\}$, $L(v_2) = \{0, 2, 3\}$, $L(v_3) = \{1, 2, 4\}$, $L(v_4) = \{2, 4\}$ and $L(v_5) = \{3, 4\}$.}

Fixer gets a winning strategy by coloring $v_1v_2$ with $0$ and applying Lemma \ref{CanColorAndPlayOnRest}.

\noindent\textbf{Case 222.  }\textit{$L(v_1) = \{0, 1\}$, $L(v_2) = \{0, 2, 3\}$, $L(v_3) = \{1, 2, 4\}$, $L(v_4) = \{3, 4\}$ and $L(v_5) = \{0, 2\}$.}

Fixer gets a winning strategy by coloring $v_4v_3$ with $4$ and applying Lemma \ref{CanColorAndPlayOnRest}.

\noindent\textbf{Case 223.  }\textit{$L(v_1) = \{0, 1\}$, $L(v_2) = \{0, 2, 3\}$, $L(v_3) = \{1, 2, 4\}$, $L(v_4) = \{3, 4\}$ and $L(v_5) = \{2, 3\}$.}

Fixer gets a winning strategy by coloring $v_1v_2$ with $0$ and applying Lemma \ref{CanColorAndPlayOnRest}.

\noindent\textbf{Case 224.  }\textit{$L(v_1) = \{0, 1\}$, $L(v_2) = \{0, 2, 3\}$, $L(v_3) = \{1, 2, 4\}$, $L(v_4) = \{3, 4\}$ and $L(v_5) = \{2, 4\}$.}

Fixer gets a winning strategy by coloring $v_1v_2$ with $0$ and applying Lemma \ref{CanColorAndPlayOnRest}.

\noindent\textbf{Case 225.  }\textit{$L(v_1) = \{0, 1\}$, $L(v_2) = \{0, 2, 3\}$, $L(v_3) = \{1, 2, 4\}$, $L(v_4) = \{3, 4\}$ and $L(v_5) = \{3, 4\}$.}

Fixer gets a winning strategy by coloring $v_1v_2$ with $0$ and applying Lemma \ref{CanColorAndPlayOnRest}.

\noindent\textbf{Case 226.  }\textit{$L(v_1) = \{0, 1\}$, $L(v_2) = \{0, 2, 3\}$, $L(v_3) = \{2, 3, 4\}$, $L(v_4) = \{0, 2\}$ and $L(v_5) = \{0, 2\}$.}

Fixer gets a winning strategy by coloring $v_1v_2$ with $0$ and applying Lemma \ref{CanColorAndPlayOnRest}.

\noindent\textbf{Case 227.  }\textit{$L(v_1) = \{0, 1\}$, $L(v_2) = \{0, 2, 3\}$, $L(v_3) = \{2, 3, 4\}$, $L(v_4) = \{0, 2\}$ and $L(v_5) = \{0, 3\}$.}

Fixer gets a winning strategy by coloring $v_1v_2$ with $0$ and applying Lemma \ref{CanColorAndPlayOnRest}.

\noindent\textbf{Case 228.  }\textit{$L(v_1) = \{0, 1\}$, $L(v_2) = \{0, 2, 3\}$, $L(v_3) = \{2, 3, 4\}$, $L(v_4) = \{0, 2\}$ and $L(v_5) = \{2, 3\}$.}

Fixer gets a winning strategy by coloring $v_1v_2$ with $0$ and applying Lemma \ref{CanColorAndPlayOnRest}.

\noindent\textbf{Case 229.  }\textit{$L(v_1) = \{0, 1\}$, $L(v_2) = \{0, 2, 3\}$, $L(v_3) = \{2, 3, 4\}$, $L(v_4) = \{1, 2\}$ and $L(v_5) = \{1, 2\}$.}

Fixer gets a winning strategy by coloring $v_1v_2$ with $0$ and applying Lemma \ref{CanColorAndPlayOnRest}.

\noindent\textbf{Case 230.  }\textit{$L(v_1) = \{0, 1\}$, $L(v_2) = \{0, 2, 3\}$, $L(v_3) = \{2, 3, 4\}$, $L(v_4) = \{1, 2\}$ and $L(v_5) = \{1, 3\}$.}

Fixer gets a winning strategy by coloring $v_1v_2$ with $0$ and applying Lemma \ref{CanColorAndPlayOnRest}.

\noindent\textbf{Case 231.  }\textit{$L(v_1) = \{0, 1\}$, $L(v_2) = \{0, 2, 3\}$, $L(v_3) = \{2, 3, 4\}$, $L(v_4) = \{1, 2\}$ and $L(v_5) = \{2, 3\}$.}

Fixer gets a winning strategy by coloring $v_1v_2$ with $0$ and applying Lemma \ref{CanColorAndPlayOnRest}.

\noindent\textbf{Case 232.  }\textit{$L(v_1) = \{0, 1\}$, $L(v_2) = \{0, 2, 3\}$, $L(v_3) = \{2, 3, 4\}$, $L(v_4) = \{2, 3\}$ and $L(v_5) = \{0, 2\}$.}

Fixer gets a winning strategy by coloring $v_1v_2$ with $0$ and applying Lemma \ref{CanColorAndPlayOnRest}.

\noindent\textbf{Case 233.  }\textit{$L(v_1) = \{0, 1\}$, $L(v_2) = \{0, 2, 3\}$, $L(v_3) = \{2, 3, 4\}$, $L(v_4) = \{2, 3\}$ and $L(v_5) = \{1, 2\}$.}

Fixer gets a winning strategy by coloring $v_1v_2$ with $0$ and applying Lemma \ref{CanColorAndPlayOnRest}.

\noindent\textbf{Case 234.  }\textit{$L(v_1) = \{0, 1\}$, $L(v_2) = \{0, 2, 3\}$, $L(v_3) = \{2, 3, 4\}$, $L(v_4) = \{2, 3\}$ and $L(v_5) = \{2, 3\}$.}

Fixer gets a winning strategy by coloring $v_1v_2$ with $0$ and applying Lemma \ref{CanColorAndPlayOnRest}.

\noindent\textbf{Case 235.  }\textit{$L(v_1) = \{0, 1\}$, $L(v_2) = \{0, 2, 3\}$, $L(v_3) = \{2, 3, 4\}$, $L(v_4) = \{0, 4\}$ and $L(v_5) = \{0, 4\}$.}

Fixer gets a winning strategy by coloring $v_1v_2$ with $0$ and applying Lemma \ref{CanColorAndPlayOnRest}.

\noindent\textbf{Case 236.  }\textit{$L(v_1) = \{0, 1\}$, $L(v_2) = \{0, 2, 3\}$, $L(v_3) = \{2, 3, 4\}$, $L(v_4) = \{1, 4\}$ and $L(v_5) = \{1, 4\}$.}

Fixer gets a winning strategy by coloring $v_1v_2$ with $0$ and applying Lemma \ref{CanColorAndPlayOnRest}.

\noindent\textbf{Case 237.  }\textit{$L(v_1) = \{0, 1\}$, $L(v_2) = \{0, 1, 2\}$, $L(v_3) = \{0, 3, 4\}$, $L(v_4) = \{0, 1\}$ and $L(v_5) = \{3, 5\}$.}

Fixer gets a winning strategy by coloring $v_5v_3$ with $3$ and applying Lemma \ref{CanColorAndPlayOnRest}.

\noindent\textbf{Case 238.  }\textit{$L(v_1) = \{0, 1\}$, $L(v_2) = \{0, 1, 2\}$, $L(v_3) = \{0, 3, 4\}$, $L(v_4) = \{0, 2\}$ and $L(v_5) = \{3, 5\}$.}

Fixer gets a winning strategy by coloring $v_1v_2$ with $1$ and applying Lemma \ref{CanColorAndPlayOnRest}.

\noindent\textbf{Case 239.  }\textit{$L(v_1) = \{0, 1\}$, $L(v_2) = \{0, 1, 2\}$, $L(v_3) = \{0, 3, 4\}$, $L(v_4) = \{3, 5\}$ and $L(v_5) = \{0, 1\}$.}

Fixer gets a winning strategy by coloring $v_4v_3$ with $3$ and applying Lemma \ref{CanColorAndPlayOnRest}.

\noindent\textbf{Case 240.  }\textit{$L(v_1) = \{0, 1\}$, $L(v_2) = \{0, 1, 2\}$, $L(v_3) = \{0, 3, 4\}$, $L(v_4) = \{3, 5\}$ and $L(v_5) = \{0, 2\}$.}

Fixer gets a winning strategy by coloring $v_1v_2$ with $1$ and applying Lemma \ref{CanColorAndPlayOnRest}.

\noindent\textbf{Case 241.  }\textit{$L(v_1) = \{0, 1\}$, $L(v_2) = \{0, 1, 2\}$, $L(v_3) = \{0, 3, 4\}$, $L(v_4) = \{3, 5\}$ and $L(v_5) = \{3, 5\}$.}

Fixer gets a winning strategy by coloring $v_1v_2$ with $1$ and applying Lemma \ref{CanColorAndPlayOnRest}.

\noindent\textbf{Case 242.  }\textit{$L(v_1) = \{0, 1\}$, $L(v_2) = \{0, 1, 2\}$, $L(v_3) = \{2, 3, 4\}$, $L(v_4) = \{0, 2\}$ and $L(v_5) = \{3, 5\}$.}

Fixer gets a winning strategy by coloring $v_1v_2$ with $1$ and applying Lemma \ref{CanColorAndPlayOnRest}.

\noindent\textbf{Case 243.  }\textit{$L(v_1) = \{0, 1\}$, $L(v_2) = \{0, 1, 2\}$, $L(v_3) = \{2, 3, 4\}$, $L(v_4) = \{3, 5\}$ and $L(v_5) = \{0, 2\}$.}

Fixer gets a winning strategy by coloring $v_1v_2$ with $1$ and applying Lemma \ref{CanColorAndPlayOnRest}.

\noindent\textbf{Case 244.  }\textit{$L(v_1) = \{0, 1\}$, $L(v_2) = \{0, 1, 2\}$, $L(v_3) = \{2, 3, 4\}$, $L(v_4) = \{3, 5\}$ and $L(v_5) = \{3, 5\}$.}

Fixer gets a winning strategy by coloring $v_1v_2$ with $0$ and applying Lemma \ref{CanColorAndPlayOnRest}.

\noindent\textbf{Case 245.  }\textit{$L(v_1) = \{0, 1\}$, $L(v_2) = \{0, 2, 3\}$, $L(v_3) = \{0, 1, 4\}$, $L(v_4) = \{0, 1\}$ and $L(v_5) = \{4, 5\}$.}

Fixer gets a winning strategy by coloring $v_5v_3$ with $4$ and applying Lemma \ref{CanColorAndPlayOnRest}.

\noindent\textbf{Case 246.  }\textit{$L(v_1) = \{0, 1\}$, $L(v_2) = \{0, 2, 3\}$, $L(v_3) = \{0, 1, 4\}$, $L(v_4) = \{0, 2\}$ and $L(v_5) = \{4, 5\}$.}

Fixer gets a winning strategy by coloring $v_5v_3$ with $4$ and applying Lemma \ref{CanColorAndPlayOnRest}.

\noindent\textbf{Case 247.  }\textit{$L(v_1) = \{0, 1\}$, $L(v_2) = \{0, 2, 3\}$, $L(v_3) = \{0, 1, 4\}$, $L(v_4) = \{1, 2\}$ and $L(v_5) = \{4, 5\}$.}

Fixer gets a winning strategy by coloring $v_5v_3$ with $4$ and applying Lemma \ref{CanColorAndPlayOnRest}.

\noindent\textbf{Case 248.  }\textit{$L(v_1) = \{0, 1\}$, $L(v_2) = \{0, 2, 3\}$, $L(v_3) = \{0, 1, 4\}$, $L(v_4) = \{4, 5\}$ and $L(v_5) = \{0, 1\}$.}

Fixer gets a winning strategy by coloring $v_4v_3$ with $4$ and applying Lemma \ref{CanColorAndPlayOnRest}.

\noindent\textbf{Case 249.  }\textit{$L(v_1) = \{0, 1\}$, $L(v_2) = \{0, 2, 3\}$, $L(v_3) = \{0, 1, 4\}$, $L(v_4) = \{4, 5\}$ and $L(v_5) = \{0, 2\}$.}

Fixer gets a winning strategy by coloring $v_4v_3$ with $4$ and applying Lemma \ref{CanColorAndPlayOnRest}.

\noindent\textbf{Case 250.  }\textit{$L(v_1) = \{0, 1\}$, $L(v_2) = \{0, 2, 3\}$, $L(v_3) = \{0, 1, 4\}$, $L(v_4) = \{4, 5\}$ and $L(v_5) = \{1, 2\}$.}

Fixer gets a winning strategy by coloring $v_4v_3$ with $4$ and applying Lemma \ref{CanColorAndPlayOnRest}.

\noindent\textbf{Case 251.  }\textit{$L(v_1) = \{0, 1\}$, $L(v_2) = \{0, 2, 3\}$, $L(v_3) = \{0, 2, 4\}$, $L(v_4) = \{1, 2\}$ and $L(v_5) = \{4, 5\}$.}

Fixer gets a winning strategy by coloring $v_5v_3$ with $4$ and applying Lemma \ref{CanColorAndPlayOnRest}.

\noindent\textbf{Case 252.  }\textit{$L(v_1) = \{0, 1\}$, $L(v_2) = \{0, 2, 3\}$, $L(v_3) = \{0, 2, 4\}$, $L(v_4) = \{2, 3\}$ and $L(v_5) = \{0, 5\}$.}

Fixer gets a winning strategy by coloring $v_1v_2$ with $0$ and applying Lemma \ref{CanColorAndPlayOnRest}.

\noindent\textbf{Case 253.  }\textit{$L(v_1) = \{0, 1\}$, $L(v_2) = \{0, 2, 3\}$, $L(v_3) = \{0, 2, 4\}$, $L(v_4) = \{2, 3\}$ and $L(v_5) = \{4, 5\}$.}

Fixer gets a winning strategy by coloring $v_1v_2$ with $0$ and applying Lemma \ref{CanColorAndPlayOnRest}.

\noindent\textbf{Case 254.  }\textit{$L(v_1) = \{0, 1\}$, $L(v_2) = \{0, 2, 3\}$, $L(v_3) = \{0, 2, 4\}$, $L(v_4) = \{0, 5\}$ and $L(v_5) = \{2, 3\}$.}

Fixer gets a winning strategy by coloring $v_1v_2$ with $0$ and applying Lemma \ref{CanColorAndPlayOnRest}.

\noindent\textbf{Case 255.  }\textit{$L(v_1) = \{0, 1\}$, $L(v_2) = \{0, 2, 3\}$, $L(v_3) = \{0, 2, 4\}$, $L(v_4) = \{0, 5\}$ and $L(v_5) = \{0, 5\}$.}

Fixer gets a winning strategy by coloring $v_1v_2$ with $0$ and applying Lemma \ref{CanColorAndPlayOnRest}.

\noindent\textbf{Case 256.  }\textit{$L(v_1) = \{0, 1\}$, $L(v_2) = \{0, 2, 3\}$, $L(v_3) = \{0, 2, 4\}$, $L(v_4) = \{4, 5\}$ and $L(v_5) = \{1, 2\}$.}

Fixer gets a winning strategy by coloring $v_4v_3$ with $4$ and applying Lemma \ref{CanColorAndPlayOnRest}.

\noindent\textbf{Case 257.  }\textit{$L(v_1) = \{0, 1\}$, $L(v_2) = \{0, 2, 3\}$, $L(v_3) = \{0, 2, 4\}$, $L(v_4) = \{4, 5\}$ and $L(v_5) = \{2, 3\}$.}

Fixer gets a winning strategy by coloring $v_1v_2$ with $0$ and applying Lemma \ref{CanColorAndPlayOnRest}.

\noindent\textbf{Case 258.  }\textit{$L(v_1) = \{0, 1\}$, $L(v_2) = \{0, 2, 3\}$, $L(v_3) = \{0, 2, 4\}$, $L(v_4) = \{4, 5\}$ and $L(v_5) = \{4, 5\}$.}

Fixer gets a winning strategy by coloring $v_1v_2$ with $0$ and applying Lemma \ref{CanColorAndPlayOnRest}.

\noindent\textbf{Case 259.  }\textit{$L(v_1) = \{0, 1\}$, $L(v_2) = \{0, 2, 3\}$, $L(v_3) = \{1, 2, 4\}$, $L(v_4) = \{0, 2\}$ and $L(v_5) = \{4, 5\}$.}

Fixer gets a winning strategy by coloring $v_5v_3$ with $4$ and applying Lemma \ref{CanColorAndPlayOnRest}.

\noindent\textbf{Case 260.  }\textit{$L(v_1) = \{0, 1\}$, $L(v_2) = \{0, 2, 3\}$, $L(v_3) = \{1, 2, 4\}$, $L(v_4) = \{2, 3\}$ and $L(v_5) = \{1, 5\}$.}

Fixer gets a winning strategy by coloring $v_1v_2$ with $0$ and applying Lemma \ref{CanColorAndPlayOnRest}.

\noindent\textbf{Case 261.  }\textit{$L(v_1) = \{0, 1\}$, $L(v_2) = \{0, 2, 3\}$, $L(v_3) = \{1, 2, 4\}$, $L(v_4) = \{2, 3\}$ and $L(v_5) = \{4, 5\}$.}

Fixer gets a winning strategy by coloring $v_1v_2$ with $0$ and applying Lemma \ref{CanColorAndPlayOnRest}.

\noindent\textbf{Case 262.  }\textit{$L(v_1) = \{0, 1\}$, $L(v_2) = \{0, 2, 3\}$, $L(v_3) = \{1, 2, 4\}$, $L(v_4) = \{1, 5\}$ and $L(v_5) = \{2, 3\}$.}

Fixer gets a winning strategy by coloring $v_1v_2$ with $0$ and applying Lemma \ref{CanColorAndPlayOnRest}.

\noindent\textbf{Case 263.  }\textit{$L(v_1) = \{0, 1\}$, $L(v_2) = \{0, 2, 3\}$, $L(v_3) = \{1, 2, 4\}$, $L(v_4) = \{1, 5\}$ and $L(v_5) = \{1, 5\}$.}

Fixer gets a winning strategy by coloring $v_1v_2$ with $0$ and applying Lemma \ref{CanColorAndPlayOnRest}.

\noindent\textbf{Case 264.  }\textit{$L(v_1) = \{0, 1\}$, $L(v_2) = \{0, 2, 3\}$, $L(v_3) = \{1, 2, 4\}$, $L(v_4) = \{4, 5\}$ and $L(v_5) = \{0, 2\}$.}

Fixer gets a winning strategy by coloring $v_4v_3$ with $4$ and applying Lemma \ref{CanColorAndPlayOnRest}.

\noindent\textbf{Case 265.  }\textit{$L(v_1) = \{0, 1\}$, $L(v_2) = \{0, 2, 3\}$, $L(v_3) = \{1, 2, 4\}$, $L(v_4) = \{4, 5\}$ and $L(v_5) = \{2, 3\}$.}

Fixer gets a winning strategy by coloring $v_1v_2$ with $0$ and applying Lemma \ref{CanColorAndPlayOnRest}.

\noindent\textbf{Case 266.  }\textit{$L(v_1) = \{0, 1\}$, $L(v_2) = \{0, 2, 3\}$, $L(v_3) = \{1, 2, 4\}$, $L(v_4) = \{4, 5\}$ and $L(v_5) = \{4, 5\}$.}

Fixer gets a winning strategy by coloring $v_1v_2$ with $0$ and applying Lemma \ref{CanColorAndPlayOnRest}.

\noindent\textbf{Case 267.  }\textit{$L(v_1) = \{0, 1\}$, $L(v_2) = \{0, 2, 3\}$, $L(v_3) = \{2, 3, 4\}$, $L(v_4) = \{2, 3\}$ and $L(v_5) = \{2, 5\}$.}

Fixer gets a winning strategy by coloring $v_1v_2$ with $0$ and applying Lemma \ref{CanColorAndPlayOnRest}.

\noindent\textbf{Case 268.  }\textit{$L(v_1) = \{0, 1\}$, $L(v_2) = \{0, 2, 3\}$, $L(v_3) = \{2, 3, 4\}$, $L(v_4) = \{2, 5\}$ and $L(v_5) = \{2, 3\}$.}

Fixer gets a winning strategy by coloring $v_1v_2$ with $0$ and applying Lemma \ref{CanColorAndPlayOnRest}.

\noindent\textbf{Case 269.  }\textit{$L(v_1) = \{0, 1\}$, $L(v_2) = \{0, 2, 3\}$, $L(v_3) = \{2, 3, 4\}$, $L(v_4) = \{2, 5\}$ and $L(v_5) = \{2, 5\}$.}

Fixer gets a winning strategy by coloring $v_1v_2$ with $0$ and applying Lemma \ref{CanColorAndPlayOnRest}.

\noindent\textbf{Case 270.  }\textit{$L(v_1) = \{0, 1\}$, $L(v_2) = \{0, 2, 3\}$, $L(v_3) = \{2, 3, 4\}$, $L(v_4) = \{2, 5\}$ and $L(v_5) = \{3, 5\}$.}

Fixer gets a winning strategy by coloring $v_1v_2$ with $0$ and applying Lemma \ref{CanColorAndPlayOnRest}.

\noindent\textbf{Case 271.  }\textit{$L(v_1) = \{0, 1\}$, $L(v_2) = \{0, 2, 3\}$, $L(v_3) = \{2, 3, 4\}$, $L(v_4) = \{4, 5\}$ and $L(v_5) = \{4, 5\}$.}

Fixer gets a winning strategy by coloring $v_1v_2$ with $0$ and applying Lemma \ref{CanColorAndPlayOnRest}.

\noindent\textbf{Case 272.  }\textit{$L(v_1) = \{0, 1\}$, $L(v_2) = \{0, 2, 3\}$, $L(v_3) = \{2, 4, 5\}$, $L(v_4) = \{2, 3\}$ and $L(v_5) = \{2, 3\}$.}

Fixer gets a winning strategy by coloring $v_1v_2$ with $0$ and applying Lemma \ref{CanColorAndPlayOnRest}.

\noindent\textbf{Case 273.  }\textit{$L(v_1) = \{0, 1\}$, $L(v_2) = \{0, 2, 3\}$, $L(v_3) = \{2, 4, 5\}$, $L(v_4) = \{2, 3\}$ and $L(v_5) = \{0, 4\}$.}

Fixer gets a winning strategy by coloring $v_1v_2$ with $0$ and applying Lemma \ref{CanColorAndPlayOnRest}.

\noindent\textbf{Case 274.  }\textit{$L(v_1) = \{0, 1\}$, $L(v_2) = \{0, 2, 3\}$, $L(v_3) = \{2, 4, 5\}$, $L(v_4) = \{2, 3\}$ and $L(v_5) = \{1, 4\}$.}

Fixer gets a winning strategy by coloring $v_1v_2$ with $0$ and applying Lemma \ref{CanColorAndPlayOnRest}.

\noindent\textbf{Case 275.  }\textit{$L(v_1) = \{0, 1\}$, $L(v_2) = \{0, 2, 3\}$, $L(v_3) = \{2, 4, 5\}$, $L(v_4) = \{2, 3\}$ and $L(v_5) = \{2, 4\}$.}

Fixer gets a winning strategy by coloring $v_1v_2$ with $0$ and applying Lemma \ref{CanColorAndPlayOnRest}.

\noindent\textbf{Case 276.  }\textit{$L(v_1) = \{0, 1\}$, $L(v_2) = \{0, 2, 3\}$, $L(v_3) = \{2, 4, 5\}$, $L(v_4) = \{2, 3\}$ and $L(v_5) = \{3, 4\}$.}

Fixer gets a winning strategy by coloring $v_1v_2$ with $0$ and applying Lemma \ref{CanColorAndPlayOnRest}.

\noindent\textbf{Case 277.  }\textit{$L(v_1) = \{0, 1\}$, $L(v_2) = \{0, 2, 3\}$, $L(v_3) = \{2, 4, 5\}$, $L(v_4) = \{2, 3\}$ and $L(v_5) = \{4, 5\}$.}

Fixer gets a winning strategy by coloring $v_1v_2$ with $0$ and applying Lemma \ref{CanColorAndPlayOnRest}.

\noindent\textbf{Case 278.  }\textit{$L(v_1) = \{0, 1\}$, $L(v_2) = \{0, 2, 3\}$, $L(v_3) = \{2, 4, 5\}$, $L(v_4) = \{0, 4\}$ and $L(v_5) = \{2, 3\}$.}

Fixer gets a winning strategy by coloring $v_1v_2$ with $0$ and applying Lemma \ref{CanColorAndPlayOnRest}.

\noindent\textbf{Case 279.  }\textit{$L(v_1) = \{0, 1\}$, $L(v_2) = \{0, 2, 3\}$, $L(v_3) = \{2, 4, 5\}$, $L(v_4) = \{0, 4\}$ and $L(v_5) = \{0, 4\}$.}

Fixer gets a winning strategy by coloring $v_1v_2$ with $0$ and applying Lemma \ref{CanColorAndPlayOnRest}.

\noindent\textbf{Case 280.  }\textit{$L(v_1) = \{0, 1\}$, $L(v_2) = \{0, 2, 3\}$, $L(v_3) = \{2, 4, 5\}$, $L(v_4) = \{1, 4\}$ and $L(v_5) = \{2, 3\}$.}

Fixer gets a winning strategy by coloring $v_1v_2$ with $0$ and applying Lemma \ref{CanColorAndPlayOnRest}.

\noindent\textbf{Case 281.  }\textit{$L(v_1) = \{0, 1\}$, $L(v_2) = \{0, 2, 3\}$, $L(v_3) = \{2, 4, 5\}$, $L(v_4) = \{1, 4\}$ and $L(v_5) = \{1, 4\}$.}

Fixer gets a winning strategy by coloring $v_1v_2$ with $0$ and applying Lemma \ref{CanColorAndPlayOnRest}.

\noindent\textbf{Case 282.  }\textit{$L(v_1) = \{0, 1\}$, $L(v_2) = \{0, 2, 3\}$, $L(v_3) = \{2, 4, 5\}$, $L(v_4) = \{2, 4\}$ and $L(v_5) = \{2, 3\}$.}

Fixer gets a winning strategy by coloring $v_1v_2$ with $0$ and applying Lemma \ref{CanColorAndPlayOnRest}.

\noindent\textbf{Case 283.  }\textit{$L(v_1) = \{0, 1\}$, $L(v_2) = \{0, 2, 3\}$, $L(v_3) = \{2, 4, 5\}$, $L(v_4) = \{2, 4\}$ and $L(v_5) = \{2, 4\}$.}

Fixer gets a winning strategy by coloring $v_1v_2$ with $0$ and applying Lemma \ref{CanColorAndPlayOnRest}.

\noindent\textbf{Case 284.  }\textit{$L(v_1) = \{0, 1\}$, $L(v_2) = \{0, 2, 3\}$, $L(v_3) = \{2, 4, 5\}$, $L(v_4) = \{2, 4\}$ and $L(v_5) = \{3, 4\}$.}

Fixer gets a winning strategy by coloring $v_1v_2$ with $0$ and applying Lemma \ref{CanColorAndPlayOnRest}.

\noindent\textbf{Case 285.  }\textit{$L(v_1) = \{0, 1\}$, $L(v_2) = \{0, 2, 3\}$, $L(v_3) = \{2, 4, 5\}$, $L(v_4) = \{3, 4\}$ and $L(v_5) = \{2, 3\}$.}

Fixer gets a winning strategy by coloring $v_1v_2$ with $0$ and applying Lemma \ref{CanColorAndPlayOnRest}.

\noindent\textbf{Case 286.  }\textit{$L(v_1) = \{0, 1\}$, $L(v_2) = \{0, 2, 3\}$, $L(v_3) = \{2, 4, 5\}$, $L(v_4) = \{3, 4\}$ and $L(v_5) = \{2, 4\}$.}

Fixer gets a winning strategy by coloring $v_1v_2$ with $0$ and applying Lemma \ref{CanColorAndPlayOnRest}.

\noindent\textbf{Case 287.  }\textit{$L(v_1) = \{0, 1\}$, $L(v_2) = \{0, 2, 3\}$, $L(v_3) = \{2, 4, 5\}$, $L(v_4) = \{3, 4\}$ and $L(v_5) = \{3, 4\}$.}

Fixer gets a winning strategy by coloring $v_1v_2$ with $0$ and applying Lemma \ref{CanColorAndPlayOnRest}.

\noindent\textbf{Case 288.  }\textit{$L(v_1) = \{0, 1\}$, $L(v_2) = \{0, 2, 3\}$, $L(v_3) = \{2, 4, 5\}$, $L(v_4) = \{4, 5\}$ and $L(v_5) = \{2, 3\}$.}

Fixer gets a winning strategy by coloring $v_1v_2$ with $0$ and applying Lemma \ref{CanColorAndPlayOnRest}.

\noindent\textbf{Case 289.  }\textit{$L(v_1) = \{0, 1\}$, $L(v_2) = \{0, 2, 3\}$, $L(v_3) = \{2, 4, 5\}$, $L(v_4) = \{2, 3\}$ and $L(v_5) = \{4, 6\}$.}

Fixer gets a winning strategy by coloring $v_1v_2$ with $0$ and applying Lemma \ref{CanColorAndPlayOnRest}.

\noindent\textbf{Case 290.  }\textit{$L(v_1) = \{0, 1\}$, $L(v_2) = \{0, 2, 3\}$, $L(v_3) = \{2, 4, 5\}$, $L(v_4) = \{4, 6\}$ and $L(v_5) = \{2, 3\}$.}

Fixer gets a winning strategy by coloring $v_1v_2$ with $0$ and applying Lemma \ref{CanColorAndPlayOnRest}.

\noindent\textbf{Case 291.  }\textit{$L(v_1) = \{0, 1\}$, $L(v_2) = \{0, 2, 3\}$, $L(v_3) = \{2, 4, 5\}$, $L(v_4) = \{4, 6\}$ and $L(v_5) = \{4, 6\}$.}

Fixer gets a winning strategy by coloring $v_1v_2$ with $0$ and applying Lemma \ref{CanColorAndPlayOnRest}.

\noindent\textbf{Case 292.  }\textit{$L(v_1) = \{0, 1\}$, $L(v_2) = \{0, 1, 2\}$, $L(v_3) = \{0, 1, 3\}$, $L(v_4) = \{0, 4\}$ and $L(v_5) = \{1, 4\}$.}

Let $S$ and $A_S$ be as in Lemma \ref{MultiMoveCombination} using colors $2$ and $4$. If the components of $A_S$ have vertex sets $\{v_1\}$ and $\{v_3, v_4\}$, then Fixer should swap 2 and 4 at $v_1$. This results in a position with lists $L(v_1) = \{0, 1\}$, $L(v_2) = \{0, 1, 4\}$, $L(v_3) = \{0, 1, 3\}$, $L(v_4) = \{0, 4\}$ and $L(v_5) = \{1, 4\}$, but then Fixer can edge-color the graph. If the components of $A_S$ have vertex sets $\{v_3\}$ and $\{v_1, v_4\}$, then Fixer should swap 2 and 4 at $v_3$. This results in a position with lists $L(v_1) = \{0, 1\}$, $L(v_2) = \{0, 1, 2\}$, $L(v_3) = \{0, 1, 3\}$, $L(v_4) = \{0, 2\}$ and $L(v_5) = \{1, 4\}$, but then Fixer wins by Case 51. If the components of $A_S$ have vertex sets $\{v_4\}$ and $\{v_1, v_3\}$, then Fixer should swap 2 and 4 at $v_4$. This results in a position with lists $L(v_1) = \{0, 1\}$, $L(v_2) = \{0, 1, 2\}$, $L(v_3) = \{0, 1, 3\}$, $L(v_4) = \{0, 4\}$ and $L(v_5) = \{1, 2\}$, but then Fixer wins by Case 53. 

\noindent\textbf{Case 293.  }\textit{$L(v_1) = \{0, 1\}$, $L(v_2) = \{0, 1, 2\}$, $L(v_3) = \{0, 3, 4\}$, $L(v_4) = \{0, 1\}$ and $L(v_5) = \{0, 1\}$.}

Let $S$ and $A_S$ be as in Lemma \ref{MultiMoveCombination} using colors $1$ and $3$. If the components of $A_S$ have vertex sets $\{v_0\}$, $\{v_1, v_2\}$ and $\{v_3, v_4\}$, then Fixer should swap 1 and 3 at $v_4$ and $v_3$. This results in a position with lists $L(v_1) = \{0, 1\}$, $L(v_2) = \{0, 1, 2\}$, $L(v_3) = \{0, 3, 4\}$, $L(v_4) = \{0, 3\}$ and $L(v_5) = \{0, 3\}$, but then Fixer wins by Case 80. If the components of $A_S$ have vertex sets $\{v_0\}$, $\{v_1, v_3\}$ and $\{v_2, v_4\}$, then Fixer should swap 1 and 3 at $v_4$ and $v_2$. This results in a position with lists $L(v_1) = \{0, 1\}$, $L(v_2) = \{0, 1, 2\}$, $L(v_3) = \{0, 1, 4\}$, $L(v_4) = \{0, 1\}$ and $L(v_5) = \{0, 3\}$, but then Fixer wins by Case 50. If the components of $A_S$ have vertex sets $\{v_0\}$, $\{v_1, v_4\}$ and $\{v_2, v_3\}$, then Fixer should swap 1 and 3 at $v_3$ and $v_2$. This results in a position with lists $L(v_1) = \{0, 1\}$, $L(v_2) = \{0, 1, 2\}$, $L(v_3) = \{0, 1, 4\}$, $L(v_4) = \{0, 3\}$ and $L(v_5) = \{0, 1\}$, but then Fixer wins by Case 52. If the components of $A_S$ have vertex sets $\{v_1\}$, $\{v_0, v_2\}$ and $\{v_3, v_4\}$, then Fixer should swap 1 and 3 at $v_1$. This results in a position with lists $L(v_1) = \{0, 1\}$, $L(v_2) = \{0, 2, 3\}$, $L(v_3) = \{0, 3, 4\}$, $L(v_4) = \{0, 1\}$ and $L(v_5) = \{0, 1\}$, but then Fixer wins by Case 160. If the components of $A_S$ have vertex sets $\{v_1\}$, $\{v_0, v_3\}$ and $\{v_2, v_4\}$, then Fixer should swap 1 and 3 at $v_1$. This results in a position with lists $L(v_1) = \{0, 1\}$, $L(v_2) = \{0, 2, 3\}$, $L(v_3) = \{0, 3, 4\}$, $L(v_4) = \{0, 1\}$ and $L(v_5) = \{0, 1\}$, but then Fixer wins by Case 160. If the components of $A_S$ have vertex sets $\{v_1\}$, $\{v_0, v_4\}$ and $\{v_2, v_3\}$, then Fixer should swap 1 and 3 at $v_1$. This results in a position with lists $L(v_1) = \{0, 1\}$, $L(v_2) = \{0, 2, 3\}$, $L(v_3) = \{0, 3, 4\}$, $L(v_4) = \{0, 1\}$ and $L(v_5) = \{0, 1\}$, but then Fixer wins by Case 160. If the components of $A_S$ have vertex sets $\{v_2\}$, $\{v_0, v_1\}$ and $\{v_3, v_4\}$, then Fixer should swap 1 and 3 at $v_2$. This results in a position with lists $L(v_1) = \{0, 1\}$, $L(v_2) = \{0, 1, 2\}$, $L(v_3) = \{0, 1, 4\}$, $L(v_4) = \{0, 1\}$ and $L(v_5) = \{0, 1\}$, but then Fixer can edge-color the graph. If the components of $A_S$ have vertex sets $\{v_3\}$, $\{v_0, v_1\}$ and $\{v_2, v_4\}$, then Fixer should swap 1 and 3 at $v_3$. This results in a position with lists $L(v_1) = \{0, 1\}$, $L(v_2) = \{0, 1, 2\}$, $L(v_3) = \{0, 3, 4\}$, $L(v_4) = \{0, 3\}$ and $L(v_5) = \{0, 1\}$, but then Fixer wins by Case 78. If the components of $A_S$ have vertex sets $\{v_4\}$, $\{v_0, v_1\}$ and $\{v_2, v_3\}$, then Fixer should swap 1 and 3 at $v_4$. This results in a position with lists $L(v_1) = \{0, 1\}$, $L(v_2) = \{0, 1, 2\}$, $L(v_3) = \{0, 3, 4\}$, $L(v_4) = \{0, 1\}$ and $L(v_5) = \{0, 3\}$, but then Fixer wins by Case 69. If the components of $A_S$ have vertex sets $\{v_2\}$, $\{v_0, v_3\}$ and $\{v_1, v_4\}$, then Fixer should swap 1 and 3 at $v_2$. This results in a position with lists $L(v_1) = \{0, 1\}$, $L(v_2) = \{0, 1, 2\}$, $L(v_3) = \{0, 1, 4\}$, $L(v_4) = \{0, 1\}$ and $L(v_5) = \{0, 1\}$, but then Fixer can edge-color the graph. If the components of $A_S$ have vertex sets $\{v_2\}$, $\{v_0, v_4\}$ and $\{v_1, v_3\}$, then Fixer should swap 1 and 3 at $v_2$. This results in a position with lists $L(v_1) = \{0, 1\}$, $L(v_2) = \{0, 1, 2\}$, $L(v_3) = \{0, 1, 4\}$, $L(v_4) = \{0, 1\}$ and $L(v_5) = \{0, 1\}$, but then Fixer can edge-color the graph. If the components of $A_S$ have vertex sets $\{v_3\}$, $\{v_0, v_2\}$ and $\{v_1, v_4\}$, then Fixer should swap 1 and 3 at $v_3$. This results in a position with lists $L(v_1) = \{0, 1\}$, $L(v_2) = \{0, 1, 2\}$, $L(v_3) = \{0, 3, 4\}$, $L(v_4) = \{0, 3\}$ and $L(v_5) = \{0, 1\}$, but then Fixer wins by Case 78. If the components of $A_S$ have vertex sets $\{v_4\}$, $\{v_0, v_2\}$ and $\{v_1, v_3\}$, then Fixer should swap 1 and 3 at $v_4$. This results in a position with lists $L(v_1) = \{0, 1\}$, $L(v_2) = \{0, 1, 2\}$, $L(v_3) = \{0, 3, 4\}$, $L(v_4) = \{0, 1\}$ and $L(v_5) = \{0, 3\}$, but then Fixer wins by Case 69. If the components of $A_S$ have vertex sets $\{v_3\}$, $\{v_0, v_4\}$ and $\{v_1, v_2\}$, then Fixer should swap 1 and 3 at $v_3$. This results in a position with lists $L(v_1) = \{0, 1\}$, $L(v_2) = \{0, 1, 2\}$, $L(v_3) = \{0, 3, 4\}$, $L(v_4) = \{0, 3\}$ and $L(v_5) = \{0, 1\}$, but then Fixer wins by Case 78. If the components of $A_S$ have vertex sets $\{v_4\}$, $\{v_0, v_3\}$ and $\{v_1, v_2\}$, then Fixer should swap 1 and 3 at $v_4$. This results in a position with lists $L(v_1) = \{0, 1\}$, $L(v_2) = \{0, 1, 2\}$, $L(v_3) = \{0, 3, 4\}$, $L(v_4) = \{0, 1\}$ and $L(v_5) = \{0, 3\}$, but then Fixer wins by Case 69. 

\noindent\textbf{Case 294.  }\textit{$L(v_1) = \{0, 1\}$, $L(v_2) = \{0, 2, 3\}$, $L(v_3) = \{0, 1, 2\}$, $L(v_4) = \{0, 4\}$ and $L(v_5) = \{2, 4\}$.}

Let $S$ and $A_S$ be as in Lemma \ref{MultiMoveCombination} using colors $1$ and $3$. If the components of $A_S$ have vertex sets $\{v_0\}$ and $\{v_1, v_2\}$, then Fixer should swap 1 and 3 at $v_0$. This results in a position with lists $L(v_1) = \{0, 3\}$, $L(v_2) = \{0, 2, 3\}$, $L(v_3) = \{0, 1, 2\}$, $L(v_4) = \{0, 4\}$ and $L(v_5) = \{2, 4\}$, but then Fixer wins by Case 63. If the components of $A_S$ have vertex sets $\{v_1\}$ and $\{v_0, v_2\}$, then Fixer should swap 1 and 3 at $v_1$. This results in a position with lists $L(v_1) = \{0, 1\}$, $L(v_2) = \{0, 1, 2\}$, $L(v_3) = \{0, 1, 2\}$, $L(v_4) = \{0, 4\}$ and $L(v_5) = \{2, 4\}$, but then Fixer can edge-color the graph. If the components of $A_S$ have vertex sets $\{v_2\}$ and $\{v_0, v_1\}$, then Fixer should swap 1 and 3 at $v_2$. This results in a position with lists $L(v_1) = \{0, 1\}$, $L(v_2) = \{0, 2, 3\}$, $L(v_3) = \{0, 2, 3\}$, $L(v_4) = \{0, 4\}$ and $L(v_5) = \{2, 4\}$, but then Fixer can edge-color the graph. 

\noindent\textbf{Case 295.  }\textit{$L(v_1) = \{0, 1\}$, $L(v_2) = \{0, 2, 3\}$, $L(v_3) = \{0, 1, 2\}$, $L(v_4) = \{1, 4\}$ and $L(v_5) = \{2, 4\}$.}

Let $S$ and $A_S$ be as in Lemma \ref{MultiMoveCombination} using colors $0$ and $4$. If the components of $A_S$ have vertex sets $\{v_0\}$, $\{v_1, v_2\}$ and $\{v_3, v_4\}$, then Fixer should swap 0 and 4 at $v_4$ and $v_3$. This results in a position with lists $L(v_1) = \{0, 1\}$, $L(v_2) = \{0, 2, 3\}$, $L(v_3) = \{0, 1, 2\}$, $L(v_4) = \{0, 1\}$ and $L(v_5) = \{0, 2\}$, but then Fixer can edge-color the graph. If the components of $A_S$ have vertex sets $\{v_0\}$, $\{v_1, v_3\}$ and $\{v_2, v_4\}$, then Fixer should swap 0 and 4 at $v_4$ and $v_2$. This results in a position with lists $L(v_1) = \{0, 1\}$, $L(v_2) = \{0, 2, 3\}$, $L(v_3) = \{1, 2, 4\}$, $L(v_4) = \{1, 4\}$ and $L(v_5) = \{0, 2\}$, but then Fixer wins by Case 216. If the components of $A_S$ have vertex sets $\{v_0\}$, $\{v_1, v_4\}$ and $\{v_2, v_3\}$, then Fixer should swap 0 and 4 at $v_3$ and $v_2$. This results in a position with lists $L(v_1) = \{0, 1\}$, $L(v_2) = \{0, 2, 3\}$, $L(v_3) = \{1, 2, 4\}$, $L(v_4) = \{0, 1\}$ and $L(v_5) = \{2, 4\}$, but then Fixer can edge-color the graph. If the components of $A_S$ have vertex sets $\{v_1\}$, $\{v_0, v_2\}$ and $\{v_3, v_4\}$, then Fixer should swap 0 and 4 at $v_4$ and $v_3$. This results in a position with lists $L(v_1) = \{0, 1\}$, $L(v_2) = \{0, 2, 3\}$, $L(v_3) = \{0, 1, 2\}$, $L(v_4) = \{0, 1\}$ and $L(v_5) = \{0, 2\}$, but then Fixer can edge-color the graph. If the components of $A_S$ have vertex sets $\{v_1\}$, $\{v_0, v_3\}$ and $\{v_2, v_4\}$, then Fixer should swap 0 and 4 at $v_4$ and $v_2$. This results in a position with lists $L(v_1) = \{0, 1\}$, $L(v_2) = \{0, 2, 3\}$, $L(v_3) = \{1, 2, 4\}$, $L(v_4) = \{1, 4\}$ and $L(v_5) = \{0, 2\}$, but then Fixer wins by Case 216. If the components of $A_S$ have vertex sets $\{v_1\}$, $\{v_0, v_4\}$ and $\{v_2, v_3\}$, then Fixer should swap 0 and 4 at $v_3$ and $v_2$. This results in a position with lists $L(v_1) = \{0, 1\}$, $L(v_2) = \{0, 2, 3\}$, $L(v_3) = \{1, 2, 4\}$, $L(v_4) = \{0, 1\}$ and $L(v_5) = \{2, 4\}$, but then Fixer can edge-color the graph. If the components of $A_S$ have vertex sets $\{v_2\}$, $\{v_0, v_1\}$ and $\{v_3, v_4\}$, then Fixer should swap 0 and 4 at $v_2$. This results in a position with lists $L(v_1) = \{0, 1\}$, $L(v_2) = \{0, 2, 3\}$, $L(v_3) = \{1, 2, 4\}$, $L(v_4) = \{1, 4\}$ and $L(v_5) = \{2, 4\}$, but then Fixer can edge-color the graph. If the components of $A_S$ have vertex sets $\{v_3\}$, $\{v_0, v_1\}$ and $\{v_2, v_4\}$, then Fixer should swap 0 and 4 at $v_3$. This results in a position with lists $L(v_1) = \{0, 1\}$, $L(v_2) = \{0, 2, 3\}$, $L(v_3) = \{0, 1, 2\}$, $L(v_4) = \{0, 1\}$ and $L(v_5) = \{2, 4\}$, but then Fixer wins by Case 104. If the components of $A_S$ have vertex sets $\{v_4\}$, $\{v_0, v_1\}$ and $\{v_2, v_3\}$, then Fixer should swap 0 and 4 at $v_4$. This results in a position with lists $L(v_1) = \{0, 1\}$, $L(v_2) = \{0, 2, 3\}$, $L(v_3) = \{0, 1, 2\}$, $L(v_4) = \{1, 4\}$ and $L(v_5) = \{0, 2\}$, but then Fixer can edge-color the graph. If the components of $A_S$ have vertex sets $\{v_2\}$, $\{v_0, v_3\}$ and $\{v_1, v_4\}$, then Fixer should swap 0 and 4 at $v_2$. This results in a position with lists $L(v_1) = \{0, 1\}$, $L(v_2) = \{0, 2, 3\}$, $L(v_3) = \{1, 2, 4\}$, $L(v_4) = \{1, 4\}$ and $L(v_5) = \{2, 4\}$, but then Fixer can edge-color the graph. If the components of $A_S$ have vertex sets $\{v_2\}$, $\{v_0, v_4\}$ and $\{v_1, v_3\}$, then Fixer should swap 0 and 4 at $v_2$. This results in a position with lists $L(v_1) = \{0, 1\}$, $L(v_2) = \{0, 2, 3\}$, $L(v_3) = \{1, 2, 4\}$, $L(v_4) = \{1, 4\}$ and $L(v_5) = \{2, 4\}$, but then Fixer can edge-color the graph. If the components of $A_S$ have vertex sets $\{v_3\}$, $\{v_0, v_2\}$ and $\{v_1, v_4\}$, then Fixer should swap 0 and 4 at $v_3$. This results in a position with lists $L(v_1) = \{0, 1\}$, $L(v_2) = \{0, 2, 3\}$, $L(v_3) = \{0, 1, 2\}$, $L(v_4) = \{0, 1\}$ and $L(v_5) = \{2, 4\}$, but then Fixer wins by Case 104. If the components of $A_S$ have vertex sets $\{v_4\}$, $\{v_0, v_2\}$ and $\{v_1, v_3\}$, then Fixer should swap 0 and 4 at $v_4$. This results in a position with lists $L(v_1) = \{0, 1\}$, $L(v_2) = \{0, 2, 3\}$, $L(v_3) = \{0, 1, 2\}$, $L(v_4) = \{1, 4\}$ and $L(v_5) = \{0, 2\}$, but then Fixer can edge-color the graph. If the components of $A_S$ have vertex sets $\{v_3\}$, $\{v_0, v_4\}$ and $\{v_1, v_2\}$, then Fixer should swap 0 and 4 at $v_3$. This results in a position with lists $L(v_1) = \{0, 1\}$, $L(v_2) = \{0, 2, 3\}$, $L(v_3) = \{0, 1, 2\}$, $L(v_4) = \{0, 1\}$ and $L(v_5) = \{2, 4\}$, but then Fixer wins by Case 104. If the components of $A_S$ have vertex sets $\{v_4\}$, $\{v_0, v_3\}$ and $\{v_1, v_2\}$, then Fixer should swap 0 and 4 at $v_4$. This results in a position with lists $L(v_1) = \{0, 1\}$, $L(v_2) = \{0, 2, 3\}$, $L(v_3) = \{0, 1, 2\}$, $L(v_4) = \{1, 4\}$ and $L(v_5) = \{0, 2\}$, but then Fixer can edge-color the graph. 

\noindent\textbf{Case 296.  }\textit{$L(v_1) = \{0, 1\}$, $L(v_2) = \{0, 2, 3\}$, $L(v_3) = \{0, 1, 2\}$, $L(v_4) = \{2, 4\}$ and $L(v_5) = \{0, 4\}$.}

Let $S$ and $A_S$ be as in Lemma \ref{MultiMoveCombination} using colors $1$ and $3$. If the components of $A_S$ have vertex sets $\{v_0\}$ and $\{v_1, v_2\}$, then Fixer should swap 1 and 3 at $v_0$. This results in a position with lists $L(v_1) = \{0, 3\}$, $L(v_2) = \{0, 2, 3\}$, $L(v_3) = \{0, 1, 2\}$, $L(v_4) = \{2, 4\}$ and $L(v_5) = \{0, 4\}$, but then Fixer wins by Case 66. If the components of $A_S$ have vertex sets $\{v_1\}$ and $\{v_0, v_2\}$, then Fixer should swap 1 and 3 at $v_1$. This results in a position with lists $L(v_1) = \{0, 1\}$, $L(v_2) = \{0, 1, 2\}$, $L(v_3) = \{0, 1, 2\}$, $L(v_4) = \{2, 4\}$ and $L(v_5) = \{0, 4\}$, but then Fixer can edge-color the graph. If the components of $A_S$ have vertex sets $\{v_2\}$ and $\{v_0, v_1\}$, then Fixer should swap 1 and 3 at $v_2$. This results in a position with lists $L(v_1) = \{0, 1\}$, $L(v_2) = \{0, 2, 3\}$, $L(v_3) = \{0, 2, 3\}$, $L(v_4) = \{2, 4\}$ and $L(v_5) = \{0, 4\}$, but then Fixer can edge-color the graph. 

\noindent\textbf{Case 297.  }\textit{$L(v_1) = \{0, 1\}$, $L(v_2) = \{0, 2, 3\}$, $L(v_3) = \{0, 1, 2\}$, $L(v_4) = \{2, 4\}$ and $L(v_5) = \{1, 4\}$.}

Let $S$ and $A_S$ be as in Lemma \ref{MultiMoveCombination} using colors $0$ and $4$. If the components of $A_S$ have vertex sets $\{v_0\}$, $\{v_1, v_2\}$ and $\{v_3, v_4\}$, then Fixer should swap 0 and 4 at $v_4$ and $v_3$. This results in a position with lists $L(v_1) = \{0, 1\}$, $L(v_2) = \{0, 2, 3\}$, $L(v_3) = \{0, 1, 2\}$, $L(v_4) = \{0, 2\}$ and $L(v_5) = \{0, 1\}$, but then Fixer can edge-color the graph. If the components of $A_S$ have vertex sets $\{v_0\}$, $\{v_1, v_3\}$ and $\{v_2, v_4\}$, then Fixer should swap 0 and 4 at $v_4$ and $v_2$. This results in a position with lists $L(v_1) = \{0, 1\}$, $L(v_2) = \{0, 2, 3\}$, $L(v_3) = \{1, 2, 4\}$, $L(v_4) = \{2, 4\}$ and $L(v_5) = \{0, 1\}$, but then Fixer can edge-color the graph. If the components of $A_S$ have vertex sets $\{v_0\}$, $\{v_1, v_4\}$ and $\{v_2, v_3\}$, then Fixer should swap 0 and 4 at $v_3$ and $v_2$. This results in a position with lists $L(v_1) = \{0, 1\}$, $L(v_2) = \{0, 2, 3\}$, $L(v_3) = \{1, 2, 4\}$, $L(v_4) = \{0, 2\}$ and $L(v_5) = \{1, 4\}$, but then Fixer wins by Case 196. If the components of $A_S$ have vertex sets $\{v_1\}$, $\{v_0, v_2\}$ and $\{v_3, v_4\}$, then Fixer should swap 0 and 4 at $v_4$ and $v_3$. This results in a position with lists $L(v_1) = \{0, 1\}$, $L(v_2) = \{0, 2, 3\}$, $L(v_3) = \{0, 1, 2\}$, $L(v_4) = \{0, 2\}$ and $L(v_5) = \{0, 1\}$, but then Fixer can edge-color the graph. If the components of $A_S$ have vertex sets $\{v_1\}$, $\{v_0, v_3\}$ and $\{v_2, v_4\}$, then Fixer should swap 0 and 4 at $v_4$ and $v_2$. This results in a position with lists $L(v_1) = \{0, 1\}$, $L(v_2) = \{0, 2, 3\}$, $L(v_3) = \{1, 2, 4\}$, $L(v_4) = \{2, 4\}$ and $L(v_5) = \{0, 1\}$, but then Fixer can edge-color the graph. If the components of $A_S$ have vertex sets $\{v_1\}$, $\{v_0, v_4\}$ and $\{v_2, v_3\}$, then Fixer should swap 0 and 4 at $v_3$ and $v_2$. This results in a position with lists $L(v_1) = \{0, 1\}$, $L(v_2) = \{0, 2, 3\}$, $L(v_3) = \{1, 2, 4\}$, $L(v_4) = \{0, 2\}$ and $L(v_5) = \{1, 4\}$, but then Fixer wins by Case 196. If the components of $A_S$ have vertex sets $\{v_2\}$, $\{v_0, v_1\}$ and $\{v_3, v_4\}$, then Fixer should swap 0 and 4 at $v_2$. This results in a position with lists $L(v_1) = \{0, 1\}$, $L(v_2) = \{0, 2, 3\}$, $L(v_3) = \{1, 2, 4\}$, $L(v_4) = \{2, 4\}$ and $L(v_5) = \{1, 4\}$, but then Fixer can edge-color the graph. If the components of $A_S$ have vertex sets $\{v_3\}$, $\{v_0, v_1\}$ and $\{v_2, v_4\}$, then Fixer should swap 0 and 4 at $v_3$. This results in a position with lists $L(v_1) = \{0, 1\}$, $L(v_2) = \{0, 2, 3\}$, $L(v_3) = \{0, 1, 2\}$, $L(v_4) = \{0, 2\}$ and $L(v_5) = \{1, 4\}$, but then Fixer can edge-color the graph. If the components of $A_S$ have vertex sets $\{v_4\}$, $\{v_0, v_1\}$ and $\{v_2, v_3\}$, then Fixer should swap 0 and 4 at $v_4$. This results in a position with lists $L(v_1) = \{0, 1\}$, $L(v_2) = \{0, 2, 3\}$, $L(v_3) = \{0, 1, 2\}$, $L(v_4) = \{2, 4\}$ and $L(v_5) = \{0, 1\}$, but then Fixer wins by Case 119. If the components of $A_S$ have vertex sets $\{v_2\}$, $\{v_0, v_3\}$ and $\{v_1, v_4\}$, then Fixer should swap 0 and 4 at $v_2$. This results in a position with lists $L(v_1) = \{0, 1\}$, $L(v_2) = \{0, 2, 3\}$, $L(v_3) = \{1, 2, 4\}$, $L(v_4) = \{2, 4\}$ and $L(v_5) = \{1, 4\}$, but then Fixer can edge-color the graph. If the components of $A_S$ have vertex sets $\{v_2\}$, $\{v_0, v_4\}$ and $\{v_1, v_3\}$, then Fixer should swap 0 and 4 at $v_2$. This results in a position with lists $L(v_1) = \{0, 1\}$, $L(v_2) = \{0, 2, 3\}$, $L(v_3) = \{1, 2, 4\}$, $L(v_4) = \{2, 4\}$ and $L(v_5) = \{1, 4\}$, but then Fixer can edge-color the graph. If the components of $A_S$ have vertex sets $\{v_3\}$, $\{v_0, v_2\}$ and $\{v_1, v_4\}$, then Fixer should swap 0 and 4 at $v_3$. This results in a position with lists $L(v_1) = \{0, 1\}$, $L(v_2) = \{0, 2, 3\}$, $L(v_3) = \{0, 1, 2\}$, $L(v_4) = \{0, 2\}$ and $L(v_5) = \{1, 4\}$, but then Fixer can edge-color the graph. If the components of $A_S$ have vertex sets $\{v_4\}$, $\{v_0, v_2\}$ and $\{v_1, v_3\}$, then Fixer should swap 0 and 4 at $v_4$. This results in a position with lists $L(v_1) = \{0, 1\}$, $L(v_2) = \{0, 2, 3\}$, $L(v_3) = \{0, 1, 2\}$, $L(v_4) = \{2, 4\}$ and $L(v_5) = \{0, 1\}$, but then Fixer wins by Case 119. If the components of $A_S$ have vertex sets $\{v_3\}$, $\{v_0, v_4\}$ and $\{v_1, v_2\}$, then Fixer should swap 0 and 4 at $v_3$. This results in a position with lists $L(v_1) = \{0, 1\}$, $L(v_2) = \{0, 2, 3\}$, $L(v_3) = \{0, 1, 2\}$, $L(v_4) = \{0, 2\}$ and $L(v_5) = \{1, 4\}$, but then Fixer can edge-color the graph. If the components of $A_S$ have vertex sets $\{v_4\}$, $\{v_0, v_3\}$ and $\{v_1, v_2\}$, then Fixer should swap 0 and 4 at $v_4$. This results in a position with lists $L(v_1) = \{0, 1\}$, $L(v_2) = \{0, 2, 3\}$, $L(v_3) = \{0, 1, 2\}$, $L(v_4) = \{2, 4\}$ and $L(v_5) = \{0, 1\}$, but then Fixer wins by Case 119. 

\noindent\textbf{Case 298.  }\textit{$L(v_1) = \{0, 1\}$, $L(v_2) = \{0, 2, 3\}$, $L(v_3) = \{0, 1, 2\}$, $L(v_4) = \{2, 4\}$ and $L(v_5) = \{2, 4\}$.}

Let $S$ and $A_S$ be as in Lemma \ref{MultiMoveCombination} using colors $0$ and $4$. If the components of $A_S$ have vertex sets $\{v_0\}$, $\{v_1, v_2\}$ and $\{v_3, v_4\}$, then Fixer should swap 0 and 4 at $v_4$ and $v_3$. This results in a position with lists $L(v_1) = \{0, 1\}$, $L(v_2) = \{0, 2, 3\}$, $L(v_3) = \{0, 1, 2\}$, $L(v_4) = \{0, 2\}$ and $L(v_5) = \{0, 2\}$, but then Fixer can edge-color the graph. If the components of $A_S$ have vertex sets $\{v_0\}$, $\{v_1, v_3\}$ and $\{v_2, v_4\}$, then Fixer should swap 0 and 4 at $v_4$ and $v_2$. This results in a position with lists $L(v_1) = \{0, 1\}$, $L(v_2) = \{0, 2, 3\}$, $L(v_3) = \{1, 2, 4\}$, $L(v_4) = \{2, 4\}$ and $L(v_5) = \{0, 2\}$, but then Fixer wins by Case 218. If the components of $A_S$ have vertex sets $\{v_0\}$, $\{v_1, v_4\}$ and $\{v_2, v_3\}$, then Fixer should swap 0 and 4 at $v_3$ and $v_2$. This results in a position with lists $L(v_1) = \{0, 1\}$, $L(v_2) = \{0, 2, 3\}$, $L(v_3) = \{1, 2, 4\}$, $L(v_4) = \{0, 2\}$ and $L(v_5) = \{2, 4\}$, but then Fixer wins by Case 197. If the components of $A_S$ have vertex sets $\{v_1\}$, $\{v_0, v_2\}$ and $\{v_3, v_4\}$, then Fixer should swap 0 and 4 at $v_4$ and $v_3$. This results in a position with lists $L(v_1) = \{0, 1\}$, $L(v_2) = \{0, 2, 3\}$, $L(v_3) = \{0, 1, 2\}$, $L(v_4) = \{0, 2\}$ and $L(v_5) = \{0, 2\}$, but then Fixer can edge-color the graph. If the components of $A_S$ have vertex sets $\{v_1\}$, $\{v_0, v_3\}$ and $\{v_2, v_4\}$, then Fixer should swap 0 and 4 at $v_4$ and $v_2$. This results in a position with lists $L(v_1) = \{0, 1\}$, $L(v_2) = \{0, 2, 3\}$, $L(v_3) = \{1, 2, 4\}$, $L(v_4) = \{2, 4\}$ and $L(v_5) = \{0, 2\}$, but then Fixer wins by Case 218. If the components of $A_S$ have vertex sets $\{v_1\}$, $\{v_0, v_4\}$ and $\{v_2, v_3\}$, then Fixer should swap 0 and 4 at $v_3$ and $v_2$. This results in a position with lists $L(v_1) = \{0, 1\}$, $L(v_2) = \{0, 2, 3\}$, $L(v_3) = \{1, 2, 4\}$, $L(v_4) = \{0, 2\}$ and $L(v_5) = \{2, 4\}$, but then Fixer wins by Case 197. If the components of $A_S$ have vertex sets $\{v_2\}$, $\{v_0, v_1\}$ and $\{v_3, v_4\}$, then Fixer should swap 0 and 4 at $v_2$. This results in a position with lists $L(v_1) = \{0, 1\}$, $L(v_2) = \{0, 2, 3\}$, $L(v_3) = \{1, 2, 4\}$, $L(v_4) = \{2, 4\}$ and $L(v_5) = \{2, 4\}$, but then Fixer wins by Case 220. If the components of $A_S$ have vertex sets $\{v_3\}$, $\{v_0, v_1\}$ and $\{v_2, v_4\}$, then Fixer should swap 0 and 4 at $v_3$. This results in a position with lists $L(v_1) = \{0, 1\}$, $L(v_2) = \{0, 2, 3\}$, $L(v_3) = \{0, 1, 2\}$, $L(v_4) = \{0, 2\}$ and $L(v_5) = \{2, 4\}$, but then Fixer wins by Case 106. If the components of $A_S$ have vertex sets $\{v_4\}$, $\{v_0, v_1\}$ and $\{v_2, v_3\}$, then Fixer should swap 0 and 4 at $v_4$. This results in a position with lists $L(v_1) = \{0, 1\}$, $L(v_2) = \{0, 2, 3\}$, $L(v_3) = \{0, 1, 2\}$, $L(v_4) = \{2, 4\}$ and $L(v_5) = \{0, 2\}$, but then Fixer wins by Case 120. If the components of $A_S$ have vertex sets $\{v_2\}$, $\{v_0, v_3\}$ and $\{v_1, v_4\}$, then Fixer should swap 0 and 4 at $v_2$. This results in a position with lists $L(v_1) = \{0, 1\}$, $L(v_2) = \{0, 2, 3\}$, $L(v_3) = \{1, 2, 4\}$, $L(v_4) = \{2, 4\}$ and $L(v_5) = \{2, 4\}$, but then Fixer wins by Case 220. If the components of $A_S$ have vertex sets $\{v_2\}$, $\{v_0, v_4\}$ and $\{v_1, v_3\}$, then Fixer should swap 0 and 4 at $v_2$. This results in a position with lists $L(v_1) = \{0, 1\}$, $L(v_2) = \{0, 2, 3\}$, $L(v_3) = \{1, 2, 4\}$, $L(v_4) = \{2, 4\}$ and $L(v_5) = \{2, 4\}$, but then Fixer wins by Case 220. If the components of $A_S$ have vertex sets $\{v_3\}$, $\{v_0, v_2\}$ and $\{v_1, v_4\}$, then Fixer should swap 0 and 4 at $v_3$. This results in a position with lists $L(v_1) = \{0, 1\}$, $L(v_2) = \{0, 2, 3\}$, $L(v_3) = \{0, 1, 2\}$, $L(v_4) = \{0, 2\}$ and $L(v_5) = \{2, 4\}$, but then Fixer wins by Case 106. If the components of $A_S$ have vertex sets $\{v_4\}$, $\{v_0, v_2\}$ and $\{v_1, v_3\}$, then Fixer should swap 0 and 4 at $v_4$. This results in a position with lists $L(v_1) = \{0, 1\}$, $L(v_2) = \{0, 2, 3\}$, $L(v_3) = \{0, 1, 2\}$, $L(v_4) = \{2, 4\}$ and $L(v_5) = \{0, 2\}$, but then Fixer wins by Case 120. If the components of $A_S$ have vertex sets $\{v_3\}$, $\{v_0, v_4\}$ and $\{v_1, v_2\}$, then Fixer should swap 0 and 4 at $v_3$. This results in a position with lists $L(v_1) = \{0, 1\}$, $L(v_2) = \{0, 2, 3\}$, $L(v_3) = \{0, 1, 2\}$, $L(v_4) = \{0, 2\}$ and $L(v_5) = \{2, 4\}$, but then Fixer wins by Case 106. If the components of $A_S$ have vertex sets $\{v_4\}$, $\{v_0, v_3\}$ and $\{v_1, v_2\}$, then Fixer should swap 0 and 4 at $v_4$. This results in a position with lists $L(v_1) = \{0, 1\}$, $L(v_2) = \{0, 2, 3\}$, $L(v_3) = \{0, 1, 2\}$, $L(v_4) = \{2, 4\}$ and $L(v_5) = \{0, 2\}$, but then Fixer wins by Case 120. 

\noindent\textbf{Case 299.  }\textit{$L(v_1) = \{0, 1\}$, $L(v_2) = \{0, 2, 3\}$, $L(v_3) = \{0, 1, 4\}$, $L(v_4) = \{0, 1\}$ and $L(v_5) = \{0, 1\}$.}

Let $S$ and $A_S$ be as in Lemma \ref{MultiMoveCombination} using colors $1$ and $2$. If the components of $A_S$ have vertex sets $\{v_0\}$, $\{v_1, v_2\}$ and $\{v_3, v_4\}$, then Fixer should swap 1 and 2 at $v_0$. This results in a position with lists $L(v_1) = \{0, 2\}$, $L(v_2) = \{0, 2, 3\}$, $L(v_3) = \{0, 1, 4\}$, $L(v_4) = \{0, 1\}$ and $L(v_5) = \{0, 1\}$, but then Fixer wins by Case 80. If the components of $A_S$ have vertex sets $\{v_0\}$, $\{v_1, v_3\}$ and $\{v_2, v_4\}$, then Fixer should swap 1 and 2 at $v_0$. This results in a position with lists $L(v_1) = \{0, 2\}$, $L(v_2) = \{0, 2, 3\}$, $L(v_3) = \{0, 1, 4\}$, $L(v_4) = \{0, 1\}$ and $L(v_5) = \{0, 1\}$, but then Fixer wins by Case 80. If the components of $A_S$ have vertex sets $\{v_0\}$, $\{v_1, v_4\}$ and $\{v_2, v_3\}$, then Fixer should swap 1 and 2 at $v_0$. This results in a position with lists $L(v_1) = \{0, 2\}$, $L(v_2) = \{0, 2, 3\}$, $L(v_3) = \{0, 1, 4\}$, $L(v_4) = \{0, 1\}$ and $L(v_5) = \{0, 1\}$, but then Fixer wins by Case 80. If the components of $A_S$ have vertex sets $\{v_1\}$, $\{v_0, v_2\}$ and $\{v_3, v_4\}$, then Fixer should swap 1 and 2 at $v_1$. This results in a position with lists $L(v_1) = \{0, 1\}$, $L(v_2) = \{0, 1, 3\}$, $L(v_3) = \{0, 1, 4\}$, $L(v_4) = \{0, 1\}$ and $L(v_5) = \{0, 1\}$, but then Fixer can edge-color the graph. If the components of $A_S$ have vertex sets $\{v_1\}$, $\{v_0, v_3\}$ and $\{v_2, v_4\}$, then Fixer should swap 1 and 2 at $v_1$. This results in a position with lists $L(v_1) = \{0, 1\}$, $L(v_2) = \{0, 1, 3\}$, $L(v_3) = \{0, 1, 4\}$, $L(v_4) = \{0, 1\}$ and $L(v_5) = \{0, 1\}$, but then Fixer can edge-color the graph. If the components of $A_S$ have vertex sets $\{v_1\}$, $\{v_0, v_4\}$ and $\{v_2, v_3\}$, then Fixer should swap 1 and 2 at $v_1$. This results in a position with lists $L(v_1) = \{0, 1\}$, $L(v_2) = \{0, 1, 3\}$, $L(v_3) = \{0, 1, 4\}$, $L(v_4) = \{0, 1\}$ and $L(v_5) = \{0, 1\}$, but then Fixer can edge-color the graph. If the components of $A_S$ have vertex sets $\{v_2\}$, $\{v_0, v_1\}$ and $\{v_3, v_4\}$, then Fixer should swap 1 and 2 at $v_2$. This results in a position with lists $L(v_1) = \{0, 1\}$, $L(v_2) = \{0, 2, 3\}$, $L(v_3) = \{0, 2, 4\}$, $L(v_4) = \{0, 1\}$ and $L(v_5) = \{0, 1\}$, but then Fixer wins by Case 160. If the components of $A_S$ have vertex sets $\{v_3\}$, $\{v_0, v_1\}$ and $\{v_2, v_4\}$, then Fixer should swap 1 and 2 at $v_1$ and $v_0$. This results in a position with lists $L(v_1) = \{0, 2\}$, $L(v_2) = \{0, 1, 3\}$, $L(v_3) = \{0, 1, 4\}$, $L(v_4) = \{0, 1\}$ and $L(v_5) = \{0, 1\}$, but then Fixer wins by Case 162. If the components of $A_S$ have vertex sets $\{v_4\}$, $\{v_0, v_1\}$ and $\{v_2, v_3\}$, then Fixer should swap 1 and 2 at $v_1$ and $v_0$. This results in a position with lists $L(v_1) = \{0, 2\}$, $L(v_2) = \{0, 1, 3\}$, $L(v_3) = \{0, 1, 4\}$, $L(v_4) = \{0, 1\}$ and $L(v_5) = \{0, 1\}$, but then Fixer wins by Case 162. If the components of $A_S$ have vertex sets $\{v_2\}$, $\{v_0, v_3\}$ and $\{v_1, v_4\}$, then Fixer should swap 1 and 2 at $v_2$. This results in a position with lists $L(v_1) = \{0, 1\}$, $L(v_2) = \{0, 2, 3\}$, $L(v_3) = \{0, 2, 4\}$, $L(v_4) = \{0, 1\}$ and $L(v_5) = \{0, 1\}$, but then Fixer wins by Case 160. If the components of $A_S$ have vertex sets $\{v_2\}$, $\{v_0, v_4\}$ and $\{v_1, v_3\}$, then Fixer should swap 1 and 2 at $v_2$. This results in a position with lists $L(v_1) = \{0, 1\}$, $L(v_2) = \{0, 2, 3\}$, $L(v_3) = \{0, 2, 4\}$, $L(v_4) = \{0, 1\}$ and $L(v_5) = \{0, 1\}$, but then Fixer wins by Case 160. If the components of $A_S$ have vertex sets $\{v_3\}$, $\{v_0, v_2\}$ and $\{v_1, v_4\}$, then Fixer should swap 1 and 2 at $v_2$ and $v_0$. This results in a position with lists $L(v_1) = \{0, 2\}$, $L(v_2) = \{0, 2, 3\}$, $L(v_3) = \{0, 2, 4\}$, $L(v_4) = \{0, 1\}$ and $L(v_5) = \{0, 1\}$, but then Fixer wins by Case 54. If the components of $A_S$ have vertex sets $\{v_4\}$, $\{v_0, v_2\}$ and $\{v_1, v_3\}$, then Fixer should swap 1 and 2 at $v_2$ and $v_0$. This results in a position with lists $L(v_1) = \{0, 2\}$, $L(v_2) = \{0, 2, 3\}$, $L(v_3) = \{0, 2, 4\}$, $L(v_4) = \{0, 1\}$ and $L(v_5) = \{0, 1\}$, but then Fixer wins by Case 54. If the components of $A_S$ have vertex sets $\{v_3\}$, $\{v_0, v_4\}$ and $\{v_1, v_2\}$, then Fixer should swap 1 and 2 at $v_4$ and $v_0$. This results in a position with lists $L(v_1) = \{0, 2\}$, $L(v_2) = \{0, 2, 3\}$, $L(v_3) = \{0, 1, 4\}$, $L(v_4) = \{0, 1\}$ and $L(v_5) = \{0, 2\}$, but then Fixer wins by Case 78. If the components of $A_S$ have vertex sets $\{v_4\}$, $\{v_0, v_3\}$ and $\{v_1, v_2\}$, then Fixer should swap 1 and 2 at $v_3$ and $v_0$. This results in a position with lists $L(v_1) = \{0, 2\}$, $L(v_2) = \{0, 2, 3\}$, $L(v_3) = \{0, 1, 4\}$, $L(v_4) = \{0, 2\}$ and $L(v_5) = \{0, 1\}$, but then Fixer wins by Case 69. 

\noindent\textbf{Case 300.  }\textit{$L(v_1) = \{0, 1\}$, $L(v_2) = \{0, 2, 3\}$, $L(v_3) = \{0, 1, 4\}$, $L(v_4) = \{0, 1\}$ and $L(v_5) = \{0, 2\}$.}

Let $S$ and $A_S$ be as in Lemma \ref{MultiMoveCombination} using colors $2$ and $4$. If the components of $A_S$ have vertex sets $\{v_1\}$ and $\{v_2, v_4\}$, then Fixer should swap 2 and 4 at $v_1$. This results in a position with lists $L(v_1) = \{0, 1\}$, $L(v_2) = \{0, 3, 4\}$, $L(v_3) = \{0, 1, 4\}$, $L(v_4) = \{0, 1\}$ and $L(v_5) = \{0, 2\}$, but then Fixer can edge-color the graph. If the components of $A_S$ have vertex sets $\{v_2\}$ and $\{v_1, v_4\}$, then Fixer should swap 2 and 4 at $v_2$. This results in a position with lists $L(v_1) = \{0, 1\}$, $L(v_2) = \{0, 2, 3\}$, $L(v_3) = \{0, 1, 2\}$, $L(v_4) = \{0, 1\}$ and $L(v_5) = \{0, 2\}$, but then Fixer can edge-color the graph. If the components of $A_S$ have vertex sets $\{v_4\}$ and $\{v_1, v_2\}$, then Fixer should swap 2 and 4 at $v_4$. This results in a position with lists $L(v_1) = \{0, 1\}$, $L(v_2) = \{0, 2, 3\}$, $L(v_3) = \{0, 1, 4\}$, $L(v_4) = \{0, 1\}$ and $L(v_5) = \{0, 4\}$, but then Fixer wins by Case 138. 

\noindent\textbf{Case 301.  }\textit{$L(v_1) = \{0, 1\}$, $L(v_2) = \{0, 2, 3\}$, $L(v_3) = \{0, 1, 4\}$, $L(v_4) = \{0, 1\}$ and $L(v_5) = \{1, 2\}$.}

Let $S$ and $A_S$ be as in Lemma \ref{MultiMoveCombination} using colors $2$ and $4$. If the components of $A_S$ have vertex sets $\{v_1\}$ and $\{v_2, v_4\}$, then Fixer should swap 2 and 4 at $v_1$. This results in a position with lists $L(v_1) = \{0, 1\}$, $L(v_2) = \{0, 3, 4\}$, $L(v_3) = \{0, 1, 4\}$, $L(v_4) = \{0, 1\}$ and $L(v_5) = \{1, 2\}$, but then Fixer can edge-color the graph. If the components of $A_S$ have vertex sets $\{v_2\}$ and $\{v_1, v_4\}$, then Fixer should swap 2 and 4 at $v_2$. This results in a position with lists $L(v_1) = \{0, 1\}$, $L(v_2) = \{0, 2, 3\}$, $L(v_3) = \{0, 1, 2\}$, $L(v_4) = \{0, 1\}$ and $L(v_5) = \{1, 2\}$, but then Fixer can edge-color the graph. If the components of $A_S$ have vertex sets $\{v_4\}$ and $\{v_1, v_2\}$, then Fixer should swap 2 and 4 at $v_4$. This results in a position with lists $L(v_1) = \{0, 1\}$, $L(v_2) = \{0, 2, 3\}$, $L(v_3) = \{0, 1, 4\}$, $L(v_4) = \{0, 1\}$ and $L(v_5) = \{1, 4\}$, but then Fixer wins by Case 139. 

\noindent\textbf{Case 302.  }\textit{$L(v_1) = \{0, 1\}$, $L(v_2) = \{0, 2, 3\}$, $L(v_3) = \{0, 1, 4\}$, $L(v_4) = \{0, 2\}$ and $L(v_5) = \{0, 1\}$.}

Let $S$ and $A_S$ be as in Lemma \ref{MultiMoveCombination} using colors $2$ and $4$. If the components of $A_S$ have vertex sets $\{v_1\}$ and $\{v_2, v_3\}$, then Fixer should swap 2 and 4 at $v_1$. This results in a position with lists $L(v_1) = \{0, 1\}$, $L(v_2) = \{0, 3, 4\}$, $L(v_3) = \{0, 1, 4\}$, $L(v_4) = \{0, 2\}$ and $L(v_5) = \{0, 1\}$, but then Fixer can edge-color the graph. If the components of $A_S$ have vertex sets $\{v_2\}$ and $\{v_1, v_3\}$, then Fixer should swap 2 and 4 at $v_2$. This results in a position with lists $L(v_1) = \{0, 1\}$, $L(v_2) = \{0, 2, 3\}$, $L(v_3) = \{0, 1, 2\}$, $L(v_4) = \{0, 2\}$ and $L(v_5) = \{0, 1\}$, but then Fixer can edge-color the graph. If the components of $A_S$ have vertex sets $\{v_3\}$ and $\{v_1, v_2\}$, then Fixer should swap 2 and 4 at $v_3$. This results in a position with lists $L(v_1) = \{0, 1\}$, $L(v_2) = \{0, 2, 3\}$, $L(v_3) = \{0, 1, 4\}$, $L(v_4) = \{0, 4\}$ and $L(v_5) = \{0, 1\}$, but then Fixer wins by Case 149. 

\noindent\textbf{Case 303.  }\textit{$L(v_1) = \{0, 1\}$, $L(v_2) = \{0, 2, 3\}$, $L(v_3) = \{0, 1, 4\}$, $L(v_4) = \{0, 2\}$ and $L(v_5) = \{0, 2\}$.}

Let $S$ and $A_S$ be as in Lemma \ref{MultiMoveCombination} using colors $1$ and $2$. If the components of $A_S$ have vertex sets $\{v_0\}$, $\{v_1, v_2\}$ and $\{v_3, v_4\}$, then Fixer should swap 1 and 2 at $v_2$ and $v_1$. This results in a position with lists $L(v_1) = \{0, 1\}$, $L(v_2) = \{0, 1, 3\}$, $L(v_3) = \{0, 2, 4\}$, $L(v_4) = \{0, 2\}$ and $L(v_5) = \{0, 2\}$, but then Fixer wins by Case 80. If the components of $A_S$ have vertex sets $\{v_0\}$, $\{v_1, v_3\}$ and $\{v_2, v_4\}$, then Fixer should swap 1 and 2 at $v_3$ and $v_1$. This results in a position with lists $L(v_1) = \{0, 1\}$, $L(v_2) = \{0, 1, 3\}$, $L(v_3) = \{0, 1, 4\}$, $L(v_4) = \{0, 1\}$ and $L(v_5) = \{0, 2\}$, but then Fixer wins by Case 50. If the components of $A_S$ have vertex sets $\{v_0\}$, $\{v_1, v_4\}$ and $\{v_2, v_3\}$, then Fixer should swap 1 and 2 at $v_4$ and $v_1$. This results in a position with lists $L(v_1) = \{0, 1\}$, $L(v_2) = \{0, 1, 3\}$, $L(v_3) = \{0, 1, 4\}$, $L(v_4) = \{0, 2\}$ and $L(v_5) = \{0, 1\}$, but then Fixer wins by Case 52. If the components of $A_S$ have vertex sets $\{v_1\}$, $\{v_0, v_2\}$ and $\{v_3, v_4\}$, then Fixer should swap 1 and 2 at $v_1$. This results in a position with lists $L(v_1) = \{0, 1\}$, $L(v_2) = \{0, 1, 3\}$, $L(v_3) = \{0, 1, 4\}$, $L(v_4) = \{0, 2\}$ and $L(v_5) = \{0, 2\}$, but then Fixer wins by Case 54. If the components of $A_S$ have vertex sets $\{v_1\}$, $\{v_0, v_3\}$ and $\{v_2, v_4\}$, then Fixer should swap 1 and 2 at $v_1$. This results in a position with lists $L(v_1) = \{0, 1\}$, $L(v_2) = \{0, 1, 3\}$, $L(v_3) = \{0, 1, 4\}$, $L(v_4) = \{0, 2\}$ and $L(v_5) = \{0, 2\}$, but then Fixer wins by Case 54. If the components of $A_S$ have vertex sets $\{v_1\}$, $\{v_0, v_4\}$ and $\{v_2, v_3\}$, then Fixer should swap 1 and 2 at $v_1$. This results in a position with lists $L(v_1) = \{0, 1\}$, $L(v_2) = \{0, 1, 3\}$, $L(v_3) = \{0, 1, 4\}$, $L(v_4) = \{0, 2\}$ and $L(v_5) = \{0, 2\}$, but then Fixer wins by Case 54. If the components of $A_S$ have vertex sets $\{v_2\}$, $\{v_0, v_1\}$ and $\{v_3, v_4\}$, then Fixer should swap 1 and 2 at $v_2$. This results in a position with lists $L(v_1) = \{0, 1\}$, $L(v_2) = \{0, 2, 3\}$, $L(v_3) = \{0, 2, 4\}$, $L(v_4) = \{0, 2\}$ and $L(v_5) = \{0, 2\}$, but then Fixer wins by Case 162. If the components of $A_S$ have vertex sets $\{v_3\}$, $\{v_0, v_1\}$ and $\{v_2, v_4\}$, then Fixer should swap 1 and 2 at $v_1$ and $v_0$. This results in a position with lists $L(v_1) = \{0, 2\}$, $L(v_2) = \{0, 1, 3\}$, $L(v_3) = \{0, 1, 4\}$, $L(v_4) = \{0, 2\}$ and $L(v_5) = \{0, 2\}$, but then Fixer wins by Case 160. If the components of $A_S$ have vertex sets $\{v_4\}$, $\{v_0, v_1\}$ and $\{v_2, v_3\}$, then Fixer should swap 1 and 2 at $v_1$ and $v_0$. This results in a position with lists $L(v_1) = \{0, 2\}$, $L(v_2) = \{0, 1, 3\}$, $L(v_3) = \{0, 1, 4\}$, $L(v_4) = \{0, 2\}$ and $L(v_5) = \{0, 2\}$, but then Fixer wins by Case 160. If the components of $A_S$ have vertex sets $\{v_2\}$, $\{v_0, v_3\}$ and $\{v_1, v_4\}$, then Fixer should swap 1 and 2 at $v_2$. This results in a position with lists $L(v_1) = \{0, 1\}$, $L(v_2) = \{0, 2, 3\}$, $L(v_3) = \{0, 2, 4\}$, $L(v_4) = \{0, 2\}$ and $L(v_5) = \{0, 2\}$, but then Fixer wins by Case 162. If the components of $A_S$ have vertex sets $\{v_2\}$, $\{v_0, v_4\}$ and $\{v_1, v_3\}$, then Fixer should swap 1 and 2 at $v_2$. This results in a position with lists $L(v_1) = \{0, 1\}$, $L(v_2) = \{0, 2, 3\}$, $L(v_3) = \{0, 2, 4\}$, $L(v_4) = \{0, 2\}$ and $L(v_5) = \{0, 2\}$, but then Fixer wins by Case 162. If the components of $A_S$ have vertex sets $\{v_3\}$, $\{v_0, v_2\}$ and $\{v_1, v_4\}$, then Fixer should swap 1 and 2 at $v_2$ and $v_0$. This results in a position with lists $L(v_1) = \{0, 2\}$, $L(v_2) = \{0, 2, 3\}$, $L(v_3) = \{0, 2, 4\}$, $L(v_4) = \{0, 2\}$ and $L(v_5) = \{0, 2\}$, but then Fixer can edge-color the graph. If the components of $A_S$ have vertex sets $\{v_4\}$, $\{v_0, v_2\}$ and $\{v_1, v_3\}$, then Fixer should swap 1 and 2 at $v_2$ and $v_0$. This results in a position with lists $L(v_1) = \{0, 2\}$, $L(v_2) = \{0, 2, 3\}$, $L(v_3) = \{0, 2, 4\}$, $L(v_4) = \{0, 2\}$ and $L(v_5) = \{0, 2\}$, but then Fixer can edge-color the graph. If the components of $A_S$ have vertex sets $\{v_3\}$, $\{v_0, v_4\}$ and $\{v_1, v_2\}$, then Fixer should swap 1 and 2 at $v_4$ and $v_0$. This results in a position with lists $L(v_1) = \{0, 2\}$, $L(v_2) = \{0, 2, 3\}$, $L(v_3) = \{0, 1, 4\}$, $L(v_4) = \{0, 2\}$ and $L(v_5) = \{0, 1\}$, but then Fixer wins by Case 69. If the components of $A_S$ have vertex sets $\{v_4\}$, $\{v_0, v_3\}$ and $\{v_1, v_2\}$, then Fixer should swap 1 and 2 at $v_3$ and $v_0$. This results in a position with lists $L(v_1) = \{0, 2\}$, $L(v_2) = \{0, 2, 3\}$, $L(v_3) = \{0, 1, 4\}$, $L(v_4) = \{0, 1\}$ and $L(v_5) = \{0, 2\}$, but then Fixer wins by Case 78. 

\noindent\textbf{Case 304.  }\textit{$L(v_1) = \{0, 1\}$, $L(v_2) = \{0, 2, 3\}$, $L(v_3) = \{0, 1, 4\}$, $L(v_4) = \{0, 2\}$ and $L(v_5) = \{1, 3\}$.}

Let $S$ and $A_S$ be as in Lemma \ref{MultiMoveCombination} using colors $2$ and $4$. If the components of $A_S$ have vertex sets $\{v_1\}$ and $\{v_2, v_3\}$, then Fixer should swap 2 and 4 at $v_1$. This results in a position with lists $L(v_1) = \{0, 1\}$, $L(v_2) = \{0, 3, 4\}$, $L(v_3) = \{0, 1, 4\}$, $L(v_4) = \{0, 2\}$ and $L(v_5) = \{1, 3\}$, but then Fixer can edge-color the graph. If the components of $A_S$ have vertex sets $\{v_2\}$ and $\{v_1, v_3\}$, then Fixer should swap 2 and 4 at $v_2$. This results in a position with lists $L(v_1) = \{0, 1\}$, $L(v_2) = \{0, 2, 3\}$, $L(v_3) = \{0, 1, 2\}$, $L(v_4) = \{0, 2\}$ and $L(v_5) = \{1, 3\}$, but then Fixer can edge-color the graph. If the components of $A_S$ have vertex sets $\{v_3\}$ and $\{v_1, v_2\}$, then Fixer should swap 2 and 4 at $v_3$. This results in a position with lists $L(v_1) = \{0, 1\}$, $L(v_2) = \{0, 2, 3\}$, $L(v_3) = \{0, 1, 4\}$, $L(v_4) = \{0, 4\}$ and $L(v_5) = \{1, 3\}$, but then Fixer wins by Case 151. 

\noindent\textbf{Case 305.  }\textit{$L(v_1) = \{0, 1\}$, $L(v_2) = \{0, 2, 3\}$, $L(v_3) = \{0, 1, 4\}$, $L(v_4) = \{1, 2\}$ and $L(v_5) = \{0, 1\}$.}

Let $S$ and $A_S$ be as in Lemma \ref{MultiMoveCombination} using colors $2$ and $4$. If the components of $A_S$ have vertex sets $\{v_1\}$ and $\{v_2, v_3\}$, then Fixer should swap 2 and 4 at $v_1$. This results in a position with lists $L(v_1) = \{0, 1\}$, $L(v_2) = \{0, 3, 4\}$, $L(v_3) = \{0, 1, 4\}$, $L(v_4) = \{1, 2\}$ and $L(v_5) = \{0, 1\}$, but then Fixer can edge-color the graph. If the components of $A_S$ have vertex sets $\{v_2\}$ and $\{v_1, v_3\}$, then Fixer should swap 2 and 4 at $v_2$. This results in a position with lists $L(v_1) = \{0, 1\}$, $L(v_2) = \{0, 2, 3\}$, $L(v_3) = \{0, 1, 2\}$, $L(v_4) = \{1, 2\}$ and $L(v_5) = \{0, 1\}$, but then Fixer can edge-color the graph. If the components of $A_S$ have vertex sets $\{v_3\}$ and $\{v_1, v_2\}$, then Fixer should swap 2 and 4 at $v_3$. This results in a position with lists $L(v_1) = \{0, 1\}$, $L(v_2) = \{0, 2, 3\}$, $L(v_3) = \{0, 1, 4\}$, $L(v_4) = \{1, 4\}$ and $L(v_5) = \{0, 1\}$, but then Fixer wins by Case 152. 

\noindent\textbf{Case 306.  }\textit{$L(v_1) = \{0, 1\}$, $L(v_2) = \{0, 2, 3\}$, $L(v_3) = \{0, 1, 4\}$, $L(v_4) = \{1, 2\}$ and $L(v_5) = \{0, 3\}$.}

Let $S$ and $A_S$ be as in Lemma \ref{MultiMoveCombination} using colors $2$ and $4$. If the components of $A_S$ have vertex sets $\{v_1\}$ and $\{v_2, v_3\}$, then Fixer should swap 2 and 4 at $v_1$. This results in a position with lists $L(v_1) = \{0, 1\}$, $L(v_2) = \{0, 3, 4\}$, $L(v_3) = \{0, 1, 4\}$, $L(v_4) = \{1, 2\}$ and $L(v_5) = \{0, 3\}$, but then Fixer can edge-color the graph. If the components of $A_S$ have vertex sets $\{v_2\}$ and $\{v_1, v_3\}$, then Fixer should swap 2 and 4 at $v_2$. This results in a position with lists $L(v_1) = \{0, 1\}$, $L(v_2) = \{0, 2, 3\}$, $L(v_3) = \{0, 1, 2\}$, $L(v_4) = \{1, 2\}$ and $L(v_5) = \{0, 3\}$, but then Fixer can edge-color the graph. If the components of $A_S$ have vertex sets $\{v_3\}$ and $\{v_1, v_2\}$, then Fixer should swap 2 and 4 at $v_3$. This results in a position with lists $L(v_1) = \{0, 1\}$, $L(v_2) = \{0, 2, 3\}$, $L(v_3) = \{0, 1, 4\}$, $L(v_4) = \{1, 4\}$ and $L(v_5) = \{0, 3\}$, but then Fixer wins by Case 153. 

\noindent\textbf{Case 307.  }\textit{$L(v_1) = \{0, 1\}$, $L(v_2) = \{0, 2, 3\}$, $L(v_3) = \{0, 1, 4\}$, $L(v_4) = \{0, 4\}$ and $L(v_5) = \{0, 4\}$.}

Let $S$ and $A_S$ be as in Lemma \ref{MultiMoveCombination} using colors $1$ and $2$. If the components of $A_S$ have vertex sets $\{v_0\}$ and $\{v_1, v_2\}$, then Fixer should swap 1 and 2 at $v_0$. This results in a position with lists $L(v_1) = \{0, 2\}$, $L(v_2) = \{0, 2, 3\}$, $L(v_3) = \{0, 1, 4\}$, $L(v_4) = \{0, 4\}$ and $L(v_5) = \{0, 4\}$, but then Fixer wins by Case 80. If the components of $A_S$ have vertex sets $\{v_1\}$ and $\{v_0, v_2\}$, then Fixer should swap 1 and 2 at $v_1$. This results in a position with lists $L(v_1) = \{0, 1\}$, $L(v_2) = \{0, 1, 3\}$, $L(v_3) = \{0, 1, 4\}$, $L(v_4) = \{0, 4\}$ and $L(v_5) = \{0, 4\}$, but then Fixer can edge-color the graph. If the components of $A_S$ have vertex sets $\{v_2\}$ and $\{v_0, v_1\}$, then Fixer should swap 1 and 2 at $v_2$. This results in a position with lists $L(v_1) = \{0, 1\}$, $L(v_2) = \{0, 2, 3\}$, $L(v_3) = \{0, 2, 4\}$, $L(v_4) = \{0, 4\}$ and $L(v_5) = \{0, 4\}$, but then Fixer can edge-color the graph. 

\noindent\textbf{Case 308.  }\textit{$L(v_1) = \{0, 1\}$, $L(v_2) = \{0, 2, 3\}$, $L(v_3) = \{0, 1, 4\}$, $L(v_4) = \{0, 4\}$ and $L(v_5) = \{2, 4\}$.}

Let $S$ and $A_S$ be as in Lemma \ref{MultiMoveCombination} using colors $1$ and $2$. If the components of $A_S$ have vertex sets $\{v_0, v_1\}$ and $\{v_2, v_4\}$, then Fixer should swap 1 and 2 at $v_1$ and $v_0$. This results in a position with lists $L(v_1) = \{0, 2\}$, $L(v_2) = \{0, 1, 3\}$, $L(v_3) = \{0, 1, 4\}$, $L(v_4) = \{0, 4\}$ and $L(v_5) = \{2, 4\}$, but then Fixer can edge-color the graph. If the components of $A_S$ have vertex sets $\{v_0, v_2\}$ and $\{v_1, v_4\}$, then Fixer should swap 1 and 2 at $v_2$ and $v_0$. This results in a position with lists $L(v_1) = \{0, 2\}$, $L(v_2) = \{0, 2, 3\}$, $L(v_3) = \{0, 2, 4\}$, $L(v_4) = \{0, 4\}$ and $L(v_5) = \{2, 4\}$, but then Fixer can edge-color the graph. If the components of $A_S$ have vertex sets $\{v_0, v_4\}$ and $\{v_1, v_2\}$, then Fixer should swap 1 and 2 at $v_4$ and $v_0$. This results in a position with lists $L(v_1) = \{0, 2\}$, $L(v_2) = \{0, 2, 3\}$, $L(v_3) = \{0, 1, 4\}$, $L(v_4) = \{0, 4\}$ and $L(v_5) = \{1, 4\}$, but then Fixer can edge-color the graph. 

\noindent\textbf{Case 309.  }\textit{$L(v_1) = \{0, 1\}$, $L(v_2) = \{0, 2, 3\}$, $L(v_3) = \{0, 1, 4\}$, $L(v_4) = \{1, 4\}$ and $L(v_5) = \{1, 4\}$.}

Let $S$ and $A_S$ be as in Lemma \ref{MultiMoveCombination} using colors $1$ and $2$. If the components of $A_S$ have vertex sets $\{v_0\}$, $\{v_1, v_2\}$ and $\{v_3, v_4\}$, then Fixer should swap 1 and 2 at $v_0$. This results in a position with lists $L(v_1) = \{0, 2\}$, $L(v_2) = \{0, 2, 3\}$, $L(v_3) = \{0, 1, 4\}$, $L(v_4) = \{1, 4\}$ and $L(v_5) = \{1, 4\}$, but then Fixer can edge-color the graph. If the components of $A_S$ have vertex sets $\{v_0\}$, $\{v_1, v_3\}$ and $\{v_2, v_4\}$, then Fixer should swap 1 and 2 at $v_0$. This results in a position with lists $L(v_1) = \{0, 2\}$, $L(v_2) = \{0, 2, 3\}$, $L(v_3) = \{0, 1, 4\}$, $L(v_4) = \{1, 4\}$ and $L(v_5) = \{1, 4\}$, but then Fixer can edge-color the graph. If the components of $A_S$ have vertex sets $\{v_0\}$, $\{v_1, v_4\}$ and $\{v_2, v_3\}$, then Fixer should swap 1 and 2 at $v_0$. This results in a position with lists $L(v_1) = \{0, 2\}$, $L(v_2) = \{0, 2, 3\}$, $L(v_3) = \{0, 1, 4\}$, $L(v_4) = \{1, 4\}$ and $L(v_5) = \{1, 4\}$, but then Fixer can edge-color the graph. If the components of $A_S$ have vertex sets $\{v_1\}$, $\{v_0, v_2\}$ and $\{v_3, v_4\}$, then Fixer should swap 1 and 2 at $v_1$. This results in a position with lists $L(v_1) = \{0, 1\}$, $L(v_2) = \{0, 1, 3\}$, $L(v_3) = \{0, 1, 4\}$, $L(v_4) = \{1, 4\}$ and $L(v_5) = \{1, 4\}$, but then Fixer can edge-color the graph. If the components of $A_S$ have vertex sets $\{v_1\}$, $\{v_0, v_3\}$ and $\{v_2, v_4\}$, then Fixer should swap 1 and 2 at $v_1$. This results in a position with lists $L(v_1) = \{0, 1\}$, $L(v_2) = \{0, 1, 3\}$, $L(v_3) = \{0, 1, 4\}$, $L(v_4) = \{1, 4\}$ and $L(v_5) = \{1, 4\}$, but then Fixer can edge-color the graph. If the components of $A_S$ have vertex sets $\{v_1\}$, $\{v_0, v_4\}$ and $\{v_2, v_3\}$, then Fixer should swap 1 and 2 at $v_1$. This results in a position with lists $L(v_1) = \{0, 1\}$, $L(v_2) = \{0, 1, 3\}$, $L(v_3) = \{0, 1, 4\}$, $L(v_4) = \{1, 4\}$ and $L(v_5) = \{1, 4\}$, but then Fixer can edge-color the graph. If the components of $A_S$ have vertex sets $\{v_2\}$, $\{v_0, v_1\}$ and $\{v_3, v_4\}$, then Fixer should swap 1 and 2 at $v_2$. This results in a position with lists $L(v_1) = \{0, 1\}$, $L(v_2) = \{0, 2, 3\}$, $L(v_3) = \{0, 2, 4\}$, $L(v_4) = \{1, 4\}$ and $L(v_5) = \{1, 4\}$, but then Fixer wins by Case 184. If the components of $A_S$ have vertex sets $\{v_3\}$, $\{v_0, v_1\}$ and $\{v_2, v_4\}$, then Fixer should swap 1 and 2 at $v_1$ and $v_0$. This results in a position with lists $L(v_1) = \{0, 2\}$, $L(v_2) = \{0, 1, 3\}$, $L(v_3) = \{0, 1, 4\}$, $L(v_4) = \{1, 4\}$ and $L(v_5) = \{1, 4\}$, but then Fixer wins by Case 187. If the components of $A_S$ have vertex sets $\{v_4\}$, $\{v_0, v_1\}$ and $\{v_2, v_3\}$, then Fixer should swap 1 and 2 at $v_1$ and $v_0$. This results in a position with lists $L(v_1) = \{0, 2\}$, $L(v_2) = \{0, 1, 3\}$, $L(v_3) = \{0, 1, 4\}$, $L(v_4) = \{1, 4\}$ and $L(v_5) = \{1, 4\}$, but then Fixer wins by Case 187. If the components of $A_S$ have vertex sets $\{v_2\}$, $\{v_0, v_3\}$ and $\{v_1, v_4\}$, then Fixer should swap 1 and 2 at $v_2$. This results in a position with lists $L(v_1) = \{0, 1\}$, $L(v_2) = \{0, 2, 3\}$, $L(v_3) = \{0, 2, 4\}$, $L(v_4) = \{1, 4\}$ and $L(v_5) = \{1, 4\}$, but then Fixer wins by Case 184. If the components of $A_S$ have vertex sets $\{v_2\}$, $\{v_0, v_4\}$ and $\{v_1, v_3\}$, then Fixer should swap 1 and 2 at $v_2$. This results in a position with lists $L(v_1) = \{0, 1\}$, $L(v_2) = \{0, 2, 3\}$, $L(v_3) = \{0, 2, 4\}$, $L(v_4) = \{1, 4\}$ and $L(v_5) = \{1, 4\}$, but then Fixer wins by Case 184. If the components of $A_S$ have vertex sets $\{v_3\}$, $\{v_0, v_2\}$ and $\{v_1, v_4\}$, then Fixer should swap 1 and 2 at $v_2$ and $v_0$. This results in a position with lists $L(v_1) = \{0, 2\}$, $L(v_2) = \{0, 2, 3\}$, $L(v_3) = \{0, 2, 4\}$, $L(v_4) = \{1, 4\}$ and $L(v_5) = \{1, 4\}$, but then Fixer wins by Case 55. If the components of $A_S$ have vertex sets $\{v_4\}$, $\{v_0, v_2\}$ and $\{v_1, v_3\}$, then Fixer should swap 1 and 2 at $v_2$ and $v_0$. This results in a position with lists $L(v_1) = \{0, 2\}$, $L(v_2) = \{0, 2, 3\}$, $L(v_3) = \{0, 2, 4\}$, $L(v_4) = \{1, 4\}$ and $L(v_5) = \{1, 4\}$, but then Fixer wins by Case 55. If the components of $A_S$ have vertex sets $\{v_3\}$, $\{v_0, v_4\}$ and $\{v_1, v_2\}$, then Fixer should swap 1 and 2 at $v_4$ and $v_0$. This results in a position with lists $L(v_1) = \{0, 2\}$, $L(v_2) = \{0, 2, 3\}$, $L(v_3) = \{0, 1, 4\}$, $L(v_4) = \{1, 4\}$ and $L(v_5) = \{2, 4\}$, but then Fixer can edge-color the graph. If the components of $A_S$ have vertex sets $\{v_4\}$, $\{v_0, v_3\}$ and $\{v_1, v_2\}$, then Fixer should swap 1 and 2 at $v_3$ and $v_0$. This results in a position with lists $L(v_1) = \{0, 2\}$, $L(v_2) = \{0, 2, 3\}$, $L(v_3) = \{0, 1, 4\}$, $L(v_4) = \{2, 4\}$ and $L(v_5) = \{1, 4\}$, but then Fixer can edge-color the graph. 

\noindent\textbf{Case 310.  }\textit{$L(v_1) = \{0, 1\}$, $L(v_2) = \{0, 2, 3\}$, $L(v_3) = \{0, 1, 4\}$, $L(v_4) = \{2, 4\}$ and $L(v_5) = \{0, 4\}$.}

Let $S$ and $A_S$ be as in Lemma \ref{MultiMoveCombination} using colors $1$ and $2$. If the components of $A_S$ have vertex sets $\{v_0, v_1\}$ and $\{v_2, v_3\}$, then Fixer should swap 1 and 2 at $v_1$ and $v_0$. This results in a position with lists $L(v_1) = \{0, 2\}$, $L(v_2) = \{0, 1, 3\}$, $L(v_3) = \{0, 1, 4\}$, $L(v_4) = \{2, 4\}$ and $L(v_5) = \{0, 4\}$, but then Fixer can edge-color the graph. If the components of $A_S$ have vertex sets $\{v_0, v_2\}$ and $\{v_1, v_3\}$, then Fixer should swap 1 and 2 at $v_2$ and $v_0$. This results in a position with lists $L(v_1) = \{0, 2\}$, $L(v_2) = \{0, 2, 3\}$, $L(v_3) = \{0, 2, 4\}$, $L(v_4) = \{2, 4\}$ and $L(v_5) = \{0, 4\}$, but then Fixer can edge-color the graph. If the components of $A_S$ have vertex sets $\{v_0, v_3\}$ and $\{v_1, v_2\}$, then Fixer should swap 1 and 2 at $v_3$ and $v_0$. This results in a position with lists $L(v_1) = \{0, 2\}$, $L(v_2) = \{0, 2, 3\}$, $L(v_3) = \{0, 1, 4\}$, $L(v_4) = \{1, 4\}$ and $L(v_5) = \{0, 4\}$, but then Fixer can edge-color the graph. 

\noindent\textbf{Case 311.  }\textit{$L(v_1) = \{0, 1\}$, $L(v_2) = \{0, 2, 3\}$, $L(v_3) = \{0, 1, 4\}$, $L(v_4) = \{2, 4\}$ and $L(v_5) = \{2, 4\}$.}

Let $S$ and $A_S$ be as in Lemma \ref{MultiMoveCombination} using colors $1$ and $2$. If the components of $A_S$ have vertex sets $\{v_0\}$, $\{v_1, v_2\}$ and $\{v_3, v_4\}$, then Fixer should swap 1 and 2 at $v_0$. This results in a position with lists $L(v_1) = \{0, 2\}$, $L(v_2) = \{0, 2, 3\}$, $L(v_3) = \{0, 1, 4\}$, $L(v_4) = \{2, 4\}$ and $L(v_5) = \{2, 4\}$, but then Fixer wins by Case 84. If the components of $A_S$ have vertex sets $\{v_0\}$, $\{v_1, v_3\}$ and $\{v_2, v_4\}$, then Fixer should swap 1 and 2 at $v_0$. This results in a position with lists $L(v_1) = \{0, 2\}$, $L(v_2) = \{0, 2, 3\}$, $L(v_3) = \{0, 1, 4\}$, $L(v_4) = \{2, 4\}$ and $L(v_5) = \{2, 4\}$, but then Fixer wins by Case 84. If the components of $A_S$ have vertex sets $\{v_0\}$, $\{v_1, v_4\}$ and $\{v_2, v_3\}$, then Fixer should swap 1 and 2 at $v_0$. This results in a position with lists $L(v_1) = \{0, 2\}$, $L(v_2) = \{0, 2, 3\}$, $L(v_3) = \{0, 1, 4\}$, $L(v_4) = \{2, 4\}$ and $L(v_5) = \{2, 4\}$, but then Fixer wins by Case 84. If the components of $A_S$ have vertex sets $\{v_1\}$, $\{v_0, v_2\}$ and $\{v_3, v_4\}$, then Fixer should swap 1 and 2 at $v_1$. This results in a position with lists $L(v_1) = \{0, 1\}$, $L(v_2) = \{0, 1, 3\}$, $L(v_3) = \{0, 1, 4\}$, $L(v_4) = \{2, 4\}$ and $L(v_5) = \{2, 4\}$, but then Fixer wins by Case 55. If the components of $A_S$ have vertex sets $\{v_1\}$, $\{v_0, v_3\}$ and $\{v_2, v_4\}$, then Fixer should swap 1 and 2 at $v_1$. This results in a position with lists $L(v_1) = \{0, 1\}$, $L(v_2) = \{0, 1, 3\}$, $L(v_3) = \{0, 1, 4\}$, $L(v_4) = \{2, 4\}$ and $L(v_5) = \{2, 4\}$, but then Fixer wins by Case 55. If the components of $A_S$ have vertex sets $\{v_1\}$, $\{v_0, v_4\}$ and $\{v_2, v_3\}$, then Fixer should swap 1 and 2 at $v_1$. This results in a position with lists $L(v_1) = \{0, 1\}$, $L(v_2) = \{0, 1, 3\}$, $L(v_3) = \{0, 1, 4\}$, $L(v_4) = \{2, 4\}$ and $L(v_5) = \{2, 4\}$, but then Fixer wins by Case 55. If the components of $A_S$ have vertex sets $\{v_2\}$, $\{v_0, v_1\}$ and $\{v_3, v_4\}$, then Fixer should swap 1 and 2 at $v_2$. This results in a position with lists $L(v_1) = \{0, 1\}$, $L(v_2) = \{0, 2, 3\}$, $L(v_3) = \{0, 2, 4\}$, $L(v_4) = \{2, 4\}$ and $L(v_5) = \{2, 4\}$, but then Fixer wins by Case 187. If the components of $A_S$ have vertex sets $\{v_3\}$, $\{v_0, v_1\}$ and $\{v_2, v_4\}$, then Fixer should swap 1 and 2 at $v_1$ and $v_0$. This results in a position with lists $L(v_1) = \{0, 2\}$, $L(v_2) = \{0, 1, 3\}$, $L(v_3) = \{0, 1, 4\}$, $L(v_4) = \{2, 4\}$ and $L(v_5) = \{2, 4\}$, but then Fixer wins by Case 184. If the components of $A_S$ have vertex sets $\{v_4\}$, $\{v_0, v_1\}$ and $\{v_2, v_3\}$, then Fixer should swap 1 and 2 at $v_1$ and $v_0$. This results in a position with lists $L(v_1) = \{0, 2\}$, $L(v_2) = \{0, 1, 3\}$, $L(v_3) = \{0, 1, 4\}$, $L(v_4) = \{2, 4\}$ and $L(v_5) = \{2, 4\}$, but then Fixer wins by Case 184. If the components of $A_S$ have vertex sets $\{v_2\}$, $\{v_0, v_3\}$ and $\{v_1, v_4\}$, then Fixer should swap 1 and 2 at $v_2$. This results in a position with lists $L(v_1) = \{0, 1\}$, $L(v_2) = \{0, 2, 3\}$, $L(v_3) = \{0, 2, 4\}$, $L(v_4) = \{2, 4\}$ and $L(v_5) = \{2, 4\}$, but then Fixer wins by Case 187. If the components of $A_S$ have vertex sets $\{v_2\}$, $\{v_0, v_4\}$ and $\{v_1, v_3\}$, then Fixer should swap 1 and 2 at $v_2$. This results in a position with lists $L(v_1) = \{0, 1\}$, $L(v_2) = \{0, 2, 3\}$, $L(v_3) = \{0, 2, 4\}$, $L(v_4) = \{2, 4\}$ and $L(v_5) = \{2, 4\}$, but then Fixer wins by Case 187. If the components of $A_S$ have vertex sets $\{v_3\}$, $\{v_0, v_2\}$ and $\{v_1, v_4\}$, then Fixer should swap 1 and 2 at $v_2$ and $v_0$. This results in a position with lists $L(v_1) = \{0, 2\}$, $L(v_2) = \{0, 2, 3\}$, $L(v_3) = \{0, 2, 4\}$, $L(v_4) = \{2, 4\}$ and $L(v_5) = \{2, 4\}$, but then Fixer can edge-color the graph. If the components of $A_S$ have vertex sets $\{v_4\}$, $\{v_0, v_2\}$ and $\{v_1, v_3\}$, then Fixer should swap 1 and 2 at $v_2$ and $v_0$. This results in a position with lists $L(v_1) = \{0, 2\}$, $L(v_2) = \{0, 2, 3\}$, $L(v_3) = \{0, 2, 4\}$, $L(v_4) = \{2, 4\}$ and $L(v_5) = \{2, 4\}$, but then Fixer can edge-color the graph. If the components of $A_S$ have vertex sets $\{v_3\}$, $\{v_0, v_4\}$ and $\{v_1, v_2\}$, then Fixer should swap 1 and 2 at $v_4$ and $v_0$. This results in a position with lists $L(v_1) = \{0, 2\}$, $L(v_2) = \{0, 2, 3\}$, $L(v_3) = \{0, 1, 4\}$, $L(v_4) = \{2, 4\}$ and $L(v_5) = \{1, 4\}$, but then Fixer can edge-color the graph. If the components of $A_S$ have vertex sets $\{v_4\}$, $\{v_0, v_3\}$ and $\{v_1, v_2\}$, then Fixer should swap 1 and 2 at $v_3$ and $v_0$. This results in a position with lists $L(v_1) = \{0, 2\}$, $L(v_2) = \{0, 2, 3\}$, $L(v_3) = \{0, 1, 4\}$, $L(v_4) = \{1, 4\}$ and $L(v_5) = \{2, 4\}$, but then Fixer can edge-color the graph. 

\noindent\textbf{Case 312.  }\textit{$L(v_1) = \{0, 1\}$, $L(v_2) = \{0, 2, 3\}$, $L(v_3) = \{0, 2, 4\}$, $L(v_4) = \{0, 1\}$ and $L(v_5) = \{0, 2\}$.}

Let $S$ and $A_S$ be as in Lemma \ref{MultiMoveCombination} using colors $1$ and $3$. If the components of $A_S$ have vertex sets $\{v_0\}$ and $\{v_1, v_3\}$, then Fixer should swap 1 and 3 at $v_0$. This results in a position with lists $L(v_1) = \{0, 3\}$, $L(v_2) = \{0, 2, 3\}$, $L(v_3) = \{0, 2, 4\}$, $L(v_4) = \{0, 1\}$ and $L(v_5) = \{0, 2\}$, but then Fixer wins by Case 60. If the components of $A_S$ have vertex sets $\{v_1\}$ and $\{v_0, v_3\}$, then Fixer should swap 1 and 3 at $v_1$. This results in a position with lists $L(v_1) = \{0, 1\}$, $L(v_2) = \{0, 1, 2\}$, $L(v_3) = \{0, 2, 4\}$, $L(v_4) = \{0, 1\}$ and $L(v_5) = \{0, 2\}$, but then Fixer can edge-color the graph. If the components of $A_S$ have vertex sets $\{v_3\}$ and $\{v_0, v_1\}$, then Fixer should swap 1 and 3 at $v_3$. This results in a position with lists $L(v_1) = \{0, 1\}$, $L(v_2) = \{0, 2, 3\}$, $L(v_3) = \{0, 2, 4\}$, $L(v_4) = \{0, 3\}$ and $L(v_5) = \{0, 2\}$, but then Fixer wins by Case 169. 

\noindent\textbf{Case 313.  }\textit{$L(v_1) = \{0, 1\}$, $L(v_2) = \{0, 2, 3\}$, $L(v_3) = \{0, 2, 4\}$, $L(v_4) = \{0, 1\}$ and $L(v_5) = \{1, 2\}$.}

Let $S$ and $A_S$ be as in Lemma \ref{MultiMoveCombination} using colors $1$ and $3$. If the components of $A_S$ have vertex sets $\{v_0, v_1\}$ and $\{v_3, v_4\}$, then Fixer should swap 1 and 3 at $v_1$ and $v_0$. This results in a position with lists $L(v_1) = \{0, 3\}$, $L(v_2) = \{0, 1, 2\}$, $L(v_3) = \{0, 2, 4\}$, $L(v_4) = \{0, 1\}$ and $L(v_5) = \{1, 2\}$, but then Fixer wins by Case 171. If the components of $A_S$ have vertex sets $\{v_0, v_3\}$ and $\{v_1, v_4\}$, then Fixer should swap 1 and 3 at $v_3$ and $v_0$. This results in a position with lists $L(v_1) = \{0, 3\}$, $L(v_2) = \{0, 2, 3\}$, $L(v_3) = \{0, 2, 4\}$, $L(v_4) = \{0, 3\}$ and $L(v_5) = \{1, 2\}$, but then Fixer wins by Case 56. If the components of $A_S$ have vertex sets $\{v_0, v_4\}$ and $\{v_1, v_3\}$, then Fixer should swap 1 and 3 at $v_4$ and $v_0$. This results in a position with lists $L(v_1) = \{0, 3\}$, $L(v_2) = \{0, 2, 3\}$, $L(v_3) = \{0, 2, 4\}$, $L(v_4) = \{0, 1\}$ and $L(v_5) = \{2, 3\}$, but then Fixer wins by Case 61. 

\noindent\textbf{Case 314.  }\textit{$L(v_1) = \{0, 1\}$, $L(v_2) = \{0, 2, 3\}$, $L(v_3) = \{0, 2, 4\}$, $L(v_4) = \{0, 2\}$ and $L(v_5) = \{0, 1\}$.}

Let $S$ and $A_S$ be as in Lemma \ref{MultiMoveCombination} using colors $1$ and $3$. If the components of $A_S$ have vertex sets $\{v_0\}$ and $\{v_1, v_4\}$, then Fixer should swap 1 and 3 at $v_0$. This results in a position with lists $L(v_1) = \{0, 3\}$, $L(v_2) = \{0, 2, 3\}$, $L(v_3) = \{0, 2, 4\}$, $L(v_4) = \{0, 2\}$ and $L(v_5) = \{0, 1\}$, but then Fixer wins by Case 57. If the components of $A_S$ have vertex sets $\{v_1\}$ and $\{v_0, v_4\}$, then Fixer should swap 1 and 3 at $v_1$. This results in a position with lists $L(v_1) = \{0, 1\}$, $L(v_2) = \{0, 1, 2\}$, $L(v_3) = \{0, 2, 4\}$, $L(v_4) = \{0, 2\}$ and $L(v_5) = \{0, 1\}$, but then Fixer can edge-color the graph. If the components of $A_S$ have vertex sets $\{v_4\}$ and $\{v_0, v_1\}$, then Fixer should swap 1 and 3 at $v_4$. This results in a position with lists $L(v_1) = \{0, 1\}$, $L(v_2) = \{0, 2, 3\}$, $L(v_3) = \{0, 2, 4\}$, $L(v_4) = \{0, 2\}$ and $L(v_5) = \{0, 3\}$, but then Fixer wins by Case 163. 

\noindent\textbf{Case 315.  }\textit{$L(v_1) = \{0, 1\}$, $L(v_2) = \{0, 2, 3\}$, $L(v_3) = \{0, 2, 4\}$, $L(v_4) = \{0, 2\}$ and $L(v_5) = \{1, 2\}$.}

Let $S$ and $A_S$ be as in Lemma \ref{MultiMoveCombination} using colors $1$ and $3$. If the components of $A_S$ have vertex sets $\{v_0\}$ and $\{v_1, v_4\}$, then Fixer should swap 1 and 3 at $v_0$. This results in a position with lists $L(v_1) = \{0, 3\}$, $L(v_2) = \{0, 2, 3\}$, $L(v_3) = \{0, 2, 4\}$, $L(v_4) = \{0, 2\}$ and $L(v_5) = \{1, 2\}$, but then Fixer wins by Case 58. If the components of $A_S$ have vertex sets $\{v_1\}$ and $\{v_0, v_4\}$, then Fixer should swap 1 and 3 at $v_1$. This results in a position with lists $L(v_1) = \{0, 1\}$, $L(v_2) = \{0, 1, 2\}$, $L(v_3) = \{0, 2, 4\}$, $L(v_4) = \{0, 2\}$ and $L(v_5) = \{1, 2\}$, but then Fixer can edge-color the graph. If the components of $A_S$ have vertex sets $\{v_4\}$ and $\{v_0, v_1\}$, then Fixer should swap 1 and 3 at $v_4$. This results in a position with lists $L(v_1) = \{0, 1\}$, $L(v_2) = \{0, 2, 3\}$, $L(v_3) = \{0, 2, 4\}$, $L(v_4) = \{0, 2\}$ and $L(v_5) = \{2, 3\}$, but then Fixer wins by Case 164. 

\noindent\textbf{Case 316.  }\textit{$L(v_1) = \{0, 1\}$, $L(v_2) = \{0, 2, 3\}$, $L(v_3) = \{0, 2, 4\}$, $L(v_4) = \{1, 2\}$ and $L(v_5) = \{0, 1\}$.}

Let $S$ and $A_S$ be as in Lemma \ref{MultiMoveCombination} using colors $1$ and $3$. If the components of $A_S$ have vertex sets $\{v_0, v_1\}$ and $\{v_3, v_4\}$, then Fixer should swap 1 and 3 at $v_1$ and $v_0$. This results in a position with lists $L(v_1) = \{0, 3\}$, $L(v_2) = \{0, 1, 2\}$, $L(v_3) = \{0, 2, 4\}$, $L(v_4) = \{1, 2\}$ and $L(v_5) = \{0, 1\}$, but then Fixer wins by Case 174. If the components of $A_S$ have vertex sets $\{v_0, v_3\}$ and $\{v_1, v_4\}$, then Fixer should swap 1 and 3 at $v_3$ and $v_0$. This results in a position with lists $L(v_1) = \{0, 3\}$, $L(v_2) = \{0, 2, 3\}$, $L(v_3) = \{0, 2, 4\}$, $L(v_4) = \{2, 3\}$ and $L(v_5) = \{0, 1\}$, but then Fixer wins by Case 59. If the components of $A_S$ have vertex sets $\{v_0, v_4\}$ and $\{v_1, v_3\}$, then Fixer should swap 1 and 3 at $v_4$ and $v_0$. This results in a position with lists $L(v_1) = \{0, 3\}$, $L(v_2) = \{0, 2, 3\}$, $L(v_3) = \{0, 2, 4\}$, $L(v_4) = \{1, 2\}$ and $L(v_5) = \{0, 3\}$, but then Fixer wins by Case 64. 

\noindent\textbf{Case 317.  }\textit{$L(v_1) = \{0, 1\}$, $L(v_2) = \{0, 2, 3\}$, $L(v_3) = \{0, 2, 4\}$, $L(v_4) = \{1, 2\}$ and $L(v_5) = \{0, 2\}$.}

Let $S$ and $A_S$ be as in Lemma \ref{MultiMoveCombination} using colors $1$ and $3$. If the components of $A_S$ have vertex sets $\{v_0\}$ and $\{v_1, v_3\}$, then Fixer should swap 1 and 3 at $v_0$. This results in a position with lists $L(v_1) = \{0, 3\}$, $L(v_2) = \{0, 2, 3\}$, $L(v_3) = \{0, 2, 4\}$, $L(v_4) = \{1, 2\}$ and $L(v_5) = \{0, 2\}$, but then Fixer wins by Case 65. If the components of $A_S$ have vertex sets $\{v_1\}$ and $\{v_0, v_3\}$, then Fixer should swap 1 and 3 at $v_1$. This results in a position with lists $L(v_1) = \{0, 1\}$, $L(v_2) = \{0, 1, 2\}$, $L(v_3) = \{0, 2, 4\}$, $L(v_4) = \{1, 2\}$ and $L(v_5) = \{0, 2\}$, but then Fixer can edge-color the graph. If the components of $A_S$ have vertex sets $\{v_3\}$ and $\{v_0, v_1\}$, then Fixer should swap 1 and 3 at $v_3$. This results in a position with lists $L(v_1) = \{0, 1\}$, $L(v_2) = \{0, 2, 3\}$, $L(v_3) = \{0, 2, 4\}$, $L(v_4) = \{2, 3\}$ and $L(v_5) = \{0, 2\}$, but then Fixer wins by Case 173. 

\noindent\textbf{Case 318.  }\textit{$L(v_1) = \{0, 1\}$, $L(v_2) = \{0, 2, 3\}$, $L(v_3) = \{0, 2, 4\}$, $L(v_4) = \{1, 2\}$ and $L(v_5) = \{1, 2\}$.}

Let $S$ and $A_S$ be as in Lemma \ref{MultiMoveCombination} using colors $1$ and $4$. If the components of $A_S$ have vertex sets $\{v_0, v_2\}$ and $\{v_3, v_4\}$, then Fixer should swap 1 and 4 at $v_2$ and $v_0$. This results in a position with lists $L(v_1) = \{0, 4\}$, $L(v_2) = \{0, 2, 3\}$, $L(v_3) = \{0, 1, 2\}$, $L(v_4) = \{1, 2\}$ and $L(v_5) = \{1, 2\}$, but then Fixer wins by Case 187. If the components of $A_S$ have vertex sets $\{v_0, v_3\}$ and $\{v_2, v_4\}$, then Fixer should swap 1 and 4 at $v_3$ and $v_0$. This results in a position with lists $L(v_1) = \{0, 4\}$, $L(v_2) = \{0, 2, 3\}$, $L(v_3) = \{0, 2, 4\}$, $L(v_4) = \{2, 4\}$ and $L(v_5) = \{1, 2\}$, but then Fixer wins by Case 108. If the components of $A_S$ have vertex sets $\{v_0, v_4\}$ and $\{v_2, v_3\}$, then Fixer should swap 1 and 4 at $v_4$ and $v_0$. This results in a position with lists $L(v_1) = \{0, 4\}$, $L(v_2) = \{0, 2, 3\}$, $L(v_3) = \{0, 2, 4\}$, $L(v_4) = \{1, 2\}$ and $L(v_5) = \{2, 4\}$, but then Fixer wins by Case 121. 

\noindent\textbf{Case 319.  }\textit{$L(v_1) = \{0, 1\}$, $L(v_2) = \{0, 2, 3\}$, $L(v_3) = \{0, 2, 4\}$, $L(v_4) = \{1, 2\}$ and $L(v_5) = \{0, 3\}$.}

Let $S$ and $A_S$ be as in Lemma \ref{MultiMoveCombination} using colors $1$ and $3$. If the components of $A_S$ have vertex sets $\{v_0, v_1\}$ and $\{v_3, v_4\}$, then Fixer should swap 1 and 3 at $v_1$ and $v_0$. This results in a position with lists $L(v_1) = \{0, 3\}$, $L(v_2) = \{0, 1, 2\}$, $L(v_3) = \{0, 2, 4\}$, $L(v_4) = \{1, 2\}$ and $L(v_5) = \{0, 3\}$, but then Fixer wins by Case 172. If the components of $A_S$ have vertex sets $\{v_0, v_3\}$ and $\{v_1, v_4\}$, then Fixer should swap 1 and 3 at $v_3$ and $v_0$. This results in a position with lists $L(v_1) = \{0, 3\}$, $L(v_2) = \{0, 2, 3\}$, $L(v_3) = \{0, 2, 4\}$, $L(v_4) = \{2, 3\}$ and $L(v_5) = \{0, 3\}$, but then Fixer can edge-color the graph. If the components of $A_S$ have vertex sets $\{v_0, v_4\}$ and $\{v_1, v_3\}$, then Fixer should swap 1 and 3 at $v_4$ and $v_0$. This results in a position with lists $L(v_1) = \{0, 3\}$, $L(v_2) = \{0, 2, 3\}$, $L(v_3) = \{0, 2, 4\}$, $L(v_4) = \{1, 2\}$ and $L(v_5) = \{0, 1\}$, but then Fixer wins by Case 66. 

\noindent\textbf{Case 320.  }\textit{$L(v_1) = \{0, 1\}$, $L(v_2) = \{0, 2, 3\}$, $L(v_3) = \{0, 2, 4\}$, $L(v_4) = \{0, 3\}$ and $L(v_5) = \{1, 2\}$.}

Let $S$ and $A_S$ be as in Lemma \ref{MultiMoveCombination} using colors $1$ and $3$. If the components of $A_S$ have vertex sets $\{v_0, v_1\}$ and $\{v_3, v_4\}$, then Fixer should swap 1 and 3 at $v_1$ and $v_0$. This results in a position with lists $L(v_1) = \{0, 3\}$, $L(v_2) = \{0, 1, 2\}$, $L(v_3) = \{0, 2, 4\}$, $L(v_4) = \{0, 3\}$ and $L(v_5) = \{1, 2\}$, but then Fixer wins by Case 161. If the components of $A_S$ have vertex sets $\{v_0, v_3\}$ and $\{v_1, v_4\}$, then Fixer should swap 1 and 3 at $v_3$ and $v_0$. This results in a position with lists $L(v_1) = \{0, 3\}$, $L(v_2) = \{0, 2, 3\}$, $L(v_3) = \{0, 2, 4\}$, $L(v_4) = \{0, 1\}$ and $L(v_5) = \{1, 2\}$, but then Fixer wins by Case 63. If the components of $A_S$ have vertex sets $\{v_0, v_4\}$ and $\{v_1, v_3\}$, then Fixer should swap 1 and 3 at $v_4$ and $v_0$. This results in a position with lists $L(v_1) = \{0, 3\}$, $L(v_2) = \{0, 2, 3\}$, $L(v_3) = \{0, 2, 4\}$, $L(v_4) = \{0, 3\}$ and $L(v_5) = \{2, 3\}$, but then Fixer can edge-color the graph. 

\noindent\textbf{Case 321.  }\textit{$L(v_1) = \{0, 1\}$, $L(v_2) = \{0, 2, 3\}$, $L(v_3) = \{0, 2, 4\}$, $L(v_4) = \{1, 4\}$ and $L(v_5) = \{2, 4\}$.}

Let $S$ and $A_S$ be as in Lemma \ref{MultiMoveCombination} using colors $1$ and $3$. If the components of $A_S$ have vertex sets $\{v_0\}$ and $\{v_1, v_3\}$, then Fixer should swap 1 and 3 at $v_0$. This results in a position with lists $L(v_1) = \{0, 3\}$, $L(v_2) = \{0, 2, 3\}$, $L(v_3) = \{0, 2, 4\}$, $L(v_4) = \{1, 4\}$ and $L(v_5) = \{2, 4\}$, but then Fixer can edge-color the graph. If the components of $A_S$ have vertex sets $\{v_1\}$ and $\{v_0, v_3\}$, then Fixer should swap 1 and 3 at $v_1$. This results in a position with lists $L(v_1) = \{0, 1\}$, $L(v_2) = \{0, 1, 2\}$, $L(v_3) = \{0, 2, 4\}$, $L(v_4) = \{1, 4\}$ and $L(v_5) = \{2, 4\}$, but then Fixer can edge-color the graph. If the components of $A_S$ have vertex sets $\{v_3\}$ and $\{v_0, v_1\}$, then Fixer should swap 1 and 3 at $v_3$. This results in a position with lists $L(v_1) = \{0, 1\}$, $L(v_2) = \{0, 2, 3\}$, $L(v_3) = \{0, 2, 4\}$, $L(v_4) = \{3, 4\}$ and $L(v_5) = \{2, 4\}$, but then Fixer wins by Case 191. 

\noindent\textbf{Case 322.  }\textit{$L(v_1) = \{0, 1\}$, $L(v_2) = \{0, 2, 3\}$, $L(v_3) = \{0, 2, 4\}$, $L(v_4) = \{2, 4\}$ and $L(v_5) = \{1, 4\}$.}

Let $S$ and $A_S$ be as in Lemma \ref{MultiMoveCombination} using colors $1$ and $3$. If the components of $A_S$ have vertex sets $\{v_0\}$ and $\{v_1, v_4\}$, then Fixer should swap 1 and 3 at $v_0$. This results in a position with lists $L(v_1) = \{0, 3\}$, $L(v_2) = \{0, 2, 3\}$, $L(v_3) = \{0, 2, 4\}$, $L(v_4) = \{2, 4\}$ and $L(v_5) = \{1, 4\}$, but then Fixer can edge-color the graph. If the components of $A_S$ have vertex sets $\{v_1\}$ and $\{v_0, v_4\}$, then Fixer should swap 1 and 3 at $v_1$. This results in a position with lists $L(v_1) = \{0, 1\}$, $L(v_2) = \{0, 1, 2\}$, $L(v_3) = \{0, 2, 4\}$, $L(v_4) = \{2, 4\}$ and $L(v_5) = \{1, 4\}$, but then Fixer can edge-color the graph. If the components of $A_S$ have vertex sets $\{v_4\}$ and $\{v_0, v_1\}$, then Fixer should swap 1 and 3 at $v_4$. This results in a position with lists $L(v_1) = \{0, 1\}$, $L(v_2) = \{0, 2, 3\}$, $L(v_3) = \{0, 2, 4\}$, $L(v_4) = \{2, 4\}$ and $L(v_5) = \{3, 4\}$, but then Fixer wins by Case 188. 

\noindent\textbf{Case 323.  }\textit{$L(v_1) = \{0, 1\}$, $L(v_2) = \{0, 2, 3\}$, $L(v_3) = \{1, 2, 4\}$, $L(v_4) = \{0, 1\}$ and $L(v_5) = \{0, 2\}$.}

Let $S$ and $A_S$ be as in Lemma \ref{MultiMoveCombination} using colors $0$ and $4$. If the components of $A_S$ have vertex sets $\{v_0\}$, $\{v_1, v_2\}$ and $\{v_3, v_4\}$, then Fixer should swap 0 and 4 at $v_4$ and $v_3$. This results in a position with lists $L(v_1) = \{0, 1\}$, $L(v_2) = \{0, 2, 3\}$, $L(v_3) = \{1, 2, 4\}$, $L(v_4) = \{1, 4\}$ and $L(v_5) = \{2, 4\}$, but then Fixer can edge-color the graph. If the components of $A_S$ have vertex sets $\{v_0\}$, $\{v_1, v_3\}$ and $\{v_2, v_4\}$, then Fixer should swap 0 and 4 at $v_4$ and $v_2$. This results in a position with lists $L(v_1) = \{0, 1\}$, $L(v_2) = \{0, 2, 3\}$, $L(v_3) = \{0, 1, 2\}$, $L(v_4) = \{0, 1\}$ and $L(v_5) = \{2, 4\}$, but then Fixer wins by Case 104. If the components of $A_S$ have vertex sets $\{v_0\}$, $\{v_1, v_4\}$ and $\{v_2, v_3\}$, then Fixer should swap 0 and 4 at $v_3$ and $v_2$. This results in a position with lists $L(v_1) = \{0, 1\}$, $L(v_2) = \{0, 2, 3\}$, $L(v_3) = \{0, 1, 2\}$, $L(v_4) = \{1, 4\}$ and $L(v_5) = \{0, 2\}$, but then Fixer can edge-color the graph. If the components of $A_S$ have vertex sets $\{v_1\}$, $\{v_0, v_2\}$ and $\{v_3, v_4\}$, then Fixer should swap 0 and 4 at $v_4$ and $v_3$. This results in a position with lists $L(v_1) = \{0, 1\}$, $L(v_2) = \{0, 2, 3\}$, $L(v_3) = \{1, 2, 4\}$, $L(v_4) = \{1, 4\}$ and $L(v_5) = \{2, 4\}$, but then Fixer can edge-color the graph. If the components of $A_S$ have vertex sets $\{v_1\}$, $\{v_0, v_3\}$ and $\{v_2, v_4\}$, then Fixer should swap 0 and 4 at $v_4$ and $v_2$. This results in a position with lists $L(v_1) = \{0, 1\}$, $L(v_2) = \{0, 2, 3\}$, $L(v_3) = \{0, 1, 2\}$, $L(v_4) = \{0, 1\}$ and $L(v_5) = \{2, 4\}$, but then Fixer wins by Case 104. If the components of $A_S$ have vertex sets $\{v_1\}$, $\{v_0, v_4\}$ and $\{v_2, v_3\}$, then Fixer should swap 0 and 4 at $v_3$ and $v_2$. This results in a position with lists $L(v_1) = \{0, 1\}$, $L(v_2) = \{0, 2, 3\}$, $L(v_3) = \{0, 1, 2\}$, $L(v_4) = \{1, 4\}$ and $L(v_5) = \{0, 2\}$, but then Fixer can edge-color the graph. If the components of $A_S$ have vertex sets $\{v_2\}$, $\{v_0, v_1\}$ and $\{v_3, v_4\}$, then Fixer should swap 0 and 4 at $v_2$. This results in a position with lists $L(v_1) = \{0, 1\}$, $L(v_2) = \{0, 2, 3\}$, $L(v_3) = \{0, 1, 2\}$, $L(v_4) = \{0, 1\}$ and $L(v_5) = \{0, 2\}$, but then Fixer can edge-color the graph. If the components of $A_S$ have vertex sets $\{v_3\}$, $\{v_0, v_1\}$ and $\{v_2, v_4\}$, then Fixer should swap 0 and 4 at $v_3$. This results in a position with lists $L(v_1) = \{0, 1\}$, $L(v_2) = \{0, 2, 3\}$, $L(v_3) = \{1, 2, 4\}$, $L(v_4) = \{1, 4\}$ and $L(v_5) = \{0, 2\}$, but then Fixer wins by Case 216. If the components of $A_S$ have vertex sets $\{v_4\}$, $\{v_0, v_1\}$ and $\{v_2, v_3\}$, then Fixer should swap 0 and 4 at $v_4$. This results in a position with lists $L(v_1) = \{0, 1\}$, $L(v_2) = \{0, 2, 3\}$, $L(v_3) = \{1, 2, 4\}$, $L(v_4) = \{0, 1\}$ and $L(v_5) = \{2, 4\}$, but then Fixer can edge-color the graph. If the components of $A_S$ have vertex sets $\{v_2\}$, $\{v_0, v_3\}$ and $\{v_1, v_4\}$, then Fixer should swap 0 and 4 at $v_2$. This results in a position with lists $L(v_1) = \{0, 1\}$, $L(v_2) = \{0, 2, 3\}$, $L(v_3) = \{0, 1, 2\}$, $L(v_4) = \{0, 1\}$ and $L(v_5) = \{0, 2\}$, but then Fixer can edge-color the graph. If the components of $A_S$ have vertex sets $\{v_2\}$, $\{v_0, v_4\}$ and $\{v_1, v_3\}$, then Fixer should swap 0 and 4 at $v_2$. This results in a position with lists $L(v_1) = \{0, 1\}$, $L(v_2) = \{0, 2, 3\}$, $L(v_3) = \{0, 1, 2\}$, $L(v_4) = \{0, 1\}$ and $L(v_5) = \{0, 2\}$, but then Fixer can edge-color the graph. If the components of $A_S$ have vertex sets $\{v_3\}$, $\{v_0, v_2\}$ and $\{v_1, v_4\}$, then Fixer should swap 0 and 4 at $v_3$. This results in a position with lists $L(v_1) = \{0, 1\}$, $L(v_2) = \{0, 2, 3\}$, $L(v_3) = \{1, 2, 4\}$, $L(v_4) = \{1, 4\}$ and $L(v_5) = \{0, 2\}$, but then Fixer wins by Case 216. If the components of $A_S$ have vertex sets $\{v_4\}$, $\{v_0, v_2\}$ and $\{v_1, v_3\}$, then Fixer should swap 0 and 4 at $v_4$. This results in a position with lists $L(v_1) = \{0, 1\}$, $L(v_2) = \{0, 2, 3\}$, $L(v_3) = \{1, 2, 4\}$, $L(v_4) = \{0, 1\}$ and $L(v_5) = \{2, 4\}$, but then Fixer can edge-color the graph. If the components of $A_S$ have vertex sets $\{v_3\}$, $\{v_0, v_4\}$ and $\{v_1, v_2\}$, then Fixer should swap 0 and 4 at $v_3$. This results in a position with lists $L(v_1) = \{0, 1\}$, $L(v_2) = \{0, 2, 3\}$, $L(v_3) = \{1, 2, 4\}$, $L(v_4) = \{1, 4\}$ and $L(v_5) = \{0, 2\}$, but then Fixer wins by Case 216. If the components of $A_S$ have vertex sets $\{v_4\}$, $\{v_0, v_3\}$ and $\{v_1, v_2\}$, then Fixer should swap 0 and 4 at $v_4$. This results in a position with lists $L(v_1) = \{0, 1\}$, $L(v_2) = \{0, 2, 3\}$, $L(v_3) = \{1, 2, 4\}$, $L(v_4) = \{0, 1\}$ and $L(v_5) = \{2, 4\}$, but then Fixer can edge-color the graph. 

\noindent\textbf{Case 324.  }\textit{$L(v_1) = \{0, 1\}$, $L(v_2) = \{0, 2, 3\}$, $L(v_3) = \{1, 2, 4\}$, $L(v_4) = \{0, 1\}$ and $L(v_5) = \{1, 2\}$.}

Let $S$ and $A_S$ be as in Lemma \ref{MultiMoveCombination} using colors $1$ and $3$. If the components of $A_S$ have vertex sets $\{v_0\}$, $\{v_1, v_2\}$ and $\{v_3, v_4\}$, then Fixer should swap 1 and 3 at $v_0$. This results in a position with lists $L(v_1) = \{0, 3\}$, $L(v_2) = \{0, 2, 3\}$, $L(v_3) = \{1, 2, 4\}$, $L(v_4) = \{0, 1\}$ and $L(v_5) = \{1, 2\}$, but then Fixer wins by Case 99. If the components of $A_S$ have vertex sets $\{v_0\}$, $\{v_1, v_3\}$ and $\{v_2, v_4\}$, then Fixer should swap 1 and 3 at $v_0$. This results in a position with lists $L(v_1) = \{0, 3\}$, $L(v_2) = \{0, 2, 3\}$, $L(v_3) = \{1, 2, 4\}$, $L(v_4) = \{0, 1\}$ and $L(v_5) = \{1, 2\}$, but then Fixer wins by Case 99. If the components of $A_S$ have vertex sets $\{v_0\}$, $\{v_1, v_4\}$ and $\{v_2, v_3\}$, then Fixer should swap 1 and 3 at $v_0$. This results in a position with lists $L(v_1) = \{0, 3\}$, $L(v_2) = \{0, 2, 3\}$, $L(v_3) = \{1, 2, 4\}$, $L(v_4) = \{0, 1\}$ and $L(v_5) = \{1, 2\}$, but then Fixer wins by Case 99. If the components of $A_S$ have vertex sets $\{v_1\}$, $\{v_0, v_2\}$ and $\{v_3, v_4\}$, then Fixer should swap 1 and 3 at $v_1$. This results in a position with lists $L(v_1) = \{0, 1\}$, $L(v_2) = \{0, 1, 2\}$, $L(v_3) = \{1, 2, 4\}$, $L(v_4) = \{0, 1\}$ and $L(v_5) = \{1, 2\}$, but then Fixer can edge-color the graph. If the components of $A_S$ have vertex sets $\{v_1\}$, $\{v_0, v_3\}$ and $\{v_2, v_4\}$, then Fixer should swap 1 and 3 at $v_1$. This results in a position with lists $L(v_1) = \{0, 1\}$, $L(v_2) = \{0, 1, 2\}$, $L(v_3) = \{1, 2, 4\}$, $L(v_4) = \{0, 1\}$ and $L(v_5) = \{1, 2\}$, but then Fixer can edge-color the graph. If the components of $A_S$ have vertex sets $\{v_1\}$, $\{v_0, v_4\}$ and $\{v_2, v_3\}$, then Fixer should swap 1 and 3 at $v_1$. This results in a position with lists $L(v_1) = \{0, 1\}$, $L(v_2) = \{0, 1, 2\}$, $L(v_3) = \{1, 2, 4\}$, $L(v_4) = \{0, 1\}$ and $L(v_5) = \{1, 2\}$, but then Fixer can edge-color the graph. If the components of $A_S$ have vertex sets $\{v_2\}$, $\{v_0, v_1\}$ and $\{v_3, v_4\}$, then Fixer should swap 1 and 3 at $v_1$ and $v_0$. This results in a position with lists $L(v_1) = \{0, 3\}$, $L(v_2) = \{0, 1, 2\}$, $L(v_3) = \{1, 2, 4\}$, $L(v_4) = \{0, 1\}$ and $L(v_5) = \{1, 2\}$, but then Fixer wins by Case 228. If the components of $A_S$ have vertex sets $\{v_3\}$, $\{v_0, v_1\}$ and $\{v_2, v_4\}$, then Fixer should swap 1 and 3 at $v_1$ and $v_0$. This results in a position with lists $L(v_1) = \{0, 3\}$, $L(v_2) = \{0, 1, 2\}$, $L(v_3) = \{1, 2, 4\}$, $L(v_4) = \{0, 1\}$ and $L(v_5) = \{1, 2\}$, but then Fixer wins by Case 228. If the components of $A_S$ have vertex sets $\{v_4\}$, $\{v_0, v_1\}$ and $\{v_2, v_3\}$, then Fixer should swap 1 and 3 at $v_4$. This results in a position with lists $L(v_1) = \{0, 1\}$, $L(v_2) = \{0, 2, 3\}$, $L(v_3) = \{1, 2, 4\}$, $L(v_4) = \{0, 1\}$ and $L(v_5) = \{2, 3\}$, but then Fixer wins by Case 194. If the components of $A_S$ have vertex sets $\{v_2\}$, $\{v_0, v_3\}$ and $\{v_1, v_4\}$, then Fixer should swap 1 and 3 at $v_4$ and $v_1$. This results in a position with lists $L(v_1) = \{0, 1\}$, $L(v_2) = \{0, 1, 2\}$, $L(v_3) = \{1, 2, 4\}$, $L(v_4) = \{0, 1\}$ and $L(v_5) = \{2, 3\}$, but then Fixer wins by Case 56. If the components of $A_S$ have vertex sets $\{v_2\}$, $\{v_0, v_4\}$ and $\{v_1, v_3\}$, then Fixer should swap 1 and 3 at $v_4$ and $v_0$. This results in a position with lists $L(v_1) = \{0, 3\}$, $L(v_2) = \{0, 2, 3\}$, $L(v_3) = \{1, 2, 4\}$, $L(v_4) = \{0, 1\}$ and $L(v_5) = \{2, 3\}$, but then Fixer wins by Case 97. If the components of $A_S$ have vertex sets $\{v_3\}$, $\{v_0, v_2\}$ and $\{v_1, v_4\}$, then Fixer should swap 1 and 3 at $v_4$ and $v_1$. This results in a position with lists $L(v_1) = \{0, 1\}$, $L(v_2) = \{0, 1, 2\}$, $L(v_3) = \{1, 2, 4\}$, $L(v_4) = \{0, 1\}$ and $L(v_5) = \{2, 3\}$, but then Fixer wins by Case 56. If the components of $A_S$ have vertex sets $\{v_4\}$, $\{v_0, v_2\}$ and $\{v_1, v_3\}$, then Fixer should swap 1 and 3 at $v_4$. This results in a position with lists $L(v_1) = \{0, 1\}$, $L(v_2) = \{0, 2, 3\}$, $L(v_3) = \{1, 2, 4\}$, $L(v_4) = \{0, 1\}$ and $L(v_5) = \{2, 3\}$, but then Fixer wins by Case 194. If the components of $A_S$ have vertex sets $\{v_3\}$, $\{v_0, v_4\}$ and $\{v_1, v_2\}$, then Fixer should swap 1 and 3 at $v_4$ and $v_0$. This results in a position with lists $L(v_1) = \{0, 3\}$, $L(v_2) = \{0, 2, 3\}$, $L(v_3) = \{1, 2, 4\}$, $L(v_4) = \{0, 1\}$ and $L(v_5) = \{2, 3\}$, but then Fixer wins by Case 97. If the components of $A_S$ have vertex sets $\{v_4\}$, $\{v_0, v_3\}$ and $\{v_1, v_2\}$, then Fixer should swap 1 and 3 at $v_4$. This results in a position with lists $L(v_1) = \{0, 1\}$, $L(v_2) = \{0, 2, 3\}$, $L(v_3) = \{1, 2, 4\}$, $L(v_4) = \{0, 1\}$ and $L(v_5) = \{2, 3\}$, but then Fixer wins by Case 194. 

\noindent\textbf{Case 325.  }\textit{$L(v_1) = \{0, 1\}$, $L(v_2) = \{0, 2, 3\}$, $L(v_3) = \{1, 2, 4\}$, $L(v_4) = \{0, 2\}$ and $L(v_5) = \{0, 1\}$.}

Let $S$ and $A_S$ be as in Lemma \ref{MultiMoveCombination} using colors $0$ and $4$. If the components of $A_S$ have vertex sets $\{v_0\}$, $\{v_1, v_2\}$ and $\{v_3, v_4\}$, then Fixer should swap 0 and 4 at $v_4$ and $v_3$. This results in a position with lists $L(v_1) = \{0, 1\}$, $L(v_2) = \{0, 2, 3\}$, $L(v_3) = \{1, 2, 4\}$, $L(v_4) = \{2, 4\}$ and $L(v_5) = \{1, 4\}$, but then Fixer can edge-color the graph. If the components of $A_S$ have vertex sets $\{v_0\}$, $\{v_1, v_3\}$ and $\{v_2, v_4\}$, then Fixer should swap 0 and 4 at $v_4$ and $v_2$. This results in a position with lists $L(v_1) = \{0, 1\}$, $L(v_2) = \{0, 2, 3\}$, $L(v_3) = \{0, 1, 2\}$, $L(v_4) = \{0, 2\}$ and $L(v_5) = \{1, 4\}$, but then Fixer can edge-color the graph. If the components of $A_S$ have vertex sets $\{v_0\}$, $\{v_1, v_4\}$ and $\{v_2, v_3\}$, then Fixer should swap 0 and 4 at $v_3$ and $v_2$. This results in a position with lists $L(v_1) = \{0, 1\}$, $L(v_2) = \{0, 2, 3\}$, $L(v_3) = \{0, 1, 2\}$, $L(v_4) = \{2, 4\}$ and $L(v_5) = \{0, 1\}$, but then Fixer wins by Case 119. If the components of $A_S$ have vertex sets $\{v_1\}$, $\{v_0, v_2\}$ and $\{v_3, v_4\}$, then Fixer should swap 0 and 4 at $v_4$ and $v_3$. This results in a position with lists $L(v_1) = \{0, 1\}$, $L(v_2) = \{0, 2, 3\}$, $L(v_3) = \{1, 2, 4\}$, $L(v_4) = \{2, 4\}$ and $L(v_5) = \{1, 4\}$, but then Fixer can edge-color the graph. If the components of $A_S$ have vertex sets $\{v_1\}$, $\{v_0, v_3\}$ and $\{v_2, v_4\}$, then Fixer should swap 0 and 4 at $v_4$ and $v_2$. This results in a position with lists $L(v_1) = \{0, 1\}$, $L(v_2) = \{0, 2, 3\}$, $L(v_3) = \{0, 1, 2\}$, $L(v_4) = \{0, 2\}$ and $L(v_5) = \{1, 4\}$, but then Fixer can edge-color the graph. If the components of $A_S$ have vertex sets $\{v_1\}$, $\{v_0, v_4\}$ and $\{v_2, v_3\}$, then Fixer should swap 0 and 4 at $v_3$ and $v_2$. This results in a position with lists $L(v_1) = \{0, 1\}$, $L(v_2) = \{0, 2, 3\}$, $L(v_3) = \{0, 1, 2\}$, $L(v_4) = \{2, 4\}$ and $L(v_5) = \{0, 1\}$, but then Fixer wins by Case 119. If the components of $A_S$ have vertex sets $\{v_2\}$, $\{v_0, v_1\}$ and $\{v_3, v_4\}$, then Fixer should swap 0 and 4 at $v_2$. This results in a position with lists $L(v_1) = \{0, 1\}$, $L(v_2) = \{0, 2, 3\}$, $L(v_3) = \{0, 1, 2\}$, $L(v_4) = \{0, 2\}$ and $L(v_5) = \{0, 1\}$, but then Fixer can edge-color the graph. If the components of $A_S$ have vertex sets $\{v_3\}$, $\{v_0, v_1\}$ and $\{v_2, v_4\}$, then Fixer should swap 0 and 4 at $v_3$. This results in a position with lists $L(v_1) = \{0, 1\}$, $L(v_2) = \{0, 2, 3\}$, $L(v_3) = \{1, 2, 4\}$, $L(v_4) = \{2, 4\}$ and $L(v_5) = \{0, 1\}$, but then Fixer can edge-color the graph. If the components of $A_S$ have vertex sets $\{v_4\}$, $\{v_0, v_1\}$ and $\{v_2, v_3\}$, then Fixer should swap 0 and 4 at $v_4$. This results in a position with lists $L(v_1) = \{0, 1\}$, $L(v_2) = \{0, 2, 3\}$, $L(v_3) = \{1, 2, 4\}$, $L(v_4) = \{0, 2\}$ and $L(v_5) = \{1, 4\}$, but then Fixer wins by Case 196. If the components of $A_S$ have vertex sets $\{v_2\}$, $\{v_0, v_3\}$ and $\{v_1, v_4\}$, then Fixer should swap 0 and 4 at $v_2$. This results in a position with lists $L(v_1) = \{0, 1\}$, $L(v_2) = \{0, 2, 3\}$, $L(v_3) = \{0, 1, 2\}$, $L(v_4) = \{0, 2\}$ and $L(v_5) = \{0, 1\}$, but then Fixer can edge-color the graph. If the components of $A_S$ have vertex sets $\{v_2\}$, $\{v_0, v_4\}$ and $\{v_1, v_3\}$, then Fixer should swap 0 and 4 at $v_2$. This results in a position with lists $L(v_1) = \{0, 1\}$, $L(v_2) = \{0, 2, 3\}$, $L(v_3) = \{0, 1, 2\}$, $L(v_4) = \{0, 2\}$ and $L(v_5) = \{0, 1\}$, but then Fixer can edge-color the graph. If the components of $A_S$ have vertex sets $\{v_3\}$, $\{v_0, v_2\}$ and $\{v_1, v_4\}$, then Fixer should swap 0 and 4 at $v_3$. This results in a position with lists $L(v_1) = \{0, 1\}$, $L(v_2) = \{0, 2, 3\}$, $L(v_3) = \{1, 2, 4\}$, $L(v_4) = \{2, 4\}$ and $L(v_5) = \{0, 1\}$, but then Fixer can edge-color the graph. If the components of $A_S$ have vertex sets $\{v_4\}$, $\{v_0, v_2\}$ and $\{v_1, v_3\}$, then Fixer should swap 0 and 4 at $v_4$. This results in a position with lists $L(v_1) = \{0, 1\}$, $L(v_2) = \{0, 2, 3\}$, $L(v_3) = \{1, 2, 4\}$, $L(v_4) = \{0, 2\}$ and $L(v_5) = \{1, 4\}$, but then Fixer wins by Case 196. If the components of $A_S$ have vertex sets $\{v_3\}$, $\{v_0, v_4\}$ and $\{v_1, v_2\}$, then Fixer should swap 0 and 4 at $v_3$. This results in a position with lists $L(v_1) = \{0, 1\}$, $L(v_2) = \{0, 2, 3\}$, $L(v_3) = \{1, 2, 4\}$, $L(v_4) = \{2, 4\}$ and $L(v_5) = \{0, 1\}$, but then Fixer can edge-color the graph. If the components of $A_S$ have vertex sets $\{v_4\}$, $\{v_0, v_3\}$ and $\{v_1, v_2\}$, then Fixer should swap 0 and 4 at $v_4$. This results in a position with lists $L(v_1) = \{0, 1\}$, $L(v_2) = \{0, 2, 3\}$, $L(v_3) = \{1, 2, 4\}$, $L(v_4) = \{0, 2\}$ and $L(v_5) = \{1, 4\}$, but then Fixer wins by Case 196. 

\noindent\textbf{Case 326.  }\textit{$L(v_1) = \{0, 1\}$, $L(v_2) = \{0, 2, 3\}$, $L(v_3) = \{1, 2, 4\}$, $L(v_4) = \{0, 2\}$ and $L(v_5) = \{0, 2\}$.}

Let $S$ and $A_S$ be as in Lemma \ref{MultiMoveCombination} using colors $0$ and $4$. If the components of $A_S$ have vertex sets $\{v_0\}$, $\{v_1, v_2\}$ and $\{v_3, v_4\}$, then Fixer should swap 0 and 4 at $v_4$ and $v_3$. This results in a position with lists $L(v_1) = \{0, 1\}$, $L(v_2) = \{0, 2, 3\}$, $L(v_3) = \{1, 2, 4\}$, $L(v_4) = \{2, 4\}$ and $L(v_5) = \{2, 4\}$, but then Fixer wins by Case 220. If the components of $A_S$ have vertex sets $\{v_0\}$, $\{v_1, v_3\}$ and $\{v_2, v_4\}$, then Fixer should swap 0 and 4 at $v_4$ and $v_2$. This results in a position with lists $L(v_1) = \{0, 1\}$, $L(v_2) = \{0, 2, 3\}$, $L(v_3) = \{0, 1, 2\}$, $L(v_4) = \{0, 2\}$ and $L(v_5) = \{2, 4\}$, but then Fixer wins by Case 106. If the components of $A_S$ have vertex sets $\{v_0\}$, $\{v_1, v_4\}$ and $\{v_2, v_3\}$, then Fixer should swap 0 and 4 at $v_3$ and $v_2$. This results in a position with lists $L(v_1) = \{0, 1\}$, $L(v_2) = \{0, 2, 3\}$, $L(v_3) = \{0, 1, 2\}$, $L(v_4) = \{2, 4\}$ and $L(v_5) = \{0, 2\}$, but then Fixer wins by Case 120. If the components of $A_S$ have vertex sets $\{v_1\}$, $\{v_0, v_2\}$ and $\{v_3, v_4\}$, then Fixer should swap 0 and 4 at $v_4$ and $v_3$. This results in a position with lists $L(v_1) = \{0, 1\}$, $L(v_2) = \{0, 2, 3\}$, $L(v_3) = \{1, 2, 4\}$, $L(v_4) = \{2, 4\}$ and $L(v_5) = \{2, 4\}$, but then Fixer wins by Case 220. If the components of $A_S$ have vertex sets $\{v_1\}$, $\{v_0, v_3\}$ and $\{v_2, v_4\}$, then Fixer should swap 0 and 4 at $v_4$ and $v_2$. This results in a position with lists $L(v_1) = \{0, 1\}$, $L(v_2) = \{0, 2, 3\}$, $L(v_3) = \{0, 1, 2\}$, $L(v_4) = \{0, 2\}$ and $L(v_5) = \{2, 4\}$, but then Fixer wins by Case 106. If the components of $A_S$ have vertex sets $\{v_1\}$, $\{v_0, v_4\}$ and $\{v_2, v_3\}$, then Fixer should swap 0 and 4 at $v_3$ and $v_2$. This results in a position with lists $L(v_1) = \{0, 1\}$, $L(v_2) = \{0, 2, 3\}$, $L(v_3) = \{0, 1, 2\}$, $L(v_4) = \{2, 4\}$ and $L(v_5) = \{0, 2\}$, but then Fixer wins by Case 120. If the components of $A_S$ have vertex sets $\{v_2\}$, $\{v_0, v_1\}$ and $\{v_3, v_4\}$, then Fixer should swap 0 and 4 at $v_2$. This results in a position with lists $L(v_1) = \{0, 1\}$, $L(v_2) = \{0, 2, 3\}$, $L(v_3) = \{0, 1, 2\}$, $L(v_4) = \{0, 2\}$ and $L(v_5) = \{0, 2\}$, but then Fixer can edge-color the graph. If the components of $A_S$ have vertex sets $\{v_3\}$, $\{v_0, v_1\}$ and $\{v_2, v_4\}$, then Fixer should swap 0 and 4 at $v_3$. This results in a position with lists $L(v_1) = \{0, 1\}$, $L(v_2) = \{0, 2, 3\}$, $L(v_3) = \{1, 2, 4\}$, $L(v_4) = \{2, 4\}$ and $L(v_5) = \{0, 2\}$, but then Fixer wins by Case 218. If the components of $A_S$ have vertex sets $\{v_4\}$, $\{v_0, v_1\}$ and $\{v_2, v_3\}$, then Fixer should swap 0 and 4 at $v_4$. This results in a position with lists $L(v_1) = \{0, 1\}$, $L(v_2) = \{0, 2, 3\}$, $L(v_3) = \{1, 2, 4\}$, $L(v_4) = \{0, 2\}$ and $L(v_5) = \{2, 4\}$, but then Fixer wins by Case 197. If the components of $A_S$ have vertex sets $\{v_2\}$, $\{v_0, v_3\}$ and $\{v_1, v_4\}$, then Fixer should swap 0 and 4 at $v_2$. This results in a position with lists $L(v_1) = \{0, 1\}$, $L(v_2) = \{0, 2, 3\}$, $L(v_3) = \{0, 1, 2\}$, $L(v_4) = \{0, 2\}$ and $L(v_5) = \{0, 2\}$, but then Fixer can edge-color the graph. If the components of $A_S$ have vertex sets $\{v_2\}$, $\{v_0, v_4\}$ and $\{v_1, v_3\}$, then Fixer should swap 0 and 4 at $v_2$. This results in a position with lists $L(v_1) = \{0, 1\}$, $L(v_2) = \{0, 2, 3\}$, $L(v_3) = \{0, 1, 2\}$, $L(v_4) = \{0, 2\}$ and $L(v_5) = \{0, 2\}$, but then Fixer can edge-color the graph. If the components of $A_S$ have vertex sets $\{v_3\}$, $\{v_0, v_2\}$ and $\{v_1, v_4\}$, then Fixer should swap 0 and 4 at $v_3$. This results in a position with lists $L(v_1) = \{0, 1\}$, $L(v_2) = \{0, 2, 3\}$, $L(v_3) = \{1, 2, 4\}$, $L(v_4) = \{2, 4\}$ and $L(v_5) = \{0, 2\}$, but then Fixer wins by Case 218. If the components of $A_S$ have vertex sets $\{v_4\}$, $\{v_0, v_2\}$ and $\{v_1, v_3\}$, then Fixer should swap 0 and 4 at $v_4$. This results in a position with lists $L(v_1) = \{0, 1\}$, $L(v_2) = \{0, 2, 3\}$, $L(v_3) = \{1, 2, 4\}$, $L(v_4) = \{0, 2\}$ and $L(v_5) = \{2, 4\}$, but then Fixer wins by Case 197. If the components of $A_S$ have vertex sets $\{v_3\}$, $\{v_0, v_4\}$ and $\{v_1, v_2\}$, then Fixer should swap 0 and 4 at $v_3$. This results in a position with lists $L(v_1) = \{0, 1\}$, $L(v_2) = \{0, 2, 3\}$, $L(v_3) = \{1, 2, 4\}$, $L(v_4) = \{2, 4\}$ and $L(v_5) = \{0, 2\}$, but then Fixer wins by Case 218. If the components of $A_S$ have vertex sets $\{v_4\}$, $\{v_0, v_3\}$ and $\{v_1, v_2\}$, then Fixer should swap 0 and 4 at $v_4$. This results in a position with lists $L(v_1) = \{0, 1\}$, $L(v_2) = \{0, 2, 3\}$, $L(v_3) = \{1, 2, 4\}$, $L(v_4) = \{0, 2\}$ and $L(v_5) = \{2, 4\}$, but then Fixer wins by Case 197. 

\noindent\textbf{Case 327.  }\textit{$L(v_1) = \{0, 1\}$, $L(v_2) = \{0, 2, 3\}$, $L(v_3) = \{1, 2, 4\}$, $L(v_4) = \{0, 2\}$ and $L(v_5) = \{1, 2\}$.}

Let $S$ and $A_S$ be as in Lemma \ref{MultiMoveCombination} using colors $2$ and $4$. If the components of $A_S$ have vertex sets $\{v_1\}$ and $\{v_3, v_4\}$, then Fixer should swap 2 and 4 at $v_1$. This results in a position with lists $L(v_1) = \{0, 1\}$, $L(v_2) = \{0, 3, 4\}$, $L(v_3) = \{1, 2, 4\}$, $L(v_4) = \{0, 2\}$ and $L(v_5) = \{1, 2\}$, but then Fixer can edge-color the graph. If the components of $A_S$ have vertex sets $\{v_3\}$ and $\{v_1, v_4\}$, then Fixer should swap 2 and 4 at $v_3$. This results in a position with lists $L(v_1) = \{0, 1\}$, $L(v_2) = \{0, 2, 3\}$, $L(v_3) = \{1, 2, 4\}$, $L(v_4) = \{0, 4\}$ and $L(v_5) = \{1, 2\}$, but then Fixer can edge-color the graph. If the components of $A_S$ have vertex sets $\{v_4\}$ and $\{v_1, v_3\}$, then Fixer should swap 2 and 4 at $v_4$. This results in a position with lists $L(v_1) = \{0, 1\}$, $L(v_2) = \{0, 2, 3\}$, $L(v_3) = \{1, 2, 4\}$, $L(v_4) = \{0, 2\}$ and $L(v_5) = \{1, 4\}$, but then Fixer wins by Case 196. 

\noindent\textbf{Case 328.  }\textit{$L(v_1) = \{0, 1\}$, $L(v_2) = \{0, 2, 3\}$, $L(v_3) = \{1, 2, 4\}$, $L(v_4) = \{0, 2\}$ and $L(v_5) = \{1, 3\}$.}

Let $S$ and $A_S$ be as in Lemma \ref{MultiMoveCombination} using colors $3$ and $4$. If the components of $A_S$ have vertex sets $\{v_1\}$ and $\{v_2, v_4\}$, then Fixer should swap 3 and 4 at $v_1$. This results in a position with lists $L(v_1) = \{0, 1\}$, $L(v_2) = \{0, 2, 4\}$, $L(v_3) = \{1, 2, 4\}$, $L(v_4) = \{0, 2\}$ and $L(v_5) = \{1, 3\}$, but then Fixer can edge-color the graph. If the components of $A_S$ have vertex sets $\{v_2\}$ and $\{v_1, v_4\}$, then Fixer should swap 3 and 4 at $v_2$. This results in a position with lists $L(v_1) = \{0, 1\}$, $L(v_2) = \{0, 2, 3\}$, $L(v_3) = \{1, 2, 3\}$, $L(v_4) = \{0, 2\}$ and $L(v_5) = \{1, 3\}$, but then Fixer can edge-color the graph. If the components of $A_S$ have vertex sets $\{v_4\}$ and $\{v_1, v_2\}$, then Fixer should swap 3 and 4 at $v_4$. This results in a position with lists $L(v_1) = \{0, 1\}$, $L(v_2) = \{0, 2, 3\}$, $L(v_3) = \{1, 2, 4\}$, $L(v_4) = \{0, 2\}$ and $L(v_5) = \{1, 4\}$, but then Fixer wins by Case 196. 

\noindent\textbf{Case 329.  }\textit{$L(v_1) = \{0, 1\}$, $L(v_2) = \{0, 2, 3\}$, $L(v_3) = \{1, 2, 4\}$, $L(v_4) = \{1, 2\}$ and $L(v_5) = \{0, 1\}$.}

Let $S$ and $A_S$ be as in Lemma \ref{MultiMoveCombination} using colors $1$ and $3$. If the components of $A_S$ have vertex sets $\{v_0\}$, $\{v_1, v_2\}$ and $\{v_3, v_4\}$, then Fixer should swap 1 and 3 at $v_0$. This results in a position with lists $L(v_1) = \{0, 3\}$, $L(v_2) = \{0, 2, 3\}$, $L(v_3) = \{1, 2, 4\}$, $L(v_4) = \{1, 2\}$ and $L(v_5) = \{0, 1\}$, but then Fixer wins by Case 101. If the components of $A_S$ have vertex sets $\{v_0\}$, $\{v_1, v_3\}$ and $\{v_2, v_4\}$, then Fixer should swap 1 and 3 at $v_0$. This results in a position with lists $L(v_1) = \{0, 3\}$, $L(v_2) = \{0, 2, 3\}$, $L(v_3) = \{1, 2, 4\}$, $L(v_4) = \{1, 2\}$ and $L(v_5) = \{0, 1\}$, but then Fixer wins by Case 101. If the components of $A_S$ have vertex sets $\{v_0\}$, $\{v_1, v_4\}$ and $\{v_2, v_3\}$, then Fixer should swap 1 and 3 at $v_0$. This results in a position with lists $L(v_1) = \{0, 3\}$, $L(v_2) = \{0, 2, 3\}$, $L(v_3) = \{1, 2, 4\}$, $L(v_4) = \{1, 2\}$ and $L(v_5) = \{0, 1\}$, but then Fixer wins by Case 101. If the components of $A_S$ have vertex sets $\{v_1\}$, $\{v_0, v_2\}$ and $\{v_3, v_4\}$, then Fixer should swap 1 and 3 at $v_1$. This results in a position with lists $L(v_1) = \{0, 1\}$, $L(v_2) = \{0, 1, 2\}$, $L(v_3) = \{1, 2, 4\}$, $L(v_4) = \{1, 2\}$ and $L(v_5) = \{0, 1\}$, but then Fixer can edge-color the graph. If the components of $A_S$ have vertex sets $\{v_1\}$, $\{v_0, v_3\}$ and $\{v_2, v_4\}$, then Fixer should swap 1 and 3 at $v_1$. This results in a position with lists $L(v_1) = \{0, 1\}$, $L(v_2) = \{0, 1, 2\}$, $L(v_3) = \{1, 2, 4\}$, $L(v_4) = \{1, 2\}$ and $L(v_5) = \{0, 1\}$, but then Fixer can edge-color the graph. If the components of $A_S$ have vertex sets $\{v_1\}$, $\{v_0, v_4\}$ and $\{v_2, v_3\}$, then Fixer should swap 1 and 3 at $v_1$. This results in a position with lists $L(v_1) = \{0, 1\}$, $L(v_2) = \{0, 1, 2\}$, $L(v_3) = \{1, 2, 4\}$, $L(v_4) = \{1, 2\}$ and $L(v_5) = \{0, 1\}$, but then Fixer can edge-color the graph. If the components of $A_S$ have vertex sets $\{v_2\}$, $\{v_0, v_1\}$ and $\{v_3, v_4\}$, then Fixer should swap 1 and 3 at $v_1$ and $v_0$. This results in a position with lists $L(v_1) = \{0, 3\}$, $L(v_2) = \{0, 1, 2\}$, $L(v_3) = \{1, 2, 4\}$, $L(v_4) = \{1, 2\}$ and $L(v_5) = \{0, 1\}$, but then Fixer wins by Case 232. If the components of $A_S$ have vertex sets $\{v_3\}$, $\{v_0, v_1\}$ and $\{v_2, v_4\}$, then Fixer should swap 1 and 3 at $v_3$. This results in a position with lists $L(v_1) = \{0, 1\}$, $L(v_2) = \{0, 2, 3\}$, $L(v_3) = \{1, 2, 4\}$, $L(v_4) = \{2, 3\}$ and $L(v_5) = \{0, 1\}$, but then Fixer wins by Case 205. If the components of $A_S$ have vertex sets $\{v_4\}$, $\{v_0, v_1\}$ and $\{v_2, v_3\}$, then Fixer should swap 1 and 3 at $v_1$ and $v_0$. This results in a position with lists $L(v_1) = \{0, 3\}$, $L(v_2) = \{0, 1, 2\}$, $L(v_3) = \{1, 2, 4\}$, $L(v_4) = \{1, 2\}$ and $L(v_5) = \{0, 1\}$, but then Fixer wins by Case 232. If the components of $A_S$ have vertex sets $\{v_2\}$, $\{v_0, v_3\}$ and $\{v_1, v_4\}$, then Fixer should swap 1 and 3 at $v_3$ and $v_0$. This results in a position with lists $L(v_1) = \{0, 3\}$, $L(v_2) = \{0, 2, 3\}$, $L(v_3) = \{1, 2, 4\}$, $L(v_4) = \{2, 3\}$ and $L(v_5) = \{0, 1\}$, but then Fixer wins by Case 93. If the components of $A_S$ have vertex sets $\{v_2\}$, $\{v_0, v_4\}$ and $\{v_1, v_3\}$, then Fixer should swap 1 and 3 at $v_3$ and $v_1$. This results in a position with lists $L(v_1) = \{0, 1\}$, $L(v_2) = \{0, 1, 2\}$, $L(v_3) = \{1, 2, 4\}$, $L(v_4) = \{2, 3\}$ and $L(v_5) = \{0, 1\}$, but then Fixer wins by Case 64. If the components of $A_S$ have vertex sets $\{v_3\}$, $\{v_0, v_2\}$ and $\{v_1, v_4\}$, then Fixer should swap 1 and 3 at $v_3$. This results in a position with lists $L(v_1) = \{0, 1\}$, $L(v_2) = \{0, 2, 3\}$, $L(v_3) = \{1, 2, 4\}$, $L(v_4) = \{2, 3\}$ and $L(v_5) = \{0, 1\}$, but then Fixer wins by Case 205. If the components of $A_S$ have vertex sets $\{v_4\}$, $\{v_0, v_2\}$ and $\{v_1, v_3\}$, then Fixer should swap 1 and 3 at $v_3$ and $v_1$. This results in a position with lists $L(v_1) = \{0, 1\}$, $L(v_2) = \{0, 1, 2\}$, $L(v_3) = \{1, 2, 4\}$, $L(v_4) = \{2, 3\}$ and $L(v_5) = \{0, 1\}$, but then Fixer wins by Case 64. If the components of $A_S$ have vertex sets $\{v_3\}$, $\{v_0, v_4\}$ and $\{v_1, v_2\}$, then Fixer should swap 1 and 3 at $v_3$. This results in a position with lists $L(v_1) = \{0, 1\}$, $L(v_2) = \{0, 2, 3\}$, $L(v_3) = \{1, 2, 4\}$, $L(v_4) = \{2, 3\}$ and $L(v_5) = \{0, 1\}$, but then Fixer wins by Case 205. If the components of $A_S$ have vertex sets $\{v_4\}$, $\{v_0, v_3\}$ and $\{v_1, v_2\}$, then Fixer should swap 1 and 3 at $v_3$ and $v_0$. This results in a position with lists $L(v_1) = \{0, 3\}$, $L(v_2) = \{0, 2, 3\}$, $L(v_3) = \{1, 2, 4\}$, $L(v_4) = \{2, 3\}$ and $L(v_5) = \{0, 1\}$, but then Fixer wins by Case 93. 

\noindent\textbf{Case 330.  }\textit{$L(v_1) = \{0, 1\}$, $L(v_2) = \{0, 2, 3\}$, $L(v_3) = \{1, 2, 4\}$, $L(v_4) = \{1, 2\}$ and $L(v_5) = \{0, 2\}$.}

Let $S$ and $A_S$ be as in Lemma \ref{MultiMoveCombination} using colors $2$ and $4$. If the components of $A_S$ have vertex sets $\{v_1\}$ and $\{v_3, v_4\}$, then Fixer should swap 2 and 4 at $v_1$. This results in a position with lists $L(v_1) = \{0, 1\}$, $L(v_2) = \{0, 3, 4\}$, $L(v_3) = \{1, 2, 4\}$, $L(v_4) = \{1, 2\}$ and $L(v_5) = \{0, 2\}$, but then Fixer can edge-color the graph. If the components of $A_S$ have vertex sets $\{v_3\}$ and $\{v_1, v_4\}$, then Fixer should swap 2 and 4 at $v_3$. This results in a position with lists $L(v_1) = \{0, 1\}$, $L(v_2) = \{0, 2, 3\}$, $L(v_3) = \{1, 2, 4\}$, $L(v_4) = \{1, 4\}$ and $L(v_5) = \{0, 2\}$, but then Fixer wins by Case 216. If the components of $A_S$ have vertex sets $\{v_4\}$ and $\{v_1, v_3\}$, then Fixer should swap 2 and 4 at $v_4$. This results in a position with lists $L(v_1) = \{0, 1\}$, $L(v_2) = \{0, 2, 3\}$, $L(v_3) = \{1, 2, 4\}$, $L(v_4) = \{1, 2\}$ and $L(v_5) = \{0, 4\}$, but then Fixer can edge-color the graph. 

\noindent\textbf{Case 331.  }\textit{$L(v_1) = \{0, 1\}$, $L(v_2) = \{0, 2, 3\}$, $L(v_3) = \{1, 2, 4\}$, $L(v_4) = \{1, 3\}$ and $L(v_5) = \{0, 2\}$.}

Let $S$ and $A_S$ be as in Lemma \ref{MultiMoveCombination} using colors $3$ and $4$. If the components of $A_S$ have vertex sets $\{v_1\}$ and $\{v_2, v_3\}$, then Fixer should swap 3 and 4 at $v_1$. This results in a position with lists $L(v_1) = \{0, 1\}$, $L(v_2) = \{0, 2, 4\}$, $L(v_3) = \{1, 2, 4\}$, $L(v_4) = \{1, 3\}$ and $L(v_5) = \{0, 2\}$, but then Fixer can edge-color the graph. If the components of $A_S$ have vertex sets $\{v_2\}$ and $\{v_1, v_3\}$, then Fixer should swap 3 and 4 at $v_2$. This results in a position with lists $L(v_1) = \{0, 1\}$, $L(v_2) = \{0, 2, 3\}$, $L(v_3) = \{1, 2, 3\}$, $L(v_4) = \{1, 3\}$ and $L(v_5) = \{0, 2\}$, but then Fixer can edge-color the graph. If the components of $A_S$ have vertex sets $\{v_3\}$ and $\{v_1, v_2\}$, then Fixer should swap 3 and 4 at $v_3$. This results in a position with lists $L(v_1) = \{0, 1\}$, $L(v_2) = \{0, 2, 3\}$, $L(v_3) = \{1, 2, 4\}$, $L(v_4) = \{1, 4\}$ and $L(v_5) = \{0, 2\}$, but then Fixer wins by Case 216. 

\noindent\textbf{Case 332.  }\textit{$L(v_1) = \{0, 1\}$, $L(v_2) = \{0, 2, 3\}$, $L(v_3) = \{1, 2, 4\}$, $L(v_4) = \{0, 4\}$ and $L(v_5) = \{2, 4\}$.}

Let $S$ and $A_S$ be as in Lemma \ref{MultiMoveCombination} using colors $1$ and $3$. If the components of $A_S$ have vertex sets $\{v_0\}$ and $\{v_1, v_2\}$, then Fixer should swap 1 and 3 at $v_0$. This results in a position with lists $L(v_1) = \{0, 3\}$, $L(v_2) = \{0, 2, 3\}$, $L(v_3) = \{1, 2, 4\}$, $L(v_4) = \{0, 4\}$ and $L(v_5) = \{2, 4\}$, but then Fixer wins by Case 99. If the components of $A_S$ have vertex sets $\{v_1\}$ and $\{v_0, v_2\}$, then Fixer should swap 1 and 3 at $v_1$. This results in a position with lists $L(v_1) = \{0, 1\}$, $L(v_2) = \{0, 1, 2\}$, $L(v_3) = \{1, 2, 4\}$, $L(v_4) = \{0, 4\}$ and $L(v_5) = \{2, 4\}$, but then Fixer can edge-color the graph. If the components of $A_S$ have vertex sets $\{v_2\}$ and $\{v_0, v_1\}$, then Fixer should swap 1 and 3 at $v_2$. This results in a position with lists $L(v_1) = \{0, 1\}$, $L(v_2) = \{0, 2, 3\}$, $L(v_3) = \{2, 3, 4\}$, $L(v_4) = \{0, 4\}$ and $L(v_5) = \{2, 4\}$, but then Fixer can edge-color the graph. 

\noindent\textbf{Case 333.  }\textit{$L(v_1) = \{0, 1\}$, $L(v_2) = \{0, 2, 3\}$, $L(v_3) = \{1, 2, 4\}$, $L(v_4) = \{2, 4\}$ and $L(v_5) = \{0, 4\}$.}

Let $S$ and $A_S$ be as in Lemma \ref{MultiMoveCombination} using colors $1$ and $3$. If the components of $A_S$ have vertex sets $\{v_0\}$ and $\{v_1, v_2\}$, then Fixer should swap 1 and 3 at $v_0$. This results in a position with lists $L(v_1) = \{0, 3\}$, $L(v_2) = \{0, 2, 3\}$, $L(v_3) = \{1, 2, 4\}$, $L(v_4) = \{2, 4\}$ and $L(v_5) = \{0, 4\}$, but then Fixer wins by Case 101. If the components of $A_S$ have vertex sets $\{v_1\}$ and $\{v_0, v_2\}$, then Fixer should swap 1 and 3 at $v_1$. This results in a position with lists $L(v_1) = \{0, 1\}$, $L(v_2) = \{0, 1, 2\}$, $L(v_3) = \{1, 2, 4\}$, $L(v_4) = \{2, 4\}$ and $L(v_5) = \{0, 4\}$, but then Fixer can edge-color the graph. If the components of $A_S$ have vertex sets $\{v_2\}$ and $\{v_0, v_1\}$, then Fixer should swap 1 and 3 at $v_2$. This results in a position with lists $L(v_1) = \{0, 1\}$, $L(v_2) = \{0, 2, 3\}$, $L(v_3) = \{2, 3, 4\}$, $L(v_4) = \{2, 4\}$ and $L(v_5) = \{0, 4\}$, but then Fixer can edge-color the graph. 

\noindent\textbf{Case 334.  }\textit{$L(v_1) = \{0, 1\}$, $L(v_2) = \{0, 2, 3\}$, $L(v_3) = \{2, 3, 4\}$, $L(v_4) = \{0, 2\}$ and $L(v_5) = \{1, 3\}$.}

Let $S$ and $A_S$ be as in Lemma \ref{MultiMoveCombination} using colors $1$ and $2$. If the components of $A_S$ have vertex sets $\{v_0\}$, $\{v_1, v_2\}$ and $\{v_3, v_4\}$, then Fixer should swap 1 and 2 at $v_0$. This results in a position with lists $L(v_1) = \{0, 2\}$, $L(v_2) = \{0, 2, 3\}$, $L(v_3) = \{2, 3, 4\}$, $L(v_4) = \{0, 2\}$ and $L(v_5) = \{1, 3\}$, but then Fixer wins by Case 56. If the components of $A_S$ have vertex sets $\{v_0\}$, $\{v_1, v_3\}$ and $\{v_2, v_4\}$, then Fixer should swap 1 and 2 at $v_0$. This results in a position with lists $L(v_1) = \{0, 2\}$, $L(v_2) = \{0, 2, 3\}$, $L(v_3) = \{2, 3, 4\}$, $L(v_4) = \{0, 2\}$ and $L(v_5) = \{1, 3\}$, but then Fixer wins by Case 56. If the components of $A_S$ have vertex sets $\{v_0\}$, $\{v_1, v_4\}$ and $\{v_2, v_3\}$, then Fixer should swap 1 and 2 at $v_0$. This results in a position with lists $L(v_1) = \{0, 2\}$, $L(v_2) = \{0, 2, 3\}$, $L(v_3) = \{2, 3, 4\}$, $L(v_4) = \{0, 2\}$ and $L(v_5) = \{1, 3\}$, but then Fixer wins by Case 56. If the components of $A_S$ have vertex sets $\{v_1\}$, $\{v_0, v_2\}$ and $\{v_3, v_4\}$, then Fixer should swap 1 and 2 at $v_1$. This results in a position with lists $L(v_1) = \{0, 1\}$, $L(v_2) = \{0, 1, 3\}$, $L(v_3) = \{2, 3, 4\}$, $L(v_4) = \{0, 2\}$ and $L(v_5) = \{1, 3\}$, but then Fixer wins by Case 97. If the components of $A_S$ have vertex sets $\{v_1\}$, $\{v_0, v_3\}$ and $\{v_2, v_4\}$, then Fixer should swap 1 and 2 at $v_1$. This results in a position with lists $L(v_1) = \{0, 1\}$, $L(v_2) = \{0, 1, 3\}$, $L(v_3) = \{2, 3, 4\}$, $L(v_4) = \{0, 2\}$ and $L(v_5) = \{1, 3\}$, but then Fixer wins by Case 97. If the components of $A_S$ have vertex sets $\{v_1\}$, $\{v_0, v_4\}$ and $\{v_2, v_3\}$, then Fixer should swap 1 and 2 at $v_1$. This results in a position with lists $L(v_1) = \{0, 1\}$, $L(v_2) = \{0, 1, 3\}$, $L(v_3) = \{2, 3, 4\}$, $L(v_4) = \{0, 2\}$ and $L(v_5) = \{1, 3\}$, but then Fixer wins by Case 97. If the components of $A_S$ have vertex sets $\{v_2\}$, $\{v_0, v_1\}$ and $\{v_3, v_4\}$, then Fixer should swap 1 and 2 at $v_1$ and $v_0$. This results in a position with lists $L(v_1) = \{0, 2\}$, $L(v_2) = \{0, 1, 3\}$, $L(v_3) = \{2, 3, 4\}$, $L(v_4) = \{0, 2\}$ and $L(v_5) = \{1, 3\}$, but then Fixer wins by Case 194. If the components of $A_S$ have vertex sets $\{v_3\}$, $\{v_0, v_1\}$ and $\{v_2, v_4\}$, then Fixer should swap 1 and 2 at $v_1$ and $v_0$. This results in a position with lists $L(v_1) = \{0, 2\}$, $L(v_2) = \{0, 1, 3\}$, $L(v_3) = \{2, 3, 4\}$, $L(v_4) = \{0, 2\}$ and $L(v_5) = \{1, 3\}$, but then Fixer wins by Case 194. If the components of $A_S$ have vertex sets $\{v_4\}$, $\{v_0, v_1\}$ and $\{v_2, v_3\}$, then Fixer should swap 1 and 2 at $v_4$. This results in a position with lists $L(v_1) = \{0, 1\}$, $L(v_2) = \{0, 2, 3\}$, $L(v_3) = \{2, 3, 4\}$, $L(v_4) = \{0, 2\}$ and $L(v_5) = \{2, 3\}$, but then Fixer wins by Case 228. If the components of $A_S$ have vertex sets $\{v_2\}$, $\{v_0, v_3\}$ and $\{v_1, v_4\}$, then Fixer should swap 1 and 2 at $v_4$ and $v_1$. This results in a position with lists $L(v_1) = \{0, 1\}$, $L(v_2) = \{0, 1, 3\}$, $L(v_3) = \{2, 3, 4\}$, $L(v_4) = \{0, 2\}$ and $L(v_5) = \{2, 3\}$, but then Fixer wins by Case 99. If the components of $A_S$ have vertex sets $\{v_2\}$, $\{v_0, v_4\}$ and $\{v_1, v_3\}$, then Fixer should swap 1 and 2 at $v_4$ and $v_0$. This results in a position with lists $L(v_1) = \{0, 2\}$, $L(v_2) = \{0, 2, 3\}$, $L(v_3) = \{2, 3, 4\}$, $L(v_4) = \{0, 2\}$ and $L(v_5) = \{2, 3\}$, but then Fixer can edge-color the graph. If the components of $A_S$ have vertex sets $\{v_3\}$, $\{v_0, v_2\}$ and $\{v_1, v_4\}$, then Fixer should swap 1 and 2 at $v_4$ and $v_1$. This results in a position with lists $L(v_1) = \{0, 1\}$, $L(v_2) = \{0, 1, 3\}$, $L(v_3) = \{2, 3, 4\}$, $L(v_4) = \{0, 2\}$ and $L(v_5) = \{2, 3\}$, but then Fixer wins by Case 99. If the components of $A_S$ have vertex sets $\{v_4\}$, $\{v_0, v_2\}$ and $\{v_1, v_3\}$, then Fixer should swap 1 and 2 at $v_4$. This results in a position with lists $L(v_1) = \{0, 1\}$, $L(v_2) = \{0, 2, 3\}$, $L(v_3) = \{2, 3, 4\}$, $L(v_4) = \{0, 2\}$ and $L(v_5) = \{2, 3\}$, but then Fixer wins by Case 228. If the components of $A_S$ have vertex sets $\{v_3\}$, $\{v_0, v_4\}$ and $\{v_1, v_2\}$, then Fixer should swap 1 and 2 at $v_4$ and $v_0$. This results in a position with lists $L(v_1) = \{0, 2\}$, $L(v_2) = \{0, 2, 3\}$, $L(v_3) = \{2, 3, 4\}$, $L(v_4) = \{0, 2\}$ and $L(v_5) = \{2, 3\}$, but then Fixer can edge-color the graph. If the components of $A_S$ have vertex sets $\{v_4\}$, $\{v_0, v_3\}$ and $\{v_1, v_2\}$, then Fixer should swap 1 and 2 at $v_4$. This results in a position with lists $L(v_1) = \{0, 1\}$, $L(v_2) = \{0, 2, 3\}$, $L(v_3) = \{2, 3, 4\}$, $L(v_4) = \{0, 2\}$ and $L(v_5) = \{2, 3\}$, but then Fixer wins by Case 228. 

\noindent\textbf{Case 335.  }\textit{$L(v_1) = \{0, 1\}$, $L(v_2) = \{0, 2, 3\}$, $L(v_3) = \{2, 3, 4\}$, $L(v_4) = \{1, 2\}$ and $L(v_5) = \{0, 3\}$.}

Let $S$ and $A_S$ be as in Lemma \ref{MultiMoveCombination} using colors $1$ and $3$. If the components of $A_S$ have vertex sets $\{v_0\}$, $\{v_1, v_2\}$ and $\{v_3, v_4\}$, then Fixer should swap 1 and 3 at $v_0$. This results in a position with lists $L(v_1) = \{0, 3\}$, $L(v_2) = \{0, 2, 3\}$, $L(v_3) = \{2, 3, 4\}$, $L(v_4) = \{1, 2\}$ and $L(v_5) = \{0, 3\}$, but then Fixer wins by Case 64. If the components of $A_S$ have vertex sets $\{v_0\}$, $\{v_1, v_3\}$ and $\{v_2, v_4\}$, then Fixer should swap 1 and 3 at $v_0$. This results in a position with lists $L(v_1) = \{0, 3\}$, $L(v_2) = \{0, 2, 3\}$, $L(v_3) = \{2, 3, 4\}$, $L(v_4) = \{1, 2\}$ and $L(v_5) = \{0, 3\}$, but then Fixer wins by Case 64. If the components of $A_S$ have vertex sets $\{v_0\}$, $\{v_1, v_4\}$ and $\{v_2, v_3\}$, then Fixer should swap 1 and 3 at $v_0$. This results in a position with lists $L(v_1) = \{0, 3\}$, $L(v_2) = \{0, 2, 3\}$, $L(v_3) = \{2, 3, 4\}$, $L(v_4) = \{1, 2\}$ and $L(v_5) = \{0, 3\}$, but then Fixer wins by Case 64. If the components of $A_S$ have vertex sets $\{v_1\}$, $\{v_0, v_2\}$ and $\{v_3, v_4\}$, then Fixer should swap 1 and 3 at $v_1$. This results in a position with lists $L(v_1) = \{0, 1\}$, $L(v_2) = \{0, 1, 2\}$, $L(v_3) = \{2, 3, 4\}$, $L(v_4) = \{1, 2\}$ and $L(v_5) = \{0, 3\}$, but then Fixer wins by Case 93. If the components of $A_S$ have vertex sets $\{v_1\}$, $\{v_0, v_3\}$ and $\{v_2, v_4\}$, then Fixer should swap 1 and 3 at $v_1$. This results in a position with lists $L(v_1) = \{0, 1\}$, $L(v_2) = \{0, 1, 2\}$, $L(v_3) = \{2, 3, 4\}$, $L(v_4) = \{1, 2\}$ and $L(v_5) = \{0, 3\}$, but then Fixer wins by Case 93. If the components of $A_S$ have vertex sets $\{v_1\}$, $\{v_0, v_4\}$ and $\{v_2, v_3\}$, then Fixer should swap 1 and 3 at $v_1$. This results in a position with lists $L(v_1) = \{0, 1\}$, $L(v_2) = \{0, 1, 2\}$, $L(v_3) = \{2, 3, 4\}$, $L(v_4) = \{1, 2\}$ and $L(v_5) = \{0, 3\}$, but then Fixer wins by Case 93. If the components of $A_S$ have vertex sets $\{v_2\}$, $\{v_0, v_1\}$ and $\{v_3, v_4\}$, then Fixer should swap 1 and 3 at $v_1$ and $v_0$. This results in a position with lists $L(v_1) = \{0, 3\}$, $L(v_2) = \{0, 1, 2\}$, $L(v_3) = \{2, 3, 4\}$, $L(v_4) = \{1, 2\}$ and $L(v_5) = \{0, 3\}$, but then Fixer wins by Case 205. If the components of $A_S$ have vertex sets $\{v_3\}$, $\{v_0, v_1\}$ and $\{v_2, v_4\}$, then Fixer should swap 1 and 3 at $v_3$. This results in a position with lists $L(v_1) = \{0, 1\}$, $L(v_2) = \{0, 2, 3\}$, $L(v_3) = \{2, 3, 4\}$, $L(v_4) = \{2, 3\}$ and $L(v_5) = \{0, 3\}$, but then Fixer wins by Case 232. If the components of $A_S$ have vertex sets $\{v_4\}$, $\{v_0, v_1\}$ and $\{v_2, v_3\}$, then Fixer should swap 1 and 3 at $v_1$ and $v_0$. This results in a position with lists $L(v_1) = \{0, 3\}$, $L(v_2) = \{0, 1, 2\}$, $L(v_3) = \{2, 3, 4\}$, $L(v_4) = \{1, 2\}$ and $L(v_5) = \{0, 3\}$, but then Fixer wins by Case 205. If the components of $A_S$ have vertex sets $\{v_2\}$, $\{v_0, v_3\}$ and $\{v_1, v_4\}$, then Fixer should swap 1 and 3 at $v_3$ and $v_0$. This results in a position with lists $L(v_1) = \{0, 3\}$, $L(v_2) = \{0, 2, 3\}$, $L(v_3) = \{2, 3, 4\}$, $L(v_4) = \{2, 3\}$ and $L(v_5) = \{0, 3\}$, but then Fixer can edge-color the graph. If the components of $A_S$ have vertex sets $\{v_2\}$, $\{v_0, v_4\}$ and $\{v_1, v_3\}$, then Fixer should swap 1 and 3 at $v_3$ and $v_1$. This results in a position with lists $L(v_1) = \{0, 1\}$, $L(v_2) = \{0, 1, 2\}$, $L(v_3) = \{2, 3, 4\}$, $L(v_4) = \{2, 3\}$ and $L(v_5) = \{0, 3\}$, but then Fixer wins by Case 101. If the components of $A_S$ have vertex sets $\{v_3\}$, $\{v_0, v_2\}$ and $\{v_1, v_4\}$, then Fixer should swap 1 and 3 at $v_3$. This results in a position with lists $L(v_1) = \{0, 1\}$, $L(v_2) = \{0, 2, 3\}$, $L(v_3) = \{2, 3, 4\}$, $L(v_4) = \{2, 3\}$ and $L(v_5) = \{0, 3\}$, but then Fixer wins by Case 232. If the components of $A_S$ have vertex sets $\{v_4\}$, $\{v_0, v_2\}$ and $\{v_1, v_3\}$, then Fixer should swap 1 and 3 at $v_3$ and $v_1$. This results in a position with lists $L(v_1) = \{0, 1\}$, $L(v_2) = \{0, 1, 2\}$, $L(v_3) = \{2, 3, 4\}$, $L(v_4) = \{2, 3\}$ and $L(v_5) = \{0, 3\}$, but then Fixer wins by Case 101. If the components of $A_S$ have vertex sets $\{v_3\}$, $\{v_0, v_4\}$ and $\{v_1, v_2\}$, then Fixer should swap 1 and 3 at $v_3$. This results in a position with lists $L(v_1) = \{0, 1\}$, $L(v_2) = \{0, 2, 3\}$, $L(v_3) = \{2, 3, 4\}$, $L(v_4) = \{2, 3\}$ and $L(v_5) = \{0, 3\}$, but then Fixer wins by Case 232. If the components of $A_S$ have vertex sets $\{v_4\}$, $\{v_0, v_3\}$ and $\{v_1, v_2\}$, then Fixer should swap 1 and 3 at $v_3$ and $v_0$. This results in a position with lists $L(v_1) = \{0, 3\}$, $L(v_2) = \{0, 2, 3\}$, $L(v_3) = \{2, 3, 4\}$, $L(v_4) = \{2, 3\}$ and $L(v_5) = \{0, 3\}$, but then Fixer can edge-color the graph. 

\noindent\textbf{Case 336.  }\textit{$L(v_1) = \{0, 1\}$, $L(v_2) = \{0, 2, 3\}$, $L(v_3) = \{0, 1, 4\}$, $L(v_4) = \{4, 5\}$ and $L(v_5) = \{4, 5\}$.}

Let $S$ and $A_S$ be as in Lemma \ref{MultiMoveCombination} using colors $1$ and $2$. If the components of $A_S$ have vertex sets $\{v_0\}$ and $\{v_1, v_2\}$, then Fixer should swap 1 and 2 at $v_0$. This results in a position with lists $L(v_1) = \{0, 2\}$, $L(v_2) = \{0, 2, 3\}$, $L(v_3) = \{0, 1, 4\}$, $L(v_4) = \{4, 5\}$ and $L(v_5) = \{4, 5\}$, but then Fixer wins by Case 241. If the components of $A_S$ have vertex sets $\{v_1\}$ and $\{v_0, v_2\}$, then Fixer should swap 1 and 2 at $v_1$. This results in a position with lists $L(v_1) = \{0, 1\}$, $L(v_2) = \{0, 1, 3\}$, $L(v_3) = \{0, 1, 4\}$, $L(v_4) = \{4, 5\}$ and $L(v_5) = \{4, 5\}$, but then Fixer can edge-color the graph. If the components of $A_S$ have vertex sets $\{v_2\}$ and $\{v_0, v_1\}$, then Fixer should swap 1 and 2 at $v_2$. This results in a position with lists $L(v_1) = \{0, 1\}$, $L(v_2) = \{0, 2, 3\}$, $L(v_3) = \{0, 2, 4\}$, $L(v_4) = \{4, 5\}$ and $L(v_5) = \{4, 5\}$, but then Fixer wins by Case 258. 

\noindent\textbf{Case 337.  }\textit{$L(v_1) = \{0, 1\}$, $L(v_2) = \{0, 1, 2\}$, $L(v_3) = \{0, 3, 4\}$, $L(v_4) = \{0, 3\}$ and $L(v_5) = \{1, 3\}$.}

Let $S$ and $A_S$ be as in Lemma \ref{MultiMoveCombination} using colors $0$ and $3$. If the components of $A_S$ have vertex sets $\{v_0\}$ and $\{v_1, v_4\}$, then Fixer should swap 0 and 3 at $v_0$. This results in a position with lists $L(v_1) = \{1, 3\}$, $L(v_2) = \{0, 1, 2\}$, $L(v_3) = \{0, 3, 4\}$, $L(v_4) = \{0, 3\}$ and $L(v_5) = \{1, 3\}$, but then Fixer wins by Case 329. If the components of $A_S$ have vertex sets $\{v_1\}$ and $\{v_0, v_4\}$, then Fixer should swap 0 and 3 at $v_1$. This results in a position with lists $L(v_1) = \{0, 1\}$, $L(v_2) = \{1, 2, 3\}$, $L(v_3) = \{0, 3, 4\}$, $L(v_4) = \{0, 3\}$ and $L(v_5) = \{1, 3\}$, but then Fixer wins by Case 330. If the components of $A_S$ have vertex sets $\{v_4\}$ and $\{v_0, v_1\}$, then Fixer should swap 0 and 3 at $v_4$. This results in a position with lists $L(v_1) = \{0, 1\}$, $L(v_2) = \{0, 1, 2\}$, $L(v_3) = \{0, 3, 4\}$, $L(v_4) = \{0, 3\}$ and $L(v_5) = \{0, 1\}$, but then Fixer wins by Case 78. 

\noindent\textbf{Case 338.  }\textit{$L(v_1) = \{0, 1\}$, $L(v_2) = \{0, 1, 2\}$, $L(v_3) = \{0, 3, 4\}$, $L(v_4) = \{1, 3\}$ and $L(v_5) = \{0, 3\}$.}

Let $S$ and $A_S$ be as in Lemma \ref{MultiMoveCombination} using colors $0$ and $3$. If the components of $A_S$ have vertex sets $\{v_0\}$ and $\{v_1, v_3\}$, then Fixer should swap 0 and 3 at $v_0$. This results in a position with lists $L(v_1) = \{1, 3\}$, $L(v_2) = \{0, 1, 2\}$, $L(v_3) = \{0, 3, 4\}$, $L(v_4) = \{1, 3\}$ and $L(v_5) = \{0, 3\}$, but then Fixer wins by Case 324. If the components of $A_S$ have vertex sets $\{v_1\}$ and $\{v_0, v_3\}$, then Fixer should swap 0 and 3 at $v_1$. This results in a position with lists $L(v_1) = \{0, 1\}$, $L(v_2) = \{1, 2, 3\}$, $L(v_3) = \{0, 3, 4\}$, $L(v_4) = \{1, 3\}$ and $L(v_5) = \{0, 3\}$, but then Fixer wins by Case 327. If the components of $A_S$ have vertex sets $\{v_3\}$ and $\{v_0, v_1\}$, then Fixer should swap 0 and 3 at $v_3$. This results in a position with lists $L(v_1) = \{0, 1\}$, $L(v_2) = \{0, 1, 2\}$, $L(v_3) = \{0, 3, 4\}$, $L(v_4) = \{0, 1\}$ and $L(v_5) = \{0, 3\}$, but then Fixer wins by Case 69. 

\noindent\textbf{Case 339.  }\textit{$L(v_1) = \{0, 1\}$, $L(v_2) = \{0, 2, 3\}$, $L(v_3) = \{0, 1, 4\}$, $L(v_4) = \{0, 2\}$ and $L(v_5) = \{1, 2\}$.}

Let $S$ and $A_S$ be as in Lemma \ref{MultiMoveCombination} using colors $0$ and $2$. If the components of $A_S$ have vertex sets $\{v_0\}$ and $\{v_2, v_4\}$, then Fixer should swap 0 and 2 at $v_0$. This results in a position with lists $L(v_1) = \{1, 2\}$, $L(v_2) = \{0, 2, 3\}$, $L(v_3) = \{0, 1, 4\}$, $L(v_4) = \{0, 2\}$ and $L(v_5) = \{1, 2\}$, but then Fixer wins by Case 325. If the components of $A_S$ have vertex sets $\{v_2\}$ and $\{v_0, v_4\}$, then Fixer should swap 0 and 2 at $v_2$. This results in a position with lists $L(v_1) = \{0, 1\}$, $L(v_2) = \{0, 2, 3\}$, $L(v_3) = \{1, 2, 4\}$, $L(v_4) = \{0, 2\}$ and $L(v_5) = \{1, 2\}$, but then Fixer wins by Case 327. If the components of $A_S$ have vertex sets $\{v_4\}$ and $\{v_0, v_2\}$, then Fixer should swap 0 and 2 at $v_4$. This results in a position with lists $L(v_1) = \{0, 1\}$, $L(v_2) = \{0, 2, 3\}$, $L(v_3) = \{0, 1, 4\}$, $L(v_4) = \{0, 2\}$ and $L(v_5) = \{0, 1\}$, but then Fixer wins by Case 302. 

\noindent\textbf{Case 340.  }\textit{$L(v_1) = \{0, 1\}$, $L(v_2) = \{0, 2, 3\}$, $L(v_3) = \{0, 1, 4\}$, $L(v_4) = \{1, 2\}$ and $L(v_5) = \{0, 2\}$.}

Let $S$ and $A_S$ be as in Lemma \ref{MultiMoveCombination} using colors $0$ and $2$. If the components of $A_S$ have vertex sets $\{v_0\}$ and $\{v_2, v_3\}$, then Fixer should swap 0 and 2 at $v_0$. This results in a position with lists $L(v_1) = \{1, 2\}$, $L(v_2) = \{0, 2, 3\}$, $L(v_3) = \{0, 1, 4\}$, $L(v_4) = \{1, 2\}$ and $L(v_5) = \{0, 2\}$, but then Fixer wins by Case 323. If the components of $A_S$ have vertex sets $\{v_2\}$ and $\{v_0, v_3\}$, then Fixer should swap 0 and 2 at $v_2$. This results in a position with lists $L(v_1) = \{0, 1\}$, $L(v_2) = \{0, 2, 3\}$, $L(v_3) = \{1, 2, 4\}$, $L(v_4) = \{1, 2\}$ and $L(v_5) = \{0, 2\}$, but then Fixer wins by Case 330. If the components of $A_S$ have vertex sets $\{v_3\}$ and $\{v_0, v_2\}$, then Fixer should swap 0 and 2 at $v_3$. This results in a position with lists $L(v_1) = \{0, 1\}$, $L(v_2) = \{0, 2, 3\}$, $L(v_3) = \{0, 1, 4\}$, $L(v_4) = \{0, 1\}$ and $L(v_5) = \{0, 2\}$, but then Fixer wins by Case 300. 

\noindent\textbf{Case 341.  }\textit{$L(v_1) = \{0, 1\}$, $L(v_2) = \{0, 2, 3\}$, $L(v_3) = \{0, 1, 4\}$, $L(v_4) = \{1, 2\}$ and $L(v_5) = \{1, 2\}$.}

Let $S$ and $A_S$ be as in Lemma \ref{MultiMoveCombination} using colors $0$ and $2$. If the components of $A_S$ have vertex sets $\{v_0, v_2\}$ and $\{v_3, v_4\}$, then Fixer should swap 0 and 2 at $v_2$ and $v_0$. This results in a position with lists $L(v_1) = \{1, 2\}$, $L(v_2) = \{0, 2, 3\}$, $L(v_3) = \{1, 2, 4\}$, $L(v_4) = \{1, 2\}$ and $L(v_5) = \{1, 2\}$, but then Fixer wins by Case 299. If the components of $A_S$ have vertex sets $\{v_0, v_3\}$ and $\{v_2, v_4\}$, then Fixer should swap 0 and 2 at $v_3$ and $v_0$. This results in a position with lists $L(v_1) = \{1, 2\}$, $L(v_2) = \{0, 2, 3\}$, $L(v_3) = \{0, 1, 4\}$, $L(v_4) = \{0, 1\}$ and $L(v_5) = \{1, 2\}$, but then Fixer wins by Case 329. If the components of $A_S$ have vertex sets $\{v_0, v_4\}$ and $\{v_2, v_3\}$, then Fixer should swap 0 and 2 at $v_4$ and $v_0$. This results in a position with lists $L(v_1) = \{1, 2\}$, $L(v_2) = \{0, 2, 3\}$, $L(v_3) = \{0, 1, 4\}$, $L(v_4) = \{1, 2\}$ and $L(v_5) = \{0, 1\}$, but then Fixer wins by Case 324. 

\noindent\textbf{Case 342.  }\textit{$L(v_1) = \{0, 1\}$, $L(v_2) = \{0, 2, 3\}$, $L(v_3) = \{0, 1, 4\}$, $L(v_4) = \{1, 4\}$ and $L(v_5) = \{2, 4\}$.}

Let $S$ and $A_S$ be as in Lemma \ref{MultiMoveCombination} using colors $1$ and $2$. If the components of $A_S$ have vertex sets $\{v_0\}$, $\{v_1, v_2\}$ and $\{v_3, v_4\}$, then Fixer should swap 1 and 2 at $v_0$. This results in a position with lists $L(v_1) = \{0, 2\}$, $L(v_2) = \{0, 2, 3\}$, $L(v_3) = \{0, 1, 4\}$, $L(v_4) = \{1, 4\}$ and $L(v_5) = \{2, 4\}$, but then Fixer can edge-color the graph. If the components of $A_S$ have vertex sets $\{v_0\}$, $\{v_1, v_3\}$ and $\{v_2, v_4\}$, then Fixer should swap 1 and 2 at $v_0$. This results in a position with lists $L(v_1) = \{0, 2\}$, $L(v_2) = \{0, 2, 3\}$, $L(v_3) = \{0, 1, 4\}$, $L(v_4) = \{1, 4\}$ and $L(v_5) = \{2, 4\}$, but then Fixer can edge-color the graph. If the components of $A_S$ have vertex sets $\{v_0\}$, $\{v_1, v_4\}$ and $\{v_2, v_3\}$, then Fixer should swap 1 and 2 at $v_0$. This results in a position with lists $L(v_1) = \{0, 2\}$, $L(v_2) = \{0, 2, 3\}$, $L(v_3) = \{0, 1, 4\}$, $L(v_4) = \{1, 4\}$ and $L(v_5) = \{2, 4\}$, but then Fixer can edge-color the graph. If the components of $A_S$ have vertex sets $\{v_1\}$, $\{v_0, v_2\}$ and $\{v_3, v_4\}$, then Fixer should swap 1 and 2 at $v_1$. This results in a position with lists $L(v_1) = \{0, 1\}$, $L(v_2) = \{0, 1, 3\}$, $L(v_3) = \{0, 1, 4\}$, $L(v_4) = \{1, 4\}$ and $L(v_5) = \{2, 4\}$, but then Fixer can edge-color the graph. If the components of $A_S$ have vertex sets $\{v_1\}$, $\{v_0, v_3\}$ and $\{v_2, v_4\}$, then Fixer should swap 1 and 2 at $v_1$. This results in a position with lists $L(v_1) = \{0, 1\}$, $L(v_2) = \{0, 1, 3\}$, $L(v_3) = \{0, 1, 4\}$, $L(v_4) = \{1, 4\}$ and $L(v_5) = \{2, 4\}$, but then Fixer can edge-color the graph. If the components of $A_S$ have vertex sets $\{v_1\}$, $\{v_0, v_4\}$ and $\{v_2, v_3\}$, then Fixer should swap 1 and 2 at $v_1$. This results in a position with lists $L(v_1) = \{0, 1\}$, $L(v_2) = \{0, 1, 3\}$, $L(v_3) = \{0, 1, 4\}$, $L(v_4) = \{1, 4\}$ and $L(v_5) = \{2, 4\}$, but then Fixer can edge-color the graph. If the components of $A_S$ have vertex sets $\{v_2\}$, $\{v_0, v_1\}$ and $\{v_3, v_4\}$, then Fixer should swap 1 and 2 at $v_2$. This results in a position with lists $L(v_1) = \{0, 1\}$, $L(v_2) = \{0, 2, 3\}$, $L(v_3) = \{0, 2, 4\}$, $L(v_4) = \{1, 4\}$ and $L(v_5) = \{2, 4\}$, but then Fixer wins by Case 321. If the components of $A_S$ have vertex sets $\{v_3\}$, $\{v_0, v_1\}$ and $\{v_2, v_4\}$, then Fixer should swap 1 and 2 at $v_3$. This results in a position with lists $L(v_1) = \{0, 1\}$, $L(v_2) = \{0, 2, 3\}$, $L(v_3) = \{0, 1, 4\}$, $L(v_4) = \{2, 4\}$ and $L(v_5) = \{2, 4\}$, but then Fixer wins by Case 311. If the components of $A_S$ have vertex sets $\{v_4\}$, $\{v_0, v_1\}$ and $\{v_2, v_3\}$, then Fixer should swap 1 and 2 at $v_4$. This results in a position with lists $L(v_1) = \{0, 1\}$, $L(v_2) = \{0, 2, 3\}$, $L(v_3) = \{0, 1, 4\}$, $L(v_4) = \{1, 4\}$ and $L(v_5) = \{1, 4\}$, but then Fixer wins by Case 309. If the components of $A_S$ have vertex sets $\{v_2\}$, $\{v_0, v_3\}$ and $\{v_1, v_4\}$, then Fixer should swap 1 and 2 at $v_2$. This results in a position with lists $L(v_1) = \{0, 1\}$, $L(v_2) = \{0, 2, 3\}$, $L(v_3) = \{0, 2, 4\}$, $L(v_4) = \{1, 4\}$ and $L(v_5) = \{2, 4\}$, but then Fixer wins by Case 321. If the components of $A_S$ have vertex sets $\{v_2\}$, $\{v_0, v_4\}$ and $\{v_1, v_3\}$, then Fixer should swap 1 and 2 at $v_2$. This results in a position with lists $L(v_1) = \{0, 1\}$, $L(v_2) = \{0, 2, 3\}$, $L(v_3) = \{0, 2, 4\}$, $L(v_4) = \{1, 4\}$ and $L(v_5) = \{2, 4\}$, but then Fixer wins by Case 321. If the components of $A_S$ have vertex sets $\{v_3\}$, $\{v_0, v_2\}$ and $\{v_1, v_4\}$, then Fixer should swap 1 and 2 at $v_3$. This results in a position with lists $L(v_1) = \{0, 1\}$, $L(v_2) = \{0, 2, 3\}$, $L(v_3) = \{0, 1, 4\}$, $L(v_4) = \{2, 4\}$ and $L(v_5) = \{2, 4\}$, but then Fixer wins by Case 311. If the components of $A_S$ have vertex sets $\{v_4\}$, $\{v_0, v_2\}$ and $\{v_1, v_3\}$, then Fixer should swap 1 and 2 at $v_4$. This results in a position with lists $L(v_1) = \{0, 1\}$, $L(v_2) = \{0, 2, 3\}$, $L(v_3) = \{0, 1, 4\}$, $L(v_4) = \{1, 4\}$ and $L(v_5) = \{1, 4\}$, but then Fixer wins by Case 309. If the components of $A_S$ have vertex sets $\{v_3\}$, $\{v_0, v_4\}$ and $\{v_1, v_2\}$, then Fixer should swap 1 and 2 at $v_3$. This results in a position with lists $L(v_1) = \{0, 1\}$, $L(v_2) = \{0, 2, 3\}$, $L(v_3) = \{0, 1, 4\}$, $L(v_4) = \{2, 4\}$ and $L(v_5) = \{2, 4\}$, but then Fixer wins by Case 311. If the components of $A_S$ have vertex sets $\{v_4\}$, $\{v_0, v_3\}$ and $\{v_1, v_2\}$, then Fixer should swap 1 and 2 at $v_4$. This results in a position with lists $L(v_1) = \{0, 1\}$, $L(v_2) = \{0, 2, 3\}$, $L(v_3) = \{0, 1, 4\}$, $L(v_4) = \{1, 4\}$ and $L(v_5) = \{1, 4\}$, but then Fixer wins by Case 309. 

\noindent\textbf{Case 343.  }\textit{$L(v_1) = \{0, 1\}$, $L(v_2) = \{0, 2, 3\}$, $L(v_3) = \{0, 1, 4\}$, $L(v_4) = \{2, 4\}$ and $L(v_5) = \{1, 4\}$.}

Let $S$ and $A_S$ be as in Lemma \ref{MultiMoveCombination} using colors $1$ and $2$. If the components of $A_S$ have vertex sets $\{v_0\}$, $\{v_1, v_2\}$ and $\{v_3, v_4\}$, then Fixer should swap 1 and 2 at $v_0$. This results in a position with lists $L(v_1) = \{0, 2\}$, $L(v_2) = \{0, 2, 3\}$, $L(v_3) = \{0, 1, 4\}$, $L(v_4) = \{2, 4\}$ and $L(v_5) = \{1, 4\}$, but then Fixer can edge-color the graph. If the components of $A_S$ have vertex sets $\{v_0\}$, $\{v_1, v_3\}$ and $\{v_2, v_4\}$, then Fixer should swap 1 and 2 at $v_0$. This results in a position with lists $L(v_1) = \{0, 2\}$, $L(v_2) = \{0, 2, 3\}$, $L(v_3) = \{0, 1, 4\}$, $L(v_4) = \{2, 4\}$ and $L(v_5) = \{1, 4\}$, but then Fixer can edge-color the graph. If the components of $A_S$ have vertex sets $\{v_0\}$, $\{v_1, v_4\}$ and $\{v_2, v_3\}$, then Fixer should swap 1 and 2 at $v_0$. This results in a position with lists $L(v_1) = \{0, 2\}$, $L(v_2) = \{0, 2, 3\}$, $L(v_3) = \{0, 1, 4\}$, $L(v_4) = \{2, 4\}$ and $L(v_5) = \{1, 4\}$, but then Fixer can edge-color the graph. If the components of $A_S$ have vertex sets $\{v_1\}$, $\{v_0, v_2\}$ and $\{v_3, v_4\}$, then Fixer should swap 1 and 2 at $v_1$. This results in a position with lists $L(v_1) = \{0, 1\}$, $L(v_2) = \{0, 1, 3\}$, $L(v_3) = \{0, 1, 4\}$, $L(v_4) = \{2, 4\}$ and $L(v_5) = \{1, 4\}$, but then Fixer can edge-color the graph. If the components of $A_S$ have vertex sets $\{v_1\}$, $\{v_0, v_3\}$ and $\{v_2, v_4\}$, then Fixer should swap 1 and 2 at $v_1$. This results in a position with lists $L(v_1) = \{0, 1\}$, $L(v_2) = \{0, 1, 3\}$, $L(v_3) = \{0, 1, 4\}$, $L(v_4) = \{2, 4\}$ and $L(v_5) = \{1, 4\}$, but then Fixer can edge-color the graph. If the components of $A_S$ have vertex sets $\{v_1\}$, $\{v_0, v_4\}$ and $\{v_2, v_3\}$, then Fixer should swap 1 and 2 at $v_1$. This results in a position with lists $L(v_1) = \{0, 1\}$, $L(v_2) = \{0, 1, 3\}$, $L(v_3) = \{0, 1, 4\}$, $L(v_4) = \{2, 4\}$ and $L(v_5) = \{1, 4\}$, but then Fixer can edge-color the graph. If the components of $A_S$ have vertex sets $\{v_2\}$, $\{v_0, v_1\}$ and $\{v_3, v_4\}$, then Fixer should swap 1 and 2 at $v_2$. This results in a position with lists $L(v_1) = \{0, 1\}$, $L(v_2) = \{0, 2, 3\}$, $L(v_3) = \{0, 2, 4\}$, $L(v_4) = \{2, 4\}$ and $L(v_5) = \{1, 4\}$, but then Fixer wins by Case 322. If the components of $A_S$ have vertex sets $\{v_3\}$, $\{v_0, v_1\}$ and $\{v_2, v_4\}$, then Fixer should swap 1 and 2 at $v_3$. This results in a position with lists $L(v_1) = \{0, 1\}$, $L(v_2) = \{0, 2, 3\}$, $L(v_3) = \{0, 1, 4\}$, $L(v_4) = \{1, 4\}$ and $L(v_5) = \{1, 4\}$, but then Fixer wins by Case 309. If the components of $A_S$ have vertex sets $\{v_4\}$, $\{v_0, v_1\}$ and $\{v_2, v_3\}$, then Fixer should swap 1 and 2 at $v_4$. This results in a position with lists $L(v_1) = \{0, 1\}$, $L(v_2) = \{0, 2, 3\}$, $L(v_3) = \{0, 1, 4\}$, $L(v_4) = \{2, 4\}$ and $L(v_5) = \{2, 4\}$, but then Fixer wins by Case 311. If the components of $A_S$ have vertex sets $\{v_2\}$, $\{v_0, v_3\}$ and $\{v_1, v_4\}$, then Fixer should swap 1 and 2 at $v_2$. This results in a position with lists $L(v_1) = \{0, 1\}$, $L(v_2) = \{0, 2, 3\}$, $L(v_3) = \{0, 2, 4\}$, $L(v_4) = \{2, 4\}$ and $L(v_5) = \{1, 4\}$, but then Fixer wins by Case 322. If the components of $A_S$ have vertex sets $\{v_2\}$, $\{v_0, v_4\}$ and $\{v_1, v_3\}$, then Fixer should swap 1 and 2 at $v_2$. This results in a position with lists $L(v_1) = \{0, 1\}$, $L(v_2) = \{0, 2, 3\}$, $L(v_3) = \{0, 2, 4\}$, $L(v_4) = \{2, 4\}$ and $L(v_5) = \{1, 4\}$, but then Fixer wins by Case 322. If the components of $A_S$ have vertex sets $\{v_3\}$, $\{v_0, v_2\}$ and $\{v_1, v_4\}$, then Fixer should swap 1 and 2 at $v_3$. This results in a position with lists $L(v_1) = \{0, 1\}$, $L(v_2) = \{0, 2, 3\}$, $L(v_3) = \{0, 1, 4\}$, $L(v_4) = \{1, 4\}$ and $L(v_5) = \{1, 4\}$, but then Fixer wins by Case 309. If the components of $A_S$ have vertex sets $\{v_4\}$, $\{v_0, v_2\}$ and $\{v_1, v_3\}$, then Fixer should swap 1 and 2 at $v_4$. This results in a position with lists $L(v_1) = \{0, 1\}$, $L(v_2) = \{0, 2, 3\}$, $L(v_3) = \{0, 1, 4\}$, $L(v_4) = \{2, 4\}$ and $L(v_5) = \{2, 4\}$, but then Fixer wins by Case 311. If the components of $A_S$ have vertex sets $\{v_3\}$, $\{v_0, v_4\}$ and $\{v_1, v_2\}$, then Fixer should swap 1 and 2 at $v_3$. This results in a position with lists $L(v_1) = \{0, 1\}$, $L(v_2) = \{0, 2, 3\}$, $L(v_3) = \{0, 1, 4\}$, $L(v_4) = \{1, 4\}$ and $L(v_5) = \{1, 4\}$, but then Fixer wins by Case 309. If the components of $A_S$ have vertex sets $\{v_4\}$, $\{v_0, v_3\}$ and $\{v_1, v_2\}$, then Fixer should swap 1 and 2 at $v_4$. This results in a position with lists $L(v_1) = \{0, 1\}$, $L(v_2) = \{0, 2, 3\}$, $L(v_3) = \{0, 1, 4\}$, $L(v_4) = \{2, 4\}$ and $L(v_5) = \{2, 4\}$, but then Fixer wins by Case 311. 

\noindent\textbf{Case 344.  }\textit{$L(v_1) = \{0, 1\}$, $L(v_2) = \{0, 2, 3\}$, $L(v_3) = \{0, 1, 4\}$, $L(v_4) = \{0, 4\}$ and $L(v_5) = \{1, 4\}$.}

Let $S$ and $A_S$ be as in Lemma \ref{MultiMoveCombination} using colors $0$ and $2$. If the components of $A_S$ have vertex sets $\{v_0\}$ and $\{v_2, v_3\}$, then Fixer should swap 0 and 2 at $v_0$. This results in a position with lists $L(v_1) = \{1, 2\}$, $L(v_2) = \{0, 2, 3\}$, $L(v_3) = \{0, 1, 4\}$, $L(v_4) = \{0, 4\}$ and $L(v_5) = \{1, 4\}$, but then Fixer can edge-color the graph. If the components of $A_S$ have vertex sets $\{v_2\}$ and $\{v_0, v_3\}$, then Fixer should swap 0 and 2 at $v_2$. This results in a position with lists $L(v_1) = \{0, 1\}$, $L(v_2) = \{0, 2, 3\}$, $L(v_3) = \{1, 2, 4\}$, $L(v_4) = \{0, 4\}$ and $L(v_5) = \{1, 4\}$, but then Fixer can edge-color the graph. If the components of $A_S$ have vertex sets $\{v_3\}$ and $\{v_0, v_2\}$, then Fixer should swap 0 and 2 at $v_3$. This results in a position with lists $L(v_1) = \{0, 1\}$, $L(v_2) = \{0, 2, 3\}$, $L(v_3) = \{0, 1, 4\}$, $L(v_4) = \{2, 4\}$ and $L(v_5) = \{1, 4\}$, but then Fixer wins by Case 343. 

\noindent\textbf{Case 345.  }\textit{$L(v_1) = \{0, 1\}$, $L(v_2) = \{0, 2, 3\}$, $L(v_3) = \{0, 1, 4\}$, $L(v_4) = \{1, 4\}$ and $L(v_5) = \{0, 4\}$.}

Let $S$ and $A_S$ be as in Lemma \ref{MultiMoveCombination} using colors $0$ and $2$. If the components of $A_S$ have vertex sets $\{v_0\}$ and $\{v_2, v_4\}$, then Fixer should swap 0 and 2 at $v_0$. This results in a position with lists $L(v_1) = \{1, 2\}$, $L(v_2) = \{0, 2, 3\}$, $L(v_3) = \{0, 1, 4\}$, $L(v_4) = \{1, 4\}$ and $L(v_5) = \{0, 4\}$, but then Fixer can edge-color the graph. If the components of $A_S$ have vertex sets $\{v_2\}$ and $\{v_0, v_4\}$, then Fixer should swap 0 and 2 at $v_2$. This results in a position with lists $L(v_1) = \{0, 1\}$, $L(v_2) = \{0, 2, 3\}$, $L(v_3) = \{1, 2, 4\}$, $L(v_4) = \{1, 4\}$ and $L(v_5) = \{0, 4\}$, but then Fixer can edge-color the graph. If the components of $A_S$ have vertex sets $\{v_4\}$ and $\{v_0, v_2\}$, then Fixer should swap 0 and 2 at $v_4$. This results in a position with lists $L(v_1) = \{0, 1\}$, $L(v_2) = \{0, 2, 3\}$, $L(v_3) = \{0, 1, 4\}$, $L(v_4) = \{1, 4\}$ and $L(v_5) = \{2, 4\}$, but then Fixer wins by Case 342. 

\end{proof}


\begin{lem}\label{chair2}
Fixer has a winning strategy against Breaker in the chronicled game on the chair, with superabundant list assignment $L$ where $|L(v_1)| = 2$, $|L(v_2)| = 2$, $|L(v_3)| = 4$, $|L(v_4)| = 2$ and $|L(v_5)| = 2$.
\end{lem}
\begin{proof}
We show that for each possible such list assignment $L$ on $G$, Fixer has a winning strategy.
Up to symmetry, the following cases cover all the possible list assignments that are not an immediate win for Fixer.

\noindent\textbf{Case 1.  }\textit{$L(v_1) = \{0, 1\}$, $L(v_2) = \{0, 1\}$, $L(v_3) = \{0, 1, 2, 3\}$, $L(v_4) = \{0, 1\}$ and $L(v_5) = \{0, 1\}$.}

Fixer gets a winning strategy by coloring $v_1v_2$ with $0$ and applying Lemma \ref{CanColorAndPlayOnRest}.

\noindent\textbf{Case 2.  }\textit{$L(v_1) = \{0, 1\}$, $L(v_2) = \{0, 2\}$, $L(v_3) = \{0, 1, 2, 3\}$, $L(v_4) = \{0, 2\}$ and $L(v_5) = \{0, 2\}$.}

Fixer gets a winning strategy by coloring $v_1v_2$ with $0$ and applying Lemma \ref{CanColorAndPlayOnRest}.

\noindent\textbf{Case 3.  }\textit{$L(v_1) = \{0, 1\}$, $L(v_2) = \{0, 2\}$, $L(v_3) = \{0, 1, 2, 3\}$, $L(v_4) = \{1, 2\}$ and $L(v_5) = \{1, 2\}$.}

Fixer gets a winning strategy by coloring $v_1v_2$ with $0$ and applying Lemma \ref{CanColorAndPlayOnRest}.

\noindent\textbf{Case 4.  }\textit{$L(v_1) = \{0, 1\}$, $L(v_2) = \{0, 2\}$, $L(v_3) = \{0, 1, 2, 3\}$, $L(v_4) = \{2, 3\}$ and $L(v_5) = \{2, 3\}$.}

Fixer gets a winning strategy by coloring $v_1v_2$ with $0$ and applying Lemma \ref{CanColorAndPlayOnRest}.

\noindent\textbf{Case 5.  }\textit{$L(v_1) = \{0, 1\}$, $L(v_2) = \{0, 1\}$, $L(v_3) = \{0, 1, 2, 3\}$, $L(v_4) = \{0, 1\}$ and $L(v_5) = \{0, 4\}$.}

Fixer gets a winning strategy by coloring $v_5v_3$ with $0$ and applying Lemma \ref{CanColorAndPlayOnRest}.

\noindent\textbf{Case 6.  }\textit{$L(v_1) = \{0, 1\}$, $L(v_2) = \{0, 1\}$, $L(v_3) = \{0, 1, 2, 3\}$, $L(v_4) = \{0, 4\}$ and $L(v_5) = \{0, 1\}$.}

Fixer gets a winning strategy by coloring $v_4v_3$ with $0$ and applying Lemma \ref{CanColorAndPlayOnRest}.

\noindent\textbf{Case 7.  }\textit{$L(v_1) = \{0, 1\}$, $L(v_2) = \{0, 1\}$, $L(v_3) = \{0, 1, 2, 3\}$, $L(v_4) = \{0, 4\}$ and $L(v_5) = \{0, 4\}$.}

Fixer gets a winning strategy by coloring $v_1v_2$ with $0$ and applying Lemma \ref{CanColorAndPlayOnRest}.

\noindent\textbf{Case 8.  }\textit{$L(v_1) = \{0, 1\}$, $L(v_2) = \{0, 1\}$, $L(v_3) = \{0, 1, 2, 3\}$, $L(v_4) = \{2, 4\}$ and $L(v_5) = \{2, 4\}$.}

Fixer gets a winning strategy by coloring $v_1v_2$ with $0$ and applying Lemma \ref{CanColorAndPlayOnRest}.

\noindent\textbf{Case 9.  }\textit{$L(v_1) = \{0, 1\}$, $L(v_2) = \{0, 1\}$, $L(v_3) = \{0, 2, 3, 4\}$, $L(v_4) = \{0, 1\}$ and $L(v_5) = \{0, 2\}$.}

Fixer gets a winning strategy by coloring $v_5v_3$ with $2$ and applying Lemma \ref{CanColorAndPlayOnRest}.

\noindent\textbf{Case 10.  }\textit{$L(v_1) = \{0, 1\}$, $L(v_2) = \{0, 1\}$, $L(v_3) = \{0, 2, 3, 4\}$, $L(v_4) = \{0, 1\}$ and $L(v_5) = \{1, 2\}$.}

Fixer gets a winning strategy by coloring $v_5v_3$ with $2$ and applying Lemma \ref{CanColorAndPlayOnRest}.

\noindent\textbf{Case 11.  }\textit{$L(v_1) = \{0, 1\}$, $L(v_2) = \{0, 1\}$, $L(v_3) = \{0, 2, 3, 4\}$, $L(v_4) = \{0, 1\}$ and $L(v_5) = \{2, 3\}$.}

Fixer gets a winning strategy by coloring $v_5v_3$ with $2$ and applying Lemma \ref{CanColorAndPlayOnRest}.

\noindent\textbf{Case 12.  }\textit{$L(v_1) = \{0, 1\}$, $L(v_2) = \{0, 1\}$, $L(v_3) = \{0, 2, 3, 4\}$, $L(v_4) = \{0, 2\}$ and $L(v_5) = \{0, 1\}$.}

Fixer gets a winning strategy by coloring $v_4v_3$ with $2$ and applying Lemma \ref{CanColorAndPlayOnRest}.

\noindent\textbf{Case 13.  }\textit{$L(v_1) = \{0, 1\}$, $L(v_2) = \{0, 1\}$, $L(v_3) = \{0, 2, 3, 4\}$, $L(v_4) = \{0, 2\}$ and $L(v_5) = \{0, 2\}$.}

Fixer gets a winning strategy by coloring $v_1v_2$ with $1$ and applying Lemma \ref{CanColorAndPlayOnRest}.

\noindent\textbf{Case 14.  }\textit{$L(v_1) = \{0, 1\}$, $L(v_2) = \{0, 1\}$, $L(v_3) = \{0, 2, 3, 4\}$, $L(v_4) = \{1, 2\}$ and $L(v_5) = \{0, 1\}$.}

Fixer gets a winning strategy by coloring $v_4v_3$ with $2$ and applying Lemma \ref{CanColorAndPlayOnRest}.

\noindent\textbf{Case 15.  }\textit{$L(v_1) = \{0, 1\}$, $L(v_2) = \{0, 1\}$, $L(v_3) = \{0, 2, 3, 4\}$, $L(v_4) = \{1, 2\}$ and $L(v_5) = \{1, 2\}$.}

Fixer gets a winning strategy by coloring $v_1v_2$ with $1$ and applying Lemma \ref{CanColorAndPlayOnRest}.

\noindent\textbf{Case 16.  }\textit{$L(v_1) = \{0, 1\}$, $L(v_2) = \{0, 1\}$, $L(v_3) = \{0, 2, 3, 4\}$, $L(v_4) = \{2, 3\}$ and $L(v_5) = \{0, 1\}$.}

Fixer gets a winning strategy by coloring $v_4v_3$ with $2$ and applying Lemma \ref{CanColorAndPlayOnRest}.

\noindent\textbf{Case 17.  }\textit{$L(v_1) = \{0, 1\}$, $L(v_2) = \{0, 2\}$, $L(v_3) = \{0, 1, 2, 3\}$, $L(v_4) = \{0, 1\}$ and $L(v_5) = \{2, 4\}$.}

Fixer gets a winning strategy by coloring $v_5v_3$ with $2$ and applying Lemma \ref{CanColorAndPlayOnRest}.

\noindent\textbf{Case 18.  }\textit{$L(v_1) = \{0, 1\}$, $L(v_2) = \{0, 2\}$, $L(v_3) = \{0, 1, 2, 3\}$, $L(v_4) = \{0, 2\}$ and $L(v_5) = \{0, 4\}$.}

Fixer gets a winning strategy by coloring $v_5v_3$ with $0$ and applying Lemma \ref{CanColorAndPlayOnRest}.

\noindent\textbf{Case 19.  }\textit{$L(v_1) = \{0, 1\}$, $L(v_2) = \{0, 2\}$, $L(v_3) = \{0, 1, 2, 3\}$, $L(v_4) = \{0, 2\}$ and $L(v_5) = \{2, 4\}$.}

Fixer gets a winning strategy by coloring $v_5v_3$ with $2$ and applying Lemma \ref{CanColorAndPlayOnRest}.

\noindent\textbf{Case 20.  }\textit{$L(v_1) = \{0, 1\}$, $L(v_2) = \{0, 2\}$, $L(v_3) = \{0, 1, 2, 3\}$, $L(v_4) = \{1, 2\}$ and $L(v_5) = \{1, 4\}$.}

Fixer gets a winning strategy by coloring $v_5v_3$ with $1$ and applying Lemma \ref{CanColorAndPlayOnRest}.

\noindent\textbf{Case 21.  }\textit{$L(v_1) = \{0, 1\}$, $L(v_2) = \{0, 2\}$, $L(v_3) = \{0, 1, 2, 3\}$, $L(v_4) = \{1, 2\}$ and $L(v_5) = \{2, 4\}$.}

Fixer gets a winning strategy by coloring $v_5v_3$ with $2$ and applying Lemma \ref{CanColorAndPlayOnRest}.

\noindent\textbf{Case 22.  }\textit{$L(v_1) = \{0, 1\}$, $L(v_2) = \{0, 2\}$, $L(v_3) = \{0, 1, 2, 3\}$, $L(v_4) = \{0, 3\}$ and $L(v_5) = \{2, 4\}$.}

Fixer gets a winning strategy by coloring $v_4v_3$ with $3$ and applying Lemma \ref{CanColorAndPlayOnRest}.

\noindent\textbf{Case 23.  }\textit{$L(v_1) = \{0, 1\}$, $L(v_2) = \{0, 2\}$, $L(v_3) = \{0, 1, 2, 3\}$, $L(v_4) = \{1, 3\}$ and $L(v_5) = \{2, 4\}$.}

Fixer gets a winning strategy by coloring $v_4v_3$ with $3$ and applying Lemma \ref{CanColorAndPlayOnRest}.

\noindent\textbf{Case 24.  }\textit{$L(v_1) = \{0, 1\}$, $L(v_2) = \{0, 2\}$, $L(v_3) = \{0, 1, 2, 3\}$, $L(v_4) = \{2, 3\}$ and $L(v_5) = \{2, 4\}$.}

Fixer gets a winning strategy by coloring $v_4v_3$ with $3$ and applying Lemma \ref{CanColorAndPlayOnRest}.

\noindent\textbf{Case 25.  }\textit{$L(v_1) = \{0, 1\}$, $L(v_2) = \{0, 2\}$, $L(v_3) = \{0, 1, 2, 3\}$, $L(v_4) = \{2, 3\}$ and $L(v_5) = \{3, 4\}$.}

Fixer gets a winning strategy by coloring $v_4v_3$ with $2$ and applying Lemma \ref{CanColorAndPlayOnRest}.

\noindent\textbf{Case 26.  }\textit{$L(v_1) = \{0, 1\}$, $L(v_2) = \{0, 2\}$, $L(v_3) = \{0, 1, 2, 3\}$, $L(v_4) = \{0, 4\}$ and $L(v_5) = \{0, 2\}$.}

Fixer gets a winning strategy by coloring $v_4v_3$ with $0$ and applying Lemma \ref{CanColorAndPlayOnRest}.

\noindent\textbf{Case 27.  }\textit{$L(v_1) = \{0, 1\}$, $L(v_2) = \{0, 2\}$, $L(v_3) = \{0, 1, 2, 3\}$, $L(v_4) = \{0, 4\}$ and $L(v_5) = \{0, 4\}$.}

Fixer gets a winning strategy by coloring $v_1v_2$ with $0$ and applying Lemma \ref{CanColorAndPlayOnRest}.

\noindent\textbf{Case 28.  }\textit{$L(v_1) = \{0, 1\}$, $L(v_2) = \{0, 2\}$, $L(v_3) = \{0, 1, 2, 3\}$, $L(v_4) = \{1, 4\}$ and $L(v_5) = \{1, 2\}$.}

Fixer gets a winning strategy by coloring $v_4v_3$ with $1$ and applying Lemma \ref{CanColorAndPlayOnRest}.

\noindent\textbf{Case 29.  }\textit{$L(v_1) = \{0, 1\}$, $L(v_2) = \{0, 2\}$, $L(v_3) = \{0, 1, 2, 3\}$, $L(v_4) = \{1, 4\}$ and $L(v_5) = \{1, 4\}$.}

Fixer gets a winning strategy by coloring $v_1v_2$ with $0$ and applying Lemma \ref{CanColorAndPlayOnRest}.

\noindent\textbf{Case 30.  }\textit{$L(v_1) = \{0, 1\}$, $L(v_2) = \{0, 2\}$, $L(v_3) = \{0, 1, 2, 3\}$, $L(v_4) = \{2, 4\}$ and $L(v_5) = \{0, 1\}$.}

Fixer gets a winning strategy by coloring $v_4v_3$ with $2$ and applying Lemma \ref{CanColorAndPlayOnRest}.

\noindent\textbf{Case 31.  }\textit{$L(v_1) = \{0, 1\}$, $L(v_2) = \{0, 2\}$, $L(v_3) = \{0, 1, 2, 3\}$, $L(v_4) = \{2, 4\}$ and $L(v_5) = \{0, 2\}$.}

Fixer gets a winning strategy by coloring $v_4v_3$ with $2$ and applying Lemma \ref{CanColorAndPlayOnRest}.

\noindent\textbf{Case 32.  }\textit{$L(v_1) = \{0, 1\}$, $L(v_2) = \{0, 2\}$, $L(v_3) = \{0, 1, 2, 3\}$, $L(v_4) = \{2, 4\}$ and $L(v_5) = \{1, 2\}$.}

Fixer gets a winning strategy by coloring $v_4v_3$ with $2$ and applying Lemma \ref{CanColorAndPlayOnRest}.

\noindent\textbf{Case 33.  }\textit{$L(v_1) = \{0, 1\}$, $L(v_2) = \{0, 2\}$, $L(v_3) = \{0, 1, 2, 3\}$, $L(v_4) = \{2, 4\}$ and $L(v_5) = \{0, 3\}$.}

Fixer gets a winning strategy by coloring $v_4v_3$ with $2$ and applying Lemma \ref{CanColorAndPlayOnRest}.

\noindent\textbf{Case 34.  }\textit{$L(v_1) = \{0, 1\}$, $L(v_2) = \{0, 2\}$, $L(v_3) = \{0, 1, 2, 3\}$, $L(v_4) = \{2, 4\}$ and $L(v_5) = \{1, 3\}$.}

Fixer gets a winning strategy by coloring $v_4v_3$ with $2$ and applying Lemma \ref{CanColorAndPlayOnRest}.

\noindent\textbf{Case 35.  }\textit{$L(v_1) = \{0, 1\}$, $L(v_2) = \{0, 2\}$, $L(v_3) = \{0, 1, 2, 3\}$, $L(v_4) = \{2, 4\}$ and $L(v_5) = \{2, 3\}$.}

Fixer gets a winning strategy by coloring $v_4v_3$ with $2$ and applying Lemma \ref{CanColorAndPlayOnRest}.

\noindent\textbf{Case 36.  }\textit{$L(v_1) = \{0, 1\}$, $L(v_2) = \{0, 2\}$, $L(v_3) = \{0, 1, 2, 3\}$, $L(v_4) = \{2, 4\}$ and $L(v_5) = \{3, 4\}$.}

Fixer gets a winning strategy by coloring $v_4v_3$ with $2$ and applying Lemma \ref{CanColorAndPlayOnRest}.

\noindent\textbf{Case 37.  }\textit{$L(v_1) = \{0, 1\}$, $L(v_2) = \{0, 2\}$, $L(v_3) = \{0, 1, 2, 3\}$, $L(v_4) = \{3, 4\}$ and $L(v_5) = \{2, 3\}$.}

Fixer gets a winning strategy by coloring $v_4v_3$ with $3$ and applying Lemma \ref{CanColorAndPlayOnRest}.

\noindent\textbf{Case 38.  }\textit{$L(v_1) = \{0, 1\}$, $L(v_2) = \{0, 2\}$, $L(v_3) = \{0, 1, 2, 3\}$, $L(v_4) = \{3, 4\}$ and $L(v_5) = \{2, 4\}$.}

Fixer gets a winning strategy by coloring $v_4v_3$ with $3$ and applying Lemma \ref{CanColorAndPlayOnRest}.

\noindent\textbf{Case 39.  }\textit{$L(v_1) = \{0, 1\}$, $L(v_2) = \{0, 2\}$, $L(v_3) = \{0, 1, 2, 3\}$, $L(v_4) = \{3, 4\}$ and $L(v_5) = \{3, 4\}$.}

Fixer gets a winning strategy by coloring $v_1v_2$ with $0$ and applying Lemma \ref{CanColorAndPlayOnRest}.

\noindent\textbf{Case 40.  }\textit{$L(v_1) = \{0, 1\}$, $L(v_2) = \{0, 2\}$, $L(v_3) = \{0, 1, 3, 4\}$, $L(v_4) = \{0, 1\}$ and $L(v_5) = \{0, 3\}$.}

Fixer gets a winning strategy by coloring $v_5v_3$ with $3$ and applying Lemma \ref{CanColorAndPlayOnRest}.

\noindent\textbf{Case 41.  }\textit{$L(v_1) = \{0, 1\}$, $L(v_2) = \{0, 2\}$, $L(v_3) = \{0, 1, 3, 4\}$, $L(v_4) = \{0, 1\}$ and $L(v_5) = \{1, 3\}$.}

Fixer gets a winning strategy by coloring $v_5v_3$ with $3$ and applying Lemma \ref{CanColorAndPlayOnRest}.

\noindent\textbf{Case 42.  }\textit{$L(v_1) = \{0, 1\}$, $L(v_2) = \{0, 2\}$, $L(v_3) = \{0, 1, 3, 4\}$, $L(v_4) = \{0, 1\}$ and $L(v_5) = \{2, 3\}$.}

Fixer gets a winning strategy by coloring $v_5v_3$ with $3$ and applying Lemma \ref{CanColorAndPlayOnRest}.

\noindent\textbf{Case 43.  }\textit{$L(v_1) = \{0, 1\}$, $L(v_2) = \{0, 2\}$, $L(v_3) = \{0, 1, 3, 4\}$, $L(v_4) = \{0, 1\}$ and $L(v_5) = \{3, 4\}$.}

Fixer gets a winning strategy by coloring $v_5v_3$ with $3$ and applying Lemma \ref{CanColorAndPlayOnRest}.

\noindent\textbf{Case 44.  }\textit{$L(v_1) = \{0, 1\}$, $L(v_2) = \{0, 2\}$, $L(v_3) = \{0, 1, 3, 4\}$, $L(v_4) = \{0, 2\}$ and $L(v_5) = \{0, 3\}$.}

Fixer gets a winning strategy by coloring $v_5v_3$ with $3$ and applying Lemma \ref{CanColorAndPlayOnRest}.

\noindent\textbf{Case 45.  }\textit{$L(v_1) = \{0, 1\}$, $L(v_2) = \{0, 2\}$, $L(v_3) = \{0, 1, 3, 4\}$, $L(v_4) = \{0, 2\}$ and $L(v_5) = \{1, 3\}$.}

Fixer gets a winning strategy by coloring $v_5v_3$ with $3$ and applying Lemma \ref{CanColorAndPlayOnRest}.

\noindent\textbf{Case 46.  }\textit{$L(v_1) = \{0, 1\}$, $L(v_2) = \{0, 2\}$, $L(v_3) = \{0, 1, 3, 4\}$, $L(v_4) = \{0, 2\}$ and $L(v_5) = \{2, 3\}$.}

Fixer gets a winning strategy by coloring $v_5v_3$ with $3$ and applying Lemma \ref{CanColorAndPlayOnRest}.

\noindent\textbf{Case 47.  }\textit{$L(v_1) = \{0, 1\}$, $L(v_2) = \{0, 2\}$, $L(v_3) = \{0, 1, 3, 4\}$, $L(v_4) = \{0, 2\}$ and $L(v_5) = \{3, 4\}$.}

Fixer gets a winning strategy by coloring $v_5v_3$ with $3$ and applying Lemma \ref{CanColorAndPlayOnRest}.

\noindent\textbf{Case 48.  }\textit{$L(v_1) = \{0, 1\}$, $L(v_2) = \{0, 2\}$, $L(v_3) = \{0, 1, 3, 4\}$, $L(v_4) = \{1, 2\}$ and $L(v_5) = \{0, 3\}$.}

Fixer gets a winning strategy by coloring $v_5v_3$ with $3$ and applying Lemma \ref{CanColorAndPlayOnRest}.

\noindent\textbf{Case 49.  }\textit{$L(v_1) = \{0, 1\}$, $L(v_2) = \{0, 2\}$, $L(v_3) = \{0, 1, 3, 4\}$, $L(v_4) = \{1, 2\}$ and $L(v_5) = \{1, 3\}$.}

Fixer gets a winning strategy by coloring $v_5v_3$ with $3$ and applying Lemma \ref{CanColorAndPlayOnRest}.

\noindent\textbf{Case 50.  }\textit{$L(v_1) = \{0, 1\}$, $L(v_2) = \{0, 2\}$, $L(v_3) = \{0, 1, 3, 4\}$, $L(v_4) = \{1, 2\}$ and $L(v_5) = \{2, 3\}$.}

Fixer gets a winning strategy by coloring $v_5v_3$ with $3$ and applying Lemma \ref{CanColorAndPlayOnRest}.

\noindent\textbf{Case 51.  }\textit{$L(v_1) = \{0, 1\}$, $L(v_2) = \{0, 2\}$, $L(v_3) = \{0, 1, 3, 4\}$, $L(v_4) = \{1, 2\}$ and $L(v_5) = \{3, 4\}$.}

Fixer gets a winning strategy by coloring $v_5v_3$ with $3$ and applying Lemma \ref{CanColorAndPlayOnRest}.

\noindent\textbf{Case 52.  }\textit{$L(v_1) = \{0, 1\}$, $L(v_2) = \{0, 2\}$, $L(v_3) = \{0, 1, 3, 4\}$, $L(v_4) = \{0, 3\}$ and $L(v_5) = \{0, 1\}$.}

Fixer gets a winning strategy by coloring $v_4v_3$ with $3$ and applying Lemma \ref{CanColorAndPlayOnRest}.

\noindent\textbf{Case 53.  }\textit{$L(v_1) = \{0, 1\}$, $L(v_2) = \{0, 2\}$, $L(v_3) = \{0, 1, 3, 4\}$, $L(v_4) = \{0, 3\}$ and $L(v_5) = \{0, 2\}$.}

Fixer gets a winning strategy by coloring $v_4v_3$ with $3$ and applying Lemma \ref{CanColorAndPlayOnRest}.

\noindent\textbf{Case 54.  }\textit{$L(v_1) = \{0, 1\}$, $L(v_2) = \{0, 2\}$, $L(v_3) = \{0, 1, 3, 4\}$, $L(v_4) = \{0, 3\}$ and $L(v_5) = \{1, 2\}$.}

Fixer gets a winning strategy by coloring $v_4v_3$ with $3$ and applying Lemma \ref{CanColorAndPlayOnRest}.

\noindent\textbf{Case 55.  }\textit{$L(v_1) = \{0, 1\}$, $L(v_2) = \{0, 2\}$, $L(v_3) = \{0, 1, 3, 4\}$, $L(v_4) = \{0, 3\}$ and $L(v_5) = \{0, 4\}$.}

Fixer gets a winning strategy by coloring $v_4v_3$ with $3$ and applying Lemma \ref{CanColorAndPlayOnRest}.

\noindent\textbf{Case 56.  }\textit{$L(v_1) = \{0, 1\}$, $L(v_2) = \{0, 2\}$, $L(v_3) = \{0, 1, 3, 4\}$, $L(v_4) = \{0, 3\}$ and $L(v_5) = \{1, 4\}$.}

Fixer gets a winning strategy by coloring $v_4v_3$ with $3$ and applying Lemma \ref{CanColorAndPlayOnRest}.

\noindent\textbf{Case 57.  }\textit{$L(v_1) = \{0, 1\}$, $L(v_2) = \{0, 2\}$, $L(v_3) = \{0, 1, 3, 4\}$, $L(v_4) = \{0, 3\}$ and $L(v_5) = \{2, 4\}$.}

Fixer gets a winning strategy by coloring $v_4v_3$ with $3$ and applying Lemma \ref{CanColorAndPlayOnRest}.

\noindent\textbf{Case 58.  }\textit{$L(v_1) = \{0, 1\}$, $L(v_2) = \{0, 2\}$, $L(v_3) = \{0, 1, 3, 4\}$, $L(v_4) = \{0, 3\}$ and $L(v_5) = \{3, 4\}$.}

Fixer gets a winning strategy by coloring $v_4v_3$ with $3$ and applying Lemma \ref{CanColorAndPlayOnRest}.

\noindent\textbf{Case 59.  }\textit{$L(v_1) = \{0, 1\}$, $L(v_2) = \{0, 2\}$, $L(v_3) = \{0, 1, 3, 4\}$, $L(v_4) = \{1, 3\}$ and $L(v_5) = \{0, 1\}$.}

Fixer gets a winning strategy by coloring $v_4v_3$ with $3$ and applying Lemma \ref{CanColorAndPlayOnRest}.

\noindent\textbf{Case 60.  }\textit{$L(v_1) = \{0, 1\}$, $L(v_2) = \{0, 2\}$, $L(v_3) = \{0, 1, 3, 4\}$, $L(v_4) = \{1, 3\}$ and $L(v_5) = \{0, 2\}$.}

Fixer gets a winning strategy by coloring $v_4v_3$ with $3$ and applying Lemma \ref{CanColorAndPlayOnRest}.

\noindent\textbf{Case 61.  }\textit{$L(v_1) = \{0, 1\}$, $L(v_2) = \{0, 2\}$, $L(v_3) = \{0, 1, 3, 4\}$, $L(v_4) = \{1, 3\}$ and $L(v_5) = \{1, 2\}$.}

Fixer gets a winning strategy by coloring $v_4v_3$ with $3$ and applying Lemma \ref{CanColorAndPlayOnRest}.

\noindent\textbf{Case 62.  }\textit{$L(v_1) = \{0, 1\}$, $L(v_2) = \{0, 2\}$, $L(v_3) = \{0, 1, 3, 4\}$, $L(v_4) = \{1, 3\}$ and $L(v_5) = \{0, 4\}$.}

Fixer gets a winning strategy by coloring $v_4v_3$ with $3$ and applying Lemma \ref{CanColorAndPlayOnRest}.

\noindent\textbf{Case 63.  }\textit{$L(v_1) = \{0, 1\}$, $L(v_2) = \{0, 2\}$, $L(v_3) = \{0, 1, 3, 4\}$, $L(v_4) = \{1, 3\}$ and $L(v_5) = \{1, 4\}$.}

Fixer gets a winning strategy by coloring $v_4v_3$ with $3$ and applying Lemma \ref{CanColorAndPlayOnRest}.

\noindent\textbf{Case 64.  }\textit{$L(v_1) = \{0, 1\}$, $L(v_2) = \{0, 2\}$, $L(v_3) = \{0, 1, 3, 4\}$, $L(v_4) = \{1, 3\}$ and $L(v_5) = \{2, 4\}$.}

Fixer gets a winning strategy by coloring $v_4v_3$ with $3$ and applying Lemma \ref{CanColorAndPlayOnRest}.

\noindent\textbf{Case 65.  }\textit{$L(v_1) = \{0, 1\}$, $L(v_2) = \{0, 2\}$, $L(v_3) = \{0, 1, 3, 4\}$, $L(v_4) = \{1, 3\}$ and $L(v_5) = \{3, 4\}$.}

Fixer gets a winning strategy by coloring $v_4v_3$ with $3$ and applying Lemma \ref{CanColorAndPlayOnRest}.

\noindent\textbf{Case 66.  }\textit{$L(v_1) = \{0, 1\}$, $L(v_2) = \{0, 2\}$, $L(v_3) = \{0, 1, 3, 4\}$, $L(v_4) = \{2, 3\}$ and $L(v_5) = \{0, 1\}$.}

Fixer gets a winning strategy by coloring $v_4v_3$ with $3$ and applying Lemma \ref{CanColorAndPlayOnRest}.

\noindent\textbf{Case 67.  }\textit{$L(v_1) = \{0, 1\}$, $L(v_2) = \{0, 2\}$, $L(v_3) = \{0, 1, 3, 4\}$, $L(v_4) = \{2, 3\}$ and $L(v_5) = \{0, 2\}$.}

Fixer gets a winning strategy by coloring $v_4v_3$ with $3$ and applying Lemma \ref{CanColorAndPlayOnRest}.

\noindent\textbf{Case 68.  }\textit{$L(v_1) = \{0, 1\}$, $L(v_2) = \{0, 2\}$, $L(v_3) = \{0, 1, 3, 4\}$, $L(v_4) = \{2, 3\}$ and $L(v_5) = \{1, 2\}$.}

Fixer gets a winning strategy by coloring $v_4v_3$ with $3$ and applying Lemma \ref{CanColorAndPlayOnRest}.

\noindent\textbf{Case 69.  }\textit{$L(v_1) = \{0, 1\}$, $L(v_2) = \{0, 2\}$, $L(v_3) = \{0, 1, 3, 4\}$, $L(v_4) = \{2, 3\}$ and $L(v_5) = \{0, 4\}$.}

Fixer gets a winning strategy by coloring $v_4v_3$ with $3$ and applying Lemma \ref{CanColorAndPlayOnRest}.

\noindent\textbf{Case 70.  }\textit{$L(v_1) = \{0, 1\}$, $L(v_2) = \{0, 2\}$, $L(v_3) = \{0, 1, 3, 4\}$, $L(v_4) = \{2, 3\}$ and $L(v_5) = \{1, 4\}$.}

Fixer gets a winning strategy by coloring $v_4v_3$ with $3$ and applying Lemma \ref{CanColorAndPlayOnRest}.

\noindent\textbf{Case 71.  }\textit{$L(v_1) = \{0, 1\}$, $L(v_2) = \{0, 2\}$, $L(v_3) = \{0, 1, 3, 4\}$, $L(v_4) = \{2, 3\}$ and $L(v_5) = \{2, 4\}$.}

Fixer gets a winning strategy by coloring $v_4v_3$ with $3$ and applying Lemma \ref{CanColorAndPlayOnRest}.

\noindent\textbf{Case 72.  }\textit{$L(v_1) = \{0, 1\}$, $L(v_2) = \{0, 2\}$, $L(v_3) = \{0, 1, 3, 4\}$, $L(v_4) = \{2, 3\}$ and $L(v_5) = \{3, 4\}$.}

Fixer gets a winning strategy by coloring $v_4v_3$ with $3$ and applying Lemma \ref{CanColorAndPlayOnRest}.

\noindent\textbf{Case 73.  }\textit{$L(v_1) = \{0, 1\}$, $L(v_2) = \{0, 2\}$, $L(v_3) = \{0, 1, 3, 4\}$, $L(v_4) = \{3, 4\}$ and $L(v_5) = \{0, 1\}$.}

Fixer gets a winning strategy by coloring $v_4v_3$ with $3$ and applying Lemma \ref{CanColorAndPlayOnRest}.

\noindent\textbf{Case 74.  }\textit{$L(v_1) = \{0, 1\}$, $L(v_2) = \{0, 2\}$, $L(v_3) = \{0, 1, 3, 4\}$, $L(v_4) = \{3, 4\}$ and $L(v_5) = \{0, 2\}$.}

Fixer gets a winning strategy by coloring $v_4v_3$ with $3$ and applying Lemma \ref{CanColorAndPlayOnRest}.

\noindent\textbf{Case 75.  }\textit{$L(v_1) = \{0, 1\}$, $L(v_2) = \{0, 2\}$, $L(v_3) = \{0, 1, 3, 4\}$, $L(v_4) = \{3, 4\}$ and $L(v_5) = \{1, 2\}$.}

Fixer gets a winning strategy by coloring $v_4v_3$ with $3$ and applying Lemma \ref{CanColorAndPlayOnRest}.

\noindent\textbf{Case 76.  }\textit{$L(v_1) = \{0, 1\}$, $L(v_2) = \{0, 2\}$, $L(v_3) = \{0, 1, 3, 4\}$, $L(v_4) = \{3, 4\}$ and $L(v_5) = \{0, 3\}$.}

Fixer gets a winning strategy by coloring $v_4v_3$ with $4$ and applying Lemma \ref{CanColorAndPlayOnRest}.

\noindent\textbf{Case 77.  }\textit{$L(v_1) = \{0, 1\}$, $L(v_2) = \{0, 2\}$, $L(v_3) = \{0, 1, 3, 4\}$, $L(v_4) = \{3, 4\}$ and $L(v_5) = \{1, 3\}$.}

Fixer gets a winning strategy by coloring $v_4v_3$ with $4$ and applying Lemma \ref{CanColorAndPlayOnRest}.

\noindent\textbf{Case 78.  }\textit{$L(v_1) = \{0, 1\}$, $L(v_2) = \{0, 2\}$, $L(v_3) = \{0, 1, 3, 4\}$, $L(v_4) = \{3, 4\}$ and $L(v_5) = \{2, 3\}$.}

Fixer gets a winning strategy by coloring $v_4v_3$ with $4$ and applying Lemma \ref{CanColorAndPlayOnRest}.

\noindent\textbf{Case 79.  }\textit{$L(v_1) = \{0, 1\}$, $L(v_2) = \{0, 2\}$, $L(v_3) = \{0, 1, 3, 4\}$, $L(v_4) = \{3, 4\}$ and $L(v_5) = \{3, 4\}$.}

Fixer gets a winning strategy by coloring $v_4v_3$ with $3$ and applying Lemma \ref{CanColorAndPlayOnRest}.

\noindent\textbf{Case 80.  }\textit{$L(v_1) = \{0, 1\}$, $L(v_2) = \{0, 2\}$, $L(v_3) = \{0, 2, 3, 4\}$, $L(v_4) = \{0, 1\}$ and $L(v_5) = \{0, 1\}$.}

Fixer gets a winning strategy by coloring $v_1v_2$ with $0$ and applying Lemma \ref{CanColorAndPlayOnRest}.

\noindent\textbf{Case 81.  }\textit{$L(v_1) = \{0, 1\}$, $L(v_2) = \{0, 2\}$, $L(v_3) = \{0, 2, 3, 4\}$, $L(v_4) = \{0, 2\}$ and $L(v_5) = \{0, 2\}$.}

Fixer gets a winning strategy by coloring $v_1v_2$ with $0$ and applying Lemma \ref{CanColorAndPlayOnRest}.

\noindent\textbf{Case 82.  }\textit{$L(v_1) = \{0, 1\}$, $L(v_2) = \{0, 2\}$, $L(v_3) = \{0, 2, 3, 4\}$, $L(v_4) = \{1, 2\}$ and $L(v_5) = \{0, 3\}$.}

Fixer gets a winning strategy by coloring $v_5v_3$ with $3$ and applying Lemma \ref{CanColorAndPlayOnRest}.

\noindent\textbf{Case 83.  }\textit{$L(v_1) = \{0, 1\}$, $L(v_2) = \{0, 2\}$, $L(v_3) = \{0, 2, 3, 4\}$, $L(v_4) = \{1, 2\}$ and $L(v_5) = \{1, 3\}$.}

Fixer gets a winning strategy by coloring $v_5v_3$ with $3$ and applying Lemma \ref{CanColorAndPlayOnRest}.

\noindent\textbf{Case 84.  }\textit{$L(v_1) = \{0, 1\}$, $L(v_2) = \{0, 2\}$, $L(v_3) = \{0, 2, 3, 4\}$, $L(v_4) = \{1, 2\}$ and $L(v_5) = \{2, 3\}$.}

Fixer gets a winning strategy by coloring $v_5v_3$ with $3$ and applying Lemma \ref{CanColorAndPlayOnRest}.

\noindent\textbf{Case 85.  }\textit{$L(v_1) = \{0, 1\}$, $L(v_2) = \{0, 2\}$, $L(v_3) = \{0, 2, 3, 4\}$, $L(v_4) = \{1, 2\}$ and $L(v_5) = \{3, 4\}$.}

Fixer gets a winning strategy by coloring $v_5v_3$ with $3$ and applying Lemma \ref{CanColorAndPlayOnRest}.

\noindent\textbf{Case 86.  }\textit{$L(v_1) = \{0, 1\}$, $L(v_2) = \{0, 2\}$, $L(v_3) = \{0, 2, 3, 4\}$, $L(v_4) = \{0, 3\}$ and $L(v_5) = \{1, 2\}$.}

Fixer gets a winning strategy by coloring $v_4v_3$ with $3$ and applying Lemma \ref{CanColorAndPlayOnRest}.

\noindent\textbf{Case 87.  }\textit{$L(v_1) = \{0, 1\}$, $L(v_2) = \{0, 2\}$, $L(v_3) = \{0, 2, 3, 4\}$, $L(v_4) = \{1, 3\}$ and $L(v_5) = \{1, 2\}$.}

Fixer gets a winning strategy by coloring $v_4v_3$ with $3$ and applying Lemma \ref{CanColorAndPlayOnRest}.

\noindent\textbf{Case 88.  }\textit{$L(v_1) = \{0, 1\}$, $L(v_2) = \{0, 2\}$, $L(v_3) = \{0, 2, 3, 4\}$, $L(v_4) = \{1, 3\}$ and $L(v_5) = \{1, 3\}$.}

Fixer gets a winning strategy by coloring $v_1v_2$ with $0$ and applying Lemma \ref{CanColorAndPlayOnRest}.

\noindent\textbf{Case 89.  }\textit{$L(v_1) = \{0, 1\}$, $L(v_2) = \{0, 2\}$, $L(v_3) = \{0, 2, 3, 4\}$, $L(v_4) = \{2, 3\}$ and $L(v_5) = \{1, 2\}$.}

Fixer gets a winning strategy by coloring $v_4v_3$ with $3$ and applying Lemma \ref{CanColorAndPlayOnRest}.

\noindent\textbf{Case 90.  }\textit{$L(v_1) = \{0, 1\}$, $L(v_2) = \{0, 2\}$, $L(v_3) = \{0, 2, 3, 4\}$, $L(v_4) = \{2, 3\}$ and $L(v_5) = \{2, 3\}$.}

Fixer gets a winning strategy by coloring $v_1v_2$ with $0$ and applying Lemma \ref{CanColorAndPlayOnRest}.

\noindent\textbf{Case 91.  }\textit{$L(v_1) = \{0, 1\}$, $L(v_2) = \{0, 2\}$, $L(v_3) = \{0, 2, 3, 4\}$, $L(v_4) = \{3, 4\}$ and $L(v_5) = \{1, 2\}$.}

Fixer gets a winning strategy by coloring $v_4v_3$ with $3$ and applying Lemma \ref{CanColorAndPlayOnRest}.

\noindent\textbf{Case 92.  }\textit{$L(v_1) = \{0, 1\}$, $L(v_2) = \{0, 2\}$, $L(v_3) = \{1, 2, 3, 4\}$, $L(v_4) = \{0, 1\}$ and $L(v_5) = \{0, 1\}$.}

Fixer gets a winning strategy by coloring $v_1v_2$ with $0$ and applying Lemma \ref{CanColorAndPlayOnRest}.

\noindent\textbf{Case 93.  }\textit{$L(v_1) = \{0, 1\}$, $L(v_2) = \{0, 2\}$, $L(v_3) = \{1, 2, 3, 4\}$, $L(v_4) = \{0, 2\}$ and $L(v_5) = \{0, 3\}$.}

Fixer gets a winning strategy by coloring $v_5v_3$ with $3$ and applying Lemma \ref{CanColorAndPlayOnRest}.

\noindent\textbf{Case 94.  }\textit{$L(v_1) = \{0, 1\}$, $L(v_2) = \{0, 2\}$, $L(v_3) = \{1, 2, 3, 4\}$, $L(v_4) = \{0, 2\}$ and $L(v_5) = \{1, 3\}$.}

Fixer gets a winning strategy by coloring $v_5v_3$ with $3$ and applying Lemma \ref{CanColorAndPlayOnRest}.

\noindent\textbf{Case 95.  }\textit{$L(v_1) = \{0, 1\}$, $L(v_2) = \{0, 2\}$, $L(v_3) = \{1, 2, 3, 4\}$, $L(v_4) = \{0, 2\}$ and $L(v_5) = \{2, 3\}$.}

Fixer gets a winning strategy by coloring $v_5v_3$ with $3$ and applying Lemma \ref{CanColorAndPlayOnRest}.

\noindent\textbf{Case 96.  }\textit{$L(v_1) = \{0, 1\}$, $L(v_2) = \{0, 2\}$, $L(v_3) = \{1, 2, 3, 4\}$, $L(v_4) = \{0, 2\}$ and $L(v_5) = \{3, 4\}$.}

Fixer gets a winning strategy by coloring $v_5v_3$ with $3$ and applying Lemma \ref{CanColorAndPlayOnRest}.

\noindent\textbf{Case 97.  }\textit{$L(v_1) = \{0, 1\}$, $L(v_2) = \{0, 2\}$, $L(v_3) = \{1, 2, 3, 4\}$, $L(v_4) = \{1, 2\}$ and $L(v_5) = \{1, 2\}$.}

Fixer gets a winning strategy by coloring $v_1v_2$ with $0$ and applying Lemma \ref{CanColorAndPlayOnRest}.

\noindent\textbf{Case 98.  }\textit{$L(v_1) = \{0, 1\}$, $L(v_2) = \{0, 2\}$, $L(v_3) = \{1, 2, 3, 4\}$, $L(v_4) = \{0, 3\}$ and $L(v_5) = \{0, 2\}$.}

Fixer gets a winning strategy by coloring $v_4v_3$ with $3$ and applying Lemma \ref{CanColorAndPlayOnRest}.

\noindent\textbf{Case 99.  }\textit{$L(v_1) = \{0, 1\}$, $L(v_2) = \{0, 2\}$, $L(v_3) = \{1, 2, 3, 4\}$, $L(v_4) = \{0, 3\}$ and $L(v_5) = \{0, 3\}$.}

Fixer gets a winning strategy by coloring $v_1v_2$ with $0$ and applying Lemma \ref{CanColorAndPlayOnRest}.

\noindent\textbf{Case 100.  }\textit{$L(v_1) = \{0, 1\}$, $L(v_2) = \{0, 2\}$, $L(v_3) = \{1, 2, 3, 4\}$, $L(v_4) = \{1, 3\}$ and $L(v_5) = \{0, 2\}$.}

Fixer gets a winning strategy by coloring $v_4v_3$ with $3$ and applying Lemma \ref{CanColorAndPlayOnRest}.

\noindent\textbf{Case 101.  }\textit{$L(v_1) = \{0, 1\}$, $L(v_2) = \{0, 2\}$, $L(v_3) = \{1, 2, 3, 4\}$, $L(v_4) = \{2, 3\}$ and $L(v_5) = \{0, 2\}$.}

Fixer gets a winning strategy by coloring $v_4v_3$ with $3$ and applying Lemma \ref{CanColorAndPlayOnRest}.

\noindent\textbf{Case 102.  }\textit{$L(v_1) = \{0, 1\}$, $L(v_2) = \{0, 2\}$, $L(v_3) = \{1, 2, 3, 4\}$, $L(v_4) = \{2, 3\}$ and $L(v_5) = \{2, 3\}$.}

Fixer gets a winning strategy by coloring $v_1v_2$ with $0$ and applying Lemma \ref{CanColorAndPlayOnRest}.

\noindent\textbf{Case 103.  }\textit{$L(v_1) = \{0, 1\}$, $L(v_2) = \{0, 2\}$, $L(v_3) = \{1, 2, 3, 4\}$, $L(v_4) = \{3, 4\}$ and $L(v_5) = \{0, 2\}$.}

Fixer gets a winning strategy by coloring $v_4v_3$ with $3$ and applying Lemma \ref{CanColorAndPlayOnRest}.

\noindent\textbf{Case 104.  }\textit{$L(v_1) = \{0, 1\}$, $L(v_2) = \{0, 1\}$, $L(v_3) = \{0, 2, 3, 4\}$, $L(v_4) = \{0, 1\}$ and $L(v_5) = \{2, 5\}$.}

Fixer gets a winning strategy by coloring $v_5v_3$ with $2$ and applying Lemma \ref{CanColorAndPlayOnRest}.

\noindent\textbf{Case 105.  }\textit{$L(v_1) = \{0, 1\}$, $L(v_2) = \{0, 1\}$, $L(v_3) = \{0, 2, 3, 4\}$, $L(v_4) = \{2, 5\}$ and $L(v_5) = \{0, 1\}$.}

Fixer gets a winning strategy by coloring $v_4v_3$ with $2$ and applying Lemma \ref{CanColorAndPlayOnRest}.

\noindent\textbf{Case 106.  }\textit{$L(v_1) = \{0, 1\}$, $L(v_2) = \{0, 1\}$, $L(v_3) = \{0, 2, 3, 4\}$, $L(v_4) = \{2, 5\}$ and $L(v_5) = \{2, 5\}$.}

Fixer gets a winning strategy by coloring $v_1v_2$ with $1$ and applying Lemma \ref{CanColorAndPlayOnRest}.

\noindent\textbf{Case 107.  }\textit{$L(v_1) = \{0, 1\}$, $L(v_2) = \{0, 2\}$, $L(v_3) = \{0, 1, 2, 3\}$, $L(v_4) = \{2, 4\}$ and $L(v_5) = \{3, 5\}$.}

Fixer gets a winning strategy by coloring $v_4v_3$ with $2$ and applying Lemma \ref{CanColorAndPlayOnRest}.

\noindent\textbf{Case 108.  }\textit{$L(v_1) = \{0, 1\}$, $L(v_2) = \{0, 2\}$, $L(v_3) = \{0, 1, 2, 3\}$, $L(v_4) = \{3, 4\}$ and $L(v_5) = \{2, 5\}$.}

Fixer gets a winning strategy by coloring $v_4v_3$ with $3$ and applying Lemma \ref{CanColorAndPlayOnRest}.

\noindent\textbf{Case 109.  }\textit{$L(v_1) = \{0, 1\}$, $L(v_2) = \{0, 2\}$, $L(v_3) = \{0, 1, 3, 4\}$, $L(v_4) = \{0, 1\}$ and $L(v_5) = \{3, 5\}$.}

Fixer gets a winning strategy by coloring $v_5v_3$ with $3$ and applying Lemma \ref{CanColorAndPlayOnRest}.

\noindent\textbf{Case 110.  }\textit{$L(v_1) = \{0, 1\}$, $L(v_2) = \{0, 2\}$, $L(v_3) = \{0, 1, 3, 4\}$, $L(v_4) = \{0, 2\}$ and $L(v_5) = \{3, 5\}$.}

Fixer gets a winning strategy by coloring $v_5v_3$ with $3$ and applying Lemma \ref{CanColorAndPlayOnRest}.

\noindent\textbf{Case 111.  }\textit{$L(v_1) = \{0, 1\}$, $L(v_2) = \{0, 2\}$, $L(v_3) = \{0, 1, 3, 4\}$, $L(v_4) = \{1, 2\}$ and $L(v_5) = \{3, 5\}$.}

Fixer gets a winning strategy by coloring $v_5v_3$ with $3$ and applying Lemma \ref{CanColorAndPlayOnRest}.

\noindent\textbf{Case 112.  }\textit{$L(v_1) = \{0, 1\}$, $L(v_2) = \{0, 2\}$, $L(v_3) = \{0, 1, 3, 4\}$, $L(v_4) = \{0, 3\}$ and $L(v_5) = \{4, 5\}$.}

Fixer gets a winning strategy by coloring $v_4v_3$ with $3$ and applying Lemma \ref{CanColorAndPlayOnRest}.

\noindent\textbf{Case 113.  }\textit{$L(v_1) = \{0, 1\}$, $L(v_2) = \{0, 2\}$, $L(v_3) = \{0, 1, 3, 4\}$, $L(v_4) = \{1, 3\}$ and $L(v_5) = \{4, 5\}$.}

Fixer gets a winning strategy by coloring $v_4v_3$ with $3$ and applying Lemma \ref{CanColorAndPlayOnRest}.

\noindent\textbf{Case 114.  }\textit{$L(v_1) = \{0, 1\}$, $L(v_2) = \{0, 2\}$, $L(v_3) = \{0, 1, 3, 4\}$, $L(v_4) = \{2, 3\}$ and $L(v_5) = \{4, 5\}$.}

Fixer gets a winning strategy by coloring $v_4v_3$ with $3$ and applying Lemma \ref{CanColorAndPlayOnRest}.

\noindent\textbf{Case 115.  }\textit{$L(v_1) = \{0, 1\}$, $L(v_2) = \{0, 2\}$, $L(v_3) = \{0, 1, 3, 4\}$, $L(v_4) = \{3, 4\}$ and $L(v_5) = \{3, 5\}$.}

Fixer gets a winning strategy by coloring $v_4v_3$ with $4$ and applying Lemma \ref{CanColorAndPlayOnRest}.

\noindent\textbf{Case 116.  }\textit{$L(v_1) = \{0, 1\}$, $L(v_2) = \{0, 2\}$, $L(v_3) = \{0, 1, 3, 4\}$, $L(v_4) = \{3, 5\}$ and $L(v_5) = \{0, 1\}$.}

Fixer gets a winning strategy by coloring $v_4v_3$ with $3$ and applying Lemma \ref{CanColorAndPlayOnRest}.

\noindent\textbf{Case 117.  }\textit{$L(v_1) = \{0, 1\}$, $L(v_2) = \{0, 2\}$, $L(v_3) = \{0, 1, 3, 4\}$, $L(v_4) = \{3, 5\}$ and $L(v_5) = \{0, 2\}$.}

Fixer gets a winning strategy by coloring $v_4v_3$ with $3$ and applying Lemma \ref{CanColorAndPlayOnRest}.

\noindent\textbf{Case 118.  }\textit{$L(v_1) = \{0, 1\}$, $L(v_2) = \{0, 2\}$, $L(v_3) = \{0, 1, 3, 4\}$, $L(v_4) = \{3, 5\}$ and $L(v_5) = \{1, 2\}$.}

Fixer gets a winning strategy by coloring $v_4v_3$ with $3$ and applying Lemma \ref{CanColorAndPlayOnRest}.

\noindent\textbf{Case 119.  }\textit{$L(v_1) = \{0, 1\}$, $L(v_2) = \{0, 2\}$, $L(v_3) = \{0, 1, 3, 4\}$, $L(v_4) = \{3, 5\}$ and $L(v_5) = \{0, 4\}$.}

Fixer gets a winning strategy by coloring $v_4v_3$ with $3$ and applying Lemma \ref{CanColorAndPlayOnRest}.

\noindent\textbf{Case 120.  }\textit{$L(v_1) = \{0, 1\}$, $L(v_2) = \{0, 2\}$, $L(v_3) = \{0, 1, 3, 4\}$, $L(v_4) = \{3, 5\}$ and $L(v_5) = \{1, 4\}$.}

Fixer gets a winning strategy by coloring $v_4v_3$ with $3$ and applying Lemma \ref{CanColorAndPlayOnRest}.

\noindent\textbf{Case 121.  }\textit{$L(v_1) = \{0, 1\}$, $L(v_2) = \{0, 2\}$, $L(v_3) = \{0, 1, 3, 4\}$, $L(v_4) = \{3, 5\}$ and $L(v_5) = \{2, 4\}$.}

Fixer gets a winning strategy by coloring $v_4v_3$ with $3$ and applying Lemma \ref{CanColorAndPlayOnRest}.

\noindent\textbf{Case 122.  }\textit{$L(v_1) = \{0, 1\}$, $L(v_2) = \{0, 2\}$, $L(v_3) = \{0, 1, 3, 4\}$, $L(v_4) = \{3, 5\}$ and $L(v_5) = \{3, 4\}$.}

Fixer gets a winning strategy by coloring $v_4v_3$ with $3$ and applying Lemma \ref{CanColorAndPlayOnRest}.

\noindent\textbf{Case 123.  }\textit{$L(v_1) = \{0, 1\}$, $L(v_2) = \{0, 2\}$, $L(v_3) = \{0, 1, 3, 4\}$, $L(v_4) = \{3, 5\}$ and $L(v_5) = \{4, 5\}$.}

Fixer gets a winning strategy by coloring $v_4v_3$ with $3$ and applying Lemma \ref{CanColorAndPlayOnRest}.

\noindent\textbf{Case 124.  }\textit{$L(v_1) = \{0, 1\}$, $L(v_2) = \{0, 2\}$, $L(v_3) = \{0, 2, 3, 4\}$, $L(v_4) = \{1, 2\}$ and $L(v_5) = \{3, 5\}$.}

Fixer gets a winning strategy by coloring $v_5v_3$ with $3$ and applying Lemma \ref{CanColorAndPlayOnRest}.

\noindent\textbf{Case 125.  }\textit{$L(v_1) = \{0, 1\}$, $L(v_2) = \{0, 2\}$, $L(v_3) = \{0, 2, 3, 4\}$, $L(v_4) = \{0, 5\}$ and $L(v_5) = \{0, 5\}$.}

Fixer gets a winning strategy by coloring $v_1v_2$ with $0$ and applying Lemma \ref{CanColorAndPlayOnRest}.

\noindent\textbf{Case 126.  }\textit{$L(v_1) = \{0, 1\}$, $L(v_2) = \{0, 2\}$, $L(v_3) = \{0, 2, 3, 4\}$, $L(v_4) = \{3, 5\}$ and $L(v_5) = \{1, 2\}$.}

Fixer gets a winning strategy by coloring $v_4v_3$ with $3$ and applying Lemma \ref{CanColorAndPlayOnRest}.

\noindent\textbf{Case 127.  }\textit{$L(v_1) = \{0, 1\}$, $L(v_2) = \{0, 2\}$, $L(v_3) = \{0, 2, 3, 4\}$, $L(v_4) = \{3, 5\}$ and $L(v_5) = \{3, 5\}$.}

Fixer gets a winning strategy by coloring $v_1v_2$ with $0$ and applying Lemma \ref{CanColorAndPlayOnRest}.

\noindent\textbf{Case 128.  }\textit{$L(v_1) = \{0, 1\}$, $L(v_2) = \{0, 2\}$, $L(v_3) = \{1, 2, 3, 4\}$, $L(v_4) = \{0, 2\}$ and $L(v_5) = \{3, 5\}$.}

Fixer gets a winning strategy by coloring $v_5v_3$ with $3$ and applying Lemma \ref{CanColorAndPlayOnRest}.

\noindent\textbf{Case 129.  }\textit{$L(v_1) = \{0, 1\}$, $L(v_2) = \{0, 2\}$, $L(v_3) = \{1, 2, 3, 4\}$, $L(v_4) = \{1, 5\}$ and $L(v_5) = \{1, 5\}$.}

Fixer gets a winning strategy by coloring $v_1v_2$ with $0$ and applying Lemma \ref{CanColorAndPlayOnRest}.

\noindent\textbf{Case 130.  }\textit{$L(v_1) = \{0, 1\}$, $L(v_2) = \{0, 2\}$, $L(v_3) = \{1, 2, 3, 4\}$, $L(v_4) = \{3, 5\}$ and $L(v_5) = \{0, 2\}$.}

Fixer gets a winning strategy by coloring $v_4v_3$ with $3$ and applying Lemma \ref{CanColorAndPlayOnRest}.

\noindent\textbf{Case 131.  }\textit{$L(v_1) = \{0, 1\}$, $L(v_2) = \{0, 2\}$, $L(v_3) = \{1, 2, 3, 4\}$, $L(v_4) = \{3, 5\}$ and $L(v_5) = \{3, 5\}$.}

Fixer gets a winning strategy by coloring $v_1v_2$ with $0$ and applying Lemma \ref{CanColorAndPlayOnRest}.

\noindent\textbf{Case 132.  }\textit{$L(v_1) = \{0, 1\}$, $L(v_2) = \{0, 2\}$, $L(v_3) = \{2, 3, 4, 5\}$, $L(v_4) = \{0, 3\}$ and $L(v_5) = \{0, 3\}$.}

Fixer gets a winning strategy by coloring $v_1v_2$ with $0$ and applying Lemma \ref{CanColorAndPlayOnRest}.

\noindent\textbf{Case 133.  }\textit{$L(v_1) = \{0, 1\}$, $L(v_2) = \{0, 2\}$, $L(v_3) = \{2, 3, 4, 5\}$, $L(v_4) = \{1, 3\}$ and $L(v_5) = \{1, 3\}$.}

Fixer gets a winning strategy by coloring $v_1v_2$ with $0$ and applying Lemma \ref{CanColorAndPlayOnRest}.

\noindent\textbf{Case 134.  }\textit{$L(v_1) = \{0, 1\}$, $L(v_2) = \{0, 2\}$, $L(v_3) = \{2, 3, 4, 5\}$, $L(v_4) = \{2, 3\}$ and $L(v_5) = \{2, 3\}$.}

Fixer gets a winning strategy by coloring $v_1v_2$ with $0$ and applying Lemma \ref{CanColorAndPlayOnRest}.

\noindent\textbf{Case 135.  }\textit{$L(v_1) = \{0, 1\}$, $L(v_2) = \{0, 2\}$, $L(v_3) = \{0, 1, 3, 4\}$, $L(v_4) = \{3, 5\}$ and $L(v_5) = \{4, 6\}$.}

Fixer gets a winning strategy by coloring $v_4v_3$ with $3$ and applying Lemma \ref{CanColorAndPlayOnRest}.

\noindent\textbf{Case 136.  }\textit{$L(v_1) = \{0, 1\}$, $L(v_2) = \{0, 2\}$, $L(v_3) = \{2, 3, 4, 5\}$, $L(v_4) = \{3, 6\}$ and $L(v_5) = \{3, 6\}$.}

Fixer gets a winning strategy by coloring $v_1v_2$ with $0$ and applying Lemma \ref{CanColorAndPlayOnRest}.

\noindent\textbf{Case 137.  }\textit{$L(v_1) = \{0, 1\}$, $L(v_2) = \{0, 1\}$, $L(v_3) = \{0, 1, 2, 3\}$, $L(v_4) = \{0, 4\}$ and $L(v_5) = \{1, 4\}$.}

Let $S$ and $A_S$ be as in Lemma \ref{MultiMoveCombination} using colors $2$ and $4$. If the components of $A_S$ have vertex sets $\{v_2\}$ and $\{v_3, v_4\}$, then Fixer should swap 2 and 4 at $v_2$. This results in a position with lists $L(v_1) = \{0, 1\}$, $L(v_2) = \{0, 1\}$, $L(v_3) = \{0, 1, 3, 4\}$, $L(v_4) = \{0, 4\}$ and $L(v_5) = \{1, 4\}$, but then Fixer can edge-color the graph. If the components of $A_S$ have vertex sets $\{v_3\}$ and $\{v_2, v_4\}$, then Fixer should swap 2 and 4 at $v_3$. This results in a position with lists $L(v_1) = \{0, 1\}$, $L(v_2) = \{0, 1\}$, $L(v_3) = \{0, 1, 2, 3\}$, $L(v_4) = \{0, 2\}$ and $L(v_5) = \{1, 4\}$, but then Fixer can edge-color the graph. If the components of $A_S$ have vertex sets $\{v_4\}$ and $\{v_2, v_3\}$, then Fixer should swap 2 and 4 at $v_4$. This results in a position with lists $L(v_1) = \{0, 1\}$, $L(v_2) = \{0, 1\}$, $L(v_3) = \{0, 1, 2, 3\}$, $L(v_4) = \{0, 4\}$ and $L(v_5) = \{1, 2\}$, but then Fixer can edge-color the graph. 

\noindent\textbf{Case 138.  }\textit{$L(v_1) = \{0, 1\}$, $L(v_2) = \{0, 1\}$, $L(v_3) = \{0, 2, 3, 4\}$, $L(v_4) = \{0, 1\}$ and $L(v_5) = \{0, 1\}$.}

Let $S$ and $A_S$ be as in Lemma \ref{MultiMoveCombination} using colors $1$ and $2$. If the components of $A_S$ have vertex sets $\{v_0\}$, $\{v_1, v_2\}$ and $\{v_3, v_4\}$, then Fixer should swap 1 and 2 at $v_4$ and $v_3$. This results in a position with lists $L(v_1) = \{0, 1\}$, $L(v_2) = \{0, 1\}$, $L(v_3) = \{0, 2, 3, 4\}$, $L(v_4) = \{0, 2\}$ and $L(v_5) = \{0, 2\}$, but then Fixer wins by Case 13. If the components of $A_S$ have vertex sets $\{v_0\}$, $\{v_1, v_3\}$ and $\{v_2, v_4\}$, then Fixer should swap 1 and 2 at $v_4$ and $v_2$. This results in a position with lists $L(v_1) = \{0, 1\}$, $L(v_2) = \{0, 1\}$, $L(v_3) = \{0, 1, 3, 4\}$, $L(v_4) = \{0, 1\}$ and $L(v_5) = \{0, 2\}$, but then Fixer wins by Case 5. If the components of $A_S$ have vertex sets $\{v_0\}$, $\{v_1, v_4\}$ and $\{v_2, v_3\}$, then Fixer should swap 1 and 2 at $v_3$ and $v_2$. This results in a position with lists $L(v_1) = \{0, 1\}$, $L(v_2) = \{0, 1\}$, $L(v_3) = \{0, 1, 3, 4\}$, $L(v_4) = \{0, 2\}$ and $L(v_5) = \{0, 1\}$, but then Fixer wins by Case 6. If the components of $A_S$ have vertex sets $\{v_1\}$, $\{v_0, v_2\}$ and $\{v_3, v_4\}$, then Fixer should swap 1 and 2 at $v_1$. This results in a position with lists $L(v_1) = \{0, 1\}$, $L(v_2) = \{0, 2\}$, $L(v_3) = \{0, 2, 3, 4\}$, $L(v_4) = \{0, 1\}$ and $L(v_5) = \{0, 1\}$, but then Fixer wins by Case 80. If the components of $A_S$ have vertex sets $\{v_1\}$, $\{v_0, v_3\}$ and $\{v_2, v_4\}$, then Fixer should swap 1 and 2 at $v_1$. This results in a position with lists $L(v_1) = \{0, 1\}$, $L(v_2) = \{0, 2\}$, $L(v_3) = \{0, 2, 3, 4\}$, $L(v_4) = \{0, 1\}$ and $L(v_5) = \{0, 1\}$, but then Fixer wins by Case 80. If the components of $A_S$ have vertex sets $\{v_1\}$, $\{v_0, v_4\}$ and $\{v_2, v_3\}$, then Fixer should swap 1 and 2 at $v_1$. This results in a position with lists $L(v_1) = \{0, 1\}$, $L(v_2) = \{0, 2\}$, $L(v_3) = \{0, 2, 3, 4\}$, $L(v_4) = \{0, 1\}$ and $L(v_5) = \{0, 1\}$, but then Fixer wins by Case 80. If the components of $A_S$ have vertex sets $\{v_2\}$, $\{v_0, v_1\}$ and $\{v_3, v_4\}$, then Fixer should swap 1 and 2 at $v_2$. This results in a position with lists $L(v_1) = \{0, 1\}$, $L(v_2) = \{0, 1\}$, $L(v_3) = \{0, 1, 3, 4\}$, $L(v_4) = \{0, 1\}$ and $L(v_5) = \{0, 1\}$, but then Fixer can edge-color the graph. If the components of $A_S$ have vertex sets $\{v_3\}$, $\{v_0, v_1\}$ and $\{v_2, v_4\}$, then Fixer should swap 1 and 2 at $v_3$. This results in a position with lists $L(v_1) = \{0, 1\}$, $L(v_2) = \{0, 1\}$, $L(v_3) = \{0, 2, 3, 4\}$, $L(v_4) = \{0, 2\}$ and $L(v_5) = \{0, 1\}$, but then Fixer wins by Case 12. If the components of $A_S$ have vertex sets $\{v_4\}$, $\{v_0, v_1\}$ and $\{v_2, v_3\}$, then Fixer should swap 1 and 2 at $v_4$. This results in a position with lists $L(v_1) = \{0, 1\}$, $L(v_2) = \{0, 1\}$, $L(v_3) = \{0, 2, 3, 4\}$, $L(v_4) = \{0, 1\}$ and $L(v_5) = \{0, 2\}$, but then Fixer wins by Case 9. If the components of $A_S$ have vertex sets $\{v_2\}$, $\{v_0, v_3\}$ and $\{v_1, v_4\}$, then Fixer should swap 1 and 2 at $v_2$. This results in a position with lists $L(v_1) = \{0, 1\}$, $L(v_2) = \{0, 1\}$, $L(v_3) = \{0, 1, 3, 4\}$, $L(v_4) = \{0, 1\}$ and $L(v_5) = \{0, 1\}$, but then Fixer can edge-color the graph. If the components of $A_S$ have vertex sets $\{v_2\}$, $\{v_0, v_4\}$ and $\{v_1, v_3\}$, then Fixer should swap 1 and 2 at $v_2$. This results in a position with lists $L(v_1) = \{0, 1\}$, $L(v_2) = \{0, 1\}$, $L(v_3) = \{0, 1, 3, 4\}$, $L(v_4) = \{0, 1\}$ and $L(v_5) = \{0, 1\}$, but then Fixer can edge-color the graph. If the components of $A_S$ have vertex sets $\{v_3\}$, $\{v_0, v_2\}$ and $\{v_1, v_4\}$, then Fixer should swap 1 and 2 at $v_3$. This results in a position with lists $L(v_1) = \{0, 1\}$, $L(v_2) = \{0, 1\}$, $L(v_3) = \{0, 2, 3, 4\}$, $L(v_4) = \{0, 2\}$ and $L(v_5) = \{0, 1\}$, but then Fixer wins by Case 12. If the components of $A_S$ have vertex sets $\{v_4\}$, $\{v_0, v_2\}$ and $\{v_1, v_3\}$, then Fixer should swap 1 and 2 at $v_4$. This results in a position with lists $L(v_1) = \{0, 1\}$, $L(v_2) = \{0, 1\}$, $L(v_3) = \{0, 2, 3, 4\}$, $L(v_4) = \{0, 1\}$ and $L(v_5) = \{0, 2\}$, but then Fixer wins by Case 9. If the components of $A_S$ have vertex sets $\{v_3\}$, $\{v_0, v_4\}$ and $\{v_1, v_2\}$, then Fixer should swap 1 and 2 at $v_3$. This results in a position with lists $L(v_1) = \{0, 1\}$, $L(v_2) = \{0, 1\}$, $L(v_3) = \{0, 2, 3, 4\}$, $L(v_4) = \{0, 2\}$ and $L(v_5) = \{0, 1\}$, but then Fixer wins by Case 12. If the components of $A_S$ have vertex sets $\{v_4\}$, $\{v_0, v_3\}$ and $\{v_1, v_2\}$, then Fixer should swap 1 and 2 at $v_4$. This results in a position with lists $L(v_1) = \{0, 1\}$, $L(v_2) = \{0, 1\}$, $L(v_3) = \{0, 2, 3, 4\}$, $L(v_4) = \{0, 1\}$ and $L(v_5) = \{0, 2\}$, but then Fixer wins by Case 9. 

\noindent\textbf{Case 139.  }\textit{$L(v_1) = \{0, 1\}$, $L(v_2) = \{0, 2\}$, $L(v_3) = \{0, 1, 2, 3\}$, $L(v_4) = \{0, 4\}$ and $L(v_5) = \{2, 4\}$.}

Let $S$ and $A_S$ be as in Lemma \ref{MultiMoveCombination} using colors $1$ and $4$. If the components of $A_S$ have vertex sets $\{v_0, v_2\}$ and $\{v_3, v_4\}$, then Fixer should swap 1 and 4 at $v_2$ and $v_0$. This results in a position with lists $L(v_1) = \{0, 4\}$, $L(v_2) = \{0, 2\}$, $L(v_3) = \{0, 2, 3, 4\}$, $L(v_4) = \{0, 4\}$ and $L(v_5) = \{2, 4\}$, but then Fixer can edge-color the graph. If the components of $A_S$ have vertex sets $\{v_0, v_3\}$ and $\{v_2, v_4\}$, then Fixer should swap 1 and 4 at $v_3$ and $v_0$. This results in a position with lists $L(v_1) = \{0, 4\}$, $L(v_2) = \{0, 2\}$, $L(v_3) = \{0, 1, 2, 3\}$, $L(v_4) = \{0, 1\}$ and $L(v_5) = \{2, 4\}$, but then Fixer wins by Case 86. If the components of $A_S$ have vertex sets $\{v_0, v_4\}$ and $\{v_2, v_3\}$, then Fixer should swap 1 and 4 at $v_4$ and $v_0$. This results in a position with lists $L(v_1) = \{0, 4\}$, $L(v_2) = \{0, 2\}$, $L(v_3) = \{0, 1, 2, 3\}$, $L(v_4) = \{0, 4\}$ and $L(v_5) = \{1, 2\}$, but then Fixer can edge-color the graph. 

\noindent\textbf{Case 140.  }\textit{$L(v_1) = \{0, 1\}$, $L(v_2) = \{0, 2\}$, $L(v_3) = \{0, 1, 2, 3\}$, $L(v_4) = \{1, 4\}$ and $L(v_5) = \{2, 4\}$.}

Let $S$ and $A_S$ be as in Lemma \ref{MultiMoveCombination} using colors $0$ and $4$. If the components of $A_S$ have vertex sets $\{v_0\}$, $\{v_1, v_2\}$ and $\{v_3, v_4\}$, then Fixer should swap 0 and 4 at $v_4$ and $v_3$. This results in a position with lists $L(v_1) = \{0, 1\}$, $L(v_2) = \{0, 2\}$, $L(v_3) = \{0, 1, 2, 3\}$, $L(v_4) = \{0, 1\}$ and $L(v_5) = \{0, 2\}$, but then Fixer can edge-color the graph. If the components of $A_S$ have vertex sets $\{v_0\}$, $\{v_1, v_3\}$ and $\{v_2, v_4\}$, then Fixer should swap 0 and 4 at $v_4$ and $v_2$. This results in a position with lists $L(v_1) = \{0, 1\}$, $L(v_2) = \{0, 2\}$, $L(v_3) = \{1, 2, 3, 4\}$, $L(v_4) = \{1, 4\}$ and $L(v_5) = \{0, 2\}$, but then Fixer wins by Case 100. If the components of $A_S$ have vertex sets $\{v_0\}$, $\{v_1, v_4\}$ and $\{v_2, v_3\}$, then Fixer should swap 0 and 4 at $v_3$ and $v_2$. This results in a position with lists $L(v_1) = \{0, 1\}$, $L(v_2) = \{0, 2\}$, $L(v_3) = \{1, 2, 3, 4\}$, $L(v_4) = \{0, 1\}$ and $L(v_5) = \{2, 4\}$, but then Fixer can edge-color the graph. If the components of $A_S$ have vertex sets $\{v_1\}$, $\{v_0, v_2\}$ and $\{v_3, v_4\}$, then Fixer should swap 0 and 4 at $v_4$ and $v_3$. This results in a position with lists $L(v_1) = \{0, 1\}$, $L(v_2) = \{0, 2\}$, $L(v_3) = \{0, 1, 2, 3\}$, $L(v_4) = \{0, 1\}$ and $L(v_5) = \{0, 2\}$, but then Fixer can edge-color the graph. If the components of $A_S$ have vertex sets $\{v_1\}$, $\{v_0, v_3\}$ and $\{v_2, v_4\}$, then Fixer should swap 0 and 4 at $v_4$ and $v_2$. This results in a position with lists $L(v_1) = \{0, 1\}$, $L(v_2) = \{0, 2\}$, $L(v_3) = \{1, 2, 3, 4\}$, $L(v_4) = \{1, 4\}$ and $L(v_5) = \{0, 2\}$, but then Fixer wins by Case 100. If the components of $A_S$ have vertex sets $\{v_1\}$, $\{v_0, v_4\}$ and $\{v_2, v_3\}$, then Fixer should swap 0 and 4 at $v_3$ and $v_2$. This results in a position with lists $L(v_1) = \{0, 1\}$, $L(v_2) = \{0, 2\}$, $L(v_3) = \{1, 2, 3, 4\}$, $L(v_4) = \{0, 1\}$ and $L(v_5) = \{2, 4\}$, but then Fixer can edge-color the graph. If the components of $A_S$ have vertex sets $\{v_2\}$, $\{v_0, v_1\}$ and $\{v_3, v_4\}$, then Fixer should swap 0 and 4 at $v_2$. This results in a position with lists $L(v_1) = \{0, 1\}$, $L(v_2) = \{0, 2\}$, $L(v_3) = \{1, 2, 3, 4\}$, $L(v_4) = \{1, 4\}$ and $L(v_5) = \{2, 4\}$, but then Fixer can edge-color the graph. If the components of $A_S$ have vertex sets $\{v_3\}$, $\{v_0, v_1\}$ and $\{v_2, v_4\}$, then Fixer should swap 0 and 4 at $v_3$. This results in a position with lists $L(v_1) = \{0, 1\}$, $L(v_2) = \{0, 2\}$, $L(v_3) = \{0, 1, 2, 3\}$, $L(v_4) = \{0, 1\}$ and $L(v_5) = \{2, 4\}$, but then Fixer wins by Case 17. If the components of $A_S$ have vertex sets $\{v_4\}$, $\{v_0, v_1\}$ and $\{v_2, v_3\}$, then Fixer should swap 0 and 4 at $v_4$. This results in a position with lists $L(v_1) = \{0, 1\}$, $L(v_2) = \{0, 2\}$, $L(v_3) = \{0, 1, 2, 3\}$, $L(v_4) = \{1, 4\}$ and $L(v_5) = \{0, 2\}$, but then Fixer can edge-color the graph. If the components of $A_S$ have vertex sets $\{v_2\}$, $\{v_0, v_3\}$ and $\{v_1, v_4\}$, then Fixer should swap 0 and 4 at $v_2$. This results in a position with lists $L(v_1) = \{0, 1\}$, $L(v_2) = \{0, 2\}$, $L(v_3) = \{1, 2, 3, 4\}$, $L(v_4) = \{1, 4\}$ and $L(v_5) = \{2, 4\}$, but then Fixer can edge-color the graph. If the components of $A_S$ have vertex sets $\{v_2\}$, $\{v_0, v_4\}$ and $\{v_1, v_3\}$, then Fixer should swap 0 and 4 at $v_2$. This results in a position with lists $L(v_1) = \{0, 1\}$, $L(v_2) = \{0, 2\}$, $L(v_3) = \{1, 2, 3, 4\}$, $L(v_4) = \{1, 4\}$ and $L(v_5) = \{2, 4\}$, but then Fixer can edge-color the graph. If the components of $A_S$ have vertex sets $\{v_3\}$, $\{v_0, v_2\}$ and $\{v_1, v_4\}$, then Fixer should swap 0 and 4 at $v_3$. This results in a position with lists $L(v_1) = \{0, 1\}$, $L(v_2) = \{0, 2\}$, $L(v_3) = \{0, 1, 2, 3\}$, $L(v_4) = \{0, 1\}$ and $L(v_5) = \{2, 4\}$, but then Fixer wins by Case 17. If the components of $A_S$ have vertex sets $\{v_4\}$, $\{v_0, v_2\}$ and $\{v_1, v_3\}$, then Fixer should swap 0 and 4 at $v_4$. This results in a position with lists $L(v_1) = \{0, 1\}$, $L(v_2) = \{0, 2\}$, $L(v_3) = \{0, 1, 2, 3\}$, $L(v_4) = \{1, 4\}$ and $L(v_5) = \{0, 2\}$, but then Fixer can edge-color the graph. If the components of $A_S$ have vertex sets $\{v_3\}$, $\{v_0, v_4\}$ and $\{v_1, v_2\}$, then Fixer should swap 0 and 4 at $v_3$. This results in a position with lists $L(v_1) = \{0, 1\}$, $L(v_2) = \{0, 2\}$, $L(v_3) = \{0, 1, 2, 3\}$, $L(v_4) = \{0, 1\}$ and $L(v_5) = \{2, 4\}$, but then Fixer wins by Case 17. If the components of $A_S$ have vertex sets $\{v_4\}$, $\{v_0, v_3\}$ and $\{v_1, v_2\}$, then Fixer should swap 0 and 4 at $v_4$. This results in a position with lists $L(v_1) = \{0, 1\}$, $L(v_2) = \{0, 2\}$, $L(v_3) = \{0, 1, 2, 3\}$, $L(v_4) = \{1, 4\}$ and $L(v_5) = \{0, 2\}$, but then Fixer can edge-color the graph. 

\noindent\textbf{Case 141.  }\textit{$L(v_1) = \{0, 1\}$, $L(v_2) = \{0, 2\}$, $L(v_3) = \{0, 1, 2, 3\}$, $L(v_4) = \{2, 4\}$ and $L(v_5) = \{0, 4\}$.}

Let $S$ and $A_S$ be as in Lemma \ref{MultiMoveCombination} using colors $1$ and $4$. If the components of $A_S$ have vertex sets $\{v_0, v_2\}$ and $\{v_3, v_4\}$, then Fixer should swap 1 and 4 at $v_2$ and $v_0$. This results in a position with lists $L(v_1) = \{0, 4\}$, $L(v_2) = \{0, 2\}$, $L(v_3) = \{0, 2, 3, 4\}$, $L(v_4) = \{2, 4\}$ and $L(v_5) = \{0, 4\}$, but then Fixer can edge-color the graph. If the components of $A_S$ have vertex sets $\{v_0, v_3\}$ and $\{v_2, v_4\}$, then Fixer should swap 1 and 4 at $v_3$ and $v_0$. This results in a position with lists $L(v_1) = \{0, 4\}$, $L(v_2) = \{0, 2\}$, $L(v_3) = \{0, 1, 2, 3\}$, $L(v_4) = \{1, 2\}$ and $L(v_5) = \{0, 4\}$, but then Fixer can edge-color the graph. If the components of $A_S$ have vertex sets $\{v_0, v_4\}$ and $\{v_2, v_3\}$, then Fixer should swap 1 and 4 at $v_4$ and $v_0$. This results in a position with lists $L(v_1) = \{0, 4\}$, $L(v_2) = \{0, 2\}$, $L(v_3) = \{0, 1, 2, 3\}$, $L(v_4) = \{2, 4\}$ and $L(v_5) = \{0, 1\}$, but then Fixer wins by Case 82. 

\noindent\textbf{Case 142.  }\textit{$L(v_1) = \{0, 1\}$, $L(v_2) = \{0, 2\}$, $L(v_3) = \{0, 1, 2, 3\}$, $L(v_4) = \{2, 4\}$ and $L(v_5) = \{1, 4\}$.}

Let $S$ and $A_S$ be as in Lemma \ref{MultiMoveCombination} using colors $0$ and $4$. If the components of $A_S$ have vertex sets $\{v_0\}$, $\{v_1, v_2\}$ and $\{v_3, v_4\}$, then Fixer should swap 0 and 4 at $v_4$ and $v_3$. This results in a position with lists $L(v_1) = \{0, 1\}$, $L(v_2) = \{0, 2\}$, $L(v_3) = \{0, 1, 2, 3\}$, $L(v_4) = \{0, 2\}$ and $L(v_5) = \{0, 1\}$, but then Fixer can edge-color the graph. If the components of $A_S$ have vertex sets $\{v_0\}$, $\{v_1, v_3\}$ and $\{v_2, v_4\}$, then Fixer should swap 0 and 4 at $v_4$ and $v_2$. This results in a position with lists $L(v_1) = \{0, 1\}$, $L(v_2) = \{0, 2\}$, $L(v_3) = \{1, 2, 3, 4\}$, $L(v_4) = \{2, 4\}$ and $L(v_5) = \{0, 1\}$, but then Fixer can edge-color the graph. If the components of $A_S$ have vertex sets $\{v_0\}$, $\{v_1, v_4\}$ and $\{v_2, v_3\}$, then Fixer should swap 0 and 4 at $v_3$ and $v_2$. This results in a position with lists $L(v_1) = \{0, 1\}$, $L(v_2) = \{0, 2\}$, $L(v_3) = \{1, 2, 3, 4\}$, $L(v_4) = \{0, 2\}$ and $L(v_5) = \{1, 4\}$, but then Fixer wins by Case 94. If the components of $A_S$ have vertex sets $\{v_1\}$, $\{v_0, v_2\}$ and $\{v_3, v_4\}$, then Fixer should swap 0 and 4 at $v_4$ and $v_3$. This results in a position with lists $L(v_1) = \{0, 1\}$, $L(v_2) = \{0, 2\}$, $L(v_3) = \{0, 1, 2, 3\}$, $L(v_4) = \{0, 2\}$ and $L(v_5) = \{0, 1\}$, but then Fixer can edge-color the graph. If the components of $A_S$ have vertex sets $\{v_1\}$, $\{v_0, v_3\}$ and $\{v_2, v_4\}$, then Fixer should swap 0 and 4 at $v_4$ and $v_2$. This results in a position with lists $L(v_1) = \{0, 1\}$, $L(v_2) = \{0, 2\}$, $L(v_3) = \{1, 2, 3, 4\}$, $L(v_4) = \{2, 4\}$ and $L(v_5) = \{0, 1\}$, but then Fixer can edge-color the graph. If the components of $A_S$ have vertex sets $\{v_1\}$, $\{v_0, v_4\}$ and $\{v_2, v_3\}$, then Fixer should swap 0 and 4 at $v_3$ and $v_2$. This results in a position with lists $L(v_1) = \{0, 1\}$, $L(v_2) = \{0, 2\}$, $L(v_3) = \{1, 2, 3, 4\}$, $L(v_4) = \{0, 2\}$ and $L(v_5) = \{1, 4\}$, but then Fixer wins by Case 94. If the components of $A_S$ have vertex sets $\{v_2\}$, $\{v_0, v_1\}$ and $\{v_3, v_4\}$, then Fixer should swap 0 and 4 at $v_2$. This results in a position with lists $L(v_1) = \{0, 1\}$, $L(v_2) = \{0, 2\}$, $L(v_3) = \{1, 2, 3, 4\}$, $L(v_4) = \{2, 4\}$ and $L(v_5) = \{1, 4\}$, but then Fixer can edge-color the graph. If the components of $A_S$ have vertex sets $\{v_3\}$, $\{v_0, v_1\}$ and $\{v_2, v_4\}$, then Fixer should swap 0 and 4 at $v_3$. This results in a position with lists $L(v_1) = \{0, 1\}$, $L(v_2) = \{0, 2\}$, $L(v_3) = \{0, 1, 2, 3\}$, $L(v_4) = \{0, 2\}$ and $L(v_5) = \{1, 4\}$, but then Fixer can edge-color the graph. If the components of $A_S$ have vertex sets $\{v_4\}$, $\{v_0, v_1\}$ and $\{v_2, v_3\}$, then Fixer should swap 0 and 4 at $v_4$. This results in a position with lists $L(v_1) = \{0, 1\}$, $L(v_2) = \{0, 2\}$, $L(v_3) = \{0, 1, 2, 3\}$, $L(v_4) = \{2, 4\}$ and $L(v_5) = \{0, 1\}$, but then Fixer wins by Case 30. If the components of $A_S$ have vertex sets $\{v_2\}$, $\{v_0, v_3\}$ and $\{v_1, v_4\}$, then Fixer should swap 0 and 4 at $v_2$. This results in a position with lists $L(v_1) = \{0, 1\}$, $L(v_2) = \{0, 2\}$, $L(v_3) = \{1, 2, 3, 4\}$, $L(v_4) = \{2, 4\}$ and $L(v_5) = \{1, 4\}$, but then Fixer can edge-color the graph. If the components of $A_S$ have vertex sets $\{v_2\}$, $\{v_0, v_4\}$ and $\{v_1, v_3\}$, then Fixer should swap 0 and 4 at $v_2$. This results in a position with lists $L(v_1) = \{0, 1\}$, $L(v_2) = \{0, 2\}$, $L(v_3) = \{1, 2, 3, 4\}$, $L(v_4) = \{2, 4\}$ and $L(v_5) = \{1, 4\}$, but then Fixer can edge-color the graph. If the components of $A_S$ have vertex sets $\{v_3\}$, $\{v_0, v_2\}$ and $\{v_1, v_4\}$, then Fixer should swap 0 and 4 at $v_3$. This results in a position with lists $L(v_1) = \{0, 1\}$, $L(v_2) = \{0, 2\}$, $L(v_3) = \{0, 1, 2, 3\}$, $L(v_4) = \{0, 2\}$ and $L(v_5) = \{1, 4\}$, but then Fixer can edge-color the graph. If the components of $A_S$ have vertex sets $\{v_4\}$, $\{v_0, v_2\}$ and $\{v_1, v_3\}$, then Fixer should swap 0 and 4 at $v_4$. This results in a position with lists $L(v_1) = \{0, 1\}$, $L(v_2) = \{0, 2\}$, $L(v_3) = \{0, 1, 2, 3\}$, $L(v_4) = \{2, 4\}$ and $L(v_5) = \{0, 1\}$, but then Fixer wins by Case 30. If the components of $A_S$ have vertex sets $\{v_3\}$, $\{v_0, v_4\}$ and $\{v_1, v_2\}$, then Fixer should swap 0 and 4 at $v_3$. This results in a position with lists $L(v_1) = \{0, 1\}$, $L(v_2) = \{0, 2\}$, $L(v_3) = \{0, 1, 2, 3\}$, $L(v_4) = \{0, 2\}$ and $L(v_5) = \{1, 4\}$, but then Fixer can edge-color the graph. If the components of $A_S$ have vertex sets $\{v_4\}$, $\{v_0, v_3\}$ and $\{v_1, v_2\}$, then Fixer should swap 0 and 4 at $v_4$. This results in a position with lists $L(v_1) = \{0, 1\}$, $L(v_2) = \{0, 2\}$, $L(v_3) = \{0, 1, 2, 3\}$, $L(v_4) = \{2, 4\}$ and $L(v_5) = \{0, 1\}$, but then Fixer wins by Case 30. 

\noindent\textbf{Case 143.  }\textit{$L(v_1) = \{0, 1\}$, $L(v_2) = \{0, 2\}$, $L(v_3) = \{0, 1, 2, 3\}$, $L(v_4) = \{2, 4\}$ and $L(v_5) = \{2, 4\}$.}

Let $S$ and $A_S$ be as in Lemma \ref{MultiMoveCombination} using colors $0$ and $4$. If the components of $A_S$ have vertex sets $\{v_0\}$, $\{v_1, v_2\}$ and $\{v_3, v_4\}$, then Fixer should swap 0 and 4 at $v_4$ and $v_3$. This results in a position with lists $L(v_1) = \{0, 1\}$, $L(v_2) = \{0, 2\}$, $L(v_3) = \{0, 1, 2, 3\}$, $L(v_4) = \{0, 2\}$ and $L(v_5) = \{0, 2\}$, but then Fixer can edge-color the graph. If the components of $A_S$ have vertex sets $\{v_0\}$, $\{v_1, v_3\}$ and $\{v_2, v_4\}$, then Fixer should swap 0 and 4 at $v_4$ and $v_2$. This results in a position with lists $L(v_1) = \{0, 1\}$, $L(v_2) = \{0, 2\}$, $L(v_3) = \{1, 2, 3, 4\}$, $L(v_4) = \{2, 4\}$ and $L(v_5) = \{0, 2\}$, but then Fixer wins by Case 101. If the components of $A_S$ have vertex sets $\{v_0\}$, $\{v_1, v_4\}$ and $\{v_2, v_3\}$, then Fixer should swap 0 and 4 at $v_3$ and $v_2$. This results in a position with lists $L(v_1) = \{0, 1\}$, $L(v_2) = \{0, 2\}$, $L(v_3) = \{1, 2, 3, 4\}$, $L(v_4) = \{0, 2\}$ and $L(v_5) = \{2, 4\}$, but then Fixer wins by Case 95. If the components of $A_S$ have vertex sets $\{v_1\}$, $\{v_0, v_2\}$ and $\{v_3, v_4\}$, then Fixer should swap 0 and 4 at $v_4$ and $v_3$. This results in a position with lists $L(v_1) = \{0, 1\}$, $L(v_2) = \{0, 2\}$, $L(v_3) = \{0, 1, 2, 3\}$, $L(v_4) = \{0, 2\}$ and $L(v_5) = \{0, 2\}$, but then Fixer can edge-color the graph. If the components of $A_S$ have vertex sets $\{v_1\}$, $\{v_0, v_3\}$ and $\{v_2, v_4\}$, then Fixer should swap 0 and 4 at $v_4$ and $v_2$. This results in a position with lists $L(v_1) = \{0, 1\}$, $L(v_2) = \{0, 2\}$, $L(v_3) = \{1, 2, 3, 4\}$, $L(v_4) = \{2, 4\}$ and $L(v_5) = \{0, 2\}$, but then Fixer wins by Case 101. If the components of $A_S$ have vertex sets $\{v_1\}$, $\{v_0, v_4\}$ and $\{v_2, v_3\}$, then Fixer should swap 0 and 4 at $v_3$ and $v_2$. This results in a position with lists $L(v_1) = \{0, 1\}$, $L(v_2) = \{0, 2\}$, $L(v_3) = \{1, 2, 3, 4\}$, $L(v_4) = \{0, 2\}$ and $L(v_5) = \{2, 4\}$, but then Fixer wins by Case 95. If the components of $A_S$ have vertex sets $\{v_2\}$, $\{v_0, v_1\}$ and $\{v_3, v_4\}$, then Fixer should swap 0 and 4 at $v_2$. This results in a position with lists $L(v_1) = \{0, 1\}$, $L(v_2) = \{0, 2\}$, $L(v_3) = \{1, 2, 3, 4\}$, $L(v_4) = \{2, 4\}$ and $L(v_5) = \{2, 4\}$, but then Fixer wins by Case 102. If the components of $A_S$ have vertex sets $\{v_3\}$, $\{v_0, v_1\}$ and $\{v_2, v_4\}$, then Fixer should swap 0 and 4 at $v_3$. This results in a position with lists $L(v_1) = \{0, 1\}$, $L(v_2) = \{0, 2\}$, $L(v_3) = \{0, 1, 2, 3\}$, $L(v_4) = \{0, 2\}$ and $L(v_5) = \{2, 4\}$, but then Fixer wins by Case 19. If the components of $A_S$ have vertex sets $\{v_4\}$, $\{v_0, v_1\}$ and $\{v_2, v_3\}$, then Fixer should swap 0 and 4 at $v_4$. This results in a position with lists $L(v_1) = \{0, 1\}$, $L(v_2) = \{0, 2\}$, $L(v_3) = \{0, 1, 2, 3\}$, $L(v_4) = \{2, 4\}$ and $L(v_5) = \{0, 2\}$, but then Fixer wins by Case 31. If the components of $A_S$ have vertex sets $\{v_2\}$, $\{v_0, v_3\}$ and $\{v_1, v_4\}$, then Fixer should swap 0 and 4 at $v_2$. This results in a position with lists $L(v_1) = \{0, 1\}$, $L(v_2) = \{0, 2\}$, $L(v_3) = \{1, 2, 3, 4\}$, $L(v_4) = \{2, 4\}$ and $L(v_5) = \{2, 4\}$, but then Fixer wins by Case 102. If the components of $A_S$ have vertex sets $\{v_2\}$, $\{v_0, v_4\}$ and $\{v_1, v_3\}$, then Fixer should swap 0 and 4 at $v_2$. This results in a position with lists $L(v_1) = \{0, 1\}$, $L(v_2) = \{0, 2\}$, $L(v_3) = \{1, 2, 3, 4\}$, $L(v_4) = \{2, 4\}$ and $L(v_5) = \{2, 4\}$, but then Fixer wins by Case 102. If the components of $A_S$ have vertex sets $\{v_3\}$, $\{v_0, v_2\}$ and $\{v_1, v_4\}$, then Fixer should swap 0 and 4 at $v_3$. This results in a position with lists $L(v_1) = \{0, 1\}$, $L(v_2) = \{0, 2\}$, $L(v_3) = \{0, 1, 2, 3\}$, $L(v_4) = \{0, 2\}$ and $L(v_5) = \{2, 4\}$, but then Fixer wins by Case 19. If the components of $A_S$ have vertex sets $\{v_4\}$, $\{v_0, v_2\}$ and $\{v_1, v_3\}$, then Fixer should swap 0 and 4 at $v_4$. This results in a position with lists $L(v_1) = \{0, 1\}$, $L(v_2) = \{0, 2\}$, $L(v_3) = \{0, 1, 2, 3\}$, $L(v_4) = \{2, 4\}$ and $L(v_5) = \{0, 2\}$, but then Fixer wins by Case 31. If the components of $A_S$ have vertex sets $\{v_3\}$, $\{v_0, v_4\}$ and $\{v_1, v_2\}$, then Fixer should swap 0 and 4 at $v_3$. This results in a position with lists $L(v_1) = \{0, 1\}$, $L(v_2) = \{0, 2\}$, $L(v_3) = \{0, 1, 2, 3\}$, $L(v_4) = \{0, 2\}$ and $L(v_5) = \{2, 4\}$, but then Fixer wins by Case 19. If the components of $A_S$ have vertex sets $\{v_4\}$, $\{v_0, v_3\}$ and $\{v_1, v_2\}$, then Fixer should swap 0 and 4 at $v_4$. This results in a position with lists $L(v_1) = \{0, 1\}$, $L(v_2) = \{0, 2\}$, $L(v_3) = \{0, 1, 2, 3\}$, $L(v_4) = \{2, 4\}$ and $L(v_5) = \{0, 2\}$, but then Fixer wins by Case 31. 

\noindent\textbf{Case 144.  }\textit{$L(v_1) = \{0, 1\}$, $L(v_2) = \{0, 2\}$, $L(v_3) = \{0, 1, 3, 4\}$, $L(v_4) = \{0, 1\}$ and $L(v_5) = \{0, 1\}$.}

Let $S$ and $A_S$ be as in Lemma \ref{MultiMoveCombination} using colors $1$ and $2$. If the components of $A_S$ have vertex sets $\{v_0\}$, $\{v_1, v_2\}$ and $\{v_3, v_4\}$, then Fixer should swap 1 and 2 at $v_0$. This results in a position with lists $L(v_1) = \{0, 2\}$, $L(v_2) = \{0, 2\}$, $L(v_3) = \{0, 1, 3, 4\}$, $L(v_4) = \{0, 1\}$ and $L(v_5) = \{0, 1\}$, but then Fixer wins by Case 13. If the components of $A_S$ have vertex sets $\{v_0\}$, $\{v_1, v_3\}$ and $\{v_2, v_4\}$, then Fixer should swap 1 and 2 at $v_0$. This results in a position with lists $L(v_1) = \{0, 2\}$, $L(v_2) = \{0, 2\}$, $L(v_3) = \{0, 1, 3, 4\}$, $L(v_4) = \{0, 1\}$ and $L(v_5) = \{0, 1\}$, but then Fixer wins by Case 13. If the components of $A_S$ have vertex sets $\{v_0\}$, $\{v_1, v_4\}$ and $\{v_2, v_3\}$, then Fixer should swap 1 and 2 at $v_0$. This results in a position with lists $L(v_1) = \{0, 2\}$, $L(v_2) = \{0, 2\}$, $L(v_3) = \{0, 1, 3, 4\}$, $L(v_4) = \{0, 1\}$ and $L(v_5) = \{0, 1\}$, but then Fixer wins by Case 13. If the components of $A_S$ have vertex sets $\{v_1\}$, $\{v_0, v_2\}$ and $\{v_3, v_4\}$, then Fixer should swap 1 and 2 at $v_1$. This results in a position with lists $L(v_1) = \{0, 1\}$, $L(v_2) = \{0, 1\}$, $L(v_3) = \{0, 1, 3, 4\}$, $L(v_4) = \{0, 1\}$ and $L(v_5) = \{0, 1\}$, but then Fixer can edge-color the graph. If the components of $A_S$ have vertex sets $\{v_1\}$, $\{v_0, v_3\}$ and $\{v_2, v_4\}$, then Fixer should swap 1 and 2 at $v_1$. This results in a position with lists $L(v_1) = \{0, 1\}$, $L(v_2) = \{0, 1\}$, $L(v_3) = \{0, 1, 3, 4\}$, $L(v_4) = \{0, 1\}$ and $L(v_5) = \{0, 1\}$, but then Fixer can edge-color the graph. If the components of $A_S$ have vertex sets $\{v_1\}$, $\{v_0, v_4\}$ and $\{v_2, v_3\}$, then Fixer should swap 1 and 2 at $v_1$. This results in a position with lists $L(v_1) = \{0, 1\}$, $L(v_2) = \{0, 1\}$, $L(v_3) = \{0, 1, 3, 4\}$, $L(v_4) = \{0, 1\}$ and $L(v_5) = \{0, 1\}$, but then Fixer can edge-color the graph. If the components of $A_S$ have vertex sets $\{v_2\}$, $\{v_0, v_1\}$ and $\{v_3, v_4\}$, then Fixer should swap 1 and 2 at $v_2$. This results in a position with lists $L(v_1) = \{0, 1\}$, $L(v_2) = \{0, 2\}$, $L(v_3) = \{0, 2, 3, 4\}$, $L(v_4) = \{0, 1\}$ and $L(v_5) = \{0, 1\}$, but then Fixer wins by Case 80. If the components of $A_S$ have vertex sets $\{v_3\}$, $\{v_0, v_1\}$ and $\{v_2, v_4\}$, then Fixer should swap 1 and 2 at $v_1$ and $v_0$. This results in a position with lists $L(v_1) = \{0, 2\}$, $L(v_2) = \{0, 1\}$, $L(v_3) = \{0, 1, 3, 4\}$, $L(v_4) = \{0, 1\}$ and $L(v_5) = \{0, 1\}$, but then Fixer wins by Case 81. If the components of $A_S$ have vertex sets $\{v_4\}$, $\{v_0, v_1\}$ and $\{v_2, v_3\}$, then Fixer should swap 1 and 2 at $v_1$ and $v_0$. This results in a position with lists $L(v_1) = \{0, 2\}$, $L(v_2) = \{0, 1\}$, $L(v_3) = \{0, 1, 3, 4\}$, $L(v_4) = \{0, 1\}$ and $L(v_5) = \{0, 1\}$, but then Fixer wins by Case 81. If the components of $A_S$ have vertex sets $\{v_2\}$, $\{v_0, v_3\}$ and $\{v_1, v_4\}$, then Fixer should swap 1 and 2 at $v_2$. This results in a position with lists $L(v_1) = \{0, 1\}$, $L(v_2) = \{0, 2\}$, $L(v_3) = \{0, 2, 3, 4\}$, $L(v_4) = \{0, 1\}$ and $L(v_5) = \{0, 1\}$, but then Fixer wins by Case 80. If the components of $A_S$ have vertex sets $\{v_2\}$, $\{v_0, v_4\}$ and $\{v_1, v_3\}$, then Fixer should swap 1 and 2 at $v_2$. This results in a position with lists $L(v_1) = \{0, 1\}$, $L(v_2) = \{0, 2\}$, $L(v_3) = \{0, 2, 3, 4\}$, $L(v_4) = \{0, 1\}$ and $L(v_5) = \{0, 1\}$, but then Fixer wins by Case 80. If the components of $A_S$ have vertex sets $\{v_3\}$, $\{v_0, v_2\}$ and $\{v_1, v_4\}$, then Fixer should swap 1 and 2 at $v_2$ and $v_0$. This results in a position with lists $L(v_1) = \{0, 2\}$, $L(v_2) = \{0, 2\}$, $L(v_3) = \{0, 2, 3, 4\}$, $L(v_4) = \{0, 1\}$ and $L(v_5) = \{0, 1\}$, but then Fixer wins by Case 7. If the components of $A_S$ have vertex sets $\{v_4\}$, $\{v_0, v_2\}$ and $\{v_1, v_3\}$, then Fixer should swap 1 and 2 at $v_2$ and $v_0$. This results in a position with lists $L(v_1) = \{0, 2\}$, $L(v_2) = \{0, 2\}$, $L(v_3) = \{0, 2, 3, 4\}$, $L(v_4) = \{0, 1\}$ and $L(v_5) = \{0, 1\}$, but then Fixer wins by Case 7. If the components of $A_S$ have vertex sets $\{v_3\}$, $\{v_0, v_4\}$ and $\{v_1, v_2\}$, then Fixer should swap 1 and 2 at $v_4$ and $v_0$. This results in a position with lists $L(v_1) = \{0, 2\}$, $L(v_2) = \{0, 2\}$, $L(v_3) = \{0, 1, 3, 4\}$, $L(v_4) = \{0, 1\}$ and $L(v_5) = \{0, 2\}$, but then Fixer wins by Case 12. If the components of $A_S$ have vertex sets $\{v_4\}$, $\{v_0, v_3\}$ and $\{v_1, v_2\}$, then Fixer should swap 1 and 2 at $v_3$ and $v_0$. This results in a position with lists $L(v_1) = \{0, 2\}$, $L(v_2) = \{0, 2\}$, $L(v_3) = \{0, 1, 3, 4\}$, $L(v_4) = \{0, 2\}$ and $L(v_5) = \{0, 1\}$, but then Fixer wins by Case 9. 

\noindent\textbf{Case 145.  }\textit{$L(v_1) = \{0, 1\}$, $L(v_2) = \{0, 2\}$, $L(v_3) = \{0, 1, 3, 4\}$, $L(v_4) = \{0, 1\}$ and $L(v_5) = \{0, 2\}$.}

Let $S$ and $A_S$ be as in Lemma \ref{MultiMoveCombination} using colors $2$ and $3$. If the components of $A_S$ have vertex sets $\{v_1\}$ and $\{v_2, v_4\}$, then Fixer should swap 2 and 3 at $v_1$. This results in a position with lists $L(v_1) = \{0, 1\}$, $L(v_2) = \{0, 3\}$, $L(v_3) = \{0, 1, 3, 4\}$, $L(v_4) = \{0, 1\}$ and $L(v_5) = \{0, 2\}$, but then Fixer can edge-color the graph. If the components of $A_S$ have vertex sets $\{v_2\}$ and $\{v_1, v_4\}$, then Fixer should swap 2 and 3 at $v_2$. This results in a position with lists $L(v_1) = \{0, 1\}$, $L(v_2) = \{0, 2\}$, $L(v_3) = \{0, 1, 2, 4\}$, $L(v_4) = \{0, 1\}$ and $L(v_5) = \{0, 2\}$, but then Fixer can edge-color the graph. If the components of $A_S$ have vertex sets $\{v_4\}$ and $\{v_1, v_2\}$, then Fixer should swap 2 and 3 at $v_4$. This results in a position with lists $L(v_1) = \{0, 1\}$, $L(v_2) = \{0, 2\}$, $L(v_3) = \{0, 1, 3, 4\}$, $L(v_4) = \{0, 1\}$ and $L(v_5) = \{0, 3\}$, but then Fixer wins by Case 40. 

\noindent\textbf{Case 146.  }\textit{$L(v_1) = \{0, 1\}$, $L(v_2) = \{0, 2\}$, $L(v_3) = \{0, 1, 3, 4\}$, $L(v_4) = \{0, 1\}$ and $L(v_5) = \{1, 2\}$.}

Let $S$ and $A_S$ be as in Lemma \ref{MultiMoveCombination} using colors $2$ and $3$. If the components of $A_S$ have vertex sets $\{v_1\}$ and $\{v_2, v_4\}$, then Fixer should swap 2 and 3 at $v_1$. This results in a position with lists $L(v_1) = \{0, 1\}$, $L(v_2) = \{0, 3\}$, $L(v_3) = \{0, 1, 3, 4\}$, $L(v_4) = \{0, 1\}$ and $L(v_5) = \{1, 2\}$, but then Fixer can edge-color the graph. If the components of $A_S$ have vertex sets $\{v_2\}$ and $\{v_1, v_4\}$, then Fixer should swap 2 and 3 at $v_2$. This results in a position with lists $L(v_1) = \{0, 1\}$, $L(v_2) = \{0, 2\}$, $L(v_3) = \{0, 1, 2, 4\}$, $L(v_4) = \{0, 1\}$ and $L(v_5) = \{1, 2\}$, but then Fixer can edge-color the graph. If the components of $A_S$ have vertex sets $\{v_4\}$ and $\{v_1, v_2\}$, then Fixer should swap 2 and 3 at $v_4$. This results in a position with lists $L(v_1) = \{0, 1\}$, $L(v_2) = \{0, 2\}$, $L(v_3) = \{0, 1, 3, 4\}$, $L(v_4) = \{0, 1\}$ and $L(v_5) = \{1, 3\}$, but then Fixer wins by Case 41. 

\noindent\textbf{Case 147.  }\textit{$L(v_1) = \{0, 1\}$, $L(v_2) = \{0, 2\}$, $L(v_3) = \{0, 1, 3, 4\}$, $L(v_4) = \{0, 2\}$ and $L(v_5) = \{0, 1\}$.}

Let $S$ and $A_S$ be as in Lemma \ref{MultiMoveCombination} using colors $2$ and $3$. If the components of $A_S$ have vertex sets $\{v_1\}$ and $\{v_2, v_3\}$, then Fixer should swap 2 and 3 at $v_1$. This results in a position with lists $L(v_1) = \{0, 1\}$, $L(v_2) = \{0, 3\}$, $L(v_3) = \{0, 1, 3, 4\}$, $L(v_4) = \{0, 2\}$ and $L(v_5) = \{0, 1\}$, but then Fixer can edge-color the graph. If the components of $A_S$ have vertex sets $\{v_2\}$ and $\{v_1, v_3\}$, then Fixer should swap 2 and 3 at $v_2$. This results in a position with lists $L(v_1) = \{0, 1\}$, $L(v_2) = \{0, 2\}$, $L(v_3) = \{0, 1, 2, 4\}$, $L(v_4) = \{0, 2\}$ and $L(v_5) = \{0, 1\}$, but then Fixer can edge-color the graph. If the components of $A_S$ have vertex sets $\{v_3\}$ and $\{v_1, v_2\}$, then Fixer should swap 2 and 3 at $v_3$. This results in a position with lists $L(v_1) = \{0, 1\}$, $L(v_2) = \{0, 2\}$, $L(v_3) = \{0, 1, 3, 4\}$, $L(v_4) = \{0, 3\}$ and $L(v_5) = \{0, 1\}$, but then Fixer wins by Case 52. 

\noindent\textbf{Case 148.  }\textit{$L(v_1) = \{0, 1\}$, $L(v_2) = \{0, 2\}$, $L(v_3) = \{0, 1, 3, 4\}$, $L(v_4) = \{0, 2\}$ and $L(v_5) = \{0, 2\}$.}

Let $S$ and $A_S$ be as in Lemma \ref{MultiMoveCombination} using colors $1$ and $2$. If the components of $A_S$ have vertex sets $\{v_0\}$, $\{v_1, v_2\}$ and $\{v_3, v_4\}$, then Fixer should swap 1 and 2 at $v_2$ and $v_1$. This results in a position with lists $L(v_1) = \{0, 1\}$, $L(v_2) = \{0, 1\}$, $L(v_3) = \{0, 2, 3, 4\}$, $L(v_4) = \{0, 2\}$ and $L(v_5) = \{0, 2\}$, but then Fixer wins by Case 13. If the components of $A_S$ have vertex sets $\{v_0\}$, $\{v_1, v_3\}$ and $\{v_2, v_4\}$, then Fixer should swap 1 and 2 at $v_3$ and $v_1$. This results in a position with lists $L(v_1) = \{0, 1\}$, $L(v_2) = \{0, 1\}$, $L(v_3) = \{0, 1, 3, 4\}$, $L(v_4) = \{0, 1\}$ and $L(v_5) = \{0, 2\}$, but then Fixer wins by Case 5. If the components of $A_S$ have vertex sets $\{v_0\}$, $\{v_1, v_4\}$ and $\{v_2, v_3\}$, then Fixer should swap 1 and 2 at $v_4$ and $v_1$. This results in a position with lists $L(v_1) = \{0, 1\}$, $L(v_2) = \{0, 1\}$, $L(v_3) = \{0, 1, 3, 4\}$, $L(v_4) = \{0, 2\}$ and $L(v_5) = \{0, 1\}$, but then Fixer wins by Case 6. If the components of $A_S$ have vertex sets $\{v_1\}$, $\{v_0, v_2\}$ and $\{v_3, v_4\}$, then Fixer should swap 1 and 2 at $v_1$. This results in a position with lists $L(v_1) = \{0, 1\}$, $L(v_2) = \{0, 1\}$, $L(v_3) = \{0, 1, 3, 4\}$, $L(v_4) = \{0, 2\}$ and $L(v_5) = \{0, 2\}$, but then Fixer wins by Case 7. If the components of $A_S$ have vertex sets $\{v_1\}$, $\{v_0, v_3\}$ and $\{v_2, v_4\}$, then Fixer should swap 1 and 2 at $v_1$. This results in a position with lists $L(v_1) = \{0, 1\}$, $L(v_2) = \{0, 1\}$, $L(v_3) = \{0, 1, 3, 4\}$, $L(v_4) = \{0, 2\}$ and $L(v_5) = \{0, 2\}$, but then Fixer wins by Case 7. If the components of $A_S$ have vertex sets $\{v_1\}$, $\{v_0, v_4\}$ and $\{v_2, v_3\}$, then Fixer should swap 1 and 2 at $v_1$. This results in a position with lists $L(v_1) = \{0, 1\}$, $L(v_2) = \{0, 1\}$, $L(v_3) = \{0, 1, 3, 4\}$, $L(v_4) = \{0, 2\}$ and $L(v_5) = \{0, 2\}$, but then Fixer wins by Case 7. If the components of $A_S$ have vertex sets $\{v_2\}$, $\{v_0, v_1\}$ and $\{v_3, v_4\}$, then Fixer should swap 1 and 2 at $v_2$. This results in a position with lists $L(v_1) = \{0, 1\}$, $L(v_2) = \{0, 2\}$, $L(v_3) = \{0, 2, 3, 4\}$, $L(v_4) = \{0, 2\}$ and $L(v_5) = \{0, 2\}$, but then Fixer wins by Case 81. If the components of $A_S$ have vertex sets $\{v_3\}$, $\{v_0, v_1\}$ and $\{v_2, v_4\}$, then Fixer should swap 1 and 2 at $v_1$ and $v_0$. This results in a position with lists $L(v_1) = \{0, 2\}$, $L(v_2) = \{0, 1\}$, $L(v_3) = \{0, 1, 3, 4\}$, $L(v_4) = \{0, 2\}$ and $L(v_5) = \{0, 2\}$, but then Fixer wins by Case 80. If the components of $A_S$ have vertex sets $\{v_4\}$, $\{v_0, v_1\}$ and $\{v_2, v_3\}$, then Fixer should swap 1 and 2 at $v_1$ and $v_0$. This results in a position with lists $L(v_1) = \{0, 2\}$, $L(v_2) = \{0, 1\}$, $L(v_3) = \{0, 1, 3, 4\}$, $L(v_4) = \{0, 2\}$ and $L(v_5) = \{0, 2\}$, but then Fixer wins by Case 80. If the components of $A_S$ have vertex sets $\{v_2\}$, $\{v_0, v_3\}$ and $\{v_1, v_4\}$, then Fixer should swap 1 and 2 at $v_2$. This results in a position with lists $L(v_1) = \{0, 1\}$, $L(v_2) = \{0, 2\}$, $L(v_3) = \{0, 2, 3, 4\}$, $L(v_4) = \{0, 2\}$ and $L(v_5) = \{0, 2\}$, but then Fixer wins by Case 81. If the components of $A_S$ have vertex sets $\{v_2\}$, $\{v_0, v_4\}$ and $\{v_1, v_3\}$, then Fixer should swap 1 and 2 at $v_2$. This results in a position with lists $L(v_1) = \{0, 1\}$, $L(v_2) = \{0, 2\}$, $L(v_3) = \{0, 2, 3, 4\}$, $L(v_4) = \{0, 2\}$ and $L(v_5) = \{0, 2\}$, but then Fixer wins by Case 81. If the components of $A_S$ have vertex sets $\{v_3\}$, $\{v_0, v_2\}$ and $\{v_1, v_4\}$, then Fixer should swap 1 and 2 at $v_2$ and $v_0$. This results in a position with lists $L(v_1) = \{0, 2\}$, $L(v_2) = \{0, 2\}$, $L(v_3) = \{0, 2, 3, 4\}$, $L(v_4) = \{0, 2\}$ and $L(v_5) = \{0, 2\}$, but then Fixer can edge-color the graph. If the components of $A_S$ have vertex sets $\{v_4\}$, $\{v_0, v_2\}$ and $\{v_1, v_3\}$, then Fixer should swap 1 and 2 at $v_2$ and $v_0$. This results in a position with lists $L(v_1) = \{0, 2\}$, $L(v_2) = \{0, 2\}$, $L(v_3) = \{0, 2, 3, 4\}$, $L(v_4) = \{0, 2\}$ and $L(v_5) = \{0, 2\}$, but then Fixer can edge-color the graph. If the components of $A_S$ have vertex sets $\{v_3\}$, $\{v_0, v_4\}$ and $\{v_1, v_2\}$, then Fixer should swap 1 and 2 at $v_4$ and $v_0$. This results in a position with lists $L(v_1) = \{0, 2\}$, $L(v_2) = \{0, 2\}$, $L(v_3) = \{0, 1, 3, 4\}$, $L(v_4) = \{0, 2\}$ and $L(v_5) = \{0, 1\}$, but then Fixer wins by Case 9. If the components of $A_S$ have vertex sets $\{v_4\}$, $\{v_0, v_3\}$ and $\{v_1, v_2\}$, then Fixer should swap 1 and 2 at $v_3$ and $v_0$. This results in a position with lists $L(v_1) = \{0, 2\}$, $L(v_2) = \{0, 2\}$, $L(v_3) = \{0, 1, 3, 4\}$, $L(v_4) = \{0, 1\}$ and $L(v_5) = \{0, 2\}$, but then Fixer wins by Case 12. 

\noindent\textbf{Case 149.  }\textit{$L(v_1) = \{0, 1\}$, $L(v_2) = \{0, 2\}$, $L(v_3) = \{0, 1, 3, 4\}$, $L(v_4) = \{1, 2\}$ and $L(v_5) = \{0, 1\}$.}

Let $S$ and $A_S$ be as in Lemma \ref{MultiMoveCombination} using colors $2$ and $3$. If the components of $A_S$ have vertex sets $\{v_1\}$ and $\{v_2, v_3\}$, then Fixer should swap 2 and 3 at $v_1$. This results in a position with lists $L(v_1) = \{0, 1\}$, $L(v_2) = \{0, 3\}$, $L(v_3) = \{0, 1, 3, 4\}$, $L(v_4) = \{1, 2\}$ and $L(v_5) = \{0, 1\}$, but then Fixer can edge-color the graph. If the components of $A_S$ have vertex sets $\{v_2\}$ and $\{v_1, v_3\}$, then Fixer should swap 2 and 3 at $v_2$. This results in a position with lists $L(v_1) = \{0, 1\}$, $L(v_2) = \{0, 2\}$, $L(v_3) = \{0, 1, 2, 4\}$, $L(v_4) = \{1, 2\}$ and $L(v_5) = \{0, 1\}$, but then Fixer can edge-color the graph. If the components of $A_S$ have vertex sets $\{v_3\}$ and $\{v_1, v_2\}$, then Fixer should swap 2 and 3 at $v_3$. This results in a position with lists $L(v_1) = \{0, 1\}$, $L(v_2) = \{0, 2\}$, $L(v_3) = \{0, 1, 3, 4\}$, $L(v_4) = \{1, 3\}$ and $L(v_5) = \{0, 1\}$, but then Fixer wins by Case 59. 

\noindent\textbf{Case 150.  }\textit{$L(v_1) = \{0, 1\}$, $L(v_2) = \{0, 2\}$, $L(v_3) = \{0, 1, 3, 4\}$, $L(v_4) = \{0, 3\}$ and $L(v_5) = \{0, 3\}$.}

Let $S$ and $A_S$ be as in Lemma \ref{MultiMoveCombination} using colors $1$ and $2$. If the components of $A_S$ have vertex sets $\{v_0\}$ and $\{v_1, v_2\}$, then Fixer should swap 1 and 2 at $v_0$. This results in a position with lists $L(v_1) = \{0, 2\}$, $L(v_2) = \{0, 2\}$, $L(v_3) = \{0, 1, 3, 4\}$, $L(v_4) = \{0, 3\}$ and $L(v_5) = \{0, 3\}$, but then Fixer wins by Case 13. If the components of $A_S$ have vertex sets $\{v_1\}$ and $\{v_0, v_2\}$, then Fixer should swap 1 and 2 at $v_1$. This results in a position with lists $L(v_1) = \{0, 1\}$, $L(v_2) = \{0, 1\}$, $L(v_3) = \{0, 1, 3, 4\}$, $L(v_4) = \{0, 3\}$ and $L(v_5) = \{0, 3\}$, but then Fixer can edge-color the graph. If the components of $A_S$ have vertex sets $\{v_2\}$ and $\{v_0, v_1\}$, then Fixer should swap 1 and 2 at $v_2$. This results in a position with lists $L(v_1) = \{0, 1\}$, $L(v_2) = \{0, 2\}$, $L(v_3) = \{0, 2, 3, 4\}$, $L(v_4) = \{0, 3\}$ and $L(v_5) = \{0, 3\}$, but then Fixer can edge-color the graph. 

\noindent\textbf{Case 151.  }\textit{$L(v_1) = \{0, 1\}$, $L(v_2) = \{0, 2\}$, $L(v_3) = \{0, 1, 3, 4\}$, $L(v_4) = \{0, 3\}$ and $L(v_5) = \{2, 3\}$.}

Let $S$ and $A_S$ be as in Lemma \ref{MultiMoveCombination} using colors $1$ and $2$. If the components of $A_S$ have vertex sets $\{v_0, v_1\}$ and $\{v_2, v_4\}$, then Fixer should swap 1 and 2 at $v_1$ and $v_0$. This results in a position with lists $L(v_1) = \{0, 2\}$, $L(v_2) = \{0, 1\}$, $L(v_3) = \{0, 1, 3, 4\}$, $L(v_4) = \{0, 3\}$ and $L(v_5) = \{2, 3\}$, but then Fixer can edge-color the graph. If the components of $A_S$ have vertex sets $\{v_0, v_2\}$ and $\{v_1, v_4\}$, then Fixer should swap 1 and 2 at $v_2$ and $v_0$. This results in a position with lists $L(v_1) = \{0, 2\}$, $L(v_2) = \{0, 2\}$, $L(v_3) = \{0, 2, 3, 4\}$, $L(v_4) = \{0, 3\}$ and $L(v_5) = \{2, 3\}$, but then Fixer can edge-color the graph. If the components of $A_S$ have vertex sets $\{v_0, v_4\}$ and $\{v_1, v_2\}$, then Fixer should swap 1 and 2 at $v_4$ and $v_0$. This results in a position with lists $L(v_1) = \{0, 2\}$, $L(v_2) = \{0, 2\}$, $L(v_3) = \{0, 1, 3, 4\}$, $L(v_4) = \{0, 3\}$ and $L(v_5) = \{1, 3\}$, but then Fixer can edge-color the graph. 

\noindent\textbf{Case 152.  }\textit{$L(v_1) = \{0, 1\}$, $L(v_2) = \{0, 2\}$, $L(v_3) = \{0, 1, 3, 4\}$, $L(v_4) = \{1, 3\}$ and $L(v_5) = \{1, 3\}$.}

Let $S$ and $A_S$ be as in Lemma \ref{MultiMoveCombination} using colors $1$ and $2$. If the components of $A_S$ have vertex sets $\{v_0\}$, $\{v_1, v_2\}$ and $\{v_3, v_4\}$, then Fixer should swap 1 and 2 at $v_0$. This results in a position with lists $L(v_1) = \{0, 2\}$, $L(v_2) = \{0, 2\}$, $L(v_3) = \{0, 1, 3, 4\}$, $L(v_4) = \{1, 3\}$ and $L(v_5) = \{1, 3\}$, but then Fixer can edge-color the graph. If the components of $A_S$ have vertex sets $\{v_0\}$, $\{v_1, v_3\}$ and $\{v_2, v_4\}$, then Fixer should swap 1 and 2 at $v_0$. This results in a position with lists $L(v_1) = \{0, 2\}$, $L(v_2) = \{0, 2\}$, $L(v_3) = \{0, 1, 3, 4\}$, $L(v_4) = \{1, 3\}$ and $L(v_5) = \{1, 3\}$, but then Fixer can edge-color the graph. If the components of $A_S$ have vertex sets $\{v_0\}$, $\{v_1, v_4\}$ and $\{v_2, v_3\}$, then Fixer should swap 1 and 2 at $v_0$. This results in a position with lists $L(v_1) = \{0, 2\}$, $L(v_2) = \{0, 2\}$, $L(v_3) = \{0, 1, 3, 4\}$, $L(v_4) = \{1, 3\}$ and $L(v_5) = \{1, 3\}$, but then Fixer can edge-color the graph. If the components of $A_S$ have vertex sets $\{v_1\}$, $\{v_0, v_2\}$ and $\{v_3, v_4\}$, then Fixer should swap 1 and 2 at $v_1$. This results in a position with lists $L(v_1) = \{0, 1\}$, $L(v_2) = \{0, 1\}$, $L(v_3) = \{0, 1, 3, 4\}$, $L(v_4) = \{1, 3\}$ and $L(v_5) = \{1, 3\}$, but then Fixer can edge-color the graph. If the components of $A_S$ have vertex sets $\{v_1\}$, $\{v_0, v_3\}$ and $\{v_2, v_4\}$, then Fixer should swap 1 and 2 at $v_1$. This results in a position with lists $L(v_1) = \{0, 1\}$, $L(v_2) = \{0, 1\}$, $L(v_3) = \{0, 1, 3, 4\}$, $L(v_4) = \{1, 3\}$ and $L(v_5) = \{1, 3\}$, but then Fixer can edge-color the graph. If the components of $A_S$ have vertex sets $\{v_1\}$, $\{v_0, v_4\}$ and $\{v_2, v_3\}$, then Fixer should swap 1 and 2 at $v_1$. This results in a position with lists $L(v_1) = \{0, 1\}$, $L(v_2) = \{0, 1\}$, $L(v_3) = \{0, 1, 3, 4\}$, $L(v_4) = \{1, 3\}$ and $L(v_5) = \{1, 3\}$, but then Fixer can edge-color the graph. If the components of $A_S$ have vertex sets $\{v_2\}$, $\{v_0, v_1\}$ and $\{v_3, v_4\}$, then Fixer should swap 1 and 2 at $v_2$. This results in a position with lists $L(v_1) = \{0, 1\}$, $L(v_2) = \{0, 2\}$, $L(v_3) = \{0, 2, 3, 4\}$, $L(v_4) = \{1, 3\}$ and $L(v_5) = \{1, 3\}$, but then Fixer wins by Case 88. If the components of $A_S$ have vertex sets $\{v_3\}$, $\{v_0, v_1\}$ and $\{v_2, v_4\}$, then Fixer should swap 1 and 2 at $v_1$ and $v_0$. This results in a position with lists $L(v_1) = \{0, 2\}$, $L(v_2) = \{0, 1\}$, $L(v_3) = \{0, 1, 3, 4\}$, $L(v_4) = \{1, 3\}$ and $L(v_5) = \{1, 3\}$, but then Fixer wins by Case 90. If the components of $A_S$ have vertex sets $\{v_4\}$, $\{v_0, v_1\}$ and $\{v_2, v_3\}$, then Fixer should swap 1 and 2 at $v_1$ and $v_0$. This results in a position with lists $L(v_1) = \{0, 2\}$, $L(v_2) = \{0, 1\}$, $L(v_3) = \{0, 1, 3, 4\}$, $L(v_4) = \{1, 3\}$ and $L(v_5) = \{1, 3\}$, but then Fixer wins by Case 90. If the components of $A_S$ have vertex sets $\{v_2\}$, $\{v_0, v_3\}$ and $\{v_1, v_4\}$, then Fixer should swap 1 and 2 at $v_2$. This results in a position with lists $L(v_1) = \{0, 1\}$, $L(v_2) = \{0, 2\}$, $L(v_3) = \{0, 2, 3, 4\}$, $L(v_4) = \{1, 3\}$ and $L(v_5) = \{1, 3\}$, but then Fixer wins by Case 88. If the components of $A_S$ have vertex sets $\{v_2\}$, $\{v_0, v_4\}$ and $\{v_1, v_3\}$, then Fixer should swap 1 and 2 at $v_2$. This results in a position with lists $L(v_1) = \{0, 1\}$, $L(v_2) = \{0, 2\}$, $L(v_3) = \{0, 2, 3, 4\}$, $L(v_4) = \{1, 3\}$ and $L(v_5) = \{1, 3\}$, but then Fixer wins by Case 88. If the components of $A_S$ have vertex sets $\{v_3\}$, $\{v_0, v_2\}$ and $\{v_1, v_4\}$, then Fixer should swap 1 and 2 at $v_2$ and $v_0$. This results in a position with lists $L(v_1) = \{0, 2\}$, $L(v_2) = \{0, 2\}$, $L(v_3) = \{0, 2, 3, 4\}$, $L(v_4) = \{1, 3\}$ and $L(v_5) = \{1, 3\}$, but then Fixer wins by Case 8. If the components of $A_S$ have vertex sets $\{v_4\}$, $\{v_0, v_2\}$ and $\{v_1, v_3\}$, then Fixer should swap 1 and 2 at $v_2$ and $v_0$. This results in a position with lists $L(v_1) = \{0, 2\}$, $L(v_2) = \{0, 2\}$, $L(v_3) = \{0, 2, 3, 4\}$, $L(v_4) = \{1, 3\}$ and $L(v_5) = \{1, 3\}$, but then Fixer wins by Case 8. If the components of $A_S$ have vertex sets $\{v_3\}$, $\{v_0, v_4\}$ and $\{v_1, v_2\}$, then Fixer should swap 1 and 2 at $v_4$ and $v_0$. This results in a position with lists $L(v_1) = \{0, 2\}$, $L(v_2) = \{0, 2\}$, $L(v_3) = \{0, 1, 3, 4\}$, $L(v_4) = \{1, 3\}$ and $L(v_5) = \{2, 3\}$, but then Fixer can edge-color the graph. If the components of $A_S$ have vertex sets $\{v_4\}$, $\{v_0, v_3\}$ and $\{v_1, v_2\}$, then Fixer should swap 1 and 2 at $v_3$ and $v_0$. This results in a position with lists $L(v_1) = \{0, 2\}$, $L(v_2) = \{0, 2\}$, $L(v_3) = \{0, 1, 3, 4\}$, $L(v_4) = \{2, 3\}$ and $L(v_5) = \{1, 3\}$, but then Fixer can edge-color the graph. 

\noindent\textbf{Case 153.  }\textit{$L(v_1) = \{0, 1\}$, $L(v_2) = \{0, 2\}$, $L(v_3) = \{0, 1, 3, 4\}$, $L(v_4) = \{1, 3\}$ and $L(v_5) = \{2, 3\}$.}

Let $S$ and $A_S$ be as in Lemma \ref{MultiMoveCombination} using colors $2$ and $4$. If the components of $A_S$ have vertex sets $\{v_1\}$ and $\{v_2, v_4\}$, then Fixer should swap 2 and 4 at $v_1$. This results in a position with lists $L(v_1) = \{0, 1\}$, $L(v_2) = \{0, 4\}$, $L(v_3) = \{0, 1, 3, 4\}$, $L(v_4) = \{1, 3\}$ and $L(v_5) = \{2, 3\}$, but then Fixer can edge-color the graph. If the components of $A_S$ have vertex sets $\{v_2\}$ and $\{v_1, v_4\}$, then Fixer should swap 2 and 4 at $v_2$. This results in a position with lists $L(v_1) = \{0, 1\}$, $L(v_2) = \{0, 2\}$, $L(v_3) = \{0, 1, 2, 3\}$, $L(v_4) = \{1, 3\}$ and $L(v_5) = \{2, 3\}$, but then Fixer can edge-color the graph. If the components of $A_S$ have vertex sets $\{v_4\}$ and $\{v_1, v_2\}$, then Fixer should swap 2 and 4 at $v_4$. This results in a position with lists $L(v_1) = \{0, 1\}$, $L(v_2) = \{0, 2\}$, $L(v_3) = \{0, 1, 3, 4\}$, $L(v_4) = \{1, 3\}$ and $L(v_5) = \{3, 4\}$, but then Fixer wins by Case 65. 

\noindent\textbf{Case 154.  }\textit{$L(v_1) = \{0, 1\}$, $L(v_2) = \{0, 2\}$, $L(v_3) = \{0, 1, 3, 4\}$, $L(v_4) = \{2, 3\}$ and $L(v_5) = \{0, 3\}$.}

Let $S$ and $A_S$ be as in Lemma \ref{MultiMoveCombination} using colors $1$ and $2$. If the components of $A_S$ have vertex sets $\{v_0, v_1\}$ and $\{v_2, v_3\}$, then Fixer should swap 1 and 2 at $v_1$ and $v_0$. This results in a position with lists $L(v_1) = \{0, 2\}$, $L(v_2) = \{0, 1\}$, $L(v_3) = \{0, 1, 3, 4\}$, $L(v_4) = \{2, 3\}$ and $L(v_5) = \{0, 3\}$, but then Fixer can edge-color the graph. If the components of $A_S$ have vertex sets $\{v_0, v_2\}$ and $\{v_1, v_3\}$, then Fixer should swap 1 and 2 at $v_2$ and $v_0$. This results in a position with lists $L(v_1) = \{0, 2\}$, $L(v_2) = \{0, 2\}$, $L(v_3) = \{0, 2, 3, 4\}$, $L(v_4) = \{2, 3\}$ and $L(v_5) = \{0, 3\}$, but then Fixer can edge-color the graph. If the components of $A_S$ have vertex sets $\{v_0, v_3\}$ and $\{v_1, v_2\}$, then Fixer should swap 1 and 2 at $v_3$ and $v_0$. This results in a position with lists $L(v_1) = \{0, 2\}$, $L(v_2) = \{0, 2\}$, $L(v_3) = \{0, 1, 3, 4\}$, $L(v_4) = \{1, 3\}$ and $L(v_5) = \{0, 3\}$, but then Fixer can edge-color the graph. 

\noindent\textbf{Case 155.  }\textit{$L(v_1) = \{0, 1\}$, $L(v_2) = \{0, 2\}$, $L(v_3) = \{0, 1, 3, 4\}$, $L(v_4) = \{2, 3\}$ and $L(v_5) = \{1, 3\}$.}

Let $S$ and $A_S$ be as in Lemma \ref{MultiMoveCombination} using colors $2$ and $4$. If the components of $A_S$ have vertex sets $\{v_1\}$ and $\{v_2, v_3\}$, then Fixer should swap 2 and 4 at $v_1$. This results in a position with lists $L(v_1) = \{0, 1\}$, $L(v_2) = \{0, 4\}$, $L(v_3) = \{0, 1, 3, 4\}$, $L(v_4) = \{2, 3\}$ and $L(v_5) = \{1, 3\}$, but then Fixer can edge-color the graph. If the components of $A_S$ have vertex sets $\{v_2\}$ and $\{v_1, v_3\}$, then Fixer should swap 2 and 4 at $v_2$. This results in a position with lists $L(v_1) = \{0, 1\}$, $L(v_2) = \{0, 2\}$, $L(v_3) = \{0, 1, 2, 3\}$, $L(v_4) = \{2, 3\}$ and $L(v_5) = \{1, 3\}$, but then Fixer can edge-color the graph. If the components of $A_S$ have vertex sets $\{v_3\}$ and $\{v_1, v_2\}$, then Fixer should swap 2 and 4 at $v_3$. This results in a position with lists $L(v_1) = \{0, 1\}$, $L(v_2) = \{0, 2\}$, $L(v_3) = \{0, 1, 3, 4\}$, $L(v_4) = \{3, 4\}$ and $L(v_5) = \{1, 3\}$, but then Fixer wins by Case 77. 

\noindent\textbf{Case 156.  }\textit{$L(v_1) = \{0, 1\}$, $L(v_2) = \{0, 2\}$, $L(v_3) = \{0, 1, 3, 4\}$, $L(v_4) = \{2, 3\}$ and $L(v_5) = \{2, 3\}$.}

Let $S$ and $A_S$ be as in Lemma \ref{MultiMoveCombination} using colors $1$ and $2$. If the components of $A_S$ have vertex sets $\{v_0\}$, $\{v_1, v_2\}$ and $\{v_3, v_4\}$, then Fixer should swap 1 and 2 at $v_0$. This results in a position with lists $L(v_1) = \{0, 2\}$, $L(v_2) = \{0, 2\}$, $L(v_3) = \{0, 1, 3, 4\}$, $L(v_4) = \{2, 3\}$ and $L(v_5) = \{2, 3\}$, but then Fixer wins by Case 15. If the components of $A_S$ have vertex sets $\{v_0\}$, $\{v_1, v_3\}$ and $\{v_2, v_4\}$, then Fixer should swap 1 and 2 at $v_0$. This results in a position with lists $L(v_1) = \{0, 2\}$, $L(v_2) = \{0, 2\}$, $L(v_3) = \{0, 1, 3, 4\}$, $L(v_4) = \{2, 3\}$ and $L(v_5) = \{2, 3\}$, but then Fixer wins by Case 15. If the components of $A_S$ have vertex sets $\{v_0\}$, $\{v_1, v_4\}$ and $\{v_2, v_3\}$, then Fixer should swap 1 and 2 at $v_0$. This results in a position with lists $L(v_1) = \{0, 2\}$, $L(v_2) = \{0, 2\}$, $L(v_3) = \{0, 1, 3, 4\}$, $L(v_4) = \{2, 3\}$ and $L(v_5) = \{2, 3\}$, but then Fixer wins by Case 15. If the components of $A_S$ have vertex sets $\{v_1\}$, $\{v_0, v_2\}$ and $\{v_3, v_4\}$, then Fixer should swap 1 and 2 at $v_1$. This results in a position with lists $L(v_1) = \{0, 1\}$, $L(v_2) = \{0, 1\}$, $L(v_3) = \{0, 1, 3, 4\}$, $L(v_4) = \{2, 3\}$ and $L(v_5) = \{2, 3\}$, but then Fixer wins by Case 8. If the components of $A_S$ have vertex sets $\{v_1\}$, $\{v_0, v_3\}$ and $\{v_2, v_4\}$, then Fixer should swap 1 and 2 at $v_1$. This results in a position with lists $L(v_1) = \{0, 1\}$, $L(v_2) = \{0, 1\}$, $L(v_3) = \{0, 1, 3, 4\}$, $L(v_4) = \{2, 3\}$ and $L(v_5) = \{2, 3\}$, but then Fixer wins by Case 8. If the components of $A_S$ have vertex sets $\{v_1\}$, $\{v_0, v_4\}$ and $\{v_2, v_3\}$, then Fixer should swap 1 and 2 at $v_1$. This results in a position with lists $L(v_1) = \{0, 1\}$, $L(v_2) = \{0, 1\}$, $L(v_3) = \{0, 1, 3, 4\}$, $L(v_4) = \{2, 3\}$ and $L(v_5) = \{2, 3\}$, but then Fixer wins by Case 8. If the components of $A_S$ have vertex sets $\{v_2\}$, $\{v_0, v_1\}$ and $\{v_3, v_4\}$, then Fixer should swap 1 and 2 at $v_2$. This results in a position with lists $L(v_1) = \{0, 1\}$, $L(v_2) = \{0, 2\}$, $L(v_3) = \{0, 2, 3, 4\}$, $L(v_4) = \{2, 3\}$ and $L(v_5) = \{2, 3\}$, but then Fixer wins by Case 90. If the components of $A_S$ have vertex sets $\{v_3\}$, $\{v_0, v_1\}$ and $\{v_2, v_4\}$, then Fixer should swap 1 and 2 at $v_1$ and $v_0$. This results in a position with lists $L(v_1) = \{0, 2\}$, $L(v_2) = \{0, 1\}$, $L(v_3) = \{0, 1, 3, 4\}$, $L(v_4) = \{2, 3\}$ and $L(v_5) = \{2, 3\}$, but then Fixer wins by Case 88. If the components of $A_S$ have vertex sets $\{v_4\}$, $\{v_0, v_1\}$ and $\{v_2, v_3\}$, then Fixer should swap 1 and 2 at $v_1$ and $v_0$. This results in a position with lists $L(v_1) = \{0, 2\}$, $L(v_2) = \{0, 1\}$, $L(v_3) = \{0, 1, 3, 4\}$, $L(v_4) = \{2, 3\}$ and $L(v_5) = \{2, 3\}$, but then Fixer wins by Case 88. If the components of $A_S$ have vertex sets $\{v_2\}$, $\{v_0, v_3\}$ and $\{v_1, v_4\}$, then Fixer should swap 1 and 2 at $v_2$. This results in a position with lists $L(v_1) = \{0, 1\}$, $L(v_2) = \{0, 2\}$, $L(v_3) = \{0, 2, 3, 4\}$, $L(v_4) = \{2, 3\}$ and $L(v_5) = \{2, 3\}$, but then Fixer wins by Case 90. If the components of $A_S$ have vertex sets $\{v_2\}$, $\{v_0, v_4\}$ and $\{v_1, v_3\}$, then Fixer should swap 1 and 2 at $v_2$. This results in a position with lists $L(v_1) = \{0, 1\}$, $L(v_2) = \{0, 2\}$, $L(v_3) = \{0, 2, 3, 4\}$, $L(v_4) = \{2, 3\}$ and $L(v_5) = \{2, 3\}$, but then Fixer wins by Case 90. If the components of $A_S$ have vertex sets $\{v_3\}$, $\{v_0, v_2\}$ and $\{v_1, v_4\}$, then Fixer should swap 1 and 2 at $v_2$ and $v_0$. This results in a position with lists $L(v_1) = \{0, 2\}$, $L(v_2) = \{0, 2\}$, $L(v_3) = \{0, 2, 3, 4\}$, $L(v_4) = \{2, 3\}$ and $L(v_5) = \{2, 3\}$, but then Fixer can edge-color the graph. If the components of $A_S$ have vertex sets $\{v_4\}$, $\{v_0, v_2\}$ and $\{v_1, v_3\}$, then Fixer should swap 1 and 2 at $v_2$ and $v_0$. This results in a position with lists $L(v_1) = \{0, 2\}$, $L(v_2) = \{0, 2\}$, $L(v_3) = \{0, 2, 3, 4\}$, $L(v_4) = \{2, 3\}$ and $L(v_5) = \{2, 3\}$, but then Fixer can edge-color the graph. If the components of $A_S$ have vertex sets $\{v_3\}$, $\{v_0, v_4\}$ and $\{v_1, v_2\}$, then Fixer should swap 1 and 2 at $v_4$ and $v_0$. This results in a position with lists $L(v_1) = \{0, 2\}$, $L(v_2) = \{0, 2\}$, $L(v_3) = \{0, 1, 3, 4\}$, $L(v_4) = \{2, 3\}$ and $L(v_5) = \{1, 3\}$, but then Fixer can edge-color the graph. If the components of $A_S$ have vertex sets $\{v_4\}$, $\{v_0, v_3\}$ and $\{v_1, v_2\}$, then Fixer should swap 1 and 2 at $v_3$ and $v_0$. This results in a position with lists $L(v_1) = \{0, 2\}$, $L(v_2) = \{0, 2\}$, $L(v_3) = \{0, 1, 3, 4\}$, $L(v_4) = \{1, 3\}$ and $L(v_5) = \{2, 3\}$, but then Fixer can edge-color the graph. 

\noindent\textbf{Case 157.  }\textit{$L(v_1) = \{0, 1\}$, $L(v_2) = \{0, 2\}$, $L(v_3) = \{0, 2, 3, 4\}$, $L(v_4) = \{0, 1\}$ and $L(v_5) = \{0, 2\}$.}

Let $S$ and $A_S$ be as in Lemma \ref{MultiMoveCombination} using colors $1$ and $3$. If the components of $A_S$ have vertex sets $\{v_0\}$ and $\{v_2, v_3\}$, then Fixer should swap 1 and 3 at $v_0$. This results in a position with lists $L(v_1) = \{0, 3\}$, $L(v_2) = \{0, 2\}$, $L(v_3) = \{0, 2, 3, 4\}$, $L(v_4) = \{0, 1\}$ and $L(v_5) = \{0, 2\}$, but then Fixer wins by Case 26. If the components of $A_S$ have vertex sets $\{v_2\}$ and $\{v_0, v_3\}$, then Fixer should swap 1 and 3 at $v_2$. This results in a position with lists $L(v_1) = \{0, 1\}$, $L(v_2) = \{0, 2\}$, $L(v_3) = \{0, 1, 2, 4\}$, $L(v_4) = \{0, 1\}$ and $L(v_5) = \{0, 2\}$, but then Fixer can edge-color the graph. If the components of $A_S$ have vertex sets $\{v_3\}$ and $\{v_0, v_2\}$, then Fixer should swap 1 and 3 at $v_3$. This results in a position with lists $L(v_1) = \{0, 1\}$, $L(v_2) = \{0, 2\}$, $L(v_3) = \{0, 2, 3, 4\}$, $L(v_4) = \{0, 3\}$ and $L(v_5) = \{0, 2\}$, but then Fixer can edge-color the graph. 

\noindent\textbf{Case 158.  }\textit{$L(v_1) = \{0, 1\}$, $L(v_2) = \{0, 2\}$, $L(v_3) = \{0, 2, 3, 4\}$, $L(v_4) = \{0, 1\}$ and $L(v_5) = \{1, 2\}$.}

Let $S$ and $A_S$ be as in Lemma \ref{MultiMoveCombination} using colors $1$ and $3$. If the components of $A_S$ have vertex sets $\{v_0, v_2\}$ and $\{v_3, v_4\}$, then Fixer should swap 1 and 3 at $v_2$ and $v_0$. This results in a position with lists $L(v_1) = \{0, 3\}$, $L(v_2) = \{0, 2\}$, $L(v_3) = \{0, 1, 2, 4\}$, $L(v_4) = \{0, 1\}$ and $L(v_5) = \{1, 2\}$, but then Fixer can edge-color the graph. If the components of $A_S$ have vertex sets $\{v_0, v_3\}$ and $\{v_2, v_4\}$, then Fixer should swap 1 and 3 at $v_3$ and $v_0$. This results in a position with lists $L(v_1) = \{0, 3\}$, $L(v_2) = \{0, 2\}$, $L(v_3) = \{0, 2, 3, 4\}$, $L(v_4) = \{0, 3\}$ and $L(v_5) = \{1, 2\}$, but then Fixer wins by Case 17. If the components of $A_S$ have vertex sets $\{v_0, v_4\}$ and $\{v_2, v_3\}$, then Fixer should swap 1 and 3 at $v_4$ and $v_0$. This results in a position with lists $L(v_1) = \{0, 3\}$, $L(v_2) = \{0, 2\}$, $L(v_3) = \{0, 2, 3, 4\}$, $L(v_4) = \{0, 1\}$ and $L(v_5) = \{2, 3\}$, but then Fixer can edge-color the graph. 

\noindent\textbf{Case 159.  }\textit{$L(v_1) = \{0, 1\}$, $L(v_2) = \{0, 2\}$, $L(v_3) = \{0, 2, 3, 4\}$, $L(v_4) = \{0, 2\}$ and $L(v_5) = \{0, 1\}$.}

Let $S$ and $A_S$ be as in Lemma \ref{MultiMoveCombination} using colors $1$ and $3$. If the components of $A_S$ have vertex sets $\{v_0\}$ and $\{v_2, v_4\}$, then Fixer should swap 1 and 3 at $v_0$. This results in a position with lists $L(v_1) = \{0, 3\}$, $L(v_2) = \{0, 2\}$, $L(v_3) = \{0, 2, 3, 4\}$, $L(v_4) = \{0, 2\}$ and $L(v_5) = \{0, 1\}$, but then Fixer wins by Case 18. If the components of $A_S$ have vertex sets $\{v_2\}$ and $\{v_0, v_4\}$, then Fixer should swap 1 and 3 at $v_2$. This results in a position with lists $L(v_1) = \{0, 1\}$, $L(v_2) = \{0, 2\}$, $L(v_3) = \{0, 1, 2, 4\}$, $L(v_4) = \{0, 2\}$ and $L(v_5) = \{0, 1\}$, but then Fixer can edge-color the graph. If the components of $A_S$ have vertex sets $\{v_4\}$ and $\{v_0, v_2\}$, then Fixer should swap 1 and 3 at $v_4$. This results in a position with lists $L(v_1) = \{0, 1\}$, $L(v_2) = \{0, 2\}$, $L(v_3) = \{0, 2, 3, 4\}$, $L(v_4) = \{0, 2\}$ and $L(v_5) = \{0, 3\}$, but then Fixer can edge-color the graph. 

\noindent\textbf{Case 160.  }\textit{$L(v_1) = \{0, 1\}$, $L(v_2) = \{0, 2\}$, $L(v_3) = \{0, 2, 3, 4\}$, $L(v_4) = \{0, 2\}$ and $L(v_5) = \{1, 2\}$.}

Let $S$ and $A_S$ be as in Lemma \ref{MultiMoveCombination} using colors $1$ and $3$. If the components of $A_S$ have vertex sets $\{v_0\}$ and $\{v_2, v_4\}$, then Fixer should swap 1 and 3 at $v_0$. This results in a position with lists $L(v_1) = \{0, 3\}$, $L(v_2) = \{0, 2\}$, $L(v_3) = \{0, 2, 3, 4\}$, $L(v_4) = \{0, 2\}$ and $L(v_5) = \{1, 2\}$, but then Fixer wins by Case 19. If the components of $A_S$ have vertex sets $\{v_2\}$ and $\{v_0, v_4\}$, then Fixer should swap 1 and 3 at $v_2$. This results in a position with lists $L(v_1) = \{0, 1\}$, $L(v_2) = \{0, 2\}$, $L(v_3) = \{0, 1, 2, 4\}$, $L(v_4) = \{0, 2\}$ and $L(v_5) = \{1, 2\}$, but then Fixer can edge-color the graph. If the components of $A_S$ have vertex sets $\{v_4\}$ and $\{v_0, v_2\}$, then Fixer should swap 1 and 3 at $v_4$. This results in a position with lists $L(v_1) = \{0, 1\}$, $L(v_2) = \{0, 2\}$, $L(v_3) = \{0, 2, 3, 4\}$, $L(v_4) = \{0, 2\}$ and $L(v_5) = \{2, 3\}$, but then Fixer can edge-color the graph. 

\noindent\textbf{Case 161.  }\textit{$L(v_1) = \{0, 1\}$, $L(v_2) = \{0, 2\}$, $L(v_3) = \{0, 2, 3, 4\}$, $L(v_4) = \{1, 2\}$ and $L(v_5) = \{0, 1\}$.}

Let $S$ and $A_S$ be as in Lemma \ref{MultiMoveCombination} using colors $1$ and $3$. If the components of $A_S$ have vertex sets $\{v_0, v_2\}$ and $\{v_3, v_4\}$, then Fixer should swap 1 and 3 at $v_2$ and $v_0$. This results in a position with lists $L(v_1) = \{0, 3\}$, $L(v_2) = \{0, 2\}$, $L(v_3) = \{0, 1, 2, 4\}$, $L(v_4) = \{1, 2\}$ and $L(v_5) = \{0, 1\}$, but then Fixer can edge-color the graph. If the components of $A_S$ have vertex sets $\{v_0, v_3\}$ and $\{v_2, v_4\}$, then Fixer should swap 1 and 3 at $v_3$ and $v_0$. This results in a position with lists $L(v_1) = \{0, 3\}$, $L(v_2) = \{0, 2\}$, $L(v_3) = \{0, 2, 3, 4\}$, $L(v_4) = \{2, 3\}$ and $L(v_5) = \{0, 1\}$, but then Fixer can edge-color the graph. If the components of $A_S$ have vertex sets $\{v_0, v_4\}$ and $\{v_2, v_3\}$, then Fixer should swap 1 and 3 at $v_4$ and $v_0$. This results in a position with lists $L(v_1) = \{0, 3\}$, $L(v_2) = \{0, 2\}$, $L(v_3) = \{0, 2, 3, 4\}$, $L(v_4) = \{1, 2\}$ and $L(v_5) = \{0, 3\}$, but then Fixer wins by Case 30. 

\noindent\textbf{Case 162.  }\textit{$L(v_1) = \{0, 1\}$, $L(v_2) = \{0, 2\}$, $L(v_3) = \{0, 2, 3, 4\}$, $L(v_4) = \{1, 2\}$ and $L(v_5) = \{0, 2\}$.}

Let $S$ and $A_S$ be as in Lemma \ref{MultiMoveCombination} using colors $1$ and $3$. If the components of $A_S$ have vertex sets $\{v_0\}$ and $\{v_2, v_3\}$, then Fixer should swap 1 and 3 at $v_0$. This results in a position with lists $L(v_1) = \{0, 3\}$, $L(v_2) = \{0, 2\}$, $L(v_3) = \{0, 2, 3, 4\}$, $L(v_4) = \{1, 2\}$ and $L(v_5) = \{0, 2\}$, but then Fixer wins by Case 31. If the components of $A_S$ have vertex sets $\{v_2\}$ and $\{v_0, v_3\}$, then Fixer should swap 1 and 3 at $v_2$. This results in a position with lists $L(v_1) = \{0, 1\}$, $L(v_2) = \{0, 2\}$, $L(v_3) = \{0, 1, 2, 4\}$, $L(v_4) = \{1, 2\}$ and $L(v_5) = \{0, 2\}$, but then Fixer can edge-color the graph. If the components of $A_S$ have vertex sets $\{v_3\}$ and $\{v_0, v_2\}$, then Fixer should swap 1 and 3 at $v_3$. This results in a position with lists $L(v_1) = \{0, 1\}$, $L(v_2) = \{0, 2\}$, $L(v_3) = \{0, 2, 3, 4\}$, $L(v_4) = \{2, 3\}$ and $L(v_5) = \{0, 2\}$, but then Fixer can edge-color the graph. 

\noindent\textbf{Case 163.  }\textit{$L(v_1) = \{0, 1\}$, $L(v_2) = \{0, 2\}$, $L(v_3) = \{0, 2, 3, 4\}$, $L(v_4) = \{1, 2\}$ and $L(v_5) = \{1, 2\}$.}

Let $S$ and $A_S$ be as in Lemma \ref{MultiMoveCombination} using colors $1$ and $3$. If the components of $A_S$ have vertex sets $\{v_0, v_2\}$ and $\{v_3, v_4\}$, then Fixer should swap 1 and 3 at $v_2$ and $v_0$. This results in a position with lists $L(v_1) = \{0, 3\}$, $L(v_2) = \{0, 2\}$, $L(v_3) = \{0, 1, 2, 4\}$, $L(v_4) = \{1, 2\}$ and $L(v_5) = \{1, 2\}$, but then Fixer wins by Case 90. If the components of $A_S$ have vertex sets $\{v_0, v_3\}$ and $\{v_2, v_4\}$, then Fixer should swap 1 and 3 at $v_3$ and $v_0$. This results in a position with lists $L(v_1) = \{0, 3\}$, $L(v_2) = \{0, 2\}$, $L(v_3) = \{0, 2, 3, 4\}$, $L(v_4) = \{2, 3\}$ and $L(v_5) = \{1, 2\}$, but then Fixer wins by Case 21. If the components of $A_S$ have vertex sets $\{v_0, v_4\}$ and $\{v_2, v_3\}$, then Fixer should swap 1 and 3 at $v_4$ and $v_0$. This results in a position with lists $L(v_1) = \{0, 3\}$, $L(v_2) = \{0, 2\}$, $L(v_3) = \{0, 2, 3, 4\}$, $L(v_4) = \{1, 2\}$ and $L(v_5) = \{2, 3\}$, but then Fixer wins by Case 32. 

\noindent\textbf{Case 164.  }\textit{$L(v_1) = \{0, 1\}$, $L(v_2) = \{0, 2\}$, $L(v_3) = \{0, 2, 3, 4\}$, $L(v_4) = \{1, 3\}$ and $L(v_5) = \{2, 3\}$.}

Let $S$ and $A_S$ be as in Lemma \ref{MultiMoveCombination} using colors $1$ and $4$. If the components of $A_S$ have vertex sets $\{v_0\}$ and $\{v_2, v_3\}$, then Fixer should swap 1 and 4 at $v_0$. This results in a position with lists $L(v_1) = \{0, 4\}$, $L(v_2) = \{0, 2\}$, $L(v_3) = \{0, 2, 3, 4\}$, $L(v_4) = \{1, 3\}$ and $L(v_5) = \{2, 3\}$, but then Fixer wins by Case 37. If the components of $A_S$ have vertex sets $\{v_2\}$ and $\{v_0, v_3\}$, then Fixer should swap 1 and 4 at $v_2$. This results in a position with lists $L(v_1) = \{0, 1\}$, $L(v_2) = \{0, 2\}$, $L(v_3) = \{0, 1, 2, 3\}$, $L(v_4) = \{1, 3\}$ and $L(v_5) = \{2, 3\}$, but then Fixer can edge-color the graph. If the components of $A_S$ have vertex sets $\{v_3\}$ and $\{v_0, v_2\}$, then Fixer should swap 1 and 4 at $v_3$. This results in a position with lists $L(v_1) = \{0, 1\}$, $L(v_2) = \{0, 2\}$, $L(v_3) = \{0, 2, 3, 4\}$, $L(v_4) = \{3, 4\}$ and $L(v_5) = \{2, 3\}$, but then Fixer can edge-color the graph. 

\noindent\textbf{Case 165.  }\textit{$L(v_1) = \{0, 1\}$, $L(v_2) = \{0, 2\}$, $L(v_3) = \{0, 2, 3, 4\}$, $L(v_4) = \{2, 3\}$ and $L(v_5) = \{1, 3\}$.}

Let $S$ and $A_S$ be as in Lemma \ref{MultiMoveCombination} using colors $1$ and $4$. If the components of $A_S$ have vertex sets $\{v_0\}$ and $\{v_2, v_4\}$, then Fixer should swap 1 and 4 at $v_0$. This results in a position with lists $L(v_1) = \{0, 4\}$, $L(v_2) = \{0, 2\}$, $L(v_3) = \{0, 2, 3, 4\}$, $L(v_4) = \{2, 3\}$ and $L(v_5) = \{1, 3\}$, but then Fixer wins by Case 25. If the components of $A_S$ have vertex sets $\{v_2\}$ and $\{v_0, v_4\}$, then Fixer should swap 1 and 4 at $v_2$. This results in a position with lists $L(v_1) = \{0, 1\}$, $L(v_2) = \{0, 2\}$, $L(v_3) = \{0, 1, 2, 3\}$, $L(v_4) = \{2, 3\}$ and $L(v_5) = \{1, 3\}$, but then Fixer can edge-color the graph. If the components of $A_S$ have vertex sets $\{v_4\}$ and $\{v_0, v_2\}$, then Fixer should swap 1 and 4 at $v_4$. This results in a position with lists $L(v_1) = \{0, 1\}$, $L(v_2) = \{0, 2\}$, $L(v_3) = \{0, 2, 3, 4\}$, $L(v_4) = \{2, 3\}$ and $L(v_5) = \{3, 4\}$, but then Fixer can edge-color the graph. 

\noindent\textbf{Case 166.  }\textit{$L(v_1) = \{0, 1\}$, $L(v_2) = \{0, 2\}$, $L(v_3) = \{1, 2, 3, 4\}$, $L(v_4) = \{0, 1\}$ and $L(v_5) = \{0, 2\}$.}

Let $S$ and $A_S$ be as in Lemma \ref{MultiMoveCombination} using colors $0$ and $3$. If the components of $A_S$ have vertex sets $\{v_0\}$, $\{v_1, v_2\}$ and $\{v_3, v_4\}$, then Fixer should swap 0 and 3 at $v_4$ and $v_3$. This results in a position with lists $L(v_1) = \{0, 1\}$, $L(v_2) = \{0, 2\}$, $L(v_3) = \{1, 2, 3, 4\}$, $L(v_4) = \{1, 3\}$ and $L(v_5) = \{2, 3\}$, but then Fixer can edge-color the graph. If the components of $A_S$ have vertex sets $\{v_0\}$, $\{v_1, v_3\}$ and $\{v_2, v_4\}$, then Fixer should swap 0 and 3 at $v_4$ and $v_2$. This results in a position with lists $L(v_1) = \{0, 1\}$, $L(v_2) = \{0, 2\}$, $L(v_3) = \{0, 1, 2, 4\}$, $L(v_4) = \{0, 1\}$ and $L(v_5) = \{2, 3\}$, but then Fixer wins by Case 17. If the components of $A_S$ have vertex sets $\{v_0\}$, $\{v_1, v_4\}$ and $\{v_2, v_3\}$, then Fixer should swap 0 and 3 at $v_3$ and $v_2$. This results in a position with lists $L(v_1) = \{0, 1\}$, $L(v_2) = \{0, 2\}$, $L(v_3) = \{0, 1, 2, 4\}$, $L(v_4) = \{1, 3\}$ and $L(v_5) = \{0, 2\}$, but then Fixer can edge-color the graph. If the components of $A_S$ have vertex sets $\{v_1\}$, $\{v_0, v_2\}$ and $\{v_3, v_4\}$, then Fixer should swap 0 and 3 at $v_4$ and $v_3$. This results in a position with lists $L(v_1) = \{0, 1\}$, $L(v_2) = \{0, 2\}$, $L(v_3) = \{1, 2, 3, 4\}$, $L(v_4) = \{1, 3\}$ and $L(v_5) = \{2, 3\}$, but then Fixer can edge-color the graph. If the components of $A_S$ have vertex sets $\{v_1\}$, $\{v_0, v_3\}$ and $\{v_2, v_4\}$, then Fixer should swap 0 and 3 at $v_4$ and $v_2$. This results in a position with lists $L(v_1) = \{0, 1\}$, $L(v_2) = \{0, 2\}$, $L(v_3) = \{0, 1, 2, 4\}$, $L(v_4) = \{0, 1\}$ and $L(v_5) = \{2, 3\}$, but then Fixer wins by Case 17. If the components of $A_S$ have vertex sets $\{v_1\}$, $\{v_0, v_4\}$ and $\{v_2, v_3\}$, then Fixer should swap 0 and 3 at $v_3$ and $v_2$. This results in a position with lists $L(v_1) = \{0, 1\}$, $L(v_2) = \{0, 2\}$, $L(v_3) = \{0, 1, 2, 4\}$, $L(v_4) = \{1, 3\}$ and $L(v_5) = \{0, 2\}$, but then Fixer can edge-color the graph. If the components of $A_S$ have vertex sets $\{v_2\}$, $\{v_0, v_1\}$ and $\{v_3, v_4\}$, then Fixer should swap 0 and 3 at $v_2$. This results in a position with lists $L(v_1) = \{0, 1\}$, $L(v_2) = \{0, 2\}$, $L(v_3) = \{0, 1, 2, 4\}$, $L(v_4) = \{0, 1\}$ and $L(v_5) = \{0, 2\}$, but then Fixer can edge-color the graph. If the components of $A_S$ have vertex sets $\{v_3\}$, $\{v_0, v_1\}$ and $\{v_2, v_4\}$, then Fixer should swap 0 and 3 at $v_3$. This results in a position with lists $L(v_1) = \{0, 1\}$, $L(v_2) = \{0, 2\}$, $L(v_3) = \{1, 2, 3, 4\}$, $L(v_4) = \{1, 3\}$ and $L(v_5) = \{0, 2\}$, but then Fixer wins by Case 100. If the components of $A_S$ have vertex sets $\{v_4\}$, $\{v_0, v_1\}$ and $\{v_2, v_3\}$, then Fixer should swap 0 and 3 at $v_4$. This results in a position with lists $L(v_1) = \{0, 1\}$, $L(v_2) = \{0, 2\}$, $L(v_3) = \{1, 2, 3, 4\}$, $L(v_4) = \{0, 1\}$ and $L(v_5) = \{2, 3\}$, but then Fixer can edge-color the graph. If the components of $A_S$ have vertex sets $\{v_2\}$, $\{v_0, v_3\}$ and $\{v_1, v_4\}$, then Fixer should swap 0 and 3 at $v_2$. This results in a position with lists $L(v_1) = \{0, 1\}$, $L(v_2) = \{0, 2\}$, $L(v_3) = \{0, 1, 2, 4\}$, $L(v_4) = \{0, 1\}$ and $L(v_5) = \{0, 2\}$, but then Fixer can edge-color the graph. If the components of $A_S$ have vertex sets $\{v_2\}$, $\{v_0, v_4\}$ and $\{v_1, v_3\}$, then Fixer should swap 0 and 3 at $v_2$. This results in a position with lists $L(v_1) = \{0, 1\}$, $L(v_2) = \{0, 2\}$, $L(v_3) = \{0, 1, 2, 4\}$, $L(v_4) = \{0, 1\}$ and $L(v_5) = \{0, 2\}$, but then Fixer can edge-color the graph. If the components of $A_S$ have vertex sets $\{v_3\}$, $\{v_0, v_2\}$ and $\{v_1, v_4\}$, then Fixer should swap 0 and 3 at $v_3$. This results in a position with lists $L(v_1) = \{0, 1\}$, $L(v_2) = \{0, 2\}$, $L(v_3) = \{1, 2, 3, 4\}$, $L(v_4) = \{1, 3\}$ and $L(v_5) = \{0, 2\}$, but then Fixer wins by Case 100. If the components of $A_S$ have vertex sets $\{v_4\}$, $\{v_0, v_2\}$ and $\{v_1, v_3\}$, then Fixer should swap 0 and 3 at $v_4$. This results in a position with lists $L(v_1) = \{0, 1\}$, $L(v_2) = \{0, 2\}$, $L(v_3) = \{1, 2, 3, 4\}$, $L(v_4) = \{0, 1\}$ and $L(v_5) = \{2, 3\}$, but then Fixer can edge-color the graph. If the components of $A_S$ have vertex sets $\{v_3\}$, $\{v_0, v_4\}$ and $\{v_1, v_2\}$, then Fixer should swap 0 and 3 at $v_3$. This results in a position with lists $L(v_1) = \{0, 1\}$, $L(v_2) = \{0, 2\}$, $L(v_3) = \{1, 2, 3, 4\}$, $L(v_4) = \{1, 3\}$ and $L(v_5) = \{0, 2\}$, but then Fixer wins by Case 100. If the components of $A_S$ have vertex sets $\{v_4\}$, $\{v_0, v_3\}$ and $\{v_1, v_2\}$, then Fixer should swap 0 and 3 at $v_4$. This results in a position with lists $L(v_1) = \{0, 1\}$, $L(v_2) = \{0, 2\}$, $L(v_3) = \{1, 2, 3, 4\}$, $L(v_4) = \{0, 1\}$ and $L(v_5) = \{2, 3\}$, but then Fixer can edge-color the graph. 

\noindent\textbf{Case 167.  }\textit{$L(v_1) = \{0, 1\}$, $L(v_2) = \{0, 2\}$, $L(v_3) = \{1, 2, 3, 4\}$, $L(v_4) = \{0, 2\}$ and $L(v_5) = \{0, 1\}$.}

Let $S$ and $A_S$ be as in Lemma \ref{MultiMoveCombination} using colors $0$ and $3$. If the components of $A_S$ have vertex sets $\{v_0\}$, $\{v_1, v_2\}$ and $\{v_3, v_4\}$, then Fixer should swap 0 and 3 at $v_4$ and $v_3$. This results in a position with lists $L(v_1) = \{0, 1\}$, $L(v_2) = \{0, 2\}$, $L(v_3) = \{1, 2, 3, 4\}$, $L(v_4) = \{2, 3\}$ and $L(v_5) = \{1, 3\}$, but then Fixer can edge-color the graph. If the components of $A_S$ have vertex sets $\{v_0\}$, $\{v_1, v_3\}$ and $\{v_2, v_4\}$, then Fixer should swap 0 and 3 at $v_4$ and $v_2$. This results in a position with lists $L(v_1) = \{0, 1\}$, $L(v_2) = \{0, 2\}$, $L(v_3) = \{0, 1, 2, 4\}$, $L(v_4) = \{0, 2\}$ and $L(v_5) = \{1, 3\}$, but then Fixer can edge-color the graph. If the components of $A_S$ have vertex sets $\{v_0\}$, $\{v_1, v_4\}$ and $\{v_2, v_3\}$, then Fixer should swap 0 and 3 at $v_3$ and $v_2$. This results in a position with lists $L(v_1) = \{0, 1\}$, $L(v_2) = \{0, 2\}$, $L(v_3) = \{0, 1, 2, 4\}$, $L(v_4) = \{2, 3\}$ and $L(v_5) = \{0, 1\}$, but then Fixer wins by Case 30. If the components of $A_S$ have vertex sets $\{v_1\}$, $\{v_0, v_2\}$ and $\{v_3, v_4\}$, then Fixer should swap 0 and 3 at $v_4$ and $v_3$. This results in a position with lists $L(v_1) = \{0, 1\}$, $L(v_2) = \{0, 2\}$, $L(v_3) = \{1, 2, 3, 4\}$, $L(v_4) = \{2, 3\}$ and $L(v_5) = \{1, 3\}$, but then Fixer can edge-color the graph. If the components of $A_S$ have vertex sets $\{v_1\}$, $\{v_0, v_3\}$ and $\{v_2, v_4\}$, then Fixer should swap 0 and 3 at $v_4$ and $v_2$. This results in a position with lists $L(v_1) = \{0, 1\}$, $L(v_2) = \{0, 2\}$, $L(v_3) = \{0, 1, 2, 4\}$, $L(v_4) = \{0, 2\}$ and $L(v_5) = \{1, 3\}$, but then Fixer can edge-color the graph. If the components of $A_S$ have vertex sets $\{v_1\}$, $\{v_0, v_4\}$ and $\{v_2, v_3\}$, then Fixer should swap 0 and 3 at $v_3$ and $v_2$. This results in a position with lists $L(v_1) = \{0, 1\}$, $L(v_2) = \{0, 2\}$, $L(v_3) = \{0, 1, 2, 4\}$, $L(v_4) = \{2, 3\}$ and $L(v_5) = \{0, 1\}$, but then Fixer wins by Case 30. If the components of $A_S$ have vertex sets $\{v_2\}$, $\{v_0, v_1\}$ and $\{v_3, v_4\}$, then Fixer should swap 0 and 3 at $v_2$. This results in a position with lists $L(v_1) = \{0, 1\}$, $L(v_2) = \{0, 2\}$, $L(v_3) = \{0, 1, 2, 4\}$, $L(v_4) = \{0, 2\}$ and $L(v_5) = \{0, 1\}$, but then Fixer can edge-color the graph. If the components of $A_S$ have vertex sets $\{v_3\}$, $\{v_0, v_1\}$ and $\{v_2, v_4\}$, then Fixer should swap 0 and 3 at $v_3$. This results in a position with lists $L(v_1) = \{0, 1\}$, $L(v_2) = \{0, 2\}$, $L(v_3) = \{1, 2, 3, 4\}$, $L(v_4) = \{2, 3\}$ and $L(v_5) = \{0, 1\}$, but then Fixer can edge-color the graph. If the components of $A_S$ have vertex sets $\{v_4\}$, $\{v_0, v_1\}$ and $\{v_2, v_3\}$, then Fixer should swap 0 and 3 at $v_4$. This results in a position with lists $L(v_1) = \{0, 1\}$, $L(v_2) = \{0, 2\}$, $L(v_3) = \{1, 2, 3, 4\}$, $L(v_4) = \{0, 2\}$ and $L(v_5) = \{1, 3\}$, but then Fixer wins by Case 94. If the components of $A_S$ have vertex sets $\{v_2\}$, $\{v_0, v_3\}$ and $\{v_1, v_4\}$, then Fixer should swap 0 and 3 at $v_2$. This results in a position with lists $L(v_1) = \{0, 1\}$, $L(v_2) = \{0, 2\}$, $L(v_3) = \{0, 1, 2, 4\}$, $L(v_4) = \{0, 2\}$ and $L(v_5) = \{0, 1\}$, but then Fixer can edge-color the graph. If the components of $A_S$ have vertex sets $\{v_2\}$, $\{v_0, v_4\}$ and $\{v_1, v_3\}$, then Fixer should swap 0 and 3 at $v_2$. This results in a position with lists $L(v_1) = \{0, 1\}$, $L(v_2) = \{0, 2\}$, $L(v_3) = \{0, 1, 2, 4\}$, $L(v_4) = \{0, 2\}$ and $L(v_5) = \{0, 1\}$, but then Fixer can edge-color the graph. If the components of $A_S$ have vertex sets $\{v_3\}$, $\{v_0, v_2\}$ and $\{v_1, v_4\}$, then Fixer should swap 0 and 3 at $v_3$. This results in a position with lists $L(v_1) = \{0, 1\}$, $L(v_2) = \{0, 2\}$, $L(v_3) = \{1, 2, 3, 4\}$, $L(v_4) = \{2, 3\}$ and $L(v_5) = \{0, 1\}$, but then Fixer can edge-color the graph. If the components of $A_S$ have vertex sets $\{v_4\}$, $\{v_0, v_2\}$ and $\{v_1, v_3\}$, then Fixer should swap 0 and 3 at $v_4$. This results in a position with lists $L(v_1) = \{0, 1\}$, $L(v_2) = \{0, 2\}$, $L(v_3) = \{1, 2, 3, 4\}$, $L(v_4) = \{0, 2\}$ and $L(v_5) = \{1, 3\}$, but then Fixer wins by Case 94. If the components of $A_S$ have vertex sets $\{v_3\}$, $\{v_0, v_4\}$ and $\{v_1, v_2\}$, then Fixer should swap 0 and 3 at $v_3$. This results in a position with lists $L(v_1) = \{0, 1\}$, $L(v_2) = \{0, 2\}$, $L(v_3) = \{1, 2, 3, 4\}$, $L(v_4) = \{2, 3\}$ and $L(v_5) = \{0, 1\}$, but then Fixer can edge-color the graph. If the components of $A_S$ have vertex sets $\{v_4\}$, $\{v_0, v_3\}$ and $\{v_1, v_2\}$, then Fixer should swap 0 and 3 at $v_4$. This results in a position with lists $L(v_1) = \{0, 1\}$, $L(v_2) = \{0, 2\}$, $L(v_3) = \{1, 2, 3, 4\}$, $L(v_4) = \{0, 2\}$ and $L(v_5) = \{1, 3\}$, but then Fixer wins by Case 94. 

\noindent\textbf{Case 168.  }\textit{$L(v_1) = \{0, 1\}$, $L(v_2) = \{0, 2\}$, $L(v_3) = \{1, 2, 3, 4\}$, $L(v_4) = \{0, 2\}$ and $L(v_5) = \{0, 2\}$.}

Let $S$ and $A_S$ be as in Lemma \ref{MultiMoveCombination} using colors $0$ and $3$. If the components of $A_S$ have vertex sets $\{v_0\}$, $\{v_1, v_2\}$ and $\{v_3, v_4\}$, then Fixer should swap 0 and 3 at $v_4$ and $v_3$. This results in a position with lists $L(v_1) = \{0, 1\}$, $L(v_2) = \{0, 2\}$, $L(v_3) = \{1, 2, 3, 4\}$, $L(v_4) = \{2, 3\}$ and $L(v_5) = \{2, 3\}$, but then Fixer wins by Case 102. If the components of $A_S$ have vertex sets $\{v_0\}$, $\{v_1, v_3\}$ and $\{v_2, v_4\}$, then Fixer should swap 0 and 3 at $v_4$ and $v_2$. This results in a position with lists $L(v_1) = \{0, 1\}$, $L(v_2) = \{0, 2\}$, $L(v_3) = \{0, 1, 2, 4\}$, $L(v_4) = \{0, 2\}$ and $L(v_5) = \{2, 3\}$, but then Fixer wins by Case 19. If the components of $A_S$ have vertex sets $\{v_0\}$, $\{v_1, v_4\}$ and $\{v_2, v_3\}$, then Fixer should swap 0 and 3 at $v_3$ and $v_2$. This results in a position with lists $L(v_1) = \{0, 1\}$, $L(v_2) = \{0, 2\}$, $L(v_3) = \{0, 1, 2, 4\}$, $L(v_4) = \{2, 3\}$ and $L(v_5) = \{0, 2\}$, but then Fixer wins by Case 31. If the components of $A_S$ have vertex sets $\{v_1\}$, $\{v_0, v_2\}$ and $\{v_3, v_4\}$, then Fixer should swap 0 and 3 at $v_4$ and $v_3$. This results in a position with lists $L(v_1) = \{0, 1\}$, $L(v_2) = \{0, 2\}$, $L(v_3) = \{1, 2, 3, 4\}$, $L(v_4) = \{2, 3\}$ and $L(v_5) = \{2, 3\}$, but then Fixer wins by Case 102. If the components of $A_S$ have vertex sets $\{v_1\}$, $\{v_0, v_3\}$ and $\{v_2, v_4\}$, then Fixer should swap 0 and 3 at $v_4$ and $v_2$. This results in a position with lists $L(v_1) = \{0, 1\}$, $L(v_2) = \{0, 2\}$, $L(v_3) = \{0, 1, 2, 4\}$, $L(v_4) = \{0, 2\}$ and $L(v_5) = \{2, 3\}$, but then Fixer wins by Case 19. If the components of $A_S$ have vertex sets $\{v_1\}$, $\{v_0, v_4\}$ and $\{v_2, v_3\}$, then Fixer should swap 0 and 3 at $v_3$ and $v_2$. This results in a position with lists $L(v_1) = \{0, 1\}$, $L(v_2) = \{0, 2\}$, $L(v_3) = \{0, 1, 2, 4\}$, $L(v_4) = \{2, 3\}$ and $L(v_5) = \{0, 2\}$, but then Fixer wins by Case 31. If the components of $A_S$ have vertex sets $\{v_2\}$, $\{v_0, v_1\}$ and $\{v_3, v_4\}$, then Fixer should swap 0 and 3 at $v_2$. This results in a position with lists $L(v_1) = \{0, 1\}$, $L(v_2) = \{0, 2\}$, $L(v_3) = \{0, 1, 2, 4\}$, $L(v_4) = \{0, 2\}$ and $L(v_5) = \{0, 2\}$, but then Fixer can edge-color the graph. If the components of $A_S$ have vertex sets $\{v_3\}$, $\{v_0, v_1\}$ and $\{v_2, v_4\}$, then Fixer should swap 0 and 3 at $v_3$. This results in a position with lists $L(v_1) = \{0, 1\}$, $L(v_2) = \{0, 2\}$, $L(v_3) = \{1, 2, 3, 4\}$, $L(v_4) = \{2, 3\}$ and $L(v_5) = \{0, 2\}$, but then Fixer wins by Case 101. If the components of $A_S$ have vertex sets $\{v_4\}$, $\{v_0, v_1\}$ and $\{v_2, v_3\}$, then Fixer should swap 0 and 3 at $v_4$. This results in a position with lists $L(v_1) = \{0, 1\}$, $L(v_2) = \{0, 2\}$, $L(v_3) = \{1, 2, 3, 4\}$, $L(v_4) = \{0, 2\}$ and $L(v_5) = \{2, 3\}$, but then Fixer wins by Case 95. If the components of $A_S$ have vertex sets $\{v_2\}$, $\{v_0, v_3\}$ and $\{v_1, v_4\}$, then Fixer should swap 0 and 3 at $v_2$. This results in a position with lists $L(v_1) = \{0, 1\}$, $L(v_2) = \{0, 2\}$, $L(v_3) = \{0, 1, 2, 4\}$, $L(v_4) = \{0, 2\}$ and $L(v_5) = \{0, 2\}$, but then Fixer can edge-color the graph. If the components of $A_S$ have vertex sets $\{v_2\}$, $\{v_0, v_4\}$ and $\{v_1, v_3\}$, then Fixer should swap 0 and 3 at $v_2$. This results in a position with lists $L(v_1) = \{0, 1\}$, $L(v_2) = \{0, 2\}$, $L(v_3) = \{0, 1, 2, 4\}$, $L(v_4) = \{0, 2\}$ and $L(v_5) = \{0, 2\}$, but then Fixer can edge-color the graph. If the components of $A_S$ have vertex sets $\{v_3\}$, $\{v_0, v_2\}$ and $\{v_1, v_4\}$, then Fixer should swap 0 and 3 at $v_3$. This results in a position with lists $L(v_1) = \{0, 1\}$, $L(v_2) = \{0, 2\}$, $L(v_3) = \{1, 2, 3, 4\}$, $L(v_4) = \{2, 3\}$ and $L(v_5) = \{0, 2\}$, but then Fixer wins by Case 101. If the components of $A_S$ have vertex sets $\{v_4\}$, $\{v_0, v_2\}$ and $\{v_1, v_3\}$, then Fixer should swap 0 and 3 at $v_4$. This results in a position with lists $L(v_1) = \{0, 1\}$, $L(v_2) = \{0, 2\}$, $L(v_3) = \{1, 2, 3, 4\}$, $L(v_4) = \{0, 2\}$ and $L(v_5) = \{2, 3\}$, but then Fixer wins by Case 95. If the components of $A_S$ have vertex sets $\{v_3\}$, $\{v_0, v_4\}$ and $\{v_1, v_2\}$, then Fixer should swap 0 and 3 at $v_3$. This results in a position with lists $L(v_1) = \{0, 1\}$, $L(v_2) = \{0, 2\}$, $L(v_3) = \{1, 2, 3, 4\}$, $L(v_4) = \{2, 3\}$ and $L(v_5) = \{0, 2\}$, but then Fixer wins by Case 101. If the components of $A_S$ have vertex sets $\{v_4\}$, $\{v_0, v_3\}$ and $\{v_1, v_2\}$, then Fixer should swap 0 and 3 at $v_4$. This results in a position with lists $L(v_1) = \{0, 1\}$, $L(v_2) = \{0, 2\}$, $L(v_3) = \{1, 2, 3, 4\}$, $L(v_4) = \{0, 2\}$ and $L(v_5) = \{2, 3\}$, but then Fixer wins by Case 95. 

\noindent\textbf{Case 169.  }\textit{$L(v_1) = \{0, 1\}$, $L(v_2) = \{0, 2\}$, $L(v_3) = \{1, 2, 3, 4\}$, $L(v_4) = \{0, 2\}$ and $L(v_5) = \{1, 2\}$.}

Let $S$ and $A_S$ be as in Lemma \ref{MultiMoveCombination} using colors $2$ and $3$. If the components of $A_S$ have vertex sets $\{v_1\}$ and $\{v_3, v_4\}$, then Fixer should swap 2 and 3 at $v_1$. This results in a position with lists $L(v_1) = \{0, 1\}$, $L(v_2) = \{0, 3\}$, $L(v_3) = \{1, 2, 3, 4\}$, $L(v_4) = \{0, 2\}$ and $L(v_5) = \{1, 2\}$, but then Fixer can edge-color the graph. If the components of $A_S$ have vertex sets $\{v_3\}$ and $\{v_1, v_4\}$, then Fixer should swap 2 and 3 at $v_3$. This results in a position with lists $L(v_1) = \{0, 1\}$, $L(v_2) = \{0, 2\}$, $L(v_3) = \{1, 2, 3, 4\}$, $L(v_4) = \{0, 3\}$ and $L(v_5) = \{1, 2\}$, but then Fixer can edge-color the graph. If the components of $A_S$ have vertex sets $\{v_4\}$ and $\{v_1, v_3\}$, then Fixer should swap 2 and 3 at $v_4$. This results in a position with lists $L(v_1) = \{0, 1\}$, $L(v_2) = \{0, 2\}$, $L(v_3) = \{1, 2, 3, 4\}$, $L(v_4) = \{0, 2\}$ and $L(v_5) = \{1, 3\}$, but then Fixer wins by Case 94. 

\noindent\textbf{Case 170.  }\textit{$L(v_1) = \{0, 1\}$, $L(v_2) = \{0, 2\}$, $L(v_3) = \{1, 2, 3, 4\}$, $L(v_4) = \{1, 2\}$ and $L(v_5) = \{0, 2\}$.}

Let $S$ and $A_S$ be as in Lemma \ref{MultiMoveCombination} using colors $2$ and $3$. If the components of $A_S$ have vertex sets $\{v_1\}$ and $\{v_3, v_4\}$, then Fixer should swap 2 and 3 at $v_1$. This results in a position with lists $L(v_1) = \{0, 1\}$, $L(v_2) = \{0, 3\}$, $L(v_3) = \{1, 2, 3, 4\}$, $L(v_4) = \{1, 2\}$ and $L(v_5) = \{0, 2\}$, but then Fixer can edge-color the graph. If the components of $A_S$ have vertex sets $\{v_3\}$ and $\{v_1, v_4\}$, then Fixer should swap 2 and 3 at $v_3$. This results in a position with lists $L(v_1) = \{0, 1\}$, $L(v_2) = \{0, 2\}$, $L(v_3) = \{1, 2, 3, 4\}$, $L(v_4) = \{1, 3\}$ and $L(v_5) = \{0, 2\}$, but then Fixer wins by Case 100. If the components of $A_S$ have vertex sets $\{v_4\}$ and $\{v_1, v_3\}$, then Fixer should swap 2 and 3 at $v_4$. This results in a position with lists $L(v_1) = \{0, 1\}$, $L(v_2) = \{0, 2\}$, $L(v_3) = \{1, 2, 3, 4\}$, $L(v_4) = \{1, 2\}$ and $L(v_5) = \{0, 3\}$, but then Fixer can edge-color the graph. 

\noindent\textbf{Case 171.  }\textit{$L(v_1) = \{0, 1\}$, $L(v_2) = \{0, 2\}$, $L(v_3) = \{0, 1, 3, 4\}$, $L(v_4) = \{3, 5\}$ and $L(v_5) = \{3, 5\}$.}

Let $S$ and $A_S$ be as in Lemma \ref{MultiMoveCombination} using colors $1$ and $2$. If the components of $A_S$ have vertex sets $\{v_0\}$ and $\{v_1, v_2\}$, then Fixer should swap 1 and 2 at $v_0$. This results in a position with lists $L(v_1) = \{0, 2\}$, $L(v_2) = \{0, 2\}$, $L(v_3) = \{0, 1, 3, 4\}$, $L(v_4) = \{3, 5\}$ and $L(v_5) = \{3, 5\}$, but then Fixer wins by Case 106. If the components of $A_S$ have vertex sets $\{v_1\}$ and $\{v_0, v_2\}$, then Fixer should swap 1 and 2 at $v_1$. This results in a position with lists $L(v_1) = \{0, 1\}$, $L(v_2) = \{0, 1\}$, $L(v_3) = \{0, 1, 3, 4\}$, $L(v_4) = \{3, 5\}$ and $L(v_5) = \{3, 5\}$, but then Fixer can edge-color the graph. If the components of $A_S$ have vertex sets $\{v_2\}$ and $\{v_0, v_1\}$, then Fixer should swap 1 and 2 at $v_2$. This results in a position with lists $L(v_1) = \{0, 1\}$, $L(v_2) = \{0, 2\}$, $L(v_3) = \{0, 2, 3, 4\}$, $L(v_4) = \{3, 5\}$ and $L(v_5) = \{3, 5\}$, but then Fixer wins by Case 127. 

\noindent\textbf{Case 172.  }\textit{$L(v_1) = \{0, 1\}$, $L(v_2) = \{0, 2\}$, $L(v_3) = \{0, 1, 3, 4\}$, $L(v_4) = \{0, 2\}$ and $L(v_5) = \{1, 2\}$.}

Let $S$ and $A_S$ be as in Lemma \ref{MultiMoveCombination} using colors $0$ and $2$. If the components of $A_S$ have vertex sets $\{v_0\}$ and $\{v_2, v_4\}$, then Fixer should swap 0 and 2 at $v_0$. This results in a position with lists $L(v_1) = \{1, 2\}$, $L(v_2) = \{0, 2\}$, $L(v_3) = \{0, 1, 3, 4\}$, $L(v_4) = \{0, 2\}$ and $L(v_5) = \{1, 2\}$, but then Fixer wins by Case 167. If the components of $A_S$ have vertex sets $\{v_2\}$ and $\{v_0, v_4\}$, then Fixer should swap 0 and 2 at $v_2$. This results in a position with lists $L(v_1) = \{0, 1\}$, $L(v_2) = \{0, 2\}$, $L(v_3) = \{1, 2, 3, 4\}$, $L(v_4) = \{0, 2\}$ and $L(v_5) = \{1, 2\}$, but then Fixer wins by Case 169. If the components of $A_S$ have vertex sets $\{v_4\}$ and $\{v_0, v_2\}$, then Fixer should swap 0 and 2 at $v_4$. This results in a position with lists $L(v_1) = \{0, 1\}$, $L(v_2) = \{0, 2\}$, $L(v_3) = \{0, 1, 3, 4\}$, $L(v_4) = \{0, 2\}$ and $L(v_5) = \{0, 1\}$, but then Fixer wins by Case 147. 

\noindent\textbf{Case 173.  }\textit{$L(v_1) = \{0, 1\}$, $L(v_2) = \{0, 2\}$, $L(v_3) = \{0, 1, 3, 4\}$, $L(v_4) = \{1, 2\}$ and $L(v_5) = \{0, 2\}$.}

Let $S$ and $A_S$ be as in Lemma \ref{MultiMoveCombination} using colors $0$ and $2$. If the components of $A_S$ have vertex sets $\{v_0\}$ and $\{v_2, v_3\}$, then Fixer should swap 0 and 2 at $v_0$. This results in a position with lists $L(v_1) = \{1, 2\}$, $L(v_2) = \{0, 2\}$, $L(v_3) = \{0, 1, 3, 4\}$, $L(v_4) = \{1, 2\}$ and $L(v_5) = \{0, 2\}$, but then Fixer wins by Case 166. If the components of $A_S$ have vertex sets $\{v_2\}$ and $\{v_0, v_3\}$, then Fixer should swap 0 and 2 at $v_2$. This results in a position with lists $L(v_1) = \{0, 1\}$, $L(v_2) = \{0, 2\}$, $L(v_3) = \{1, 2, 3, 4\}$, $L(v_4) = \{1, 2\}$ and $L(v_5) = \{0, 2\}$, but then Fixer wins by Case 170. If the components of $A_S$ have vertex sets $\{v_3\}$ and $\{v_0, v_2\}$, then Fixer should swap 0 and 2 at $v_3$. This results in a position with lists $L(v_1) = \{0, 1\}$, $L(v_2) = \{0, 2\}$, $L(v_3) = \{0, 1, 3, 4\}$, $L(v_4) = \{0, 1\}$ and $L(v_5) = \{0, 2\}$, but then Fixer wins by Case 145. 

\noindent\textbf{Case 174.  }\textit{$L(v_1) = \{0, 1\}$, $L(v_2) = \{0, 2\}$, $L(v_3) = \{0, 1, 3, 4\}$, $L(v_4) = \{1, 2\}$ and $L(v_5) = \{1, 2\}$.}

Let $S$ and $A_S$ be as in Lemma \ref{MultiMoveCombination} using colors $1$ and $2$. If the components of $A_S$ have vertex sets $\{v_0\}$ and $\{v_1, v_2\}$, then Fixer should swap 1 and 2 at $v_0$. This results in a position with lists $L(v_1) = \{0, 2\}$, $L(v_2) = \{0, 2\}$, $L(v_3) = \{0, 1, 3, 4\}$, $L(v_4) = \{1, 2\}$ and $L(v_5) = \{1, 2\}$, but then Fixer wins by Case 15. If the components of $A_S$ have vertex sets $\{v_1\}$ and $\{v_0, v_2\}$, then Fixer should swap 1 and 2 at $v_1$. This results in a position with lists $L(v_1) = \{0, 1\}$, $L(v_2) = \{0, 1\}$, $L(v_3) = \{0, 1, 3, 4\}$, $L(v_4) = \{1, 2\}$ and $L(v_5) = \{1, 2\}$, but then Fixer wins by Case 7. If the components of $A_S$ have vertex sets $\{v_2\}$ and $\{v_0, v_1\}$, then Fixer should swap 1 and 2 at $v_2$. This results in a position with lists $L(v_1) = \{0, 1\}$, $L(v_2) = \{0, 2\}$, $L(v_3) = \{0, 2, 3, 4\}$, $L(v_4) = \{1, 2\}$ and $L(v_5) = \{1, 2\}$, but then Fixer wins by Case 163. 

\noindent\textbf{Case 175.  }\textit{$L(v_1) = \{0, 1\}$, $L(v_2) = \{0, 2\}$, $L(v_3) = \{0, 1, 3, 4\}$, $L(v_4) = \{0, 3\}$ and $L(v_5) = \{1, 3\}$.}

Let $S$ and $A_S$ be as in Lemma \ref{MultiMoveCombination} using colors $0$ and $2$. If the components of $A_S$ have vertex sets $\{v_0\}$ and $\{v_2, v_3\}$, then Fixer should swap 0 and 2 at $v_0$. This results in a position with lists $L(v_1) = \{1, 2\}$, $L(v_2) = \{0, 2\}$, $L(v_3) = \{0, 1, 3, 4\}$, $L(v_4) = \{0, 3\}$ and $L(v_5) = \{1, 3\}$, but then Fixer can edge-color the graph. If the components of $A_S$ have vertex sets $\{v_2\}$ and $\{v_0, v_3\}$, then Fixer should swap 0 and 2 at $v_2$. This results in a position with lists $L(v_1) = \{0, 1\}$, $L(v_2) = \{0, 2\}$, $L(v_3) = \{1, 2, 3, 4\}$, $L(v_4) = \{0, 3\}$ and $L(v_5) = \{1, 3\}$, but then Fixer can edge-color the graph. If the components of $A_S$ have vertex sets $\{v_3\}$ and $\{v_0, v_2\}$, then Fixer should swap 0 and 2 at $v_3$. This results in a position with lists $L(v_1) = \{0, 1\}$, $L(v_2) = \{0, 2\}$, $L(v_3) = \{0, 1, 3, 4\}$, $L(v_4) = \{2, 3\}$ and $L(v_5) = \{1, 3\}$, but then Fixer wins by Case 155. 

\noindent\textbf{Case 176.  }\textit{$L(v_1) = \{0, 1\}$, $L(v_2) = \{0, 2\}$, $L(v_3) = \{0, 1, 3, 4\}$, $L(v_4) = \{1, 3\}$ and $L(v_5) = \{0, 3\}$.}

Let $S$ and $A_S$ be as in Lemma \ref{MultiMoveCombination} using colors $0$ and $2$. If the components of $A_S$ have vertex sets $\{v_0\}$ and $\{v_2, v_4\}$, then Fixer should swap 0 and 2 at $v_0$. This results in a position with lists $L(v_1) = \{1, 2\}$, $L(v_2) = \{0, 2\}$, $L(v_3) = \{0, 1, 3, 4\}$, $L(v_4) = \{1, 3\}$ and $L(v_5) = \{0, 3\}$, but then Fixer can edge-color the graph. If the components of $A_S$ have vertex sets $\{v_2\}$ and $\{v_0, v_4\}$, then Fixer should swap 0 and 2 at $v_2$. This results in a position with lists $L(v_1) = \{0, 1\}$, $L(v_2) = \{0, 2\}$, $L(v_3) = \{1, 2, 3, 4\}$, $L(v_4) = \{1, 3\}$ and $L(v_5) = \{0, 3\}$, but then Fixer can edge-color the graph. If the components of $A_S$ have vertex sets $\{v_4\}$ and $\{v_0, v_2\}$, then Fixer should swap 0 and 2 at $v_4$. This results in a position with lists $L(v_1) = \{0, 1\}$, $L(v_2) = \{0, 2\}$, $L(v_3) = \{0, 1, 3, 4\}$, $L(v_4) = \{1, 3\}$ and $L(v_5) = \{2, 3\}$, but then Fixer wins by Case 153. 

\noindent\textbf{Case 177.  }\textit{$L(v_1) = \{0, 1\}$, $L(v_2) = \{0, 2\}$, $L(v_3) = \{1, 2, 3, 4\}$, $L(v_4) = \{0, 1\}$ and $L(v_5) = \{1, 2\}$.}

Let $S$ and $A_S$ be as in Lemma \ref{MultiMoveCombination} using colors $0$ and $1$. If the components of $A_S$ have vertex sets $\{v_1\}$ and $\{v_2, v_4\}$, then Fixer should swap 0 and 1 at $v_1$. This results in a position with lists $L(v_1) = \{0, 1\}$, $L(v_2) = \{1, 2\}$, $L(v_3) = \{1, 2, 3, 4\}$, $L(v_4) = \{0, 1\}$ and $L(v_5) = \{1, 2\}$, but then Fixer wins by Case 157. If the components of $A_S$ have vertex sets $\{v_2\}$ and $\{v_1, v_4\}$, then Fixer should swap 0 and 1 at $v_2$. This results in a position with lists $L(v_1) = \{0, 1\}$, $L(v_2) = \{0, 2\}$, $L(v_3) = \{0, 2, 3, 4\}$, $L(v_4) = \{0, 1\}$ and $L(v_5) = \{1, 2\}$, but then Fixer wins by Case 158. If the components of $A_S$ have vertex sets $\{v_4\}$ and $\{v_1, v_2\}$, then Fixer should swap 0 and 1 at $v_4$. This results in a position with lists $L(v_1) = \{0, 1\}$, $L(v_2) = \{0, 2\}$, $L(v_3) = \{1, 2, 3, 4\}$, $L(v_4) = \{0, 1\}$ and $L(v_5) = \{0, 2\}$, but then Fixer wins by Case 166. 

\noindent\textbf{Case 178.  }\textit{$L(v_1) = \{0, 1\}$, $L(v_2) = \{0, 2\}$, $L(v_3) = \{1, 2, 3, 4\}$, $L(v_4) = \{1, 2\}$ and $L(v_5) = \{0, 1\}$.}

Let $S$ and $A_S$ be as in Lemma \ref{MultiMoveCombination} using colors $0$ and $1$. If the components of $A_S$ have vertex sets $\{v_1\}$ and $\{v_2, v_3\}$, then Fixer should swap 0 and 1 at $v_1$. This results in a position with lists $L(v_1) = \{0, 1\}$, $L(v_2) = \{1, 2\}$, $L(v_3) = \{1, 2, 3, 4\}$, $L(v_4) = \{1, 2\}$ and $L(v_5) = \{0, 1\}$, but then Fixer wins by Case 159. If the components of $A_S$ have vertex sets $\{v_2\}$ and $\{v_1, v_3\}$, then Fixer should swap 0 and 1 at $v_2$. This results in a position with lists $L(v_1) = \{0, 1\}$, $L(v_2) = \{0, 2\}$, $L(v_3) = \{0, 2, 3, 4\}$, $L(v_4) = \{1, 2\}$ and $L(v_5) = \{0, 1\}$, but then Fixer wins by Case 161. If the components of $A_S$ have vertex sets $\{v_3\}$ and $\{v_1, v_2\}$, then Fixer should swap 0 and 1 at $v_3$. This results in a position with lists $L(v_1) = \{0, 1\}$, $L(v_2) = \{0, 2\}$, $L(v_3) = \{1, 2, 3, 4\}$, $L(v_4) = \{0, 2\}$ and $L(v_5) = \{0, 1\}$, but then Fixer wins by Case 167. 

\noindent\textbf{Case 179.  }\textit{$L(v_1) = \{0, 1\}$, $L(v_2) = \{0, 2\}$, $L(v_3) = \{1, 2, 3, 4\}$, $L(v_4) = \{0, 3\}$ and $L(v_5) = \{2, 3\}$.}

Let $S$ and $A_S$ be as in Lemma \ref{MultiMoveCombination} using colors $0$ and $1$. If the components of $A_S$ have vertex sets $\{v_1\}$ and $\{v_2, v_3\}$, then Fixer should swap 0 and 1 at $v_1$. This results in a position with lists $L(v_1) = \{0, 1\}$, $L(v_2) = \{1, 2\}$, $L(v_3) = \{1, 2, 3, 4\}$, $L(v_4) = \{0, 3\}$ and $L(v_5) = \{2, 3\}$, but then Fixer wins by Case 164. If the components of $A_S$ have vertex sets $\{v_2\}$ and $\{v_1, v_3\}$, then Fixer should swap 0 and 1 at $v_2$. This results in a position with lists $L(v_1) = \{0, 1\}$, $L(v_2) = \{0, 2\}$, $L(v_3) = \{0, 2, 3, 4\}$, $L(v_4) = \{0, 3\}$ and $L(v_5) = \{2, 3\}$, but then Fixer can edge-color the graph. If the components of $A_S$ have vertex sets $\{v_3\}$ and $\{v_1, v_2\}$, then Fixer should swap 0 and 1 at $v_3$. This results in a position with lists $L(v_1) = \{0, 1\}$, $L(v_2) = \{0, 2\}$, $L(v_3) = \{1, 2, 3, 4\}$, $L(v_4) = \{1, 3\}$ and $L(v_5) = \{2, 3\}$, but then Fixer can edge-color the graph. 

\noindent\textbf{Case 180.  }\textit{$L(v_1) = \{0, 1\}$, $L(v_2) = \{0, 2\}$, $L(v_3) = \{1, 2, 3, 4\}$, $L(v_4) = \{2, 3\}$ and $L(v_5) = \{0, 3\}$.}

Let $S$ and $A_S$ be as in Lemma \ref{MultiMoveCombination} using colors $0$ and $1$. If the components of $A_S$ have vertex sets $\{v_1\}$ and $\{v_2, v_4\}$, then Fixer should swap 0 and 1 at $v_1$. This results in a position with lists $L(v_1) = \{0, 1\}$, $L(v_2) = \{1, 2\}$, $L(v_3) = \{1, 2, 3, 4\}$, $L(v_4) = \{2, 3\}$ and $L(v_5) = \{0, 3\}$, but then Fixer wins by Case 165. If the components of $A_S$ have vertex sets $\{v_2\}$ and $\{v_1, v_4\}$, then Fixer should swap 0 and 1 at $v_2$. This results in a position with lists $L(v_1) = \{0, 1\}$, $L(v_2) = \{0, 2\}$, $L(v_3) = \{0, 2, 3, 4\}$, $L(v_4) = \{2, 3\}$ and $L(v_5) = \{0, 3\}$, but then Fixer can edge-color the graph. If the components of $A_S$ have vertex sets $\{v_4\}$ and $\{v_1, v_2\}$, then Fixer should swap 0 and 1 at $v_4$. This results in a position with lists $L(v_1) = \{0, 1\}$, $L(v_2) = \{0, 2\}$, $L(v_3) = \{1, 2, 3, 4\}$, $L(v_4) = \{2, 3\}$ and $L(v_5) = \{1, 3\}$, but then Fixer can edge-color the graph. 

\noindent\textbf{Case 181.  }\textit{$L(v_1) = \{0, 1\}$, $L(v_2) = \{0, 1\}$, $L(v_3) = \{0, 2, 3, 4\}$, $L(v_4) = \{0, 2\}$ and $L(v_5) = \{1, 2\}$.}

Let $S$ and $A_S$ be as in Lemma \ref{MultiMoveCombination} using colors $0$ and $2$. If the components of $A_S$ have vertex sets $\{v_0\}$ and $\{v_1, v_4\}$, then Fixer should swap 0 and 2 at $v_0$. This results in a position with lists $L(v_1) = \{1, 2\}$, $L(v_2) = \{0, 1\}$, $L(v_3) = \{0, 2, 3, 4\}$, $L(v_4) = \{0, 2\}$ and $L(v_5) = \{1, 2\}$, but then Fixer wins by Case 178. If the components of $A_S$ have vertex sets $\{v_1\}$ and $\{v_0, v_4\}$, then Fixer should swap 0 and 2 at $v_1$. This results in a position with lists $L(v_1) = \{0, 1\}$, $L(v_2) = \{1, 2\}$, $L(v_3) = \{0, 2, 3, 4\}$, $L(v_4) = \{0, 2\}$ and $L(v_5) = \{1, 2\}$, but then Fixer wins by Case 170. If the components of $A_S$ have vertex sets $\{v_4\}$ and $\{v_0, v_1\}$, then Fixer should swap 0 and 2 at $v_4$. This results in a position with lists $L(v_1) = \{0, 1\}$, $L(v_2) = \{0, 1\}$, $L(v_3) = \{0, 2, 3, 4\}$, $L(v_4) = \{0, 2\}$ and $L(v_5) = \{0, 1\}$, but then Fixer wins by Case 12. 

\noindent\textbf{Case 182.  }\textit{$L(v_1) = \{0, 1\}$, $L(v_2) = \{0, 1\}$, $L(v_3) = \{0, 2, 3, 4\}$, $L(v_4) = \{1, 2\}$ and $L(v_5) = \{0, 2\}$.}

Let $S$ and $A_S$ be as in Lemma \ref{MultiMoveCombination} using colors $0$ and $2$. If the components of $A_S$ have vertex sets $\{v_0\}$ and $\{v_1, v_3\}$, then Fixer should swap 0 and 2 at $v_0$. This results in a position with lists $L(v_1) = \{1, 2\}$, $L(v_2) = \{0, 1\}$, $L(v_3) = \{0, 2, 3, 4\}$, $L(v_4) = \{1, 2\}$ and $L(v_5) = \{0, 2\}$, but then Fixer wins by Case 177. If the components of $A_S$ have vertex sets $\{v_1\}$ and $\{v_0, v_3\}$, then Fixer should swap 0 and 2 at $v_1$. This results in a position with lists $L(v_1) = \{0, 1\}$, $L(v_2) = \{1, 2\}$, $L(v_3) = \{0, 2, 3, 4\}$, $L(v_4) = \{1, 2\}$ and $L(v_5) = \{0, 2\}$, but then Fixer wins by Case 169. If the components of $A_S$ have vertex sets $\{v_3\}$ and $\{v_0, v_1\}$, then Fixer should swap 0 and 2 at $v_3$. This results in a position with lists $L(v_1) = \{0, 1\}$, $L(v_2) = \{0, 1\}$, $L(v_3) = \{0, 2, 3, 4\}$, $L(v_4) = \{0, 1\}$ and $L(v_5) = \{0, 2\}$, but then Fixer wins by Case 9. 

\end{proof}


\begin{lem}\label{CanColorAndPlayOnRest}
Let $G$ be a multigraph and $L$ a list assignment on $G$.  Suppose we have an edge-coloring $\pi$ of $H \subseteq G$ where $\pi(xy) \in L(x) \cap L(y)$ for all $xy \in E(H)$.  Put $G' \DefinedAs G - E(H)$ and 
$L'(v) \DefinedAs L(v) - \pi(E_H(v))$ for all $v \in V(G')$.  If Fixer has a winning strategy against Breaker in the chronicled game on $G'$ with lists $L'$, then Fixer has a winning strategy against Breaker in the chronicled game on $G$ with lists $L$.
\end{lem}

\begin{lem}\label{MultiMoveCombination}
Let $G$ be a multigraph, $L$ a list assignment on $G$ and $\alpha, \beta \in \pot(L)$. Let $S \subseteq V(G)$ be those vertices $v$ with $\card{\set{\alpha, \beta} \cap L(v)} = 1$.  Then there is a graph $A_S$ with vertex set $S$ and $\Delta(A_S) \le 1$ such that Fixer has a sequence of moves against Breaker in the chronicled game resulting in a list assignment where Fixer has chosen to swap $\alpha$ and $\beta$ all or none of the vertices in each component of $A_S$.
\end{lem}
\begin{proof}
For each $v \in S$, Fixer should swap $\alpha$ and $\beta$ at $v$ twice in a row.  Now every $v \in S$ is incident to an edge in $\C$; that is, as long as Fixer only does swaps with $\alpha$ and $\beta$, Breaker's moves are already foretold in the chronicle.  Now add an edge in $A_S$ for each $xy \in \C - \infty$ labeled $\set{\alpha, \beta}$. The lemma follows.
\end{proof}

\end{document}



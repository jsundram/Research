\documentclass[12pt]{amsart}
\usepackage{amsmath, amsthm, amssymb}
\usepackage[top=1.25in, bottom=1.25in, left=1.0in, right=1.0in]{geometry}
\usepackage{hyperref}
\usepackage{color}
\usepackage{verbatim}
\usepackage{hyperref}

\makeatletter
\newtheorem*{rep@theorem}{\rep@title}
\newcommand{\newreptheorem}[2]{
\newenvironment{rep#1}[1]{
 \def\rep@title{#2 \ref{##1}}
 \begin{rep@theorem}}
 {\end{rep@theorem}}}
\makeatother

\theoremstyle{plain}
\newtheorem{thm}{Theorem}
\newreptheorem{thm}{Theorem}
\newtheorem*{Brooks}{Brooks' Theorem}
\newtheorem*{KernelLemma}{Kernel Lemma}
\newtheorem{prop}[thm]{Proposition}
\newreptheorem{prop}{Proposition}
\newtheorem{lem}[thm]{Lemma}
\newreptheorem{lem}{Lemma}
\newtheorem{conj}[thm]{Conjecture}
\newreptheorem{conj}{Conjecture}
\newtheorem{cor}[thm]{Corollary}
\newreptheorem{cor}{Corollary}
\newtheorem{prob}[thm]{Problem}
\theoremstyle{definition}
\newtheorem{defn}{Definition}
\theoremstyle{remark}
\newtheorem*{remark}{Remark}
\newtheorem{example}{Example}
\newtheorem*{question}{Question}
\newtheorem*{observation}{Observation}
\newtheorem*{FixerMove}{\bf {Fixer's turn}}
\newtheorem*{BreakerMove}{\bf {Breaker's turn}}
\newtheorem*{ChronicleUpdate}{\bf {Chronicle update}}

%\title{Fixer-Breaker and Short Tashkinov Trees }
%\author{Daniel W. Cranston \and Landon Rabern}

\newcommand{\fancy}[1]{\mathcal{#1}}
\newcommand{\C}[1]{\fancy{C}_{#1}}
\newcommand{\IN}{\mathbb{N}}
\newcommand{\IR}{\mathbb{R}}
\newcommand{\G}{\fancy{G}}
\newcommand{\LB}{\mathcal{L}_B}
\newcommand{\col}{{\textrm{col}}}
\newcommand{\chil}{{\chi_{\ell}}}
\newcommand{\chiol}{{\chi_{OL}}}

\newcommand{\inj}{\hookrightarrow}
\newcommand{\surj}{\twoheadrightarrow}

\newcommand{\set}[1]{\left\{ #1 \right\}}
\newcommand{\setb}[3]{\left\{ #1 \in #2 \mid #3 \right\}}
\newcommand{\setbs}[2]{\left\{ #1 \mid #2 \right\}}
\newcommand{\card}[1]{\left|#1\right|}
\newcommand{\size}[1]{\left\Vert#1\right\Vert}
\newcommand{\ceil}[1]{\left\lceil#1\right\rceil}
\newcommand{\floor}[1]{\left\lfloor#1\right\rfloor}
\newcommand{\func}[3]{#1\colon #2 \rightarrow #3}
\newcommand{\funcinj}[3]{#1\colon #2 \inj #3}
\newcommand{\funcsurj}[3]{#1\colon #2 \surj #3}
\newcommand{\irange}[1]{\left[#1\right]}
\newcommand{\join}[2]{#1 \mbox{\hspace{2 pt}$\ast$\hspace{2 pt}} #2}
\newcommand{\djunion}[2]{#1 \mbox{\hspace{2 pt}$+$\hspace{2 pt}} #2}
\newcommand{\parens}[1]{\left( #1 \right)}
\newcommand{\brackets}[1]{\left[ #1 \right]}
\newcommand{\DefinedAs}{\mathrel{\mathop:}=}
\newcommand{\im}{\operatorname{im}}
\newcommand{\mic}{\operatorname{mic}}
\newcommand{\pot}{\operatorname{Pot}}

\renewcommand{\S}{\fancy{S}}
\newcommand{\W}{\fancy{W}}
\newcommand{\T}{\fancy{T}}
\renewcommand{\P}{\fancy{P}}
\renewcommand{\C}{\fancy{C}}

\begin{document}

\begin{lem}\label{P_4-tash2}
Fixer has a winning strategy against Breaker in the chronicled game on $G$, with superabundant list assignment $L$ where $|L(v_1)| = 2$, $|L(v_2)| = 2$, $|L(v_3)| = 3$ and $|L(v_4)| = 2$.
\end{lem}
\begin{proof}
We show that for each possible such list assignment $L$ on $G$, Fixer has a winning strategy.
Up to symmetry, the following cases cover all the possible list assignments that are not an immediate win for Fixer.

\noindent\textbf{Case 1.  }\textit{$L(v_1) = \{0, 1\}$, $L(v_2) = \{0, 2\}$, $L(v_3) = \{0, 1, 2\}$ and $L(v_4) = \{2, 3\}$.}

Fixer gets a winning strategy by coloring $v_3v_4$ with $2$ and applying Lemma \ref{CanColorAndPlayOnRest}.

\noindent\textbf{Case 2.  }\textit{$L(v_1) = \{0, 1\}$, $L(v_2) = \{0, 2\}$, $L(v_3) = \{0, 1, 3\}$ and $L(v_4) = \{1, 3\}$.}

Fixer gets a winning strategy by coloring $v_3v_4$ with $3$ and applying Lemma \ref{CanColorAndPlayOnRest}.

\noindent\textbf{Case 3.  }\textit{$L(v_1) = \{0, 1\}$, $L(v_2) = \{0, 2\}$, $L(v_3) = \{0, 1, 3\}$ and $L(v_4) = \{1, 2\}$.}

Let $S$ and $A_S$ be as in Lemma \ref{MultiMoveCombination} using colors $2$ and $3$. If the components of $A_S$ have vertex sets $\{v_1\}$ and $\{v_2, v_3\}$, then Fixer should swap 2 and 3 at $v_1$. This results in a position with lists $L(v_1) = \{0, 1\}$, $L(v_2) = \{0, 3\}$, $L(v_3) = \{0, 1, 3\}$ and $L(v_4) = \{1, 2\}$, but then Fixer can edge-color the graph. If the components of $A_S$ have vertex sets $\{v_2\}$ and $\{v_1, v_3\}$, then Fixer should swap 2 and 3 at $v_2$. This results in a position with lists $L(v_1) = \{0, 1\}$, $L(v_2) = \{0, 2\}$, $L(v_3) = \{0, 1, 2\}$ and $L(v_4) = \{1, 2\}$, but then Fixer can edge-color the graph. If the components of $A_S$ have vertex sets $\{v_3\}$ and $\{v_1, v_2\}$, then Fixer should swap 2 and 3 at $v_3$. This results in a position with lists $L(v_1) = \{0, 1\}$, $L(v_2) = \{0, 2\}$, $L(v_3) = \{0, 1, 3\}$ and $L(v_4) = \{1, 3\}$, but then Fixer wins by Case 2. 

\noindent\textbf{Case 4.  }\textit{$L(v_1) = \{0, 1\}$, $L(v_2) = \{0, 2\}$, $L(v_3) = \{0, 2, 3\}$ and $L(v_4) = \{1, 2\}$.}

Let $S$ and $A_S$ be as in Lemma \ref{MultiMoveCombination} using colors $1$ and $3$. If the components of $A_S$ have vertex sets $\{v_0\}$ and $\{v_2, v_3\}$, then Fixer should swap 1 and 3 at $v_0$. This results in a position with lists $L(v_1) = \{0, 3\}$, $L(v_2) = \{0, 2\}$, $L(v_3) = \{0, 2, 3\}$ and $L(v_4) = \{1, 2\}$, but then Fixer wins by Case 1. If the components of $A_S$ have vertex sets $\{v_2\}$ and $\{v_0, v_3\}$, then Fixer should swap 1 and 3 at $v_2$. This results in a position with lists $L(v_1) = \{0, 1\}$, $L(v_2) = \{0, 2\}$, $L(v_3) = \{0, 1, 2\}$ and $L(v_4) = \{1, 2\}$, but then Fixer can edge-color the graph. If the components of $A_S$ have vertex sets $\{v_3\}$ and $\{v_0, v_2\}$, then Fixer should swap 1 and 3 at $v_3$. This results in a position with lists $L(v_1) = \{0, 1\}$, $L(v_2) = \{0, 2\}$, $L(v_3) = \{0, 2, 3\}$ and $L(v_4) = \{2, 3\}$, but then Fixer can edge-color the graph. 

\noindent\textbf{Case 5.  }\textit{$L(v_1) = \{0, 1\}$, $L(v_2) = \{0, 2\}$, $L(v_3) = \{0, 1, 3\}$ and $L(v_4) = \{0, 1\}$.}

Let $S$ and $A_S$ be as in Lemma \ref{MultiMoveCombination} using colors $0$ and $2$. If the components of $A_S$ have vertex sets $\{v_0\}$ and $\{v_2, v_3\}$, then Fixer should swap 0 and 2 at $v_0$. This results in a position with lists $L(v_1) = \{1, 2\}$, $L(v_2) = \{0, 2\}$, $L(v_3) = \{0, 1, 3\}$ and $L(v_4) = \{0, 1\}$, but then Fixer can edge-color the graph. If the components of $A_S$ have vertex sets $\{v_2\}$ and $\{v_0, v_3\}$, then Fixer should swap 0 and 2 at $v_2$. This results in a position with lists $L(v_1) = \{0, 1\}$, $L(v_2) = \{0, 2\}$, $L(v_3) = \{1, 2, 3\}$ and $L(v_4) = \{0, 1\}$, but then Fixer can edge-color the graph. If the components of $A_S$ have vertex sets $\{v_3\}$ and $\{v_0, v_2\}$, then Fixer should swap 0 and 2 at $v_3$. This results in a position with lists $L(v_1) = \{0, 1\}$, $L(v_2) = \{0, 2\}$, $L(v_3) = \{0, 1, 3\}$ and $L(v_4) = \{1, 2\}$, but then Fixer wins by Case 3. 

\end{proof}


\begin{lem}\label{CanColorAndPlayOnRest}
Let $G$ be a multigraph and $L$ a list assignment on $G$.  Suppose we have an edge-coloring $\pi$ of $H \subseteq G$ where $\pi(xy) \in L(x) \cap L(y)$ for all $xy \in E(H)$.  Put $G' \DefinedAs G - E(H)$ and 
$L'(v) \DefinedAs L(v) - \pi(E_H(v))$ for all $v \in V(G')$.  If Fixer has a winning strategy against Breaker in the chronicled game on $G'$ with lists $L'$, then Fixer has a winning strategy against Breaker in the chronicled game on $G$ with lists $L$.
\end{lem}

\begin{lem}\label{MultiMoveCombination}
Let $G$ be a multigraph, $L$ a list assignment on $G$ and $\alpha, \beta \in \pot(L)$. Let $S \subseteq V(G)$ be those vertices $v$ with $\card{\set{\alpha, \beta} \cap L(v)} = 1$.  Then there is a graph $A_S$ with vertex set $S$ and $\Delta(A_S) \le 1$ such that Fixer has a sequence of moves against Breaker in the chronicled game resulting in a list assignment where Fixer has chosen to swap $\alpha$ and $\beta$ all or none of the vertices in each component of $A_S$.
\end{lem}
\begin{proof}
For each $v \in S$, Fixer should swap $\alpha$ and $\beta$ at $v$ twice in a row.  Now every $v \in S$ is incident to an edge in $\C$; that is, as long as Fixer only does swaps with $\alpha$ and $\beta$, Breaker's moves are already foretold in the chronicle.  Now add an edge in $A_S$ for each $xy \in \C - \infty$ labeled $\set{\alpha, \beta}$. The lemma follows.
\end{proof}

\end{document}



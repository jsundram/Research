\documentclass[12pt]{amsart}
\usepackage{amsmath, amsthm, amssymb}
\usepackage[top=1.25in, bottom=1.25in, left=1.0in, right=1.0in]{geometry}

\allowdisplaybreaks
\pagestyle{headings}

\usepackage{tkz-graph}
\usetikzlibrary{arrows}
\usetikzlibrary{shapes}
\usepackage[position=bottom]{subfig}

\makeatletter
\newtheorem*{rep@theorem}{\rep@title}
\newcommand{\newreptheorem}[2]{
\newenvironment{rep#1}[1]{
 \def\rep@title{#2 \ref{##1}}
 \begin{rep@theorem}}
 {\end{rep@theorem}}}
\makeatother

\theoremstyle{plain}
\newtheorem{thm}{Theorem}[section]
\newreptheorem{thm}{Theorem}
\newtheorem{prop}[thm]{Proposition}
\newreptheorem{prop}{Proposition}
\newtheorem{lem}[thm]{Lemma}
\newreptheorem{lem}{Lemma}
\newtheorem{conjecture}[thm]{Conjecture}
\newreptheorem{conjecture}{Conjecture}
\newtheorem{cor}[thm]{Corollary}
\newreptheorem{cor}{Corollary}
\newtheorem{prob}[thm]{Problem}
\newtheorem*{GoldBerg}{Goldberg's Conjecture}
\theoremstyle{definition}
\newtheorem{defn}{Definition}
\theoremstyle{remark}
\newtheorem*{remark}{Remark}
\newtheorem*{FixerMove}{\bf {Fixer's turn}}
\newtheorem*{BreakerMove}{\bf {Breaker's turn}}
\newtheorem{example}{Example}
\newtheorem*{question}{Question}
\newtheorem*{observation}{Observation}

\newcommand{\fancy}[1]{\mathcal{#1}}
\newcommand{\C}[1]{\fancy{C}_{#1}}
\newcommand{\IN}{\mathbb{N}}
\newcommand{\IR}{\mathbb{R}}
\newcommand{\G}{\fancy{G}}

\newcommand{\inj}{\hookrightarrow}
\newcommand{\surj}{\twoheadrightarrow}

\newcommand{\set}[1]{\left\{ #1 \right\}}
\newcommand{\setb}[3]{\left\{ #1 \in #2 \mid #3 \right\}}
\newcommand{\setbs}[2]{\left\{ #1 \mid #2 \right\}}
\newcommand{\card}[1]{\left|#1\right|}
\newcommand{\size}[1]{\left\Vert#1\right\Vert}
\newcommand{\ceil}[1]{\left\lceil#1\right\rceil}
\newcommand{\floor}[1]{\left\lfloor#1\right\rfloor}
\newcommand{\func}[3]{#1\colon #2 \rightarrow #3}
\newcommand{\funcinj}[3]{#1\colon #2 \inj #3}
\newcommand{\funcsurj}[3]{#1\colon #2 \surj #3}
\newcommand{\irange}[1]{\left[#1\right]}
\newcommand{\join}[2]{#1 \mbox{\hspace{2 pt}$\ast$\hspace{2 pt}} #2}
\newcommand{\djunion}[2]{#1 \mbox{\hspace{2 pt}$+$\hspace{2 pt}} #2}
\newcommand{\parens}[1]{\left( #1 \right)}
\newcommand{\brackets}[1]{\left[ #1 \right]}
\newcommand{\DefinedAs}{\mathrel{\mathop:}=}
\newcommand{\im}{\operatorname{im}}


\renewcommand{\S}{\fancy{S}}
\newcommand{\W}{\fancy{W}}
\newcommand{\T}{\fancy{T}}
\renewcommand{\P}{\fancy{P}}


\title{Proposal for 2012 block grant:\\Edge coloring via two-player games}
\author{Landon Rabern}

\begin{document}
\maketitle

\section{Introduction}
We propose to study the following two-player game which has applications to edge
coloring graphs.  The game is played on a graph $G$ by \emph{Fixer} and
\emph{Breaker}.  The game is set up by assigning a list of colors $L(v)$ to each
$v \in V(G)$ and choosing a \emph{pot} $P$ with $\bigcup_{v \in V(G)} L(v)
\subseteq P$.  Fixer always gets the first move and he wins iff before moving on
his turn he can construct a proper edge coloring of $G$, say $\func{\pi}{E(G)}{P}$ such that
$\pi(xy) \in L(x) \cap L(y)$ for each $xy \in E(G)$.

\begin{FixerMove}
Pick $\alpha \in P$ and $v \in V(G)$ with $\alpha \not \in L(v)$ and set $L(v)
\DefinedAs L(v) \cup \set{\alpha} \smallsetminus \set{\beta}$ for some $\beta
\in L(v)$.
\end{FixerMove}

\begin{BreakerMove}
If Fixer modified $L(v)$ by inserting $\alpha$ and removing $\beta$, Breaker can
pick at most one vertex in $V(G) \smallsetminus \set{v}$ and modify its list by
swapping $\alpha$ for $\beta$ or $\beta$ for $\alpha$.
\end{BreakerMove}

The game setup corresponds to the situation that arises after picking some edges
in a (multi)graph, removing them and coloring the rest of the graph.  A Fixer
move followed by a Breaker move corresponds to switching colors along a
two-colored path (see the proof of Vizing's theorem below for an example). 
Studying this game on its own apart from edge coloring has many advantages just as studying list coloring as a separate concept has advantages
over merely viewing it as pre-coloring extension---more induction parameters,
indirect minimizations and hypothesis generalizations to name a few.

In \cite{HallGame}, we have carried out the analysis of winning strategies for
Fixer when $G$ is a fan $K_{1,t}$.  The result is a simultaneous generalization
of Hall's theorem and Vizing's theorem.  Other results such as Vizing's
adjacency lemma and generalizations also follow from the analysis.  In the
next section, we will outline this analysis as an example.  

Our proposal is to take these results much further by analyzing the game on more
classes of graphs.  One goal will be to fold the acceptable paths of Kierstead
\cite{kierstead1984chromatic} and the acceptable trees of Tashkinov
\cite{tashkinov} into this framework.  We also hope that this approach will help
make progress on the big conjecture of Goldberg \cite{goldberg}. For a graph
$G$, define the \emph{overfull parameter} of $G$ as

\[w(G) \DefinedAs \max_{H \subseteq G} \ceil{\frac{\size{H}}{\floor{\frac12
\card{H}}}}.\]

\begin{GoldBerg}
For any graph $G$ we have $\chi'(G) \leq \max \set{\Delta(G) + 1, w(G)}$.
\end{GoldBerg}

\section{The game on a fan}
As an example, we have completed the analysis of the game on a fan $K_{1,t}$.
From this, all of the classical edge coloring results follow.  We need a couple
of preconditions to make the game tenable.  First we always assume $\card{P}
\geq t$ and second we assume that the root $r$ of the $K_{1,t}$ has $L(r) = P$. 
With these assumptions, the game amounts to doing Hall's theorem with a little
extra power. For the remainder of this proposal, let $\S$ be a finite family of
finite sets. A \emph{transversal} of $\S$ is an injection $\funcinj{f}{\S}{\bigcup \S}$ such that $f(S) \in S$ for each $S \in \S$.  
Hall's theorem \cite{hall} gives the precise conditions under
which $\S$ has a transversal.

\begin{thm}[Hall \cite{hall}]
$\S$ has a transversal iff $\card{\bigcup \W} \geq \card{\W}$ for each $\W
\subseteq \S$.
\end{thm}

The precise conditions under which Fixer has a winning strategy will look very
similar.  To state the main theorem we need a couple pieces of notation. Define
the \emph{degree} in $\W \subseteq \S$ of $x \in P$ as

\[d_{\W}(x) \DefinedAs \card{\setb{S}{\W}{x \in S}}.\]

\noindent Now define the \emph{value} of $\W \subseteq \S$ as

\[\nu(\W) \DefinedAs \sum_{x \in P} \floor{\frac{\max\set{0, d_{\W}(x) -
1}}{2}}.\]

\begin{thm}\label{MainTheorem}
Fixer has a winning strategy against Breaker iff $\card{\bigcup \W} \geq
\card{\W} - \nu(\W)$ for each $\W \subseteq \S$.
\end{thm}

Vizing's theorem follows from a special case of Theorem \ref{MainTheorem}.  
In fact, the strategy employed by Fixer is based, in part, on Ehrenfeucht, Faber
and Kierstead's proof of Vizing's theorem \cite{Ehrenfeucht1984159}.

\begin{cor}\label{Simple}
If $\S = \set{S_1, \ldots, S_k}$ with $\card{S_k} \geq 1$ and $\card{S_i} \geq
2$ for all $i \in \irange{k-1}$, then Fixer has a winning strategy against
Breaker.
\end{cor}

\begin{proof}
Let $\W \subseteq \S$. Then $\nu(\W) \geq \sum_{x
\in \bigcup \W} \frac{d_{\W}(x) - 2}{2} = \frac12 \sum_{S \in \W} \card{S} -
\card{\bigcup \W} \geq \frac12 (2\card{\W} - 1) -
\card{\bigcup \W}$.  Hence $\nu(\W) \geq \card{\W} - \card{\bigcup \W}$ as
desired.
\end{proof}

\begin{cor}[Vizing \cite{vizing}]
Every simple graph satisfies $\chi' \leq \Delta + 1$.
\end{cor}
\begin{proof}
Suppose not and let $G$ be a counterexample minimizing $\card{G}$. Put $\Delta
\DefinedAs \Delta(G)$.  Pick $v \in V(G)$ with
degree $\Delta$, say $v_1, \ldots, v_\Delta$ are the neighbors of $v$ in $G$. 
By minimality of $\card{G}$, we have a $(\Delta + 1)$-edge-coloring of $G -
v$. Let $S_i$ be the colors not incident with $v_i$ in this coloring. 
Each $v_i$ has degree at most $\Delta - 1$ in $G - v$ and hence
$\card{S_i} \geq 2$. Also, if $a \in S_i$ and $b \not \in S_i$ we
may exchange colors on a maximum length path starting at $v_i$ and alternating between colors $b$ and $a$. 
This gives a Fixer move followed by a Breaker move.  Apply Corollary
\ref{Simple} to get a transversal of the $S_i$. Now we may complete the $(\Delta +
1)$-edge-coloring to all of $G$ by using the corresponding element of the
transversal on $vv_i$ for each $i \in \irange{\Delta}$.
\end{proof}

To deal with edge coloring multigraphs, we need to generalize our game slightly.
Instead of looking for a transversal, we will look for a system of
disjoint representatives. For $\func{\eta}{\S}{\IN^+}$ an
\emph{$\eta$-transversal} of $\S$ is a function $\func{f}{\S}{\P\parens{\bigcup
\S}}$ such that $f(S) \subseteq S$, $\card{f(S)} = \eta(S)$ for $S \in \S$ and
$f(A) \cap f(B) = \emptyset$ for different $A,B \in \S$.  By making $\eta(S)$
copies of each $S \in \S$ and applying Hall's theorem we get the following.

\begin{thm}
$\S$ has an $\eta$-transversal iff $\card{\bigcup \W} \geq \sum_{W \in \W}
\eta(W)$ for each $\W \subseteq \S$.
\end{thm}

Call the game where Fixer wins iff he creates an $\eta$-transversal \emph{the
$\eta$-game}.  Using the above simple generalization of Hall's theorem in the
proof of Theorem \ref{MainTheorem} gives the following.

\begin{thm}\label{MainTheoremMulti}
Fixer has a winning strategy against Breaker in the $\eta$-game iff
$\card{\bigcup \W} \geq \sum_{W \in \W}
\eta(W) - \nu(\W)$ for each $\W \subseteq \S$.
\end{thm}

With this theorem in hand, many edge coloring results such as Vizing's adjacency
lemma and generalizations follow.

\bibliographystyle{amsplain}
\bibliography{hall}
%\nocite*{}
%\bibliographystyle{amsplain}
%\renewcommand\refname{Landon Rabern's Publications}
%\bibliography{landon}

\renewcommand{\refname}{Landon Rabern's Publications} 
\begin{thebibliography}{1}

\bibitem{krs_one}
A.V. Kostochka, L.~Rabern, and M.~Stiebitz, \emph{{Graphs with chromatic number
  close to maximum degree}}, Discrete Math (Forthcoming).

\bibitem{rabern2008simple}
B.~Rabern and L.~Rabern, \emph{A simple solution to the hardest logic puzzle
  ever}, Analysis \textbf{68} (2008), no.~2, 105--112.

\bibitem{DBLP:journals/siamdm/Rabern06}
L.~Rabern, \emph{{On Graph Associations}}, SIAM J. Discrete Math. \textbf{20}
  (2006), no.~2, 529--535.

\bibitem{rabern2007applying}
L.~Rabern, \emph{{Applying Gr{\"o}bner basis techniques to group theory}},
  Journal of Pure and Applied Algebra \textbf{210} (2007), no.~1, 137--140.

\bibitem{DBLP:journals/combinatorics/Rabern07}
L.~Rabern, \emph{{The Borodin-Kostochka Conjecture for Graphs Containing a Doubly
  Critical Edge}}, Electr. J. Comb. \textbf{14} (2007), no.~1.

\bibitem{DBLP:journals/siamdm/Rabern08}
L.~Rabern, \emph{{A Note On Reed's Conjecture}}, SIAM J. Discrete Math.
  \textbf{22} (2008), no.~2, 820--827.

\bibitem{rabern2011strengthening}
L.~Rabern, \emph{{A strengthening of Brooks' Theorem for line graphs}}, Electron.
  J. Combin. \textbf{18} (2011), no.~p145, 1.

\bibitem{rabernhitting}
L.~Rabern, \emph{{On hitting all maximum cliques with an independent set}},
  Journal of Graph Theory \textbf{66} (2011), no.~1, 32--37.

\bibitem{rabern2010a}
L.~Rabern, \emph{{$\Delta$-Critical graphs with small high vertex cliques}},
  Journal of Combinatorial Theory Series B \textbf{102} (2012), no.~1,
  126--130.

\bibitem{rabern2010destroying}
L.~Rabern, \emph{Destroying non-complete regular components in graph partitions},
  J. Graph Theory (Forthcoming).

\bibitem{Danger}
L.~Rabern, B.~Rabern, and M.~Macauley, \emph{{Dangerous reference graphs and semantic paradoxes}}, Submitted
  (2012).

\bibitem{partitiondegree}
L.~Rabern, \emph{Partitioning and coloring with degree constraints}, Submitted (2012).

\bibitem{HallGame}
L.~Rabern, \emph{{A game generalizing Hall's
  theorem}}, Submitted (2012).

\end{thebibliography}


\end{document}

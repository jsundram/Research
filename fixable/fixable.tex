\documentclass[12pt]{article}
\usepackage{amsmath, amsthm, amssymb}
\usepackage{hyperref}
\usepackage{verbatim}
\usepackage[top=1.0in, bottom=1.0in, left=1.0in, right=1.0in]{geometry}

\pagestyle{plain}

\usepackage{tkz-graph}
\usetikzlibrary{arrows}
\usetikzlibrary{shapes}
\usepackage[position=bottom]{subfig}

\usepackage{longtable}
\usepackage{array}

\usepackage{sectsty}
\allsectionsfont{\sffamily}

\setcounter{secnumdepth}{5}
\setcounter{tocdepth}{5}

\makeatletter
\newtheorem*{rep@theorem}{\rep@title}
\newcommand{\newreptheorem}[2]{
\newenvironment{rep#1}[1]{
 \def\rep@title{#2 \ref{##1}}
 \begin{rep@theorem}}
 {\end{rep@theorem}}}
\makeatother

\theoremstyle{plain}
\newtheorem{thm}{Theorem}[section]
\newreptheorem{thm}{Theorem}
\newtheorem{prop}[thm]{Proposition}
\newreptheorem{prop}{Proposition}
\newtheorem{lem}[thm]{Lemma}
\newreptheorem{lem}{Lemma}
\newtheorem{conjecture}[thm]{Conjecture}
\newreptheorem{conjecture}{Conjecture}
\newtheorem{cor}[thm]{Corollary}
\newreptheorem{cor}{Corollary}
\newtheorem{prob}[thm]{Problem}
\newtheorem{observation}{Observation}
\newtheorem*{mainconj}{Main Conjecture}
\newtheorem*{mainthm}{Main Theorem}

\theoremstyle{definition}
\newtheorem{defn}{Definition}
\theoremstyle{remark}
\newtheorem*{remark}{Remark}
\newtheorem*{problem}{Problem}
\newtheorem{example}{Example}
\newtheorem*{question}{Question}


\newcommand{\fancy}[1]{\mathcal{#1}}
\newcommand{\C}[1]{\fancy{C}_{#1}}
\newcommand{\IN}{\mathbb{N}}
\newcommand{\IR}{\mathbb{R}}
\newcommand{\G}{\fancy{G}}
\newcommand{\CC}{\fancy{C}}
\newcommand{\D}{\fancy{D}}

\newcommand{\inj}{\hookrightarrow}
\newcommand{\surj}{\twoheadrightarrow}

\newcommand{\set}[1]{\left\{ #1 \right\}}
\newcommand{\setb}[3]{\left\{ #1 \in #2 \mid #3 \right\}}
\newcommand{\setbs}[2]{\left\{ #1 \mid #2 \right\}}
\newcommand{\card}[1]{\left|#1\right|}
\newcommand{\size}[1]{\left\Vert#1\right\Vert}
\newcommand{\ceil}[1]{\left\lceil#1\right\rceil}
\newcommand{\floor}[1]{\left\lfloor#1\right\rfloor}
\newcommand{\func}[3]{#1\colon #2 \rightarrow #3}
\newcommand{\funcinj}[3]{#1\colon #2 \inj #3}
\newcommand{\funcsurj}[3]{#1\colon #2 \surj #3}
\newcommand{\irange}[1]{\left[#1\right]}
\newcommand{\join}[2]{#1 \mbox{\hspace{2 pt}$\ast$\hspace{2 pt}} #2}
\newcommand{\djunion}[2]{#1 \mbox{\hspace{2 pt}$+$\hspace{2 pt}} #2}
\newcommand{\parens}[1]{\left( #1 \right)}
\newcommand{\brackets}[1]{\left[ #1 \right]}
\newcommand{\nint}[1]{\widetilde{N}\left(#1\right)}
\newcommand{\DefinedAs}{\mathrel{\mathop:}=}

\def\adj{\leftrightarrow}
\def\nonadj{\not\!\leftrightarrow}

\def\D{\fancy{D}}
\def\C{\fancy{C}}
\def\Q{\fancy{Q}}
\def\Z{\fancy{Z}}

\newcommand{\pot}{\operatorname{Pot}}

% any changes to \claim should be mirrored in \claimnonum and \subclaim
\newcommand{\claim}[2]{{\bf Claim #1.}~{\it #2}~~}
\newcommand{\claimnonum}[1]{{\bf Claim.}~{\it #1}~~}
\newcommand{\subclaim}[2]{{\bf Subclaim #1.}~{\it #2}~~}

\begin{document}
\title{Edge-coloring via fixable subgraphs}
\maketitle

\section{Introduction}
All multigraphs are loopless.

\section{Completing edge-colorings}
Our goal is to convert a partial $k$-edge-coloring of a multigraph $M$ into a (total) $k$-edge-coloring of $M$.  For a partial $k$-edge-coloring $\pi$ of $M$, let $M_\pi$ be the subgraph of $M$ induced on the uncolored edges and let $L_\pi$ be the list assignment on the vertices of $M_\pi$ given by 
$L_\pi(v) = \irange{k} - \setbs{\tau}{\pi(vx) = \tau \text{ for some  } vx \in E(M)}$. 

Kempe chains give a powerful technique for converting a partial $k$-edge-coloring into a total $k$-edge-coloring.  The idea is to repeatedly exchange colors on two-colored paths until $M_\pi$ has an edge-coloring $\zeta$ such that $\zeta(xy) \in L_\zeta(x) \cap L_\zeta(y)$ for all $xy \in E(M_\pi)$.  In this sense the original list assignment $L_\pi$ on $M_\pi$ is \emph{fixable}. In the next section, we give an abstract definition of this notion that frees us from the embedding in the ambient graph $M$.

\subsection{Fixable graphs}
Let $G$ be a multigraph and $L$ a list assignment on $V(G)$.  For different colors $a,b \in \pot(L)$, let $S_{a,b}$ be all the vertices of $G$ that have exactly one of $a$ or $b$ in their list; more precisely, $S_{a,b} = \setb{v}{V(G)}{\card{\set{a,b} \cap L(v)} = 1}$.  We say that $G$ is \emph{$L$-fixable} if either
\begin{enumerate}
\item $G$ has an edge-coloring $\pi$ such that $\pi(x) \in L(x) \cap L(y)$ for all $xy \in E(G)$; or
\item there are different $a,b \in \pot(L)$ such that for every partition $P_1, \ldots, P_k$ of $S_{a,b}$ into sets of size at most two, 
      there is $J \subseteq \irange{k}$ so that $G$ is $L'$-fixable where $L'$ is formed from $L$ by swapping $a$ and $b$ in $L(v)$ for every $v \in \bigcup_{i \in J} P_i$.
\end{enumerate}

\begin{lem}\label{FixableCompletesColoring}
If a multigraph $M$ has a partial $k$-edge-coloring $\pi$ such that $M_\pi$ is $L_\pi$-fixable, then $M$ is $k$-edge-colorable.
\end{lem}
\begin{proof}
\end{proof}

\subsection{A necessary condition}
Since the edges incident to a given vertex must all get different colors, we have the following.

\begin{lem}\label{DegreeNecessaryCondition}
If $G$ is $L$-fixable, then $|L(v)| \ge d_G(v)$ for all $v \in V(G)$.
\end{lem}

By considering the maximum size of matchings in each color, we get a much more interesting necessary condition.
For $C \subseteq \pot(L)$ and $H \subseteq G$, let $H_{L, C}$ be the
subgraph of $H$ induced on the vertices $v$ with $L(v) \cap C \ne \emptyset$. 
When $L$ is clear from context, we may write $H_C$ for $H_{L,C}$. If $C =
\set{\alpha}$, we may write $H_\alpha$ for $H_C$.  For $H \subseteq G$, put

\[\psi_L(H) = \sum_{\alpha \in \pot(L)} \floor{\frac{\card{H_{L, \alpha}}}{2}}.\]

Each term in the sum gives an upper bound on the size of a matching in color
$\alpha$. So $\psi_L(H)$ is an upper bound on the the number of edges in a
partial $L$-edge-coloring of $H$.  We say that $(H, L)$ is \emph{abundant} if
$\psi_L(H) \ge \size{H}$ and that $(G,L)$ is \emph{superabundant} if for every
$H \subseteq G$, the pair $(H, L)$ is abundant.  

\begin{lem}\label{SuperabundanceIsNecessary}
If $G$ is $L$-fixable, then $(G, L)$ is superabundant.
\end{lem}
\begin{proof}
\end{proof}

Intuitively, superabundance requires the potential for a large enough matching in each color. If instead we require the existence of a large enough matching in each color, we get a stronger requirement that has been studied before. For a multigraph $H$, let $\nu(H)$ be the number of edges in a maximum matching of $H$. 
For a list assignment $L$ on $H$, put $\eta_L(G) = \sum_{\alpha \in \pot(L)} \nu(G_\alpha)$.

The following generalization of Hall's theorem was proved by Marcotte and Seymour \cite{marcotte1990extending} and independently by Cropper, Gy{\'a}rf{\'a}s and Lehel \cite{cropper2003edge}.  By a \emph{multitree} we mean a tree that possibly has edges of multiplicity greater than one.

\begin{lem}\label{MultiTreeHall}
Let $T$ be a multitree and $L$ a list assignment on $V(T)$.  If $\eta_L(H) \ge \size{H}$ for all $H \subseteq T$, then $T$ has an edge-coloring $\func{\pi}{E(T)}{\pot(L)}$ such that
$\pi(xy) \in L(x) \cap L(y)$ for each $xy \in E(T)$.
\end{lem}

\subsection{Fixability of stars}
When $G$ is a star, the conjunction of our two necessary conditions is sufficient. This generalizes Vizing fans \cite{Vizing76}; in the next section we will define ``Kierstead-Tashkinov-Vizing assignments'' and show that they are always superabundant.  In \cite{HallGame}, the second author proved a common generalization of Theorem \ref{FixabilityOfStars} and Hall's theorem.  In particular, Theorem \ref{FixabilityOfStars} holds for multistars as well; the proof for multistars is nearly identical, but notationally cumbersome.

\begin{thm}\label{FixabilityOfStars}
If $G$ is a star, then $G$ is $L$-fixable if and only if $(G, L)$ is superabundant and $|L(v)| \ge d_G(v)$ for all $v \in V(G)$.
\end{thm}
\begin{proof}
\end{proof}

\subsection{Kierstead-Tashkinov-Vizing assignments}
Many edge-coloring results have been proved using a specific kind of
superabundant pair $(G, L)$ where superabundance can be proved via a special
ordering. That is, the orderings given by the definition of Vizing fans,
Kierstead paths, and Tashkinov trees.  In this section, we show how
superabundance easily follows from these orderings.

We say that a list assignment $L$ on $G$ is a \emph{Kierstead-Tashkinov-Vizing} assignment (henceforth \emph{KTV-assignment}) if for some $xy \in E(G)$, there is a total ordering `$<$' of $V(G)$ such that

\begin{enumerate}
\item there is and edge-coloring $\pi$ of $G-xy$ such that $\pi(uv) \in L(u) \cap L(v)$ for each $uv \in E(G - xy)$; 
\item $x < z$ for all $z \in V(G - x)$; 
\item $G\brackets{w \mid w \le z}$ is connected for all $z \in V(G)$; 
\item for each $wz \in E(G - xy)$, there is $u < \max\set{w, z}$ such that $\pi(wz) \in L(u) - \setbs{\pi(e)}{e \in E(u)}$;
\item there are different $s, t \in V(G)$ such that $L(s) \cap L(t) - \setbs{\pi(e)}{e \in E(s) \cup E(t)} \ne \emptyset$.
\end{enumerate}

\begin{lem}\label{KTVImpliesSuperabundant}
If $L$ is a KTV-assignment on $G$, then $(G, L)$ is superabundant.
\end{lem}
\begin{proof}
Let $L$ be a KTV-assignment on $G$, and let $H \subseteq G$.  We will show that
$(H,L)$ is abundant.  
Clearly it suffices to consider the case when $H$ is an induced subgraph, so we
assume this.
Property (1) gives that $G-xy$ has an edge-coloring
$\pi$, so $\psi_L(H)\ge \size{H}-1$; also $\psi_L(H)\ge \size{H}$ if
$\{x,y\}\not\subseteq V(H)$.  Furthermore $\psi_L(H)\ge \size{H}$ if $s$ and
$t$ from property (5) are both in $V(H)$, since then $\psi_L(H)$ gains 1 over
the naive lower bound, due to the color in $L(s)\cap L(t)$.  So $V(G)-
V(H)\ne \emptyset$.

Now choose $z \in V(G) - V(H)$ that is smallest under $<$.  
Put $H' = G\brackets{w \mid w \le z}$.  By the minimality of $z$, we have $H' - z \subseteq H$. By property (2), $\card{H'} \ge 2$.  
By property (3), $H'$ is connected and thus there is $w \in V(H' - z)$ adjacent to $z$. So, we have $w < z$ and $wz\in E(G)-E(H)$.
Now $\pi(wz)\in L(w)$.  By the definition of a KTV-assignment, 
property (4) implies that there exists $u$ with $u < \max\set{w, z} = z$ and $\pi(wz) \in
L(u)-\{\pi(e)|e\in E(u)\}$.  Then $u \in V(H' - z) \subseteq V(H)$ and
again we gain 1 over the naive lower bound on $\psi_L(H)$, due to the color
in $L(u)\cap L(w)$.  So $\psi_L(H)\ge \size{H}$.
\end{proof}

\subsection{Stars with one edge subdivided}
\begin{thm}
Let $G$ be a star with one edge subdivided and root $r$. If $(G, L)$ is superabundant, $|L(v)| \ge d_G(v)$ for all $v \in V(G)$ and $|L(r)| > d_G(r)$, then $G$ is $L$-fixable.
\end{thm}
\begin{proof}

\end{proof}
\section{Applications}

\section{Conjectures}
\begin{conjecture}
Any multigraph $G$ is $L$-fixable if $(G, L)$ is superabundant and $|L(v)| > d_G(v)$ for all $v \in V(G)$.
\end{conjecture}

\begin{conjecture}
Any tree $G$ is $L$-fixable if $(G, L)$ is superabundant and $|L(v)| > d_G(v)$ for all $v \in V(G)$.
\end{conjecture}

\begin{conjecture}
Any path $G$ is $L$-fixable if $(G, L)$ is superabundant and $|L(v)| > d_G(v)$ for all $v \in V(G)$.
\end{conjecture}

\bibliographystyle{amsplain}
\bibliography{GraphColoring}
\end{document}
